\باب{غیر تابع وقت شروڈنگر مساوات}


باب اول میں ہم نے تفار موج پر بات کی جہاں اسکا استعمال کرتے ہوئے دلچسپی کے مختلف مقداروں کا حساب کیا. اب وقت آن پہنچا ہے کہ ہم شروڈنگر مساوات سے \( \Psi (x,t) \)   حاصل کرنا سیکھیں. 

\begin{align}
i \hbar \frac{\partial \Psi}{\partial t} = - \frac{\hbar^{2}}{2 m} \frac{\partial^{2} \Psi}{\partial x^{2}} + V\Psi
\end{align}
  

ہم مساوات شروڈنگر کو کسی مخصوص خفی توانائی \( V (x,t) \) کیلیئے حل کرنا سیکھتے ہیں. اس باب میں, بلکہ کتاب کے بیشتر حصے میں میں فرض کروں گا کہ  \( V \) وقت t کا تابع نہیں ہے. ایسی صورت میں مساوات شروڈنگر کو علیحدگی متغیرات کے طریقے سے حل کیا جا سکتا ہے. ہم ایسے حل تلاش کرتے ہیں جنہیں حاصل ضرب کی شکل میں لکھنا ممکن ہو. 

\begin{align}
\Psi (x,t) = \psi (x)  \phi (t),
\end{align}

جہاں \( \psi \) صرف\( x \) 
 کا تفاعل ہے جبکہ \( \phi \)
  صرف  \( t \)  کا تفاعل ہے. ظاہری طور پر حل پر یہ شرط مسلط کرنا درست قدم نظر نہیں آتا. لیکن حقیقت میں یوں حاصل ہوئے حل بہت کار آمد ثابت ہوتے ہیں. مزید (جیسا کہ علیحدگی متغیرات کیلئے عموماً ہوتا ہے)  ہم علیحدگی متغیرات سے حاصل حلوں کو یوں آپس میں جوڑ سکتے ہیں کہ ان سے عمومی حل حاصل کرنا ممکن ہوتا ہے. قابل علیحدگی حل کیلئے درج ذیل ہو گا. 

\begin{align*}
\frac{\partial \Psi }{\partial t} = \psi \frac{d \phi}{d t}, \quad \frac{\partial^{2} \Psi}{\partial x^{2}} = \frac{d^{2} \Psi}{d x^{2}} \phi
\end{align*}

جو سادہ تفرقی مساوات ہے,  انکی مدد سے مساوات شروڈنگر درج ذیل لکھی جا سکتی ہے. 

\begin{align*}
i \hbar \psi \frac{d \phi}{d t} = - \frac{\hbar^{2}}{2m} \frac{d^{2} \psi}{d x^{2}} \phi + V\psi\phi
\end{align*}

\begin{align}
i\hbar\frac{1}{\phi} \frac{d \phi}{d t} = - \frac{\hbar^{2}}{2m} \frac{1}{\psi} \frac{d^{2} \psi}{d x^{2}} + V
\end{align}

اب بایاں ہاتھ تفاعل صرف \( t \) کے تابع ہے جبکہ دایاں ہاتھ تفاعل صرف \( x \) کے تابع ہے. یاد رہے اگر \( V \) از خود \( x \) اور \( t \) دونوں پر منحصر ہو تب ایسا نہیں ہو گا. یوں اگر ہم صرف \( t \) تبدیل کریں, دایاں ہاتھ کسی صورت تبدیل نہیں ہو سکتا. اور چونکہ بایاں ہاتھ اور دایاں ہاتھ ہر صورت ایک دوسرے کے برابر ہیں لحاضہ \( t \) تبدیل کرنے سے بایاں ہاتھ بھی تبدیل نہیں ہو گا. یوں ہم کہہ سکتے ہیں کہ بایاں ہاتھ ایک مستقل کے برابر ہے. اسی طرح اگر ہم صرف \( x \) تبدیل کریں اور \( t \) کو مستقل تصور کریں تب صرف دایاں ہاتھ تبدیل ہونے کا امکان ہے. لیکن دایاں ہاتھ اور بایاں ہاتھ ایک برابر ہے,  جبکہ \( x \) تبدیل کرنے سے بایاں ہاتھ تبدیل نہیں ہوتا لحاضہ دایاں ہاتھ بھی تبدیل نہیں ہو گا. یوں دایاں ہاتھ بھی ایک مستقل کے برابر ہے. اس مستقل کو ہم علیحدگی مستقل کہتے ہیں جسکو ہم \( E \) سے ظاہر کرتے ہیں. یو مساوات 2.3 درج ذیل سے گا. 

\begin{align*}
i\hbar \frac{1}{\phi} \frac{d \phi}{d t} = E
\end{align*}

\begin{align}
\frac{d \phi }{d t} = - \frac{i E}{\hbar} \phi ,
\end{align}

\begin{align*}
- \frac{\hbar^{2}}{2m} \frac{1}{\psi} \frac{d^{2} \psi}{d x^{2}} + V = E,
\end{align*}

\begin{align}
- \frac{\hbar^{2}}{2m} \frac{d^{2} \psi}{dx^{2}} +V\psi = E\psi.
\end{align}

علیحدگی متغیرات نے ایک جزوی تفرقی مساوات کو دو سادہ تفرقی مساوات میں علیحدہ کیا یعنی مساوات 2.4 اور 2.5. ان میں سے پہلی مساوات, یعنی 2.4 کو حل کرنا بہت آسان ہے. دونوں اطراف کو \( dt \) سے ضرب دیتے ہوئے تکمل لیں. یوں عمومی حل \( C e^{-iEt/\hbar} \) حاصل ہو گا. چونکہ ہماری دلچسپی کا حل \( \psi\phi \) ہوگا لہذا ہم مستقل \( C \) کو \( \psi \)  میں ضم کر سکتے ہیں. اور مساوات 2.4 کے حل کو درج ذیل لکھ سکتے ہیں. 

%eq 2.6
\begin{align}
\phi(t) = e^{-iEt/\hbar}.
\end{align}


دوسریساوات, یعنی مساوات 2.5  کو وقت کا غیر تابع شروڈنگر مساوات کہتے ہیں. ہم یہاں مزید آگے بڑھنے سے قاصر ہیں جب تک ہمیں \( V(x) \) پوری طرح معلوم نا ہو. 

اس باب کے بقیہ پورے حصے میں ہم وقت کا غیر تابع شروڈنگر مساوات کو مختلف سادہ خفی توانائی کیلئے حل کریں گے. ایسا کرنے سے پہلے آپ پوچھ سکتے ہیں کہ علیحدگی متغیرات کی کیا مقبولیت ہے. چونکہ وقت کے تابع شروڈنگر مساوات کے زیادہ تر حل \( \psi (x) \phi (t) \)  کی صورت میں نہیں لکھے جا سکتے. میں اس کے تین جوابات دیں سکتا ہوں. ان میں سے دو طبعی اور ایک ریاضیاتی ہو گا. 

یہ ساکن حال ہیں, اگرچہ طفاعل موج ازخود 

%Eq 2.7
\begin{align}
\Pi (x,t) = \psi (x) e^{-iEt/\hbar},
\end{align}

وقت \( t \) کا تابع ہے. کثافت احتمال 

%Eq 2.8
\begin{align}
\left| \Psi (x,t) \right|^{2} = \Psi^{*}\Psi = \psi^{*}e^{+iEt/\hbar} \psi e^{-iEt/\hbar} = \left| \psi (x) \right|^{2},
\end{align}


وقت کا تابع نہیں ہے. چونکہ اس میں وقت کے تابع اجزاء آپس میں کٹ جاتے ہیں. یہی کچھ کسی بھی حرکی متغیر کے حساب میں ہو گا. مساوات 1.39 تخفیف کے بعد درج زیل صورت اختیار کرتی ہے. 

%Eq 2.9
\begin{align}
\langle Q(x,p) \rangle = \int \psi^{*} Q \left( x, \frac{\hbar}{i} \frac{d}{dx} \right) \psi dx.
\end{align}


ہر توقعاتی قیمت وقت میں مستقل ہو گی. یہاں تک کہ ہم \( \phi (t) \) کو رد کر کے \( \Psi \) کی جگہ \( \psi \) استعمال کر کے وہی نتائج حاصل لر سکتے ہیں. اگرچہ بعض اوقات  \( \psi \)  کو ہی تفاعل موج پکارا جاتا ہے, لیکن ایسا کرنا حقیقتاً غلط ہے جس سے مسئلے کھڑے ہو سکتے ہیں. یہ ضروری ہے کہ آپ یاد رکھیں کہ اصل تفاعل موج ہر صورت تابع وقت ہو گا. خاص طور پر \( \langle x \rangle \) ایک مستقل  ہو گا. لھذا مساوات 1.33, \( \langle p \rangle = 0 \) دے گا. ساکن حالت میں کبھی بھی کچھ نہیں ہوتا ہے. 

یہ غیر مبہم کل توانائی کے حال کو ظاہر کرتے ہیں. کلاسیکی میکانیات میں کل توانائی )یعنی حرکی توانائی جمع خفی  توانائی( کو حیملٹونین کہتے ہیں. 

%eq 2.10
\begin{align}
H (x,p) = \frac{p^{2}}{2m} + V(x).
\end{align}


اس کا مطابقتی حیملٹنی عامل, قواعدو ظوابط کے تحت \( p \rightarrow (\hbar/i)(\partial/\partial x \)  پر کر کے درج ذیل حاصل ہو گا. 
%Eq2. 11
\begin{align}
\hat{H} = - \frac{\hbar^{2}}{2m} \frac{\partial^{2}}{\partial x^{2}} + V(x) 
\end{align}

یوں وقت کے غیر تابع شروڈنگر مساوات 2.5 درج ذیل صورت اختیار کرتی ہے. 
% Eq 2.12
\begin{align}
\hat{H} \psi = E\psi
\end{align}

اور کل توانائی کئ توقعاتی قیمت درج ذیل ہو گی. 
2.13 

\begin{align}
\langle H \rangle = \int \psi^{*} \hat{H}\psi dx = E \int \left| \psi \right|^{2} dx = E \int \left| \Psi \right|^{2} dx = E.
\end{align}

آپ دیکھ سکتے ہیں کہ \( \Psi \) کو معمول پہ لانا \( \psi \) کومعمول پر لانے کے مترادف ہے. مزید درج ذیل ہو گا

\begin{align*}
\hat{H}^{2} \psi = \hat{H} (\hat{H}\psi ) = \hat{H} ( E\psi ) = E (\hat{H} \psi ) = E^{2} \psi ,
\end{align*}

جسکی بنا پر درج ذیل ہوگا 

\begin{align*}
\langle H^{2} \rangle = \int \psi^{*} \hat{H}^{2} \psi dx = E^{2} \int \left| \psi \right|^{2} dx = E^{2}.
\end{align*}

یوں\( H \) کی تغیریت درج ذیل ہو گی. 
%Eq 2.14 
\begin{align}
\sigma^{2}_{H} = \langle H^{2} \rangle - \langle H \rangle^{2} E^{2} - E^{2} = 0.
\end{align}

اب یاد رہے اگر \( \sigma = 0 \)  ہو ,تب نمونوں کا ہر جزو ٹھیک ایک جیسی قیمت دے گا. یعنی تقسیم کا پھیلاؤ صفر ہو گا. نتیجتاً قابل علیحدگی حل کی ایک خاصیت یہ ہو گی کہ کل توانائی کی ہر پیمائش یقیناً ایک ہی قیمت دے گی جو \( E \)  کے برابر ہو گی. اسی بنا پر ہم نے علیحدگی مستقل کو \( E \)  سے ظاہر کیا تھا. 



عمومی حل قابل علیحدگی حلوں کا خطی جوڑ ہو گا جیسا ہم جلد دیکھیں گے, وقت کا غیر تابع شروڈنگر مساوات یعنی مساوات 2.5 لامتناعی تعداد کے حل \( \psi_{1}(x),\, \psi_{2}(x),\, \psi_{3}(x), \cdots \)  دے گا. جہاں ہر ایک کے ساتھ قیمت منسلک ہو گی. اور ایک علیحدگی مستقل منسلک ہو گا. \( E_{1}, \, E_{2}, \, E_{3}, \, \cdots \) . لحضہ ہر جائز توانائی ایک منفرد تفاعل موج دے گا. 

\begin{align*}
\Psi_{1} (x,t) = \psi_{1}(x)e^{-iE_{1}t/\hbar} , \quad \Psi_{2} (x,t) = \psi_{2}(x)e^{-iE_{2}t/\hbar}, \, \cdots 
\end{align*}

اب جیسا کہ آپ خود تصدیق کر سکتے ہیں کہ وقت کے غیر تابع شروڈنگر مساوات یعنی مساوات 2.1 , کی ایک خاصیت یہ ہے کہ اس کے حلوں کا ہر خطی جوڑ ازخود مساوات کا حل ہو گا. ایک بار قابل علیحدہ حل تلاش کرنے کے بعد ہم عمومی حل کی درج ذیل صورت بنا سکتے ہیں. 

% Eq2.15 
\begin{align}
\Psi (x,t) = \sum_{n=1}^{\infty} c_{n} \psi_{n}(x)e^{-iE_{n}t/\hbar}.
\end{align}





حقیقت میں وقت کے تابع شروڈنگر مساوات کے ہر حل کو ہم درج بالا صورت میں لکھ سکتے ہیں. ایسا کرنے کیلئے ہمیں مخصوص مستقل \( c_{1},\, c_{2}, \, \cdots \) وغیرہ تلاش کرنے ہوں گے. تا کہ ابتدائی معلومات پر یہ حل پورا اترے. آپ آنے والے حصوں میں دیکھیں گے کہ ہم کا طرح یہ سب کچھ کر پائیں گے. باب 3 میں  ہم اس کو زیادہ مضبوط بنیادوں پر رکھ پائیں گے. بنیادی نقطہ یہ ہے کہ ایک بار وقت کا غیر تابع شروڈنگر مساوات حل کرنے کے بعد آپ کے مسائل ختم ہو جاتے ہیں. یہاں سے تابع وقت شروڈنگر مساوات کا عمومی حل حاصل کرنا آسان کام ہے. 


گذشتہ چار صفحوں میں ہم بہت کچھ کہہ چکے ہیں. میں ان کو مختصراً اور مختلف نقطہ نظر سے دوبارہ پیش کرتا ہوں. آپ لو حس عمومی مسئلے سے واسطہ ہو گا اس میں دیے گئے خفی توانائی \( V(x) \) جو وقت کا تابع نہیں ہے اور آپ کو ابتدائی طفاعل موج \( \Psi (x,0) \) دی گئی ہو گی. آپ کا کام \( \Psi (x,t) \) تلاش کرنا ہوگا جو مستقبل کے تمام \( t \)  کیلئے درست ہو. ایسا کرنے کی خاطر آپ کو تابع وقت شروڈنگر مساوات یعنی مساوات 2.1 حل کرنا ہو گا. ایسا کرنے کی خاطر آپ سب سے پہلے وقت کا غیر تابع شروڈنگر مساوات یعنی مساوات 2.5 کو حل کریں گے. حس سے آپ کے پاس لا متناہی تعداد کے حلوں کا سلسلہ  \( \psi_{1}(x),\, \psi_{2}(x),\, \psi_{3}(x), \cdots \)  حاصل ہوں گے جہاں ہر ایک کی منفرد توانائی  \( E_{1}, \, E_{2}, \, E_{3}, \, \cdots \)  ہو گی. \( \Psi (x,0) \) پر ٹھیک بیٹھنے کی خاطر آپ عمومی مساوات کو ان تمام کے خطی جوڑ کی صورت میں لکھیں گے. 

%Eq 2.16 
\begin{align}
\Psi (x,t) = \sum_{n=1}^{\infty} c_{n} \psi_{n}(x)
\end{align}



یہاں کمال کئ بات یہ ہے کہ کسی بھی ابتدائی حال کی صورت میں آپ ہر صورت \( c_{1}, \,c_{2}, \,c_{3}, \cdots \) مستقل حاصل کر پائیں گے. \( |psi (x,t) \)  حاصل کرنے کی خاطر آپ ہر جزو کے ساتھ اپنے تابع وقت جزو \( e^{-iEt/\hbar} \) کو منسلک کرتے ہیں. 

%Eq 2.17 
\begin{align}
\Psi (x,t) = \sum_{n=1}^{\infty} c_{n} \psi_{n}(x)e^{-iE_{n}t/\hbar} = \sum_{n=0}^{\infty} c_{n} \Psi_{n} (x,t).
\end{align}

قابل علیحدگی حل از خود 

%Eq 2.18 
\begin{align}
\Psi_{n} (x,t) = \psi_{n}(x) e^{-iE_{n}t/\hbar},
\end{align}

ساکن حال ظاہر کرتے ہیں. یعنی ان کے تمام احتمال اور توقعاتی قیمتیں وقت کی تابع نہیں ہو گی. اگرچہ عمومی حل یعنی مساوات 2.17 یہ خاصیت نہیں رکھتا.
چونکہ انفرادی ساکن حالات کئ توانائیاں ایک دوسرے سے مختلف ہوں گی, لحضہ \( \left| \Psi \right|^{2} \) کا حساب کرتے وقت قوت نمائی ایک دوسرے کو نہیں کاٹتی ہیں. 


مثال 2.1 

فرض کریں کہ ایک ذرہ ابتدائی طور پر دو ساکن حالات کا خطی جوڑ ہو 
\begin{align}
\Psi (x,t) = c_{1} \psi_{1}(x) + c_{2} \psi_{2}(x) 
\end{align}


چیزوں کو سادہ رکھنے کی خاطر میں فرض کرتا ہوں کے مستقل \( c_{n} \) اور حالات \( \psi_{n} (x) \) حقیقی ہیں. مستقبل وقت t کیلئے تفاعل موج \( \Psi (x,t) \) کیا ہو گا ؟ کثافت احتمال تلاش کریں اور ذرے کی حرکت بھی بیان کریں. 

Solution 

اسکا پہلا حصہ آسان ہے 

\begin{align*}
\Psi (x,t) = c_{1} \psi_{1}(x)e^{-iE_{1}t/\hbar} + c_{2} \psi_{2}(x)e^{-iE_{2}t/\hbar}
\end{align*}

جہاں \( E_{1} \) اور \( E_{2} \) , تفاعل \( \psi_{1} \) اور \( \psi_{2} \) سے منسلک توانائیاں ہیں. یوں درج ذیل ہو گا. 

\begin{align*}
\left| \Psi (x,t) \right|^{2} &= \left( c_{1} \psi_{1} e^{iE_{1}t/\hbar} + c_{2} \psi_{2} e^{iE_{2}t/\hbar} \right) \left( c_{1} \psi_{1} e^{-iE_{1}t/\hbar} + c_{2} \psi_{2} e^{-iE_{2}t/\hbar} \right) \\
&= c_{1}^{2} \psi_{1}^{2} + c_{2}^{2} \psi_{2}^{2} + 2c_{1}c_{2}\psi_{1}\psi_{2} \cos [ ( E_{2} - E_{1})t/\hbar].
\end{align*}

