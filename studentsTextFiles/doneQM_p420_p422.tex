
\باب{پس نوشت}
اب چونکہ میں توقع کرتا ہوں آپ کوانٹم میکانیات کو سمجھتے ہیں ہم \حوالہء{حصہ \num{1.2}} میں کیا گیا سوال دوبارہ اٹھاتے ہیں کوانٹم میکانیات کے نتائج سے کیا مطلن اغز کرنا چاہیئے مسئلہ کا جڑ تفاعل موج کے ساتھ وابستہ شماریتای مفہوم کی عدم تعینیت ہے۔ تفاعل \عددی{\ساے} یا کوانٹم حال کہنا بہتر ہوگا جو مثال کے طور پر چکرکار ہو سکتا ہے صرف ممکنہ نتائج کی شماریاتی تقسیم مہیا کرتا ہے اور کسی بھی پیمائش کا نتیجہ یکتا طور پر تعین نہیں کرتا اس سے ایک اہم سوال کھڑا ہوتا ہے کیا پیمائش سے قبل نظام یہ مخصوص خاصیت حقیقتاً رکھتا تھا جسے حقیقت پسند نقطہ نظر کہتے ہیں یا پیمائش کے عامل نے اس خاصیت کو جنم دیا جو تافعل موج کی شماریاتی پابندی کو مطمعن کرتا ہے۔ تقلید پسند نقطہ نظر یا ہم اس سوال کو ان بنیادوں پر رد کرتے ہیں کہ یہ سوال ایک فرضی سوال ہے انکاری نقطہ نظر۔

حقیقت پسند کے نقطہ نظر سے کوانٹم میکانیات ایک نا مکمل نظریہ ہے چونکہ کوانٹم میکانیات کی تمام فراہم کردہ معلومات یعنی اس کا تفاعل موج جانتے ہوئے آپ خواص تعین نہیں کر سکتے ہیں۔ ظاہر ہے ایسی صورت میں کوانٹم میکانیات سے باہر کوئی اور معلومات ہوگی جس کو \عددی{\ساے} کے ساتھ ملا کر طبعی حقائق کو مکلم طور پر بیان کرنا ممکن ہوگا۔

تقلید پسند نقطہ نظر اس سے بھی زیادہ سنگین سوالات کھڑے کرتا ہے چونکہ اگر پیمائشی عمل نظام کو ایک خاصیت اختیار کرنے پر مجبور کرتا ہو تب پیمائش ایک عجیب عمل ہوگا ساتھ ہی یہ جانتے ہوئے کہ ایک پیمائش کے فوراً بعد دوسری پیمائش وہی نتیجہ دیتی ہے ہمیں ماننا ہوگا کہ پیمائشی عمل تفاعل موج کو یوں منحداً کرتا ہے جو مساوات شروڈنگر کی تجویز کردہ ارتقا کے برعکس ہے۔

ان سب کی روشنی میں ہم دیکھ سکتے ہیں کہ نسل در نسل ماہرِ طبیعیات انکاری سوچ کے پیچھے پنا لینے پر مجبور کیوں ہوئے اور اپنے شاگردوں کو نصیحت کرتے رہے کہ نظریہ کے تصوراتی بنیادوں پر غور و فکر کر کے اپنا وقت ضائع نہ کریں۔
\حصہ{آئنسٹائن پوڈلسکیو روزن تضاد}
سن \num{1935} میں آئنسٹائن پوڈلسکی اور روزن نے مل کر آئنسٹائن پوڈلسکی اور روزن تضاد پیش کیا جسکا مقصد خالصتاً نظریاتی بنیادوں پر یہ ثابت کرنا تھا کہ صرف حقیقت پسندانا نقطہ نظر درست ہوسکتا ہے۔ میں اس تضاد کی ایک سادہ روپ جو داؤد بام نے پیش کی پر تبصرہ کرتا ہوں۔ تادیلی پاے میزان کی ایک الیکٹران اور ایک پرٹون میں تحلیل پر غور کریں
\begin{align*}
	\pi^0\to e^{-}+e^{+}
\end{align*}
ساکن پائون کی صورت میں الیکٹران اور پروٹان ایک دوسرے کے مخالف رخ جائیں گے \حوالہء{شکل \num{12.1}}۔ اب چونکہ پائون کا چکر صفر ہے لحاظہ زاویائی معیارِ حرکت کی بقا کے تحت یہ الیکٹران اور پوزیٹران یکتا تنظیم میں ہوں گے
\begin{align}
	\frac{1}{\sqrt{2}}(\uparrow_{-}\downarrow_{+}-\downarrow_{-}\uparrow_{+})
\end{align}
اگر دیکھا جائے کہ الیکٹران ہم میدان ہے تب پوزیٹران لاظماً خلافِ میدان ہوگا اور اسی طرح اگر الیکٹران خلاف میدان پایا جائے تب پوزیٹران ہم میدان ہوگا۔ کوانٹم میکانیات آپ کو یہ بتناے سے قاصر ہے کہ کس پایون تحویل میں آپ کو کونسی صورت حال ملے گی تاہم کوانٹم میکانیات یہ ضرور بتا سکتی ہے کہ ان پیمائش کا ایک دوسرے کے ساتھ تعلق ہوگا اور اوسطاً نصف وقت ایک قسم اور نصف وقت دوسری قسم کی جوڑیاں پیدا ہوں گے۔ اب فرض کریں ہم ان الیکٹران اور پوزیٹران کو ایک عملی تجربہ کے لیئے دس میٹر تک جانے دیں یا اصولاً دس نوری سال تک جانے دیں اور اس کے بعد الیکٹران کے چکر کی پیمائش کریں۔ فرض کریں آپ کو ہم میدان ملتا ہے۔ آپ فوراً جان پائیں گے کہ بیس میٹر یا بیس نوری سال دور کوئی دوسرا شخص پوزیٹران کو خلاف میدان پائے گا۔

حقیقت پسند کے نقطہ نظر سے اس میں کوئی حیرانی کی بات نہیں ہے چونکہ انکی پیدائش کے وقت سے ہی الیکٹران حقیقتاً ہم میدان اور پوزیٹران خلاف میدان تھے ہاں کوانٹم میکانیات ان کے بارے میں جاننے سے قاصر تھا۔ تاہم تقلید پسند نقطہ نظر کے تحت پیمائش سے قبل دونوں ذرات نہ ہم میدان اور نہ ہی خلاف میدان تھے الیکٹران پر پیمائش تفاعل موج کو منحداً کرتی ہے جو فوراً بیس میٹر یا بیس نوری سال دور پوزیٹران کو خلاف میدان بناتا ہے۔ آئنسٹائن پوڈلسکی اور روزن اس قسم کے دور عمل کرنے والے عوامل میں یقین نہیں رکھتے تھے۔ یوں انہوں نے تقلید پسند نقطہ نظر کو ناقابلِ قبول قرار دیا چاہے کوانٹم میکانیات جانتا ہو یا نہ جانتا ہو الیکٹران اور ہوزیٹران لاظماً کسی مخصوص چکر کے حامل تھے۔

ان کی دلیل اس بنیادی مفروضہ پر کھڑی ہے کہ کوئی ھبی اثر روشنی کی رفتار سے تیز سفر نہیں کرسکتا ہے۔ ہم اسے اصول مقامیت کہتے ہیں۔ آپ کو شبہ ہوسکتا ہے کہ تفاعل موج کی انہدام کی خبر کسی متناہی سمتی رفتار سے سفر کرتی ہے۔ تاہم ایسی صورت میں زاویائی معیارِ حرکت کی بقا متمعن نہیں ہوگی چونکہ پوزیٹران تک انہدام کی خبر پہنچنے سے پہلے اگر ہم اس کے چکر کی پیمائش تو ہمیں دونوں اقسام کے چکر پچاس پچاس فیصد احتمال سے حاصل ہوں گے۔ آپ کا نظریہ جو بھی کہے تجربات کے تحت دونوں کے چکر ہر صورت ایک دوسرے کے مخلاف ہوتے ہیں۔ ظاہر ہے تفاعل موج کا انہدام یک دم ہوتا ہے۔

