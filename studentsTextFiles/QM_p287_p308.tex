\documentclass{book}
\usepackage{fontspec}
\usepackage{makeidx}
\usepackage{amsmath}                                                                         %\tfrac for fractions in text
\usepackage{amssymb}    
\usepackage{gensymb}  
\usepackage{amsthm}      						%theorem environment. started using in the maths book
\usepackage{mathtools}
\usepackage{multicol}
\usepackage{commath}									%differentiation symbols
\usepackage{polyglossia}    
\setmainlanguage[numerals=maghrib]{arabic}     %for english numbers use numerals=maghrib, for arabic numerals=arabicdigits
\setotherlanguages{english}

\newfontfamily\arabicfont[Scale=1.0,Script=Arabic]{Jameel Noori Nastaleeq} 
\setmonofont{DejaVu Sans Mono}                                                                  %had to add this and the next line to get going after ubuntu upgrade
\let\arabicfontt\ttfamily                                                                                  %had to add this and the above line to get going after ubuntu upgrade
\newfontfamily\urduTechTermsfont[Scale=1.0,Script=Arabic]{AA Sameer Sagar Nastaleeq Bold}
\newfontfamily\urdufont[WordSpace=1.0,Script=Arabic]{Jameel Noori Nastaleeq}
\newfontfamily\urdufontBig[Scale=1.25,WordSpace=1.0,Script=Arabic]{Jameel Noori Nastaleeq}
\newfontfamily\urdufontItalic[Scale=1.25,WordSpace=1.0,Script=Arabic]{Jameel Noori Nastaleeq Italic}
\setlength{\parskip}{5mm plus 4mm minus 3mm}
\begin{document}
سوال6.32:
\\
فرض کریں ایک مخصوص کوانٹم نظام کا  
Hamiltonian 
کسی مقدار معلوم
\(\lambda\)
 کا تفعال ہو .
\(H(\lambda)\)
کے امتیازی اقدار کو اور امتیازی تفعالات
\(E_{n}(\lambda)\)
اور
\(\psi_{n}(\lambda)\)
لیں۔ 
مسلہ Feynman-Hellmann درج ذیل کہتا ہے
\[\frac{\partial E_{n}}{\partial \lambda}=\big\langle{\psi_{n}|\frac{\partial{H}}{\partial{\lambda}}|\psi_{n}}\big\rangle\]
جہاں 
\(E_{n}\)
کو غیر انحطاطی تصور کریں اور اگر انحطاطی ہوں تب تمام 
\(\psi_{n}\)
کو انحطاطی امتیازی تفعالات کے موضوع خطی جوڑ تصور کریں۔\\
(جزو الف):مسلہ Feynman-Hellmann ثابت کریں۔(اشارہ : مسلہ 6.9 استمال کریں ۔ )\\
(جزو ب): درج ذیل یقبودی هارمونی مدار اسکا اطلاق کریں۔\\
ایک)\\
\(\lambda=\omega\)\\
لیں جس سے
V
کی توقعاتی قیمت کا کلیہ اخذ ہوگا ۔\\
دو)\\
\(\lambda=\hslash\)\\
لیں جو
\(\langle T \rangle\)
دے گا اور\\
تین)\\
\(\lambda=m\)\\
جو 
\(\langle T \rangle\)
اور 
\(\langle V \rangle\)
کے درمیان رشتہ دے گا۔اپنے جوابات کا سوال2.12 اور مسلہ virial کی پیشنگویوں  کے ساتھ موعازنا کریں ۔\\
سوال 6.33 : 
\\
مسلہ Feynman-Hellmann استعمال کرتے ہوے ھاٰےڈروجنکے لئے 
\(1/r\)
اور 
\(1/r^{2}\)
کی توقعاتی قیمتیں تین کی جا سکتی ہیں راداسی تفعالات امواج کا موثر 
Hamiltonian 
مساوات 4.53 درج ذیل ہے :
\[H=-\frac{\hslash^{2}}{2m}\frac{d^{2}}{dr^{2}}+\frac{\hslash^{2}}{2m}\frac{l(l+1)}{r^{2}}-\frac{e^{2}}{4\pi\epsilon}\frac{1}{r}\]
اور امتیازی اقدار جنہیں 
L
کی صورت میں لکھا گیا ہے مساوات 4.70 درج ذیل ہونگے 
\[E_n=-\frac{me^{4}}{32\pi^{2}\epsilon^{2}\hslash^{2}(j_{max}+l+1)^{2}}\]
(جزو الف):\\
 مسلہ Feynman-Hellmann میں 
\(\lambda=e\)
استعمال کرتے ہوے
\(\langle1/r\rangle\)
تلاش کریں۔ اپنے نتیجے کی تصدیق مساوات 6.55 کے ساتھ کریں۔\\
(جزو ب) :\\
\(\lambda=l\)
کو استعمال کرتے ہوے
\(\langle1/r^{2}\rangle\)
تلاش کریں۔ اپنے نتیجے کی تصدیق مساوات 6.56 کے ساتھ کریں۔\\
سوال 6.34 :\\
 رشتہ Kramers' 
\[\frac{s+1}{n^{2}}\langle r^{s}\rangle -(2s+1)a\langle r^{s-1}\rangle n+\frac{s}{4}[(2l+1)^{2}-s^{2}]a^{2}\langle r^{s-2}\rangle =0\]
صابط کریں جو ھاٰےڈروجنکےحال 
\(\psi_{nlm}\)
میں الیکٹران کے لئے R کی توقعاتی قیمتوں کی تین مختلف طاقتوں
\((s,s-1,\)
اور
\(s-2)\)
کا۔ تعلق پیش کرتا ہے۔ اشارہ : راداسی مساوات 4.53 کو درج ذیل روپ میں لکھ کر
\[u''=\big[\frac{l(l+1)}{r^{2}}-\frac{2}{ar}+\frac{1}{n^{2}a^{2}}\big]u.\]
۔
\(\int(ur^{s}u'')dr\)	 
کو 
\(\langle r^{s}\rangle\)
،
\(\langle r^{s-1}\rangle\)
،
\(\langle r^{s-2}\rangle \)
کی  صورت میں لکھیں اسکے بعد تکامل
bilhisis
کے ذریےدوہرا تفروق کو بیٹھایں۔ دیکھایں کے\\
\[\int(ur^{s}u')=-(s/2)<r^{s-1}>\]
اور
\[\int(u'r^{s}u')dr=-[2/(s+1)]\int(u''r^{s+1}u')dr\]
ہوگا اسی کو لے کر آگے چلیں)\\
سول 6.35\\ 
(جزو الف ):\\
 رشتہ Kramers' مساوات 6.104 میں
\(s=0,s=1,s=2\)
اور 
\(s=3\)
ڈال کے 
\(\langle r^{-1}\rangle,\langle r\rangle,\langle r^{2}\rangle\)
اور 
\(\langle r^{3}\rangle\)
کے قلیات حاصل کریں۔ آپ دیکھ سکتے ہیں کے آپ اس طرح چلتے ہے کسی بھی مثبت طاقت کے لئے قلیہ دریافت کر سکتے ہیں ۔\\
(جزو ب):\\ 
دوسرے رخ آپکو مثلا درپیش  ہوگا آپ 
\(s=-1\)
پر کرکے دیکھیں کے آپکو صرف 
\(\langle r^{-2}\rangle\)
اور 
\(\langle r^{-3}\rangle\)
کے بیچ رشتہ حاصل ہوگا ۔\\
جزو ج:\\
 اگر آپ کسی طریقہ سے
\(\langle r^{-2}\rangle\)
دریافت کر پایں تب آپ رشتہ Kramers' استعمال کرکے باکی تمام منفی قوعتوں کے لئے قلیات دریافت کر سکتے ہیں ۔\\
مساوات 6.56 : جسے سوال 6.33 میں اخذ کیا گیا ہے اسے استعمال کرتے ہوے 
\(\langle r^{-3}\rangle\)
تعین کریں اور اپنے نتیجہ کی تصدیق مساوات 6.64 کے ساتھ کریں۔\\
سوال 6.36: \\
ایک جوہر کو يقسا بیرونی برقی میدان
\(E_{ext}\)
میں رکھنے سے توانائی کی سطحیں بٹتی ہیں جسے سٹارک اثر کہا جاتا ہے اور جو zemann اثر کا برقی مماسل ہے اس سوال میں ہم ھاٰےڈروجن کے 
\(n=1\)
اور 
\(n=2\)
حالات کے لئے سٹارک اثر کا تجزیہ کرتے ہیں۔ فرض کریں میدان
Z
رخ ہے لہٰذا الیکٹران کی مخفی توانائی درج ذیل ہوگی: 
\[H'_{S}=eE_{ext}z=eE_{ext}r\cos{\theta}\]
اسکو bohr hamiltonian  مساوات 6.42 میں اضطراب تصور کریں اس مسلہ میں چکر کا کوئی کردار نہیں ہے لہذا اسے نظر انداز کرتے ہوے عمدہ ساخت کو رعد کریں۔\\
(جزو الف ) :\\
 اول رتبہ میں زمینی حال توانائی اس اضطراب سے اثر انداذ نہیں ہوتی ۔\\
(جزو ب) :\\  
پہلا هیجان حال
4
پرتہ
\(\psi_{200},\psi_{211},\psi_{210},\psi_{21-1},\)
انحطاطی نظریہ اضطراب استعمال کرتے ہوے، توانائی کی رتبہِ اول کا سہی تعین کریں ۔توانائی 
\(E_{2}\)
کتنے سطحوں میں بٹے گا؟\\
(جزو ج) :\\
 درج بالہ جزو ب میں موضوع تفعالات موج کیا ہونگے ؟ ان میں سے ہر ایک موضوع حالات میں برقی جوعف قطب میعارِاثر
\((p_{e}=-er)\)
کی توقعاتی قیمت معلوم کریں ۔ آپ دیکھیں گے کہ نتائج  لاگو میدان کے تعابع نہیں ہونگے اس طرح ظاہر ہے کے پہلی هیجان حال میں ھاٰےڈروجن برقی جوعفت قطب میعارِاثر کا حامل ہوگا۔اشارہ:اس سوال میں بہت سارے تاكملات پاے جاتے ہیں تاہم تقریبن تمام کی قیمت سِفر ہے لہذا حساب سے قبل غور کریں اگر 
\(\phi\)
تكمل سفر ہو تب 
\(r\)
اور 
\(\theta\)
تکملات حل کرنے  کی ضرورت نہیں ہوگی جزوی جواب
\[W_{13}=W_{31}=-3eaE_{ext};\]
باقی تمام ارکان سفر ہیں۔ )\\
سوال 6.37:
 ھاٰےڈروجن کی 
\(n=3\)
حالات کے لئے سٹارک  اثر سوال 6.36 پر غور کریں ابتدائ طور پر چکر کو نظر انداز کرتے ہوے اب انحطاطی حالات 
\(\psi_{3lm}\)
ہونگے اور اب ہم 
\(z\)
رخ برقی میدان چالو کرتے ہیں ۔\\
(جزو الف) :\\ 
اضطرابی hamiltonian کو ظاہر کرنے والا 
\(9\times9\)
کا کالم  تیار کریں\\
 جزوی جواب 
\[\langle 300|z|310\rangle =-3\sqrt{6}a,\langle 310|z|320\rangle =-3\sqrt{3}a,\langle 31\pm1|z|32\pm1\rangle=-(9/2)a.\]\\
(جزو ب) :\\ 
امتیازی اقتدار اور انکی انحطاط دریافت کریں .\\
سوال 6.38 :
  ڈوٹرئم
کی زمینی حال میں نہایت موحین منتقلی کے دوران خارج کردہ پھوٹان کا طولِ موج
میں تلاش کریں ۔ ڈوٹرئم درحقیقت بھاری ھاٰےڈروجن ہے جسکے مرکز میں ایک اضافی نوٹران پایا جاتا ہے پروٹان اور نوٹران ساتھ جڑ کر  ڈوٹرئم بناتے ہیں جسکا چکر ایک مقناطیسی دارِاثر
\[\mu_{d}=\frac{g_{d}e}{2m_{d}}S_{d};\]
اور ڈوٹرئم کا 
g-
جزو 1.71 ہے ۔\\
سوال 6.39 :\\
 ایک کالم میں قریبی باردارا کا بجلی میدان جوہر کی توانائی کی سطحوں کو مضطرب کرتا ہے ایک تازہ نمونہ کے طور پر فرض کریں hydrogen جوہر کی پڑوس میں نقطہ باروں کی تین جوڑیاں پای جاتی ہیں شکل 6.15۔(چونکے اس۔ سوال کے ساتھ چکر کا کوئی۔ واستہ نہیں ہے لہٰذا اسے نظرانداز کریں )\\
(جزو الف ):\\
 درج ذیل 
\[r<<d_{1},r<<d_{2},andr<<d_{3},\]
کی صورت میں دیکھاے
\[H'=V_{o}+3(\beta_{1}x^{2}+\beta_{2}y^{2}+\beta_{3}z^{2})-(\beta_{1}+\beta_{2}+\beta_{3})r^{2},\]
جہا درج ذیل ہیں
\[\beta_{i}\equiv-\frac{e}{4\pi\epsilon}\frac{\eta_{i}}{d_{i}^{3}},\quad\quad\]
اور 
\[V_{o}=2(\beta_{1}d_{1}^{2}+\beta_{2}d_{2}^{2}+\beta_{3}d_{3}^{2}).\]\\
(جزو ب) : \\زمینی حال توانائی کی رتباےاول کی تخفیف تلاش کریں ۔\\
(جزو ج) :\\ پہلی۔ هیجان حالات 
\((n=2)\)
کی توانائی کے لئے رتباےاول کی تخفیف تلاش کریں ۔ درجذیل صورتوں میں یہ چار پڑتہ انحطاطی نظام کتنی سطحوں میں بٹے گا ۔\\
ایک ) کابی تشاقلی
\[\beta_{1}=\beta_{2}=\beta_{3},\]
کی۔ صورت میں ۔\\
دو ) چوں زاویہ تشاقلی
\[\beta_{1}=\beta_{2}\neq\beta_{3}:\]
کی صورت میں۔\\
تین ) آرتھو ھامبک تشاقل کی صورت میں تینوں مختلف ہونگیں ۔\\
سوال 6.40 : \\بازاوقات 
\(\psi_{n}^{1}\)
کو غیر مضطرب طفعالات امواج میں پهلاۓ  مساوات 6.11 بغیر مساوات 6.10 کو بلہ واستہ حال کرنا ممکن ہوتا ہے اسکی دو بلخصوص خوبصورت مثالین درج ذیل ہیں۔\\
(الف )\\
ایک) ھاٰےڈروجن کی زمینی حال میں سٹارک اثر ایک یکساں بیرونی برقی میدان 
\(E_{ext}\)
کی۔ موجودگی میں ھاٰےڈروجن  کی زمینی حال کا رتبہ اول تخفیف تلاش کریں ( سوال 6.36 stark اثر دیکھیں ۔) ۔اشارہ : حل کی درج ذیل روپ :\\
\[(A+Br+Cr^{2})e^{-r/n}cos\theta;\]
استعمال کرکے دیکھیں اپ نے مستقلات 
\(A,B,\)
اور 
\(C\)
کی ایسی قیمتیں تلاش کرنی ہیں جو مساوات 6.10 کو مطمئن کرتے ہوں ۔\\
دو ) زمینی حال توانائی کی رتبہ دوم تخفیف مساوات 6.14 کی مدد سے  تعین کریں جیسا اپنے سوال 6.36 (الف ) میں دیکھا رتبہ اول تخفیف سفر ہوگی ۔جواب :\\
\[-m(3a^{2}eE_{ext}/2\hslash)^{2}.\]\\
(جزو ب)\\ اگر پروٹان کا برقی جست قطب میعارِاثر 
\(p\)
ہوتا تب  ھاٰےڈروجن کے  الیکٹرانکی مخفی توانائی درجذیل مقدار سے مضطرب ہوتی۔
\[H'=\frac{epcos\theta}{4\pi\epsilon r^{2}}\]
ایک ) زمینی حال طفعال موج کی رتبی اول تخفیف کو مساوات 6.10 حل کرکے تلاش کریں ۔\\
دو ) دیکھایں کہ رتبہ تک جوہر کا قل برقی جوعفت قطب میعارِاثر حیرت کی۔ بات ہے سفر ہوگا۔ \\
تین ) زمینی حال توانائی کی۔ رتبہ دوم تخفیف مساوات 6.14 سے تعین کریں رتبہ اول تخفیف کیا ہوگا ؟\\
\\

باب 7 اصول تغيوریت حصہ 7.1 نظریہ\\
فرض کریں آپ ایک نظام جس کو ہیملٹنی H بیان کرتا ہو٬ کی زمینی حال توانائی
 \(E_{gs}\)
   کا حساب کرنا چاہتے ہیں لیکن آپ غير تابع وقت شروڈنگر مساوات حاصل کرنے سے قاصر ہوتے ہیں . اصول تغيوریت آپ کو 
  \(E_{gs}\)
   کی بالائی حد دیتا ہے . بعض اوقعات آپ کو صرف اسی سے غرض ہوتا ہے اور عموماً ہوشیاری سے کام لیتے ہوئے آپ بالکل ٹھیک قیمت کے قريب قیمت حاصل کر سکتے ہیں . آئیں اس کا استعمال دیکھے. کوئی بھی معمول شده تفاعل 
\(\psi\)
    لیں ۔ میں درج ذيل دعوه کرتا ہوں:
\[E_{gs}\le \langle \psi |H|\psi\rangle \equiv \langle H \rangle\]
یعنی کسی بھی شائد غير درست حال 
\(\psi\)
 میں H کی توقعاتی قیمت زمینی حال توانائی سے زياده ہوگی. یقیناً اگر 
 \(\psi\) 
 اتفاقاً ایک ہیجان حال ہو تب H کی قیمت
  \(E_{gs}\) 
 سے تجاوز کرے گی. اصل نقطہ یہ ہے کہ کسی بھی تفاعل 
 \(\psi\)
  کے لیے بھی ایسا ہوگا.\\
ثبوت: \\
چونکہ H کے نامعلوم امیتازی تفاعلات مكمل سلسلہ دیتے ہیں. لحاظہ ہم 
\(\psi\) 
کو ان کا خطی جوڑ لکھ سکتے ہیں.\\
جہان
\[\psi=\sum c_{n}\psi_{n}, \quad H\psi_{n}=E_{n}\psi_{n}\] 
ہے .چونکہ 
\(\psi\) 
معمول شده ہے\\
\[1=\langle \psi |\psi\rangle=\big\langle \sum_{m} c_{m}\psi_{m}|\sum_{n}c_{n}\psi_{n}\big\rangle=\sum_{m}\sum_{n}c_{m}^{*}c_{n}\langle \psi_{m}|\psi_{n}\rangle=\sum_{n}\abs{c_{n}}^{2}\]
جہاں فرض کیا گیا ہے کے امتیازی تفاعلات از خد معیاری معمول شدہ ہے.
\[\langle \psi_{m}|\psi_{n}\rangle=\delta_{mn}\]
ساتھ ہی درج ذیل ہوگا
\[\langle H \rangle=\big\langle \sum_{m} c_{m}\psi_{m}|H\sum_{n}c_{n}\psi_{n}\big\rangle=\sum_{m}\sum_{n}c_{m}^{*}E_{n}c_{n}\langle \psi_{m}|\psi_{n}\rangle=\sum_{n}E_{n}\abs{c_{n}}^{2}\]
ليكن تعريف کی رو سے زمینی حال توانائی کم سے کم  امتیازی قیمت ہوگی. لحاظہ
\(E_{gs}\le E_{n}\)
ہوگا. جس کے تحط درج ذیل ہوگا
\[\langle H \rangle \ge E_{gs}\sum_{n}\abs{c_{n}}^{2}=E_{gs}\]
جس کو ہم ثابت کرنا چاہتے تھے.\\
مثال \\7.1
فرض كرے ہم یک بودی ہارمونی مورتیش
\[H=-\frac{\hbar^{2}}{2m}\frac{\dif^{2}}{\dif{x^{2}}}+\frac{1}{2}m\omega^{2}x^{2}\]
کی زمینی حال توانائی جاننا چاہتے ہیں . یقیناً ہم اس کا ٹھیک ٹھیک جواب جانتے ہیں. جو مساوات 2.61
\(E_{gs}=(1/2)\hbar\omega\)
جسے استعمال کرکے اس ترقيب کو پرکا جاسکتا ہے. ہم گاوسی تفاعل
 \[\psi(x)=Ae^{-bx^{2}}\] 
کو اپنا پرکیا تفاعل موج منتخب کرتے ہے جہاں b ایک مستقل ہے اور A کو معمول زنی سے تعائن کیا جاسکتا ہے.
\[1=\abs{A}^{2}\int_{-\infty}^{+\infty}e^{-2bx^{2}}\dif{x}=\abs{A}^{2}\sqrt{\frac{\pi}{2b}}\Rightarrow \big (\frac{2b}{pi}\big )^{1/4}\]
اب درج ذیل ہے
 \[ \langle H \rangle=\langle T \rangle + \langle V \rangle\]
جبکہ یہاں
\[\langle T \rangle=-\frac{\hbar^{2}}{2m}\abs{A}^{2}\int_{-\infty}^{+\infty}e^{-bx^{2}}\frac{\dif^{2}}{\dif{x^{2}}}(e^{-bx^{2}}\dif{x}=\frac{\hbar^{2}b}{2m}\]
اور
\[\langle V\rangle=\frac{1}{2}m\omega^{2}x^{2}\abs{A}^{2}\int_{-\infty}^{+\infty}e^{-2bx^{2}}x^{2}\dif{x}=\frac{m\omega^{2}}{8b}\]
ہونے کی بنا درج ذیل ہوگا
\[ \langle H \rangle=\frac{\hbar^{2}b}{2m}+\frac{m\omega^{2}}{8b}\]
/
مساوات 7.1 کے تحط یہ b کی تمام قیمتوں کے لیے
\(E_{gs}\)
 سے تجاوز کرے گا. سخت سے سخت حد بندی کی خاطر ہم
 \( \langle H\rangle\)
 کی کم سے کم قیمت حاصل کرتے ہے.
\[\frac{d}{db}\langle H\rangle=\frac{\hbar^{2}}{2m}-\frac{m\omega^{2}}{8b^{2}}=0\Rightarrow b=\frac{m\omega}{2\hbar}\]
اس کو واپس
 \( \langle H\rangle \)
میں پُھر کرتے ہوئے درج ذیل حاصل ہوگا.
 \[\langle H\rangle _{min} =\frac{1}{2}\hbar\omega\]
یہاں ہم بالکل ٹھیک زمینی حال توانائی حاصل کرپائے ہے. جو حیرانی کی بات نہیں ہے چونکہ میں نے اتفاقی طور پر ایسا پرکیا تفاعل منتخب کیا جس کا روپ ٹھیک حقیقی  زمینی حال مثاوات 2.59 کی طرح ہے.   تاہم گاوسی کے ساتھ کام کرنا انتہائی آسان ثابت ہوتا ہے لحاظہ یہ ایک مقبول پرکیا تفاعل ہے. جسے وہاں بھی استعمال کیا جاتا ہے جب حقیقی زمینی حال کے ساتھ اس کی کوئی مشابہت نہ ہو.\\
مثال \\7.2
فرض کرے ہم Delta تفاعل مخفیا
\[H=-\frac{\hbar^{2}}{2m}\frac{\dif^{2}}{\dif{x^{2}}}-\alpha\delta(x)\]
کی زمینی حال توانائی جاننا چاہتے ہے. یہاں بھی ہمیں ٹھیک جواب
\(E_{gs}=-m\alpha^{2}/2\hbar^{2}\)
معلوم ہے. یہاں بھی ہم گاوسی پرکیا تفاعل مساوات 7.2 کا انتخاب کرتے ہیں. چونکہ ہم معمول زنی کرچکے ہے اور  
\(\langle T\rangle\)
 کا حساب کرچکے ہے لحاظہ ہمیں صرف درجہ ذیل کرنا ہوگا
\[\langle V\rangle=-\alpha\abs{A}^{2}\int_{-\infty}^{+\infty}e^{-2bx^{2}}\delta(x)\dif{x}=-\alpha\sqrt{\frac{2b}{\pi}}\]
ظاہر ہے کے درج ذيل ہوگا
 \[\langle H\rangle=\frac{\hbar^{2}b}{2m}-\alpha\sqrt{\frac{2b}{\pi}}\]
اور ہم جانتے ہے کے یہ تمام B  کے لیے یہ
\( E_{gs}\)
 سے تجاوز کرے گا. اس کی کم سے كم قیمت تلاش کرتے ہے\\
\[\frac{d}{db}\langle H\rangle=\frac{\hbar^{2}}{2m}-\frac{\alpha}{\sqrt{2\pi b}}=0\Rightarrow b=\frac{2m^{2}\alpha^{2}}{\pi\hbar^{4}}\]
لحاظہ درجہ ذیل ہوگا 
 \[\langle H\rangle _{min} =-\frac{-m\alpha^{2}}{\pi\hbar^{2}}\]
جو کہ یقناً
\( E_{gs}\)
 سے یہ قدرے بلند ہوگا، چونکہ
\(\pi>2\) \\
میں نے کہا آپ کسی بھی معمول شده پرکیا تفاعل 
\(\psi\) 
کا انتخاب کر سکتے ہے جو ایک لحاظ سے درست ہے. البتہ غیر استمراری تفاعلات کے دوہرہ تفرق جو
\( \langle T \rangle\)
کی قیمت حاصل کرنے کے لیے درکار ہوگا٬ کو معنی خیز مطلب مختص کرنے کے لیے انوکے چال چلنا ہوگا. ہاں, اگر آپ محتاط ہو تو استمراری تفاعلات جن میں بل پائے جاتے ہو٬ کو استعمال کرنا نسبتاً آسان ہوگا. اگلی مثال میں انہیں استعمال کرنا دکھایا گیا ہے.\\
مثال \\7.3
دکونی پرکیا تفاعل موج شکل 7.1
\[\psi(x)=\begin{cases}
Ax & 0\le x\le a/2\\
A(a-x) & a/2\le x\le a\\
0 & otherwise\\
\end{cases}\]
استعمال کرتے ہوئے یک بودی لا متناہی چکور كواں کی زمینی حال توانائی کی بالائی حد بندی تلاش کرے. A کو معمول زنی سے تعائن کیا جائے گا.
\[1=\abs{A}^{2}\big [\int_{0}^{a/2} x^{2}\dif{x}+\int_{a/2}^{a}(a-x)^{2}\dif{x} \big ]=\abs{A}^{3}\frac{a^{3}}{12}\Rightarrow A=\frac{2}{a}\sqrt{\frac{3}{a}}\]
جیسا شکل 7.2 میں دکھایا گیا ہے یہاں درجہ ذيل ہوگا
\[\frac{\dif{\psi}}{\dif{x}}=\begin{cases}
A & 0<x<a/2\\
-A & a/2<x<a\\
0& otherwise\\
\end{cases}\]
دیگر صورت. اب سیڑھی تفاعل کا تفرق ایک Delta تفاعل ہے. سوال 2.24 ب دیکھے.
\[\frac{\dif^{2}{\psi}}{\dif{x^{2}}}=A\delta(x)-2A\delta(x-a/2)+A\delta(x-a)\]
لحاظہ درج ذیل ہوگا 
\begin{align*}
\langle H \rangle &= -\frac{\hbar^{2}A}{2m}\int[\delta(x)-2A\delta(x-a/2)+\delta(x-a)]\psi(x)\dif{x}\\
&= -\frac{\hbar^{2}A}{2m}[\psi(0)-2\psi(a/2)+\psi(a)]=\frac{\hbar^{2}A^{2}a}{2m}=\frac{12\hbar^{2}}{2ma^{2}}\\
\end{align*}
ٹھیک زمینی حال توانائی
\(E_{gs}=\frac{\pi^{2}\hbar^{2}}{2ma^{2}}\)
مساوات 2.27 ہے. لحاظہ یہ مثلہ کار آمد ہے. 
\(12>\pi^{2}\)\\

اصول تغيوریت انتہائی طاقتور اور استعمال کے نقطہ نظر سے شرمناک حد تک آسان ہے. کسی پیچده سالمہ کی زمینی حال توانائی جاننے کی خاطر ماہر کیمیا ایک ایسا پرکیا تفاعل موج منتخب کر کے٬ جس میں متعدد مقدار معلوم پائے جاتے ہو اور ان کی قیمتیں تبديل کرتے ہوئے
\(  \langle H \rangle\)
کی کم سے کم ممکنہ قیمت تلاش کرے گا. اصل تفاعل موج کے ساتھ
\(\psi\) 
کی کوئی مشابہت نہ پائے جانے کی صورت میں بھی آپ کو
\( E_{gs}\)
 کی حیرت کن حد تک درست قیمت حاصل ہوگی۔ ظاہر ہے اگر آپ
\(\psi\) 
 کو حقیقی تفاعل کے زیادہ قریب منتخب کر پائے تو اتنا بہتر ہوگا. اس ترقيب کے ساتھ مسلہ یہ ہے کہ آپ کبھی بھی جان نہیں سکتے کہ آپ درست جواب کے کتنے قريب ہو. آپ صرف اتنا جانتے ہو کہ اصل جواب آپ کے نتیجہ سے كم ہوگا. مزید اس روپ میں یہ ترقيب صرف زمینی حال کے لیے کارآمد ہے.  البتہ سوال 7.4 دیکھے.\\
سوال  \\7.1
درجہ زیل مخفیہ کے لئے زمينی حال توانائی جاننے کی خاطر  گاوسی پرکیا تفاعل\\:
مساوات 7.2 کی كم سے كم بالائی حد بندی تلاش كرے.\\
ا) خطی مخفیہ
\[V(x)=\alpha\abs{x}\]
ب) طاقت چار مخفیہ
\[V(x)=\alpha x^{4}\]
سوال \\7.2
 یک بودی ہارمونی مورتیش 
 \(E_{gs}\)
  کی بہترین حد بندی کو درج ذیل روپ کی پرکیا تفاعل موج
\[\psi(x)=\frac{A}{x^{2}+b^{2}}\]
استعمال کرکے تلاش كریں. جہاں معمول زنی سے تعائن ہوگا. جبکہ بھی قابل تبديل مقدار معلوم ہے.
 سوال 7.3 : \\
 ڈلتا تفاعل مخفیہ
\[-\alpha\delta(x)\]
کی
\(E_{gs}\) 
کی بہترین بالائی حد بندی کو دکونی پرکیا تفاعل مساوات 7.10. لیکن جس کا وسط مبده پر ہو استعمال کرکے تلاش كريں. یہاں a ایک قابل تبديل مقدار معلوم ہے.\\
سوال \\7.4
ا) اصول تغيوريت کے درج ذیل زمنی نتیجہ کو ثابت کریں. اگر
\(\langle \psi | \psi_{gs} \rangle =0\)
تب
\(\langle H \rangle \ge E_{fc}\)
ہوگا. جہاں پہلی ہیجان حال کی توانائی
 \(E_{fc}\)
 ہے. یوں اگر ہم کسی طرح ٹھیک زمینی حال کو امودی ایک پرکیا تفاعل تلاش كر سکے تب ہم پہلی ہیجان حال کی بالائی حد بندی جان سکتے ہیں. عموماً چونکہ ہم زمینی حال تفاعل
 \(\psi_{gs}\)
  کو نہیں جانتے ہے لحاظہ یہ کہنا مشکل ہوگا کہ ہمارا پرکی تفاعل 
 \(\psi\)
   اس کو امودی ہوگا. ہاں٬ اگر x کے لحاظ سے مخفی
  \(V(x)\)
     ایک جفت تفاعل ہو تب زمینی حال بھی جفت ہوگا. لحاظہ کوئی بھی تاگ پرکیا تفاعل خود بخود اس زمنی نتیجہ کے شرط پر پورا اترے گا.\\
ب) درج ذیل پرکیا تفاعل
\[\psi(x)=Axe^{-bx^{2}}\]
استعمال کرتے ہوئے یک بودی ہارمونی مورتیش کی پہلی ہیجان حال کا بہترین بالائی حد بندی تلاش كرے.\\
سوال 7.5        \\
ا) اصول تغيوریت استعمال کرکے ثابت کریں کہ رتبہ اول غير انحطاطی نظریہ استراب ہر صورت زمینی حال توانائی کی قیمت سے تجاوز کرے گا يا كم سے كم کبھی بھی اس سے كم قیمت نہیں دے گا.\\
ب) آپ جز آ جانتے ہوئے توقع کریں گے کہ زمینی حال کی دو رتبی تصحیح لاظمن منفی ہوگی. مساوات 6.15 کا معائنہ كرتے ہوئے تصدیق كريں کہ ایسا ہی ہوگا. \\
\\
حصہ \\7.2
ہیلیم کا زمینی حال\\ 
ہیلیم جوہر کے مرکزا میں دو پروٹون اور دو نیوٹران جن کا یہاں کوئی کردار نہیں ہوگا پائے جاتے ہیں اور مرکزا کے گرد مدار میں دو الیکٹران حرکت کرتے ہیں۔\\
شکل \\7.3
مہین ساخت اور باریک طزہی کو نظر انداز کرتے ہوئے اس نظام کا ہملٹھنی درج ذیل ہوگا 
\[H=-\frac{\hbar^{2}}{2m}(\nabla_{1}^{2}+\nabla_{2}^{2})-\frac{e^{2}}{4\pi\epsilon_{0}}\big (\frac{2}{r_{1}}+\frac{2}{r_{2}}-\frac{1}{\abs{\vec{r}_{1}-\vec{r}_{2}}}\big )\]
ہم نے زمینی حال توانائی 
 \(E_{gs}\)
  کا حساب کرنا ہوگا ۔ طبی طور پر یہ دونوں الیکٹران اکھاڑنے کے لیے درکار توانائی کو ظاہر کرتا ہے۔ 
 \(E_{gs}\)
   جانتے ہوئے ہم ایک الیکٹران اکھاڑنے کے لیے درکار توانائی برداری عمل معلوم کر سکتے ہیں۔\\
سوال 7.6 دیکھیں\\  
تجربہ گاہ میں ہیلیم کی زمینی حل توانائی کی قیمت کو انتہائی زیادہ درستگی تک پیمائش کیا گیا ہے ۔
\[E_{gs}=-78.975 \text{eV}\]
ہم نظریا سے اسی عدد کو حاصل کرنا چاہنگے۔ یہ تجسس کی بات ہے کہ ابھی تک اتنی سادہ اور اہم مسلے کا ٹھیک حل نہیں ڈھونڈا جا سکا ہے۔
مسلہ الیکٹران الیکٹران دفعہ 
\[V_{ee}=\frac{e^{2}}{4\pi\epsilon_{0}}\frac{1}{\abs{\vec{r}_{1}-\vec{r}_{2}}}\]
پیدا کرتا ہے۔ اس جز کو نظر انداز کرنے سے
 \(H \)
حایڑروجن ہملٹنیو میں الہدگا ہو جاتا ہے جہاں مرکزوی بار e کی بجائے
 \(2e\)
 ہوگا۔ اس کا ٹھیک ٹھیک حل حایڑروجن دفالاج ماج کا حاصل ظرب ہوگا ۔
\[\psi_{0}(\vec{r}_{1},\vec{r}_{2})\equiv \psi_{100}(\vec{r}_{1})\psi_{100}(\vec{r}_{2})=\frac{8}{\pi a^{3}e^{-2(r_{1}+r_{2})/a}}\]
اور توانائی
 \(8E_{1}=-109\text{eV}\)
 الیکٹران وولٹ  
مساوات 5.31 ہوگی ۔ یہ قیمت
 \(-79\text{eV}\) 
 الیکٹران وولٹ سے بہت دور ہے ۔ تاہم یہ صرف آغاز ہے ۔ ہم فایے ناٹ کو بھر کیا افعال معاج لیتے ہوئے  
 \(E_{gs}\)
  کی بہتر تخمیم کو اصول تغیریت سے حاصل کرتے ہیں چونکہ یہ زیادہ تر ہملٹھنی کا امتیازی  دفعال ہے لہذا یہ خصوصی طور پر بہتر انتخاب ہے۔
\[H\psi_{0}=(8E_{1}+V_{ee})\psi_{0}\]
یوں درج ذیل ہوگا۔ 
\[\langle H \rangle=8E_{1}+\langle V_{ee}\rangle\]
جہاں درج ذیل ہے 
\[\langle V_{ee}\rangle=\big(\frac{e^{2}}{4\pi\epsilon_{0}}\big )\big (\frac{8}{\pi a^{3}}\big )^{2}\int \frac{e^{-4(r_{1}+r_{2})/a}}{\abs{\vec{r}_{1}-\vec{r}_{2}}}d^{3}\vec{r}_{1}d^{3}\vec{r}_{2}\]
میں 
\(r_{2}\)
 تکمل کو پہلے حل کرتا ہوں ۔یوں
  \(r_{1}\)
کو مستقل تصور کیا جائے گا ۔\\
ہم 
\(r_{2}\)
 محددی نظام کو یوں رکھتے ہیں کہ
 \(r_{1}\)
  قددی محور پر پایا جاتا ہو ۔\\
قانون قوساین کے تحت 
\[\abs{\vec{r}_{1}-\vec{r}_{2}}=\sqrt{r_{1}^{2}+r_{2}^{2}-2r_{1}r_{2}\cos{\theta_{2}}}\]
لحاضہ درج ذیل ہوگا \\
\[I_{2}\equiv\int\frac{e^{-4r^{2}/a}}{\abs{\vec{r}_{1}-\vec{r}_{2}}}d^{3}r_{2}=\int\frac{e^{-4r^{2}/a}}{\sqrt{r_{1}^{2}+r_{2}^{2}-2r_{1}r_{2}\cos{\theta_{2}}}}r_{2}^{2}\sin{\theta_{2}}dr_{2}d\theta_{2}d\phi_{2}\]
متغیر 
\(\phi_{2}\)
کا تکمل 
\(2\pi\)
 دے گا۔\\
متغیر 
\(\theta_{2}\)
 کا تکمل  درج ذیل ہوگا
\begin{align*}
\int_{0}^{\pi}\frac{\sin{\theta_{2}}}{\sqrt{r_{1}^{2}+r_{2}^{2}-2r_{1}r_{2}\cos{\theta_{2}}}}\dif{\theta_{2}}=\frac{\sqrt{r_{1}^{2}+r_{2}^{2}-2r_{1}r_{2}\cos{\theta_{2}}}}{r_{1}r_{2}}|_{0}^{\pi}
\end{align*}
\begin{align*}
&=\frac{1}{r_{1}r_{2}}(\sqrt{r_{1}^{2}+r_{2}^{2}+2r_{1}r_{2}}-\sqrt{r_{1}^{2}+r_{2}^{2}-2r_{1}r_{2}})\\
&=\frac{1}{r_{1}r_{2}}[(r_{1}+r_{2})-\abs{r_{1}-r_{2}}]=\begin{cases}
2/r_{1} & r_{2}<r_{1}\\
2/r_{2}&r_{2}>r_{1}\\
\end{cases}\\
\end{align*}
یوں درج ذیل ہوگا 
\begin{align*}
I_{2}&=4\pi(\frac{1}{r_{1}}\int_{0}^{r_{1}}e^{-4r_{2}/a}r_{2}^{2}dr_{2}+\int_{r_{1}}^{\infty}e^{-4r_{2}/a}r_{2}dr_{2})\\
&=\frac{\pi a^{3}}{8r_{1}}[1-(1+\frac{2r_{1}}{a})e^{-4r_{1}/a}]\\
\end{align*}
اس طرح 
\(\langle V_{ee} \rangle \)
درج ذیل ہوگا ۔
\[(\frac{e^{2}}{4\pi\epsilon_{0}})(\frac{8}{\pi a^{3}})\int[1-(1+\frac{2r_{1}}{a})e^{-4r_{1}/a}]e^{-4r_{1}/a}r_{1}\sin{\theta_{1}}dr_{1}d\theta_{1}d\phi_{1}\]
ظوایائی تکملات 
\(4\pi\)
دینگے جبکہ 
\(r_{1}\)
 کا تکمل درج ذیل ہوگا
\[\int_{0}^{\infty}[re^{-4r/a}-(r+\frac{2r^{2}}{a})e^{-8r/a}]dr=\frac{5a^{2}}{128}\]
آخر میں اس طرح درج ذیل ہوگا 
\[\langle V_{ee} \rangle\frac{5}{4a}(\frac{e^{2}}{4\pi\epsilon_{0}}=-\frac{5}{2}E_{1}=34\text{eV}\]
جس کی بنا درج ذیل ہوگا 
\[\langle H \rangle =-109 \text{eV}+34 \text{eV}=-75\text{eV}\]
یہ جواب زیادہ برا نہیں ہے۔
یاد رہے کہ تجرباتی قیمت 79- الیکٹران وولٹ ہے۔\\
تاہم ہم اس سے بھی بہتر کر سکتے ہیں۔
ہم
\(\psi_{0}\)
جو دو الیکٹرانوں کو یوں تصور کرتا ہے جیسا ایک دوسرے پر اصر انداز نہیں ہوتے ہیں۔
سے بہتر زیادہ حقیقت پسندانہ پھر کیا دفعال کا سوچ سکتے ہیں۔
ایک الیکٹران کا دوسرے الیکٹران پر اصر کو مکمل طور پر نظر انداز کرنے کی بجائے ہم کہتے ہیں کہ ایک الیکٹران قواسطن منفی بار کی بطل کی طرح ہوگا جو مرکزا کو جزوی طور پر سپر کرتا ہے جس کی بنا دوسرے الیکٹران کو موثر مرکزوی بار z کی قیمت 2 سے کچھ کم نظر آئے گی۔ اس سے ہمیں خیال آتا ہے کہ ہم درج ذیل روپ کا برقی دفعال استعمال کریں۔
\[\psi_{1}(r_{1},r_{2})=\frac{Z^{3}}{\pi a^{3}e^{-Z(r_{1}+r_{2})/a}}\]
ہم z کو تخیریت کا مقدار معلوم تصور کر کہ اس کی وہ تمام قیمت منتخب کر کے جس سے H کی کم سے کم قیمت حاصل ہو ۔
دیہان رہے کہ فضول تغیریت کی ترقیب کبھی بھی ہمیلٹنی کو تبدیل نہیں کرتا ہے ۔
ہیلیم کا ہمیلٹنی اب بھی مساوات 
مساوات 7.14
دیگی البتہ
تصور میں ہمیلٹنی کی تخمیمی قیمت کے بارے میں سوچ کے بہتر بلکیا دفعال معاج حاصل کیا جا سکتا ہے ۔
یہ دفعال معاج اس غیر مضطرب ہمیلٹنی جو الیکٹران کی دفعہ کو نظر انداز کرتا ہو جس میں جز coulumb میں دو کی جگہ z پایا جاتا ہو کا امتیازی حال ہوگا۔
اس کو ذہن میں رکھتے ہوئے ہم 7.14 H کو روپ میں لکھتے ہیں 
\[-\frac{\hbar^{2}}{2m}(\nabla_{1}^{2}+\nabla_{2}^{2})-\frac{e^{2}}{4\pi\epsilon_{0}}(\frac{Z}{r_{1}}+\frac{Z}{r_{2}})+\frac{e^{2}}{4\pi\epsilon_{0}}(\frac{(Z-2)}{r_{1}}+\frac{(Z-2)}{r_{2}}+\frac{1}{\abs{\vec{r_{1}}-\vec{r}_{2}}})\]
ظاہر ہے کہ H کی تحقیقاتی قیمت درج ذیل ہوگی 
\[\langle H \rangle = 2Z^{2}E_{1}+2(Z-2)(\frac{e^{2}}{4\pi\epsilon_{0}})\langle \frac{1}{r}\rangle + \langle V_{ee} \rangle \]
یہاں
\(\langle 1/r \rangle \) 
سی مراد ایک ظرہ ہائڈروجن زمینی حال سائے 100 جس میں مرکزوی بار 
Z
 ہو میں
\(1/r\)
کی تحقیقاتی قیمت ہے ۔\\
یوں مساوات 6.55 کے تحت درج ذیل ہوگا 
\[\langle \frac{1}{r} \rangle = \frac{Z}{a}\]
یہاں بھی vee کی توقیاتی قیمت وہی ہوگی جو پہلے تھی۔
مساوات 7.65لیکن اب ہم z =2 کی بجائے اختیار z استعمال کریں گے لہذا ہم a کو 
\(2/Z\)
 سے ظرب کرتے ہیں 
 \[\langle V_{ee} \rangle =\frac{5Z}{8a}(\frac{e^{2}}{4\pi\epsilon_{0}})=\frac{5Z}{4}E_{1}\]
 ۔ان تمام کو اکٹھے کر کہ درج ذیل حاصل ہوگا 
 \[\langle H \langle =[2Z^{2}-4Z(Z-2)-(5/4)Z]E_{1}=[-2Z^{2}+(27/4)Z]E_{1}\]
اصول تغیریت کے تحت z کی کسی قیمت کے لیے بھی یہ مقدار 
\(E_{gs}\)
سے تجاوظ کرے گی ۔\\
بالائی حد بندی کی کم سے کم قیمت وہاں پائی جانے گی جب
 \(\langle H \rangle  \)
  کی قیمت کن سے کم ہو۔\\
\[\frac{d}{dZ}\langle H \rangle =[-4Z+(27/4)]E_{1}=0\]
جس سے درج ذیل حاصل ہوگا۔
\[Z=\frac{27}{16}=1.69\]
یہ ایک معقول نتیجہ نظر آتا ہے جو کہتا ہے دوسرا الیکٹران مرکزا کو سپر کرتا ہے جس کی بنا اس کی موثر بار 2 کی بجائے 1.69 نظر آتی ہے ۔اس قیمت کو z میں پر کر کہ درج ذیل حاصل ہوگا ۔\\
\[\langle H \rangle =\frac{1}{2}(\frac{3}{2})^{6}E_{1}=-77.5\text{eV}\]
قبلے تقدیر معا معلوم کی تعداد بڑھا کر زیادہ پیچیدہ پرکیا دفعالات معاج لے کر ہیلیم کی زمینی حال توانائی کو اسی ترقیب سے انتہائی زیادہ درستگی تک حاصل کیا گیا ہے ہم ٹھیک جواب کے دو فیسٹ قریب ہیں لحاضہ اس کو یہیں پر چھوڑتے ہیں ۔\\
سوال  
7.6\\
ہیلیم کی زمینی حال توانائی 
\(E_{gs}=-79\text{eV}\)
 لیتے ہوئے توانائی بار داری حمل صرف ایک الیکٹران اکھاڑنے کے لیے درکار توانائی کا حساب کریں ۔\\
اشارہ پہلے ہیلیم بارداریا
 \(He^{+}\)
 جس کے مرکزا کے گرد صرف ایک الیکٹران مدار میں حرکت کرتا ہے کی زمینی حال توانائی تلاش کریں ۔\\
اس کے بعد دونوں توانائیوں کا فرق لیں 
سوال 7۔7
\\
اس حصہ میں ملتمل ترقیبات کا اتلاک
 \(H^{-}\)
 اور
  \(Li^{+}\)
  بارداریا جن میں ہلیم کی طرح دو الیکٹران پائے جاتے ہیں اور جن کی مرکزوی بار بالترتیب z=1 , z=3 ہیں کریں ۔\\
باریک باریک ایک ایک بارداریا کے لیے کا موثر جزوی سپر شدا مرکزوی بار تلاش کر کہ 
\(E_{gs}\)
 کی بہترین بالائی حقبندی متعین کریں ۔\\
 بارداریا
 \(\text{H}^{-}\)
  کی صورت میں آپ دیکھیں گے کہ 
  \(\langle  H \rangle  > -13.6\text{eV}\) 
  ہوگا جس کے تحت کوئی مقید حآل نہیں ہوگا ۔
توانائی کی نقطہ نظر سے زیادہ بہتر صورت حال یہ ہوگی کہ الیکٹران درست ہو کر پیچھے مدرل حاڈروجن جوہر چھوڑے۔ یہ زیادہ حیرانگی کی بات نہیں ہے چونکہ ہیلیم کے لحاظ سے یہاں الیکٹران اور مرکزا کے بیچ قوت کشش کم ہے ۔ جبکہ الیکٹرانوں کے بیچ قوت دفعہ زیادہ ہے۔
جو اس جوہر کے توڑے گا حقیقت میں یہ نتیجہ درست نہیں ہے ۔ زیادہ نفیس برکیا دفعال معاج ساتھ 
7.18
دیکھیں 
منتخب کر کے دکھایا جا سکتا ہے کہ 
\(E_{gs}<-13.6\text{eV}\)
ہوگا لحاضہ مقید حآل موجود ہوگا ۔
البتہ یہ بمشکل مقید ہوگا اور کوئی حجانی مقید حالات نہیں پائے جاتے ہیں یوں
\(\text{ H}^{-}\)
 کا غیر مسلسل طیف نہیں پایا جاتا ہے ۔
تمام عبور ازتمراریا کو اور ازتمراریا سے ہوں گے اسی لیے ان کا متالیا تجربہ گاہ میں کرنا دشوار ثابت ہوتا ہے اگرچہ سورج کی سطح پر ان کی کثیر تعداد پائی جاتی ہے ۔\\
\
حصہ 
7.3\\
ہائیڈروجن سالمہ بار داریہ \\
اصول تغیریت کی ایک اور پلاسی کی استعمال۔ ہائیڈروجن سالمہ بار داریہ
\(\text{ H}_{2}^{+}\)
 کا معائنہ ہے۔ ہائیڈروجن سالمہ بار داریہ دو پروٹان کی کلوم میدان میں ایک الیکڑان پر مشتمل ہے۔  شکل 7.5 
میں فلوقت فرض کرتا ہوں کہ دونوں پروٹان ساکن ہیں اور ان کے بیچ فاصلہ R  ہے۔ اگرچہ اس حساب کا ایک دلچسپ ذیلی نتیجہ R کی اصل قیمت ہوگی۔  ہمیٹنی درجہ ذیل ہوگا۔
 \[H=-\frac{\hbar^{2}}{2m}\nabla^{2}-\frac{e^{2}}{4\pi\epsilon_{0}}(\frac{1}{r_{1}}+\frac{1}{r_{2}}\]
جہاں
\( r_{1}\)  
اور 
\(r_{2}\) 
الیکڑان سے متعلقہ پروٹان تک فاصلہ ہے۔ ہمیشہ کی طرح ہم کوشش کریں گے کہ ایک ایسا پھر کی طفال موج کا انتخاب کریں جس کو استعمال کرتے ہوئے زمینی حال توانائی کی حد بندی اصول تغیریت سے حاصل ہو۔ در حقیقت ہم صرف اتنا جاننا چاہتے ہیں کہ آیا اس نظام میں  بند پیدا ہوگا یعنی آیا ایک مادل ہائیڈروجن جوہر اور ایک آزاد پروٹان سے کیا اس نظام کی توانائی کم ہوگی۔ آگر ہماری پھر کی طفال موج دکھائے کہ ایک مکید حال پایا جاتا ہے۔ اس سے زیادہ بہتر پھر کی طفال اس بند کو مزید طاقتور بنائے گا۔ پھر کی طفال موج تیار کرنے کی خاطر فرض کریں زمینی حال مہوار 4.80
\[\psi_{0}(r)=\frac{1}{\sqrt{\pi a^{3}}}e^{-r/a}\]
 میں ایک ہائیڈروجن جوہر کے قریب لا متناہی دوسرا پروٹان قریب لا کر فاصلہ  R پر رکھ کر بار داریہ پیدا کیا جاتا ہے۔ اگر رداس بوہر سے r کافی بڑا ہو تب الیکڑان کی طفال موج غالبا زیادہ تبدیل نہیں ہو گا۔ تاہم ہمیں دونوں پروٹانوں کو ایک نظر سے دیکھنا ہوگا۔ لہذا کسی ایک کے ساتھ الیکڑان کی وابستگی کا احتمال ایک دوسرے جیسا ہوگا۔ اس سے ہمیں خیال آتا ہے کہ ہم درجہ ذیل روپ کے پھر کی طفال 
  \[\psi=A[\psi_{0}(r_{1})+\psi_{0}(r_{2})]\]
 پر غور کریں\\
۔ ماہر کوانٹم کیمیا اس ترکیب کو جوہری مدارچوں کا خطی جوڑ کہتے ہیں۔ \\
سب سے پہلا کام پھر کی طفال کی معمول زنی ہے۔
 \[1=\int\abs{\psi}^{2}d^{3}r=\abs{A}^{2}\big [\int \abs{\psi_{0}(r_{1})}^{2}d^{3}r+\int \abs{\psi_{0}(r_{2})}^{2}d^{3}r+2\int\psi_{0}(r_{1})\psi_{0}(r_{2})d^{3}r\big ]\]
پہلے دو تکلملات کا نتیجہ ایک ہے۔ چونکہ  
\(\psi_{0}\)
خود معمول شدہ ہے۔ تیسرا زیادہ پیچیدہ ہے۔ درجہ ذیل فرض کریں۔\\
 \[I\equiv\langle \psi_{0}(r_{1})|\psi_{0}(r_{2})\rangle=\frac{1}{\pi a^{3}}\int e^{-(r_{1}+r_{2})/a}d^{3}r\]
ایسا معتدی نظام کھڑا کریں جس کہ نقطہ پر پروٹان 1 پایا جاتا ہو جبکہ Z مہور پر فاصلہ R پر پروٹان 2 پایا جاتا ہو۔ شکل 7.6 یوں درجہ ذیل ہوگا۔ \\
\[r_{1}=r \quad r_{2}=\sqrt{r^{2}+R^{2}-2rR\cos{\theta}}\]
لہذا درجہ ہوگا \\
\[I=\frac{1}{\pi a^{3}}\int e^{-r/a}e^{-\sqrt{r^{2}+R^{2}-2rR\cos{\theta/a}}}r^{2}\sin{\theta}drd\theta d\phi\]
متغیر 
\(\phi\)
 کا تکمل 
 \(2\pi\)
  دے گا۔ متغیر 
  \(\theta\)
   کا تکمل حل کرنے کی خاطر درجہ ذیل لیں۔\\
\[y\equiv\sqrt{r^{2}+R^{2}-2rR\cos{\theta}}\]
لہذا 
\[d(y^{2})=2ydy=2rR\sin{\theta}d\theta\]
ہوگا۔ تب درجہ ذیل ہوگا۔
\[\int_{0}^{\pi}e^{-\sqrt{r^{2}+R^{2}-2rR\cos{\theta/a}}}\sin{\theta}d\theta=\frac{1}{rR}\int_{\abs{r-R}}^{r+R}e^{-y/a}ydy=-\frac{-a}{rR}[e^{-(r+R)/a}(r+R+a)-e^{-\abs{r-R}/a}(\abs{r-R}+a)]\]
اب تکمل r با آسانی حل ہوگا۔ 
\[I=\frac{2}{a^{2}R}[-e^{-R/a}\int_{0}^{\infty}(r+R+a)e^{-2r/a}rdr+e^{-R/a}\int_{0}^{R}(R-r+a)rdr+e^{R/a}\int_{R}^{\infty}(r-R+a)e^{-2r/a}rdr]\]
ان تکملات کی قیمتیں حاصل کرنے کے بعد کچھ الجبرائی تصحیل کے بعد درجہ ذیل حاصل ہوگا۔
\[I=e^{-R/a}\big[1+(\frac{R}{a}+\frac{1}{3}(\frac{R}{a})^{2}\big]\]
ےھاں  ٰI کو مکمل ڈمب کہتے ہیں جو۔
\(\psi_{0}(r_{1})\)
 کا۔
 \(\psi_{0}(r_{2}\)
 پر چڑھنے کی مقدار کی پیمائش ہے۔ دیہان رہے کہ 
 \(R\rightarrow 0\)
  کی صورت میں یہ ایک پہنجتا ہے۔جبکہ 
  \(R\rightarrow \infty\)
   کی صورت میں یہ صفر کو پہنجتا ہے۔ تکمل ڈنب i میں جز زربی معمول زنی مساوات 7.38 درجہ ذیل ہوگا ۔\\
\[\abs{A}^{2}=\frac{1}{2(l+1)}\]
اس کے بعد ہمیں پھر کی حال 
\(\psi\)
 میں H کی توقعاتی قیمت کا حساب کرنا ہوگا۔ درجہ ذیل ۔
 \[\big(-\frac{\hbar^{2}}{2m}\nabla^{2}-\frac{e^{2}}{4\pi\epsilon_{0}}\frac{1}{e_{1}}\big)\psi_{0}(r_{1})=E_{1}\psi_{0}(r_{1})\]
  جہاں 
  \(E_{1}=-13.6\text{eV}\)
  جوہری ہائیڈروجن کی زمینی حال توانائی ہے اور r1 کی جگہ r2 کے لئے بھی یہی کچھ کے بنا  درجہ ذیل ہوگا ۔
\begin{align*}
H\psi&=A\big[-\frac{\hbar^{2}}{2m}\nabla^{2}-\frac{e^{2}}{4\pi\epsilon_{0}}\big(\frac{1}{r_{1}}+\frac{1}{r_{2}}\big)\big][\psi_{0}(r_{1})+\psi_{0}(r_{2})]\\
&=E_{1}\psi-A(\frac{e^{2}}{4\pi\epsilon_{0}}\big[\frac{1}{r^{2}}\psi_{0}(r_{1})+\frac{1}{r_{1}}\psi_{0}(r_{2})\big]
\end{align*}
یوں H کی توقعاتی قیمت درجہ ذیل ہوگی۔
\[\langle H \rangle =E_{1}-2\abs{A}^{2}\big(\frac{e^{2}}{4\pi\epsilon_{0}}\big)\big[\langle \psi_{0}(r_{1})\abs{\frac{1}{r_{2}}}\psi_{0}(r_{1})\rangle +\langle \psi_{0}(r_{1})\abs{\frac{1}{r_{1}}}\psi_{0}(r_{2})\rangle\big]\]
میں آپ کے لئے باقی دو مقدار جو بلا واسطہ تکمل 
\[D\equiv a\langle \psi_{0}(r_{1})\abs{\frac{1}{r_{2}}}\psi_{0}(r_{1})\rangle\]
اور
مبادلہ تکمل 
\[X\equiv a\langle\psi_{0}(r_{1})\abs{\frac{1}{r_{1}}}\psi_{0}(r_{2})\rangle\]
کہلاتا ہے۔ حل کرنے کے لئے چھورتا ہوں۔ بلا واسطہ تکمل کا نتیجہ درجہ ذیل 
\[D=\frac{a}{R}-\big(1+\frac{a}{R}\big)e^{-2R/a}\]
اور مبادلہ تکمل کا نتیجہ درجہ ذیل ہے ۔ 
\[X=\big(1+\frac{R}{a}\big)e^{-R/a}\]
ان تمام نتائج کو اکھٹے کرتے ہوئے اور یاد رکھتے ہوئے مساوات 4.70 اور 4.72 کہ 
\(E_{1}=-(e^{2}/4\pi\epsilon_{0})(1/2a)\)
ہے ۔ ہم درجہ ذیل آخذ کرتے ہیں۔
\[\langle H \rangle =\big[a+2\frac{(D+X)}{(1+L)}\big]E_{1}\]
اصول تغیریت کے تحت زمینی حال توانائی  
\(\langle H \rangle\)
 سے کم گی۔ یقینا  یہ صرف الیکڑان کی توانائی ہے۔ اس کے ساتھ پروٹان پروٹان دفع سے وابستہ مخفی توانائی بھی پائ جائے گی۔.
\[V_{pp}+\frac{e^{2}}{4\pi\epsilon_{0}}\frac{1}{R}=-\frac{2a}{R}E_{1}\]
یوں نظام کی  کل توانائی مائنس
\(E_{1}\)
  کی  اکائیوں میں 
  \(x\equiv R/a\)
  کا طفال لکھتے ہوئے درجہ ذیل سے کم ہو گا۔
  \[F(x)=-1+\frac{2}{X}\big\{\frac{(1-(2/3)x^{2})e^{-x}+(1+x)e^{-2x}}{1+(1+x+(1/3)x^{2})e^{-x}}\big\}\]
اس طفال کو شکل 7.7 میں ترسیم کیا گیا ہے۔ اس ترسیم کا کچھ حصہ منفی ایک سے نیچے ہے۔ جہاں معدل جوہر جمع ایک آزاد پروٹان کی توانائی مائنس  13.6الیکڑان وولٹ سے توانائی کم ہے۔ لہذا اس نظام میں بند پیدا ہوگا۔ یہ ایک شریک گرفتی بند ہوگا، جہاں دونوں پروٹانوں کا الیکڑان میں ایک دوسرے کے برابر حصہ ہوگا۔ پروٹانوں کے بیچ توازنی فاصلہ تقریبا 2.4 رداس بوہر یعنی 1.3 اینگسڑروم ہے۔ جس کی تجرباتی قیمت 1.06 اینگسڑروم ہے۔ توانائی بندش کی حساب سے حاصل قیمت 1.8 الیکڑان وولٹ جبکہ پیمائشی قیمت 2.8 الیکڑان وولٹ ہے۔ چونکہ اصول تغیریت ہر صورت زمینی حال توانائی سے تجاوز کرتا ہے لہذا یہ بندش کی طاقت کی قیمت کم دے گا۔ بہرحال اس کی فکر نہ کریں۔ یہاں اہم نقطہ یہ ہےکہ بندش پایا جاتا ہے۔ ایک بہتر تغیراتی طفال اس مخفیہ کو مزید گہرا کرے گا۔ \\
سوال 
7.8\\
بلاواسطہ تکمل D اور مبادلہ تکمل X مساوات 7.45 اور 7.46 کی قیمتیں تلاش کریں۔ اپنے جوابات کا موازنہ مساوات 7.47 اور 7.48 کے ساتھ کریں۔ \\
سوال 
7.9\\
فرض کریں ہم نے پھرکی طفال موج مساوات 7.37 میں منفی علامت استعمال کی ہوتی ۔
\[\psi=A[\psi_{0}(r_{1})-\psi_{0}(r_{2})]\]
کوئی نیا تکمل حل کیے بغیر مساوات  7.51 کا مماسل
\( F(x)\) 
معلوم کر کے ترسیم کریں۔ دکھائیں کہ ایسی صورت میں بند پیدا نہیں ہوگا۔ چونکہ اصول تغیریت صرف بالائی حد بندی دیتا ہے لہذا اس سے یہ ثابت نہیں ہوگا کہ ایسے حال میں بند نہیں پایا جائے گا۔ تاہم اس سے زیادہ امید بھی نہیں کرنی چاہیئے۔ تبصرہ در حقیقت درجہ ذیل روپ کا کوئی طفال 
\[\psi=A[\psi_{0}(r_{1})+e^{i\phi}\psi_{0}(r_{2})]\]
کی ایک خاصیت یہ ہے کہ  الیکڑان دونوں پروٹان کے ساتھ برابر کا وابستگی رکھتا ہے۔ تاہم چونکہ باہمی ادل بدل 
\(P: r_{1}\leftrightarrow r_{2}\)
کی صورت میں ہملٹنی مساوات 7.35 غیر متغیر ہے۔ لہذا اس کے امتیازی طفالات کو بیک وقت P کے امتیازی طفالات چنا جا سکتا ہے۔ امتیازی قدر
\(+1\)
کے ساتھ مثبت علامت۔ مساوات 7.37 اور امتیازی قدر منفی 1 کے ساتھ منفی علامت مساوات 7.52 ہوگا۔ زیادہ عمومی صورت مساوات 7.53 کا استعمال مزید فائدہ نہیں دے گا۔ اگرچہ آپ چاہیں تو اسے استعمال کر کے دیکھ سکتے ہیں۔\\
سوال 
7.10\\
نقطہ توازن پر
\( F(x)\)
 کی دوہرا تفرق سے ہائیڈروجن سالمہ بار داریہ حصہ 2.3 میں دونوں پروٹانوں کی ارتعاش کی قدرتی تعدد اومیگہ کی انداز قیمت تلاش کی جا سکتی ہے۔ اگر اس موردیش کی زمینی حال توانائی 
 \(\hbar\omega/2\)
 نظام کی بندشی توانائی سے زیادہ ہو تب نظام بکھر کر ٹوٹ جائے گا ۔ دکھائیں کہ حقیقت میں موردیش توانائی اتنی کم ہے کہ ایسا کبھی بھی نہیں ہوگا۔ ساتھ ہی مکید لرزشی سطحوں کی انداز  تعداد دریافت کریں۔ تبصرہ 
آپ دہلیلی طور پر کم سے کم نقطہ یا اس نقطہ پر دوہرا تفرق حاصل نہیں کر پائیں گے۔ اعدادی طریقہ یا کمپیوٹر کی مدد سے  ایسا کیجئے گا۔\\



\end{document}
