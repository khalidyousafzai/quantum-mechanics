
\جزوحصہ{تسلسل بارن}
تخمین بارن روح کے لحاظ سے کلاسیکی نظریہ بکھراؤ میں تخمین ضرب کی طرح ہے۔ ایک ذرہ کو منتقل عرضی ضرب کا حساب کرنے کے لیئے ہم تخمین ضرب میں فرض کرتے ہیں کہ ذرہ ایک سیدھی لیکر پر ہی چلے جاتا ہے \حوالہء{شکل \num{11.12}} ایسی صورت میں درج ذیل ہوگا
\begin{align}
	I=\int F_\perp\dif t
\end{align}
اگر ذرہ زیادہ نہیں مڑے تب یہ ذرہ کو منتقل معیارِ حرکت کی ایک اچھی تخمین ہوگی اور یوں زاویہ بکھراؤ درج ذیل ہوگا جہاں \عددی{p} آمدی معیارِ حرکت ہے 
\begin{align}
	\theta\cong\tan^{-1}(I/p)
\end{align}
اسے ہم رتبہ اوّل تخمین ضرب کہہ سکتے ہیں نہ مڑنےکی صورت کو صفر رتبی کہا ھائے گا اسی طرح صفر رتبی تخمین بارن میں آمدی مستوی موج بغیر کسی تبدیلی کے گزرے گی اور ہم نے جو کچھ گزشتہ حصہ میں دیکھا وہ در حقیقت اس کی رتبہ اوّل تصحیح ہے۔ ہم توقع کر سکتے ہیں کہ اسی تصور کو بار بار استعمال کرتے ہوئے ہم زیادہ بلند رتبی تصحیح کا ایک تسلسل پیدا کر کے بلکل ٹھیک جواب پر مرکوز ہو سکتے ہیں۔

مساوات شروڈنگر کی تکملی روپ درج ذیل ہے
\begin{align}
	\psi(r)=\psi_0(r)+\int g(r-r_0)V(r_0)\psi(r_0)\dif^3r_0
\end{align}
جہاں \عددی{\psi_0} آمدی موج ہے
\begin{align}
	g(r)\equiv-\frac{m}{2\pi\hslash^2}\frac{e^{ikr}}{r}
\end{align}
تفاعل گرین ہے۔ جس میں میں نے اپنی آسانی کے لیئے جز ضربی \عددی{2m/\hslash^2} شامل کیا ہے اور \عددی{V} مخفیہ بکھراؤ ہے۔ اس کو درج ذیل دیکھا جا سکتا ہے
\begin{align}
	\psi = \psi_0+\int gV\psi
\end{align}
فرض کریں ہم \عددی{\psi} کی اس ریاضی جملہ کو لیکر اسے تکمل کی علامت کے اندر لکھیں 
\begin{align}
	\psi=\psi_0+\int gV\psi_0+\iint gVgV\psi
\end{align}
اس عمل کہ بار بار دوہرانے سے ہمیں \عددی{\psi} کا ایک تسلسل حاصل ہوگا
\begin{align}
	\psi=\psi_0+\int gV\psi_0+\iint gVgV\psi_0+\iiint gVgVgV\psi_0+\dots
\end{align}
ہر متکمل میں آمدی تفاعل موج \عددی{\psi_0} کے علاوہ \عددی{gV} کے مزید زیادہ طاقتیں پائی جاتی ہیں۔ بارن کی تخمین اوّل اس تسلسل کو دوسرے جز کے بعد ختم کرتا ہے تاہم آپ دیکھ سکتے ہیں کہ بلند رتبی تصحیح کس طرح پیدا کی جائیں گی۔

بارن تسلسل کا خاکہ \حوالہء{شکل \num{11.13}} میں پیش کیا گیا ہے۔ صفر رتبی \عددی{\psi} پر مخفیہ کا کوئی اثر نہیں ہوگا رتبی اوّل میں اسے ایک چوٹ پڑتی ہے جس کے بعد یہ کسی نئے رخ چلے جائے گا۔ دوم رتبی میں اسے ایک چوٹ پڑتی ہے جس کے بعد یہ ایک نئے مقام پر پہنچتا ہے جہاں اسے دوبارہ ایک چوٹ پڑتی ہے جس کے بعد یہ ایک نئے راہ پر چل نکلتا ہے وغیرہ وغیرہ۔ اسی کے بنا بعض اوقات تفاعل گرین کو اشاعت کار کہا جاتا ہے جو ایک باہم عمل اور سورے کے بیچ خلل کی اشاعت کس طرح ہوتی ہے۔ تسلسل بارن اضافیتی کوانٹم میکانیات کی فینمن تشریح کا سبب بنا جس میں اشکال فینمن میں جز ضربی راس \عددی{V} اور اشاعت کار \عددی{g} کو ایک دوسرے کے ساتھ جوڑ کر سب کچھ بیان کیا جاتا ہے۔

\ابتدا{سوال}
تخمین ضرب میں ردرفورڈ بکھراؤ کے لیئے \عددی{\theta} کو ٹکراؤ مقدار معلوم کا تفاعل تلاش کریں۔ دیکھائیں کہ مناسب حدوں کے اندر آپ کا نتیجہ بلکل ٹھیک ریاضی فکرہ \حوالہء{سوال \num{11.1} (الف)} کے مطابق ہے۔
\انتہا{سوال}
\ابتدا{سوال}
بارن کی دوسری تخمین میں کم توانائی نرم کرہ بکھراو کے لیئے حیطہ بکھراو تلاش کریں۔

جواب: \عددی{-(2mV_0a^3/3\hslash^2)[1-(4mV_0a^2/5\hslash^2)]}
\انتہا{سوال}
\ابتدا{سوال}
یک بُعدی مساوات شروڈنگر کے لیئے تفاعل گریں تلاش کر کے \حوالہء{مساوات \num{11.67}} کا مماثل تکملی روپ تیار کریں۔

جواب:
\begin{align}
	\psi(x)=\psi_0(x)-\frac{im}{\hslash^2k}\int_{-\infty}^{\infty}e^{ik\abs{x-x_0}}V(x_0)\psi(x_0)\dif x_0
\end{align}
\انتہا{سوال}
\ابتدا{سوال}
مبدہ پر بغیر اینٹون کی دیوار کی صورت میں وقفہ \عددی{-\infty<x<\infty} پر یک بُعدی بکھراو کے لیئے \حوالہء{سوال \num{11.16}} کا نتیجہ استعمال کرتے ہوئے تخمین بارن تیار کریں۔ یعنی \عددی{\psi(x_0)\cong\psi_0(x_0)} تصور کرتے ہوئے \عددی{\psi_0(x)=Ae^{ikx}} منتخب کر کت تکمل کی قیمت تلاش کریں۔ دیکھائیں کہ انعکاسی عددی سر درج ذیل روپ اختیار کرتا ہے
\begin{align}
	R\cong\left(\frac{m}{\hslash^2k}\right)^2\abs{\int_{-\infty}^{\infty}e^{2ikx}V(x)\dif x}^2
\end{align}
\انتہا{سوال}
\ابتدا{سوال}
ایک ڈیلٹا تفاعل \حوالہء{مساوات \num{2.114}} اور ایک متناہی چکور کنواں \حوالہء{مساوات \num{2.145}} سے بکھراو کے لیئے تفصیلی عددی سر \عددی{(T = 1 - R)} کو یک بُعدی تخمین بارن \حوالہء{سوال \num{11.17}} کی مدد سے حاصل کریں۔ اپنے جوابات کا بلکل ٹھیک جوابات \حوالہء{مساوات \num{2.141} اور \num{2.169}} کے ساتھ موازنہی کریں۔
\انتہا{سوال}
\ابتدا{سوال}
آگے رخ ھیطہ بکھراو کے خیالی جز اور کل عمودی تراش کے بیچ رشتہ دینے والا مسئلہ بصریات ثابت کریں 
\begin{align}
	\sigma = \frac{4\pi}{k}Im(f(0))
\end{align}
اشارہ: \حوالہء{مساوات \num{11.47} اور \num{11.48}} استعمال کریں۔
\انتہا{سوال}
\ابتدا{سوال}
Missing Question
\begin{align}
	V(r) = Ae^{-\mu r^2}
\end{align}
\انتہا{سوال}

