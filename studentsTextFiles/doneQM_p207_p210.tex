
\جزوحصہ{قوت مبادلہ}
میں ایک سادہ یک بُعدی مثال کے ذریع آپ کو ضرورتِ تشاکل کی وضاحت کرنا چاہتا ہوں۔ فرض کریں ایک ذرہ حال \عددی{\psi_a(x)} میں اور دوسرا حال \عددی{\psi_b(x)} میں ہو اور یہ دونوں حالات عمودی اورمعمول شدہ ہوں اگر یہ ذرات قابلِ ممیز ہوں اور ذرہ ایک حال \عددی{\psi_a} میں ہو تب انکا مجموعی تفاعل موج درج ذیل ہوگا
\begin{align}
	\psi(x_1, x_2)=\psi_a(x_1)\psi_b(x_2)
\end{align}
اگر یہ یکساں بوزون ہوں تب انکا مرکب تفاعل موج \حوالہء{سوال \num{5.4}} معمولزنی کے لیئے دیکھں درج ذیل ہوگا
\begin{align}
	\psi_+(x_1, x_2) = \frac{1}{\sqrt{2}}[\psi_a(x_1)\psi_b(x_2)+\psi_b(x_1)\psi_a(x_2)]
\end{align}
اور اگر یہ یکساں فرمیونز ہوں تب درج ذیل ہوگا
\begin{align}
	\psi_-(x_1, x_2)=\frac{1}{\sqrt{2}}[\psi_a(x_1)\psi_b(x_2)-\psi_b(x_1)\psi_a(x_2)]
\end{align}
آئیں ان ذرات کے بیچ علیحدگی کے فاصلی کے مربع کی توقعاتی قیمت معلوم کریں
\begin{align}
	\langle(x_1-x_2)^2\rangle=\langle x^2_1\rangle+\langle x_2^2\rangle-2\langle x_1x_2\rangle
\end{align}
\موٹا{پہلی صورت: قابل ممیز ذرات۔} \حوالہء{مساوات \num{5.15}} میں دی گئی تفاعل موج کے لیئے ایک ذرہ حال \عددی{\psi_a} میں \عددی{x^2} کی توقعاتی قیمت 
\begin{align*}
	\langle x_1^2\rangle=\int x_1^2\abs{\psi_a(x_1)}^2\dif x_1\int\abs{\psi_b(x_2)}^2\dif x_2=\langle x^2\rangle_a
\end{align*}
\begin{align*}
	\langle x_2^2\rangle=\int\abs{\psi_a(x_1)}^2\dif x_1\int x_2^2\abs{\psi_b(x_2)}^2\dif x_2=\langle x^2\rangle_b
\end{align*}
اور
\begin{align*}
	\langle x_1x_2 \rangle=\int x_1\abs{\psi_a(x_1)}^2\dif x_1\int x_2\abs{\psi_b(x_2)}^2\dif x_2=\langle x\rangle_a\langle x \rangle_b
\end{align*}
یوں اس صورت درج ذیل ہوگا
\begin{align}
	\langle(x_1-x_2)^2\rangle_d=\langle x^2\rangle_a+\langle x^2 \rangle_b-2\langle x \rangle_a\langle x \rangle_b
\end{align}
یہی جواب ذرہ ایک حال \عددی{\psi_b} میں اور ذرہ دو حال \عددی{\psi_a} میں ہونے کی صورت میں بھی حاصل ہوتا۔

\موٹا{دوم صورت: یکساں ذرات۔} \حوالہء{مساوات \num{5.16} اور \num{5.17}} کے تفاعل امواج کے لیئے 
\begin{gather*}
	\begin{aligned}
		\langle x_1^2 \rangle =& \frac{1}{2}\Big[\int x_1^2\abs{\psi_a(x_1)}^2\dif x_1\int\abs{\psi_b(x_2)}^2\dif x_2 \\
		&+ \int x_1^2\abs{\psi_b(x_1)}^2\dif x_1\int\abs{\psi_a(x_2)}^2\dif x_2 \\
		&\pm\int x_1^2\psi_a(x_1)^*\psi_b(x_1)\dif x_1\int\psi_b(x_2)^*\psi_a(x_2)\dif x_2 \\
		&\pm\int x_1^2\psi_b(x_1)^*\psi_a*x_1\dif x_1\int\psi_a(x_2)^*\psi_b(x_2)\dif x_2\Big] \\
		=& \frac{1}{2}\left[\langle x^2 \rangle_a+\langle x^2 \rangle_b\pm0\pm0\right]=\frac{1}{2}\left(\langle x^2 \rangle_a+\langle x^2\rangle_b\right)
	\end{aligned}
\end{gather*}
بلکل اسی طرح
\begin{align*}
	\langle x_2^2 \rangle=\frac{1}{2}\left(\langle x^2 \rangle_b+\langle x^2 \rangle_a\right)
\end{align*}
ظاہر ہے \عددی{\langle x^2_2 \rangle=\langle x^2_1 \rangle} ہوگا کیونکہ آپ ان میں تمیز نہیں کرسکتے ہیں۔ تاہم
\begin{gather*}
	\begin{aligned}
		\langle x_1x_2 \rangle=&\frac{1}{2}\Big[\int x_1\abs{\psi_a(x_1)}^2\dif x_1\int x_2\abs{\psi_b(x_2)}^2\dif x_2 \\
		&+\int x_1\abs{\psi_b(x_1)}^2\dif x_1\int x_2\abs{\psi_a(x_2)}^2\dif x_2 \\
		&\pm\int x_1\psi_a(x_1)^*\psi_b(x_1)\dif x_1\int x_2\psi_b(x_2)^*\psi_a(x_2)\dif x_2 \\
		&\pm\int x_1\psi_b(x_1)^*\psi_a(x_1)\dif x_1\int x_2\psi_a(x_2)^*\psi_b(x_2)\dif x_2\Big] \\
		=& \frac{1}{2}\left(\langle x \rangle_a\langle x \rangle_b+\langle x \rangle_b\langle x \rangle_a\pm\langle x \rangle_{ab}\langle x \rangle_{ba}\pm\langle x \rangle_{ba}\langle x \rangle_{ab}\right)\\
		=&\langle x \rangle_a\langle x \rangle_b\pm\abs{\langle x \rangle_{ab}}^2
	\end{aligned}
\end{gather*}
جہاں درج ذیل ہوگا
\begin{align}
	\langle x \rangle_{ab}\equiv\int x\psi_a(x)^*\psi_b(x)\dif x
\end{align}
ظاہر ہے کہ درج ذیل ہوگا
\begin{align}
	\langle(x_1-x_2)^2\rangle_{\pm}=\langle x^2\rangle_a+\langle x^2 \rangle_b-2\langle x \rangle_a\langle x \rangle_b\mp2\abs{\langle x \rangle_{ab}}^2
\end{align}
\حوالہء{مساوات \num{5.19} اور \num{5.21}} کا موازنہ کرتے ہوئے ہم دیکھتے ہیں کہ فرق صرف آخری ھز میں پایا جاتا ہےٍ
\begin{align}
	\langle(\Delta x)^2\rangle_{\pm}=\langle(\Delta x)^2\rangle_d\mp2\abs{\langle x \rangle_{ab}}^2
\end{align}
قابل ممیز ذرات کے لھاظ سے انہی دو حالات کے یکساں بوزان ملائی علامت نسبتاً ایک دوسرے کے زیادہ قریب جبکہ یکساں فرمیون زیریں علامت نسبتاً ایک دوسرے سے زیادہ دور ہوںگے۔ دیہان رہے کہ جب تک یہ دو تفاعل امواج ایک دوسرے کو ڈھانپے نہیں \عددی{\langle x \rangle_{ab}} صفر ہوگا غیر صفر \عددی{\psi_b(x)} کی صورت میں جب بھی \عددی{\psi_a(x)} صفر ہو تب \حوالہء{مساوات \num{5.20}} میں تکمل کی قیمت صفر ہوگی۔ یوں اگر کراچی میں ایک جوہر کے اندر الیکٹران کو \عددی{\psi_a} ظاہر کرتا ہو جبکہ صوابی میں ایک جوہو کے اندر الیکٹران کو \عددی{\psi_b} ظاہر کرتا ہو تب تفاعل موج کو غیر تشاکلی بنانے یا نہ بنانے سے کوئی فرق نہیں پڑے گا یوں عملی نقطہ نظر سے ایسے الیکٹران جن کے تفاعل امواج ایک دوسرے کو ڈھانپتے نہ ہوں کو آپ قابلِ ممیز ہونے کا ڈھونگ رچا سکتے ہیں۔ درحقیقت اسی کی بنا ماہرِ طبیعیات اور کیمیات آگے بڑھ سکتے ہیں چونکہ اصولاً جائنت میں ہر ایک الیکٹران باقی تمام کے ساتھ تفاعل امواج کے ذریعہ عدم تشاکلی کی بنا جڑا ہے اور اگر اس سے کوئی ففرق پڑتا تب تمام کائنات کے الیکٹرانوں کی بات کیئے بغیر ہم کسی ایک الیکٹران کی بات کرنے سے قاصر ہوتے۔

دلچسپ صورت تب پیدا ہوتے ہے جب انکی موجی تفاعلات ایک دوسرے کو ڈھانپتے ہیں۔ ایسی صورت میں نظام کا رویہ کچھ یوں ہوگا جیسا یکساں بوزون کے بیچ قوت کشش پائی جاتی ہو جو انہیں قریب کھینچتی ہے جبکہ یکساں فرمیونز کے بیچ قوت دفع پائے جاتی ہے جو انہیں ایک دوسرے سے دور دھکا دیتے ہیں۔ یاد رہے کہ ہم فل حال چکر کو نظرانداز کر رہے ہیں۔ ہم اس کو قوت مبادلہ کہتے ہیں اگرچہ یہ حقیقیتاً ایک وقت نہیں ہے کوئی بھی چیز ان ذرات کو دکھیل نہیں رہی ہے یہ صرف ضرورت تشاکل کی جومیٹرائی نتیجہ ہے ساتھ ہی یہ کوانٹم میکانی مظہر ہے جس کا کلاسیکی میکنیات میں کوئی مماثل نہیں پایا جاتا ہے۔ بہرحال اس کے دورست نتائج پائے جاتے ہیں۔ مثال کے طعر پر ہائڈروجن سالمہ \عددی{H_2} پر غور کریں اندازاً بات کرتے ہوئے مرکزہ ایک پر وسط رکھے ہوئے جوہری زمینی حال \حوالہء{مساوات \num{4.80}} میں ایک الیکٹران اور مرکزہ دو پر وسط رکھے ہوئے جوہری زمین حال دو میں ایک الیکتران پر زمینی حال مشتمل ہوگا اگر الیکٹران بوزون ہوتے تب ضرورت تشاکل یا اگر آپ قوت مبرہ پسند کرتے ہیں کوشش کرتے کہ دونوں پروٹان کے بیچ الیکٹرانوں کو جمع کریں \حوالہء{شکل \num{5.1} الف} نتیجتاً منفی بار کا امبار دونوں پروٹانوں کو اندر کی طرف ایک دوسرے کی جانب کھینچتا جو شریک گرفتی بند کا سبب ہوتا۔ بدقسمتی سے  الیکتران در حقیقیت فرمیون ہیں نہ کہ بوزون جس کی بنا منفی بار اطراف کی جانب منتقل ہوتا ہے \حوالہء{شکل \num{5.1} ب} جو سالمہ کو توڑنے کی کوشش کرتا ہے۔

ذرا رکیئے گا اب تک ہم نے چکر کو نظرانداز کیا ہے الیکٹران کے مکمل حال کو نہ صرف الیکٹران کا مکام تفاعل موج بلکہ الیکٹران کے  چکر کی سمت بندی کو بیان کرنے والا چکر کار تعین کرتے ہیں
\begin{align}
	\psi(r)\chi(s)
\end{align}
دو الیکتران حال کو تشکیل دیتے ہوئے ہمیں صرف فضائی جزو کو مبادلہ کے لحاظ سے عدم تشاکلی بنانا ہوگا بلکہ پورے کو عدم تشاکلی بنانا ہوگا۔ مرکب چکری حال \حوالہء{مساوات \num{4.177} اور \num{4.178}} پر نظریں ڈالتے ہوئے ہم دیکھتے ہیں کہ یکتا ملاپ خلاف تشاکل ہے لحاظہ اس کو تشاکل فضائی تفاعل کے ساتھ جوڑنا ہوگا جبکہ تین سہتا حالات تشاکلی ہیں لحاظہ انہیں خلاف تشاکل فضائی تفاعل کے ساتھ منسلک کرنا ہوگا۔ ظاہر ہے کہ یوں یکتا حال بندھن پیدا کرے گا جنکہ سہتا حال خلاف بندھن ہوگا۔ یقیناً ماہرِ کیمیات ہمیں بتاتے ہیں کہ شریک گرفتی بند کے لیئے ضراری ہے کہ دونوں الیکٹران یکتا حال کے مکین ہوں جہاں انکا کل چکر صفر ہوگا۔

