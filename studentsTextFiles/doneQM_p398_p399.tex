
\جزوحصہ{کوانٹم نظریہ بکھراو}
بکھراو کے کوانٹم نظریہ میں فرض کرتے ہیں کہ ایک آمدی مستوی موج \عددی{\psi(z) = Ae^{ikz}} جو محور \عددی{z} رخ حرکت کرتی ہو کا سامنا ایک بکھراو مخفیہ سے ہوتا ہے جس کے نتیجہ میں ایک کروی رخصتی موج پیدا ہوتی ہے \حوالہء{شکل \num{11.4}} یعنی ہم مساوات شروڈنگر کے وہ حل تلاش کرنا چاہتے ہیں جن کی عمومی روپ درج ذیل ہو
\begin{align}
	\psi(r, \theta)\approx A\left\{e^{ikz}+f(\theta)\frac{e^{ikr}}{r}\right\}, && \text{\RL{کے لیئے}} r \text{\RL{بڑے}}
\end{align}
کروی موج میں جز ضربی \عددی{1/r} پایا جاتا ہے چونکہ احتمال کی بقا کے خاطر \عددی{\abs{\psi}^2} کا یہ حصہ \عددی{1/r^2} کے لحاظ سے تبدیل ہوگا۔ عدد موج \عددی{k} کا آمدی ذرات کی توانائی کے ساتھ ہمیشہ کی طرح درج ذیل رشتہ ہوگا 
\begin{align}
	k\equiv\frac{\sqrt{2mE}}{\hslash}
\end{align}
یہاں بھی میں فرض کرتا ہوں کہ ہدف اسمتی تشاکلی ہے زیادہ عمومی صورت میں رخصتی کروی موج کا حیطہ \عددی{f} متغیرات \عددی{\فاے} اور \عددی{\تھیٹا} کا تابع ہوگا۔

ہمیں حیطہ بکھراو \عددی{f(\تھیٹا)} تعین کران ہوگا۔ یہ ہمیں کسی مخصوص رخ \عددی{\تھیٹا} میں بکھراو کا احتمال دیتا ہے اور یوں اس کا تعلق تفریقی عمودی تراش سے ہوگا۔ یقیناً سمتی رفتار \عددی{v} پر چلتے ہوئے ایک آمدی ذرہ کا وقت \عددی{\dif t} میں لامتناہی چھوٹی رقبہ \عددی{\dif\سگما} میں سے گزرنے کا احتمال \حوالہء{شکل \num{11.5}} دیکھیں درج ذیل ہوگا
\begin{align*}
	\dif P = \abs{\psi_{\text{\RL{آمدی}}}}^2\dif V = \abs{A}^2(v\dif t)\dif\sigma
\end{align*}
لیکن مطابقتی ٹھوس زاویہ \عددی{\dif\Omega} میں اس ذرہ کے بکھاو کا احتمال 
\begin{align*}
	\dif P = \abs{\psi_{\text{\RL{بکھرا}}}}^2\dif V = \frac{\abs{A}^2\abs{f}^2}{r^2}(v\dif t)r^2\dif\Omega
\end{align*}
بھی یہی ہوگا لحاظہ \عددی{\dif\sigma=\abs{f}^2\dif\Omega} اور درج ذیل ہوں گے
\begin{align}
		D(\theta) = \frac{\dif\sigma}{\dif\Omega} = \abs{f(\theta)}^2
\end{align}
ظاہر ہے کہ تفرقی عمودی تراش جس میں تجربہ کرنے والا دلچسمی رکھتا ہے حیطہ بکھراو جو مساوات ژروڈنگر کے حل سے حاصل ہوگا کی مطلق مربع کے برابر ہوگا آنے والے حصوں میں ہم حیطہ بکھراو کی حساب کے دو تراکیب جزوی موج تجزیہ اور بارن تخمین پر غور کریں گے۔

\ابتدا{سوال}
ایک بُعدی اور دو ابعادی بکھراو کے لیئے \حوالہء{مساوات \num{11.12}} کے مماثل تیار کریں۔
\انتہا{سوال} 

