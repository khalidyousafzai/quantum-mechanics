
میں آپ کو 
$1/2$
چکر سے متلقہ ایک فرضی پیمائسی تجربا سے گزرتا ہوں۔ چونکہ یہ ان تصوراتی خیالات کی وضاحت کرتا یے جن پر باب ۱ میں تبصرا کیا گیا۔ فرض کریں ایک زرا حال
$\psi_+$
 میں پایا جاتا ہے۔ اب اگر کوئی سوال پوچھے کہ اس زرے کی زاویائی چکری میارِ حرکت کا 
 z
 جز کیا ہے۔ تب ہم پورے یقین کے ساتھ جواب دے سکتے ہیں کہ اس کا جواب   
 $+\hslash/2$
 یوگا۔ چونکہ 
 z
 کی پیمائس لازمن یہی قیمت دے گی۔ اس کے بجائے اگر پوچھنے والا سوال کرے کہ اس زرے کی چکریا زاویائی میارِ حرکت کا
 x
 جز کیا ہوگا۔ تب ہم یہ کہنے پر مجبور یونگے کہ 
 $S_x$
 کی پیمائس سے 
 $+\hslash/2$
  یا
  $-\hslash/2$
کے حصول کا احتمال آدھا آدھا ہے۔ گر  سوال پوچھنے والا کلاسیکی ماحرِ تبیات یا حصہ ۱۔۲ کے نقطہِ نزر سے حقیقت پسند ہو تو وہ اس جواب کو ناکافی سمجھے گا۔ کیا آپ یہ کہنا چاہتے ہیں کہ آپ کو اس زرے کا حقیقی حال معلوم نہیں ہے۔ نہیں میں نے یہ تو نہیں کہا!۔ مجھے زرے کا حال تھیک تھیک معلوم ہے اور یہ 
$\psi_+$
یے۔ یب ایسا کیوں ہے کہ آپ مجھے اس کے چکر کا 
x
جز نہیں بتا سکتے اس لیئے کہ اس کے چکر کا کوئی مخصوس 
x
جز نہیں پایا جاتا ہے۔ یقینن ایسا ہی ہوگا۔ اگر 
$S_x$
اور
$S_z$
کی قیمتیں تائین ہوں تب اصولِ ادم یقینیت متمئن نہیں ہوگا۔ یہ سنتے ہی سوال کرنے والا زرے کی چکر کا 
x
 جز از خود پیمائس کرتا ہے۔ اب فرض کریں کہ وہ
 $+\hslash/2$
 قیمت حاصل کرتا ہے۔ وہ خوشی سے چلا اٹھا ہے۔ اس زرے کی 
 $S_x$
 قیمت ٹھیک
 $+\hslash/2$
 یے۔ جی آپ درست فرماتے ہیں اب اس کی یہی قیمت ہے۔ جس سے یہ بلکل سابت نہیں ہوتا کہ تجربہ سے پہلے بھی اس کی یہی قیمت تھی۔ اب ظاہر ہے آپ بال کی کھال اتار رہے ہو اور آپ کی ادم یقینیت اصول کا کیا بنا۔ میں اب 
 $S_x$
 اور 
  $S_z$
  دونوں کو جانتا ہوں۔ جی نہیں آپ نہیں جانتے  
ہیں۔ آپ نے پیمائس کے دوران زرے کا حال تبدیل کر دیا ہے۔ اب وہ  
$\psi_+$
%x^3 5:48
اور اگرچہ آپ اس کے 
$S_x$
کی قیمت جانتے ہیں۔ آپ
$S_z$
کی قیمت اب نہیں جانتے ہیں۔ لیکن میں نے 
 $S_x$
 کی پیمائس کے دوران ہم نے پوری کوسس کی کہ میں زرے کا سکون برباد نہ کروں۔ اچھا اگر آپ میری بات پر یقین نہیں کرتے تو خود تصدیق کریں۔ آپ
$S_z$
کی پیمائس کریں اور دیکھیں کہ کیا نتیجہ حاصل ہوتا ہے۔ عین ممکن ہے کہ وہ 
 $\hslash/2$
حاصل کرے جو میرے لیئے سرمندگی کا عصر ہوگا۔ اگر ہم اس پورے عمل کو بار بار دورائیں تو یہ سب اوقات اسے   
$-\hslash/2$
حاصل ہوگا۔ یہ کام آدمی کے لیئے

