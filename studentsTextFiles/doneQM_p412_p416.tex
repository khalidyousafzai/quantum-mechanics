
\جزوحصہ{بارن تخمین اوّل}
فرض کریں \عددی{r_0 = 0} پر \عددی{V(r_0)} مکامی مخفیہ ہے یعنی کسی متناہی خطہ کے باہر مخفیہ کی قیمت صفر ہے جو عموماً مسئلہ بکھراؤ میں ہعگا اور ہم مرکز بکھراؤ سے دور نکات پر \عددی{\psi(r)} جاننا چاہتے ہیں۔ ایسی صورت میں \حوالہء{مساوات \num{11.67}} کی تکمل میں حصہ ڈالنے والے تمام نکات کے لیئے \عددی{\abs{r}\gg\abs{r_0}} ہوگا لحاظہ
\begin{align}
	\abs{r-r_0}^2 = r^2+r_0^2-2r\cdot r_0 \cong r^2\left(1-2\frac{r\cdot r_0}{r^2}\right)
\end{align}
اور یوں درج ذیل ہوگا
\begin{align}
	\abs{r-r_0}^2\cong r-\hat{r}\cdot r_0
\end{align}
ہم 
\begin{align}
	k\equiv k\hat{r}
\end{align}
لیتے ہیں۔ یوں
\begin{align}
	e^{ik\abs{r-r_0}}\cong e^{ikr}e^{-ik\cdot r_0}
\end{align}
ہوگا۔ لحاظہ درج ذیل ہوگا 
\begin{align}
	\frac{e^{ik\abs{r-r_0}}}{\abs{r-r_0}}\cong\frac{e^{ikr}}{r}e^{-ik\cdot r_0}
\end{align}
نصب نما میں ہم زیادہ بڑی تخمین \عددی{\abs{r-r_0}\cong r} دے سکتے ہیں قوت نما میں ہمیں دوسرا جز بھی رکھنا ہوگا۔ اگر آپ یقین نہیں کر سکتے ہیں تو نصب نما میں دوسرے جز کو پہلا کر دیکھیں ہم یہاں ایک چھوٹی مقدار \عددی{(r_0/r)} کی قوتوں میں پھیلا کر کم سے کم رتبی جز کے علاوہ باقی تمام کو رد کرتے ہیں۔

بکھراؤ کی صورت میں ہم درج ذیل چاہتے ہیں۔ جو آمدی مستوی موج کو ظاہر کرتا ہے
\begin{align}
	\psi_0(r) = Ae^{ikz}
\end{align}
یوں بڑی \عددی{r} کے لیئے درج ذیل ہوگا 
\begin{align}
	\psi(r)\cong Ae^{ikz}-\frac{m}{2\pi\hslash^2}\frac{e^{ikr}}{r}\int e^{ik\cdot r_0}V(r_0)\psi(r_0)\dif^3r_0
\end{align}
یہ معیاری روپ \حوالہء{مساوات \num{11.12}} ہے جس سے ہم حیطہ بکھراؤ پڑھ سکتے ہیں
\begin{align}
	f(\theta, \phi) = -\frac{m}{2\pi\hslash^2A}\int e^{-ik\cdot r_0}V(r_0)\psi(r_0)\dif^3r_0 
\end{align}
یہاں تک یہ بلکل ایک درست جواب ہے ہم اب بارن تخمین باروہِ کار لاتے ہیں۔ فرض کریں آمدیہ مستوی موج کو مخفیہ قابلِ ذکر تبدیل نہیں کرتا ہو ایسی صورت میں درج ذیل استعمال کرنا معقول ہوگا
\begin{align}
	\psi(r_0)\approx\psi_0(r_0) = Ae^{ikz_0} = Ae^{ik'\cdot r_0}
\end{align}
جہاں تکمل کے اندر \عددی{k'} درج ذیل ہے
\begin{align}
	k'\equiv k\hat{z}
\end{align}
مخفیہ \عددی{V} صفر ہونے کی صورت میں یہ بلکل ٹھیک تفاعل موج ہوتا یہ بنیادی طور پر کمزور مخفیہ تخمین ہے۔ بارن تخمین میں یوں درج ذیل ہوگا 
\begin{align}
	f(\theta, \phi)\cong-\frac{m}{2\pi\hslash^2}\int e^{i(k'-k)\cdot r_0}V(r_0)\dif^3r_0
\end{align}
ہوسکتا ہے کہ آپ \عددی{k'} اور \عددی{k} کی تعریفات بھول چکے ہوں دونوں کی مقدار \عددی{k} ہے تاہم اوّل الذکر کا رخ آمدی شعاع کے رخ ہے جبکہ معاخرالذکر کا رخ کاشف کے رخ ہے \حوالہء{شکل \num{11.11}} دیکھیں۔ اس عمل میں \عددی{\hslash(k-k')} منتقلی معیارِ حرکت کو ظاہر کرے گا بلخصوص خطہ بکھراؤ پر کم توانائی لمبی طولِ موج بکھراؤ کے لیئے قوتِ نمائی جز ضربی بنیادی طر پر مستقل ہوگا اور یوں تخمین بارن درج ذیل سادہ روپ اختیار کرے گا   
\begin{align}
	f(\theta, \phi)\cong-\frac{m}{2\pi\hslash}\int V(r)\dif^3r, && \text{\RL{کم توانائی}}
\end{align}
میں نے یہاں \عددی{r} کے زیرِ نوشت میں کچھ نہیں لکھا اُید کی جاتی اس سے کوئی پریشانی پیدا نہیں ہوگی۔

\ابتدا{مثال}
کم توانائی نرم کرہ بکھراؤ درج ذیل مخفیہ لیں 
\begin{align}
	V(r)=
	\begin{cases}
		V_0, & r\leq a \text{\RL{اگر}} \\
		0, & r>a \text{\RL{اگر}}
	\end{cases}
\end{align}
کم توانائی کی صورت میں \عددی{\theta} اور \عددی{\phi} کا غیر تابع حیطہ مکھراؤ درج ذیل ہوگا۔
\begin{align}
	f(\theta, \phi)\cong-\frac{m}{2\pi\hslash^2}V_0\left(\frac{4}{3}\pi a^3\right)
\end{align}
تفریقی عمودی تراش 
\begin{align}
	\frac{\dif\sigma}{\dif\Omega}=\abs{f}^2\cong\left(\frac{2mV_0a^3}{3\hslash^2}\right)^2
\end{align}
اور کل عمودی تراش درج ذیل ہوگا۔ 
\begin{align}
	\sigma\cong4\pi\left(\frac{2mV_0a^3}{3\hslash^2}\right)^2
\end{align}
\انتہا{مثال}
ایک کروی تشاکلی مخفیہ \عددی{V(r)=V(r)} کے لیئے جو ضروری نہیں کہ کم توانائی پر ہو تخمین بارن دوبارہ سادہ روپ اختیار کرتا ہے۔ درج ذیل متعارف کرتے ہوئے 
\begin{align}
	\kappa\equiv k'-k
\end{align}
\عددی{r_0} تکمل کے قطبی محور کو \عددی{\kappa} پر رکھتے ہوئے درج ذیل ہوگا 
\begin{align}
	(k'-k)\cdot r_0 = \kappa r_0\cos\theta_0
\end{align}
یوں درج ذیل حاصل ہوگا
\begin{align}
	f(\theta)\cong-\frac{m}{2\pi\hslash^2}\int e^{i\kappa r_0\cos\theta_0}V(r_0)r^2_0\sin\theta_0\dif r_0\dif\theta_0\dif\phi_0
\end{align}
متغیر \عددی{\phi_0} کے لحاظ سے تکمل \عددی{2\pi} دیگا اور \عددی{\theta_0} تکمل کو ہم پہلے دیکھ چکے ہیں \حوالہء{مساوات \num{11.59}} دیکھیں۔ یوں \عددی{r} کے زیرِنوشت کو نہ لکھتے ہوئے درج ذیل رہ جائے گا
\begin{align}
	f(\theta)\cong-\frac{2m}{\hslash^2\kappa}\int_{0}^{\infty}rV(r)\sin(\kappa r)\dif r، && \text{\RL{کروی تشاکل}}
\end{align}
\عددی{f} کی زیویائی تابیعت \عددی{\kappa} میں سموئی گئی ہے \حوالہء{شکل \num{11.11}} کو دیکھ کر درج ذیل لکھا جا سکتا ہے
\begin{align}
	\kappa = 2k\sin(\theta/2)
\end{align}
\ابتدا{مثال}
یوکاوا بکھراؤ۔ یوکاوا مخفیہ جو جوہری مرکزہ کے بیچ بندشی قوت کا ایک سادہ نمونہ پیش کرتا ہے کا روپ درج ذیل ہے جہاں \عددی{\beta} اور \عددی{\mu} مستقلات ہیں
\begin{align}
	V(r) = \beta\frac{e^{-\mu r}}{r}
\end{align}
تخمین بارن درج ذیل دیگا 
\begin{align}
	f(\theta)\cong-\frac{2m\beta}{\hslash^2\kappa}\int_{0}^{\infty}e^{-\mu r}\sin(\kappa r)\dif r=-\frac{2m\beta}{\hslash(\mu^2+\kappa^2)}
\end{align}
آپ کو \حوالہء{سوال \num{11.11}} میں یہ تکمل حل کرنے کو کہا گیا ہے۔
\انتہا{مثال}
\ابتدا{مثال}
ردرفورڈ بکھراؤ۔ مخفیہ یوکاوا میں \عددی{\beta=q_1q_2/4\pi\epsilon_0} اور \عددی{\mu=0} پُر کرنے سے مخفیہ کولمب حاصل ہوگا جو دو نقطی باروں کے بیچ برقی باہم عمل کو بایان کرتا ہے۔ ظاہر ہے کہ حیطہ بکھراؤ درج ذیل ہوگا 
\begin{align}
	f(\theta)\cong-\frac{2mq_1q_2}{4\pi\epsilon_0\hslash^2\kappa^2}
\end{align}
یا \حوالہء{مساوات \num{11.89} اور \num{11.51}} استعمال کرتے ہوئے درج ذیل ہوگا 
\begin{align}
	f(\theta)\cong-\frac{q_1q_2}{16\pi\epsilon_0E\sin^2(\theta/2)}
\end{align}
اس کا مربع ہمیں تفریقی عمودی تراش دیگا 
\begin{align}
	\frac{\dif\sigma}{\dif\Omega}=\left[\frac{q_1q_2}{16\pi\epsilon_0E\sin^2(\theta/2)}\right]^2
\end{align}
جو ٹھیک کلیہ ردرفورڈ \حوالہء{مساوات \num{11.11}} ہے۔ آپ دیکھ سکتے ہیں کہ کولمب مخفیہ کے لیئے کالسیکی میکانیات تخمین بارن اور کوانٹم نظریہ میدان تمام ایک دوسرے جیسا نتیجہ دیتے ہیں۔ ہم کہہ سکتے ہیں کہ کلیہ ردرفورڈ ایک مضبوط کلیہ ہے۔
\انتہا{مثال}
\ابتدا{سوال}
اختیاری توانائی کے لیئے نرم کرہ بکھراؤ کا حیطہ بکھراؤ بارن تخمین سے حاصل کریں دیکھائیں کہ کم توانائی حد میں اس سے \حوالہء{مساوات \num{11.82}} حاصل ہوگا۔
\انتہا{سوال}
\ابتدا{سوال}
\حوالہء{مساوات \num{11.91}} میں تکمل کی قیمت تلا کر کے دائیں ہاتھ ریاضی فکرہ کی تصدیق کریں۔
\انتہا{سوال}
\ابتدا{سوال}
بارن تخمین میں یوکاوا مخفیہ سے بکھراؤ کا کل عمودی تراش تلاش کریں۔ اپنے جواب کو \عددی{E} کا تفاعل لکھیں۔
\انتہا{سوال}
\ابتدا{سوال}
درج ذیل اقدام \حوالہء{سوال \num{11.4}} کے مخفیہ کے لیئے کریں۔

(الف) کم توانائی تخمین بارن میں \عددی{f(\theta, D(\theta))} اور \عددی{\sigma} کا ھساب لگائیں۔

(ب) تخمین بارن میں اختیاری توانائیوں کے لیئے \عددی{f(\theta)} کا حساب لگائیں۔

(ج) دیکھائیں کہ آپ کے نتائج مناسب خطوں میں \حوالہء{سوال \num11.4} کے جواب کے مطابق ہیں۔
\انتہا{سوال}

