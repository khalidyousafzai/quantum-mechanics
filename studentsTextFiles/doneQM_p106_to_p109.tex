%%%%%%%%%%%%%%%%%%%%%%%%%%%%%%%%%%%%%%%%%%%%%%%%%%%%%%%%%%%%%%%%%%%
%		Page 106-109
%		Contributors:
%			Mirza Akbar Ali
%			Ayesha Jamshaid
%			Misbah Batool
%
%%%%%%%%%%%%%%%%%%%%%%%%%%%%%%%%%%%%%%%%%%%%%%%%%%%%%%%%%%%%%%%%%%%
% 			Section 3.4 Generalized Statistical 
% 						Interpretation
%%%%%%%%%%%%%%%%%%%%%%%%%%%%%%%%%%%%%%%%%%%%%%%%%%%%%%%%%%%%%%%%%%%
\حصہ{عمومی شماریاتی مفہوم}
ایک ذرے کی کسی ایک مخصوص مقام پر پائے جانے والے احتمال کا حساب اور کسی بھی قابل مشاہدہ مقدار کی توقعاتی قیمت تعین کرنا میں نے آپ کو باب \حوالہ{1} میں دکھایا ۔باب \حوالہ{2} میں آپ نے توانائی کی پیمائش کے ممکنہ نتائج اور ان کا احتمال حاصل کرنا سیکھا۔میں اب اس قابل ہوں کہ آپ کو عمومی شماریاتی مفہوم پیش کروں جس میں یہ تمام شامل ہے اور جو ہر پیمائش کی تمام ممکنہ نتائج اور ان کا احتمال حاصل کرنے کے قابل بناتی ہے۔ اس کے ساتھ سکروڈنگر مساوات( جو وقت کے ساتھ تفاعل موج کی تبدیلی کے بارے میں ہمیں بتاتی ہے) کوانٹم مکینیات کی بنیاد ہے۔
عمومی شماریاتی مفہوم : حال \عددی{  \Psi(x,t) } میں ایک ذرے کی قابل مشاہدہ \عددی{ Q(x,P)  } کی پیمائش ہر صورت ہرمیشی حامل \عددی{ \hat{Q}(x, -i\hbar d/dx)  } کی کوئی ایک امتیازی قدر دے گی اگر \عددی{  \hat{Q} } کا طیف غیر مسلسل ہو تب معیاری عمودی امتیازی تفاعل \عددی{  f_{n}(x) } سے منسلک کوئی مخصوص امتیازی قدر \عددی{  q_{n} } کے حصول کا احتمال درج ذیل ہو گا 
\begin{align}
|c_{n}|^{2}, \quad \text{ جہاں} c_{n} = \langle f_{n}|\Psi \rangle
\end{align}
استمراری طیف کی صورت میں جہاں حقیقی امتیازی اقدار \عددی{  q(z) }اور منسلک ڈیراک معیاری عمودی امتیازی تفاعل \عددی{  f_{z}(x) } ہو، ساتھ \عددی{  dz } میں نتیجے کا حصول درج ذیل ہو گا 
\begin{align}
|c(z)|^{2} dz \quad \text{where} \quad c(z) = \langle f_{z}|\Psi \rangle
\end{align}
پیمائش کی عمل کے بناتفاعل موج مطابقتی امتیازی حال پر آ بیٹھتا ہے۔
شماریاتی تشریح یا مفہوم ان تمام چیزوں سے یکسر مختلف ہے جس سے کلاسیکی طبیعیات میں ہمارا آمنا سامنا ہوا ۔اس کو ایک مختلف نقطہ نظر سے دیکھنا بہتر ہو گا: چونکہ قابل مشاہدہ حامل کے امتیازی تفاعل مکمل ہیں لہٰذا تفاعل موج کو ان کا ایک خطی جوڑ لکھا جا سکتا ہے۔
\begin{align}
\Psi(x,t) = \sum_{n} c_{n}f_{n}(x)
\end{align}
اپنی آسانی کے لیے میں فرض کرتا ہوں کہ طیف غیر مسلسل ہے ان دلائل کی تعمیل استمراری صورت کے لیے بآسانی کی جا سکتی ہے چونکہ امتیازی تفاعل معیاری عمودی ہیں لہٰذا ان کے عددی سر کو فوررئیر کی ترکیب سے حاصل کیا جا سکتا ہے۔
\begin{align}
c_{n} = \langle f_{n} | \Psi \rangle = \int f_{n}(x)^{*} \Psi(x,t)dx
\end{align}
طیفی طور پر \عددی{ \Psi} میں \عددی{  f_{n} } کی مقدار کو \عددی{   c_{n}} ظاہر کرتی ہے اور چونکہ کوئی بھی پیمائش \عددی{  \hat{Q} } کی کوئی ایک امتیازی قدر دے گی لہٰذا ہم توقع کرتے ہیں کہ وہ مخصوص امتیازی قدر \عددی{  q_{n} } کے حصول کا احتمال \عددی{  \Psi } میں \عددی{  f_{n} } کی مقدار پر منحصر ہو گا۔لیکن چونکہ احتمال کو تفاعل موج کی مطلق قیمت کا مربع تعین کرتا ہے لہذا پیمائش کی ٹھیک ٹھیک قدر \عددی{  |c_{n}|^{2} } ہو گی ۔ عمومی شماریاتی تشریح کا یہ ایک اثر ہے ہاں تمام ممکنات کا کل احتمال ایک ہو گا۔
\begin{align}
\sum_{n} |c_{n}|^{2} = 1
\end{align}
جو یقیناً تفاعل موج کو معمول پر لانے سے حاصل ہوتا ہے۔
\begin{align}
1 &= \langle \Psi | \Psi \rangle = \left\langle \left( \sum_{n^{'}}c_{n^{'}}f_{n^{'}}\right) \bigg| \left( \sum_{n} c_{n}f_{n} \right) \right\rangle = \sum_{n^{'}}\sum_{n} c_{n^{'}}^{*}c_{n}\langle f_{n^{'}} | f_{n} \rangle \nonumber \\
&= \sum_{n^{'}}\sum_{n} c_{n^{'}}^{*}c_{n} \delta_{n^{'}n} = \sum_{n}c_{n}^{*}c_{n} = \sum_{n}|c_{n}|^{2}
\end{align}
اسی طرح تمام ممکنہ امتیازی قدروں اور ان قدروں کے حصول کے احتمال کا حاصل ضرب کا مجموعہ\عددی{  Q } کی توقعاتی قیمت ہو گی۔
\begin{align}
\langle Q \rangle = \sum_{n}q_{n}|c_{n}|^{2}.
\end{align}
یقیناً درج ذیل ہو گا 
\begin{align}
\langle Q \rangle = \langle \Psi | \hat{Q} \Psi \rangle =  \left\langle \left( \sum_{n^{'}}c_{n^{'}}f_{n^{'}}\right) \bigg| \left( \hat{Q} \sum_{n} c_{n}f_{n} \right) \right\rangle
\end{align}
لیکن \عددی{  \hat{Q}f_{n} = q_{n}f_{n} } لہٰذا درج ذیل ہو گا۔
\begin{align}
\langle Q \rangle = \sum_{n^{'}}\sum_{n} c_{n^{'}}^{*}c_{n} q_{n} \langle f_{n^{'}} | f_{n} \rangle = \sum_{n^{'}}\sum_{n} c_{n^{'}}^{*}c_{n} q_{n} \delta_{n^{'}n} \sum_{n}q_{n}|c_{n}|^{2}.
\end{align}
کم از کم یہاں تک چیزیں ٹھیک نظر آ رہی ہیں۔
کیا ہم مقام کی پیمائش کی اصل شماریاتی مفہوم کو اس زبان میں پیش کر سکتے ہیں؟ جی ہاں آئیں اس کی تصدیق کرتے ہیں۔ حال \عددی{  \Psi } میں ایک ذرے کے لیے \عددی{  x } کی پیمائش لازماً حامل مقام کا کوئی ایک امتیازی قدر دے گا۔ ہم مثال \حوالہ{3.3} میں دیکھ چکے ہیں کہ ہر حقیقی عدد \عددی{   y} متغیر \عددی{  x } کا امتیازی قدر ہے اور اس کا مطابقاتی ڈیراک معیاری عمودی امتیازی تفاعل \عددی{  g_{y}(x) = \delta (x-y) } ہے ۔ ظاہر ہے کہ درج ذیل ہو گا۔
\begin{align}
c(y) = \langle g_{y} | \Psi \rangle  \int_{-\infty}^{\infty} \delta(x-y) \Psi(x,t)dx = \Psi(y,t).
\end{align}
لہٰذا ساتھ \عددی{  dy } میں نتیجے حاصل کرنے کا احتمال \عددی{ | \Psi(y,t)|^{2}  } ہو گا جو ٹھیک اصل شماریاتی مفہوم ہے۔
معیار حرکت کے لیے کیا ہو گا؟ ہم مثال \حوالہ{3.2} میں دیکھ چکے ہیں کہ معیار حرکت کے حامل کی امتیازی تفاعل \عددی{ f_{p}(x) = (1/\sqrt{2\pi\hbar} exp(ipx/\hbar)  } ہے لہٰذا درج ذیل ہو گا
\begin{align}
c(p) = \langle f_{p}  | \Psi \rangle = \frac{1}{\sqrt{2\pi\hbar}} \int_{-\infty}^{\infty} e^{-ipx/\hbar} \Psi (x,t)dx
\end{align}
یہ ایک اتنی اہم مقدار ہے کہ ہم اسے ایک مخصوص نام اور ایک مخصوص علامت سے ظاہر کرتے ہیں اس کو معیار حرکت و فضائی تفاعل موج پکارا جاتا ہے جسے \عددی{  \Psi(p,t) } سے ظاہر کیا جاتا ہے ۔یہ بنیادی طور پر مقام اور فضائی تفاعل موج \عددی{  \Psi(x,t) } کا فوررئیر بدل ہے جو مسئلہ پلانشرال کے تحت اس کا الٹ فوررئیر بدل ہے
\begin{align}
\Phi(p,t) &= \frac{1}{2\pi\hbar} \int_{-\infty}^{\infty} \Psi (x,t) dx, \\ 
\Psi(x,t) &= \frac{1}{2\pi\hbar} \int_{-\infty}^{\infty} \Phi (p,t) dp,
\end{align}
عمومی شماریاتی مفہوم کے تحت ساتھ \عددی{  dp } میں معیار حرکت کی پیمائش کے حصول کا احتمال درج ذیل ہو گا 
\begin{align}
| \Psi(p,t)|^{2}dp
\end{align}
مثال 3.4 : ایک ذرہ جس کی کمیت \عددی{  m } ہے ڈیلٹا طفاعل کنواں \عددی{ V(x) = -\alpha \delta (x)  } میں مقید ہے۔ اس بات کا کیا احتمال ہے کہ اس کے معیار حرکت کی پیمائش \عددی{ p_{0} = m\alpha /\hbar } سے بڑی قیمت دے گا؟ 
حل: اس کا مقام و فضائی تفاعل موج مساوات \حوالہ{2.129} سے پیش کیا گیا ہے
\begin{align}
\Psi(x,t) = \frac{\sqrt{m\alpha}}{\hbar} e^{-m\alpha|x|/\hbar^{2}}e^{-iEt/\hbar}
\end{align}
 جہاں \عددی{ E = -m\alpha^{2}/2\hbar^{2} } ہے۔ لہٰذا معیار حرکت و فضائی تفاعل موج درج ذیل ہو گا
\begin{align*}
\Phi(p,t) = \frac{1}{\sqrt{2\pi\hbar}} \frac{\sqrt{m\alpha}}{\hbar} e^{-iEt/\hbar} \int_{-\infty}^{\infty} e^{-ipx/\hbar} e^{-m\alpha|x|/\hbar^{2}}dx = \sqrt{\frac{2}{\pi}} \frac{p_{0}^{3/2}e^{-iEt/\hbar}}{p^{2}+p_{0}^{2}}
\end{align*}
میں نے یہ تکمل جدول سے لکھا ہے اس کا احتمال درج ذیل ہو گا 
\begin{align*}
\frac{2}{\pi} p_{0}^{3}\int_{p_{0}}^{\infty} \frac{1}{(p^{2}+p_{0}^{2})^{2}}dp &= \frac{1}{\pi} \left[ \frac{pp_{0}}{p^{2}+p_{0}^{2}} + \tan^{-1}\left( \frac{p}{p_{0}} \right) \right) \bigg|_{p_{0}}^{\infty} \\
&= \frac{1}{4} - \frac{1}{2\pi} = 0.0908
\end{align*}
یہاں بھی میں نے تکمل کی قیمت جدول سے دیکھ کر لکھی ہے ۔


\عددی{  }




% Problem 3.11
\ابتدا{سوال}
ہارمونی مرتعش کی زمینی حال میں ایک ذرے کی معیاری حرکت و فضائی تفاعل موج \عددی{  \Phi(p,t) } تلاش کریں ۔اس حال میں ایک ذرہ کے \عددی{  p } کی پیمائش کا اسی توانائی کے کلاسیکی ذات کے باہر نتیجے کا احتمال دو با معنی ہندسوں تک کیا ہو گا؟ اشارہ :عددی جواب کے لیے عمومی تقسیم یا تفاعل حلل سے حاصل کریں۔
\انتہا{سوال}
% Problem 3.12
\ابتدا{سوال}
درج ذیل دکھائیں
\begin{align}
\langle x \rangle = int \Psi^{*} \left( - \frac{\hbar}{i} \frac{\partial}{\partial p} \right) \Psi dp.
\end{align}
اشارہ : دھیان رہے کہ \عددی{ x exp(ipx/\hbar) = -i\hbar (d/dp) exp(ipx/\hbar) } یوں معیار حرکت فضا میں حامل مقام \عددی{ i\hbar \partial/\partial p } ہو گا ۔زیادہ عمومی درج ذیل ہوں گے
\begin{align}
\langle Q(x,p)\rangle =  \begin{array}{ll} 
\int \Psi^{*} \hat{Q} \left( x, \frac{\hbar}{i} \frac{\partial}{\partial x} \right) \Psi dx, & \text{in position space} \\
\int \Psi^{*}\hat{Q} \left( -\frac{\hbar}{i} \frac{\partial}{\partial p},p \right) \Phi dp, & \text{in momentum space} 
\end{array}
\end{align}
اصولی طور پر آپ تمام حساب و کتاب مقامی فضا کی بجائے معیار حرکت فضا میں کر سکتے ہیں)اگرچہ ایسا کرنا عموماً زیادہ آسان نہیں ہوگا۔(
\انتہا{سوال}
