
\حصہ{مسئلہ کلمیہ}
کوانٹم پیمائش عموماً تباہ کن ہوتے ہیں یعنی یہ پیمائش کردہ نظام کے حال کو تبدیل کرتا ہے۔ یہی تجربہ گاہ میں اصول عدم یقینیت کو یقینی بناتا ہے ہم کیوں اصل حال کی کئی متماثل نقل کلمیہ بنا کر اصل نظام کو چھوئے بغیر ان کی پیمائش نہیں کرتے ایسا کرنا ممکن نہیں ہے۔ اگر آپ کلمیہ بنانے والا ایسا آلا بنا پائیں تو کوانٹم میکانیات کو خداحافظ کہنا ہوگا۔

مثال کے طور پر آئنسٹائن، پوڈلسکی، روزن اور بوہم تجربہ کے ذریعہ روشنی سے تیز رفتار پر خبر بھیجنا ممکن ہوگا فرض کریں پوزیٹران کاشف چلانے ولا شخص ہاں یا نہیں کی خبر ترسیل کرتا ہے۔ خبر ہاں ہونے کی صورت میں بھیجنے والا پوزیٹران کا \عددی{S_z} ناپتا ہے یہ جاننے کی ضرورت نہیں کہ پیمائشی نتیجہ کیا ہے صرف اتنا جاننا ضروری ہے کہ پیمائش کی گئی ہے یوں الیکٹران کسی غیر مبہم حال \عددی{\uparrow} یا \عددی{\downarrow} میں ہوگا جسکا جاننا غیر اہم ہے۔ خبر وصول کرنے والا جلدی سے الیکٹران کی دس لاکھ کلمیہ تیار کر کے ہر ایک کی \عددی{S_z} ناپتا ہے اگر تمام کا ایک ہی جواب ہو کونسا جواب یہ جاننا ضروری نہیں ہم یقین سے کہہ سکیں گے کہ الیکٹران کی پیمائش کی گئی لحاظہ خبر ہاں ہوگی۔ اس کے برعکس اگر نصف الیکٹران ہم میدان اور نصف خلاف میدان ہوں تب یقیناً الیکٹران کی پیمائش نہیں کی گئی اور خبر نہیں ہوگا۔

لیکن سن \num{1982} ووٹرز، زورک اور ڈائکس نے ثابت کیا کہ ایسا مشین تیار نہیں کیا جا سکتا ہے جو کوانٹم متماثل ذرات پیدا کرتا ہو ہم چاہیں گے کہ یہ مشین حال \عددی{\mid\psi\rangle} میں ایک ذرہ جس کا نقل بنانا مقصود ہو اور حال \عددی{\mid X\rangle} میں ایک اضافی ذرہ لی کر  حال \عددی{\mid\psi\rangle} میں دو ذرات  اصل اور نقل دیتا ہو 
\begin{align}
	\mid\psi\rangle\mid X\rangle\to\mid\psi\rangle\mid\psi\rangle
\end{align}
فرض کریں ہم ایسا مشین بنانے میں کامیاب ہوتے ہیں جو حال \عددی{\mid\psi_1\rangle} کا کلمہ تیار کرتا ہو 
\begin{align}
	\mid\psi_1\rangle\mid X\rangle\to\mid\psi_1\rangle\mid\psi_1\rangle
\end{align}
اور \عددی{\mid\psi_2\rangle} پر بھی کام کرنے کے قابل ہو
\begin{align}
	\mid\psi_2\rangle\mid X\rangle\to\mid\psi_2\rangle\mid\psi_2\rangle
\end{align}
مثال کے طوور پر اگر ذرہ ایک الیکٹران ہو تب \عددی{\mid\psi_1\rangle} اور \عددی{\mid\psi_2\rangle} ہم میدان اور خلاف میدان ہو سکتے ہیں۔ یہاں تک کوئی مسئلہ پیدا نہیں ہوتا یہ دیکھان ہوگا کہ ان کا خطی جوڑ \عددی{\mid\psi\rangle=\alpha\mid\psi_1\rangle+\beta\mid\psi_2\rangle} کی صورت میں کیا ہوگا؟ ظاہر ہے ایسی صورت میں درج ذیل ہوگا
\begin{align}
	\mid\psi\rangle\mid X\rangle\to\alpha\mid\psi_1\rangle\mid\psi_1\rangle+\beta\mid\psi_2\rangle\mid\psi_2\rangle
\end{align}
جو ہم نہیں چاہتے ہیں۔ ہم درج ذیل چاہتے ہیں  
\begin{align}
	\mid\psi\rangle\mid X\rangle\to\mid\psi\rangle\mid\psi\rangle &= [\alpha\mid\psi_1\rangle+\beta\mid\psi_2\rangle][\alpha\mid\psi_1\rangle+\beta\mid\psi_2\rangle]\nonumber \\
	&= \alpha^2\mid\psi_1\rangle\mid\psi_1\rangle+\beta^2\mid\psi_2\rangle\mid\psi_2\rangle+\alpha\beta[\mid\psi_1\rangle\mid\psi_2\rangle+\mid\psi_2\rangle\mid\psi_1\rangle]
\end{align}
آپ ہم میدان الیکٹران اور خلاف میدان الیکٹران کے کلمہ بنانے کی مشین بنا سکتے ہیں لیکن وہ کسی بھی اہم خطی جوڑ کی صورت میں ناکامی کا شکار ہوگا یہ بلکل ایسا ہوگا جیسا نقل بنانے کی مشین  افکی لکیروں اور انتسابی لکیروں کی نقل خوش اصلوبی سے کرتا ہو لیکن وتری لکیروں کو مکمل طور پر بگاڑتا ہو۔

