
\حصہ{احتمال} 
\جزوحصہ{ غیر مسلسل متغیرات} 
چونکہ کوانٹم میکانیات کی شماریاتی تشریح کی جاتی ہے  لہٰذا  اس میں احتمال کلیدی کردار ادا کرتا ہے۔ اسی لیے میں اصل موضوع سے ہٹ کر  نظریہ احتمال پر تبصرہ کرتا ہوں۔ ہمیں چند نئی علامتیں  اور اصطلاحات سیکھنا ہو گا جنہیں میں ایک سادہ مثال کی مدد سے  واضح کرتا ہوں۔ 
فرض کریں ایک کمرہ   میں  \( 14 \)  حضرات موجود ہوں جن  کی  عمریں درج ذیل ہوں۔ 
\begin{itemize}
\item
 \( 14 \) سال عمر کا ایک شخص، 
\item
 \( 15 \) سال عمر کا ایک شخص، 
\item
 \( 16 \) سال عمر کے تین اشخاص، 
\item
 \( 22 \) سال عمر کے دو اشخاص، 
\item
 \( 24 \) سال عمر کے دو اشخاص، 
\item
 اور  \( 25 \) سال عمر کے پانچ  اشخاص۔
\end{itemize} 
اگر ہم \( j \) سال عمر کی تعداد کے لوگوں کو\( N(j) \) لکھیں تب ہم یوں لکھ سکتے ہیں۔ 
\begin{align*}
N(14) &= 1 \\
N(15) &= 1 \\
N(16) &= 3 \\
N(22) &= 2 \\
N(24) &= 2 \\
N(25) &= 5 \\
\end{align*}
جبکہ 
\begin{align*}
N(17) = 0
\end{align*}


کمرے میں لوگوں کئ کل تعداد کو ہم یوں لکھ سکتے ہیں 
\begin{align*}
N(j) = \sum_{j=0}^{\infty} N(j)
\end{align*}

چونکہ اس مثال میں کل \( 14 \) افراد ہیں۔
\( N=14 \)شکل 1.4 میں اس مواد کئ مثططیلی ترسیم دکھائ گئی ہے۔ اس تقسیم کے بارے میں ہم درج ذیل سوالات پوچھ سکتے ہیں۔ 

سوال 1
اگر ہم اس گروہ سے بلامنصوبہ ایک شخص منتخب کریں تو اس بات کی کیا احتمال ہو گی کہ اس شخص کی عمر 15 سال ہو؟
جواب 
چودہ میں ایک امکان ممکن ہے کیونکہ کل  ممکنات ہیں اور ان میں سے ایک شخص کئ عمر \( 15 \) سال ہے۔ اگر J سال عمر کے شخص کے انتخاب کا احتمال\( P(j) \) ہو تب 
\begin{align*}
P(14) &= \frac{1}{14} \\
P(15) &= \frac{1}{15} \\
P(16) &= \frac{3}{14}
\end{align*}
اور آپ اسی طرح باقی بھی دریافت کر سکتے ہیں۔ 

اس کا عمومی کلیہ ہم یوں لکھ سکتے ہیں۔ 
\[P(j) = \frac{N(j)}{N} \]

دھیان رہے کی چودہ یا پندرہ سال عمر کے شخص کے انتخاب کا احتمال ان دونوں کی انفرادی احتمال کا مجموعہ ہو گا۔ یعنی \( \frac{1}{7} \)۔
بالخصوص تمام احتمال کا مجموعہ اکائ کے برابر ہو گا چونکہ آپ کسی نہ کسی عمر کے شخص کو ضرور منتخب کر پائیں گے۔ 

\[ \sum_{j=0}^{\infty} P(j) = 1 \]

سوال 2

کس عمر کا احتمال سب سے زیادہ ہے۔

جواب 

\( 25 \)

پانچ اشخاص اتنی عمر رکھتے ہیں جبکہ اس کے بعد ایک جیسی عمر کے لوگوں کئ تعداد تین ہے۔ عموماً سب سے زیادہ احتمال کا J وہی \( j \) ہو گا حس کیلئے\( P(j) \) کی قیمت زیادہ سے زیادہ ہو گی۔ 


سوال 3

وسطانیہ عمر کیا ہے۔ 

چونکہ سات لوگوں کئ عمر تئیس سے کم اور سات لوگوں  کی  عمر تئیس سے زیادہ ہے۔ لحظہ جواب \( 23 \) ہو گا۔ 

عمومی طور پر وسطانیہ وہ \( j \) ہو گا جس سے زیادہ زیادہ کئ تعداد کا احتمال اور اس سے کم کے احتمال کا امکان ایک جیسا ہوگا 

سوال 4

ان  کی  اوسط عمر کتنی ہے ؟

جواب 

\[ \frac{14+15+3(16)+2(22)+2(24)+5(25)}{14} = \frac{294}{14} \]
 
\( \langle j \rangle \) 

عمومی طور پر \( j \) کی اوسط قیمت جسکو ہم  \( \langle j \rangle \) لکھتے ہیں, درج ذیل ہو گی۔ 

\[ \langle j \rangle = \frac{\sum  j N(j)}{N} = \sum_{j=0}^{\infty} jP(j)  \]

\( 23 \)
دھیان رہے کہ عین ممکن ہے کہ گروہ میں کسی کی عمر بھی اوسط عمر کے برابر نہ ہو یا وسطانیہ عمر کے برابر ہو۔ مثال کے طور پر,  اس مثال میں کسی کی عمر \( 21 \) یا \( 23 \) سال نہیں تھی۔ 
کوانٹم میکانیات میں ہماری دلچسپی عموماً اوسط قیمت میں ہوتی ہے۔ کوانٹم میکانیات میں اسکو قیمت توقع کہتے ہیں۔ 


سوال 5

عمروں کے مربعوں کا اوسط کیا ہو گا ؟

جواب

 
آپ اس کو \( 14^{2} = 196 \) لکھ سکتے ہیں۔ جسکا احتمال\( \frac{1}{14} \) ہے 
اسی طرح\( 15^{2} = 225 \) ہے جسکا احتمال\( \frac{1}{14} \) ہے۔\( 16^{2} = 256 \) جسکا احتمال\( \frac{3}{14} \)ہے۔ 
اور اسی طرح باقی۔ 
یوں اسکا اوسط درج ذیل ہو گا۔ 

\[ \langle j^{2} \rangle  = \sum_{j=0}^{\infty} j^{2} P(j) \] 

عمومی طور پر \( j \) کے کسی بھی تفاعل کئ اوسط قیمت درج ذیل ہو گی۔ 

\[ \langle f(j) \rangle = \sum_{j=0}^{\infty} f(j) P(j) \]

\( 3 \)
مساوات 1.6,1.7,1.8 اس مساوات کی خصوصی صورتی ہیں۔ دھیان رہے کہ مربع کی اوسط قیمت عموماً اوسط کے مربع کے برابر نہیں ہو گا۔ مثال کے طور پر اگر ایک کمرے میں صرف دو بچے ہوں جنکی عمریں\( 1 \) اور\( 3 \) ہو تب 
\( \langle j^{2} \rangle = 5 \)  جبکہ
\( \langle j\rangle ^{2} = 4 \)  ہے 


اشکال 1.5 کی صورتوں میں واضح فرق پایا جاتا ہے اگرچہ انکی اوسط قیمت, وسطانیہ اور زیادہ احتمال کی قیمت اور اجزاء کئ تعداد ایک کیسے ہے۔ ان میں پہلی شکل اوسط کے قریب نوکیلی صورت رکھتی ہے جبکہ دوسری چوڑی اور افقی صورت رکھتی ہے۔ مثال کے طور پر کسی بڑے شہر شہر میں ایک جماعت میں طلباء کئ تعداد پہلی شکل کئ مانند ہو گی۔ جبکہ دیہی علاقے میں ایک ہی کمرے پر مبنی سکول میں بچوں کی تعداد دوسری شکل ظاہر کرتی ہے۔ ہمیں کسی بھی تقسیم میں اوسط قیمت کے لحاظ سے مقدار کی پھیلاؤ عددی صورت میں درکار ہو گی۔ اس کا ایک سیدھا طریقہ یہ ہو سکتا ہے کہ ہم ہر انفرادی جزو کی قیمت اور اوسط قیمت کے درمیان فرق دیکھیں۔ 

\[ \Delta j = j-\langle j \rangle \]

اور پھر تمام \( \Delta j \) کی اوسط تلاش کریں۔ مسئلہ یہ پیش آتا ہے کہ خواب صفر ہو گا چونکہ اوسط کی تعریف کے تحت اوسط سے زیادہ اور اوسط سے کم قیمتیں ایک برابر ہوں گی۔ 

\begin{align*}
\langle \Delta j \rangle &= \sum \left( j -  \langle j \rangle  \right) P(j) = \sum jP(j) - \langle j \rangle \sum P(j) \\
 &= \langle j \rangle - \langle j \rangle = 0
\end{align*}

ایک جزو سے دوسرے جزو کی بات کرتے ہوئے \( \langle j \rangle \)   مستقل ہو گی۔ اس سے جان چھڑوانے کیلئے آپ  \( \Delta j \)   
 کی مطلق قیمت کا اوسط لے سکتے ہیں۔ لیکن آپ دیکھیں گے\( \Delta j \)    کی مطلق قیمتوں کا اوسط زیادہ مشکلات پیدا کرتا ہے۔ مطلق قیمتوں کے ساتھ کام کرنا پیچیدہ ثابت ہوتا ہے۔ اس کی بجائے ہم مربع لینے کے بعد اوسط حاصل کرتے ہیں۔ 
\[ \sigma^{2} \equiv \langle \left( \Delta j \right)^{2} \rangle \]


اس قیمت کو تقسیم کی تغیرت کہتے ہیں جبکہ \( \sigma \) جو کی تغیرت کا جزر ہے, کو معیاری انحراف کہتے ہیں۔ ہم \( \langle j \rangle \)  کے اطراف مقدار کے پھیلاؤ کو \( \sigma \)  سے بیان کرتے ہیں۔ تغیریت کا ایک چھوٹا مسئلہ ہم پیش کرتے ہیں۔ 

\begin{align*}
\sigma^{2} &= \langle ( \Delta j )^{2} \rangle = \sum ( \Delta j )^{2} P(j) = \sum (j- \langle j \rangle )^{2} P(j) \\
&= \sum (j^{2} -2j \langle j \rangle + \langle j \rangle ^{2} ) P(j) \\ 
&= \sum j^{2} P(j) -2 \langle j \rangle \sum jP(j) + \langle j \rangle ^{2} \sum P(j) \\ 
&= \langle j^{2} -2\langle j \rangle \langle j \rangle + \langle j \rangle ^{2} = \langle j^{2} \rangle - \langle j \rangle ^{2}
\end{align*}

اسکا جزر لے کر ہم معیاری انحراف کو درج ذیل لکھ سکتے ہیں۔ 
\[ \sigma = \sqrt{\langle j^{2} \rangle - \langle j \rangle ^{2}} \]


عملی استعمال میں \( \sigma \) کا حصول اس کلیے سے بہت جلد حاصل ہو گا۔ آپ \( \langle j^{2} \rangle \) اور \( \langle j \rangle ^{2} \) معلوم کر کہ انکا فرق لیں اور انکا جزر لیں۔ جیسا آپکو یاد ہوگا میں نے ذکر کیا \( \langle j^{2} \rangle \) اور \( \langle j \rangle ^{2} \) عموماً ایک دوسرے کے برابر نہیں ہوں گے۔ چونکہ \( \sigma ^{2} \) غیر منفی ہو گا جو آپ مساوات 1.11 سے دیکھ سکتے ہیں۔ لحظہ مساوات 1.12 کے تحت 

\[ \langle j^2 \rangle \geq \langle j \rangle ^{2} \]

اور یہ دونوں صرف اس صورت برابر ہو سکتے ہیں جب \( \sigma = 0 \),   جو تب ممکن ہو گا جب تقسیم میں کوئی پھیلاؤ نہ پایا جاتا ہو یعنی ہر جزو ایک ہی قیمت کا ہو۔ 

