
\ابتدا{سوال}
ایک بُعدی لامتناہی چکور کنواں کی زمینی حال میں کمیت \عددی{m} کا ایک ذرہ ابتدائی طور پر پایا جاتا ہے۔ لمحہ \(t=0\) پر ایک اینٹ اس کنواں میں گرائی جاتی ہے جس سے مخفیہ درج ذیل ہو جاتا ہے جہاں \(V_0<<E_1\) ہے۔
\begin{align*}
	V(x)=
	\begin{cases}
		V_0 & 0\leq x\leq a/2 \text{\RL{جب}} \\
		0 & a/2<x\leq a \text{\RL{جب}} \\
		\infty & \text{\RL{دیگرصورت}}
	\end{cases}
\end{align*}
کچھ وقت \عددی{T} کے بعد اینٹ ہٹائی جاتی ہے اور ذرہ کی توانائی ناپی جاتی ہے۔ رتبہ اوّل نظریہ اضطراب میں نتیجہ \عددی{E_2} ہونے کا احتمال کیا ہوگا؟
\انتہا{سوال}
\ابتدا{سوال}
ہم تحرقی اخراج، تحرقی انجزاب اور خود با خود اخراج دیکھ چکے ہیں۔ خود با خود انجزاب کیوں نہیں پایا جاتا ہے؟
\انتہا{سوال}
\ابتدا{سوال}
مقناطیسی گمک ساکن مقناطیسی میدان \(B_0\hat{k}\) میں \(1/2\) چکر کا ایک ذرہ جس کی مسکن مقناطیسی نسبت \(\gamma\) ہو لارمر تعدد \(\omega_0 = \gamma B_0\) مثال \num{4.3} سے استقبالی حرکت کرتا ہے۔ اب ہم ایک کمزور عارضی ریڈیائی تعدد میدان \(B_{rf}[\cos(\omega t)\hat{i}-\sin(\omega t)\hat{j}]\) چالو کرتے ہیں جس سے کل میدان درج ذیل ہو جاتا ہے۔
\begin{align}
	B = B_{rf}\cos(\omega t)\hat{i} - B_{rf}\sin(\omega t)\hat{j} + B_0\hat{k}
\end{align} 
(الف) اس نظام کے لیئے \(2\times2\) ہیملٹی قالب مساوات \num{4.158} تیار کریں۔

(ب) وقت \عددی{t} پر \(\chi(t) = \begin{pmatrix}a(t) \\b(t)\end{pmatrix}\) چکر حال ہونے کی صورت میں درج ذیل دیکھائیں۔
\begin{align}
	\dot{a} = \frac{i}{2}\big(\Omega e^{i\omega t}b+\omega_0 a\big):\quad\dot{b} = \frac{i}{2}\big(\Omega e^{i\omega t}a-\omega_0 b\big)
\end{align}
جہاں \(\Omega\equiv\gamma B_{rf}\) کا تعلق ریڈیائی تعدد میدان کی زور کے ساتھ پایا جاتا ہے۔

(ج) ابتدائی قیمتیں \عددی{a_0} اور \عددی{b_0} کی صورت میں \عددی{a(t)} اور \عددی{b(t)} کا عمومی حل تلاش کریں۔ جواب: 
\begin{align*}
	a(t) &= \bigg\{a_0\cos(\omega^\prime t/2)+\frac{i}{\omega^\prime}[a_0(\omega_0-\omega)+b_0\Omega]\sin(\omega^\prime t/2)\bigg\}e^{i\omega t/2} \\
	b(t) &= \bigg\{b_0\cos(\omega^\prime t/2)+\frac{i}{\omega^\prime}[b_0(\omega-\omega_0)+a_0\Omega]\sin(\omega^\prime t/2)\bigg\}e^{-i\omega t/2}
\end{align*}
جہاں درج ذیل ہوگا
\begin{align}
	\omega^\prime\equiv\sqrt{(\omega-\omega_0)^2+\Omega^2}
\end{align}
(د) ہواں میدان چکر حال یعنی \(a_0 = 1, b_0 = 0\) سے ایک ذرہ آغاز کرتا ہے۔ مخالف میدان چکر میں تحویل کی احتمال کو بطور وقت کا تفاعل تلش کریں۔

جواب: \(P(t) = \{\Omega^2/[(\omega-\omega_0)^2+\Omega^2]\}\sin^2(\omega^\prime t/2)\)

(و) منحنی گمک
\begin{align}
	P(\omega) = \frac{\Omega^2}{(\omega-\omega_0)^2+\Omega^2}
\end{align}
کو غیر متغیر \(\omega_0\) اور \(\Omega\) کیصورت میں متحرق تعدد \(\omega\) کی تفاعل کے طور پر ترسیم کریں۔ آپ دیکھیں گے کہ \(\omega = \omega_0\) پر اس کی زیادہ سے زیادہ قیمت پائی جاتی ہے۔ زیادہ سے زیادہ قیمت کی نصف پر پوری چوڑائی \(\Delta\omega\) تلاش کریں۔

(ھ) چونکہ \(\omega_0 = \gamma B_0\) ہے لحاظہ ہم تجرباتی طور گمک کا مشاہدہ کر کے ذرہ کی مقناطیسی جفت کتب معیارِ اثر تعین کر سکتے ہیں۔ ایک مرکزی مقناطیسی گمک تجربہ میں فوٹان کا \عددی{g} جزو ضربی ایک ٹیسلا کے ساکن میدان اور ایک مائکرو ٹیسلا حیطہ کے ریڈیائی تعدد میدان کی مدد سے ناپا جاتا ہے۔ تعدد گمک کیا ہوگا؟ پروٹان کی مقناطیسی معیارِ اثر کے لیئے حصۃ \num{6.5} دیکھیں۔ منحنی گمک کی چوڑائی تلاش کریں۔ اپنا جواب \(\si{\hertz}\) میں دیں۔
\انتہا{سوال}
\ابتدا{سوال}
میں نے مساوات \num{9.31} میں فرض کیا تھا کہ جوہر روشنی کی طولِ موج کے لحاظ سے اتنا چھوٹا ہے کہ میدان کی فضائی تغیر کو نظر انداز کیا جا سکتا ہے۔ حقیقی برقی میدان درج ذیل ہوگا
\begin{align}
	E(r,t) = E_0\cos(k.r-\omega t)
\end{align}
اگر جوہر کا مرکز مبدا پر ہو تب متعلقہ حجم پر \(k.r<<1\) \((\abs{k} = 2\pi/\lambda, \text{\RL{لحاظہ}}k.r\sim r/\lambda<<1)\) ہوگا جس کی بنا ہم اس جزو کو نظرانداز کر سکتے تھے۔ فرض کریں ہم رتبہ اوّل درستگی۔
\begin{align}
	E(r,t) = E_0[\cos(\omega t)+(k.r)\sin(\omega t)]
\end{align}
استعمال کریں۔ اس کا پہلا جزو وہ اجازتی برقی جفت کتب تحویلات پیدا کرتا ہے جن پر متن میں بات کی چکی ہے۔ دوسرا جزو وہ تحویلات پیدا کرتا ہے جنہیں ممنوعہ مقناطیسی جفت کتب اور برقی چو کتب تحویل کہتے ہیں \(k.r\) کی اس سے زیادہ بڑی طاقتیں مزید زیادہ ممنوعہ تحویلات پیدا کرتی ہے جو زیادہ بلند متعدد کتبی معیارِ اثر کے ساتھ وابستہ ہوں گے۔

(الف) ممنوعہ تحویلات کی خود با خود اخراجی شرح حاصل کریں اس کی تکتیب اور حرکت کے رخ پر اوسط قیمت تلاش کرنے کی ضرورت نہیں ہے اگر چہ مکمل جواب کے لیئے ایسا کرنا ضروری ہوگا۔ جواب:
\begin{align}
	R_{b\rightarrow a} = \frac{q^2\omega^5}{\pi\epsilon_0\hbar c^5}|\langle a|(\hat{n}.r)(\hat{k}.r)|b \rangle|^2
\end{align}
(ب) دیکھائیں کہ ایک بُعدی مرتعش کے لیئے ممنوعہ تحویلات سطح \عددی{n} سے سطح \(n-2\) میں ہوگی اور تحویلی شرح جس کی اوسط قیمت \(\hat{n}\) اور \(\hat{k}\) پر حاصل کی گئی ہو درج ذیل ہوگا۔
\begin{align}
	R = \frac{\hbar q^2\omega^3n(n-1)}{15\pi\epsilon_0m^2c^5}
\end{align}
تبصرہ: یہاں \(\omega\) سے مراد فوٹان کا تعدد ہے نا کہ مرتعش کا تعدد۔ اجازتی شرح کے لحاظ سے ممنوعہ شرح کا نصبط تلاش کریں۔ ان اصطلاح پر تبصرہ کریں۔

(ج) دیکھائیں کہ ہائڈروجن میں ممنوعہ تحویل بھی \(2S\rightarrow 1S\) کی اجازت نہیں دیتا۔ درحقیقت یہ تمام بلند متعدد کتب کے لیئے بھی درست ہوگا غالب تنزل دو فوٹان اخراج کی بنا ہوگا جس کا عرصہ حیات تقریباً ایک  سیکنڈ کا دسواں حصہ ہوگا۔
\انتہا{سوال}
\ابتدا{سوال}
دیکھائیں کہ \عددی{n,l} سے \(n^\prime, l^\prime\) میں تحویل کے لیئے ہائڈروجن کا خود با خود اخراجی شرح مساوات \num{9.56} درج ذیل ہوگا۔
\begin{align}
	\frac{e^2\omega^3I^2}{3\pi\epsilon_0\hbar c^3}\times
	\begin{cases}
		\frac{l+1}{2l+1}, & l^\prime = l+1 \text{\RL{جب}} \\
		\frac{l}{2l-1}, & l^\prime = l-1 \text{\RL{جب}}
	\end{cases}
\end{align}
جہاں \عددی{I} درج ذیل ہے۔
\begin{align}
	I\equiv\int_{0}^{\infty} r^3R_{nl}(r)R_{n^\prime l^\prime}(r)dr
\end{align}
جوہر \عددی{m} کی کسی مخصوص قیمت سے آغاز کر کے انتخابی قواعد \(m^\prime = m+1, m, \text{\RL{یا}} m-1\) کے تحت \(m^\prime\) حالات میں سے کسی ایک میں پہنچتا ہے۔ دیہان رہے کہ جواب \عددی{m} پر منحصر نہیں ہے۔ اشارہ: پہلے \(l^\prime = l+1\) صورت کے لیئے \(\mid nlm\rangle\) اور \(\mid n^\prime l^\prime m^\prime \rangle\) کے بیچ \عددی{x, y} اور \عددی{z} کے تمام غیر صفر قالبی ارکان معلوم کریں۔ ان سے درج ذیل مقدار تعین کریں
\begin{align*}
	|\langle n^\prime, l+1,m+1\abs{r}nlm \rangle|^2 + |\langle n^\prime, l+1, m\abs{r}nlm \rangle|^2+|\langle n^\prime, l+1, m-1\abs{r}nlm\rangle|^2
\end{align*}
یہی کچھ \(l^\prime = l-1\) کے لیئے بھی کریں۔
\انتہا{سوال}

