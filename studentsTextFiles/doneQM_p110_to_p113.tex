%%%%%%%%%%%%%%%%%%%%%%%%%%%%%%%%%%%%%%%%%%%%%%
% 			Page 110-111

\حصہ{عدم یقینیت کا اصول}
میں نے عدم یقینیت کے اصول کو \عددی{  \sigma_{x} \sigma_{p} \geq \hbar /2 } کی صورت میں حصہ \حوالہ{1.6} میں  بیان کیا  جس کو آپ کئی سوالات حل کرتے ہوئے دیکھ چکے ہیں. تاہم اس کا ثبوت ہم نے ابھی پیش نہیں کیا. اس حصہ میں ہم عدم یقینیت کے اصول کی عمومی صورت پیش کریں گے اور اس .کے چند مضمرات جانیں گے. ثبوت کا دلیل خوبصورت ضرور ہے لیکن ساتھ ہی پیچیدہ بھی ہے لہذا غور سے دیکھیے گا
\جزوحصہ{عدم یقینیت کے اصول کا ثبوت}
کسی بھی قابل مشاہدہ \عددی{  A } کے لیے درج ذیل ہو گا مساوات \حوالہ{ 3.21}
\begin{align*}
\sigma_{A}^{2} = \langle (\hat{A} - \langle A \rangle ) \Psi | (\hat{A} - \langle A \rangle ) \Psi \rangle = \langle f|f \rangle
\end{align*}
جہاں \عددی{f \equiv (\hat{A} - \langle A \rangle ) \Psi } ہے اسی طرح کسی دوسرے قابل مشاہدہ \عددی{ B } کے لیے
\begin{align*}
\sigma_{B}^{2} = \langle g|g \rangle , \quad \text{where}\, g \equiv ( \hat{B} - \langle B \rangle ) \Psi
\end{align*}
یوں شوارز عدم مساوات مساوات \حوالہ{ 3.7}  کے تحت درج ذیل ہو گا
\begin{align}
\sigma_{A}^{2} \sigma_{B}^{2} = \langle f|f \rangle \langle g|g \rangle \geq | \langle f|g \rangle |^{2}
\end{align}
اب کسی بھی مخلوط عدد \عددی{ z } کے لیے درج ذیل ہو گا 
\begin{align}
|z|^{2} = [Re(z)]^{2} + [Im(z)]^{2} \geq [Im(z)]^{2} = \left[\frac{1}{2i} (z - z^{*} \right]^{2}
\end{align}
یوں \عددی{ z = \langle f|g \rangle } لیتے ہوئے درج ذیل ہو گا
\begin{align}
\sigma_{A}^{2} \sigma_{B}^{2} \geq \left(\frac{1}{2i} [ \langle f|g \rangle - \langle g|f \rangle ] \right)^{2}
\end{align}
لیکن \عددی{\langle f|g \rangle } کو درج ذیل لکھا جا سکتا ہے
\begin{align*}
\langle f|g \rangle &= \langle (\hat{A} - \langle A \rangle )\Psi \hat{B} - \langle B \rangle ) \Psi \rangle = \langle \Psi | ( \hat{A} - \langle A \rangle )(\hat{B}-\langle B \rangle )\Psi \rangle \\
&= \langle \Psi |(\hat{A} \hat{B} - \hat{A} \langle B \rangle -\hat{B} \langle A \rangle + \langle A \rangle \langle B \rangle ) \Psi ) \\
&= \langle \Psi |(\hat{A} \hat{B} \Psi \rangle - \langle B \rangle \langle \Psi | \hat{A} \Psi \rangle - \langle A \rangle \langle \Psi |\hat{B} \Psi \rangle + \langle A \rangle \langle B \rangle  \langle \Psi |\Psi \rangle \\
&= \langle \hat{A}\hat{B} \rangle - \langle B \rangle \langle A \rangle - \langle A \rangle \langle B \rangle  +
\langle A \rangle \langle B \rangle \\
&= \langle \hat{A} \hat{B} \rangle - \langle A \rangle \langle B \rangle .
\end{align*}
اسی طرح درج ذیل بھی لکھا جا سکتا ہے 
\begin{align*}
\langle g|f \rangle = \langle \hat{B}\hat{A} \rangle - \langle A \rangle \langle B \rangle ,
\end{align*}
لہذا \begin{align*}
\langle f|g \rangle - \langle g|f \rangle = \langle \hat{A}\hat{B} \rangle -\langle \hat{B}\hat{A} \rangle = \langle [ \hat{A} , \hat{B}] \rangle ,
\end{align*}
ہو گا
جہاں
\begin{align*}
[\hat{A} , \hat{B}] \equiv \hat{A} \hat{B} - \hat{B}\hat{A}
\end{align*}
ہے
ان دو حاملین کا تبادل کار ہے مساوات \حوالہ{ 2.48} . نتیجتاً درج ذیل ہو گا 
\begin{align}
\sigma_{A}^{2} \sigma_{B}^{2} \geq \left(\frac{1}{2i} \langle [\hat{A} , \hat{B} \rangle \right)^{2}
\end{align}
یہ عدم یقینیت کے اصول کی عمومی صورت ہے. آپ یہاں سوچ سکتے ہیں کہ مساوات کا دایاں ہاتھ منفی ہے؟ یقیناً ایسا نہیں ہے دو ہرمیشی حاملین کے تبادلکار میں \عددی{ i } کا جزر بھی پایا جاتا ہے جو مساوات میں موجود \عددی{ i } کے ساتھ کٹ جاتا ہے
مثال کے طور پر ہم پہلا قابل مشاہدہ مقام \عددی{ (\hat{A} = x) } لیتے ہیں اور دوسرا معیار حرکت \عددی{(\hat{B} = (\hbar/i)d/dx)  } ہم باب دو کی مساوات \حوالہ{ 2.51}  میں اس کا تبادل کار حاصل کر چکے ہیں.
\begin{align*}
[ \hat{x} . \hat{p} ] = i \hbar
\end{align*}
لہذا
\begin{align*}
\sigma_{x}^{2} \sigma_{p}^{2} \geq \left( \frac{1}{2i} i \hbar \right)^{2} = \left( \frac{\hbar}{2} \right)^{2}
\end{align*}
یا چونکہ تعریف کی رو سے معیاری انخراف مثبت ہوتے ہیں لہذا درج ذیل ہو گا
\begin{align}
\sigma_{x} \sigma_{p} \geq \frac{h}{2}
\end{align}
یہ ہیسن برگ کی اصل عدم یقینیت کا اصول ہے جو ہم دیکھتے ہیں کہ زیادہ عمومی مسئلے کی ایک مخصوص صورت ہے
حقیقت میں ایسے ہر دو حاملین جو قابل تبادل نہ ہوں کی جوڑی کے لیے عدم یقینیت کا اصول پایا جاتا ہے ہم انہیں غیر ہم آہنگ قابل مشاہدہ کہتے ہیں. غیر ہم آہنگ قابل مشاہدہ کے مشترکہ امتیازی تفاعل نہیں پائے جاتے کم از کم ان کے مشترکہ امتیازی تفاعلوں کا مکمل 
سلسلہ نہیں ہو گا. ( سوال \حوالہ{ 3.15}  دیکھیں ). اس کے برعکس ہم آہنگ قابل تبادل مشاہدہ مشترکہ امتیازی تفاعلوں کا مکمل سلسلہ ممکن ہے
%%%%%%%%%%%%%%%%%%%%%%%%%%%%%%%%
% 		Page 112
مثال کے طور پر جیسا ہم باب \حوالہ{ 4} میں دیکھیں گے.  ہائیڈروجن جوہر جی ہیملٹونی اس کی زاویائی معیار حرکت کی مقدار اور زاویائی معیار حرکت کا \عددی{z   } جز باہمی ہم آہنگ قابل مشاہدہ ہیں. اور ہم ان تینوں کے بیک وقت امتیازی تفاعل تیار لر لے انہیں متعلقہ امتیازی اقدار کے لحاظ سے نام دیں گے. اس کے برعکس مقام کا ایسے کوئی امتیازی تفاعل نہیں پایا جاتا جو معیار حرکت کا بھی امتیازی تفاعل ہو. چونکہ یہ دو عملین غیر ہم آہنگ ہیں. یاد رہے کہ عدم یقینیت کا اصول کوانٹم نظریہ میں ایک اضافی مفروضہ نہیں ہے بلکہ یہ شماریاتی مفہوم کا ایک نتیجہ ہے. آپ تعجب سے پوچھ سکتے ہیں کہ تجربہ گاہ میں ہم ایک ذرے کا مقام اور معیار حرکت کیوں تعین نہیں کر سکتے ہیں. آپ ایک ذرے کا مقام ضرور ناپ سکتے ہیں لیکن پیمائش سے تفاعل موج ایک نقطے پر نوکیلی صورت اختیار کرتے ہوئے گرتی ہے. اور آپ فوریئر نظریہ سے جانتے ہیں کہ طول موج کی وسیع ساعت نوکیلی تفاعل موج پیدا کرتی ہے. لہذا اس کی معیار حرکت کی وسعت بھی زیادہ ہو گی. اب اگر آپ معیار حرکت کی پیمائش کریں تو تفاعل موج ایک لمبی \عددی{   \sin} نما موج پر گرے گی. جسکا اب طول موج بہت واضح ہو گا لیکن ذرے کا مقام پہلی پیمائش سے مختلف ہو گا. یہاں مسئلہ یہ ہے کہ دوسری پیمائش پہلی پیمائش کے نتیجہ غیر مشتمل کرتا ہے. صرف اس صورت جب تفاعل موج بیک وقت دونوں قابل مشاہدہ کے امتیازی حال ہوں, یہ ممکن ہو گا کہ دوسری پیمائش ذرہ کی حال پر اثر انداز نا ہو. ایسی صورت میں موج کا دوبارہ گرنے سے کوئی چیز تبدیل نہیں ہو گی. لیکن ایسا عموماً تب ممکن ہو گا کہ دونوں قابل مشاہدہ ہم آہنگ ہوں.
\ابتدا{سوال}
\begin{enumerate}[a.]
\item درج ذیل تبادل کار مماثل ثابت کریں.
\begin{align}
[ AB,C ] = A[B,C] + [A,C]B
\end{align}
\item درج ذیل دکھائیں. 
\begin{align*}
[x^{n},p] = i\hbar nx^{n-1}
\end{align*}
\item دکھائیں کہ عمومی طور پر درج ذیل ہو گا. 
\begin{align}
[f(x) , p] = i\hbar \frac{df}{dx}
\end{align}
جہاں \عددی{   f(x) }  کوئی بھی طفاعل ہو سکتا ہے.
\end{enumerate}
\انتہا{سوال}
\ابتدا{سوال}
مقام \عددی{   A=x} میں عدم یقینیت اور توانائی  \عددی{ B = p^{2}/2m + V } میں عدم یقینیت کا درج ذیل اصول ثابت کریں.
\begin{align*}
\sigma_{x}\sigma{H} \geq \frac{\hbar}{2m}|\langle p \rangle |.
\end{align*}
ساکن حالات کیلئے یہ آپ کو کوئی زیادہ معلومات فراہم نہیں کرتا, ایسا کیوں ہے؟
\انتہا{سوال}
\ابتدا{سوال}
دکھائیں کہ دو ناقابل تبادل عملین کے مشترکہ امتازی طفاعلوں کا مکمل سلسلہ نہیں پایا جائے گا.
اشارہ : دکھائیں اگر \عددی{  \hat{P} } اور \عددی{  \hat{Q} } کے مشترکہ امتیازی طفاعلوں کا مکمل سلسلہ پایا جاتا ہو, تب ہلبرٹ فضاء میں کسی بھی طفاعل کیلئے \عددی{ [ \hat{P}, \hat{Q}]f = 0  } ہو گا.
\انتہا{سوال}
\جزوحصہ{کم سے کم عدم یقینیت کا موجی اکٹھ}
ہم ہارمونی مرتعش کی زمینی حال سوال \حوالہ{  2.11  } اور آزاد ذرے کی گوثی موجی اکٹھ سوال \حوالہ{   2.22 } کے تفاعل موج دیکھ چکے ہیں جو مقام اور معیار حرکت کی عدم یقینیت کی حد \عددی{ \sigma_{x}\sigma_{p} = \hbar /2 } کو چھوتے ہیں. اس سے ایک دلچسپ سوال پیدا ہوتا ہے. کم سے کم عدم یقینیت کا سب سے زیادہ عمومی موجی اکٹھ کیا ہو گا؟ عدم یقینیت کے اصول کے ثبوت کے دلائل میں عدم مساوات دو نقتوں پر پیش کیا گیا. مساوات \حوالہ{ 3.59   } اور مساوات \حوالہ{ 3.60   }. ہم دونوں جگہ مساوات لیتے ہوئے دیکھتے ہیں کہ \عددی{  \Psi  }  کے بارے میں کیا معلومات فراہم ہوتی ہے. جب ایک طفاعل دوسرے طفاعل جا مضرب ہو \عددی{  g(x)=cf(x)  }  جہاں \عددی{   c } مخلوط عدد ہے تب شوارڈز عدم مساوات ایک مساوات بن جاتی ہے. سوال \حوالہ{ A5   } دیکھیں. ساتھ ہی میں مساوات \حوالہ{   3.60 } میں \عددی{ z   } کے حقیقی جز کو رد کرتا ہوں. یوں \عددی{ Re(z) = 0 } جسکا مطلب ہے  \عددی{ Re\langle f|g \rangle = Re (c\langle f|f \rangle ) = 0 } کی صورت میں مساوات ہو گا. اب \عددی{ \langle f|f \rangle } یقینا حقیقی ہے. لحظہ مستقل \عددی{  c  } لازماً خلص خیالی ہو گا. جسے ہم \عددی{  ia  } لکھتے ہیں. یوں کم سے کم عدم عدم یقینیت کیلئے کافی شرط درج ذیل ہو گا. 
\begin{align}
g(x) = iaf(x), \quad \text{where a is real}
\end{align}
مقام و معیار حرکت عدم یقینیت کے اصول کیلئے یہ شرط درج ذیل صورت اختیار کرتا ہے. 
\begin{align}
\left( \frac{\hbar}{i} \frac{d}{dx} - \langle p \rangle \right) \Psi = ia(x-\langle x \rangle ) \Psi 
\end{align}
جو متغیر \عددی{ x   } کے تفاعل \عددی{    \Psi } کا تفرقی مساوات ہے. اس کا عمومی حل درج ذیل ہے. سوال \حوالہ{ 3.16   } 
\begin{align}
\Psi(x) = Ae^{-a(x-\langle x \rangle )^{2}/2\hbar}e^{t\langle p \rangle x / \hbar}.
\end{align}
آپ دیکھ سکتے ہیں کہ کم سے کم عدم یقینیت کا موجی اکٹھ درحقیقت گوثی ہو گا. اور جو دو مثالیں ہم دے چکے وہ بھی گوثی تھے. 
\ابتدا{سوال}
\حوالہ{ 3.67} کو \عددی{ \Psi(x) }کیلئے حل کریں. دھیان رہے کہ \عددی{   \langle x \rangle  } اور \عددی{  \langle p \rangle  } مستقل ہیں.
\انتہا{سوال}
 
