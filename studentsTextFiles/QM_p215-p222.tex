
تاہم درج ذیل وجہ کی بنا الیکٹران کی توانائی دفا
 \(l\) 
کی کم سے کم قیمت کی طرف داری کرتی ہے۔ زاویائی معیارے حرکت الیکٹران کو بےرونی روح دھکیلنے کی کوشش کرتا ہے اور الیکٹران جتنا مرکزا سے دور ہوتا ہے اتنا ہی یہ مرکزا بہتر چھپاتا ہے۔ ہم کہہ سکتے ہیں کہ اندرونی الیکٹران کو مرکزا کا پورا 
\(Ze\)
نظر آتا ہے جب کہ بےرونی الیکٹران کو مشکل سے 
\(e\)
 سے زیادہ موثر  نظر آتا ہے ۔یوں کسی بھی ایک ہول میں کم سے کم توانائی کا حال یعنی دوسرے لفظوں میں سب سے سخت مقید الیکٹران
 \( l=0\) 
 ہوگا ۔ اور بڑھتے
  \(l\) 
 کے ساتھ توانائی بڑھے گی اس طرح لتیم میں تیسرا الیکٹران مدارجہ
  \((2,0,0)\)
 کا مقید ہوگا۔ اگلا جوہر بیریلیم جس کا
  \(Z=4\) 
 ہے اسی حال میں ہوگا لیکن اس کا چکر مخالف رخ ہوگا لیکن بوران
  \(Z=5\) 
 کو
 \( l=1\) 
 استعمال کرنا ہوگا۔ اسی طرح چلتے ہوئے ہم  نیین
 \( Z=10\) 
 تک پہنچتے ہیں جہاں
  \(n=2\) 
 ہول مکمل بھرا ہوگا اور ہم دوری جدول کی اگلی صف کو پہنچ کر
  \(n= 3\) 
 ہول کو بھرنا شروع کرتے ہیں ۔ آغاز میں دو جوہر سوڈیم اور میگنیشیم ہیں جنکا
 \( l=0\) 
 ہے اور اس کے بعد الیمینیم سے آرگان تک چھ ایسے جوہر ہیں جن کے لیے 
 \(L=1\)
  ہوگا۔ آرگان کے بعد ہم توقع کرتے ہیں کہ دس ایسے جوہر پائے جائنگے جن کے لیے 
  \(n=3\) 
  اور 
  \(l=2\) 
  ہوگا البتہ یہاں پہنچ کر اندرونی الیکٹران مرکزا کو اتنی خوش اسلوبی کے ساتھ پردہ کرتے ہیں کہ یہ اگلے ہول کو بھی ڈنگتا ہے لہذا پوٹیشیم
\(( Z=19)\) 
  اور کیلشیم
  \((Z=20)\)
  ،
  \((n=3),(l=2)\)
 کی بجائے
 \( (n=4),( L=0 )\)
 منتخب کرتے ہیں  ۔ اس کے بعد ہم نیچے اتر کر سکینڈیم سے زنک تک کے جوہر اٹھاتے ہیں جن کے لیے
 \( n=3\)
  اور
\( l=2\)
   ہوگا ۔ اس کے بعد گیلیم سے کرپٹان تک
\( n=4، l=1\)
    ہوگا جس کے آخر میں ہم دوبارہ قبل از وقت اگلی صف
\( n=5\) 
    کو چھلانگ لگاتے ہیں اور بعد میں واپس اتر کر
\( n= 4\)  
    ہول کے ۔ وہ مدارجے جن کے لیے
\( l=2، l=3\) 
    ہوں پر کرتے ہیں ۔ یہاں جوہری حالات کی قدیم نام جنہیں تمام ماہر کیمیات اور تبیات کے زیادہ تر ماہرین استعمال کرتے ہیں پر تبصرہ کرنا ضروری ہوگا اس کی وجہ شاید صرف انیسویں صدی کے تیز پیمائی کاروں کو معلوم ہوگا کہ
 \(l=0\) 
    کو
  \(s\)
      کہتے ہیں 
 \(l=1\)
       کو 
 \(p\)
        کہتے ہیں،
\( l=2\) 
        کو
   \(d\)
          کہتے ہیں اور
   \(l=3\)
            کو
\( f\)
  کہتے ہیں۔ میرے خیال سے اس کے بعد وہ سیدھی راس پر آ گئے اور انہوں نے عروف تہجی کے تحت
\( (g,h,i,,k,l)\)
   وغیرہ نام دینا شروع کیا۔ انہوں نے ہماری ناک میں دم کرنے کی خاطر 
  \(j\)
    کو نظر انداز کیا ۔ کسی ایک الیکٹران کے حال کو
    \( (n,l)\) 
    کی جوڑی ظاہر کرتی ہے جہاں عدد n حال کو اور حرف
    \(l\)
      مدارجی زاویائی معار حرکت کو ظاہر کرتا ہے ۔ کوانٹم عدد
   \(m\) 
      کا زکر نہیں کیا جاتا لیکن قوت نما میں حال کے مقین الیکٹرانوں کی تعداد لکھی جاتی ہے ۔ یوں درج ذیل تنظیم 
\[(1s)^{2}(2s)^{2}(2p)^{2} \]
کہتی ہے کہ مدارجہ
\((1,0,0)\)
 میں 2 الیکٹران، 
مدارجہ
\(( 2,0,0)\) 
میں 2 جبکہ مدارجے
\(( 2,1,1)\)
، 
\((2,1,0)\) 
اور
 \((2,1,-1)\) 
کے کسی ملاپ میں 2 الیکٹران پائے جاتے ہیں ۔ یہ درحقیقت کاربن کا زمینی حل ہے ۔ \\
اس مثال میں 2 الیکٹران ایسے پائے جاتے ہیں جن کے مدارجی زاویائی معارے ہرکت کوانٹم  عدد ایک ہے لہذا مدارچی زاوییائ معیار حرکت کوانٹم عدد ایک ہے لہذا کل مدارچی زاوییائ معیار حرکت کوانٹم نمبر 
\(l\)
کسی ایک ذرہ کی جبک
\(L\)
کل قیمت کو ظاہر کرتا ہے ۔ ایک، دو یا صفر ہو  سکتا ہے۔ جبکہ
 \((1s)\) 
کے دو الیکڑان ایک دوسرے کے ساتھ یکتا حال میں بندھے ہیں اور ان کا کل چکر صفر ہوگا۔ یہی کچھ
\(( 2S)\)
 کے دو الیکڑانوں کے لئے بھی ہوگا لیکن
 \((2p)\)
 کے دو الیکڑان یا تو یکتا نظام اور یا سہتا نظام میں ہوں گے۔ یوں کل چکر کوانٹم عدد S کل کو ظاہر کرنے کے لئے بڑا حرف استعمال ہوگا۔ جس کی قیمت ایک یا صفر ہو سکتی ہے۔ ظاہر ہے میزان کل مدارچی جمع چکر J کی قیمت تین، دو، ایک یا صفر ہو سکتی ہے۔ کسی ایک جوہر کے لئے ان کل قیمتوں کو ہن قواعد ( سوال 5.1 دیکھیں)  سے حاصل کیا جا سکتا ہے ۔نتیجہ کو درجہ ذیل روپ میں لکھا جا سکتا ہے. 
 \[^{2s+1}L_{J}\]
  جہاں J اور S اعداد جبکہ L ایک حرف ہو گا اور چونکہ ہم کل کی بات کر رہے ہیں لہذا یہ بڑا حرف ہو گا کاربن کا زمینی حالِ 3D ہے جس کا کل چکّر ایک ہے جس کی بنا 3 لکھا گیا ہے کل مدارچی زاویای معیار حرکت ایک ہے لہذا
  \(1p\) 
  لکھا گیا ہے اورمیزان  کل زاوییائ معیار حرکت صفر ہے لہذا صفر لکھا گیا ہے۔ جدول  5.1 میں دوری جدول کے ابتدائی چار صفحوں کے لئے انفرادی  تنظیم اور کل زاوییائ معیار حرکت مساوات 5.34 کی روپ میں پیش کئے گئے ہیں ۔ \\
  سوال 5.12 \\
  جز الف: دوری جدول کے ابتدائی  دو صفحوں  کے لئے نییوون تک مساوات 5.33 کی روپ میں تنظیم الیکڑان پیش کر کے ان کی تصدیق جدول 5.1 کے ساتھ کریں ۔ \\
  جز ب :ابتدائی  چار عناصر  کے لئے مساوات  5.34 کی روپ میں ان کا مطابقتی کل زاوییائ معیار حرکت تلاش کریں ۔بوران، کاربن اور نایڑوجن کے لئے تمام ممکنات پیش کریں۔\\
   سوال 5.13\\ 
   جز الف: ہن کا پہلا قاعدہ کہتا ہے کہ باقی چیزیں ایک جیسا ہونے کے لیے صورت میں وہ حال جس کا کل چکری زیادہ سے زیادہ ہوگی کم سے کم توانائی ہو گی۔ ہیلیم کے ھجان حالات کے لیے یہ کیا پیشنگوئی کرتا ہے۔\\
    جز ب: ہن کا دوسرا قاعدہ کہتا ہے کہ کسی ایک چکر کی صورت میں مجموعی طور پر خلاف تشاکلییت پر پورا اترتا ہو۔ وہ حال جس کی مدارچی زاوییائ معیار حرکتLl زیادہ سے زیادہ ہو گی توانائی کم سے کم ہو گی ۔ کاربن کے لئے 2=L کیوں نہیں ہوگا؟ اشارہ سیڑھی کا بالائی سر
    \((M_{L}=L)\)
      تشاکلی ہے۔\\
      جز ج: ہن کا تیسرا قاعدہ کہتا ہے کہ اگر ایک ذیلی خول 
   \((n,l)\)
       نصف سے زیادہ بھرا نا ہو تب کم سے کم توانائی کی سطح کے لئے 
    \(J=\abs{L-S}\)
        ہو گا۔ اگر یہ نصف سے زیادہ بھرا ہو تب
\( J=L+S\)
 کی توانائی کم سے کم ہوگی۔ اس حقیقت کو استعمال کرتے ہوئے سوال 5.12 ب میں بوران کے مسائلہ سے شک دور کرے۔\\ 
جز د: قواعد ہن کے ساتھ یہ حقیقت استعمال کرتے ہوئے کہ تشاکلی چکری حال کے ساتھ خلاف تشاکلی موزہ حال کے ساتھ خلاف تشاکل چکر حال استعمال ہوگا ۔سوال 5.12 ب میں کاربن اور نایڑوجن میں درپیش مشکلات سے چھٹکارا حاصل کریں ۔ اشارہ کسی بھی حال کی تشاکلی جاننے کی خاطر سیڑھی کے بالائی سر سے آغاز کریں ۔\\
 سوال 5.14\\
  دوری جدول کے چھٹے صف میں عنصر چار ساٹھ ڈسپروسییم کا زمینی حال
  \(^{5}I_{8}\) 
  ہے۔ اس کے کل چکر کل مدارچے اور میزان کل زاوییائ معیار حرکت کوانٹم کل حالات کیا ہوں گے۔ ڈسپروسییم کے الیکڑان  کی تنظیم کا خاکہ کیا ہو سکتا ہے۔\\
حصہ
 5.3\\
 
ٹھوس حال میں ہر جوہر کے بیرونی ڈیلے مقید گرفتی الیکٹرانوں میں سے چند ایک علیحدہ ہو کر کسی مخصوص موروثی مرکزا کے کولوم میدان سے آزاد، تمام قلمی جال کے مخفیا کے زیرِ اثر حرکت کرنا شروع کرتے ہے اس حصہ میں ہم تو بہت سادے نمونوں پے غور کرے گے۔ پہلا نمونہ الیکٹرون گیس نظریہ ہے جو سمرفيل نے پیش کیا اس نمونے میں سرحد کے اثرات کے علاوہ باقی تمام قوتوں کو نظرانداز کیا جاتا ہے اور الیکٹرانوں کو لامتناہی چاکور کنواں کے تین آبادی مماثل کی طرح ڈبے میں آزاد ذرات تصویر کیا جاتا ہے۔ دوسرا نمونہ بلخ نظریہ کہلایا جاتا ہے الیکٹرون کی بہمی دفاع کو نظر انداز کرتے ہوئے باقاعدگی سے ایک جیتنے فاصلے پر مثبت بار کے مرکزہ کو دوری مخفیہ سے ظاہر کرتا ہے، یہ نمونے ٹھوس اجسام کی کوانٹم نظریے کی طرف پہلے لڑکھڑاتے قدم ہیں۔ اس کے باوجود یہ پولی حصولمنات کا جموت میں گہرا کردار اور موصال، غیر موصل اور نیم موصل کی حیرت کن برقی خواص پر روشنی ڈالنے میں مدد دیتی ہے۔\\
جز حصہ
 5.3.1\\
 
آزاد الیکٹرون گیس\\
، فرض کرے ایک ٹھوس جسم مستطیل چکل کا ہے جس کے اصلا
 \(l_{x}\)
 ،
 \(l_{y}\)
  اور
 \(l_{z}\)
    ہے  اور فرض کرے کے اِس کے اندر الیکٹرون پر کوئی قوت اثر انداز نہیں ہوسکی ما سوائے ناقابلِ گزر دیواروں کے۔
\begin{equation}
V(x,y,z)=
\begin{cases}
0 & 0<x<l_{x}, \quad 0<y<l_{y}, \quad 0<z<l_{z}\\
\infty & otherwise\\
\end{cases}
\end{equation}
شرودنگر مساوات
\[\frac{-\hbar^{2}}{2m}\nabla^{2}\psi=E\psi\]
\[\psi(x,y,z)=X(x)Y(y)Z(z)\]
\[\frac{-\hbar^{2}}{2m}\frac{\dif^{2}X}{\dif{x}^{2}}=E_{x}X ; \frac{-\hbar^{2}}{2m}\frac{\dif^{2}Y}{\dif{y}^{2}}=E_{y}Y ; \frac{-\hbar^{2}}{2m}\frac{\dif^{2}Z}{\dif{z}^{2}}=E_{z}Z\]
اور
\[E=E_{x}+E_{y}+E_{z}\]
درج ذیل لیتے ہوئے،
\[k_{x}\equiv \frac{\sqrt{2mE_{x}}}{\hbar}, k_{y}\equiv\frac{\sqrt{2mE_{y}}}{\hbar}, k_{z}\equiv \frac{\sqrt{2mE_{z}}}{\hbar}\]
 ہم عمومی حل حاصل کرتے ہے۔
\begin{equation}
X(x)=A_{x}\sin{(K_{x}x)}+B_{x}\cos{(K_{x}x)} \quad Y(y)=A_{y}\sin{(K_{y}y)}+B_{y}\cos{(K_{y}y)}\\
Z(z)=A_{z}\sin{(K_{z}z)}+B_{z}\cos{(K_{z}z)}\\
\end{equation}
سرحدی شرائط کے تحد 
\[X(0)=Y(0)=Z(0), B_{x}=B_{y}=B_{z}=0, X(l_{x})=Y(l_{y})=Z(l_{z})=0\]
ہوگا۔ لہٰذا درج ذیل ہوگا۔
\[k_{x}l_{x}=n_{x}\pi, k_{y}l_{y}=n_{y}\pi, k_{z}l_{z}=n_{z}\pi\]
جہاں ہر n ایک مثبت عدد صحیح ہے۔
\[n_{x}=1,2,3,\dotsc \quad n_{y}=1,2,3,\dotsc \quad n_{z}=1,2,3,\dotsc\]
معمول شدہ تفلات مٌوج درج ذیل ہونگے۔
\[\psi_{n_{x}n_{y}n_{z}}=\sqrt{\frac{8}{l_{x}l_{y}l_{z}}}\sin{\big(\frac{n_{x}\pi}{l_{x}} x\big)}\sin{\big(\frac{n_{y}\pi}{l_{y}} y\big)}\sin{\big(\frac{n_{z}\pi}{l_{z}} z\big)}\]
اور اجازاتی توانائياں درج ذیل ہونگی۔
\[E_{n_{x}n_{y}n_{z}}=\frac{\hbar^{2}\pi^{2}}{2m}\big(\frac{n_{x}^{2}}{l_{x}^{2}}+\frac{n_{y}^{2}}{l_{y}^{2}}+\frac{n_{z}^{2}}{l_{z}^{2}}\big )=\frac{\hbar^{2}k^{2}}{2m}\]
جہاں سمتیاں موج، 
\(k\equiv (k_{x},k_{y},k_{z})\)
 کی مطلق قیمت K ہوگی ۔
اگر آپ ایک تین آبادی فضا کا تصویر کرے جس کے محور
\(k_{x}، k_{y} ،k_{z}\)
 ہو اور جس پر
\(k_{x}=(\pi/l_{x})(2\pi/l_{x})(3\pi/l_{x})\dotsc\)
اور
\(k_{y}=(\pi/l_{y})(2\pi/l_{y})(3\pi/l_{y})\dotsc\)
اور
\(k_{z}=(\pi/l_{z})(2\pi/l_{z})(3\pi/l_{z})\dotsc\)
پر سیدھے سطوت پاے جاتے ہو تب ہر انفرادی نقطہ تکاتے ایک منفرد یک ذرا ساکن حال دیگا۔\\
اس جال میں ہر ایک خانہ لہٰذا ہر ایک حال کی فضا میں درج ذیل حجم گہیرے گا، جہاں  پورے جسم کا حجم ہے۔
\[\frac{\pi^{3}}{l_{x}l_{y}l_{z}}=\frac{\pi^{3}}{V}\]


