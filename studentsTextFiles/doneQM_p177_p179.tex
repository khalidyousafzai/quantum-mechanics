
ایک عام آدمی، ایک فلسفی یا ایک کلاسیکی مایرِ تبیات کا یہ کینا کہ کس زرے  کا ٹھیک ٹھیک مکام یا میعارِ حرکت یا چکری زاویائی میارِ حرکت کا
x
جز یا وغیرہ نہیں پایا جاتا،  ایک گول مول جواب ہے۔ جو آپ کی نااہلی کے سوا کچھ نزر نہیں آتا۔ حقیقت میں ایسا کچھ بھی نہیں ہے لیکن اس کے اصل معنی کسی ایسے شخص کو سمجھانا جس نے کونٹم مکینیانیات کا گہرا مطلع نہ کیا ہو تقریبن ناممکن ہے۔ اگر آپ کی عقل دنگ رہ گئی ہے اور اگر آپ کی عقل دنگ نہیں دہی تو اس کا مطلب ہوگا کہ آپ کو کوئی بات سمجھ ہی نہیں آئی یو  \عددی{1/2}
چکر نظام پر دوبارا غور کی جئے گا۔ یہ  ونٹم مکینیانیات کی پیچیدہ تفرقات سمجھنے کی سادہ ترین مثال ہے۔

سوال 4.26
(الف) تصدیق کی جئے گا کہ چکری کا لپ مساوات 4.145 اور 4.147 زاویائی میارِ حرکت کے بنیادی تبادلی رشتوں کو مطمین کرتے ہیں۔ 

(ب) دیکھائیں کہ پولی چکری کا لپ  مثال  4.148 درج ذیل زروی قائدہ کو مطمین کرتی ہے۔ 
\begin{align}
\sigma_j\sigma_k = \delta_{jk}+i\sum_l \epsilon_{jkl}\sigma_l
\end{align}
جہاں اشاریا x,y اور z کو ظاہر کرتے ہیں۔ جبکہ  \عددی{\epsilon_{jkl}}
Levi-Civita
 علامت ہے۔ جو \عددی{jkl = 1,2,3}
یا 
\عددی{2,3,1}
یا  
\عددی{3,1,2}
کی سورت میں 
\عددی{+1}
جبکہ  
\عددی{jkl = 1,3,2}
یا
\عددی{2,1,3}
یا
\عددی{3,2,1}
کی سورت میں 
\عددی{-1}
جبکہ باسورت دیگر
0
ہوگا۔ 

سوال 
4.27
ایک الیکٹروں درج ذیل جکری حال میں ہے۔
$\psi =A\begin{bmatrix}
3i \\ 4
\end{bmatrix} $
(الف)مامولزنی مستقل A تاین کریں۔
 (ب) \عددی{S_x,S_y, S_z} کی تقواتی قیمتیں تلاش کریں۔
 (ج) عدم یقینیت  \عددی{\sigma_{S_x}, \sigma_{S_y}}   اور
  \عددی{\sigma_{S_z}} تلاش کریں۔ دیحان رہے کہ یہاں \عددی{\sigma} سے مراد میارِ انہراف ہے۔ پولی کالپ 
  (د)تصدیق کی جیئے گا کہ آپ کے نتائج تینوں اصول عدمی کی نیت کے عین متابک ہیں۔ مساوات   4.100 اور اس کے دوہری ترتیبی استعمال جہاں زاہر ہے۔
   l
کی جگہ
   s
   ہوگا۔
   
   سوال  4.28  سب سے زیادہ عمومی معمول سدا spinor  \عددی{\chi}  مساوات 4.139  کے لیے 
   \عددی{S_x,S_y,S_z , S_x^2, S_y^2}
   اور
   .\عددی{S_z^2}
 تلاش کریں۔ تسدیق کریں کہ 
 \عددی{S_x^2 + S_y^2 + S_z^2 = S^2}
 ہوگا۔ 
 
 سوال 
 4.29
 (الف)  امتیازی spinor 
\عددی{S_y}
کے امتیازی عدداد تلاش لریں۔
(ب) عمومی حال \عددی{\chi} مساوات 
4.139
 میں پائے جانے والا ایک زرے کے 
 \عددی{S_y}
کی پیمائس سے کیا قیمتیں متوقے ہیں اور ہر قیمت کا احتمال کیا ہوگا۔ تصدیق کی جئے گا کہ تمام احتمال کا مجموعہ  1 ہوگا۔ دیہان رہے کہ a اور b غیر حقیقی بھی ہو سکیے ہیں۔  
(ج) $S_y$ کی پیمائس سے کیا قیمتیں متوقع ہیں اور ان کے احتمالات کیا ہوں گے۔ 

سوال 
4.30
کسی اختیاری رکھ r کے ہم رہ چکری زاویائی میارِ حرکت کے اجزاء کا کالپ 
\عددی{S_r}
تیار کریں۔ کروی محدد استعمال کریں جہاں درج ذیل ہوگا۔
\begin{align}
\hat{r} = \sin\theta \cos\phi \hat{i} + \sin\theta \sin\phi \hat{j} + \cos\theta \hat{k}
\end{align}
 \عددی{S_r}
کی امتیازی عدداد اور معمور سدا امتیازی spinor
تلاش کریں۔
\begin{align}
\chi_{+}^{(r)} = \begin{bmatrix} \cos(\theta/2)\\ e^{i\phi}\sin(\theta/2)\end{bmatrix};
\quad
\chi_{-}^{(r)} = \begin{bmatrix} e^{i\phi}\sin(\theta/2)\\ -\cos(\theta/2)\end{bmatrix};
\end{align}
چونکہ آپ اپنی مرضی کے دوہری جزضرب 
\عددی{e^{i\phi}}
سے ضرب دے سکتے ہو۔ لہازا آپ کا جواب کچھ مختلف ہو سکتا ہے۔ 

سوال 
4.31
ایک زرا جس کا چکر ایک ہے کے لیے چکری کالپ 
\عددی{S_x, S_y}
اور
\عددی{S_z}
تیار کریں۔ اشعارہ 
\عددی{S_z}
کے کتنے امتیازی حالات ہونگے ہر ایسے حال پر 
\عددی{S_z, S_+}
اور
\عددی{S_{-}}
کا عمل تاین کریں۔ نصاب میں 
\عددی{1/2}
چکر کے لیے استعمال کی گئی ترقیب استعمال کریں
%بجا لا سکتے ہیں

\subsection{ مقناطیسی میداں میں ایک الیکٹران }
ایک چکر کاٹتے ہوئے باربار زرا پر مقناطیسی جفد کتب مشتمل ہوگا۔ اس کا مقناطیسی جفد کتبی معیارِ اثر $\mu$ ، زرے کی چکری زاویائی معیارِ حرکت $S$ کو راست متناسب ہوگا۔
\begin{align}
\mu = \gamma S
\end{align}
جہاں تناسبی مستقل $\gamma$ مسقن مقناطیسی نسبت کہلاتا ہے۔ مقناطیسی میدان $B$ میں رکھے گئے مقناطیسی جفد کتب پر قوتِ مروڑ
\عددی{\mu \times B}
عمل کرتا ہے۔ جو کمپس کی سوئے کی طرح اس کو میدان کے متوازج لانے کی کوسس کرتا ہے۔ اس قوتِ مروٹ کے ساتھ وبستا توانائی درج ذیل ہوگی۔
\begin{align}
H = -\mu . B
\end{align}
لہازا مقناطیسی میدان $B$ میں ایک نقطہ پر رہتے ہوئے ایک باردار چکر کھاتے ہوئے زرے کا ہیملٹونیں درج زیل ہوگا۔
\begin{align}
H = -\gamma B.S
\end{align}

