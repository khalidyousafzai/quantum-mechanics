\documentclass{book}
\usepackage{fontspec}
\usepackage{makeidx}
\usepackage{amsmath}                                                                         %\tfrac for fractions in text
\usepackage{amssymb}    
\usepackage{gensymb}  
\usepackage{amsthm}      						%theorem environment. started using in the maths book
\usepackage{mathtools}
\usepackage{multicol}
\usepackage{commath}									%differentiation symbols
\usepackage{polyglossia}    
\setmainlanguage[numerals=maghrib]{arabic}     %for english numbers use numerals=maghrib, for arabic numerals=arabicdigits
\setotherlanguages{english}

\newfontfamily\arabicfont[Scale=1.0,Script=Arabic]{Jameel Noori Nastaleeq} 
\setmonofont{DejaVu Sans Mono}                                                                  %had to add this and the next line to get going after ubuntu upgrade
\let\arabicfontt\ttfamily                                                                                  %had to add this and the above line to get going after ubuntu upgrade
\newfontfamily\urduTechTermsfont[Scale=1.0,Script=Arabic]{AA Sameer Sagar Nastaleeq Bold}
\newfontfamily\urdufont[WordSpace=1.0,Script=Arabic]{Jameel Noori Nastaleeq}
\newfontfamily\urdufontBig[Scale=1.25,WordSpace=1.0,Script=Arabic]{Jameel Noori Nastaleeq}
\newfontfamily\urdufontItalic[Scale=1.25,WordSpace=1.0,Script=Arabic]{Jameel Noori Nastaleeq Italic}
\setlength{\parskip}{5mm plus 4mm minus 3mm}
\begin{document}
%126-128
سوال 
3.32\\
توانائی اور وقت کی عدم یقینیت کے اصول کی ایک دلچسپ روپ
\(\Delta t=\tau/\pi\)
ہے جہاں ابتدائی حال
\(\Psi (x,0)\)
کے عموری حال تک ارتقا کے لیے درکار وقت
\(\tau\)
ہے . دو معیاری عموری ساکن حالات کے دو برابر حصوں پر مشتمل اختیاری مخفی قو
\(\Psi(x,0)=1/\sqrt{2}[\psi_{1}(x)+\psi_{2}(x)]\)
کا تفال موج استعمال کرتے ہوئے اس کی چانچ پڑتال کریں .\\
سوال 
3.33\\
ہارمونی مرتعش کی ساکن حالات کی معیاری عموری اساس مساوات 2.67
میں
\(\langle n|x|n'\rangle\)
اور
\(\langle n|p|n'\rangle\)
تلاش کریں . آپ سوال 2.12 میں
\(n=n'\)
تلاش کر چکے ہیں . وہی ترقیب موجودہ عمومی مسئلے میں استعمال کریں۔ متکانبی لامتناہی
\(\textbf{X}\)
اور
\(\textbf{P}\)
تلاش کریں . دکھائیں کہ اس اساس میں
\((1/2m)\text{\textbf{P}}^{2}+(m\omega^{2}/2)\text{\textbf{X}}^{2}=\text{\textbf{H}}\)
وتری ہے . کیا اس کے وتر ی ارکان آپکے توقع کے مطابق ہیں .
جزوی جواب
\[\langle n|x|n'\rangle=\sqrt{\frac{\hbar}{2m\omega}}(\sqrt{n'}\delta_{n,n'-1}+\sqrt{n}\delta_{n',n-1})\]
سوال 3.34
ایک ہارمونی مرتعش ایسی حال میں ہے کہ اس کی توانائی کو پیمائش
\((1/2)\hbar\omega\)
یا
\((3/2)\hbar\omega\)
ایک دوسرے جیسے احتمال کے ساتھ دے گی . اس حال میں
\(\langle p\rangle\)
کی زیادہ سے زیادہ ممکنہ قیمت کیا ہوگی۔ اگر لمحہ
\(t=0\)
پر اس کی
\(\langle p\rangle\)
کی زیادہ سے زیادہ ممکنہ قیمت ہو تو
\(\Psi(x,t)\)
کیا ہوگا .\\
سوال 
3.35\\
ہارمونی مرتعش کے منطقی حالات. \\
ہارمونی مرتعش کے ساکن حالات 
\(|n\rangle =\psi_{n}(x)\)
مساوات 2.67 میں صرف
\(n=0\)
عین عدم یقینیت کی حد
\(\sigma_{x}\sigma_{p}=\hbar/2\)
پر بیٹھتا ہے جیسا آپ سوال 3.12 میں معلوم کر چکے ہیں عمومی طور پر
\(\sigma_{x}\sigma_{p}=(2n+1)\hbar/2\)
ہوگا . تاہم چند خطی جوڑ جنہیں منطقی حالات کہتے ہیں بھی عدم یقینیت کے حاصل ضرب تو کم سے کم کرتے ہیں جیسا ہم دیکھتے ہیں یہ عامل س تکیل کے امتیازی تفال ہوتے ہیں۔\\
\[a_{-}|\alpha\rangle =\alpha|\alpha\rangle\]
جہاں امتیاز ی 
\(\alpha\)
کوئی بھی مخلوط عود ہوسکتا ہے .\\
جز الف\\
حال 
\(|\alpha\rangle\)
میں
\(\langle x \rangle\)
,
\(\langle x^{2} \rangle\)
,
\(\langle p \rangle\)
اور
\(\langle p^{2} \rangle\)
دریافت کریں۔ مثال 2.5 کی ترقیب استعمال کیں۔ اور یاد رکھیں کہ
\(a_{-}\)
منفی کا پرمش جوڑی دار
\(a_{+}\)
ہے ساتھ ہی یہ فرض نہ کریں کہ
\(\alpha\)
حقیقی ہے .\\
جز ( ب )\\
\(\sigma_{x}\)
اور
\(\sigma_{p}\)
تلاش کریں . دکھائیں کہ
\(\sigma{x}\sigma{p}=\hbar/2\)
ہوگا .\\
جز ( ج )\\
کسی بھی دوسرے تفال موج کی طرح منطقی حال کو توانائی امتیازی حالات کی صورت میں پھیلایا جا سکتا ہے۔\\
\[|\alpha\rangle=\sum_{n=0}^{\infty}C_{n}|n\rangle\]
رکھا ئیں کہ پھیلاو کے عددی سر درج ذیل ہونگے۔\\
\[c_{n}=\frac{\alpha^{n}}{\sqrt{n!}}c_{0}\]
جر ( د )\\
\(|\alpha\rangle\)
کو معمول پر لاتے ہوئے
\(c_{0}\)
تعین کریں.
جواب
\(\text{exp}(-\abs{\alpha}^{2}/2)\)\\
جز ( ح)\\
اس کے ساتھ وقت کی تابعیت\\
\[|n\rangle\rightarrow e^{-iE_{n}t/\hbar}|n\rangle\]
منسلک کرکے دکھائیں کہ
\(|\alpha(t)\rangle\)
اب بھی
\(a{-}\)
کا امتیازی حال ہوگا لیکن وقت کے ساتھ امتیازی قیمت ارتقا پزیر ہوگا\\
\[\alpha(t)=e^{-i\omega t}\alpha\]
یوں منظقی حال ہمیشہ منطقی حال بی رہے گا اور عدم یقینیت کے حاصل ضرب کو تم سے کم برقرار رکھا ہے .\\
جز ( و ) کیا زمینی حال
\(|n=0\rangle\)
از خود منطقی حال ہے اگر ایسا ہو تب امتیازی قدر کیا ہوگا \\
%128-130
سوال 
3.36\\
مقصود عدم یقینیت کا اصول:\\
عمومی عدم یقینيت کا اصول مساوات  3.62 درجہ ذيل کہتا ہے
\[\sigma_{A}^{2}\sigma_{B}^{2}\geq\frac{1}{4}\langle C^{2} \rangle\]
جہاں
\[\hat{C}\equiv i[\hat{A},\hat{B}]\]
ہے\\ 
جذ الف\\
دکھائے کہ اس کو زیادہ مستحكم کرتے ہوئے درجہ ذیل روٹ میں لکھا جاسکتا ہے
\[\sigma_{A}^{2}\sigma_{B}^{2}\geq\frac{1}{4}(\langle C\rangle^{2}+\langle D \rangle ^{2})\]
جہاں
\(\hat{D}\equiv \hat{AB}+\hat{BA}+2\langle A \rangle \langle B \rangle\)
ہوگا\\
اشاره: مساوات 3.60 میں 
\(\text{Re}(z)\)
 اجزاه لیں\\
ب)\\
مساوات 3.99 کو
 A=B
  کے لئیے جانچیں چونکہ اس صورت میں
 C=0
   ہے لہذا معیاری عدم یقینیت غیر اہم ہوگا بدقسمتی سے مقصود عدم یقینيت کا اصول بھی زیادہ مددگار ثابت نہیں ہوتا\\
سوال
 3.37:\\
ایک نظام جو 3 صدی ہے کہ ہیملٹونین کو درجہ ذيل کا الف ظاہر کرتی ہے
\[\text{\textbf{H}}=\begin{pmatrix}
a&0&b\\
0&c&0\\
b&0&a\\
\end{pmatrix}\]
جہاں c اور a, b حقیقی اعداد ہیں.\\
ا) اگر اس نظام کا ابتدائی حال درجہ ذيل ہو
\[|\delta(0) \rangle=\begin{pmatrix}
0\\
1\\
0\\
\end{pmatrix}\]

تب
\(\delta(t)\)
  کیا ہو گا؟\\
ب) اگر اس نظام کا ابتدائی حال درجہ ذيل ہو
\[| \delta(0)\rangle =\begin{pmatrix}
0\\
0\\
1\\
\end{pmatrix}\]
تب
 \(\delta{t}\)
  کیا ہوگا؟\\
سوال
 3.38:\\
ایک نظام جو 3 صدی ہے کہ ہيملٹونین کو درجہ ذیل کا الف ظاہر کرتی ہے:\\
\[\text{\textbf{H}}=\hbar\omega\begin{pmatrix}
1&0&0\\
0&2&0\\
0&0&2\\
\end{pmatrix}\]
دیگر دو مشہود B اور A کو درجہ ذيل کا الف ظاہر کرتی ہے\\
\[\text{A}=\lambda\begin{pmatrix}
0&1&0\\
1&0&0\\
0&0&2\\
\end{pmatrix} , \text{B}=\mu\begin{pmatrix}
2&0&0\\
0&0&1\\
0&1&0\\
\end{pmatrix}
\]
جہاں
\(\omega, \lambda\)
اور
\(\mu\)
حقیقی مثبت اعداد ہیں.\\
H، A
اور
B
 کی امتیازی اقدار اور معمول پر لائے  گئے امتیازی تفاعل تلاش کریں. \\
ب) یہ نظام کسی عمومی حال
\[| \delta(0)  \rangle=\begin{pmatrix}
c_{1}\\
c_{2}\\
c_{3}\\
\end{pmatrix}\]
سے ابتداء کرتا ہے جہاں
\(\abs{c_{1}}^{2}+\abs{c_{2}}^{2}+\abs{c_{3}}^{2}=1\)
ہے.\\
لمحہ t=0 پر B اور ,A Hکی توقعاتی قیمت تلاش كريں.\\
ج)
 \(\delta(s)\) 
کیا ہوگا ؟
لمحہ t پر اس نظام کی توانائی کی پیمائش کیا قيمتيں دے سکتی ہے؟ اور ہر ایک کا انفرادى احتمال کیا ہوگا؟ اسی سوال کے جوابات B اور A کے لیے بھی تلاش كریں.\\
سوال 3.39:
ا) ایک تفاعل
 \(f(x)\)
 جس کو Taylor تسلسل کی صورت میں پھیلایا جاسکتا ہے کے لیے درجہ ذيل  دکھائیں:\\
\[f(x+x_{0})=e^{i\hat{p}x_{0}/\hbar}f(x)\]
جہاں
\(x_{0}\)
کوئی بھی مستقل فاصلہ ہو سکتا ہے. اسی وجہ کے بنا 
\(\hat{p}/\hbar\)
کو فضا میں  انتقال کا پیداکار کہتے ہیں. یہاں دیہان رہے کہ ایک حامل کی قوت نمائی کی قوتی تسلسلی پھیلاؤ کی تعریف درجہ ذیل ہے
\(e^{\hat{Q}}=1+\hat{Q}+(1/2)\hat{Q}^{2}+(1/3!)\hat{Q}^{3}+\dotsc\)\\
ب) اگر
\(\Psi(x,t)\)
وقت تابع schrodinger مساوات کو مطمئن کرتا ہو تب درجہ ذیل دکھائیں\\
\[\Psi(x,t+t_{0})=e^{-i\hat{\text{H}}t_{0}/\hbar}\Psi(x,t)\]
جہاں
\(t_{0}\)
کوئی بھی مستقل وقت ہوگا. اسی وجہ کی بنا
\(-\hat{\text{H}}/\hbar\)
وقت میں انتقال کا پیدا کار کہتے ہے.\\
ج) دکھائیں کے لمحہ
\(t+t_{0}\)
پر ہر کی متغير Q کی توقعاتی قیمت کو درجہ ذيل لکھا جا سکتا ہے\\
\[\langle Q \rangle _{t+t_{0}}=\langle \Psi(x,t)|e^{i\hat{\text{H}}t_{0}/\hbar}\hat{Q}(x,p,t+t_{0}e^{-i\hat{\text{H}}t_{0}/\hbar}|\Psi(x,t)\rangle\]
اس کو استعمال کرتے ہوئے مساوات 3.71 حاصل كریں.\\
اشاره:
 \(t_{0}=\dif{t}\) 
لیتے ہوئے dt میں پہلے رتبے تک پھیلائیں.\\
سوال 3.40:\\
ا) ایک آزاد ذرہ کے لیے وقت تابع schrodinger مساوات کو معیار حرکت فضاه میں لکھ کر حل کریں.
جواب:  
\(\text{exp}(-ip^{2}t/2m\hbar)\Phi(p,0\))\\
ب) متحرک گاسی موجی اکٹ سوال 2.43 کے لئیے
\(\Phi(p,0)\)
تلاش كريں اور
\(\Phi(p,t)\)
تیار كریں. ساتھ ہی
\(\abs{\Phi(p,t)}^{2}\)
تیار كريں. اب دیکھیں گے کہ یہ وقت کا تابع نہیں ہوگا.\\
ج)
\(\Phi\)
پر مبنی موضوع تكملات حل کرتے ہوے
\(\langle p \rangle\)
اور
\(\langle p^{2} \rangle\)
تلاش کرکے سوال 2.43 کی حاصل كرده جوابات کے ساتھ موازنہ كريں.\\
د) دکھائیں کے
\(\langle H \rangle =\langle p \rangle ^{2}/2m+\langle H \rangle _{0}\)
ہوگا جہاں زیر نوہشت میں 0 ساکن گاسی کو ظاہر کرتا ہے اور اپنے نتیجے پہ تبصرہ کریں.\\
%131-137
باب \\4
3 حابات میں کوانٹم میکانیات \\
حصہ 4.1 \\
گروی محدد میں سکروڈنگر مساوات \\
اس کو باآسانی 3 آباد تک وسعت دی جا سکتی ہے۔
سکروڈنگر مساوات درج ذیل کہتی ہے \\
\[i\hbar\frac{\dif{\Psi}}{\dif{t}}=H\Psi;\]
معاری در عمل کو x کے ساتھ ساتھ y اور z پر لاگو کرتے ہوئے
\[p_{x}\rightarrow \frac{\hbar}{i}\frac{\partial}{\partial{x}},p_{y}\rightarrow \frac{\hbar}{i}\frac{\partial}{\partial{y}},p_{z}\rightarrow \frac{\hbar}{i}\frac{\partial}{\partial{z}}\]
حا
حملٹونی حامل ایچ کو لا سیکی توانائ
\[\frac{1}{2}mv^{2}+V=\frac{1}{2m}(p_{x}^{2}+p_{y}^{2}+p_{z}^{2})+V\]
سے حاصل کیا جاتا ہے\\
مختصرن درج ذیل لیا جاتا ہے
\[p\rightarrow \frac{\hbar}{i}\nabla\]
یوں درج ذیل ہوگ\\ا
\[i\hbar\frac{\partial{\Psi}}{\partial{t}}=-\frac{\hbar^{2}}{2m}\nabla^{2}\Psi+V\Psi\]
جہاں
\[\nabla^{2}\equiv \frac{\partial{2}}{\partial{x}}+\frac{\partial{2}}{\partial{y}}+\frac{\partial{2}}{\partial{z}}\]
کارتیسی محدد میں لیپلا سی ہے\\
مخفی توانائی
 V
  اور طفال موج 
 \(\Psi\)
   اب
\(r=x,y,z\)
اور t کے طفال ہیں 
لا متاناہی چھوٹی ہجم 
\(d^{3}r=\dif{x}\dif{y}\dif{z}\)
میں ایک ذرے کے ملنے کا احتمال
 \(\abs{\Psi(r,t)}^{2}d^{3}r\)
  ہوگا اور معمول پر لانے کی شرط اب درج ذیل ہوگی \\
\[\int\abs{\Psi}^{2}d^{3}r=1\]
جہاں تکمل کو پوری فضا پر لینا ہوگا .
اگر مخفی توانائی وقت کے طابع نہ ہو تو تب ساکن حالات کا مکمل سلسلہ پایا جائے گا \\
\[\Psi_{n}(r,t)=\psi_{n}(r)e^{-iE_{n}t/\hbar}\]
جہاں فضائی طفال موج 
\(\psi_{n}\)
 وقت کے غیر طابع سکروڈنگر مساوات 
\[\boxed{-\frac{\hbar^{2}}{2m}\nabla^{2}\psi+V\psi=E\psi}\]
کو مطمعن کرتی ہیں \\
طابع وقت سکروڈنگر مساوات کا عمومی حل درج ذیل ہوگا
\[\Psi(r,t)=\sum c_{n}\psi_{n}(r)e^{-iE_{n}t/\hbar}\]
جہاں مستقلات 
\(c_{n}\)
 کو ہمیشہ کی طرح ابتدائی طفال موج 
 \(\Psi(r,0)\)
  سے حاصل کیا جائے گا .
اگر مخفی توانائی ایسے حالات دیتے ہوں جو غیر مسلسل نہ ہوں تب مساوات 4.9 میں مجموعہ کی بجائے تکمل ہوگا\\
سوال 
4.1\\
جز (ا) عاملین r اور p کے تمام باضابطہ تبادلی رشتے
\[[x,y],[x,p_{y}],[x,p_{x}],[p_{y},p_{z}]\]
وغیرہ وغیرہ حاصل کریں\\
جواب
 \[[r_{i},p_{j}]=-[p_{i},r_{j}]=i\hbar\delta_{i,j},[r_{i},r_{j}]=[p_{i},p_{j}]=0\]
جہاں اشاریہ x, y اور z کو ظاہر کرتے ہیں 
جبکہ 
\(r_{x}=x,r_{y}=y\)
اور 
\(r_{z}=z\)
ہونگے\\
جز (ب) 3 آباد کے لیے مسئلہ ارن فیسٹ کی تصدیق کریں \\
\[\frac{d}{\dif{t}}\langle r\rangle =\frac{1}{m}\langle p\rangle \quad \frac{d}{\dif{t}}\langle p\rangle=\langle -\nabla V\rangle\]
ان میں سے ہر ایک در حقیقت تین مساوات کو ضاہر کرتا ہے .
ایک مساوات ایک جز کے لیے اشارہ 
اول تصدیق کریں کہ مساوات 3.71 تین آباد کے لیے بھی کارآمد ہے .\\
جز (ج) ہیسن برگ عدم یقینیت کے اصول کو تین آباد کے لیے بیان کریں \\
جواب\\
 \[\sigma_{x}\sigma{p}\geq\hbar/2,\sigma_{y}\sigma{p}\geq\hbar/2,\sigma_{z}\sigma{p}\geq\hbar/2\]
جہاں مسلن 
\(\sigma_{x}\sigma_{py}\)
 پر کوئی پابندی نہیں ہے\\
حصہ 4.1.1 \\
طغیراط کی الہدگی\\ 
  عمومن مخفی توانائی صرف مبدا سے فاصلے کا طفال ہوگا 
ایسی صورت میں گروی مہدد
 \((r,\theta,\phi)\)
  کا استعمال بہتر ثابت ہوگا 
شکل 4.1 
گروی مہددی زبان میں لیپلاسی درج ذیل روپ اختیار کرتا ہے \\
\[\nabla^{2}(r)=\frac{1}{r^{2}}\frac{\partial}{\partial{r}}\big (r^{2}\frac{\partial}{\partial{r}}\big )+\frac{1}{r^{2}\sin{\theta}\dif{\theta}}\frac{\partial}{\partial{\theta}}\big(\sin{\theta}\frac{\partial}{\partial{\theta}}\big )+\frac{1}{r^{2}\sin^{2}{\theta}}\big(\frac{\partial^{2}}{\partial{\phi^{2}}}\big )\]
یوں گروی مہدد میں وقت طابع سکروڈنگر مساوات درج ذیل ہوگی 
\[-\frac{\hbar^{2}}{2m}\big [\frac{1}{r^{2}}\frac{\partial}{\partial{r}}\big (r^{2}\frac{\partial}{\partial{r}}\big )+\frac{1}{r^{2}\sin{\theta}\dif{\theta}}\frac{\partial}{\partial{\theta}}\big(\sin{\theta}\frac{\partial}{\partial{\theta}}\big )+\frac{1}{r^{2}\sin^{2}{\theta}}\big(\frac{\partial^{2}}{\partial{\phi^{2}}}\big )\big ]+V\psi=E\psi\]
ہم ایسے حلوں کی تلاش میں ہیں جن کو حصل ظرب کی صورت میں الہدہ الہدہ کرنا مشکل ہو \\
\[\psi(r,\theta,\phi)=R(r)Y(\theta,\phi)\]
اس کو مساوات 4.14 میں پر کرتے ہیں \\
\[-\frac{\hbar^{2}}{2m}\big [\frac{Y}{r^{2}}\frac{\partial}{\partial{r}}\big (r^{2}\frac{\partial{R}}{\partial{r}}\big )+\frac{R}{r^{2}\sin{\theta}\dif{\theta}}\frac{\partial}{\partial{\theta}}\big(\sin{\theta}\frac{\partial{Y}}{\partial{\theta}}\big )+frac{R}{r^{2}\sin^{2}{\theta}}\big(\frac{\partial^{2}{Y}}{\partial{\phi^{2}}}\big )\big ]+VRY=ERY\]
دونوں اطراف کو
\(RY\)
سے تقسیم کر کہ 
\(-2mr^{2}/\hbar^{2}\)
سے ظرب دیتے ہیں \\
\[\big\{\frac{1}{R}\frac{d}{\dif{r}}\big(r^{2}\frac{\dif{R}}{\dif{r}}\big)-\frac{2mr^{2}}{\hbar^{2}}[V(r)-E]\big\}\]
\[+\frac{1}{Y}\big\{\frac{1}{\sin{\theta}}\big(\sin{\theta}\frac{\partial{Y}}{\partial{\theta}}\big)+\frac{1}{\sin^{2}{\theta}}\frac{\partial^{2}{Y}}{\partial{\phi^{2}}}\big\}=0\]
پہلی خمدار کوسین کے اندر جز صرف r کے طابع ہے جبکہ باقی حصہ صرف
 \(\theta\)
  اور
  \(\phi\)
    پر منحصر ہے .
لحاضہ یہ حصے ایک مستقل کے برابر ہونگے. 
اس الہدگی مستقل کو ہم الہدگی
\(l(l+1)\)
کے روپ میں لکھتے ہیں جس کی وجہ کچھ دیر میں واضح ہوگی \\
\[\frac{1}{R}\frac{d}{\dif{r}}\big(r^{2}\frac{\dif{R}}{\dif{r}}\big)-\frac{2mr^{2}}{\hbar^{2}}[V(r)-E]=l(l+1)\]
\[\frac{1}{Y}\big\{\frac{1}{\sin{\theta}}\big(\sin{\theta}\frac{\partial{Y}}{\partial{\theta}}\big)+\frac{1}{\sin^{2}{\theta}}\frac{\partial^{2}{Y}}{\partial{\phi^{2}}}\big\}=-l(l+1)\]
سوال 
4.2\\
کارتیسی محدد میں الہدگی متہیرات کی ترکیب استعمال کرتے ہوئے لا متاناہی مربی کنواں کے مسئلے کو حل کریں
\[V(x,y,z)=\begin{cases}
0\\
\infty\\
\end{cases}\]
جز (ا) ساکن حالات تلاش کریں اوران کی مطابقتی توانائیاں دریافت کریں \\
جز (ب) 
بڑھتی توانائی کی ترتیب کے لحاظ سے انفرادی توانائیوں کو 
\(E_{1}, E_{2}, E_{3}\dotsc\)
سے ظاہر کریں 
اب 
\(E_{1},E_{2},E_{3},E_{4},E_{5}, E_{6}\)
تلاش کریں 
ان کی انحتاتی حالوں کی تعداد تلاش کریں 
یعنی ایک ہی توانائی کے مختلف حالاتوں کی تعداد 
آپ نے سوال 2.5 میں دیکھا ہوگا کہ یک بودی صورت میں انحتاتی مقید حال نہیں پائے جاتے لیکن تین آبادی صورت میں یہ عمومی طور پر پائے جاتے ہیں \\
جز (ج) توانائی
 \(E_{14}\)
  کی انحتاتی تعداد کتنی ہے اور یہ صورت کیوں دلچسپ ہے۔ \\
حصہ 4.1.2 \\
ظوایائی مساوات \\
مساوات 4.17 میں
 \(\theta\)
  اور
   \(\phi\)
  \(\psi\)
     طابع ہے .
جس کو
 \(Y\sin^{2}{\theta}\)
  سے ضرب دے کر درج ذیل حاصل ہوگا .\\
\[\sin{\theta}\frac{\partial}{\partial{\theta}}\big(\sin{\theta}\frac{\partial{Y}}{\partial{\theta}}\big)+\frac{\partial^{2}{Y}}{\partial{\phi^{2}}}=-l(l+1)\sin^{2}{\theta Y}\]
ہا سکتا ہے آپ اس مساوات کو پہچانتے ہوں .
یہ کلاسیکی برقی ہلکیات میں لیپلا سی مساوات کی حل میں پایا جاتا ہے ۔.
ہمیشہ کی طرح ہم الہدگی متہیرات استعمال۔
\[Y(\theta,\phi)=\Theta(\theta)\Phi(\phi)\]
اس کو پر کر کے 
\(\Theta\Phi\)
سے تقسیم کر کہ درج ذیل حاصل ہوگا 
\[\big\{\frac{1}{\Theta}\big[\sin{\theta}\big(\frac{\dif{\Theta}}{\dif{\theta}}\big)\big]+l(l+1)\sin^{2}{\theta}\big\}+\frac{1}{\Phi}\frac{\dif^{2}{\Phi}}{\dif{\phi^{2}}}=0\]
پہلا جز صرف
 \(\theta\)
  کا طفال ہے جبکہ دوسرا صرف
  \(\phi\)
    کا طفال ہے لہازا ہر ایک جز ایک مستقل ہوگا۔
اس بار ہم الہدگی مستقل کو
 \(m^{2}\)
  لکھتے ہیں .
\[\frac{1}{\Theta}\big[\sin{\theta}\big(\frac{\dif{\Theta}}{\dif{\theta}}\big)\big]+l(l+1)\sin^{2}{\theta}=m^{2}\]
\[\frac{1}{\Phi}\frac{\dif^{2}{\Phi}}{\dif{\phi^{2}}}=-m^{2}\]
 \(\phi\)
 کی مساوات زیادہ آسان ہے۔\\
 \[\frac{\dif^{2}{\Phi}}{\dif{\phi^{2}}}=-m^{2}\Phi\Rightarrow\Phi(\phi)=e^{im\phi}\]
درحقیقت اس کے دو حل پائے جاتے ہیں 
\(exp(im\phi)\)
اور 
\(exp(-im\phi)\)
لیکن m کو منفی ہونے کی اجازت دیتے ہوئے ہم معاخر الزکر کو بھی شامل کرتے ہیں۔.
اس کے آگے ایک مستقل جز بھی ہو سکتا ہے جسے ہم
 \(\Theta\)
  میں جزب کرتے ہیں چونکہ برقی مخفی توانائی لازمن حقیقی ہوگی لہازا برقی حلقیات میں
  \((\Phi)\)
    طفال کو sin اور cosin کی صورت میں نہ کہ قوت نمائی صورت میں لکھا جاتا ہے۔.
کوانٹم میکانیات میں ایسی کوئی پابندی نہیں پائی جاتی ہے اور قوت نمائی کے ساتھ کام کرنا زیادہ آسان ہوتا ہے۔
اب جب
 \(\phi\)
  کی قیمت
  \(2\pi\)
    بڑھتی ہے ہم فضا میں واپس اسی نقطہ پر پہنچتے ہیں۔.
شکل 4.1 دیکھیں۔
لہازا درج ذیل ہوگا.\\
\[\Phi(\phi+2\pi)=\Phi(\phi)\]
دوسرے الفاظ میں 
\(exp[im(\phi+2\pi)]=exp(im\phi)\)
یا
\(exp(2\pi im)=1\)
ہوگا۔
یوں لازم ہے کہ m حدد سہی ہوگا۔
\[m=0,\pm 1,\pm 2,\dotsc\]
 \(\theta\)
  کی مساوات
\[\sin{\theta}\frac{\dif}{\dif{\theta}}\big(\sin{\theta}\frac{\dif{\Theta}}{\dif{\theta}}\big)+[l(l+1)\sin^{2}{\theta}-m^{2}]\Theta=0\]
اتنی سادہ نہیں ہے۔.
اس کا حل درج ذیل ہے۔\\
\[\Theta(\theta)=AP_{l}^{m}(\cos{\theta})\]
جہاں
 \(P_{l}^{m}\) 
شریک یارینڈر طفال ہے جس کی تعاریف درج ذیل ہے \\
\[P_{l}^{m}(x)\equiv (1-x^{2})^{\abs{m}/2}\big(\frac{\dif}{\dif{x}}\big)^{\abs{m}}P_{l}(x)\]
اور یارینڈر طفیل رکنے کی جز کو
 \(P_{l}(x)\)
  ظاہر کرتا ہے جس کی تعاریف گوڈریف کا قلیہ دیتی ہے. \\
\[P_{l}(x)\equiv\frac{1}{2^{l}l!}\big(\frac{\dif}{\dif{x}}\big)^{l}(x^{2}-1)^{l}\]
مثال کے طور پر درج ذیل ہونگے \\
\[P_{0}(x)=1,P_{1}(x)=\frac{1}{2}\frac{\dif}{\dif{x}}(x^{2}-1)=x\]
\[P_{2}(x)=\big(\frac{\dif}{\dif{x}}\big)^{2}(x^{2}-1)^{2}=\frac{1}{2}(3x^{2}-1)\]
جدول 4.1 میں ابتدائی چند لییانڈر قثیر رکنیہ پیش کی گئی ہیں جیسا کہ نام سی ظاہر ہے 
\(P_{1}(x)\)
ایک ایسا قثیر رکنی ہے جس کا متبیر x ارو جس کا درجا L ہے۔
لہازا L طہ کرتا ہے کہ یہ جفط یا طاق ہے۔
تاہم
 \(P_{l}^{m}(x)\)
  ایک تسر رکنی نہیں ہوگا۔
اگر  m تاک ہو تب اس کا ایک جزروی 
\(\sqrt{1-x^{2}}\)
ہوگا ۔\\
\[P_{2}^{0}(x)=\frac{1}{2}(3x^{2}-1), P_{2}^{1}(x)=(1-x^{2})^{1/2}\frac{\dif}{\dif{x}}\big[\frac{1}{2}(3x^{2}-1)\big]=3x\sqrt{1-x^{2}}\]
\[p_{2}^{2}(x)=(1-x^{2})\big(\frac{\dif}{\dif{x}}\big)^{2}\big[\frac{1}{2}(3x^{2}-1)\big]=3(1-x^{2})\]
بلا بلا وغیرہ وغیرہ 
%137-145
اب ہمیں
\(P_{l}^{m}(cos\theta)\)
چاہیے اور
\(\sqrt{1-cos^{2}\theta}=sin\theta\)
ہوتا ہے لہٰذا
\(P_{l}^{m}(cos\theta)\)
ہر صورت
\(cos\theta\)
کا قصور ركنی ہوگا جو تاک m کی صورت میں
\(sin\theta\)
سے ضرب ہوا ہوگا۔
جو جدول 
4.2.
میں چند شریک legenders دافات پیش کیے گئے ہیں ۔\\
دیہان رہے کے قلیہ rodrigues صرف اُس صورت کوئی معنی رکھتا ہے جب l غیر منفی عدد سہی ہو ۔
مزید
\(|m|\textgreater{l}\)
کی صورت میں مساوات 
4.27
کے تہت
\(P_{l}^{m}=0\)
ہوگا ۔یوں l کی کسی بھی مخصوص قیمت کے لئے m کی
\((2l+1)\)
ممکنہ قیمتیں ہونگیں:\\
\[l=0,1,2,...;\quad{m=-l,-l+1,...-1,0,1,...l-1,l.}\]
لیکن رکیے مساوات 
4.25
ایک رتبی تفرقی مساوات ہے جسکے l اور m کی کسی بھو قیمتوں کے لئے دو خطی غیرتابہ حل ہونگے ۔ باکی حل کہا ہیں؟
جواب :
یہ مساوات کی رعایتی حلوں کی  صورت میں ضرور موجود ہیں لیکن یہ طبی بنیادوں پر ناقابلِ قبول ہیں چونکہ
\(\theta=0\)
اور/یا
\(\theta=\pi\)
پر یہ بےقابو بڑھتے ہیں (سوال 4.4 دیکھیں )۔\\
اب قروی محدد میں ھجلی رکن درجذیل ہوگا
\[d^{3}r=r^{2}sin\theta{dr}d\theta{d\phi}.\]
لہٰذا معمول پر لانے کی شرط (مساوات 
4.6)
درجزیل اختیار کرے گی
\[\int|\psi|^{2}r^{2}sin\theta{drd\theta}d\phi-\int|R|^{2}r^{2}dr\int|Y|^{2}sin\theta{d\theta}d\phi=1.\]
R اور Y کو علیحدہ علیحدہ طور پر معمول پر لانا زیادہ آسان صابط ہوتا ہے:\\
\[\int_{0}^{\infty}|R|^{2}r^{2}dr=1\quad{and}\int_{0}^{2\pi}\int_{0}^{\pi}|Y|^{2}sin\theta{d\theta}d\phi=1.\]
معمول شدہ  دائریعی موجی دفاتوں کو قروی ہارمونیات کہتے ہیں :\\
\[\boxed{Y_{l}^{m}(\theta,\phi)=\epsilon\sqrt{\frac{(2l+1)}{4\pi}\frac{(l-|m|)!}{(l+|m|)!}}e^{\iota{m\phi}}P_{l}^{m}(cos\theta.}\]
جہاں
\(\epsilon=(-1)^{m}\)
کے لئے
\(m\geq0\)
اور
\(\epsilon=1\)
کے لئے
\(m\leq0\)
ہوگا ۔ جیسا کے ہم بعد میں ثابت کریں گے یہ ازخود عمودی ہیں \\
\[\int_{0}^{2\pi}\int_{0}^{\pi}[Y_{l}^{m}(\theta,\phi)]^{*}[Y_{l'}^{m'}(\theta,\phi)]sin\theta{d\theta}d\phi=\delta_{ll'},\delta_{mm'}.\]
جو جدول
4.3
میں چند ابتدائ قروی ہارمونیات پیش کیے گئے ہیں ۔تاریخی وجوہات کی بینا L کو شمالی quanti عدد جبکے M کو مقناطیسی quanti عدد کہتے ہیں ۔\\
سوال4.3:
مساوات 4.27,4.28اور 4.32 استعمال کرتے ہوئے 
\(Y_{0}^{0}\)
اور
\(Y_{2}^{1}\)
تیار کریں ۔ اسکی تصدیق کریں کے یہ عمودی اور معمول شدہ ہیں ۔\\
سوال 4.4:
دیکھائیں کہ
\[\Theta(\theta)=Aln[tan(\theta/2)]\]
تمام
\(l=m=0\)
کے لئے مساوات
\(\theta\)
(مساوات 4.25) کو مطمئین کرتا ہے ۔یہ وہ دوسرا ناقابلِ قبول حل ہے ۔اس میں کیا خرابی ہے ؟\\
سوال 4.5:
مساوات 4.32 کو استعمال کرتے ہوے
\(Y_{1}^{1}(\theta,\phi)\)
اور
\(Y_{3}^{2}(\theta,\phi)\)
تیار کریں ۔(آپ
\(P_{3}^{2}\)
کو جو جدول 4.2 سے لے سکتے ہیں،جبکے
\(P_{1}^{1}\)
کو مساوات 4.27 ,4.28کی مدد سے آپکو تیار کرنا ہوگا)۔تصدیق کیجئے کے l اور m کی موضوں قیمتوں کیلئے یہ زاویائ مساوات (مساوات 4.18)، کو  مطمعین کرتے ہیں ۔\\
سوال 4.6:
قلیہ rodrigues سے ابتدہ کرتے ہوئے legendre کثیرقنیوں کی معیاری امودیت کی شرط\\
\[\int_{-1}^{1}P_{l}(x)P_{l'}(x)dx=\big(\frac{2}{2l+1}\big)\delta_{ll'}.\]
اخذ کریں۔ (اشارہ :تکمل حصۓ استعمال کریں )\\
حصہ 4.1.3 راداسی مساوات:\\
دیہان رہے کہ قروی طور پر تشاقلی تمام مخفی توانائیوں کے لئے تفعال موج کا زاوئی حصہ
\(Y(\theta,\phi)\)
ایک جیسا ہوگا ؛مخفی توانائی
\(V(r)\)
کی شقل وصورت تفعال موج کی صرف راداسی حصہ
\(R(r)\)
پر اثر انداز ہوگا جسے مساوات 4.16 تعین کرے گا:\\
\[\frac{d}{dr}\big(r^{2}\frac{dR}{dr}\big)-\frac{2mr^{2}}{\hslash^{2}}[V(r)-E]R=l(l+1)R.\]
نیے متغیورات استعمال کرتے ہوے اس مساوات کی سادہ روپ حاصل کی جاسکتی ہیں۔ درج زیل فرض کریں\\
\[u(r)\equiv{rR(r).}\]  
تاکہ
\(R=u/r,dR/dr=[r(du/dr)-u]/r^{2},(d/dr)[r^{2}(dR/dr)]=rd^{2}u/dr^{2},\)
ہو لہٰذا درجزیل ہوگا \\
\[\boxed{-\frac{\hslash^{2}}{2m}\frac{d^{2}u}{dr^{2}}+\big[V+\frac{\hslash^{2}}{2m}\frac{l(l+1)}{r^{2}}\big]u=Eu}\]
اسکو قداسی مساوات کہتے ہیں ۔یہ شکل و صورت کے لہٰذ سے یقبودی shrondinger مساوات 2.5 کی طرح ہے ماسواۓ مؤثر مغفی توانائی \\
\[V_{eff}=V+\frac{\hslash^{2}}{2m}\frac{l(l+1)}{r^{2}}.\]
کہ جس میں
\((\hslash^{2}/2m)[l(l+1)/r^{2}]\)
کا اضافی جزو پایا جاتا ہے جسے مرکز گریز جزو کہتے ہیں ۔ یہ كلاسیكی مکانیات کے مرکز گریز نقلی قوت کی طرح زرہ کو ممدہ سے دور دکھیلتا ہے اب معمول پر لانے کی شرط (مساوات 4.31 )درج ذیل روپ اختیار کرتی ہے\\ 
\[\int_{0}^{\infty}|u|^{2}dr=1\]
کوئی مخصوص مخفی توانائی
\(V(r)\)
نہ جانتے ہوۓ ہم صرف یہاں تک جاسکتے ہیں ۔\\
مثال 4.1:\\
درج ذیل لامتناحی قروی قواں پر غور کریں,
\[V(r)=\begin{cases}0&r\leq{a}\\\infty&r>a\\\end{cases}\]
اسکی تفعال موج اور اجازتی توانائیاں تلاش کریں ۔\\
حل :
قواں کے باہر تفعال موج صفر ہے جبکے قواں کے اندر راداسی مساوات درج ذیل کہتی ہے\\
\[\frac{d^{2}u}{dr^{2}}=\big[\frac{l(l+1)}{r^{2}}-k^{2}\big]u.\]
جہاں ہمیشہ کی طرح درج ذیل ہوگا 
\[k\equiv\frac{\sqrt{2mE}}{\hslash}\]
ہم نے اس مساوات کو سرحدی شرط
\(u(a)=0\)
مسلط کرتے ہوۓ حل کرنا ہے ۔ سب سے آسان صورت L= 0 کی ہے:\\
\[\frac{d^{2}u}{dr^{2}}=-k^{2}u\Rightarrow{u(r)=Asin(kr)+Bcos(kr).}\]
یاد رہے کے اصل تفعال موج
\(R(r)=u(r)/r,\)
ہے لیکن
\([cos(kr)]/r\)
کی صورت میں
\(r\rightarrow{0}\)
بےقابو بڑھتا ہے ۔یوں ہمیں
\(B=0\)
منتخب کرنا ہوگا اب سرحدی شرط پر پورا اترنے کے لئے ضروری ہے کہ
\(sin(ka)=0,\)
لہٰذا
\(ka=n\pi\)
جہا n کوئی بھی عدد سہی ہو سکتا ہے ظاہر ہے کہ اِجارتی توانائیاں درج ذیل ہونگی\\ 
\[E_{\pi0}=\frac{n^{2}\pi^{2}\hslash^{2}}{2ma^{2}},\quad(n=1,2,3,...).\]
جو ہیں یقبودی لامتناہی چکور قواں کی توانائیاں ہیں (مساوات 2.7)
\(u(r)\)
کو معمول پر لانے سے
\(A=\sqrt{2/a};\)
حاصل ہوگا۔ زاویائ جزو
\(Y_{0}^{0}(\theta,\phi)=1/\sqrt{4\pi},\)
ہونے کے بینا غیر اہم ہے کو ساتھ منسلک کرتے ہوئے درج ذیل حاصل ہوگا\\
\[\psi_{\pi00}=\frac{1}{\sqrt{2\pi{a}}}\frac{sin(n\pi{r/a})}{r}\]
]دیہان کیجیے گا کہ ساکن حالت کو تین quanti اعداد n,l اور m سے نام دیے گئے ہیں:
\(\psi_{nml}(r,\theta,\phi).\)
جبکہ توانائی, صرف n اور l پر منحصر ہوگی:
\(E{nl}.]\)\\
کسی بھی ایک اختیاری عدد سےہی L کے لئے مساوات 4.41 کا عمومی حل اتنا جانا پہچانا نہیں ہے 
\[u(r)=Arj_{l}(kr)+Brn_{l}(kr).\]
جہاں
\(j_{l}(x)\)
رتبی قروی Bessel دفعال ہے اور 
\(l,n_{i}(x)\)
رتبی قروی Neumann دفعال l ہے انکی تعریفات درج ذیل ہیں 
\[j_{l}(x)\equiv(-x)^{l}\big(\frac{1}{x}\frac{d}{dx}\big)^{l}\frac{sinx}{x}:\quad{n_{i}(x)\equiv-(-x)^{l}}\big(\frac{1}{x}\frac{d}{dx}\big)^{l}\frac{cosx}{x}.\]
مثال کے طور پر,\\
\begin{align*}
j_{0}(x)&=\frac{sinx}{x};\quad{n_{0}(x)=-\frac{cosx}{x};}\\
j_1(x)&=(-x)\frac{1}{x}\frac{d}{dx}\big(\frac{sinx}{x}\big)=\frac{sinx}{x^{2}}-\frac{cosx}{x};\\
j_{2}(x)&=(-x)^{2}\big(\frac{1}{x}\frac{d}{dx}\big)^{2}\frac{sinx}{x}=x^{2}\big(\frac{1}{x}\frac{d}{dx}\big)\frac{xcosx-sinx}{x^{3}}\\
\end{align*}
\[=\frac{3sinx-3xcosx-x^{2}sinx}{x^{3}}:\]
جو جدول 4.4 میں ابتدائ چند قروی Bessel اور Neumann دفعال پیش کیے گئے ہیں متغیر x کی چھوٹی قیمت کے لئے (یہاں
\(sinx=x-x^{3}/3!+x^{5}/5!...\)
اور
\(cosx=1-x^{2}/2+x^{4}/4!...),\)\\
\[j_{0}(x)\approx1;\quad{n_{0}(x)\approx-\frac{1}{x}};\quad{j_{1}(x)\approx\frac{x}{3}};\quad{j_{2}(x)\approx\frac{x^{2}}{15}};\]
وغیرہ ہونگے ۔ دیہان رہے ممدا پر Bessel دفعال متناہی رہتے ہیں جبکہ Neumann دفعالات بےقابو بڑھتے ہیں یوں لازم ہے کے ہم
\(B_{l}=0\)
منتخب کریں لہٰذا درج ذیل ہوگا \\
\[R(r)=Aj_{l}(kr)\]
اب سرحدی شرط,
\(R(a)=0\)
کو مطمئن کرنا باکی ہے ظاہر ہے کے k کو درج ذیل کے تحت منتخب کرنا ہوگا\\
\[j_{l}(ka)=0;\]
یعنی l رتبی قروی Bessel دفعال کا(ka) ایک صفر ہوگا اب ایک Bessel دفعالات ایہتحاشی ہیں (شکل 4.2 دیکھیں) ; ہر ایک کے لامتناہی تعداد کےصفر پاے جاتے ہیں, لیکن ہماری بدقسمتی ہے کہ یہ ایک جیسے فاصلوں پر نہیں پائے جاتے ہیں جیسا کہ نقطہ n یا نقطہ
\(n\pi\)
وغیرہ پر ۔انہیں اعدادی طور پر حاصل کرنا ہوگا ۔بحرحال سرحدی شرط کے تحت درج ذیل ہوگا \\
\[k=\frac{1}{a}\beta_{nl}\]
جہاں
\(\beta_{nl}\)
قروی Bessel دفعال کا
\(nth\)
صفر
\(lth\)
ہوگا ۔یوں اجارتی توانائیاں درج ذیل ہونگی \\
\[E_{nl}=\frac{\hslash^{2}}{2ma^{2}}\beta_{nl}^{2}.\]
اور دفعال موج درج ذیل ہونگے\\ 
\[\psi_{nlm}(r,\theta,\phi)=A_{nl}j_{l}(\beta_{nl}r/a)Y_{l}^{m}(\theta,\phi).\]
جہاں معمول زنی مستقل 
\(A_{n1}\)
 تعین کریں گے۔چونکہ l کی ہر ایک قیمت کے لئے m کی
\((2l+1)\)
مختلف قیمتیں پای جاتی ہیں لہٰذا توانائی کی ہر ستہ
\(2l+1\)
گناہ انہتاتی ہوگا۔\\
سوال 4.7:\\
جزو الف :
\(n_{1}(x)\)
اور
\(n_{2}(x)\)
کو (مساوات 4.46) میں پیش کی گئی تعریف سے تیار کریں ۔\\
جزو ب :
\(sines\)
اور
\(cosines\)
کو پھیلا کر
\(n_{1}(x)\)
اور
\(n_{2}(x)\)
کی تخمینی قلیات اخذ کریں جو
\(x\ll1\)
کے لئے كارامد ہونگیں ۔تصدیق کریں کے یہ ممدا پر بےقابو بڑھتے ہیں ۔\\
سوال 4.8:\\
جزو الف : تصدیق کریں کہ
\(V(r)=0\)
اور
\(l=1\)
کی صورت میں
\(Arj_{1}(kr)\)
راداسی مساوات کو مطمئن كرتا ہے ۔\\
جزو ب : لامتناہی قروی قواں کیلئے L =1 کی صورت میں اجاراتی توانایاں ترسیم کی مدد سے تعین کریں ۔ دیکھائیں کے n کی بڑی قیمت کے لئے 
\(E_{n1}\approx(\hslash^{2}\pi^{2}/2ma^{2})(n+1/2)^{2}\)
ہوگا ۔(اشارہ : پہلے دیکھائیں کے
\(j_{1}(x)=0\Rightarrow{x}=tanx\)
اسکے بعد
\(x\)
اور
\(tanx\)
کو ایک ہی جگہ ترسیم کرتے ہوئے انکے نقاط تلاش کریں ۔ ) \\
سوال 4.9: ایک ذرہ جسکی قیمت m ہے کو متناہی قروی قواں:
\[V(r)=\begin{cases}-V_{0}&r<a\\0&r>a\\\end{cases}\]
میں رکھا جاتا ہے اسکے زمینی حال کو l=0 لیتے ہوئے راداسی مساوات کے حل سے حاصل کریں ۔ دیکھائیں کے
\(V_{0}a^{2}\textless\pi^{2}\hslash^{2}/8m\)
کی صورت میں مقیت حل نہیں پایا جاتا ہے ۔
%145-150
حصہ ۴.۲
ہائیڈروجن جوہر ایک بھاری پروٹان جس کو ساکن تصور کیا جا سکتا ہے اور جس کا بار e ہے کے  گرد ایک حلقے الیکٹرون جس کا بار منفی e طواف کرتا ہے پر مشتمل ہوتا ہے ان دونوں کے مخالف بارو کے بیچ قوتِ کشش پائی جاتی ہے۔
 شکل ۴.۳
 کولوم کے قانون کے تحت مخفی توانائی درج ذیل ہوگی۔  
 \[V(r)=\frac{-e^{2}}{4\pi\epsilon_{0}}\frac{1}{r}\]
لہٰذا رداسی مساوات ۴.۳۷ درج ذیل روپ اختیار کرے گی۔
\[\frac{-\hbar^{2}}{2m}\frac{\dif^{2}{u}}{\dif{r^{2}}}+\big[\frac{-e^{2}}{4\pi\epsilon_{0}}\frac{1}{r}+\frac{\hbar^{2}}{2m}\frac{l(l+1)}{r^{2}}\big]u=Eu\]
 ہم نے اس مساوات کو 
 \(u(r)\)
  کے لئے حل کرکے اجازتی توانائیاں E تعین کرنی ہے۔  ہائیڈروجن جوہر کا حل اتنا اہم ہے کہ میں اس کو ہارمونی مرتاہش کے تہلیلی حل کے طریقے سے قدم با قدم حل کر کے پیش کرتا ہوں، جس قدم میں آپ کو دشواری پیش آئے، حصہ ۲.۳.۲ سے مدد لیں وہاں مکمل تفصیل پیش کی گئی ہے۔ کولوم مخفی توانائی مساوات۴.۵۲ 
 \(E>0\)
 استمراری حالات جو الیکٹران پروٹون ٹکراؤ کو ظاہر کرتے ہیں تسلیم کرنے کے ساتھ ساتھ غیر مسلسل مقید حالات جو ہائیڈروجن جوہر کو ظاہر کرتے ہے بھی تسلیم کرتا ہے۔ ہماری دلچپسی موخر الذِکر میں ہے۔ \\
 حصہ ۴.۲.۱ \\
 رداسی تفاعل موج، سب سے پہلے نئی علامتیں متعارف کرتے ہوئے مساوات کی صدا صورت حاصل کرتے ہیں درج ذیل لیتے ہوئے جہاں مقید حالات کے لئے e منفی ہونے کی وجہ سے k  حقیقی ہوگا۔ \\
 \[k\equiv \frac{\sqrt{-2mE}}{\hbar}\]
 مساوات ۴.۵۳ کو E سے تقسیم کرتے ہوئے درج ذیل حاصل ہوگا \\
\[\frac{1}{k^{2}}\frac{\dif^{2}{u}}{\dif{r^{2}}}=\big[1-\frac{me^{2}}{2\pi\epsilon_{0}\hbar^{2}k}\frac{1}{(kr)}+\frac{l(l+1)}{(kr)^{2}}\big]u\]
اس کو دیکھ کر ہمیں خیال آتا ہے کہ ہم درج ذیل علامتیں متعارف کرائے 
\[\rho\equiv kr \quad \rho_{0}\equiv\frac{me^{2}}{2\pi\epsilon_{0}\hbar^{2}k}\]
لہٰذا درج ذیل لکھا جائے گا۔
\[\frac{\dif^{2}{u}}{\dif{\rho^{2}}}=\big[1-\frac{\rho_{0}}{\rho}+\frac{l(l+1)}{\rho^{2}}\big]u\]
اِس کے بعد ہم حلو کی متاکاربی روپ پر غور کرتے ہیں۔ اب
\(\rho\rightarrow\infty\)
کرنے سے کوسین کے اندر مستقل جز غالب ہوگا لہٰذا تخمینی طور پر درج ذیل لکھا جا سکتا ہے۔
\[\frac{\dif^{2}{u}}{\dif{\rho^{2}}}=u\]
جس کا عمومی حال درج زیل ہے۔
\[u(\rho)=Ae^{-\rho}+Be^{+\rho}\]
تا ہم 
\(\rho\rightarrow\infty\)
کی صورت میں 
\(e^{\rho}\)
پے قابو برتا ہے لہذا ہمیں بھی
B=0
یوں 
\(\rho\)
کی بڑی قیمتوں کے لیے درج  ذیل ہوگا 
\[u(\rho)\sim Ae^{-\rho}\]
اس کے برعکس 
\(\rho\rightarrow 0\)
کی صورت میں مرکز گریز جز غالب ہوگا۔ لہٰذا تخمينی طور پر درج ذیل لکھا جاسکتا ہے۔ 
\[\frac{\dif^{2}{u}}{\dif{\rho^{2}}}=\frac{l(l+1)}{\rho^{2}}u\]
 جس کا عمومی حل تصدیق کیجئے درج ذیل ہوگا۔
 \[u(\rho)=C\rho^{l+1}+D\rho^{-l}\]
 تاہم
 \((\rho\rightarrow\infty)\)
 کی صورت میں
 \(\rho^{-l}\)
 پے قابو بھڑتا ہے لہذا
 D=0
 ہوگا۔
 \(\rho\)
 یوں 
 کی چھوٹی قیمتوں کے لیے درج ذیل ہوگا۔
 \[u(\rho)\sim C\rho^{l+1}\]
  دوسری قدم پر ہمیں متاکاربی رویئے کو ہٹانا ہوگا اس کی خاطر ہم نیا تفالوی رو اس اُمید سے متعارف کرتے ہے کے یہ
   \(u(\rho)\)
    رو سے زیادہ سادہ ہوگا۔
  \[u(\rho)=\rho^{l+1}e^{-\rho}x(\rho)\]
  پہلے اشارے اچھے نظر نہیں آتے

  اس طرح v رو کی صورت میں رداسی مساوات ۴.۵۶ 
  درج ذیل روپ اختیار کرتی ہے۔
  \[\rho\frac{\dif^{2}{v}}{\dif{\rho^{2}}}+2(1+l-\rho)\frac{\dif{v}}{\dif{\rho}}+[\rho_{0}-2(l+1)]v=0\]
  آخرمیں ہم فرض کرتے ہیں کہ حل
  \(v\)
    رو کو رو کی طاقتی تسلسل لکھا جا سکتا ہے۔
  \[v(\rho)=\sum_{j=0}^{\infty}c_{j}\rho^{j}\]
اب ہمیں عددی سر 
\((c_{0},c_{1},c_{2}...)\)
تعین کرنے ہے ہم جز در جز تفرق لیتے ہیں۔
\[\frac{\dif{v}}{\dif{\rho}}=\sum_{j=0}^{\infty}jc_{j}\rho^{j-1}=\sum_{j=0}^{\infty}(j+1)c_{j+1}\rho^{j}\]
میں نے دوسرے مجموعہ میں فرضی اشاریہ j کو 
\(j\rightarrow j+1\)
کہا ہے اگر آپکو اس سے پریشانی ہو تو آپ اولین چند اجزاء صریحاً لکھ کر اس کی درستگی کی تصدیق کر سکتے ہیں۔ آپ سوال اٹھا سکتے ہیں کے اب مجموعہ
\(j=-1\)
شروع ہونا چاہئے لیکن
\((j+1)\)
کا جز اس جز کو ختم کرتا ہے لہذا ہم ۰ سے بھی شروع کر سکتے ہیں۔ دوبارہ تفرق لیتے ہیں
\[\frac{\dif^{2}{v}}{\dif{\rho^{2}}}=\sum_{j=0}^{\infty}j(j+1)c_{j+1}\rho^{j-1}\]
مساوات ۴.۶۱ میں پر کرتے ہیں
\[\sum_{j=0}^{\infty}j(j+1)c_{j+1}\rho^{j}+2(l+1)+\sum_{j=0}^{\infty}(j+1)c_{j+1}\rho^{j}\]
\[-2\sum_{j=0}^{\infty}jc_{j}\rho^{j}+[\rho_{0}-2(l+1)]\sum{j=0}^{\infty}c_{j}\rho^{j}=0\]
ایک جیسی طاقتوں کے عددی سروں کو مساوی رکھتے ہوئے درج ذیل حاصل ہوگا۔
\[j(j+1)c_{j+1}+2(l+1)(j+1)c_{j+1}-2jc_{j}+[\rho_{0}-2(l+1)]c_{j}=0\]
یا
\[c_{j+1}=\big\{\frac{2(j+l+1)-\rho_{0}}{(j+1)(j+2l+2)}\big\}c_{j}\]
یہ کلیہ توالی عددی سر تعین کرتے ہوئے تفالوي رو تعین کرتے ہیں۔ ہم
\(c_{0}\)
سے شروع کر کے جو مجموعی مستقل کا روپ اختیار کرتا ہے جس کو آخر میںے معمول پے لیتے ہوئے حاصل کیا جائےگا۔
مساوات ۴.۶۳ کی مدد سے 
\(c_{1}\)
حاصل کرتے ہے جس کو واپس اسی مساوات میں پر کرتے ہوئے
\(c_{2}\)
حاصل ہوگا وغیرہ وغیرہ آئے J کی بری قیمت کے لئے عددی سروں کی صورت دیکھے۔ J کی بڑی قیمت 
\(\rho\)
 کی بڑی قیمت کو ظاہر کرتی ہے جہاں بلند طاقتیں غالب ہونگی۔ اس صورت میں کلیہ توالي درج ذیل رہتی ہے۔\\
\[c_{j+1}\cong\frac{2j}{j(j+1)}c_{j}=\frac{2}{(j+1)}c_{j}\]
ایک لمحہ کے لیے فرض کرے کہ یہ بلکل ٹھیک رشتہ ہے تب درج ذیل ہوگا۔
\[c_{j}=\frac{2^{j}}{j!}c_{0}\]
جس کی بنا پر درج ذیل لکھا جا سکتا ہے۔
\[v(\rho)=c_{0}\sum_{j=0}^{\infty}\frac{2^{j}}{j!}\rho^{j}=c_{0}e^{2\rho}\]
لہذا درج ذیل ہوگا۔
\[u(\rho)=c_{0}\rho^{l+1}e^{\rho}\]
 جو
  \(\rho\) 
 کی بڑی قیمتوں کے لیے پے قابو بڑھتا ہے۔ مثبت قوت نما وہی متاکربی رویا دیتا ہے جو ہمیں مساوات ۴.۵۷ میں نہیں چاہیے تھا۔ حقیقت میں متاکربی حل بھی رداسی مساوات کے جائز حل ہے البتہ ہم ان میں دلچسپی نہیں رکھتے ہے چونکہ یہ معمول پر نہیں لائے جاسکتے ہیں۔ اس علمیہ سے نجات کا صرف ایک ہی راستہ ہے ، تسلسل کو کہیں نہ کہیں ختم ہونا ہوگا لازمی طور پر ایسا زیادہ سے زیادہ عادت سے ہی J بلند تر پایا جائیگا جہاں درج ذیل ہوگا۔
 \[c_{j_{\text{max}}+1}=0\]
 یوں کلیہ توالی کے تحت باقی تمام عددی سر ۰ ہونگے ظاہر ہے مساوات ۴.۶۳ کے درج ذیل ہوگا
 \[2(j_{\text{max}}+l+1)-\rho_{0}=0\]
 صدر کوانٹم 
 \[n\equiv j_{\text{max}}+l+1\]
 متعارف کرتے ہوئے درج ذیل ہوگا۔
 \[\rho_{0}=2n\]
 اب E کو 
 \(\rho_{0}\)
 تعین کرتا ہے مساوات ۴.۵۴ اور ۴.۵۵
 \[E=\frac{-\hbar^{2}k^{2}}{2m}=\frac{-me^{4}}{8\pi^{2}\epsilon^{2}\hbar^{2}\rho^{2}}\]
 لہٰذا اجزاتی توانائیاں درج ذیل ہونگی 
 \[\boxed{E_{n}=-\big[\frac{m}{2\hbar^{2}}\big(\frac{e^{2}}{4\pi\epsilon}\big)^{2}\big]\frac{1}{n^{2}}=\frac{E_{1}}{n^{2}}\quad n=1,2,3,\dotsc}\]
 یہ مشہور زمانہ کلیہِ بہر ہے جو غالباً پورے کوانٹم میکانیات میں  سب سے اہم ترین نتیجہ ہے جناب بہر نے سن۱۹۱۳ میں یہ کلیہ کوانٹم میکانیات سے قبل تقریبن اندازے سے اخذ کیا۔ شروڈنگر مساوات سن ۱۹۲۴ میں منظر پر آی مساوات  ۴.۵۵ اور ۴.۶۸ کو ملا کر درج ذیل حاصل ہوگا۔
\[k=\big(\frac{me^{2}}{4\pi\epsilon_{0}\hbar^{2}}\big)\frac{1}{n}=\frac{1}{an}\]
جہاں
\[\boxed{a\equiv\frac{4\pi\epsilon_{0}\hbar^{2}}{me^{2}}=0.529\times 10^{-10}\text{m}}\]
رداس بہر کہلاتا ہے۔ یوں مساوات ۴.۵۵ کے تحت درج ذیل ہوگا۔
\[\rho=\frac{r}{an}\]
%150-153
ہائیڈروجن جوہر لے فضائی تفاعل  امواج کو 3 کوانٹم اعداد M اور N٫L سے نام دیا جاتا ہے 
 \[\psi_{n,l,m}(r,\theta,\phi)=R_{nl}(r)Y_{l}^{m}(\theta,\phi)\]
 جہاں مساوات 4.36 اور 4.60 کو دیکھتے ہوئے
 \[R_{n,l}(r)=\frac{1}{r}\rho^{l+1}e^{-\rho}v(\rho)\] 
 ہوگا. جبکہ وی رو متغير رو میں جے بلندتر
\(j_{max}=n-l-1\)
درجہ کا كثير رکنی ہوگا جس کے عددی سر جنہے معمول پر لانا باقی ہوگا  درجہ ذیل کلیہ توالی دے گا
 \[c_{j+1}=\frac{2(j+l+1-n)}{(j+1)(j+2l+2)}c_{j}\]
كم سے كم توانائی کا حال جسے زمینی حال کہتے ہیں٬ کے لیے
 \(n=1\) 
ہوگا. تبی مستقلوں کی یہ قیمتیں پر کرتے ہوئے درجہ ذیل حاصل ہوگا
 \[\boxed{E_{1}=-\big[\frac{m}{2\hbar^{2}}\big(\frac{e^{2}}{4\pi\epsilon}\big)^{2}\big]=13.6\text{eV}}\]
. اس سے ظاہر ہے کہ  ہائیڈروجن کی توانائی بھندن 13.6 eV ہے. یہ وہ توانائی ہے جو زمینی حال میں الیکٹرون کو محیا کرنے سے ایٹم بدارا بن جائے گا. مساوات 4.67 کے تحت l=0  لہذا m=0  ہوگا. مساوات 4.29 دیکھے یوں درجہ ذیل ہوگا
 \[\psi_{100}(r,\theta,\phi)=R_{10}(r)Y_{0}^{0}(\theta,\phi)\]
 کلیہ توالی پہلے جز پرشی رکھ جاتا ہے. j=0 کے لئیے مساوات 4.76 سے
 \(c_{1}=0\) 
 حاصل ہوگا. یوں وی رو ایک مستقل 
  \(c_{0}\)
  ہوگا. لہذا درجہ ذیل ہوگا
   \[R_{10}(r)=\frac{c_{0}}{a}e^{-r/a}\]
  . اس کو مساوات 4.31 کے تحت معمول پر لانے
سے
..\[\int_{0}^{\infty}\abs{R_{10}}^{2}r^{2}\dif{r}=\frac{c_{0}^{2}}{a^{2}}\int_{0}^{\infty}e^{-2r/a}r^{2}\dif{r}=\abs{c_{0}}^{2}\frac{a}{4}=1\]
حاصل ہوگا. ساتھ ہی
\(c_{0}=2/\sqrt{a}\)
لہذا بائیڈروجن کا زمینی حال درجہ ذیل ہوگا
\[\psi_{100}(r,\theta,\phi)=\frac{1}{\sqrt{\pi a^{3}}}e^{-r/a}\]
اسی طرح n=2  کے لئے توانائی درجہ ذیل ہوگی
\[E_{2}=\frac{-13.6\text{eV}}{4}=-3.4\text{eV}\]
چونکہ
 \(m=2\) 
کے لئے یا 
\(l=0\) 
ہوگا جو
\( m=0\) 
دیتا ہے يا
 \(l=1\) 
ہوگا جو 1 اور M= -1,0 دیتا ہے لہذا چار مختلف حالات کی توانائی E2 ہوگی. کلیہ توالی مساوات 4.76  L=0 کے لیئے درجہ ذیل دے گا:
\[c_{1}=-c_{0} (j=0) \quad c_{2}=0 (j=1)\]
لہذا
\(v(\rho)=c_{0}(1-\rho)\)
اور درجہ ذیل ہونگے
 \[R_{20}(r)=\frac{c_{0}}{2a}\big(1-\frac{r}{2a}\big)e^{-r/2a}\]
دیہان رہے کہ مختلف کوانٹم اعداد l اور N کے لئے پھیلاؤ کے عددی سر
 \(c_{j}\)
 مکمل تور پر مختلف ہونگے. کلیہ توالی
  \(l= 1\)
  کی صورت میں پہلے جز پر اختتام پذیر ہوگا. وی رو ایک مستقل ہوگا لہذا درجہ ذیل حاصل ہوگا.
   \[R_{21}(r)=\frac{c_{0}}{4a^{2}}re^{-r/2a}\]
ہر ایک صورت میں معمول پر لانے سے
 \(c_{0}\) 
تعین ہوگا.\\
سوال 4.11 دیکھے. کسی بھی n کے لئے مساوات 4.67 کی بلا تضاد L کی ممکنہ قیمتیں درجہ ذیل ہوں گی 
\[0,1,2\dotsc n-1\]
جبکہ ہر
 \(l \)
کے لئیے 
\(m\)
 کی ممکنہ قیمتوں کی تعداد
 \( 2l+1\) 
 ہوگی. (مساوات 4.29) . لہذا
  \(E_{n}\)
  سطح کی توانائی کے لئے کل انحطاطيت درجہ ذیل ہوگی.
\[d(n)=\sum_{l=0}^{n-1}(2l+1)=n^{2}\]
كثير رکنی وی رو جو مساوات 4.76 سے حاصل ہو گی ایک ایسا تفاعل ہے جس سے عملی ریاضی دان بخوبی واقف ہیں. ما سوائے معمول زنی کے اسے تجذيل لکھا جاسکتا ہے.
 \[v(\rho)=L_{n-l-1}^{2l+1}(2\rho)\]
 جہاں
  \[L_{q-p}^{p}(x)\equiv(-1)^{p}\big(\frac{\dif}{\dif{x}}\big)^{p}L_{q}(x)\] 
 ایک شریک لاگیغ  كثير رکنی ہے جبکہ 
\[ L_{q}(x)\equiv e^{x}\big(\frac{\dif}{\dif{x}}\big)^{q}(e^{-x}x^{q})\]
 لاگیغ كثير رکنی ہے. جدول 4.5 میں چند ابتدائی لاگيغ كثور رکنیا پیش کی گئی ہیں.  جبکہ جدول 4.6 میں چند شریک لاگيغ كثير رکنیا پیش کئے گئی ہیں. جدول 4.7 میں چند ابتدائی رداسی تفاعل امواج پیش کئے گئے ہیں جنہیں شکل 4.4 میں ترسیم کیا گیا ہے. ہائیڈروجن کے معمول شده تفاعل امواج درجہ ذیل ہیں.
  \[\boxed{\psi_{nlm}=\sqrt{\big(\frac{2}{na^{3}}\big)\frac{(n-l-1)!}{2n[(n+1)!]^{3}}}e^{-r/na}\big(\frac{2r}{na}\big)^{l}[L_{n-l-1}^{2l+1}(2r/na)]Y_{l}^{m}(\theta,\phi)}\]
یہ تفاعل کچ خوفناک ہیں٬ ليكن شكوه نہ کیجیے گا. یہ اُن چند حقیقی نظاموں میں سے ایک ہے جن کا مکمل حل حاصل کرنا ممکن ہے.  دیہان رہے کہ اگرچہ تفاعل امواج تینوں کوانٹم اعداد پر منحصر ہے جبکہ توانائیاں مساوات 4.70 کو صرف n تعین کرتا ہے. یہ کوولومب( coulomb) توانائی کی ایک خاصیت ہے. آپ کو یاد ہوگا کہ کروی كنواں کی صورت میں توانائیاں L پر منحصر تھی (مساوات 4.50).
تفاعل موج باہمی امودی ہوں گے 
.\[\int\psi_{nlm}^{*}\psi_{nlm}r^{2}\sin{\theta}\dif{\theta}\dif{\phi}=\delta_{nn'}\delta_{ll'}\delta_{mm'}\]
 یہ کروی ہارمونیات کی امودیت مساوات 4.33 کی بنا اور
\(n\neq n'\)
کی صورت میں تفاعلات موج کا H کی الگ تلگ امتیازی اعقدار کے امتیازی تفاعل ہونے کی بنا ہے. \\
پائیڈروجن تفاعلات موج کی تصویر کشی آسان کام نہیں ہے. ماحر کیمیا کسافتی اشکال بناتے ہیں جہاں چمک
\(\abs{\psi}^{2}\)
کا راست متناسب ہوتا ہے (شکل 4.5). ان سے زیادہ معلومات مستقل كسافت کے احتمال کی سطحوں کے اشکال دیتی ہے جنہیں پڑھنا نسبتاً مشکل ہوگا. (شکل 4.6).\\
\end{document}
