\documentclass{book}
\usepackage{polyglossia}             
\setmainlanguage[numerals=maghrib]{arabic}   
\setotherlanguages{english}
\newfontfamily\arabicfont[Scale=1.0,Script=Arabic]{Urdu Typesetting} 
\newfontfamily\urdufont[Scale=1.25,WordSpace=60.0,Script=Arabic]{Urdu Typesetting}
%\setlength{\parskip}{5mm plus 4mm minus 3mm}
\begin{document}
ہم تفعال معاج کے شماریعاتی تشریح ( مساوات 1٫3) پر دوبارہ غور کرتے ہیں ، جس کے تحت لمحہ t پر ایک ذرے کا نقطہ x پر پاےٗ جانے کی قفافت 
$ | \psi (x , t) |^2 $
ہو گی۔  یوں (مساوات 1٫16) کے تحت 
\begin{equation}
\int_{-\infty}^{infty}|\psi(x,t)|^2 = 1
\end{equation}
کے برابر ہو گا۔  چونکہ ذرہ کہی نہ کہی پر تو ضرور پایا جاےٗ گا۔  اس حقیقت کے بغیر شماریعاتی تشریح بے معنی ہو جاتی ہے۔ البتہ یہ شرط آپ کے لیے  پریشان کن  ہونا چاہیے ۔ آخر کار  تفعال موج کو مساوات شروڈنگر سے حاصل کرتے ہیں اور 
$ \psi $
پر بیرونی شراءط مصلط کرنا تب جاءز ہو گا کہ یہ دونو حقاءق کے بیچ اختلاف نہ پایا جاتا ہو ۔
%\newpage
مساوات 1٫1 کو دیکھ کر  آپ دیکھ سکتے ہیں کہ اگر 
$ \psi (x , t) $
اس کا ایک حل ہو تب 
$ A \psi (x , t ) $
اس کا ایک حل ہو گا۔ جہاں A مخلوط مستقل ہو گا۔ اس طرح ہم یہ کر سکتے ہیں کہ نا معلوم  ضربی مستقل کو یوں منتخب کریں کہ یہ مساوات  1٫20 پر پورا اترے  اس عمل کو معمول پر لانا کہتے ہیں۔ مساوات شروڈنگر کے چند حل کا تکمل علامت نہیں ہو گا۔ ایسی صورت میں کوئ بھی ضربی مستقل مساوات 1٫20 کے تکمل کو 1 کے برابر نہیں کر سکتا ۔ یہی کچھ 
$ \psi = 0 $
کے لیے بھی درست ہے ۔ ایسے نا قابل معمول پر لاےٗ جانے والے حل کسی صورت ایک ذرے ظاہر نہیں کر سکتا ہے۔ لہذا انہیں رد کیا جاےٗ گا۔  طبی طور پر پاےٗ جانے والے حل شروڈنگر مساوات کے قابل مربع تکمل ہونگے۔ یہاں رک کر ذرا غور کریں۔ فرض کریں لمحہ
$ t = 0 $
پر میں ایک تفعال معاج کو معمول پر لاتا ہوں۔ مجھے کیسا معلوم ہو گا کہ وقت گزرتے ہوےٗ  جیسے جیسے 
$ \psi $
میں تبدیلی رونما ہو گی۔ یہ تفعال معاج معمول شدہ ہی رہے گا۔ آپ یہ نہیں کر سکتے کہ بار بار تفعال معاج کو معمول پر لائیں چونکہ A اس طرح مستقل نہیں رہے گا  بلکہ T کا تفعال بن جاےٗ گا۔ یوں مساوات شروڈنگر  کا حل نہیں پایا جاےٗ گا۔ خوش قسمتی سے مساوات شروڈنگر کی ایک خاصیت ہے  جس کے تحت یہ تفعال معاج  کی معمول شدہ صورت برقرار رکھتا ہے۔ اس خاصیت کے بغیر مساوات شروڈنگر  اور شماریعاتی تشریح ہم آہنگ نہ ہونگے اور  کوانٹم نظریہ بےمعنی ہو جاےٗ گا چونکہ یہ ایک بہت اہم نقطہ ہے لہذا ہم اس کے ثبوت کو غور سے دیکھتے ہیں۔ ہم درج ذیل مساوات سے شروع کرتے ہیں۔
\begin{equation}
\frac{d}{dt} \int_{-\infty}^{\infty} | \psi (x , t)|^2 dx = \int_{- \infty}^{\infty} \frac{ \partial}{\partial t} | \psi (x , t)|^2 dx 
\end{equation}
اب غور کریں کہ تکمل صرف T  کا تفعال ہے لہذا میں نے قل تفرق 
$ \frac{d}{dt} $
استعمال کیا۔  پہلے بائیں ہاتھ جبکہ دائیں ہاتھ تکمل  تفعال پطعغیر t اور x 


\end{document}
