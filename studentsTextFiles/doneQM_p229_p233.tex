
\ابتدا{سوال}

(الف) مساوات \num{5.59} اور مساوات \num{5.63} استعمال کرتے ہوئے دیکھائیں کہ دوری ڈیلٹا تفاعل مخفیہ میں ایک ذرے کی تفاعل موج درج ذیل روپ میں لکھی جا سکتی ہے 
\begin{align*}
	\psi(x) = C[\sin(kx)+e^{-iKa}\sin k(a-x)], (0\leq x\leq a).
\end{align*} 
معمولزنی مستقل \عددی{C} تعین کرنے کی ضرورت نہیں ہے۔

(ب)  البتہ پٹی کے بالائی سر پر جہاں \عددی{z} $\pi$ کا عدد صحیح مضرب ہوگا شکل \num{5.6} (الف) سے \(\psi(x) = 0\) حاصل ہوگا ایسی صورت میں درست تفاعل موج تلاش کریں دیکھیئے گا کہ ہر ایک ڈیلٹا تفاعل پر \(\psi\) کو کیا ہوتا ہے؟
\انتہا{سوال}
\ابتدا{سوال}
پہلی اجازتی پٹی کے نچھلے نقطہ پر \(\beta = 10\) کی صورت میں توانائی کی قیمت تین با معنی ہندسوں تک تلاش کریں۔ دلائل پیش کرتے ہوئے آپ فرض کر سکتے ہیں کہ\(\frac{\alpha}{a} = \SI{1}{\electronvolt}\) ہو گا۔
\انتہا{سوال}
\ابتدا{سوال}
فرض کریں ہم ڈیلٹا تفاعل سولن کے بجائے ڈیلٹا تفاعل کنواں پر غور کر رہے ہیں یعنی مساوات \num{5.57} میں \(\alpha\) کی علامت تبدیل کریں۔ ایسی صورت میں شکل \num{5.6} اور \num{5.7} کی طرح کے شکال بنائیں۔ مثبت توانائی حلوں کے لیئے آپ کو کوئی نیا حساب کرنے کی ضرورت نہیں ہے بس مساوات \num{5.66} میں موضوع تبدیلیاں لائیں لیکن منفی توانائی حلوں کے لیئے آپ کو کام کرنا ہوگا اور انہیں ترسیم پر شامل کرنا مت بھولیئے گا جو اب \(-z\) تک وسیع ہوگا۔ پہلی اجازتی پٹی میں اب کتنے حالات ہونگے؟
\انتہا{سوال}
\ابتدا{سوال}
دیکھائیں کہ مساوات \num{5.64} میں حاصل زیادہ تر توانائیاں دوہری انحطاطی ہے۔ کن صورتوں میں ایسا نہیں ہے؟ اشارہ: \((N=1, 2, 3, 4, \dots)\) لیتے ہوئے دیکھیئے گا کیا ہوتا ہے۔ ایسی ہر صورت میں \(\cos(Ka)\) کی کیا ممکنا قیمتیں ہوں گی؟
\انتہا{سوال}
\حصہ{کوانٹم شماریاتی میکانیات}
مطلق صفر حرارت پر ایک طبی نظام اپنے کم سے کم اجازتی توانائی تنظیم کا مکین ہوگا۔ درجہ حرارت بڑھاتے ہوئے بلا منصوبہ حراری سرگرمیوں کے بنا ہیجانی حالات ابھرنے شروع ہونگے جس سے درج ذیل سوال پیدا ہوتا ہے: اگر \عددی{T} درجہ حرارت پر حراری توازن میں ایک بڑی تعداد \عددی{N} کے ذرات پائے جاتے ہوں تب اسکا کیا احتمال ہے کہ ایک ذرہ جس کو بلا منصوبہ منتخب کیا گیا ہو کی مخصوص توانائی \عددی{E_j}؟ ہوگی دیہان رہے کہ اس احتمال کا کوانٹم عدم تعین کے ساتھ کوئی تعلق نہیں ہے بلکل یہی سوال کلاسیکی شماریاتی میکانیات میں بھی کھڑا ہوتا ہے۔ ہمیں احتمالی جواب اس لیئے منظور ہوگا کہ جن ذرات کی ہم بات کر رہے ہیں انکی تعداد اتنی بڑی ہوگی کہ یہ کسی صورت ممکن نہیں ہوگا کہ ہم ہر ایک پر علیحدہ علیحدہ نظر رکھ سکیں چاہے یہ قابلِ تعین ہو یا نہ ہوں۔

شماریاتی میکانیات کا بنیادی مفروضہ یہ ہے کہ حراری توازن میں ہر وہ منفرد حال جس کی ایک جیسی کل توانائی \عددی{E} ہو ایک جتنا معتمل ہوگا۔ بلا واسطہ حراری حرکتوں کی بنا مستقل طور پر توانائی ایک ذرہ سے دوسرا ذرہ ایک روپ حرکی، گردشی، گھومتی وغیرہ سے دوسری روپ میں منتقل ہوگی لیکن بیرونی مداخلت کی عدم موجودگی میں بقاء توانائی کی بنا کل مقررہ ہوگا۔ یہاں مفروضہ یہ ہے  کہ توانائی کی لگاتار نئی تقسیم کسی مخصوص حال کو ترجیح نہیں دیتا ہے۔ یہ ایک گہرا مفروضہ ہے جو سوچنے کے قابل ہے درجہ حرارت \عددی{T} حراری توازن میں ایک نظام کی کل توانائی کی بس پیمائش ہے۔ ان منفرد حالات کی گنتی میں کوانٹم میکانیات ایک نئی پیچیدگی پیدا کرتی ہے لیکن چونکہ حالات غیر مسلسل ہیں لحاظہ یہ کلاسیکی نظریہ سے زیادہ آسان ہے اور اسکا فیصلہ کن انحصار اس بات پر ہوگا کہ یہ ذرات قابلِ ممیز، یکساں بوزان یا یکساں فرمیون ہیں۔ ان کے دلائل نسبتاً سیدھے لیکن ریاضی کافی گہری ہے لحاظہ میں ایک انتہائی سادھا مثال سے شروع کروں گا تاکہ آپ بنیادی حقائق سمجھ سکیں۔
\جزوحصہ{ایک مثال} 
فرض کریں ہمارے پاس یک بعدی لامتناہی چکور کنواں حصہ\num{2.2} میں کمیت \عددی{m} کے صرف تین باہم غیر متعمل ذرات پائے جاتے ہیں۔ ان کی کل توانائی درج ذیل ہوگی ماساوات \num{2.27} دیکھیں
\begin{align}
	E = E_A + E_B + E_C = \frac{\pi^2 \hbar ^2}{2ma^2}(n^2_A + n^2_B + n^2_C)
\end{align}
جہاں \(n_A\)، \(n_B\) اور \(n_C\) مثبت عدد صحیح ہوں گے۔ اب تبصرہ جاری رکھنے کی خاطر فرض کریں کہ\(E=363(\frac{\pi ^2 \hbar ^2}{2ma^2})\) یعنی درج ذیل
\begin{align}
	n^2_A + n^2_B + n^2_C = 363.
\end{align}  
جیسے آپ تصدیق کرسکتے ہیں ہمارے پاس تین مثبت عدد صحیح اعداد کے تیرہ ایسے ملاپ  پائے جاتے ہیں جن کے مربعوں کا مجموعہ \num{363} ہوگا: تینوں اعداد گیارہ ہوسکتے ہیں دو اعداد تیرہ  اور ایک پانچ جو تین مرتب اجتماعات میں ہوگا ایک عدد اُٗنّیس اور دو ایک یہاں نھی تین مرتب اجتماعات میں یا ایک عدد سترہ ایک ساٹھ اعر ایک پانچ چھ مرتب اجتماعات میں ہوسکتے ہیں۔ یوں \(n_A, n_B, n_C\)  درج ذیل میں سے ایک ہوگا:
\begin{align*}
	(11, 11, 11)\\
	(13, 13, 5), (13, 5, 13), (5, 13, 13)\\
	(1, 1, 19), (1, 19, 1), (19, 1, 1)\\
	(5, 7, 17), (5, 17, 7), (7, 5, 17), (7, 17, 5), (17, 5, 7), (17, 7, 5).
\end{align*}
اگر یہ ذرات قابلِ ممیز ہوں تب ان میں سے ہر ایک کسی ایک منفرد کوانٹم حال کو ظاہر کرے گا اور شماریاتی میکانیات کے بنیادی مفرضہ کے تحت حراری توزن میں یہ سب برابر محتمل ہوں گے۔ لیکن میں اس میں دلچسپی نہیں رکھتا ہوں کہ کونسا ذرہ کس یک ذرہ حال میں پایا جاتا ہے بلکہ میں یہ جاننا چاہتا ہوں کہ ہر ایک حال میں کل کتنے ذرات پائے جاتے ہیں حال\(\psi_n\) کی تعداد مکین \(N_n\)۔ ہم اس دن ذرہ حال کے تمام تعدادِ مکین کے اجتماع کو تنظیم کہتے ہیں۔ اگر تینوں حال \(\psi_{11}\)  میں ہوں تب تنظیم درج ذیل ہوگا
\begin{align}
	(0, 0, 0, 0, 0, 0, 0, 0, 0, 0, 3, 0, 0, 0, 0, 0, 0, 0, \dots)
\end{align} 
یعنی \(N_{11}=3\) باقی تمام صفر اگر دو حال \(\psi_{13}\) میں اور ایک \(\psi_5\) میں ہو تب تنظیم درج ذیل ہوگا
\begin{align}
	(0, 0, 0, 0, 1, 0, 0, 0, 0, 0, 0, 0, 2, 0, 0, 0, 0, \dots)
\end{align}  
یعنی \(N_5=1, N_{13}=2\) باقی تمام صفر اگر دو \(\psi_1\) میں ایک \(\psi_{19}\) میں تب تنظیم درج ذیل ہوگا
\begin{align}
	(2, 0, 0, 0, 0, 0, 0, 0, 0, 0, 0, 0, 0, 0, 0, 0, 0, 0, 1, 0, \dots)
\end{align} 
یعنی\(N_1 = 2, N_{19} = 1\) باقی تمام صفر اور اگر ایک ذرہ \(\psi_5\) میں ایک \(\psi_7\) میں اور ایک \(\psi_{17}\) میں تب تنظیم درج ذیل ہوگا 
\begin{align}
	(0, 0, 0, 0, 1, 0, 1, 0, 0, 0, 0, 0, 0, 0, 0, 0, 1, 0, 0, \dots)
\end{align} 
یعنی \(N_5 = N_7 = N_{17} = 1, \text{\RL{باقی تمام صفر}}\) ان تمام میں آخری تنظیم زیادہ سے زیادہ محتمل ہوگی چونکہ اسکو چھ مختلف طریقوں سے حاصل کیا جاسکتا ہے جبکہ درمیانی دو کو تین طریقوں سے اور پہلی کو صرف ایک طریقہ سے حاصل کیا جاسکتا ہے۔

میں اب دوبارہ اپنے اصل سوال پر آتا ہوں کہ بلا واسطہ تین ذرات منتخب کرتے ہوئے کوئی مخصوص اجازتی توانائی \عددی{E_n} حاصل کرنے کا احتمال \عددی{P_n} کیا ہوگا؟ توانائی \عددی{E_1} صرف اس صورت حاصل ہوگا جب ذرہ تیسری تنظیم مساوات \num{5.71} میں ہو اس تنظیم میں نظام ہونے کا اتفاق تیرہ میں سے تین ہے اور اس تنظیم میں \عددی{E_1} کے حصول کا احتمال \(\frac{2}{3}\) لحاظہ\(P_1 =(\frac{3}{13})\times (\frac{2}{3})= \frac{2}{13}\)۔ آپ \عددی{E_5} کو تنظیم دو مساوات \num{5.70} تیرہ میں سے تین کا امکان جس کا احتمال \(\frac{1}{3}\) 	یا تنظیم چار مساوات \num{5.72} تیرہ میں سے چھ امکان اور احتمال \(\frac{1}{3}\) لحاظہ\(P_5 = (\frac{3}{13})\times(\frac{1}{3}) + (\frac{6}{13})\times(\frac{1}{3}) = \frac{3}{13}\)۔ آپ \عددی{E_7} کو صرف چار سے حاصل کرسکتے ہیں لحاظہ\(P_7 = (\frac{6}{13})\times(\frac{1}{3}) = \frac{2}{13}\)۔ اسی طرح \عددی{E_{11}} صرف پہلی تنظیم سے مساوات \num{5.69} سے تیرہ میں سے ایک امکان اور احتمال ایک کے ساتھ حاصل ہوگا لحاظہ\(P_{11} = (\frac{1}{13})\) ہوگا۔ اسی طرح \(P_{13} = (\frac{3}{13})\times(\frac{2}{3}) =\frac{2}{13}\)، \(P_{17} = (\frac{6}{13})\times(\frac{1}{3}) = \frac{2}{13}\) اور \(P_{19} = (\frac{3}{13})\times(\frac{1}{3}) = \frac{1}{13}\) ہوگا۔ انکی تصدیق درج ذیل سے ہوگی 
\begin{align*}
	P_1 + P_5 + P_7 + P_{11} + P_{13} + P_{17} + P_{19} = \frac{2}{13} + \frac{3}{13} + \frac{2}{13} + \frac{1}{13} + \frac{2}{13} + \frac{2}{13} + \frac{1}{13} = 1.
\end{align*} 

یہ قابلِ ممیز ذرات کے لیئے تھا۔ اس کی بجائے اگر ذرات یکساں فرمیان ہوتے اپنی آسانی کے لیئے چکر کع نظراندا کرتے ہوئے یا اگر آپ چاہیں تو یہ تصور کرتے ہوئے کہ تمام ایک جیسے چکر حال میں ہیں ضرورت خلاف تشاکلیت کی بنا پہلی تین تنظیم جو دو یا اس سے بھی برا تین ذرات کع ایک ہی حال میں ڈالتے ہیں خارجل امکان ہوں گے لحاظہ چوتھی تنظیم میں صرف ایک حال ہوگا سوال \num{5.22} الف دیکھیں۔ یکساں فرمیونز کے لیئے \(P_5 = P_7 = P_{17} = \frac{1}{3}\) ہوگا اور اب بھی احتمالات کا مجموعہ ایک ہے اس کے برعکس اگر ذرات یکساں بوزان ہوتے تب ضرورت تشاکلیت ہر تنظیم میں صرف ایک حال کی اجازت دیتا سوال \num{5.22} ب دیکھیں۔ لحاظہ\(P_1 = (\frac{1}{4})\times(\frac{2}{3}) = \frac{1}{6}\)، \(P_5 = (\frac{1}{4})\times(\frac{1}{3}) + (\frac{1}{4})\times(\frac{1}{3}) = \frac{1}{6}\)، \(P_7 = (\frac{1}{4})\times(\frac{1}{3}) = \frac{1}{12}\)، \(P_{11} = (\frac{1}{4})\times(1) = \frac{1}{4}\)،\(P_{13} = (\frac{1}{4})\times(\frac{2}{3}) = \frac{1}{6}\)، \(P_{17} = (\frac{1}{4})\times(\frac{1}{3}) = \frac{1}{12}\) اور \(P_{19} = (\frac{1}{4})\times(\frac{1}{3}) = \frac{1}{12}\) ہوتا۔ ہمیشہ کی طرح احتمالات کا مجموعہ ایک ہے۔

اس مثال کا مقصد آپ کو یہ دیکھانا تھا کہ ذرات کی قسم پر حالات کی شمار کس طرح منحصر ہے۔ ایک لحاظ سے ایک حقیقی صورتِحال سے جہاں \عددی{N} ایک بہت بڑا عدد ہوگا سے یہ مثال زیادہ پیچیدہ تھا۔ چونکہ \عددی{N} کی قیمت بڑھانے سے زیادہ محتمل تقسیم جو قابلِ ممیز ذرات کے لیئے اس مثال میں \(N_5 = N_7 = N_{17} = 1\) ہے پائے جانے کا امکان اتنا زیادہ ہوجائے گا کہ کسی بھی شماریاتی نقطہ نظر سے باقی تمام امکانات کو رد کیا جا سکتا ہے۔ توازن کی صورت میں انفرادی ذرہ توانائیوں کی تقسیم درحقیقت انکی زیادہ سے زیادہ محتمل تنظیم میں تقسیم ہے۔ اگر یہ \(N = 3\)  کے لیئے درست ہوتا جوکہ یہ نہیں ہے ہم قابلِ ممیز ذرات کے لیئے \(N = 3\) کی صور میں اخذ کرتے \(P_5 = P_7 = P_{17} = \frac{1}{3}\) میں حصہ 5.4.3 میں اس نقطہ پر دوبارہ آؤں گا لیکن اس سے پہلے گنتی کی ترکیب کو عمومیت دیتے ہیں۔

\ابتدا{سوال}

(الف) حال \(\psi_5\) میں ایک حال \(\psi_7\) میں ایک اور حال \(\psi_{17}\) میں ایک یکساں تین فرمیون کا مکمل خلاف تشاکل تفاعل موج \(\psi(x_A, x_B, x_C)\) تیار کریں۔

(ب) تین یکساں بوزان کے لیئے مکمل تشاکل تفاعل موج\(\psi(x_A, x_B, x_C)\) درج ذیل صورتوں میں تیار کریں (ا) تینوں حال \(\psi_{11}\) میں ہوں، (ب) اگر دو \(\psi_1\) اور ایک \(\psi_{19}\) میں ہو، (ج) اگر ایک حال \(\psi_5\) ایک حال \(\psi_7\) اور ایک حال \(\psi_{17}\) میں ہو۔ 
\انتہا{سوال}
\ابتدا{سوال}
فرض کریں یک بُعدی حارمونی ارتعاشی مخفیہ میں آپ کے پاس تین باہم غیر متعمل ذرات ہیں جو حراری توازن میں پائے جاتے ہیں جن کی کل توانائی\(E = (\frac{9}{2})\hbar\omega\) ہے۔

(الف) اگر یہ تمام ایک جیسی کمیت کے قابلِ ممہز ذرات ہوں تب انکی کتنی عدد مکین تنظیمات ہوں گے اور ہر ایک کے لیئے کتنے منفرد تین ذرہ حالات ہوں گے؟ سب سے زیادہ محتمل تنظیم کیا ہوگی؟ اگر آپ ایک ذرہ بلا منصوبہ منتخب کریں اور اسکی توانائی کی پیمائش کریں تب کیا قیمتیں متوقع ہوں گی؟ اور ہر ایک کا احتمال کیا ہوگا؟ سب سے زیادہ محتمل توانائی کیا ہوگی؟

(ب) یہی کچھ یکساں فرمیونز کے لیئے کریں چکر کو نظر انداز کریں جیسا ہم نے حصہ 5.4.1 میں کیا۔

(ج) یہی کچھ یکساں بوزان کے لیئے کریں چکر کو نظرانداز کریں۔ 
\انتہا{سوال}

