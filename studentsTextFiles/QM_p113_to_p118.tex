\جزوحصہ{توانائی و وقت عدم یقینیت کا اصول}
مقام و معیار حرکت عدم یقینیت کے اصول کو عموماً درج ذیل روپ میں لکھا جاتا ہے.
\begin{align}
\Delta x \Delta p \geq \frac{\hbar}{2} ;
\end{align}
یکساں تیار کردہ نظام کی بار باد پیمائش کے نتائج کے معیاری انحراف کو بعض اوقات لاپرواہی سے \عددی{  \Delta x  } لکھا جاتا ہے جو ایک کمزور علامت ہے. مساوات \حوالہ{    3.96} کی طرح کا توانائی و وقت عدم یقینیت کا اصول درج ذیل ہے.
\begin{align}
\Delta t \Delta E \geq \frac{\hbar}{2}
\end{align}
چونکہ خصوصی نظریہ اضافت کی مقام و وقت چار سمتیات میں \عددی{ x   } اور \عددی{ t   } اکٹھے بلکہ \عددی{   ct } شامل ہوتے ہیں. جبکہ توانائی کو معیار حرکت چار سمتیات میں \عددی{  p  } اور \عددی{ E   } اکٹھے بلکہ \عددی{  E/c  } شامل ہوتے ہیں. لحظہ خصوصی نظریہ اضافت کے نقطہ نظر سے توانائی و وقت کے روپ کو مقام و معیار حرکت روپ کا نتیجہ تصور کیا جا سکتا ہے. یوں نظریہ اضافت میں مساوات \حوالہ{   3.70 } اور مساوات \حوالہ{   3.69 } ایک دوسرے کیلئے لازم و ملزوم ہیں. لیکن ہم اضافیت کوانٹم میکانیات نہیں کر رہے ہیں. شروڈنگر مساوات صریحاً غیر اضافی ہے. یہ \عددی{   t    } اور \عددی{   x    } کو ایک جیسی اہمیت نہیں دیتی ہے. بطور تفرقی مساوات کے یہ \عددی{    t   } میں یک رتبی جبکہ \عددی{    x    } میں دو رتبی ہے. اور مساوات \حوالہ{    3.70} از خود مساوات \حوالہ{   3.69 } نہیں دیتی ہے. میں اب توانائی و وقت عدم یقینیت کا اصول اخذ کرتا ہوں اور ایسا کرتے ہوئے کوشش کروں گا کہ آپ کو مطمئن کروں کہ مقام و معیار حرکت کے عدم یقینیت اصول کے ساتھ اسکی ظاہری مشابہت گمراہ کن ہے.
اب مقام، معیار حرکت اور توانائی؟؟؟کی متغیرات ہیں جو کسی بھی وقت پر ایک نظام کے قابل پیمائش خواص ہیں لیکن کم از کم غیر اضافی نظریہ میں وقت ایک حرکی متغیر نہیں ہے. آپ مقام اور توانائی کی پیمائش کا طرح ایک ذرے کا وقت نہیں ناپ سکتے ہیں. وقت ایک غیر تابع متغیر ہے جبکہ حرکی مقداریں اس کی تفاعل ہیں. بالخصوص توانائی و وقت عدم یقینیت کے اصول میں متعدد وقت کی پیمائشوں کی معیاری انحراف کو \عددی{  \Delta t } ظاہر نہیں کرتا ہے. آپ کہہ سکتے ہیں. اور میں جلد اسکی زیادہ درست صورت پیش کروں گا کہ یہ اس وقت کو ظاہر کرتا ہے جس میں نظام معقول حد تک تبدیل ہوتا ہے. یہ دیکھنے کیلئے کہ نظام کتنی تیزی سے تبدیل ہوتا ہے, آئیں ہم کسی قابل مشاہدہ \عددی{ Q(x,p,t) } کی توقعاتی قیمت کی وقت کے لحاظ سے تفرق کا حساب لگائیں. 
\begin{align*}
\frac{d}{dt} \langle Q \rangle = \frac{d}{dt} \langle \Psi | \hat{Q} \Psi \rangle = \left\langle \frac{\partial\Psi}{\partial t} \vert \hat{Q} \Psi \right\rangle + \left\langle \Psi \vert \frac{\partial \hat{Q}}{\partial t} \Psi \right\rangle + \left\langle \Psi \vert \hat{Q} \frac{\partial \Psi}{\partial t} \right\rangle .
\end{align*}
اب شروڈنگر مساوات درج ذیل کہتی ہے. جہاں \عددی{  H = p^{2}/2m + V } ہیملٹونی ہے. 
\begin{align*}
i \hbar \frac{\partial \Psi}{\partial t} = \hat{H} \Psi
\end{align*}
یوں درج ذیل ہو گا. 

اب \عددی{ \hat{H}} ہرمیشی ہے, لہذا \عددی{  \langle \hat{H} \Psi | \hat{Q} \Psi \rangle = \langle \Psi \hat{H} \hat{Q} \Psi \rangle  } ہو گا. لہذا درج ذیل ہو گا.
\begin{align}
\frac{d}{dt} \langle Q \rangle = \frac{i}{\hbar} \langle [ \hat{H} , \hat{Q} ] \rangle + \left\langle \frac{\partial \hat{Q}}{\partial t} \right\rangle .
\end{align}
یہ اپنے طور پر ایک دلچسپ اور کار آمد نتیجہ ہے. سوال \حوالہ{ 3.17 } اور \حوالہ{ 3.31} دیکھیں. اس صورت میں جب عامل صریحاً وقت کا تابع نہ ہو یہ کہتی ہے کہ توقعاتی قیمت کو حامل اور ہیملٹونی کا تبادل کار تعین کرتا ہے. بالخصوص اگر \عددی{ \hat{H}} اور { \عددی{  \hat{Q} } آپس میں قابل تبدل ہوں تب \عددی{ \langle Q \rangle  } مستقل ہو گا. اور اس نقطہ نظر سے \عددی{Q } بقائی مقدار ہو گا.
اب فرض کریں ہم عمومی عدم یقینیت کے اصول مساوات \حوالہ{ 3.62} میں \عددی{ A=H} اور \عددی{ B=Q} لیں. اور فرض کریں کہ \عددی{ Q } صریحاً \عددی{ t } کا تابع نہیں ہے.
\begin{align*}
\sigma_{H}^{2} \sigma_{Q}^{2} \geq \left( \frac{1}{2i} \langle [ \hat{H}, \hat{Q} ] \rangle \right)^{2} = \left( \frac{1}{2i} \frac{\hbar}{i} \frac{d \langle Q \rangle}{dt} \right)^{2} = \left( \frac{\hbar}{2} \right)^{2} \left( \frac{d \langle Q \rangle}{dt} \right)^{2} .
\end{align*}
جسے صریحاً درج ذیل لکھا جا سکتا ہے. 
\begin{align*}
\sigma_{H}^{2} \sigma_{Q}^{2} \geq \frac{\hbar}{2} \left| \frac{d\langle Q \rangle}{dt} \right| .
\end{align*}
آئیں \عددی{ \Delta E } اور \عددی{  \Delta t } کی تعریفات درج ذیل لیں. 
\begin{align}
\Delta E &\equiv \sigma_{H} \\
\Delta t &\equiv \frac{\sigma_{Q}}{|d\langle Q \rangle / dt} .
\end{align}
تب درج ذیل ہو گا. 
\begin{align}
\Delta E \Delta t \geq \frac{\hbar}{2}
\end{align}
جو توانائی و وقت کی عدم یقینیت کا اصول ہے. یہاں \عددی{  \Delta t } کی معنی کو دھیان دیں. چونکہ 
\begin{align*}
\sigma_{Q} = \left| \frac{d \langle Q \rangle}{dt} \right| \Delta t ,
\end{align*}
لہذا \عددی{\Delta t } اتنے وقت کو ظاہر کرتا ہے جتنے میں \عددی{ Q } کی قیمت ایک معیاری انحراف کے برابر تبدیل ہو. بالخصوص \عددی{  \Delta t } اس قابل مشاہدہ \عددی{  Q } پر منحصر ہو گی حس پر آپ غور کر رہے ہوں. کسی ایک قابل مشاہدہ کی تبدیلی بہت تیز ہو سکتی ہے جبکہ دوسرے کی بہت آہستہ ہو سکتی ہے. لیکن بہت چھوٹی \عددی{  \Delta E } کی صورت میں تمام قابل مشاہدہ کی تبدیلی کی شرح بہت آہستہ ہو گی. اس کو یوں بھی بیان کیا جا سکتا ہے کہ اگر ایک قابل مشاہدہ بہت تیزی سے تبدیل ہو تب توانائی میں عدم یقینیت بہت زیادہ ہو گی.
مثال: ساکن حال کی انتہائی صورت میں جہاں توانائی یکتا طور پر معین ہے تمام توقعاتی قیمتیں وقت کے لحاظ سے مستقل ہوں گی \عددی{\Delta E = 0 \Rightarrow \Delta t = \infty } جیسا ہم نے مساوات \حوالہ{ 2.9}میں دیکھا ۔کچھ ہونے کے لیے ضروری ہے کہ کم از کم دو ساکن حالات کا خطی جوڑ لیا جائے مثلاً 
\begin{align*}
\Psi (x,t) = a\psi_{1}(x)e^{-iE_{1}t/\hbar} + b\psi_{2}(x)e^{-iE_{2}t/\hbar}
\end{align*}
اگر \عددی{a, \, b, \, \psi_{1}}اور \عددی{ \psi_{2}}حقیقی  ہوں تب درج ذیل ہو گا
\begin{align*}
| \Psi(x,t)|^{2} = a^{2}(\psi_{1}(x))^{2} + b^{2}(\psi_{2}(x))^{2} + 2a\psi_{1}(x)\psi_{2}(x)\cos \left( \frac{E_{2}-E_{1}}{\hbar} t \right).
\end{align*}
ایک ارتعاش کا دوری عرصہ \عددی{ \tau = 2\pi\hbar/(E_{2}-E_{1}) } ہو گا ۔اندازاً بات کرتے ہوئے \عددی{ \Delta E = E_{2} - E_{1} } اور \عددی{ \Delta t = \tau } لکھ کر درج ذیل لکھ سکتے ہیں
\begin{align*}
\Delta E \Delta t = 2\pi\hbar
\end{align*}
جو \عددی{ \geq \hbar/2 } ہے (بالکل ٹھیک ٹھیک حساب کے لیے  سوال\حوالہ{ 3.18 } دیکھیں).
مثال: کسی ایک مخصوص نکتہ سے آزاد ذرے کی موجی اکٹھ کتنی دیر میں گزرتا ہے شکل \حوالہ{3.1}؟ کیفی طور پر \عددی{ \Delta t = \Delta x/v = m\Delta x/p } ہو گا لیکن \عددی{ E = p^{2}/2m } ۔ہے لہذا \عددی{ \Delta E = p\Delta p/m } ہو گا. یوں درج ذیل ہو گا
\begin{align*}
\Delta E \Delta t = \frac{p\Delta p}{m} \frac{m\Delta x}{p} = \Delta x \Delta p
\end{align*}
جو مقام و معیار حرکت عدم یقینیت کے اصول تحت \عددی{ \geq \hbar /2 } ہے ( بالکل ٹھیک ٹھیک حساب کے لیے سوال \حوالہ{3.19 } دیکھیں)۔
مثال: ذرہ \عددی{ \Delta} تقریباً \عددی{ 10^{-23} } رہنے کے بعد خود بخود ٹکڑے ہو جاتا ہے ۔اس کی کمیت کی تمام پیمائشوں کا مستطیلی ترسیل جرز کی شکل کا قوس دے گا جس کا وسط \عددی{ \SI{1232}{\mega\electronvolt/c^{2}} } ہو گا اور جس کی چوڑائی تقریباً \عددی{ \SI{120}{\mega\electronvolt/c^{2}} } ہو گی (شکل \حوالہ{ 3.2}). ساکن صورت توانائی \عددی{ mc^{2}} کیوں بعض اوقات \عددی{ 1232 } سے زیادہ اور بعض اوقات اس سے کم حاصل ہوتی ہے؟ کیا یہ تجرباتی پیمائش کی حلل کے بنا ہے؟ جی نہیں کیوں کہ 
\begin{align*}
\Delta E \Delta t = \left( \frac{120}{2} MeV \right)  ( 10^{-23} sec = 6 \times 10^{-23} MeV sec
\end{align*}
جہاں \عددی{ \har /2 = 3 \times 10^{-22} MeV sec } ہے ۔یوں کمیت میں پھیلاؤ اتنا ہی کم ہے جتنا عدم یقینیت کا اصول اجازت دیتا ہے اتنی کم عرصہ حیات کے ذرے کی ایک مخصوص کمیت نہیں ہو سکتی ہے۔ ان مثالوں میں ہم نے جز \عددی{ \Delta t } کے کئی مخصوص مطلب دیکھے:مثال \حوالہ{ 3.5} میں اس سے مراد طولی موج تھا؛ مثال \حوالہ{ 3.6} میں اس سے مراد وہ دورانیہ تھا جس میں ایک ذرہ کسی نکتہ سے گزرتا ہے؛ مثال \حوالہ{3.7 } میں یہ ایک غیر مستحکم ذرے کی عرصہ حیات کو ظاہر کرتا ہے البتہ ان تمام میں \عددی{ \Delta t } اس دورانیہ کو ظاہر کرتا ہے جس میں نظام میں ایک معقول تبدیلی رونما ہو۔
عموماً کہا جاتا ہے کہ عدم یقینیت کے اصول کے بنا کوانٹم مکینیات میں توانائی صحیح معنوں میں بقائ نہیں ہے یعنی آپ کو اجازت ہے کہ آپ توانائی \عددی{ \Delta E }  ادھار لے کر وقت \Delta t \approx \hbar /(2\Delta E ) کے اندر واپس کریں آپ جتنی زیادہ توانائی لیں یا توانائی کی بقائ کی جتنی زیادہ خلاف ورزی کریں اتنا وہ دورانیہ کم ہو گا جس پر ایسا کیا جائے۔ اب توانائی و وقت کی عدم یقینیت کے اصول کے کئی جائز مطلب لیے جا سکتے ہیں جن میں یہ شامل نہیں ہے ۔ ہمیں کہیں پر بھی کوانٹم مکینیات توانائی کی بقا کی خلاف ورزی کی اجازت نہیں دیتی ہے اور نہ ہی مساوات \حوالہ{3.74} کے حصول میں کوئی ایسی اجازت شامل کی گئی لیکن حقیقت یہ ہے کہ عدم یقینیت کا اصول انتہائی زیادہ مضبوط ہے ۔ اس کے غلط استعمال کے باوجود نتائج زیادہ غلط نہیں ہوتے ہیں اور یہی وجہ ہے کہ ماہر طبیعیات عموماً ایسا ہی کرتے ہیں ۔
% Problem 3.17
\ابتدا{سوال}
درج ذیل زیل مخصوص صورتوں پر مساوات \حوالہ{3.71 } کی اطلاق کریں
\begin{enumerate}[a.]
\item \عددی{ Q=1 }
\item \عددی{ Q = H }
\item \عددی{ Q = x }
\item \عددی{ Q = p }
\end{enumerate}
 ہر ایک صورت میں مساوات \حوالہ{1.27 }، \حوالہ{ 1.33 }، \حوالہ{1.38 } اور توانائی کی بقا کو مدنظر رکھتے ہوئے نتیجے پر بحث کریں۔
\انتہا{سوال}
%Problem 3.18
\ابتدا{سوال}
معیاری انخراف \عددی{\sigma_{H} }، \عددی{  \sigma_{x}  } اور \عددی{d\langle x \rangle /dt } کی ٹھیک ٹھیک قیمت کا حساب کرتے ہوئے سوال \حوالہ{2.5 } کے تفاعل موج اور قابل مشاہدہ \عددی{x} کے لیے توانائی کو وقت عدم یقینیت کے اصول کو پرکھیں ۔
\انتہا{سوال}
% Problem 3.19
\ابتدا{سوال}
معیاری انخراف \عددی{\sigma_{H} }، \عددی{  \sigma_{x}  } اور \عددی{d\langle x \rangle /dt } کی ٹھیک ٹھیک قیمت کا حساب کرتے ہوئے سوال\حوالہ{2.43} میں آزاد ذرے کی موجی اکٹھ اور قابل مشاہدہ \عددی{x} کے لیے توانائی و وقت عدم یقینیت کے اصول کو پرکھیں ۔
\انتہا{سوال}
% Prolem 3.20
\ابتدا{سوال}
دکھائیں کہ \عددی{x} قابل مشاہدہ کے لیے توانائی و وقت عدم یقینیت کا اصول تخفیف کے بعد سوال \حوالہ{3.14} کی عدم یقینیت کے اصول کا روپ اختیار کرتی ہے۔
\انتہا{سوال}
