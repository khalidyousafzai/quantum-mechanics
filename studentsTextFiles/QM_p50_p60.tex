\documentclass{book}
\usepackage{fontspec}
\usepackage{makeidx}
\usepackage{amsmath}                                                                         %\tfrac for fractions in text
\usepackage{amssymb}    
\usepackage{gensymb}  
\usepackage{amsthm}      						%theorem environment. started using in the maths book
\usepackage{mathtools}
\usepackage{multicol}
\usepackage{commath}									%differentiation symbols
\usepackage{polyglossia}    
\setmainlanguage[numerals=maghrib]{arabic}     %for english numbers use numerals=maghrib, for arabic numerals=arabicdigits
\setotherlanguages{english}

\newfontfamily\arabicfont[Scale=1.0,Script=Arabic]{Jameel Noori Nastaleeq} 
\setmonofont{DejaVu Sans Mono}                                                                  %had to add this and the next line to get going after ubuntu upgrade
\let\arabicfontt\ttfamily                                                                                  %had to add this and the above line to get going after ubuntu upgrade
\newfontfamily\urduTechTermsfont[Scale=1.0,Script=Arabic]{AA Sameer Sagar Nastaleeq Bold}
\newfontfamily\urdufont[WordSpace=1.0,Script=Arabic]{Jameel Noori Nastaleeq}
\newfontfamily\urdufontBig[Scale=1.25,WordSpace=1.0,Script=Arabic]{Jameel Noori Nastaleeq}
\newfontfamily\urdufontItalic[Scale=1.25,WordSpace=1.0,Script=Arabic]{Jameel Noori Nastaleeq Italic}
\setlength{\parskip}{5mm plus 4mm minus 3mm}
\begin{document}

مثال
2.5\\
ہارمونی مرتائش کی
n
حال کی مخفی توانائی کی توقعاتی قیمت تلاش کریں
حل
\begin{align*}
\langle V  \rangle =\big\langle\frac{1}{2}m\omega ^{2}x^{2}\big\rangle=\frac{1}{2}m\omega ^{2}\int_{-\infty}^{\infty}\psi_{n}^{*}x^{2}\psi_{n}\dif{x}\\
\end{align*}
اس قسم کے تکملات جن میں
x
اور
p
کے طافت پائے جاتے ہوں کے حصول کے لیے ایک بہترین طریقہ پایا جاتا ہے .
مساوات
2.47
میں دیے گیے تعریفات استعمال کرتے ہوئے
X
اور
p
کو عامل رفعت اور عامل تکمیل کی صورت میں لکھیں
\begin{align*}
x=\sqrt{\frac{\hbar}{2m\omega}}(a_{+}+a_{-}):\quad p=\iota\sqrt{\frac{\hbar m\omega}{2}}(a_{+}-a_{-})\\
\end{align*}
اس مثال میں ہم
\(x^{2}\)
میں دلچسپی رکھتے ہیں۔
\begin{align*}
x^{2}=\frac{\hbar}{2m\omega}[(a_{+})^{2}+(a_{+}a_{-}+(a_{-}a_{+})+(a_{-})^{2}] \\
\end{align*}
لہذا
\begin{align*}
\langle V \rangle =\frac{\hbar\omega}{4}\int \psi_{n}^{*}\big[(a_{+})^{2}+(a_{+}a_{-}+(a_{-}a_{+})+(a_{-})^{2}\big ]\psi_{n}\dif{x}\\
\end{align*}
اب اگرچه
\((a_{+})^{2}\psi_{n}\)
معمول شدہ نہیں ہے لیکن یہ
\(\psi_{n+2}\)
کو ظاہر کرتا ہے۔ جو
\(\psi\)
نمودی ہے یہی کچھ
\((a_{-})^{2}\psi_{n}\)
کے بارے میں کیا جا سکتا ہے۔ جو
\(\psi_{n-2}\)
کے راست متناسع ہے یوں یہ اجزاء خارج ہو جاتی ہے اور ہم مساوات
2.65
تو استعال کرتے ہوئے باقی دوکی قیمت حاصل کر سکتے ہیں۔
\begin{align*}
\langle V \rangle =\frac{\hbar\omega}{4}(n+n+1)=\frac{1}{2}\hbar\omega\big (n+\frac{1}{2} \big )\\
\end{align*}
جیسا آپ نے دیکھا مخفی توانائی کی توقعاتی قیمت قل توانائی کی بالکل نصف ہے . باقی آدھی قیمت حرکی توانائی ہوگی۔ جیسا ہم بعد میں دیکھیں گے یہ ہارمونی مر تالش کی خا ص قیمت ہے.\\
سوال
2 .10\\
( الف )\\
\(\psi_{2}(x)\)
بنائیں۔\\
( ب )\\
\(\psi_{0}\),\(\psi_{1}\)\(\psi_{2}\)
کا خاقا کھینچیں .\\
( ج )\\
\(\psi_{0}\),\(\psi_{1}\)\(\psi_{2}\)
کو مدیت کی تصدیق  تکمل لے کر سرین کریں .\\
انشاره : اگر آپ تفال کی جفت پن اور تاک پن کو بروکار لائیں تو حقیقتاً صرف ایک تکمل حاصل کرنا ہوگا .\\
سوال
2 11\\
(الف )\\
حال
\(\psi_{0}\)
مساوات
2.59
اور حال
\(\psi_{1}\)
مساوات
2.62
کا سر بہن تکمل لے کر
\(\langle x \rangle , \langle x^{2} \rangle , \langle p \rangle \)
اور
\(\langle p^{2} \rangle \)
کی قیمین معلوم کریں .
ہارمونی متائش کے مسائل میں ایک نیا متغیر
\(\xi\equiv\sqrt{m\omega/\hbar}x\)
اور مستقل
\(\alpha\equiv (m\omega/\pi\hbar)^{1/4}\)
کی طاقت
1 /4
لیتے ہوئے مسئلہ سادہ صورت اختیار کرتا ہے .\\
( ب) . کے حصول کو پر کھیں\\
( ج ) ان حالات کے لیے اوسط رکی توانائی
\(\langle T \rangle\)
اور اوسط مخفی توانائی
\(\langle ٧ \rangle\)
کی قیمتیں حاصل کریں .
آپکو کوئی نیا تکمل حاصل کرنے کی اجازت نہیں ہے . کیا انکا مجموعہ آپکی توقع کے مطابق ہے .\\
سوال
2.12\\
ہارمونی مرتائش کے
nth
ساکن حال کو مثال
2.5
کی ترکیب استعمال کرتے ہوئے
\(\langle x \rangle , \langle x^{2} \rangle , \langle p \rangle , \langle p^{2} \rangle\)
اور
\(\langle T \rangle\)
تلاش کریں .\\
سوال
2.13\\
ہارمونی مرتانش کی مخفی قومیں ایک ذرہ درج ذیل حال سے ابتداء کرتا ہے .
\begin{align*}
\Psi(x,0)=A[3\psi_{0}+4\psi_{1}]\\
\end{align*}
( الف )
A
تلاش کریں\\
( ب )
\(\Psi(x,t)\)
اور
\(\abs{|\Psi(x,t)}^{2}\)
بنائیں\\
( ج )
\(\langle x \rangle\)
اور
\(\langle p \rangle\)
تلاش کریں . انھیں کلایسے کی تعدد پر ارتعاش پذیر ہونے پر زیادہ خوش مت ہوں اگر میں
\(\psi_{1}\)
تو جگہ
\(\psi_{2}\)
دیتا تب جواب کیا ہوتا . دیکھیں اس تفال موج کے لیے مساوات
1.38
کارآمد ہے\\
( د ) اس زرے کو رنائی کی پیمائش میں کون کون سی قیمتیں متوقع ہیں اوران کا احتمال کیا ہوگا۔\\
سوال
2.14\\
ہارمونی مرتائش کے زمینی حال میں ایک زرہ کلاسیے کی تعدد
\(\omega\)
پر ارتعاش پذیر ہے کہ ایک دم لچک
4
گناہوجاتا ہے .
\(\omega^{\prime}=2\omega\).
ہو جبکہ ابتدائی تفال موج تبدیل نہیں کیا جاتا ہیملٹونی تبدیل ہونے کے بنا
\(\Psi\)
اب مختلف طریقوں سے آگے بڑھتا ہے . اس کا احتمال کتنا ہوگا کہ توانائی کی پیمائش
\(\hbar \omega/2\)
تو قیمت دے ؟ پیمائش
\(\hbar\omega\)
حاصل کرنے کا احتمال کیا ہوگا۔\\
حصہ
2.3.2\\
تحلیلی طریقہ کار\\
ہم اب ہارمونی مرتانش کی شرودنگر مساوات پر دوبارہ لوٹتے ہیں۔
\begin{align*}
\frac{-\hbar^{2}}{2m}\frac{d^{2}\psi}{dx^{2}}+\frac{1}{2}m\omega^{2}x^{2}\psi=E\psi\\
\end{align*}
اور اس تو تسلسل کے طریقے سے بلا واسطه حل کرتے ہیں۔ درج ذیل بے بودی متغیر متعارف کرنے سے چیزیں کچو صاف نظر آئیں گی۔
\begin{align*}
\xi=\sqrt{\frac{m\omega}{\hbar}}x;\\
\end{align*}
شروڈنگر مساوات اب درج ذیل روپ اختیار کرتی ہے۔
\begin{align*}
\frac{d^{2}\psi}{d\xi^{2}}=(\xi^{2}-K)\psi\\
\end{align*}
جہاں
K
توانائی ہے جس کی اکائی
\(1/2\hbar\omega\)
ہے .
\begin{align*}
K\equiv \frac{2E}{\hbar\omega}\\
\end{align*}
ہم نے مساوات
2.72
کو حل کرنا ہوگا۔ ایسا کرتے ہوئے ہم
K
اور یوں
E
کو اجارتی قیمتیں معلوم کرپائیں گے۔ ہم شروع کرتے ہیں اس صورت میں جب
\(\xi\)
کی قیمت بہت بڑی ہو یعنی
x
تی قیمت بہت زیادہ ہو ایس صورت میں
\(\xi^{2}\)
تو قیمت
K
کی قیمت سے بہت زیادہ ہوگی لہذا مساوات درج ذیل صورت اختیار کر لے گا۔
\begin{align*}
\frac{d^{2}\psi}{d\xi^{2}}\approx \xi^{2}\psi\\
\end{align*}
جس کا تخمیمی حل درج ذیل ہے۔ اس کی تصدیق کیجے گا .
\begin{align*}
\psi(\xi)\approx Ae^{-\xi^{2}/2}+Be^{+\xi^{2}/2}\\
\end{align*}
اس میں
B
والاجز معمول پر لانے کے قابل نہیں ہے چونکہ
\(\abs{x}\infty\)
کرنے سے اس کی قیمت لا متناہی ہوتی ہے . طبی طور پر قابل قبول حل درج ذیل متکارب صورت کا ہوگا۔
\begin{align*}
\psi(\xi)=(\hspace{1cm})e^{-\xi^{2}/2}\\
\end{align*}
اس سے ہمیں خیال آتا ہے کہ ہم درج ذیل پر غور کریں .
\begin{align*}
\psi(\xi)=h(\xi)e^{-\xi^{2}/2}\\
\end{align*}
اور یہ توقع کریں کے
\(h(\xi)\)
کو صورت
\(\psi(\xi)\)
تفال موج سے سادہ ہوگی۔ مساوات 2.77 کا تفرک لیتے ہیں .
\begin{align*}
\frac{\dif{\psi}}{\dif{\xi}}=\big(\frac{\dif{h}}{\dif{\xi}}-\xi h\big )e^{-\xi^{2}/2}\\
\end{align*}
اور
\begin{align*}
\frac{\dif^{2}{\psi}}{\dif{\xi^{2}}}=\big (\frac{\dif^{2}{h}}{\dif{\xi^{2}}}-2\xi\frac{\dif{h}}{\dif{\xi}}+(\xi^{2}-1)h\big )e^{-\xi^{2}/2}\\
\end{align*}
لہذا شروڈنگر مساوات 2.72 درج ذیل صورت اختیار کرتی ہے .
\begin{align*}
\frac{\dif^{2}{h}}{\dif{\xi^{2}}}-2\xi\frac{\dif{h}}{\dif{\xi}}+(K-1)h=0\\
\end{align*}
ہم فروبینی طریقہ استعمال کرتے ہوئے مساوات 2.78 کاحل
\(\xi\)
کی طاقتی کی صورت میں حاصل کرتے ہیں .
\begin{align*}
h(\xi)=a_{0}+a_{1}\xi+a_{2}\xi^{2}+\dotsc = \sum_{j=0}^{\infty}a_{j}\xi^{j}\\
\end{align*}
جزا درجز اس تسلسل کا تفرق لیتے ہیں .
\begin{align*}
\frac{\dif{h}}{\dif{\xi}}=a_{1}+2a_{2}\xi+3a_{3}\xi^{2}+\dotsc =\sum_{j=0}^{\infty}ja_{j}\xi^{j-1}\\
\end{align*}
اور
\begin{align*}
\frac{\dif^{2}{h}}{\dif{\xi^{2}}}=2a_{2}+2.3a_{3}\xi+3.4a_{4}\xi^{2}+\dotsc =\sum_{j=0}^{\infty}(j+1)(j+2)a_{j+2}\xi^{j}\\
\end{align*}
ا نھیں مساوات 2.78 میں پر کر کہ درج ذیل حاصل ہوگا۔
\begin{align*}
\sum_{j=0}^{\infty}[(j+1)(j+2)a_{j+2}-2ja_{j}+(K-1)a_{j}]\xi^{j}=0\\
\end{align*}
طاقتی تسلسلی پھیلاؤ کی یکتائی کے بنا
\(\xi\)
تو برطاقت کی عددی سر کو غائب ہونا ہوگا۔
\begin{align*}
(j+1)(j+2)a_{j+2}-2ja_{j}+(K-1)a_{j}=0\\
\end{align*}
لہذا درج ذیل ہوگا۔
\begin{align*}
a_{j+2}+\frac{2j+1-K}{(j+1)(j+2)}a_{j}\\
\end{align*}
درج بالا کلیہ توالی مکمل طور پر شروڈنگر مساوات کا مبادل ہے
\(a_{0}\)
سے ابتداء کرتے ہوئے تمام جفت حدد ی سر پیدا کرتا ہے .
\begin{align*}
a_{2}=\frac{1-K}{2}a_{0} \hspace{1cm} a_{4}=\frac{5-K}{12}a_{2}=\frac{(7-K)(K-3)}{120}a_{0}\dotsc\\
\end{align*}
اور
\(a_{1}\)
سے شروع کرکہ یہ تمام تاک حددی پیدا کرتا ہے۔
\begin{align*}
a_{3}=\frac{3-K}{6}a_{1}\hspace{1cm}a_{5}=\frac{(7-K)}{20}a_{3}=\frac{(7-K)(3-K)}{120}a_{1}\dotsc\\
\end{align*}
ہم مکمل حل تو درج ذیل صورت میں لکھتے ہیں
\begin{align*}
h(\xi)=h_{\text{even}}(\xi)+h_{\text{odd}}(\xi)\\
\end{align*}
جہاں
\(h\)
جفت
\begin{align*}
h_{\text{even}}(\xi)=a_{0}+a_{2}\xi^{2}+a_{4}\xi^{4}+\dotsc\\
\end{align*}
\(\xi\)
کا جفت تقال ہے جو
\(a_{0}\)
پرمبنی ہے اور
\(h\)
تاک
\begin{align*}
h_{\text{odd}}(\xi)=a_{1}\xi+a_{3}\xi^{3}+a_{5}\xi^{5}+\dotsc\\
\end{align*}
تقال ہے جو
\(a_{1}\)
پر مبنی ہے یو مساوات 2.81 دو اختیاری
\(a_{0}\)
اور
\(a_{1}\)
کی صورت میں
\(\xi\)
تعین کرتے ہیں۔ جوہم دوررجی تفرقی مساوات کے حل سے توقع کرتے ہیں۔ البتہ اس طرح حاصل شده حلوں میں کافی حل معمول پر لانے کے قابل نہیں ہونگے۔ چونکہ
\(j\)
کی بہت بڑی قیمت پر کلیہ توالی تقریباً درج ذیل روپ اختیار کرتی ہے .
\begin{align*}
a_{j+2}\approx\frac{2}{j}a_{j}\\
\end{align*}
جس کا تخمیمی حل
\begin{align*}
a_{j}\approx\frac{C}{(j/2)!}\\
\end{align*}
ہوگا۔ جہاں
C
لیک مستقل ہوگا اور اس سے
\(\xi\)
کی قیمتیں کے لیے طاقین غالب ہوگی درج ذیل حاصل ہوگا۔
\begin{align*}
h(\xi)\approx C\sum\frac{1}{(j/2)!}\xi^{j}\approx C\sum\frac{1}{j!}\xi^{2j}\approx Ce^{\xi^{2}}\\
\end{align*}
اب اگر
\(h\)
قیمت
\(e^{\xi^{2}}\)
کے لحاظ سے بڑھے تب
\(\psi\)
جس تو ہم حاصل کرنا چاہتے ہیں
\(e^{\xi^{2}/2}\)
یعنی مساوات 2.77
کے لحاظ سے بڑھے گا جو وہ ہی متکاربی روپ ہے جو ہم نہیں چاہتے اس سے نکلنے کا ایک ہی طریقہ ہے وہ یہ کہ معمول پر لانے کی صورت میں طاقتی تسلسل ہر صورت میں ختم ہوتا ہو
\(j\)
کی ایسی بلندترین قیمت جسے ہم
n
کہتے ہیں ہونا ہوگا جو
\(a_{n+2}=0\)
دیتا ہو یوں یا
\(h_{\text{even}}\)
جفت کا تسلسل یا
\(h_{\text{odd}}\)
تاک کا تسلسل ختم ہوگا۔ جبکہ ان میں سے دوسرا ابتداء سے ہی صفر ہوگا۔ یوں جفت کی صورت میں
\(a_{1}=0\)
ہوگا اور تاک n تو صورت میں
\(a_{0}=0\)
ہوگا۔ یوں قابل قبول حل کے لیے مساوات 2.81 کے لیے
\[K=2n+1\]
درکار ہوگا۔ جہاں n کوئی غیر منفی عدد ہے۔ یعنی ہم کہنا چاہتے ہیں کہ مساوات 2.73 کو دیکھیے کہ توانائی پر صورت درج ذیل ہوگی
\begin{align*}
E_{n}=\big(n+\frac{1}{2}\big )\hbar\omega\hspace{1cm} n=0,1,2\dotsc\\
\end{align*}
یوں ہم نے ایک بالکل مختلف طریقہ سے مساوات 2.61 میں الجبرائی طریقے سے بنیادی کو انٹائیز ایشن کی کنڈیشن حاصل کی ابتدائی طور پر یہ ایک حیران کن بات نظر آتی ہے کہ توانائی کی تو انٹائیزیشن شروڈنگر مساوات کے حل میں طاقتی تسلسل تو ایک تکنیکی نقطہ سے حاصل ہوا لیکن آئیے اسکو ایک دو مختلف نقطہ نظر سے دیکھیں۔ مساوات 2.70 کا E
کو کسی بھی قیمت کے لیے حل ممکن ہے درحقیقت پر
E
کے لیے اسکے دوخطی طور پر بے تابع حل پائے جاتے ہیں . لیکن تقریباً ایسے تمام حل
x
کو بڑی قیمت کے لیے لامتناہی کی طرف بڑھتے ہیں . لہذا یہ یہ معمول پر لانے کے قابل نہیں ہوتے فرض کریں کہ
E
کی قیمت اجازتی قیمت سے تھوڑی سی کم ہو مثلاًً
\(0.49\hbar\omega\)
اس کے حل کو ترسیم کرتے ہوئے
Figure 2.6(a)
دیکھتے ہیں کہ اس کی دم لامتناہی کی طرف بڑھتی ہے . اب E کی قیمت کسی ایک اجازتی قیمت سے زیادہ تصور کریں مثلاً
\(0.51\hbar\omega\)
اب حل تو دم دوسری سمت میں لامتناہی کی طرف بڑھتی ہے .
Figure 2.6(b)
اب اگر  آپ اس مقدار معلوم تو قیمت
0.49
سے بہت چھوٹی چھوٹی قیمتوں میں بڑھا کہ
0.51
کی طرف لے جائیں تو آپ دیکھیں گے کہ حل کی دم
0.51
سے گزر کر ایک طرف لامتناہی کی بجائے دوسری طرف لامتناہی کی طرف بڑھتی ہے . ٹھیک 0.5 پراسکی دم 0 کی طرف بڑھتی ہے .اور یہ وہ حل ہوگا جو معمول پر لانے پر قابل ہو . T کی اجارتی قیمتوں کے لیے کلیه توالی درج ذیل روپ اختیار کرتی ہے
\begin{align*}
a_{j+2}=\frac{-2(n-j)}{(j+1)(j+2)}a_{j}\\
\end{align*}
اگر
n=0
ہو تب تسلسل میں ایک جز پایا جائے گا۔ ہمیں
\(a_{1}=0\)
لینا ہوگا تاکہ
h
تاک ختم ہو اور مساوات 2.84 میں
\(j=0\)
لینا ہوگا تاکہ
\(a_{2}=0\)
ہو
\begin{align*}
h_{0}(\xi)=a_{0}\\
\end{align*}
لہذا
\begin{align*}
\psi_{0}(\xi)=a_{0}e^{-\xi^{2}/2}\\
\end{align*}
جو معمول پر لانے کے ساتھ ساتھ مساوات 2.59
دورباره دیتا ہے . اسی طرح n = 1 کے لیے ہم
\(a_{0}=0\)
لیتے ہیں اور مساوات 2.84 میں
\(j=1\)
کے کر
\(a_{3}=0\)
حاصل کرتے ہیں تاکہ
\begin{align*}
h_{1}(\xi)=a_{1}(\xi)\\
\end{align*}
ہو
اور
\begin{align*}
\psi_{1}(\xi)=a_{1}\xi e^{-\xi^{2}/2}\\
\end{align*}
ہو جس سے مساوات 2.62 کی تصدیق ہوتی ہے . ہم
n=2
کے لیے
j= 0
لے کر
\(a_{2}=-2a_{0}\)
اور
j=2
کر
\(a_{4}=0\)
حاصل کرتے ہیں۔ یوں
\begin{align*}
h_{2}(\xi)=a_{0}(1-2\xi^{2})\\
\end{align*}
اور
\begin{align*}
\psi_{2}(\xi)=a_{0}(1-2\xi^{2})e^{-\xi^{2}/2}\\
\end{align*}
وغیره وغیره . یہاں سوال 2.10 کے ساتھ موازنہ کریں جہاں یہ آخری نتیجه الجبرائی طریقے سے حاصل کیا گیا۔ عمومی طور پر اگر
\(h_{n}(\xi)\)
جفت طور پر ہو تب
n
متغیر
\(\xi\)
کے صرف جفت طاقتوں کے
n
درجے کا
hn
رکنی ہوگا اور تاک عدد کی صورت میں صرف تاک طاقتی رکن ہوگا۔ جو ضربی
\(a_{0}\)
اور
\(a_{1}\)
کے علاوہانھیں ہر مشی کسے رکنی کہتے ہیں
\(H_{n}(\xi)\)
 . جدول 2.1 میں ان کی ابتدائی چند رکنیات دی گئی ہیں۔ روایتی طور پر ان کی اختیار ی جو ضربی کو یوں منتخب کیا جاتا ہے کہ
\(\xi\)
کے بلند تر طاقتی سر
\(2^{n}\)
ہو اس روایت کے ساتھ ہارمونی مرتائش کے معمول پر لائے گئے ساکن حالات درج ذیل ہونگے
\begin{align*}
\psi_{n}(x)=\big (\frac{m\omega}{\pi\hbar}\big )^{1/4}\frac{1}{\sqrt{2^{n}n!}}H_{n}(\xi)e^{-\xi^{2}/2}\\
\end{align*}
جو مساوات 2.67 میں الجبرائی طریقے سے حاصل نتائج کے متماسل ہیں . شکل 2.7 (a) میں میں نے چند ابتدائی n کی قیموں کے لیے
\(\psi_{n}(x)\)
ترسیم کی ہے۔ کوانٹم مرتائش حیران کن حد تک کلایسے کی مرتائش سے مختلف ہے . اس کے نہ صرف توانائی کو انٹائیز ہیں بالکہ اس کی مقام کی تقسیم بھی عجیب ہے . مثال کی طور پر کلاسکی طور پر اجارتی ساعت کے باہر یعنی جہاں توانائی کی کلاسکی طول سے
x
 کی قیمت زیادہ ہو زرے کے پائے جانے کا احتمال غیر صفر ہے .\\
سوال 
2.15
تمام تاک حالت میں پائے جانے والے عین وسط پر زرہ پائے جانے کا احتمال صفر ہے۔ کلاسکی اور کوانٹم میں
n
کو بڑی قیمت پر مشابہت پائی جاتی ہے . میں نے n=100 کے لیے کلاسکی مقام کی تقسیم کو توانٹم صورت کے اوپر ترسیم کیا ہے . انھیں ہموار کرنے سے یہ ایک دوسرے پر کافی اچھا بیٹھتے ہیں البته کلاسکی صورت میں ہم ایک مرتعاش کے مقام کی تقسیم کی بات کرتے ہیں۔ جبکہ توانٹم میں ہم بالکل یکساں تیار کرده سگره تقسیم کی بات کر رہے ہیں .
سوال 
2.15\\
ہارمونی مرتائش کے زمینی حال میں کلاسکی طور پر اجازتی خطے کے باہر ایک زرے کی موجودگی کا احتمال 3 با معنی ہندسوں تک تلاش کریں .
اشاره : کلاسکی طور پر ایک مرتائش
\(E=(1/2)ka^{2}=(1/2)m\omega^{2}a^{2}\)
جہاں 
a
بیتا یو E توانائی کے مرتائش کا کلاسکی اجارتی خطا منفی
\(-\sqrt{2E/m\omega^{2}}\)
سے مثبت
\(+\sqrt{2E/m\omega^{2}}\)
ہو گا . تکمل کی قیمت عمومی تقسیم یا خلل کی جدول سے دیکھئے گا.\\
سوال 2.16
کلیہ توالی مساوات 2.48 استعمال کرتے ہوئے
\(H_{5}(\xi)\)
اور
\(H_{6}(\xi)\)
تلاش کریں . مجموعی مستقل تاعين کرنے کی خاطر
 \(\xi\) 
کی بلند ترین طافت کا عددی سر
\(2^{n}\)
اور
n
لیں\\
سوال 
2.17\\
اس سوال میں ہم ہرمشی رکنی کے اہم مسائل جن کاثبوت پیش نہیں کیا گیا ہے پر غور کریں گے۔
( الف ) روڈریگز کلیہ درج ذیل ہے۔
\begin{align*}
H_{n}(\xi)=(-1)^{n}e^{\xi^{2}}\big (\frac{d}{d\xi}\big )^{n}e^{-\xi^{2}}\\
\end{align*}
اس کم استعمال کرتے ہوئے
\(H_{3}\)
اور
\(H_{4}\)
اخذ کریں .
( ب ) درج ذیل کلیہ تو الی آپکو سابقہ ہر مشی رکنی کی صورت میں
\(H_{n+1}\)
دیتا ہے .
\begin{align*}
H_{n+1}(\xi)=2\xi H_{n}(\xi)-2nH_{n-1}(\xi)\\
\end{align*}
اس کو مختلف طریقوں سے استعمال کرتے ہو
\(H_{5}\)
اور
\(H_{6}\)
تلاش کریں\\
( ج )
اگر آپ n رتبی کثیر رکنی کا جذر لیں تو آپکو
\(n-1\)
رتبی کثیر رکنی حاصل ہوگا . ہرمشی کثیر رکنی کے لیے درج ذیل ہوگا .\\
\begin{align*}
\frac{\dif{H_{n}}}{\dif{\xi}}=2nH_{n-1}(\xi)\\
\end{align*}
ہرمشی کثیر رکنی کے لیے
\(H_{5}\)
اور
\(H_{6}\)
تلاش کریں۔\\
( د ) پیدا کار تفال
\(e^{-z^{2}+2z\xi}\)
کا
\(z=0\)
پر
n
کا تفرق
\(H_{n}(\xi)\)
ہوگا۔ یا اس کو یوں کہہ سکتے ہیں کر درج ذیل تفال میں ٹیلر پھیلاؤ میں
\(z^{n}/n!\)
کا عددی سر hn ہوگا۔
\begin{align*}
e^{-z^{2}+2z\xi}=\sum_{n=0}^{\infty}\frac{z^{n}}{n!}H_{n}(\xi)\\
\end{align*}
اسکو استعمال کرتے ہوئے
\(H_{0}, H_{1}\)
اور
\(H_{2}\)
دوباره حاصل کریں .\\
حصہ 
2.4\\
آزاد زرہ\\
ہم اب آزاد زرہ جس کے لیے پر جگہ
\(V(x)\)
صفر ہوگا پر غور کرتے ہیں جس کو سب سے زیادہ ساده ہونا چاہیے تھا . کلاسکی طور پر یہ مستقل سمتی رفتار کی حرکت کو ظاہر کرے گا۔ لیکن کوانٹم مقانیات میں یہ مسئلہ بہت پیچیده اور پراسرار ثابت ہوتا ہے . اس کا وقت سے غیر تابع شروڈنگر مساوات درج ذیل ہوگا
\begin{align*}
\frac{-\hbar^{2}}{2m}\frac{\dif^{2}{\psi}}{\dif{x^{2}}}=E\psi\\
\end{align*}
یا
\begin{align*}
\frac{\dif^{2}{\psi}}{\dif{x^{2}}}=-K^{2}\psi
\end{align*}
جہاں
\begin{align*}
K\equiv\frac{\sqrt{2mE}}{\hbar}\\
\end{align*}
ہوگا۔ یہاں تک یہ لامتناہی چکور كوما کی طرح ہے مساوات 2.21 جہاں مخفی قو صفر ہے . اس بار البتہ میں عمومی مساوات کو قوت نمائی صورت میں ناکہ
sin
اور
cosine
کی صورت میں لكهنا چاہوں گا . جس کو وجہ آپ پر جلد عیاں ہوگی
\begin{align*}
\psi(x)=Ae^{ikx}+Be^{-ikx}\\
\end{align*}
لامتناہی قوا کے برعکس جہاں کوئی سرحدی شرائط نہیں ہیں جو
K
کی ممکنہ قیمتوں اور یوں
E
کی ممکنہ قیمتوں کو محدود کرتا ہو آزاد زرہ کوئی بھی مثبت توانائی رکھ سکتا ہے . اس کے ساتھ وقت کی تابعیت
\(e^{-iEt/\hbar}\)
جوڑتے ہوئے ذیل حاصل ہوگا۔
\begin{align*}
\Psi(x)=Ae^{ik(x-\frac{\hbar k}{2m}t)}+Be^{-ik(x+\frac{\hbar k}{2m}t)}\\
\end{align*}
اب
x
اور
t
متغیرات کے تابع کوئی بھی تفاعل جو ان مخصوص مجموعا یعنی
\(x\pm vt\)
ہے جہاں
لا
مستقل ہے غیر تغیر صورت کی موج تو ظاہر کرتا ہے جو
\(\mp x\)
رخ اور
v
رفتار سے حرکت کرتا ہو اس موج پر کوئی نقطہ مثلاً موج یا نشیب دلیل کی ایک اٹل قیمت ہے . لہذا
x
اور
t
یوں متوقع ہوگا کہ
\begin{align*}
x\pm vt=\text{constant} x=\mp vt+\text{constant}\\
\end{align*}
یا
\begin{align*}
 x=\mp vt+\text{constant}\\
\end{align*}
چونکہ موج پر ہر نقطہ ایک ہی سمتی رفتار سے حرکت کرتا ہے لہذا موج کی شکل و صورت حرکت کے ساتھ تبدیل نہیں ہوگی۔ یوں مساوات 2.93 کا پہلا جز دائیں رخ حرکت کرتی موج کو ظاہر کرتا ہے جبکہ اسکا دوسرا رخ بائیں طرف رخ کرتی موج کو ظاہر کرتا ہے . چونکہ ان میں فرق صرف
K
کی علامت ہے لہذا نھیں درج ذیل بھی لکھا جا سکتا ہے 
\begin{align*}
\Psi_{k}(x,t)=Ae^{i(Kx-\frac{\hbar k^{2}}{2m}t)}\\
\end{align*}
جہاں
K
کی قیمت منفی لینے سے بائیں رخ حرکت کرتی موج کو ظاہر کرتا ہے
\begin{align*}
k\equiv\frac{\sqrt{2mE}}{\hbar}\\
\end{align*}
\begin{align*}
\begin{cases}
k>0\Rightarrow \\
k<0\Rightarrow \\
\end{cases}
\end{align*}
صاف ظاہر ہے کہ آزاد ذرے کی ساکن صورتیں حرکت کرتی امواج کو ظاہر کرتی ہیں . جن کی قولی موج
\(\lambda=2\pi/\abs{K}\)
ہوگا۔ اور ڈی بر و گلی کے کلیہ مساوات 3.19 کے تحت ان کا معیار حرکت درج ذیل ہوگا۔
\begin{align*}
p=\hbar k\\
\end{align*}
ان امواج کی رفتار یعنی
x
کا عددی سر درج ذیل ہوگا۔
\begin{align*}
v_{\text{quantum}}=\frac{\hbar\abs{k}}{2m}=\sqrt{\frac{E}{2m}}\\
\end{align*}
اس کے برعکس ایک ذرہ جس کی توانائ
\(E=1/2mv^{2}\)
ہوگی جو خالصتاً حرکی توانائی ہوگی کی کلا سکی رفتار درج ذیل ہوگی۔
\begin{align*}
v_{\text{classical}}=\sqrt{\frac{E}{2m}}=2v_{\text{quantum}}\\
\end{align*}
ظاہری طور پر کوانٹم مکانی تفال موج اور ذرے کی رفتار جو نصف رفتار سے حرکت کرتا ہو ظاہر کرتا ہے . اس تضاد پر ہم جلد غور کریں گے۔ اس سے پہلے ہم ایک زیاده سنگین مسئلے پر غور کرتے ہیں . یہ تفال موج معمول پر لانے کے قابل نہیں ہے 
\begin{align*}
\int_{-\infty}^{+\infty}\Psi_{k}^{*}\Psi_{k}\dif{x}=\abs{A}^{2}\int_{-\infty}^{+\infty}\dif{x}=\abs{A}^{2}(\infty)\\
\end{align*}
یوں آزاد ذرے کی صورت میں الحیدگی مساوات سے حاصل کردہ طبی طور پہ قابل حصول حالات کو ظاہر نہیں کرتے ہیں . ایک آزاد زده ساکن حالت میں نہیں ہو سکتا یا ہم یوں کہہ سکتے ہیں کہ ایک غیر مبہم توانائی کے ایک آزاد ذرہ کا تصور بے معنی ہیں اس کا ہرگز یہ مطلب نہیں کہ قابل الحیدگی مساوات سے حاصل حل ہمارے کسی کام نہیں کیونکہ یہ حساب میں کردار ادا کرتے ہیں جن کا ان کے طبی تشربع سے کوئی تعلق نہیں وقت تو تابع شروڈنگر مساوات کا حل اب بھی قابل الحبرگی مساوات کے حل کا جوڑ ہوگا۔ بس اس بار استمراری متغیر کے لحاظ سے تکمل لینا ہوگا نہ کہ غیر مسلسل اشائیه
n
پر مجموعہ لینا ہوگا۔
\begin{align*}
\Psi(x,t)=\frac{1}{\sqrt{2\pi}}\int_{-\infty}^{+\infty}\phi(k)e^{i(kx-\frac{\hbar k^{2}}{2m}t)}\dif{k}\\
\end{align*}
\end{document}
