

سوال 1.12  ہم گزشتہ سوال کے رفتار پیمانہ کی سوئی پر دوبارہ بات کرتے ہیں ۔ لیکن اس بار ہم سوئی کے سر کے  \عددی{x} محدد میں دلچسپی رکھتے ہیں ۔ یعنی  عرف کی لکیر پر سوئ کے سایہ میں ہم دلچسپی رکھتے ہیں ۔ \عددی{\rho(x)} کی کسافت اہتمال کیا ہو گی ؟  \عددی{x} کے لحاظ  سے   \عددی{\rho( x )}  کو  \عددی{-2 r} تا  \عددی+ 2 r میں تقسیم کریں ۔ جہاں  \عددی{r}
سوئی کی لمبائ ہے ۔ تصدیق کر لیں کہ قل اہتمال \عددی{1} کے برابر ہو گا۔  اشارہ :  \عددی{x} اور \عددی{x + dx}  کے بیچ   \عددی{\psi} کی موجودگی کا اہتمال \عددی{\rho(x) dx} ہو گا۔ آپ سوال 1.11 سے  \عددی{\theta}  کا اہتمال جانتے ہیں کہ کس کے ساتھ پر پایا جاتا ہے۔ اب یہاں سوال یہ ہے کہ \عددی{d \theta}  کے وقفے مطابقطی  \عددی{dx}  کیا ہو گا ؟ \عددی{\langle x \rangle}
, \عددی{\langle x^2 \rangle} اور  \عددی{\sigma}
 اس تقسیم کے لیے تیار کریں ۔ آپ ان قیمتوں کو سوال 1.11 کے جزو (ج) سے کس طرح حاصل کر سکتے ہیں۔  

سوال 1.13 کاغذ پر \عددی{L}
فاصلے پر اف کی لکیریں بنائی جاتی ہیں ۔ کچھ  بکندی سے اس کاغذ پر ایک سوئی گرائی جاتی ہے  سوئی کی لمبائی \عددی{L}  ہے۔  کیا اہتمال ہو گا کہ سوئ لکیر کو کاٹ کر صفے پر آن ٹھرے ۔ اشارہ : سوال 1.12 سے رجوع کریں ۔ 

سوال 1.14  لمحہ  \عددی{T} پر  \عددی{( a < x < b )} کے بیچ ایک ذرہ پاۓ جانے کا اہتمال \عددی{P_{ab} (t)} ہے ۔ 

(الف) دکھائیں کہ 
\begin{align}
\frac{ dP_{ab} }{ dt } = J ( a , t ) - J ( b , t )
\end{align}
جہاں 
\begin{align}
j ( x , t ) =  \frac{ i \hbar }{ 2 m } ( \psi \frac{ d \partial \psi^\star }{ \partial x } - \psi^\star \frac{ \partial \psi }{ \partial x } )
\end{align}
\عددی{J ( x , t )}
کی اکائی کیا ہو گی ؟ چونکہ \عددی{J} آپ کو بتاتا ہے کہ نقطہ \عددی{x} عرصے اہتمال کس رفتار سے گزرتا ہے ۔ لہذا  \عددی{J} کو اہتمال \عددی{\rho} کہتے ہیں - اگر \عددی{P_{ab} (t)} بڑھ رہا ہو تب کسی خطے میں ایک جانب سے اہتمال کی آمد خطے کے دوسری طرف کے اہتمال کے خوائف سے زیادہ ہو گا۔  

(ب) سوال 1.9 میں طفال معاج کا اہتمال \عددی{\rho} کیا ہو گا ؟ 

سوال 1.15 فرض کریں آپ ایک غیر مستحکم ذرہ کے بارے میں بات کرنا چاہیں ، جس کا عرصہ حیات 
\عددی{\tau} ہو۔ ایسی صورت میں کہیں پر بھی ذرہ پاۓ جانے کا قل اہتمال مستقل نہیں ہو گا ۔ بلکہ وقت کے ساتھ قوت نمائ طور پر کم ہو گا۔ ااس نتیجے کو بے ڈھنگے طریقے سے یوں حاصل کیا جا سکتا 
ہے۔
\begin{align}
P(t) = \int_{-\infty }^{ \infty } \mid \psi ( x , t ) \mid^2 dx
\end{align} 
اگر ہم مساوات 1.24 میں فرض کر لیں کہ 
\عددی{V} (ممکنہ توانائ ) ایک حقیقی مقدار ہے ۔ یہ ایک معقول بات ہے  ۔ لیکن اس سے مساوات 1.27 میں اہتمال کی اٹل پن معخوظ ہوتی ہے۔ آئیں دیکھیں
\عددی{V}  
کا خیالی جزو بھی پایا جاتا ہو۔
\begin{align}
V = V_o - \iota T
\end{align}
جہاں 
\عددی{V_o} حقیقی ممکنہ توانائ ہے ۔  اور یہ مثبت حقیقی مستقل ہے ۔ 
 
(الف) دکھائیں کہ اب مساوات 1.27 کی جگہ ہمیں درج ذیل ملتا ہے ۔  
\begin{align}
\frac{ dP }{ dt } = -\frac{ 2T }{ \hbar } p
\end{align}

(ب)
\عددی{P(t)} 
کے لیے حل کریں اور ذرے کا عرصہ حیات  
\عددی{T}
کی صورت میں حاصل کریں۔

سوال :1.16
دکھائیں کہ shroudinger مساوات کے کائ بھی 2 معمول پر لاۓ جانے کے قابل حل یہاں 
\عددی{\psi_1}
اور 
\عددی{\psi_2}
درج ذیل پر پورا اترتے ہیں ۔ 
\begin{align}
\frac{ d }{ dt } \int_{ - \infty }^{ \infty } \psi_1^\star \psi_2 dx = 0
\end{align}

سوال : 1.17
لمحہ 
\عددی{T=0}
پر ایک ذرے کو درج ذیل طفعال معاج پیش کرتا ہے ۔ 
$$ \begin{cases}
\psi ( x , 0 ) = A( a^2  - b^2 )  if  -a \leq x \leq +a \\
0 , otherwise 
\end{cases}$$
(الف) معمول پر لاۓ جانے والا A تلاش کریں ۔ 

(ب) لمحہ 
\عددی{}
$ t = 0 $
پر 
\عددی{}
$ x $ 
کی توقعاتی قیمت تلاش کریں ۔

(ج) لمحہ 
\عددی{}
$ t = 0 $
پر
\عددی{}
$ p $
تلاش کریں  اور یہاں دیہان رہے ۔ آپ اس کو 
\عددی{}
$ P = \frac{ md \langle x \rangle }{ dt } $
 سے حاصل نہیں کر سکتے ہیں ۔ ایسا کیوں ہے 

(د) 
\عددی{x^2}
کی توقعاتی قیمت دریافت کریں۔ 

(ہ) 
\عددی{p^2}
کی توقعاتی قیمت دریافت کریں ۔ 
\عددی{x ( \sigma_x)}
کی غیر یقینی صورتحال دریافت کریں ۔ 
\عددی{p ( \sigma_p )}
کی غیر یقینی صورتحال دریافت کریں ۔ تصدیق کریں کہ آپ کے نتائج حصول  لاتیکن کے عین مطابق ہیں ۔

سوال 1۔18 : 
عموما کونٹم مکانیات تب کرامت ہو گا جب  ایک ذرہ کی ڈی بروگلی طول معاج نظام کی جسامت سو زیادہ ہو ۔ درجہ  
\عددی{T Kelvin}
پر ذرے کی اوسط توانائی 
\begin{align}
\frac{ p^2 }{ 2 m } =  \frac{ 3 }{ 2 } k_b T
\end{align}
ہو گی ۔ جہاں 
\عددی{k_b}
boltzman مستقل ہے ۔ 
لہذا ڈی بروگلی طعل معاج درج ذیل ہو گا ۔
\begin{align}
\lambda = \frac{ h }{ \sqrt{ 3 m k_B T } }
\end{align}
ہم نے معلوم کرنا ہے کہ کونسا نظام کونٹم سے حل ہو گا اور کونسا نظام کلاسیکل سے حل ہو گا اور کیا کرنا مناسب ہو گا ۔ 

(الف ) ٹھوس جسم میں فاصلہ جال
\عددی{d = 0.33 mm}
ہوتا ہے ۔ وہ درجہ حرارت تلاش کریں  جہاں آزاد برکی زرا کونٹم مکانیاتی ہوں گے ۔ وہ درجہ حرارت تلاش کریں جس سو کم درجہ حرارت پر Atomic Particles کونٹم مکانیاتی ہوں گے ۔ sodium کو مثا ل بنائیں ۔ ہم دیکھیں گے کہ کچھ اجسام میں آزاد برکی زرے ہر صورت کونٹم مکانیاتی ہوں گے ۔  جبکہ Atomic مرکز کبھی بھی کونٹم مکانی نہیں ہوں گے ۔ یہی سب کچھ  معیا کے لیے بھی درست ہے ۔ جہاں Atoms کے بیچ فاصلے اتنا ہی ہو گا۔ سواۓ helium کے یا ان Atoms k جن کا  درجہ حرارت 4 kelvin پر موجود Helium کے 

(ب) گیس : میکانی دباؤ p پر کس درجہ حرارت سے نیچے کامل گیس کے Atoms کونٹم مکانی ہوں گے۔ اشارہ : 
