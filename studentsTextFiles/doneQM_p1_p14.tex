\documentclass{book}
\usepackage{polyglossia}     
\usepackage{amsmath,amssymb}      
\setmainlanguage[numerals=maghrib]{arabic}   
\setotherlanguages{english}
\newfontfamily\arabicfont[Scale=1.0,Script=Arabic]{Urdu Typesetting} 
\newfontfamily\urdufont[Scale=1.25,WordSpace=60.0,Script=Arabic]{Urdu Typesetting}
%\setlength{\parskip}{5mm plus 4mm minus 3mm}
\begin{document}
اب تک ہم غیر مسلسل متعغیرات کی بات کرتے آ رہے ہیں۔  جن کی قیمتیں الگ تھلگ ہوتی ہیں۔  گزشتہ مسال میں ہم نے افراد کی عمر کی بات کی  جو سالوں میں ناپا جاتا ہے۔ لہذا j  عدد صہی تھا۔ ہم اس  تصور کو توصیع دے کر استمراری تقسیم کے لیے بھی استعمال کر سکتے ہیں ۔ اگر میں  گلی میں کسی شخص کی عمر پوچھوں  تو اس کا امکان  تقریباۡۡ صفر ہو گا۔ کہ اس کی  عمر ٹھیک ٹھیک 16 سال 4 گھنٹے 27 منٹ 3٫34527 سیکنڈ ہو   اس شخص سے عمر پوچھتے وقت معنی خیز جواب یہ ہو گا کہ اس کی عمر 16 سال سے 17 سال کے درمیان ہے۔ بہت کم وقفہ کی صورت میں اہتمال وقفہ کی لمباٗی کا راست متناصب ہو گا۔ مثال کے طور پر اس بات کا امکان یہ کہ کسی شخص کی عمر 16 سال اور 16 سال جمع 2 دن کے بیچ  ہو۔ اس سے دگنا ہو گا  اس شخص کا عمر  16 سال اور 16 سال اور ۱ دن کے بیچ ہو۔  ہاں اگر 16 سال قبل اس مخصوص دن پر کسی وجہ سے بہت زیادہ  بچے پیدہ ہوےٗ ہوں تب ایسا نہیں ہو گا۔ ایسی صورت میں ہم نے 2 دن کا یا 1 دن کا وقفہ جو منتخب کیا۔
%\newpage
 وہ چھوٹا وقفہ تصور نہیں کیا جاےٗ گا۔ ہم مزید چھوٹا وقفہ لے کر اس مثال کو دوبارہ دیکھینگے۔ یوں آپ کہ سکتے ہیں کہ لا متناحی طور پر چھوٹے وقفے کی بات کر رہے ہیں۔ یوں درج ذیل ہو گا۔ 
\begin{equation}
\rho (x) dx
\end{equation}
اس مساوات میں تناصبی مستقل 
$ \rho (x) $
کا ثقافت اہتمال کہلاتا ہے۔ متناہی وقفہ ایک تعبی کے بیچ 
$ x $
کا اہتمال 
$ \rho (x) $
  کے تکمل سے حاصل ہو گا۔ غیر مسلسل تقسیمات  کے قواعد جو ہم نے حاصل کیے،  اثتمراری تقسیم کے لیے درج ذیل ہوں گے۔ 
\begin{equation}
P_{ab} = \int_{-\infty}^{\infty} \rho (x) dx
\end{equation}
\begin{equation}
 1 = \int_{- \infty}^{\infty} \rho (x) dx
\end{equation}
\begin{equation}
x = \int_{ - \infty }^{ \infty} x \rho (x) dx
\end{equation}
\begin{equation}
\langle f(x) \rangle = \int_{ - \infty }^{\infty } f(x) \rho (x) dx 
\end{equation}
\begin{equation}
\sigma^2 \equiv \langle ( \delta x)^2 \rangle=\langle x^2\rangle -  \langle  x  \rangle ^2
\end{equation}
مثال 1٫1  مثال کے طور پر میں 1 چٹان جس کی بلندی h ہو ، سے 1 پتھر کو نیچے گرنے دیتا ہوں۔ جیسے جیسے پتھر گرتا ہے ، میں طلاواسطہ وقتی فاصلوں پر اس کے 10 لاکھ تصاویر کھینچتا ہوں اور یوں میں ہر تصویر میں فاصلہ ناپتا ہوں جو پتھر طے کرتا ہے۔ اب سوال پیدا ہوتا ہے کی ان تمام فاصلوں کی اوسط  کی مد کیا ہے ۔ ان کی وقتی اوسط کیا ہو گی۔ 

حل   پتھر سا کن صورتحال سے شروع ہو کر بتدریج بڑھتی رفتار سے نیچے گرتا ہے۔ یہ چٹان کے بالکل بالایٗ سر کے قریب زیادہ وقت گزارتا ہے۔ لہذا  ہم توقع کرتے ہیں کہ فاصلہ 
$ \frac{h}{2} $
سے کم ہو گا۔ ہم ہوایٗ رگڑ کو نظر انداذ کرتے ہیں۔  ہم لمحہ t پر فاصلہ x درج ذیل ہو گا۔ 

\begin{align}
x (t) = \frac{1}{2} g t^2
\end{align}
اس کی صمتی رفتار 
$ \frac{dx}{dt} = gt $
ہو گی اور گرنے کا قل دورانیہ
$  T = \sqrt{\frac{2h}{g}} $
 جزو ہو گا۔  وقفہ dT میں یصویر کھینچنے کا اہتمال 
$ \frac{dT}{t} $
ہو گا۔ 
\newpage 
یوں اس کا اہتمال کہ کسی 1 تصویر میں مطابقطی فاصلہ
$ dx $ 
ہو گا ، درج ذیل ہے۔ 
\begin{align}
\frac{dt}{T} = \frac{ dx }{ gt } \sqrt{ \frac{ g }{ 2h } } = \frac{ 1 }{ 2 \sqrt{ hx } } dx
\end{align}
 ظاہر ہے  کہ کثافت اہتمال یعنی  مساوات 1٫14 درج ذیل ہو گا۔
\begin{align}
\rho (x) = \frac{ 1 }{ 2 \sqrt{ h x } ,     ( 0 \leq  x \leq h )
\end{align}
  جبکہ اس وقفہ سے باہر کثافت اہتمال صفر ہو گا۔ ہم مساوات  1٫16 استعمال کر کے اس نتیجے کی تصدیق کر سکتے ہیں 
\begin{align}
 \int_{0}^{h} x \frac{ 1 }{ 2 \sqrt{ h x }} dx =   \frac{ 1 }{ 2 \sqrt{ h } } (  2 x^\frac{ 1 }{ 2 } {0}^{h} = 1
\end{align}
مساوات 1٫17 سے ہم اوسط فاصلہ حاصل کرتے ہیں۔ 
\begin{align}
\langle x \rangle =  \int_{0}^{h} x \frac{1}{2 \sqrt{hx}} dx = \frac{1}{2} \sqrt{h} ( \frac{2}{3} x^\frac{3}{2} 0 ^{ h } = \frac{ h }{ 3 }
\end{align}
شکل 1٫6 میں  
$ \rho ( x ) $
کی ترثیم دکھایٗ گیٗ ہے۔ آپ دیکھ سکتے ہیں کہ کثافت اہتمال ارز خود لا متناحی ہو سکتا ہے ۔ اگرچہ اہتمال یعنی
$ \rho $
کا تکمل ہر صورت  متناحی ہو گا۔ بلکہ یہ 1 کے برابر یا 1 سے کم ہو گا۔ 
\newpage 
سوال 1٫1  حصہ 1٫3٫1 میں اشخاص کی عمر کی تقثیم  کے لیے درج ذیل حاصل کریں۔ 
الف۔  
$ \langle j^2 \rangle $
اور  
$ \langle j \rangle^2 $
کی اوسط اور اوسط کا مربع حاصل کریں۔
\\
ب۔ ہر j کے لیے 
$ \delta j $ 
دریافت کریں۔ اور مساوات 1٫11 استعمال کرتے ہوےٗ معیاری انحراف  حاصل کریں۔ 
\\
ج۔  الف اور ب کے نتاجٗ استعمال کرتے ہوےٗ مساوات 1٫12 کی تصدیق کریں۔ 
\\
سوال 1٫2 
\\
الف۔ مثال 1٫1 کی تقثیم کے لیے معیاری  انحراف تلاش کریں۔
\\ 
ب۔  اس کا احتمال کتنا ہو گا کہ بلاواسطہ منتخب کیے گےٗ کسی تصویر میں فاصلہ اوسط کی مد کے فاصلے سے ایک معاری انحریف زیادہ ہو گا۔ 
\\
سوال 1٫3 
gaussian تقثیم 
\begin{align}
\rho (x) = A e^\lambda(x   a)^2
\end{align}
جہاں A , a اور 
$ \lambda $
مثبت  حقیقی مستقل ہیں ۔ درج ذیل حاصل کریں۔  اگر ضرورت پیش آےٗ تو تکمل کسی جدول سے دیکھ لیں۔
\\
 الف۔  مساوات 1٫16 ااستعمال کرتے ہوےٗ A کی قیمت تعین کریں۔
\\
ب۔ اوسط 
$ (x) $
اور
$ (x^2) $
اور
$ \sigma $
تلاش کریں۔ 
\\
ج۔ x کی ترثین کا خاکہ بنایٗں۔
\end{document}
