
\باب{بکھراو}
\حصہ{تعارف}
\جزوحصہ{کلاسیکی نظریہ بکھراو}
فرض کریں کسی مرکز بکھراو پر ایک ذرہ کا آمد ہوتا ہے مثلاً ایک پروٹان کو ایک بھاری مرکزہ پر داغا جاتا ہے یہ توانائی \عددی{E}  اور ٹکراو مقدار معلوم \عددی{b} کے ساتھ آکر کسی زاویائے بکھراو \عددی{\theta} پر اُبھرتا ہے \حوالہء{شکل \num{11.1}} دیکھیں۔ میں اپنی آسانی کے لیئے فرض کرتا ہوں کہ ہدف اسمتی تشاکلی ہے یوں خطِ حرکت ایک مستوی میں پایا جائے گا اور کہ نشانہ بھاری ہے لحاظہ تصداً کی بنا اس کی حرکت اُچھلنے کو نظرانداز کیا جاسکتا ہے۔ کلاسیکی نظریہ بکھراو کا بنیادی مسئلہ یہ ہوگا: ٹکراو مقدار معلوم کو جانتے ہوئے زاویائے بکھراو کا حساب کریں۔ یقیناً عام طور پر ٹکراو مقدار معلوم جتنا چھوٹا ہو زاویہ بکھراو اتنا بڑا ہوگا۔

\ابتدا{مثال}
\موٹا{سخت کرہ کا بکھراو}۔ فرض کریں ہدف رداس \عددی{R} کا ایک ٹھوس بھاری گیند ہے جبکہ آمدی ذرہ ہوائی بندوق کا ایک چھرہ ہے جو لچھکیلی ٹپکی کھا کر مڑتا ہے \حوالہء{شکل \num{11.2}}۔ زاویہ \عددی{\ایلفا} کی صورت میں ٹکراو مقدار معلوم \عددی{b=R\sin\ایلفا} اور زاویہ بکھراو \عددی{\تھیٹا=\پاے-2\ایلفا} ہوں گے۔ یوں درج ذیل ہوگا
\begin{align}
	b = R\sin\left(\frac{\pi}{2}-\frac{\theta}{2}\right) = R\cos\left(\frac{\theta}{2}\right)
\end{align}
ظاہری طور پر درج ذیل ہوگا
\begin{align}
	\theta =
	\begin{cases}
		2\cos^{-1}(b/R), & b\leq R \text{\RL{اگر}} \\
		0, & b\geq R \text{\RL{اگر}}
	\end{cases}
\end{align}
\انتہا{مثال}
عمومی طور پر لامتناہی چھوٹے رقبہ عمودی تراش \عددی{\dif\سگما} میں آمدی ذرات مطابقتی لامتناہی چھوٹے ٹھوس زاویہ \عددی{\dif\بڑااومیگا} میں بکھریں گے \حوالہء{شکل \num{11.3}}۔ بڑی \عددی{\dif\سگما} کی صورت میں \عددی{\dif\بڑااومیگا} بھی بڑا ہوگا تناسبی جز ضربی \عددی{D(\تھیٹا)\equiv\dif\سگما/\dif\بڑااومیگا} کو تفریقی بکھراو عمودی تراش کہتے ہیں 
\begin{align}
	\dif\sigma = D(\theta)\dif\Omega
\end{align}
ٹکراو مقدار معلوم اور اسمتی زاویہ \عددی{\فاے} کی صورت میں \عددی{\dif\سگما=b\dif b\dif\فاے} اور \عددی{\dif\بڑااومیگا=\sin\تھیٹا\dif\تھیٹا\dif\فاے} ہوں گے لحاظہ درج ذیل ہوگا
\begin{align}
	D(\theta) = \frac{b}{\sin\theta}\abs{\frac{\dif b}{\dif\theta}}
\end{align}
چونکہ عمومی طور پر \عددی{\تھیٹا} مقدار معلوم \عددی{b} کا گٹھتا ہوا تفاعل ہوگا لحاظہ یہ تفرق در حقیقت منفی ہوگا اسی لیئے مطلق قیمت لی گئی ہے۔

\ابتدا{مثال}
\موٹا{سخت کرہ کے بکھراو کی مثال جاری رکھتے ہیں}۔ سخت کرہ بکھراو \حوالہء{مثال \num{11.1}} کی صورت میں 
\begin{align}
	\frac{\dif b}{\dif\theta}=-\frac{1}{2}R\sin\left(\frac{\theta}{2}\right)
\end{align}
لحاظہ درج ذیل ہوگا 
\begin{align}
	D(\theta) = \frac{R\cos(\theta/2)}{\sin\theta}\left(\frac{R\sin(\theta/2)}{2}\right) = \frac{R^2}{4}
\end{align}
اس مثال میں تفریقی عمودی تراش \عددی{\تھیٹا} کا تابع نہیں ہے جو ایک غیر معمولی بات ہے۔
\انتہا{مثال}
کل عمودی تراش تمام ٹھوس زاویوں پر \عددی{D(\تھیٹا)} کا تکمل ہوگا
\begin{align}
	\sigma\equiv\int D(\theta)\dif\Omega	
\end{align}
اندازاً بات کرتے ہوئے یہ آمدی شعاع کا وہ رقبہ ہوگا جسے ہدف بکھیرتا ہے۔ مثال کے طور پر سخت کرہ بکھراو کی صورت میں درج ذیل ہوگا
\begin{align}
	\sigma = (R^2/4)\int \dif\Omega = \pi R^2
\end{align}
جو ہمارے توقعات کے عین مطابق ہے۔ یہ کرہ کا رقبہ عمودی تراش ہے۔ اس رقبہ میں آمدی چھرے ہدف کو نشانہ بنائیں گے جبکہ اس سے باہر چھرے ہدف کو خطا کریں گے۔ یہی تصورات نرم اہداف مثلاً مرکزہ کا کولمب میدان کے لیئے بھی کار آمد ہے جن میں صرف نشانے پر لگنا یا نہ لگنا نہیں ہوگا۔

آخر میں فرض کریں ہمارے پاس آمدی ذرات کی یکساں شدت تابندگی کی ایک شعاع ہو 
\begin{align}
	\mathcal{L}\equiv\text{number of incident particles per unit area per unit time}
\end{align}
فی اکائی وقت رقبہ \عددی{\dif\سگما} میں داخل ہونے والے ذرات اور یوں ٹھوس زاویہ \عددی{\dif\بڑااومیگا} میں بکھراو والے ذرات کی تعداد \عددی{\dif N = \mathcal{L}\dif\سگما=\mathcal{L}D(\تھیٹا)\dif\بڑااومیگا}  ہوگی لحاظہ درج ذیل ہوگا 

\begin{align}
	D(\theta) = \frac{1}{\mathcal{L}}\frac{\dif N}{\dif\Omega}
\end{align}
چونکہ یہ صرف ان مقداروں کی بات کرتا ہے جنہیں تجربہ گاہ میں با آسانی ناپا جا سکتا ہو لحاظہ اس کو عموماً تفریقی عمودی تراش کی تعریف لیا جاتا ہے۔ اگر ٹھوس زاویہ \عددی{\dif\بڑااومیگا} میں بکھرے ذرات کو محسوس کار دیکھتا ہو تب ہم اکائی وقت میں معلوم شدہ ذرات کی تعداد کو \عددی{\dif\بڑااومیگا} سے تقسیم کر کے آمدی شعاع کی تابندگی کے لحاظ سے معمول شدہ کرتے ہیں۔

\ابتدا{سوال}
\موٹا{ردرفورڈ بکھراو}۔ بار \عددی{q_1} اور حرکی توانائی \عددی{E} کا ایک آمدی ذرہ ایک بھاری ساکن ذرہ جس کا بار \عددی{q_2} ہو 	سے بکھرتا ہے۔

(الف) ٹکراو مقدار معلوم اور زاویہ بکھراو کے بیچ رشتہ اغز کریں۔

جواب: \عددی{b=(q_1q_2/8\pi\epsilon_0E)\cot(\theta/2)}

(ب) تفریقی بکھراو عمودی تراش تعین کریں۔

جواب:
\begin{align}
	D(\theta)=\left[\frac{q_1q_2}{16\pi\epsilon_0E\sin^2(\theta/2)}\right]^2
\end{align}
(ج) دیکھائیں کہ ردرفورڈ بکھراو کا کل عمودی تراش لامتناہی ہوگا۔ ہم کہتے ہیں \عددی{1/r} مخفیہ لامتناہی ساتھ رکھتا ہے آپ کولمب قوت سے بچ نہیں سکتے ہیں۔
\انتہا{سوال}

