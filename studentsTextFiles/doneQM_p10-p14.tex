\documentclass{article}
\usepackage{amsmath}
\usepackage{polyglossia}             
\setmainlanguage[numerals=maghrib]{arabic}   
\setotherlanguages{english}
\newfontfamily\arabicfont[Scale=1.0,Script=Arabic]{Urdu Typesetting} 
\newfontfamily\urdufont[Scale=1.25,WordSpace=60.0,Script=Arabic]{Urdu Typesetting}
%\setlength{\parskip}{5mm plus 4mm minus 3mm}
\begin{document}
%حصہ
\section*{1.3.2  استمراری متغیرات}
اب تک ہم غیر مسلسل تغیرات کی بات کرتے آ رہے ہیں۔ جن کی قیمتیں الگ تھلگ ہوتی ہیں۔گزستہ مثال میں ہم نے افراد کی عمر کی بات کی جو سالوں میں ناپا جاتا ہے ، لہاذا j عدد صحیح تھا۔ ہم اس تسور کو توسیع دے کر  استمراری تقسیم کے لیے بھی استعمال کر سکتے ہیں۔ اگر میں گلی میں سکی شخص کا عمر پوچھوں تو اس کا امکان بہت کم ہوگا کہ وہ اپنی عمر 16 سال 4 گھنٹے ،27 منٹ اور 3.37524 سیکند بتاۓ۔ جواب یہ ہوگا کہ اس کا عمر 16 سے 17 سال کے بیچ ہے۔بہت کم وقفے کی صورت میں اہتمام وقفے کی لمبائی کے راس متناسب ہوگا۔مثال کے تور پر 16 سال قبل کسی مقسوس دن پر کسی وجہ سے بہت زیادہ بچے پیدا ہوئے ہوں تب ایسا نہیں ہوگا۔ایسی صورت میں ہم نے اہک یا دو دن کا وقفہ لیا ہو تو یہ وقفہ چھوٹا وقفہ تسور نہیں کیا جاۓ گا۔ہم مزید چھوٹا وقفہ لے کر اس مثال کو دوبارہ دیکھیں گے۔یوں ہم کہ سکتے ہیں کہ ہم لامتنای طور پر اہک چھوٹے وقفے کی بات کر رہے ہیں۔
\begin{align}
 p(x)dx =\lbrace ایک بلا واسطہ منتخب جزو کی قیمت x اور (x+dx)  
\end{align}
 کے بہچ ہونے کا احتمال ، اس مساوات میں تناسبتی مستقل $ \rho(x) $ کثافت احتمال کہلاتا ہے۔
 متناہی وقفہ a تا b کے درمیان x کا احتمال $ \rho(x) $ کے تکلم سے حاصل ہوگا۔
 \begin{align}
 P_{ab}=\int_a^b \rho(x)dx 
 \end{align}
 استمراری تقسیم کے لیے درج ذیل ہوں گے۔
\begin{align}
1=\int_{-\infty}^{\infty} \rho(x)dx  \\
<x> =\int_{-\infty}^{\infty} x\rho(x)dx \\
 \sigma^2=<(\Delta x)^2> = <x^2>-<x>^2 
\end{align}

\section*{مثال 1.1}
مثال کے طور پر اگر میں ایک چٹان جس کی بلندی h ہو سے اہک پٹھر کو نیچے گرنے دیتا ہوں۔جہسے جیسے پتھر گرتا ہے۔مہں بلا واسطہ وقتی فاصلوں پر اس کی دس لاکھ تصاویر کھینچتا ہوں اور میں ہر تصویر پر وہ فاصلہ ناپتا ہوں جو پتھر تہ کرتا ہے۔اب سوال پیدا ہوتا ہے کہ ان تمام فیصلوں کی اوسط قیمت کیا ہو گی۔یعنی ان کیوقتی اوسط کیا ہوگی۔ 
حل: 

پتھر ساکن صورتِحالسے سروع ہو کر بتدریج بھرتی رفتار سے نیچے گرتا ہے۔یہ چٹھان کے بلکل بالائی سر کے قریب زیادہ وقت گزارتا ہے۔لہازہ ہم توقع کرتے ہیں کی فاصلہ $   \frac{h}{2}   $ سے کم ہوگا۔ہم ہوائی رگر کو نزر انداز کر رہے ہیں۔ لمبائی پر فاصلہ $x$ درج زیل ہوگا۔
$$ x(t) = \frac{1}{2} gt^2 $$
اس کی سمتی رفتار $ \frac{dx}{dt}=gt $ ہوگی اور گرنے کا قل دورانیہ $           \sqrt{\frac{2h}{g}}  $ ہوگا۔ وقفہ dt میں تصویر کھینچنے کا احتمال $\frac{dt}{T}$ ہوگا۔یوں ایک تصویر اس کا احتمال کہ کسی ایک تصویر میں مطابقتی فاصلہ dx ہوگا درج  ذہل ہے۔ 
 \begin{align}
 \frac{dt}{T} = \frac{dx}{gt}\sqrt{g}{2h} = \frac{1}{2\sqrt{hx}}dx
\end{align}   
 ظاہر ہے کہ کسافتِ احتمال یصنی مساوات 1.14 درج ذیل ہوگا۔
 \begin{align}
 \rho(x)=\frac{1}{2\sqrt{hx}}  (0\leq x\leq h)
 \end{align}
 جبکہ اس وقفہسے باہر کسافتِ احتمال سِفر ہوگا۔ہم مساوات 1.16 استعمالکر کے اس نتیجےکی تصدیق کر سکتے ہیں۔
 \begin{align}
 \int_o^h \frac{1}{2\sqrt{hx}} =\frac{1}{2\sqrt{h}}(2x^{\frac{1}{2}})|_0^h =1
 \end{align}
 مساوات 1.17 سے اوسط فاصلہ معلوم کرتے ہیں
 \begin{align}
 <x> =\int_0^h x\frac{1}{2\sqrt{hx}}dx= \frac{1}{2\sqrt{h}} 2x^{\frac{3}{2}}|_0^h = \frac{h}{3}
 \end{align}
 جو  $ \frac{h}{2}$   سے کچھکم ہے جیساکہ ہم توقع رکھتے  تھے۔ شکل $1.6$ میں   $ \rho(x) $ کی ترسیم دیکھائی گئی ہے۔آپ دیکھ سکتے ہیں کہ کسافتِ احتمال ازخود لامتنای ہو سکتا ہے۔ اکرچے احتمال یعنی $ \rho $ کا تکمل ہر صورت متناہی ہوگا۔بلکہ یہ 1 کے برابر یا 1 سے کم ہوگا۔
\section*{ سوال 1.1} 
 حصہ 1.31 میں اشخاص کی عمر کی تقسیم کے لیے درج ذیل حاصل کریں۔ 
 (ا) $ <j^2>$ کی اوسط اور اوسط کا مربع $ ^2<j> $ حاصل کریں۔   
 (ب) ہر j  کے لیے $ \Delta j $ دریافت کریں اور مساوات 1.1 استعمال کرتے ہوۓ مایاری انحراف حاصل کریں۔
 (ج) ا اور ب کے نتائج استعمال کرتے ہوئے مساوات   $1.2  $ کی تسدیق کریں۔ 
 \section*{سوال 1.2} 
 (ا) مثال 1.1 کی تقسیم کے لیے مایاری انحراف تلاش کریں۔
 (ب) اس کا احتمال کتنا ہوگا کہ بلاواسطہ حاصل کی گئی تصویر میں فاصلہ اوسط قیمت کے فاصلے سے اہک مایاری انحراف زیادہ ہوگا۔
\section*{ گاوسی تقسیم}
 $$ P(x)=Ae^{-\lambda(x-a)^2} $$
 جہاں    $ A, \sigma $ $ \lambda $ اور مسبت حقیقی مستقل ہیں۔ درج ذیل حاصل کریں۔
   اگر ضرورت پیش آۓ تو تکمل کسی جدول سے دیکھ لیں۔
   (ا) مساوات 1.16 استعمال کرتے ہوئے  A کی قیمت معلوم کریں۔
   (ب)  اوسط $ x> , <x>^2>  اور   \sigma  $ تلاش کریں۔
   (ج)$ \rho(x) $ کی ترسیم کا خاکہ بنائیں۔
\end{document}
