
جزوحصہ{دوری جدول}
بھاری جوہروں کے زمینی حال الیکٹرانی تنظیم اسی طرح جوڑ کر حاصل کی جاتی ہے۔ پہلی تخمین کی حد میں انکی باہمی توانائی دفع کو مکمل طور پر نظرانداز کرتے ہوئے بار \عددی{Z_e} کے مرکزہ کے کولمب مخفیہ میں یک ذرہ ہائڈروجن حالات \عددی{(n, l, m)} جنہیں مدارچے کہتے ہیں کہ انفرادی الیکٹران مکین ہوں گے۔ اگر الیکٹران بوزان یا قابلِ ممیز ذرات ہوتے تب یہ زمینی حال \عددی{(1, 0, 0)} گر جاتے اور کیمیا انتی دلچسپ نہ ہوتی۔ حقیقت میں الیکٹران یکساں فرمیان ہے جن پر پولی اصول منات لاگو ہتا ہے لحاظہ کسی ایک مدارچہ میں صرف دو الیکٹران رہ سکتے ہیں ایک ہم میدان اور ایک خلاف میدان بلکہ یہ کہنا زیادہ درست کہ یکتا تنظیم میں الیکٹران رہ سکتے ہیں۔ کسی بھی \عددی{n} کی قیمت کے لیئے \عددی{n^2} ہائڈروجینی تفاعلات موج پائے جاتے ہیں جن میں سے ہر ایک کی توانائی \عددی{E_n} ہوگی یوں \عددی{n=1} خول میں دو الیکٹرانوں کی جگہ \عددی{n=2} خول میں آٹھ \عددی{n=3} میں اٹھارہ اور \عددی{n}ویں خول میں \عددی{2n^2} الیکٹرانوں کی جگہ ہوگی۔ کیفی طور پر بات کرتے ہئے دوری جدول کے اُفکی صف انفرادی خول کو بھرنے کے مترادف ہے اگرثہ یہ پوری کہانی نہیں ہے چونکہ ایسا ہونے کی صورت میں انکی لمبائیاں \عددی{2, 8, 18, 32, 50,} وغیرہ ہوتی نا کہ \عددی{2, 8, 8, 18, 18} وغیرہ ہم جلد دیکھیں گے کہ الیکٹرانوں کی باہمی توانائی دفع اس شمار کو کس طرح خراب کرتا ہے۔

ہیلیم کا \عددی{n=1} خول مکمل طور پر بھرا ہوگا لحاظہ اگلا جوہر لیتھیم \عددی{Z=3} کو ایک الیکٹران \عددی{n=2} خول میں رکھنا ہوگا۔ اب \عددی{n=2} کی صورت میں \عددی{l=0} یا \عددی{l=1} ہوسکتا ہے۔ تیسا الیکٹران ان میں سے کس ایک کا انتخاب کرے گا؟ چونکہ بوہر توانائی \عددی{n} پر منحصر ہوتی ہے نا کہ \عددی{l} پر لھاظہ الیکٹران کا باہمی عمل نہ ہونے کی صورت میں ان دونوں کی توانائی ایک دوسرے جیسی ہوگی۔

