	\باب{اصول و ضوابط}
	\حصہ{ہلبرٹ فضا}
	گشتہ دو ابواب میں سادہ ہارمونی نظاموں کے چند دلچسپ خواص ہماری نظروں سے گزرے۔ان میں سے کئی مخفیکو کی بنا تھے۔ مثلاً ہارمونی مرتعش میں توانائی کی سطح میں جفت فاصلے جبکہ باقی زیادہ عموماً نظر آتے ہیں، جنہیں ایک بار ثابت کرنا مفید ثابت ہوگا اِنکی مثالیں عدم یقینیت کا اصول اور ساکن حالات کی عمودیت ہے۔ اسکو ذہن میں رکھتے ہوئے اس باب میں نظریہ کو زیادہ مضبوط روپ میں پیش کیا جائے گا یہاں کوئی نئی بات نہیں کی جائے گی بلکہ مخصوص صورتوں میں دیکھے گئے خواص سے معقول نتائج اخذ کیا جائے گا۔
	
	کوانٹائی نطریہ کا دارومدار تفاعل موج اور عامل کے تصور پر مبنی ہے۔ نظام کے حال کو تفاعل موج ظاہر کرتی ہے۔ جبکہ قابلِ مشاہدہ خواص کو عاملین ظاہر کرتے ہیں۔ ریاضیاتی طور پر تصوراتی، سمتیات کی تعریفی، حالات پر تفاعل موج پورا اترتے ہیں۔ جبکہ عاملین ان پر خطی تبادلہ کے طور پر عمل کرتے ہیں۔ یوں کوانٹم میکانیات کی قدرتی زبان خطی الجبرائی ہے۔
	
	لیکن مجھے خدشہ ہے کہ اس طرز کی خطی الجبرا سے آپ واقف نہیں ہون گے۔ ایک بُعدی فضا میں سمتیہ\عددی{\rangle} \عددی{\upalpha} کو ایک مخصوص معیاری عمودی اساس
	\begin{align}
		|\upalpha \rangle \rightarrow a = \begin{pmatrix} 
			a_{1}\\
			a_{2}\\
			\vdots\\
			a_{N}
		 \end{pmatrix}
	\end{align}
کے لحاظ سے N عدد اجزاء  
\عددی{{a_n}}
 سے ظاہر کرنا سادہ ترین ثابت ہوتاہے۔ دو سمتیات کا اندرونی ضرب \عددی{\rangle \upbeta | \upalpha \langle}  تین بُعدی نقطہ ضرب کو وسط دیتے ہوئے درج ذیل مخلوط عدد ہوگا،
\begin{align}
	\langle \upalpha | \upbeta \rangle =a_1^{\ast}b_1 + a_2^{\ast}b_2 + \dotsb +a_N^{\ast}b_N
\end{align}
خطی تبادلہ T جنہیں اِنہی مخصوص اساس کے لحاظ سے قالب سے ظاہر کیا جاتا ہے۔ قالبی ضرب کے سادہ قواعد کے تحت سمتیات پر عمل کرتے ہوئے نئے سمتیات پیدا کرتا ہے۔
\begin{align}
	|\upbeta \rangle = T|\upalpha \rangle \rightarrow b = Ta = \begin{pmatrix}
		t_{11} & t_{12} & \hdots & t_{1N} \\ 
		t_{21} & t_{22} & \hdots & t_{2N} \\
		\vdots & \vdots &        & \vdots \\
		t_{N1} & t_{N2} & \hdots & t_{NN} 
	\end{pmatrix}
\begin{pmatrix}
	a_{1}\\
	a_{2}\\
	\vdots\\
	a_{N}
\end{pmatrix}
\end{align}
کوانٹم میکانیات میں پائے جانے والے سمتیات زیادہ تر تفاعل ہوتے ہیں جو لامتناہی بُعدی فضا میں رہتے ہیں انہیں N اجزائی قالب کے علامت سے ظاہر کرنا کچھ زیادہ ٹھیک نہیں ہوگا اور متناہی ابعادی صورت میں ٹھیک رکھنے والے ریاضیاتی عمل لامتناہی ابعادی صورت  میں پریشان کن صورت اختیار کر سکتے ہیں۔ اس کی بنیادی وجہ یہ ہے کہ اگرچہ  مساوات 2.3 متناہی مجموعہ ہر صورت موجود ہوگا لا متناہی مجموعہ یا تکمل عدم مرکوزیت کا شکار ہوسکتا ہے اور ایسی صورت میں اندرونی ضرب غیر موجود ہوگا۔ لحاظہ اندرونی ضرب پر مبنی کوئی بھی دلیل بے معنی ہوگی۔ یوں اگرچہ خطی الجبرا کی اصطلاحات اور علامتیت سے واقف ہوں گے بہتر ہوگا کہ یہاں آپ ہوشیار رہیں۔

متغیر x کے تمام تفاعل مل کر سمتی فضا پیدا کرتے ہیں، لیکن ہمارے لیئے یہ بہر بڑا ہوگا۔ کسی بھی ممکناتبی حال کو ظاہر کرنے کے لیئے ضروری ہے کہ تفاعل موج
\عددی{\Uppsi}
  معمول پر لانے کے قابل ہو:
\[\int |\Uppsi|^2 dx = 1\]
کسی مخصوص وقفہ پر تمام قابل تکامل مربع تفاعل
\begin{align}
	f(x) \text{\RL{that Such}} \int_{a}^b |f(x)|^2 dx < \infty
\end{align}
اس سے بہت چھوٹا سمتی فضا دے گا سوال 1.3 (الف) کعو دیکھیئے گا۔ ریاضی دان اسے $L_2 (a,b)$ کہتے ہیں جبکہ ماہرِ طبیعیات اسے ہلبرٹ فضا کہتے ہیں۔ یوں کوانٹم میکانیات میں
\begin{align}
	\text{\RL{تفاعل موج ہلبرٹ فضا میں بستے ہیں}}
\end{align}
ہم دو تفاعلوں کی اندرونی ضرب کی تعریف درج ذیل لیتے ہیں۔ جہاں \عددی{f(x)} اور \عددی{g(x)} دو تفاعل ہیں۔
\begin{align}
	\langle f | g \rangle \equiv \int_{a}^b f(x)^{\ast} g(x) dx
\end{align}
اگر f اور g  دونوں قابل مربع تکمل ہوں یعنی دونوں ہلبرٹ فضا میں پائے جاتے ہوں تب ہم ضمانت کے ساتھ کہہ سکتے ہیں کہ انکا اندرونی ضرب موجود ہوگا مساوات 6.3 کا تکمل ایک متناہی عدد پر مرکوز ہوگا۔ یہ شوارز عدم مساوات کی درج ذیل تکملی صورت کے پیشِ نظر ہوگا۔
\begin{align}
	\abs{\int_{a}^b f(x)^{\ast} g(x) dx} \leq \sqrt{\int_{a}^b \abs{f(x)}^2 dx \int_{a}^b \abs{g(x)}^2 dx}
\end{align}
آپ تصدیق کر سکتے ہیں کہ مساوات 6.3 اندرونی ضرب کی تمام شرائط پر پورا اتر تا ہے سوال 1۔3 (ب)۔ بلخصوص درج ذیل پر دیہان دیں
\begin{align}
	\langle g | f \rangle = \langle f | g \rangle^{\ast}
\end{align}  
مزید $f(x)$ کا اپنے ہی ساتھ اندرونی ضرب
\begin{align}
	\langle f | f \rangle = \int_{a}^b \abs{f(x)}^2 dx
\end{align}
حقیقی اور غیر منفی ہوگا یہ صرف اس صورت صفر ہوگا جب $f(x)=0$ ہو۔

ایک تفاعل اس صورت معمول شدہ کہلاتا ہے جب اسکا اپنے ہی ساتھ اندرونی ضرب ایک کے برابر ہو دو تفاعل اس صورت عمودی ہوں گے جب انکا اندرونی ضرب صفر ہو اور تفاعلوں کا سلسلہ\عددی{{f_n}} اس صورت معیاری عمودی ہوگا جب تمام معمول شدہ اور باہمی طور پر عمودی ہوں:
\begin{align}
	\langle f_m | f_n \rangle = \delta_{mn}
\end{align}
آخر میں تفاعلوں کا ایک سلسلہ اس صورت مکمل ہوگا جب ہلبرٹ فضا میں ہر تفاعل کو انکا خطی جوڑ لکھا جا سکے:
\begin{align}
	f(x) = \sum\limits_{n=1}^\infty c_{n} f_{n}(x)
\end{align}
معیاری عمودی تفاعلوں 
\عددی{f_n(x)}
 کے عددی سر فوریر تسلسل کے عددی سروں کی طرح حاصل کیئے جاتے ہیں:
\begin{align}
	c_{n} = \langle f_{n} | f \rangle
\end{align}
آپ اسکی تصدیق کر سکتے ہیں۔ میں نے باب 2 میں یہی اصطلاح استعمال کی تھی۔ لا متناہی چکور کنواں کے ساکن حالات مساوات 28.2 وقفہ (0,a) پر مکمل معیاری عمودی سلسلہ دیتے ہیں۔ ہارمونی مرتعش کے ساکن حالات مساوات (67.2 اور 85.2) وقفہ \عددی{(-\infty , \infty)} مکمل معیاری عمودی سلسلہ دیتے ہیں۔\\
سوال 1.3\\
(الف) دیکھائیں کہ تمام قابلِ تکمل مربع تفاعلوں کا سلسلہ سمری فضا دے گا آپ حصہ A.1 میں تعریف کا موازنہ کریں اشارہ: آپ نے دیکھنا ہوگا کے دو عدد قابلِ مربع تفاعلوں کا مجموعہ ازخود قابلِ تکمل مربع ہوگا مساوات 7.3 استعمال کریں۔ کیا تمام عمودی تفاعلوں کا سلسلہ سمری فضا ہوگا؟\\
(ب) دیکھائیں کہ مساوات 6.3 کا تکمل اندرونی ضرب ضرب کے تمام شرائط پر پورا اتر تا ہے حصہ A.2۔\\
سوال 2.3\\
(الف) تفاع  \عددی{f(x)=x^v} متغیر v کے کس مقداری سعت وقفہ (0,1) پر ہلبرٹ فضا میں ہوگا؟ متغیر v کو حقیقی تصور کریں جو ضروری نہیں مثبت ہو۔\\
(ب) کیا \عددی{v=\frac{1}{2}} کی صورت میں \عددی{f(x)} ہلبرٹ فضا میں پایا جائے گا؟ تفاعل \عددی{xf(x)}
  کے بارے میں آپ کیا کہیں گے؟ اور تفاعل /عددی{(\frac{d}{dx}) f(x)} کے بارے میں آپ کیا کہہ سکتے ہیں؟
\end{document}
