
\ابتدا{سوال}
آپ کو صرف کیفی تجزیہ کی اجازت ہے حساب کرکے نتیجہ اغز کرنے کی اجازت نہیں ہے۔ \حوالہء{شکل \num{2.21}} موین دہرا چکور کنواں پر غور کریں جہاں گرائی \عددی{V_0} اور چوڑائی \عددی{a} مقررہ ہیں جو اتنے بڑے ضرور ہیں کہ کئی مقید حال ممکن ہوں۔

(الف) زمینی تفاعل موج \عددی{\psi_1} اور پہلا ہیجان حال \عددی{\psi_2} کا خاکہ درج ذیل صورت میں کھینچیں۔

(ا) \عددی{b=0} (ب) \عددی{b\approx a} (ج) \عددی{b\gg a}

(ب) \عددی{b} کی قیمت صفر سے لامتناہی تک بڑھتے ہوئے توانائیاں \عددی{E_1} اور \عددی{E_2} کس طرح تبدیل ہوتی ہیں اس کا کیفی جواب دیں۔

(ج) دو جوہری سالمہ میں الیکٹران پر اثرانداز مخفی توانائی کا تاریخی یک دوری نمونہ دوہرا کنواں پیش کرتا ہے۔ مرکزوں کی قوت کشش کو دو کنویں ظاہر کرتی ہیں آزاد صورت میں یہ مرکزے کم سے کم توانائی کی صورت اختیار کریں گے۔ (ب) میں حاصل نتائج کے تحت کیا الیکٹران ان مرکزوں کو ایک دوسرے کے قریب کھینچے گا یا انہیں ایک دوسرے سے دور ہتنے پہ مجبور کرے گا۔ اگرچہ دو مرکزوں کے بیچ قوت دفع بھی پایا جاتا ہے لیکن اس کی بات یہاں نہیں کی جارہی ہے۔
\انتہا{سوال}
\ابتدا{سوال}
آپ نے \حوالہء{مساوات \num{2.39}} کے تسلسل کا مجموعہ لیتے ہوئے \حوالہء{سوال \num{2.7} (د)} میں توانائی کی توقعاتی قیمت تلاش کی جہاں ہاشیہ میں میں نے آپ کو آگاہ کیا کہ اس سے پرانے طریقہ \عددی{\langle H \rangle=\int\psi(x, 0)^*H\psi(x, 0)\dif x} سے حاصل نہ کیجیئے گا چونکہ \عددی{\psi(x, 0)} کے پہلے تفرق میں عدم استمرار دوسرے تفرق کو پریشان کن بناتا ہے۔ حقیقت میں آپ تکمل بالحصص کے ذریعے اسے حل کر سکتے تھے لیکن ڈراک ڈیلٹا تفاعل اس طرح کے انوکھے مسائل کے حل کرنے کا ایک بہترین طریقہ فراہم کرتا ہے۔

(الف) آپ \حوالہء{سوال \num{2.7}} میں \عددی{\psi(x, 0)} کا پہلا تفرق حاصل کر کے اسکو سیڑھی تفاعل \عددی{\theta(x-a/2)} کی صورت میں لکھیں جسے \حوالہء{مساوات \num{2.143}} میں پیش کیا گیا آخری سروں کی فکر نہ کریں صرف اندرونی خطہ \عددی{0<x<a} کے لیئے لکھیں۔

(ب) ابتدائی موجی تفاعل \عددی{\psi(x, 0)} کے دوسرے تفرق کو \حوالہء{سوال \num{2.24} (ب)} کا نتیجہ استعمال کرتے ہوئے ڈیلٹا تفاعل کی صورت میں لکھیں۔

(ج) تکمل \عددی{\int\psi(x, 0)^*H\psi(x, 0)\dif x} کو حل کرکے اس کے قیمت حاصل کریں اور تصدیق کریں کہ یہ جواب وہی ہے جو آپ نے پہلے حاصل کیا تھا۔
\انتہا{سوال}
\ابتدا{سوال}

(الف) دیکھائیں کہ ہارمونی مرتعش مخفی توانائی کے وقت کے غیر تابع شروڈنگر مساوات \حوالہء{مساوات \num{2.43}} پر درج ذیل پورا اترتا ہے
\begin{align*}
	\psi(x, t)=\left(\frac{m\omega}{\pi\hslash}\right)^{1/4}\exp\left[-\frac{m\omega}{2\hslash}\left(x^2+\frac{a^2}{2}(1+e^{-2i\omega t})+\frac{i\hslash t}{m}-2axe^{-i\omega t}\right)\right]
\end{align*}
یہاں \عددی{a} ایک حقیقی مستقل ہے جس کا بُعد لمبائی ہے۔

(ب) \عددی{\abs{\psi(x, t)}^2} تلاش کریں اور موجی اکتھ کی حرکت پر تبصرہ کریں۔

(ج) \عددی{\langle x \rangle} اور \عددی{\langle p \rangle} کا حساب لگائیں اور دیکھیں کہ آیہ مسئلہ اہرنفسٹ \حوالہء{مساوات \num{1.38}} پر یہ پورا اترتے ہیں۔
\انتہا{سوال}
\ابتدا{سوال}
درج ذیل حرکت کرتے ہوئے ڈیلٹا تفاعل کنواں پر غور کریں 
\begin{align*}
	V(x, t)=-\alpha\delta(x-\upsilon t)
\end{align*}
جہاں \عددی{\upsilon} ایک مستقل ہے کنواں کی سمتی رفتار کو ظاہر کرتا ہے۔

(الف) دیکھائیں کہ وقت کے غیر تابع شروڈنگر مساوات کا حل درج ذیل ہے
\begin{align*}
	\psi(x, t)=\frac{\sqrt{m\alpha}}{\hslash}e^{-m\alpha\abs{x-\upsilon t}/\hbar^2}e^{-i[(E+(1/2)m\upsilon^2)t-m\upsilon x]/\hslash}
\end{align*}
جہاں \عددی{E=-,\alpha^2/2\hslash^2} ساکن ڈیلٹا تفاعل کے مقیط حال کی توانائی ہے۔ اشارہ: اس حل کو شروڈنگر مساوات مویں پُر کر کے آپ تصدیق کر سکتے ہیں \حوالہء{سوال \num{2.24}(ب)} کا نتیجہ استعمال کریں۔

(ب) اس حال میں ہیملٹونی کی تواقعاتی قیمت تلاش کریں اور نتیجے پر تبصرہ کریں۔
\انتہا{سوال}
\ابتدا{سوال}
درج ذیل مخفی توانائی پر غور کریں 
\begin{align*}
	V(x)=-\frac{\hslash^2a^2}{m}\operatorname{sech}^2(ax)
\end{align*}
جہاں \عددی{a} ایک مثبت مستقل ہے۔

(الف) اس مخفی توانائی کو ترسیم کریں۔

(ب) تصدیق کریں کہ اس مخفی توانائی کا زمینی حال درج ذیل ہے
\begin{align*}
	\psi_0(x)=A\operatorname{sech}(ax)
\end{align*}
اور اسکی توانائی تلاش کریں۔ \عددی{\psi_0} کو معمول پر لائیں اور اس کا خط کھینچیں۔

(ج) دیکھائیں کہ درج ذیل تفاعل کسی بھی مثبت توانائی \عددی{E} کے لیئے شروڈنگر مساوات کو حل کرتا ہے 
\begin{align*}
	\psi_k(x)=A\left(\frac{ik-a\tanh(ax)}{ik+a}\right)e^{ikx}
\end{align*}
جہاں معمول کی طرح یہاں بھی \عددی{k\equiv\sqrt{2mE}/\hslash} ہے۔ چونکہ \عددی{z\to-\infty} کرنے سے \عددی{\tanh z\to -1} ہوگا لحاظہ \عددی{x} کی بہت بڑی منفی قیمتوں کے لیئے 
\begin{align*}
	\psi_k(x)\approx Ae^{ikx}, && \text{\RL{کے لیئے}}x\text{\RL{بڑی منفی}}
\end{align*}
جہاں \عددی{\exp(-ikx)} کی عدم موھودگی کی بنا یہ بائیں سے آپد ایک معج کو ظاہر کرتا ہے اور اس میں کوئی انعکاسی موج نہیں پایا جاتا۔ \عددی{x}  کی بڑی مثبت قیمتوں کے لیئے \عددی{\psi_k(x)} کی متاقاربی صورت  کیا ہوگی؟ اس مخفی توانائی کے لیئے \عددی{R} اور \عددی{T} کیا ہوں گے؟ یہ بلا انعکاس مخفی توانائی کی ایک بہت مشہور ایک مثال ہے کسی بھی توانائی کا ہر آمدی ذرہ اس سے سیدھا گزر جائے گا۔
\انتہا{سوال}
\ابتدا{سوال}
\موٹا{بکھراو قالب۔}مکامی مخفی توانائی کے لیئے بکھراو کا نظریہ ایک عمومی صورت اختیار کرتا ہے \حوالہء{شکل \num{2.22}}۔ بائیں ہاتھ خطہ ایک میں \عددی{V(x)=0} ہے لحاظہ درج ذیل ہوگا
\begin{align}
	\psi(x)=Ae^{ikx}+Be^{-ikx},&& k\equiv\frac{\sqrt{2mE}}{\hslash}\text{\RL{جہاں}}
\end{align}
دائیں ہاتھ خطہ تین جہاں بھی \عددی{V(x)=0} ہے درج ذیل ہوگا
\begin{align}
	\psi(x)=Fe^{ikx}+Ge^{-ikx}
\end{align}
ان دونوں کے بیچ خطہ دو میں میں مخفی توانائی جانے بغیر آپ کو \عددی{\psi} کے بارے میں کچھ نہیں بتا سکتا لیکن چونکہ شروڈنگر مساوات خطی ہے اور دو رتبی تفرقی مساوات ہے لحاظہ اس کا عمومی حل درج ذیل روپ کا ہوگا
\begin{align*}
	\psi(x)=Cf(x)+Dg(x)
\end{align*}
جہاں \عددی{f(x)} اور \عددی{g(x)} دو خطی غیر تابع مخصوص حل ہیں۔ یہاں چار عدد سرحدی شرائط ہوںگے جن میں سے دو خطہ ایک اور دو کو جوڑیں گے اور باقی دو خطہ دو اور تین کو جوڑیں گے۔ ان میں سے دو کو استعمال کرتے ہوئے \عددی{C} اور \عددی{D} کو خارج کر کے باقی دو کو حل کرتے ہوئے \عددی{A} اور \عددی{G} کی صورت میں \عددی{B} اور \عددی{F} تلاش کیئے جا سکتے ہیں
\begin{align*}
	B=S_{11}A+S_{12}G,&& F=S_{21}A+S_{22}G
\end{align*}
یہ چار عددی سر \عددی{S_{ij}} \عددی{k} پر منحصر ہوںگے لحاظہ \عددی{E} پر منحصر ہوںگے اور \عددی{2\times2} کا قالب ٍ\عددی{S} بنائیں گے جس سے بکھراو قالب یا مختصراً \عددی{S} قالب کہتے ہیں۔ قالب \عددی{S} آپ کو آتے ہوئے حیطوں \عددی{A} اور \عددی{G} کی صورت میں جاتے ہوئے حیطوں \عددی{B} اور \عددی{F} کی قیمت دیتے ہیں
\begin{align}
	\begin{pmatrix}
		B\\
		F
	\end{pmatrix}
	=
	\begin{pmatrix}
		S_{11} & S_{12}\\
		S_{21} & S_{22}
	\end{pmatrix}
	\begin{pmatrix}
		A\\
		G
	\end{pmatrix}
\end{align}
بائیں سے بکھراو کی صورت میں \عددی{G=0} ہوگا لحاظہ انعکاسی اور ترسیلی شرح درج ذیل ہوںگی
\begin{align}
	R_l=\frac{\abs{B}^2}{\abs{A}^2}\bigg|_{G=0}=\abs{S_{11}}^2,&& T_l=\frac{\abs{F}^2}{\abs{A}^2}\bigg|_{G=0}=\abs{S_{21}}^2
\end{align}
دائیں سے بکھراو کی صورت میں \عددی{A=0} ہوگا لحاظہ درج ذیل ہوںگے
\begin{align}
	R_r=\frac{\abs{F}^2}{\abs{G}^2}\bigg|_{A=0}=\abs{S_{22}}^2,&& T_r=\frac{\abs{B}^2}{\abs{G}^2}\bigg|_{A=0}=\abs{S_{12}}^2
\end{align}
(الف) ڈیلکٹا تفاعل کنواں \حوالہء{مساوات \num{2.114}} کے لیئے بکھراو کا \عددی{S} قالب تیار کریں۔

(ب) لامتناہی چکور کنواں \حوالہء{مساوات \num{2.145}} کے لیئے \عددی{S} قالب تیار کریں۔ اشارہ: مسئلے کی تشاکلی بروہِ کار لاتے ہوئے آپ کو کوئی نیا کام کرنے کی ضرورت نہیں ہوگی۔ 
\انتہا{سوال}
\ابتدا{سوال}
\موٹا{ترسیلی قالب۔} قالب \عددی{S} \حوالہء{سوال \num{2.52}} آپ کو جانے والے حیطے \عددی{B} اور \عددی{F} کو آنے والے حیطوں \عددی{A} اور \عددی{G} کی صورت میں پیش کرتا ہے \حوالہء{مساوات \num{2.175}}۔ بعض اوقات ترسیلی قالب \عددی{M} کے ساتھ کام کرنا زیادہ آسان ثابت ہوتا ہے جو مخفی توانائی کے دائیں جانب حیطوں \عددی{F} اور \عددی{G} کو بائیں جانب حیطوں \عددی{A} اور \عددی{B} کو صورت میں پیش کرتا ہے
\begin{align}
	\begin{pmatrix}
		F\\
		G
	\end{pmatrix}
	=
	\begin{pmatrix}
		M_{11} & M_{12} \\
		m_{21} & M_{22}
	\end{pmatrix}
	\begin{pmatrix}
		A\\
		B
	\end{pmatrix}
\end{align}
(الف) قالب \عددی{S} کے اجزا کی صورت میں قالب \عددی{M} کے چار اجزا تلاش کریں اسی طرح قالب \عددی{M} کے چار اجزا کی صورت میں قالب \عددی{S} کے اجزا تلاش کریں۔ \حوالہء{مساوات \num{2.176} اور \num{2.177}} میں دیئے گئے \عددی{R_l, T_l, R_r} اور \عددی{T_r} کو \عددی{M} قالب کے ارکان کی صورت میں لکھیں۔

(ب) فرض کریں آپ کے پاس ایک ایسی مخفی توانائی ہو جو دو تنہا ٹکڑوں پر مشتمل ہو \حوالہء{شکل \num{2.23}}۔ دیکھائیں کہ اس پورے نظام کا \عددی{M} قالب ان دو مخفی توانائیوں کے انفرادی \عددی{M} قالب کا حاصلِ ضرب ہوگا 
\begin{align}
	M=M_2M_1
\end{align}
ظاہر ہے کے آپ دو سے زیادہ عدد انفرادی مخفی توانائیں بھی استعمال کر سکتے تھے یہی \عددی{M} قالب کی انفردیت کا سبب ہے۔

(ج) نقطہ \عددی{a} پر واحد ایک ڈیلٹا تفاعل مخفی توانائی سے بکھراو کا \عددی{M} قالب تلاش کریں
\begin{align*}
	V(x)=-\alpha\delta(x-a)
\end{align*}
(د) جزو(ب) کا طریقہ استعمال کرتے ہوئے دوہرا ڈیلٹا تفاعل
\begin{align*}
	V(x)=-\alpha[\delta(x+a)+\delta(x-a)]
\end{align*}
کے لیئے \عددی{M} قالب تلاش کریں۔ اس مخفی توانائی کی ترسیلی شرح کیا ہوگی؟
\انتہا{سوال}
\ابتدا{سوال}
دم ہلانے کی ترکیب سے ہارمونی مرتعش کی زمینی حال کی توانائیوں کو پانچ معانی خیز ہندسوں تک تلاش کریں یعنی \عددی{K} کو تبدیل کرتے ہوئے \حوالہء{مساوات \num{2.72}} کو اعدادی طریقہ سے یوں حل کریں کہ \عددی{\xi} کی بڑی قیمت کے لیئے حاصل تفاعل موج صفر تک پہنچنے کی کوشش کریں۔ ماتھیمٹیکا میں درج ذیل پُر کرنے سے ایسا ہوگا
\begin{gather*}
\begin{aligned}
	\text{Plot}[\text{Evaluate}[u[x]/.\text{NDSolve}[{u''[x]-(x^2-K)^*u[x]==0, u[0]==1, u'[0]==0},u[x],{x, 10^{-8}, 10}, \text{MaxSteps}->\num{10000}]], {x, a, b}, \text{PlotRange}->{c, d}]
\end{aligned}
\end{gather*}
یہاں \عددی{a, b} ترسیم کی کی افکی ساتھ جبکہ \عددی{c, d} اسکی انتصابی ساتھ ہے \عددی{a=0, b=10, c=-10, d=10} لیتے ہوئے شروع کریں۔ ہم جانتے ہیں کہ اس کا درست جواب \عددی{K=1} ہے لحاظہ آپ \عددی{K=0.9} سے شروع کر سکتے ہیں۔ دیکھیں تفاعل موج کی دم کیا کرتی ہے۔ اب \عددی{K=1.1} لیں ہم دیکھیں گے کہ دم دوسری طرف چلے جائے گی۔ ان دونوں کی بیچ  کہیں پر درست حل موجود ہے۔ \عددی{K} کو درست جواب کے دونوں اطراف قریب سے قریب کرتے ہوئے درست جواب حاصل ہوگا۔
\انتہا{سوال}
\ابتدا{سوال}
دم ہلانے کا طریقہ \حوالہء{سوال \num{2.54}} استعمال کرتے ہوئے ہارمونی مرتعش کے ہیجانی حال کی توانائی پانچ با معانی ہندسوں تک تلاش کریں۔ پہلی اور تیسری ہیجان حال کے لیئے آپ کو \عددی{u[0]==0}  اور \عددی{u'[0]==1} لینا ہوگا۔
\انتہا{سوال}
\ابتدا{سوال}
دم ہلانے کی ترکیب سے لامتناہی چکور کنواں کی اوّلین چار توانائیوں کی قیمتہانچ با معانی ہندسوں تک تلاش کریں۔ اشارہ: \حوالہء{سوال \num{2.54}} میں کی تفرقی مساوات میں موضوں تبدیلیاں لائیں اس بار آپ کو \عددی{u(1)=0} پر نظر رکھنی ہوگی۔
\انتہا{سوال}

