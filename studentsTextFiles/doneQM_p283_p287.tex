\حصہ{نہایت مہین بٹوارہ}
پروٹان ازخود ایک مقناطیسی جفت کتب  ہے اگرچہ نسب نما میں کمیت کی بنا اس کا جفت کتب معیار اثر الیکٹران کے جفت کتب معیار اثر سے بہت کم ہوگا مساوات 6.60 
\begin{align}
\kvec{\mu}_p = \frac{g_p e}{2 m_p} \kvec{S}_p, \quad \kvec{\mu}_e = - \frac{e}{m_e} \kvec{S}_e
\end{align}
پروٹان ایک مخلوط ساخت کا ذرہ ہے جو تین کوارکوں پر مشتمل ہے لہذا اس کا مسکن مقناطیسی نسبت الیکٹران کی مسکن مقناطیسی نسبت کی طرح سادہ نہیں ہوگا جس کی بنا صريحى \عددی{g} جذب ضربی \عددی{g_p} لکھا گیا ہے جس کی پیمائشی قیمت 5.59 ہے جو الیکٹران کی قیمت دو سے مختلف ہے کلاسیکی برقی حرکیات کے تحت جفت کتب  \عددی{\mu} درج ذیل مقناطیسی میدان پیدا کرتا ہے 
\begin{align}
\kvec{B} = \frac{\mu_0}{4 \pi r^3} [3 (\kvec{\mu} \cdot \hat{r}) \hat{r} - \kvec{\mu}] + \frac{2 \mu_0}{3} \kvec{\mu} \delta^3 (\kvec{r})
\end{align}
یو پروٹان کے مقناطیسی جفت کتب معیار اثر سے پیدا مقناطیسی میدان میں الیکٹران کا ہیملٹنی درج ذیل ہوگا مساوات 6.58 
\begin{align}
H'_{hf} = \frac{\mu_0 g_p e^2}{8 \pi m_p m_e} \frac{[3(\kvec{S}_p \cdot \hat{r})(\kvec{S}_e \cdot \hat{r}) - \kvec{S}_p \cdot \kvec{S}_e]}{r^3} + \frac{\mu_0 g_p e^2}{3m_p m_e} \kvec{S}_p \cdot \kvec{S}_e \delta^3 ((\kvec{r}))
\end{align}
نظریہ اضطراب کے تحت توانائی کی اول رتبی تخفیف مساوات 6.9 اس طرح بھی ہيملٹنی کی توقعاتی قیمت ہوگی 
\begin{align}
E_{hf}^1 = \frac{\mu_0 g_p e^2}{8 \pi m_p m_e} \langle \frac{3(\kvec{S}_p \cdot \hat{r})(\kvec{S}_e \cdot \hat{r} - \kvec{S}_p \cdot \kvec{S}_e)}{r^3} \rangle + \frac{\mu _0 g_p e^2}{3 m_p m_e} \langle \kvec{S}_p \cdot \kvec{S}_e \rangle |\psi (0)|^2
\end{align}
زمینی ہال میں یا کسی دوسری ایسے حال میں جس میں \عددی{l = 0} ہو تفاعل موج کروی تشاکلی ہوگا لہذا اول توقعاتی قیمت صفر ہوگی سوال 6.27 دیکھیں ساتھ ہی مساوات 4.80 کے تحت \عددی{
|\psi_{100} (0)|^2 = 1/(\pi a^3)
} ہوگا لہذا زمینی ہال میں درج ذیل ہوگا 
\begin{align}
E_{hf}^1 = \frac{\mu_0 g_p e^2}{3 \pi m_p m_e a^3} \langle \kvec{S}_p \cdot \kvec{S}_e \rangle
\end{align}
چونکہ اس میں دو چکروں کے بیچ ضرب نقطہ پایا جاتا ہے لہذا اس کو چکر چکر ربط کہتے ہیں جیسا چکر مدار ربط میں \عددی{
\kvec{S} \cdot \kvec{L}
} پایا جاتا ہے چکر چکر ربط کی موجودگی میں انفرادی چکر زاویائی معیار اثر بقائی نہیں رہتے ہیں موزوں حالات کل چکر کے امتیازی سمتیات ہونگے 
\begin{align}
\kvec{S} \equiv \kvec{S}_e + \kvec{S}_p
\end{align}
پہلے کی طرح ہم اس کا مربع لے کر درج ذیل حاصل کرتے ہیں 
\begin{align}
\kvec{S}_p \cdot \kvec{S}_e = \frac{1}{2} (S^2 - S_e^2 - S_p^2)
\end{align}
اب الیکٹران اور پروٹون دونوں کا چکر ایک بٹا دو ہے لہذا \عددی{
S_e^2 = S_p^2 = (3/4) \hslash^2
} ہوگا سہ تا حال تمام چکر متوازی میں کل چکر ایک ہوگا جس کے تحت \عددی{
S^2 = 2 \hslash^2
} ہوگا یکتا حال میں کل چکر صفر لہذا \عددی{S^2 = 0} ہوگا یوں درج ذیل ہوگا 
\begin{align}
E_{hf}^1 = \frac{4g_p \hslash^4}{3m_p m_e^2 c^2 a^4} 
\begin{cases}
+ 1/4, & \text{\RL{سہ تا}} \\
- 3/4, & \text{\RL{یک تا}}
\end{cases}
\end{align}
چکر چکر ربط زمینی نیحال کے چکر انحطاط کو توڑ کر سہ تا تنظیم کو اٹھاتا جبکہ یک تا کو نیچے کرتا ہے شکل 6.13 یوں ان کے درمیان توانائی کا فاصلہ درج ذیل ہوگا 
\begin{align}
\Delta E = \frac{4g_p \hslash^4}{3m_p m_e^2 c^2 a^4} = 
\SI{5.88e-6}{\electronvolt}
\end{align}
سہ تا حال سے یک تا حال انتقال کے دوران خارج فوٹان کا تعدد درج ذیل ہوگا 
\begin{align}
\nu = \frac{\Delta E}{h} = \SI{1420}{\mega \hertz}
\end{align}
اور اس کی مطابقتی طول موج \عددی{
c/ \nu = \SI{21}{\centi \meter}
} ہوگی جو خود موج خطے میں پایا جاتا ہے یہ کائنات میں اخراج کی صورت میں وہ مشہور 21 سینٹی میٹر تحفی خط ہے جو ہر طرف پایا جاتا ہے 
\ابتدا{سوال} 
فرض کریں \عددی{\kvec{a}} اور \عددی{\kvec{b}} دو مستقل سمتیات ہیں درج ذیل دکھائیں 
\begin{align}
\add (\kvec{a} \cdot \kvec{\hat{r}}) (\kvec{b} \cdot \kvec{\hat{r}}) \sine \theta d\theta d \phi = \frac{4 \pi}{3} (\kvec{a} \cdot \kvec{b})
\end{align}
تکمل ہمیشہ کی طرح \عددی{
0 < \theta < \pi 
}، \عددی{
0 < \phi < 2 \phi 
} کر لیا گیا ہے اس نتیجہ کو استعمال کرتے ہوئے ان حالات کے لئے جن کے لیے \عددی{l = 0} ہو درج ذیل دکھائیں 
\begin{align*}
\langle \frac{3 (\kvec{S}_p \cdot \hat{r})(\kvec{S}_e \cdot \hat{r}) - \kvec{S}_p \cdot \kvec{S}_e}{r^3} \rangle = 0
\end{align*}
اشارہ: \عددی{
\hat{r} = \sine \theta \cosine \phi \hat{i} + \sine \theta \sine \phi \hat{j} + \cosine \theta \hat{k}
} 
\انتہا{سوال}
\ابتدا{سوال}
ہائیڈروجن کلیہ میں موزوں ترمیم کرتے ہوئے درج ذیل کے لیے زمینی حال کی مہین ساخت تعین کریں (الف) میونی ہائیڈروجن جس میں ایکٹر ان کی بجائے میون ہوگا جس کا بار اور \عددی{g} جزو ضرب الیکٹرون کے بار اور \عددی{ g} جزو ضرب کے برابر لیکن کمیت \عددی{207} گنا زیادہ ہے ( ب) پوزیٹرانیا جس میں پروٹان کی جگہ پوزیٹران ہوگا جس کی کمیت اور \عددی{ g} جزو ضرب اور الیکٹران کی کمیت اور \عددی{ g} جزو ضرب لیکن علامت الٹ ہے ( ج) میونیم جس میں پروٹان کی جگہ زد میون ہوگا جس کی کمیت اور \عددی{ g} جزو ضرب 
\انتہا{سوال}

