
\حصہ{اشعاعی اخراج اور انجذاب}
\جزوحصہ{برقناطیسی امواج}
ایک برقناطیسی موج جس کو میں رشنی کہوں گا اگرچہ یہ زیریں سرخ، بالائے بصری شعاع، خرد امواج، ایکس رے وغیرہ ہو سکتی ہے۔ جن میں صرف تعدد کا فرق ہوتا ہے۔ عرضی اور باہم قائمہ ارتعاشی برقی اور مقناطیسی میدانوں پر مشتمل ہوگا شکل \num{9.3}۔ ایک جوہر گزرتی ہوئی بصری موج کی موجودگی میں بنیادی طور پر صرف برقی جز کو ردعمل دیتا ہے۔ اگر طولِ موج جوہر کی جسامت کے لحاظ سے لمبی ہو تب ہم میدان کی فضائی تغیر کو نظرانداز کرسکتے ہیں۔ تب جوہر سائن نما ارتعاشی برقی میدان
\begin{align}
	E = E_0\cos(\omega t)\hat{k}
\end{align}
کے زیرِ اثر ہوگا۔ فل حال میں فرض کرتا ہوں کہ روشنی یک رنگی اور \عددی{z} رخ ترتیب شدہ ہے۔ اضطرابی ہیملٹنی درج ذیل ہوگا جہاں \عددی{q} الیکٹران کا بار ہے 
\begin{align}
	H^\prime =-qE_0z\cos(\omega t)
\end{align}	
ظاہر ہے درج ذیل ہوگا
\begin{align}
	H^{\prime}_{ba} =-pE_0 \cos(\omega t). \text{where} p \equiv q\langle \phi_b \abs{z}\phi_a\rangle
\end{align}
عمومی طور پر \(\psi\) متغیر \عددی{z} کا جفت یا طاق تفاعل ہوگا یہ ہماری اُس مفروضہ کا سبب ہے جس کے تحت ہم کہتے ہیں کہ \(H^\prime\) کے وتری قالبی ارکان صفر ہوں گے۔ یوں روشنی اور مادہ کا باہم عمل ٹھیک اُسی قسم کے ارتعاشی اضطراب کہ تحت ہوگا جن پر ہم نے حصہ 9.3.1 میں غور کیا۔ یہاں درج ذیل ہوگا۔  
\begin{align}
	V_{ba} = -pE_0
\end{align}
\جزوحصہ{انجزاب، تحرق شدہ اخراج اور خود با خود اخراج}
ایک جوہر جو ابتدائی طور پر زیری حال \(\phi_a\) میں پایا جاتا ہو پر تقطیب شدہ یک رنگی روشنی کی شعاع ڈالی جاتی ہے۔ بالائی حال \(\psi_b\) میں انتقال کا احتمال مساوات \num{9.28} دیتی ہے جو مساوات \num{9.34} کی روشنی میں درج ذیل روپ اختیار کرتی ہے۔
\begin{align}
	P_{a\rightarrow b} (t) = (\frac{\abs{p}E_0}{\hbar})^2 \frac{\sin^2[(\omega_0 - \omega)t/2]}{(\omega_0 - \omega)^2}
\end{align}
اس عمل میں برقناطیسی میدان سے جوہر \(E_b - E_a = \hbar\omega_0\) توانائی جزب کرتا ہے۔ ہم کہتے ہیں اس میں ایک فوٹان جزب کیا شکل \num{9.4} (الف) جیسا میں ذکر کر چکا ہوں لفظ فوٹان در حقیقت کوانٹم برقی حرقیات برقناطیسی میدان کی کوانٹم نظریہ سے تعلق رکھتا ہے جبکہ ہم میدان کو کلاسیکی نقطہ نظر سے دیکھ رہے ہیں۔ یہ زبان اُس وقت تک استعمال کرنا مناسب ہے جب تک آپ اس سے زیادہ گہرا مطلب نہ لیں۔

یقیناً میں بالائی حال \((c_a(0)=0, c_b(0)=1)\) سے آغاز کرتے ہوئے پورا عمل دوبارہ کرسکتا ہوں۔ آپ سے گزارش ہے کہ ایسا کریں نتائج بلکل وہی ہوں گے البتہ اس بار \(P_{b\rightarrow a} = \abs{C_a(t)}^2\) حاصل ہوگا جو نیطے رخ زیریں لیول میں منتقل کا احتمال ہوگا۔
\begin{align}
	P_{b\rightarrow a} (t) = (\frac{\abs{p}E_0}{\hbar})^2 \frac{\sin^2[(\omega_0 - \omega)t/2]}{(\omega_0 - \omega)^2}
\end{align}
چونکہ ہم \(a\leftrightarrow b\) کو آپس میں بدل رہے ہیں جو \(\omega_0\) کی جگہ \(-\omega_0\) ڈالتا ہے لحاظہ لاظماً یہی نتیجہ حاصل ہوتا مساوات \num{9.25} پر اب پہنچ کر ہم پہلا جز چنتے ہیں جس کے نصب نما میں \(-\omega_0 + \omega\) پایا جاتا ہے باقی حصاب پہلے کی طرح ہے لیکن اگر آپ ایک بار رک کر سوچیں تو یہ نتیجہ حیرت انگیز ہے۔ بالائی حال میں پائے جانے والے ذرہ پر روشنی کی شعاع ڈالنے سے ذرہ زیریں حال میں منتقل ہوتا ہے اور اس ک احتمال بلکل ٹھیک وہی ہوگا جو زیریں حال سے بالائی حال منتقلی کا ہے اس عمل کو تحرق زدہ اخراج کہتے ہیں۔ جس کی پیشً گوئی آئینسٹائین نے ی تھی۔

تحرق زدہ اخراج کی صورت میں براقناطیسی میدان توانائی \(\hbar\omega_0\) جوہر سے حاصل کرتا ہے۔ ہم کہتے ہیں ایک فوٹان داخل ہوا اور دو فوٹان ایک اصل جس نے تحرق پیدا کیا اور ایک تحرق کی بنا پیدا باہر نکلے شکل \num{9.4} (ب)۔ اگر ایک بوتل میں بہت سارے جوہر بالائی حال میں ہوں تب واحد ایک آمدی فوٹان دو فوٹان پیدا کرے گا اور یہ دو فوتان از خود چار پیدا کریں گے وغیرہ وغیرہ۔ یوں ایمپلیفیکیشن ممکن ہوگا تقریباً ایک ہی وقت پر ایک ہی تعدد کی بہت بڑی تعداد کے فوٹان خارج ہوں گے لیزر اسی اصول کے تحت پیدا کی جاتی ہے۔ دیہان رہے کہ لیزر عمل کے لیئے ضروری ہے کہ جوہر کی اکثریت کو بالائی حال میں جائے جس کو پاپولیشن انورزن کہتے ہیں چونکہ انجزاب ھس کی بنا ایک فوٹان کم ہوتا ہے تحرقی اخراج جو ایک پیدا کرتا ہے بل مقابل ہوں گے لحاظہ دونوں حالات کی برابر تعداد سے آغاز کرتے ہوئے ایمپلیفیکیشن پیدا نہیں ہوگا۔

انجزاب اور تحرقی اخراج کے ساتھ ساتھ روشنی اور مادہ کی باہم عمل کا ایک تیسرا طریقہ بھی پایا جاتا ہے جس کو خود با خود اخراج کہتے ہیں۔ اس میں بیرونی برقناطیسی میدان کی عدم موجودگی میں جو اخراج پیدا کرسکتا ہے ہیجان جوہر زیریں حال میں منتقل ہو کر ایک فوٹان خارج کرتا ہے شکل \num{9.4} (ج)۔ 	ہیجان حال سے ایک جوہر عموماً اسی زریعہ زمینی حال میں پہنچتا ہے پہلی نظر میں یہ سمجھ نہیں آتی کہ خود با خود اخراج کیوں کر ہو گا۔ ایک ساکن حال اگرچہ ہیجان جوہر کو کیا ضرورت پیش آتی ہے کہ وہ بیرونی اضطراب کی عدم موجودگی میں زمینی حال کو منتقل ہو۔ درحقیقت ایسا ہی ہوتا اگر اس پر کسی قسم کا بیرونی اضطراب اثر انداز نہ ہوتا۔ درحقیقت کوانٹم برقی حرقیات میں زمینی حال میں بھی میدان غیر صفر ہوتے ہیں۔ مثلاً ہارمونی مرتعش زمینی حال میں بھی غیر صفر توانائی \(\hbar\omega/2\) کا حامل ہوگا۔ آپ تمام روشنی کو روک لیں جوہر کو مطلق صفر حرارت پر لے جائیں تب بھی برقناطیسی شعاع پائی جائے گی اور یہی صفر نقطی اخراج خود با خود اخراج کا سبب بنتی ہے۔ اگر جڑ سے دیکھا جائے تو درحقیقت تمام اخراج تحرقی اخراج ہوگی۔ آپ کو یہ امتیاز کرنا ہو گا کہ آیہ آپ نے میدان پیدا کیا یا قدرت نے اس نقطہ نظر سے یہ کلاسیکی اخراجی عمل کے بلکل اُلٹ ہے جہاں تمام خراج خود با خود ہوتا ہے اور تحرقی اخراج کا تصور نہیں پایا جاتا ہے۔

