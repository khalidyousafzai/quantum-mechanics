
\حصہ{سرنگزنی}
اب تک میں \عددی{E>V} فرض کرتا رہا ہوں لحاظہ \عددی{V(x)} حقیقی تھا۔ میں غیر کلاسیکی خطہ \عددی{E<V} کے لیئے بھی بلکل اسے طرح مطابقتی نتیجہ لکھ سکتا ہوں جو عین \حوالہء{مساوات \num{8.10}} ہوگا تاہم اب \عددی{p(x)} تخیلی ہوگا
\begin{align}
	\boxed{\psi(x)\cong\frac{C}{\sqrt{\abs{p(x)}}}e^{\pm \frac{1}{\hslash}\int\abs{p(x)}\dif x}}
\end{align}
ایک مثال کے طور پر ایک مستطیلی رکاوٹ جس کی بالائی سطح غیر ہموار ہو \حوالہء{شکل \num{8.3}} سے بکھراو کا مسئلہ پر غور کریں۔ درکاوٹ کے بائیں جانب \عددی{x<0} 
\begin{align}
	\psi(x)=Ae^{ikx}+Be^{-ikx}.
\end{align}
جہاں \عددی{A} آمدی حیطہ اور \عددی{B} منعکس حیطہ ہے جبکہ \عددی{k\equiv\sqrt{2mE}/\hslash} ہے \حوالہء{حصہ \num{2.5}} دیکھیں۔ دکاوٹ کے دائیں جانب \عددی{x>a}
\begin{align}
	\psi(x)=Fe^{ikx};
\end{align}
\عددی{F} ترسیلی حیطہ جبکہ ترسیلی احتمال درج ذیل ہوگا
\begin{align}
	T=\frac{\abs{F}^{2}}{\abs{A}^{2}}.
\end{align}
سرنگزنی خطہ \عددی{0\leq x\leq a} میں ونزل، کرامرز، برلوان تخمین درج ذیل دیگی
\begin{align}
	\psi(x)\cong\frac{C}{\sqrt{\abs{p(x)}}}e^{\frac{1}{\hslash}\int^{x}_{0}\abs{p(x')}\dif x'}+\frac{D}{\sqrt{\abs{p(x)}}}e^{-\frac{1}{\hslash}\int^{x}_{0}\abs{p(x')}\dif x'}.
\end{align}
اگر رکاوٹ بہت بلند یا اور بہت چوڑا ہو یعنی جب سرنگزنی کا احتمال بہت کم ہو قوت نمائی بڑھتے جزو کا عددی سر \عددی{C} لاظماً چھوٹا ہوگا درحقیقت لامتناہی چوڑے رکاوٹ کی صورت میں یہ صفر ہوگا اور تفاعل موج کچھ \حوالہء{شکل \num{8.4}} کے نقش پر ہوگی۔ غیر کلاسیکی خطہ پر قوتِ نمائی میں کل کمی 
\begin{align*}
	\frac{\abs{F}}{\abs{A}}\sim e^{-\frac{1}{\hslash}}\int^{a}_{0}\abs{p(x')}\dif x'.
\end{align*}
آمدی اور ترسیلی امواج کے اظافی حیطے تعین کرتا ہے لحاظہ درج ذیل ہوگا
\begin{align}
	\boxed{T\cong e^{-2\gamma},   \text{\RL{جہاں}}   \gamma\equiv\frac{1}{\hslash}\int^{a}_{0}\abs{p(x)}\dif x}.۔
\end{align}
\ابتدا{مثال}
\موٹا{ایلفا تحلیل کا نظریہ گامو۔} سن \num{1928} میں جارج گامو نے \حوالہء{مساوات \num{8.22}} استعمال کرتے ہوئے ایلفا تحلیل کی پہلی کامیاب وجہ پیش کی ایلفا تحلیل سے مراد چند مخصوص تابکار مرکزہ سے ایلفا ذرہ جو دو پروٹان اور دو نیوٹران پر مشتمل ہوتا ہے کا اخراج ہے۔ چونکہ ایلفا ذرہ مثبت بار \عددی{2e} کا حامل ہے لحاظہ جیسے ہی یہ مرکزہ سے اتنا دور ہوجاتا ہے کہ یہ مرکزی بندشی قوت سے فرار کر سکے مرکزہ کے باقی حصہ کا بار \عددی{Ze} اس کو برقی قوتِ دفع سے دور جانے پر مجبور کرے گا۔ تاہم اسکو پہلے اس مخفی رکاوٹ سے گزرنا ہوگا جو یورینیم کی صورت میں خارجی ایلفا ذرہ کی توانائی سے دوگنے سے بھی زیادہ ہے۔ گامو نے اس مخفی توانائی کو تخمینی طور  پر \حوالہء{شکل \num{8.5}} کے مخفیہ سے ظاہر کیا جس نے مرکزہ کے رداس \عددی{r_1} وصت تک مرکزی قوتِ کشش کو متناہی چکور کنواں سے ظاہر کیا گیا جس کو کولومب قوتِ دفع کی دم کے ساتھ جوڑا گیا ہے۔ گامو نے کوانٹم سرنگزنی کو ایلفا ذرہ کی فرار کی وجہ کرار دیا یوں پہلی بار کوانٹم میکانیات کا اطلاق مرکزوی طبیعیات پر کیا گیا۔

اگر خارجی ایلفا ذرے کی توانائی \عددی{E} ہو تب بیرونی واپسی نقطہ \عددی{r_2} درج ذیل تعین کرے گا
\begin{align}
	\frac{1}{4\pi\epsilon_{0}}\frac{2Ze^{2}}{r_{2}}=E.
\end{align}
ظاہر ہے \حوالہء{مساوات \num{8.22}} میں قوت نما \عددی{\gamma} درج ذیل ہوگا
\begin{align*}
	\gamma=\frac{1}{\hslash}\int^{r_{2}}_{r_{1}}\sqrt{2m\Big(\frac{1}{4\pi\epsilon_{0}}\frac{2Ze^{2}}{r}-E\Big)}\dif r=\frac{\sqrt{2mE}}{\hslash}\int^{r_{2}}_{r_{1}}\sqrt{\frac{r_{2}}{r}-1}\dif r.
\end{align*}
اس تکمل میں \عددی{r\equiv r_2\sin^2u} پُر کرتے ہوئے نتیجہ حاصل کیا جاسکتا ہے
\begin{align}
	\gamma=\frac{\sqrt{2mE}}{\hslash}\left[r_{2}\left(\frac{\pi}{2}-\sin^{-1}\sqrt{\frac{r_{1}}{r_{2}}}\right)-\sqrt{r_{1}(r_{2}-r_{1})}\right].
\end{align}
عام طور پر \عددی{r_1\ll r_2} ہوگا لحاظہ ہم چھوٹے زاویوں کے تخمین \عددی{\sin\epsilon\cong\epsilon} استعمال کر کے نتیجہ کی سادہ روپ حاصل کرتے ہیں
\begin{align}
	\gamma\cong\frac{\sqrt{2mE}}{\hslash}\left[\frac{\pi}{2}r_{2}-2\sqrt{r_{1}r_{2}}\right]=K_{1}\frac{Z}{\sqrt{E}}-K_{2}\sqrt{Zr_{1}}.
\end{align}
جہہاں 
\begin{align}
	K_{1} \equiv \left(\frac{e^{2}}{4\pi\epsilon_{0}}\right)\frac{\pi\sqrt{2m}}{\hslash}= \SI{1.980}{\mega\electronvolt^{1/2}},
\end{align}
اور درج ذیل ہوگا
\begin{align}
	K_{2} \equiv \left(\frac{e^{2}}{4\pi\epsilon_{0}}\right)^{1/2}\frac{4\sqrt{m}}{\hslash}= \SI{1.485}{\femto\meter^{-1/2}}.
\end{align}
ایک عمومی مرکزہ کی جسامت تقریباً ایک \si{\femto\meter} \SI{e-15}{\meter} ہوتی ہے۔

اگر ہم مرکزہ کے اندر ایلفا ذرہ کو محسور تصور کریں اور کہیں کہ اسکی اوسط سمتی رفتار \عددی{v} ہے تب دیواروں کے ساتھ تصادم کے بیچ اوسط وقفہ تقریباً \عددی{2r_1/v} ہوگا لحاظہ تصادم کا تعدد \عددی{v/2r_1} ہوگا۔ ہر تصادم پر فراق ہونے کا احتمال \عددی{e^{-2\gamma}} ہے لحاظہ اکایئ وقت میں اخراج کا احتمال \عددی{(v/2r_1)e^{-2\gamma}} ہوگا اور یوں ولدہ مرکزہ کا عرصہ حیات تقریباً درج ذیل ہوگا
\begin{align}
	\tau=\frac{2r_{1}}{v} e^{2\gamma}.
\end{align}
بدقسمتی سے ہم \عددی{v} نہیں جانتے ہیں لیکن اس سے زیادہ فرق نہیں پڑتا ہے چونکہ ایک تابکار مرکزہ سے اور دوسرے تابکار مرکزہ کے بیچ قوتِ نمائی جزضربی پچیس رتبی مقدار تک تبدیل ہوتا ہے جس کے سامنے \عددی{v} کی تبدیلی قابلِ نظرانداز ہے۔ بالخصوص عرصہ حیات کی تجرباتی پیمائشی قیمتوں کو \عددی{1/\sqrt{E}} کے ساتھ ترسیم کرنے سے ایک خوبصورت سیدھا خط \حوالہء{شکل \num{8.6}} حاصل ہوتا ہے جو عین \حوالہء{مساوات \num{8.25} اور \num{8.28}} کے تحت ہوگا۔
\انتہا{مثال}
\ابتدا{سوال}
ایک متناہی چکور رکاوٹ جس کی انچائی \عددی{V_0>E} اور چوڑائی \عددی{2a} ہو سے ایک ایسا ذرہ جس کی توانائی \عددی{E} ہو کی تخمینی ترسیمی احتمال \حوالہء{مساوات \num{8.22}} استعما کرتے ہوئے حاصل کریں۔ اپنے جواب کا موانہ بلکل ٹھیک نتیجہ \حوالہء{سوال \num{2.33}} کے ساتھ کریں۔
\انتہا{سوال}
\ابتدا{سوال}
\حوالہء{مساوات \num{8.25} اور \num{8.28}} استعمال کرتے ہوئے \عددی{U^{238}} اور \عددی{Po^{212}} کے عرصہ حیات تلاش کریں۔ تمام مرکزہ میں مرکزوی مادہ کی کثافت تقریباً مستقل ہوتی ہے لحاظہ \عددی{(r_1)^3} اور \عددی{A} پروٹان اور نیوٹرانوں کی تعدادوں کا مجموعہ تقریباً برابر ہوںگے۔ تجرباتی طور پر درج ذیل حاصل کیا گیا ہے
\begin{align}
	r_{1}\cong(\SI{1.07}{\femto\meter})A^{1/3}.
\end{align}
خارج شدہ ایلفا ذرہ کی توانائی کلیہ آئنسٹائن \عددی{E=mc^2} سے اغز کیا جاسکتا ہے
\begin{align}
	E=m_{p}c^{2}-m_{d}c^{2}-m_{\alpha}c^{2}.
\end{align}
جہاں \عددی{m_p} ولدہ مرکزہ کی کمیت \عددی{m_d} بیٹی مرکزہ کی کمیت اور \عددی{m_\alpha} ایلفا ذرہ یعنی \عددی{He^4} مرکزہ کی کمیت ہے۔ یہ دیکھنے کی خاطر کہ بیٹی مرکزہ کیا ہوگا یاد رکھیں کہ ایلفا ذرہ دو پروٹان اور دو نیوٹران لیکر فرار ہوتا ہے لحاظہ \عددی{Z} سے دو منفی کریں اور \عددی{A} سے چار منفی کریں گے۔ حاصل جوابات استعمال کرتے ہوئے دوری جدول سے کیمیائی انصر تعین کریں۔ سمتی رفتار \عددی{v} کی اندازاً قیمت \عددی{E=(1/2)m_\alpha v^2} سے حاصل کریں یہ مرکزہ کے اندر منفی مخفی توانائی کو نظرانداز کرتا ہے لحاظہ \عددی{v} کی قیمت اصل سے زیادہ دیگا تاہم اس مرحلہ پر ہم صرف اتنا ہی کر سکتے ہیں۔ اتفاقی طور پر ان کیمیائی اناصر کی تجربہ سے حاصل عرصہ حیات بالترتیب \عددی{6\times10^9} سال اور \SI{0.5}     {\micro\second} ہے۔
\انتہا{سوال}

