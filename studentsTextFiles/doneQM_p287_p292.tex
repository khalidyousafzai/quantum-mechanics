
سوال6.32:
\\
فرض کریں ایک مخصوص کوانٹم نظام کا  
Hamiltonian 
کسی مقدار معلوم
\(\lambda\)
 کا تفعال ہو .
\(H(\lambda)\)
کے امتیازی اقدار کو اور امتیازی تفعالات
\(E_{n}(\lambda)\)
اور
\(\psi_{n}(\lambda)\)
لیں۔ 
مسلہ Feynman-Hellmann درج ذیل کہتا ہے
\[\frac{\partial E_{n}}{\partial \lambda}=\big\langle{\psi_{n}|\frac{\partial{H}}{\partial{\lambda}}|\psi_{n}}\big\rangle\]
جہاں 
\(E_{n}\)
کو غیر انحطاطی تصور کریں اور اگر انحطاطی ہوں تب تمام 
\(\psi_{n}\)
کو انحطاطی امتیازی تفعالات کے موضوع خطی جوڑ تصور کریں۔\\
(جزو الف):مسلہ Feynman-Hellmann ثابت کریں۔(اشارہ : مسلہ 6.9 استمال کریں ۔ )\\
(جزو ب): درج ذیل یقبودی هارمونی مدار اسکا اطلاق کریں۔\\
ایک)\\
\(\lambda=\omega\)\\
لیں جس سے
V
کی توقعاتی قیمت کا کلیہ اخذ ہوگا ۔\\
دو)\\
\(\lambda=\hslash\)\\
لیں جو
\(\langle T \rangle\)
دے گا اور\\
تین)\\
\(\lambda=m\)\\
جو 
\(\langle T \rangle\)
اور 
\(\langle V \rangle\)
کے درمیان رشتہ دے گا۔اپنے جوابات کا سوال2.12 اور مسلہ virial کی پیشنگویوں  کے ساتھ موعازنا کریں ۔\\
سوال 6.33 : 
\\
مسلہ Feynman-Hellmann استعمال کرتے ہوے ھاٰےڈروجنکے لئے 
\(1/r\)
اور 
\(1/r^{2}\)
کی توقعاتی قیمتیں تین کی جا سکتی ہیں راداسی تفعالات امواج کا موثر 
Hamiltonian 
مساوات 4.53 درج ذیل ہے :
\[H=-\frac{\hslash^{2}}{2m}\frac{d^{2}}{dr^{2}}+\frac{\hslash^{2}}{2m}\frac{l(l+1)}{r^{2}}-\frac{e^{2}}{4\pi\epsilon}\frac{1}{r}\]
اور امتیازی اقدار جنہیں 
L
کی صورت میں لکھا گیا ہے مساوات 4.70 درج ذیل ہونگے 
\[E_n=-\frac{me^{4}}{32\pi^{2}\epsilon^{2}\hslash^{2}(j_{max}+l+1)^{2}}\]
(جزو الف):\\
 مسلہ Feynman-Hellmann میں 
\(\lambda=e\)
استعمال کرتے ہوے
\(\langle1/r\rangle\)
تلاش کریں۔ اپنے نتیجے کی تصدیق مساوات 6.55 کے ساتھ کریں۔\\
(جزو ب) :\\
\(\lambda=l\)
کو استعمال کرتے ہوے
\(\langle1/r^{2}\rangle\)
تلاش کریں۔ اپنے نتیجے کی تصدیق مساوات 6.56 کے ساتھ کریں۔\\
سوال 6.34 :\\
 رشتہ Kramers' 
\[\frac{s+1}{n^{2}}\langle r^{s}\rangle -(2s+1)a\langle r^{s-1}\rangle n+\frac{s}{4}[(2l+1)^{2}-s^{2}]a^{2}\langle r^{s-2}\rangle =0\]
صابط کریں جو ھاٰےڈروجنکےحال 
\(\psi_{nlm}\)
میں الیکٹران کے لئے R کی توقعاتی قیمتوں کی تین مختلف طاقتوں
\((s,s-1,\)
اور
\(s-2)\)
کا۔ تعلق پیش کرتا ہے۔ اشارہ : راداسی مساوات 4.53 کو درج ذیل روپ میں لکھ کر
\[u''=\big[\frac{l(l+1)}{r^{2}}-\frac{2}{ar}+\frac{1}{n^{2}a^{2}}\big]u.\]
۔
\(\int(ur^{s}u'')dr\)	 
کو 
\(\langle r^{s}\rangle\)
،
\(\langle r^{s-1}\rangle\)
،
\(\langle r^{s-2}\rangle \)
کی  صورت میں لکھیں اسکے بعد تکامل
bilhisis
کے ذریےدوہرا تفروق کو بیٹھایں۔ دیکھایں کے\\
\[\int(ur^{s}u')=-(s/2)<r^{s-1}>\]
اور
\[\int(u'r^{s}u')dr=-[2/(s+1)]\int(u''r^{s+1}u')dr\]
ہوگا اسی کو لے کر آگے چلیں)\\
سول 6.35\\ 
(جزو الف ):\\
 رشتہ Kramers' مساوات 6.104 میں
\(s=0,s=1,s=2\)
اور 
\(s=3\)
ڈال کے 
\(\langle r^{-1}\rangle,\langle r\rangle,\langle r^{2}\rangle\)
اور 
\(\langle r^{3}\rangle\)
کے قلیات حاصل کریں۔ آپ دیکھ سکتے ہیں کے آپ اس طرح چلتے ہے کسی بھی مثبت طاقت کے لئے قلیہ دریافت کر سکتے ہیں ۔\\
(جزو ب):\\ 
دوسرے رخ آپکو مثلا درپیش  ہوگا آپ 
\(s=-1\)
پر کرکے دیکھیں کے آپکو صرف 
\(\langle r^{-2}\rangle\)
اور 
\(\langle r^{-3}\rangle\)
کے بیچ رشتہ حاصل ہوگا ۔\\
جزو ج:\\
 اگر آپ کسی طریقہ سے
\(\langle r^{-2}\rangle\)
دریافت کر پایں تب آپ رشتہ Kramers' استعمال کرکے باکی تمام منفی قوعتوں کے لئے قلیات دریافت کر سکتے ہیں ۔\\
مساوات 6.56 : جسے سوال 6.33 میں اخذ کیا گیا ہے اسے استعمال کرتے ہوے 
\(\langle r^{-3}\rangle\)
تعین کریں اور اپنے نتیجہ کی تصدیق مساوات 6.64 کے ساتھ کریں۔\\
سوال 6.36: \\
ایک جوہر کو يقسا بیرونی برقی میدان
\(E_{ext}\)
میں رکھنے سے توانائی کی سطحیں بٹتی ہیں جسے سٹارک اثر کہا جاتا ہے اور جو zemann اثر کا برقی مماسل ہے اس سوال میں ہم ھاٰےڈروجن کے 
\(n=1\)
اور 
\(n=2\)
حالات کے لئے سٹارک اثر کا تجزیہ کرتے ہیں۔ فرض کریں میدان
Z
رخ ہے لہٰذا الیکٹران کی مخفی توانائی درج ذیل ہوگی: 
\[H'_{S}=eE_{ext}z=eE_{ext}r\cos{\theta}\]
اسکو bohr hamiltonian  مساوات 6.42 میں اضطراب تصور کریں اس مسلہ میں چکر کا کوئی کردار نہیں ہے لہذا اسے نظر انداز کرتے ہوے عمدہ ساخت کو رعد کریں۔\\
(جزو الف ) :\\
 اول رتبہ میں زمینی حال توانائی اس اضطراب سے اثر انداذ نہیں ہوتی ۔\\
(جزو ب) :\\  
پہلا هیجان حال
4
پرتہ
\(\psi_{200},\psi_{211},\psi_{210},\psi_{21-1},\)
انحطاطی نظریہ اضطراب استعمال کرتے ہوے، توانائی کی رتبہِ اول کا سہی تعین کریں ۔توانائی 
\(E_{2}\)
کتنے سطحوں میں بٹے گا؟\\
(جزو ج) :\\
 درج بالہ جزو ب میں موضوع تفعالات موج کیا ہونگے ؟ ان میں سے ہر ایک موضوع حالات میں برقی جوعف قطب میعارِاثر
\((p_{e}=-er)\)
کی توقعاتی قیمت معلوم کریں ۔ آپ دیکھیں گے کہ نتائج  لاگو میدان کے تعابع نہیں ہونگے اس طرح ظاہر ہے کے پہلی هیجان حال میں ھاٰےڈروجن برقی جوعفت قطب میعارِاثر کا حامل ہوگا۔اشارہ:اس سوال میں بہت سارے تاكملات پاے جاتے ہیں تاہم تقریبن تمام کی قیمت سِفر ہے لہذا حساب سے قبل غور کریں اگر 
\(\phi\)
تكمل سفر ہو تب 
\(r\)
اور 
\(\theta\)
تکملات حل کرنے  کی ضرورت نہیں ہوگی جزوی جواب
\[W_{13}=W_{31}=-3eaE_{ext};\]
باقی تمام ارکان سفر ہیں۔ )\\
سوال 6.37:
 ھاٰےڈروجن کی 
\(n=3\)
حالات کے لئے سٹارک  اثر سوال 6.36 پر غور کریں ابتدائ طور پر چکر کو نظر انداز کرتے ہوے اب انحطاطی حالات 
\(\psi_{3lm}\)
ہونگے اور اب ہم 
\(z\)
رخ برقی میدان چالو کرتے ہیں ۔\\
(جزو الف) :\\ 
اضطرابی hamiltonian کو ظاہر کرنے والا 
\(9\times9\)
کا کالم  تیار کریں\\
 جزوی جواب 
\[\langle 300|z|310\rangle =-3\sqrt{6}a,\langle 310|z|320\rangle =-3\sqrt{3}a,\langle 31\pm1|z|32\pm1\rangle=-(9/2)a.\]\\
(جزو ب) :\\ 
امتیازی اقتدار اور انکی انحطاط دریافت کریں .\\
سوال 6.38 :
  ڈوٹرئم
کی زمینی حال میں نہایت موحین منتقلی کے دوران خارج کردہ پھوٹان کا طولِ موج
میں تلاش کریں ۔ ڈوٹرئم درحقیقت بھاری ھاٰےڈروجن ہے جسکے مرکز میں ایک اضافی نوٹران پایا جاتا ہے پروٹان اور نوٹران ساتھ جڑ کر  ڈوٹرئم بناتے ہیں جسکا چکر ایک مقناطیسی دارِاثر
\[\mu_{d}=\frac{g_{d}e}{2m_{d}}S_{d};\]
اور ڈوٹرئم کا 
g-
جزو 1.71 ہے ۔\\
سوال 6.39 :\\
 ایک کالم میں قریبی باردارا کا بجلی میدان جوہر کی توانائی کی سطحوں کو مضطرب کرتا ہے ایک تازہ نمونہ کے طور پر فرض کریں hydrogen جوہر کی پڑوس میں نقطہ باروں کی تین جوڑیاں پای جاتی ہیں شکل 6.15۔(چونکے اس۔ سوال کے ساتھ چکر کا کوئی۔ واستہ نہیں ہے لہٰذا اسے نظرانداز کریں )\\
(جزو الف ):\\
 درج ذیل 
\[r<<d_{1},r<<d_{2},andr<<d_{3},\]
کی صورت میں دیکھاے
\[H'=V_{o}+3(\beta_{1}x^{2}+\beta_{2}y^{2}+\beta_{3}z^{2})-(\beta_{1}+\beta_{2}+\beta_{3})r^{2},\]
جہا درج ذیل ہیں
\[\beta_{i}\equiv-\frac{e}{4\pi\epsilon}\frac{\eta_{i}}{d_{i}^{3}},\quad\quad\]
اور 
\[V_{o}=2(\beta_{1}d_{1}^{2}+\beta_{2}d_{2}^{2}+\beta_{3}d_{3}^{2}).\]\\
(جزو ب) : \\زمینی حال توانائی کی رتباےاول کی تخفیف تلاش کریں ۔\\
(جزو ج) :\\ پہلی۔ هیجان حالات 
\((n=2)\)
کی توانائی کے لئے رتباےاول کی تخفیف تلاش کریں ۔ درجذیل صورتوں میں یہ چار پڑتہ انحطاطی نظام کتنی سطحوں میں بٹے گا ۔\\
ایک ) کابی تشاقلی
\[\beta_{1}=\beta_{2}=\beta_{3},\]
کی۔ صورت میں ۔\\
دو ) چوں زاویہ تشاقلی
\[\beta_{1}=\beta_{2}\neq\beta_{3}:\]
کی صورت میں۔\\
تین ) آرتھو ھامبک تشاقل کی صورت میں تینوں مختلف ہونگیں ۔\\
سوال 6.40 : \\بازاوقات 
\(\psi_{n}^{1}\)
کو غیر مضطرب طفعالات امواج میں پهلاۓ  مساوات 6.11 بغیر مساوات 6.10 کو بلہ واستہ حال کرنا ممکن ہوتا ہے اسکی دو بلخصوص خوبصورت مثالین درج ذیل ہیں۔\\
(الف )\\
ایک) ھاٰےڈروجن کی زمینی حال میں سٹارک اثر ایک یکساں بیرونی برقی میدان 
\(E_{ext}\)
کی۔ موجودگی میں ھاٰےڈروجن  کی زمینی حال کا رتبہ اول تخفیف تلاش کریں ( سوال 6.36 stark اثر دیکھیں ۔) ۔اشارہ : حل کی درج ذیل روپ :\\
\[(A+Br+Cr^{2})e^{-r/n}cos\theta;\]
استعمال کرکے دیکھیں اپ نے مستقلات 
\(A,B,\)
اور 
\(C\)
کی ایسی قیمتیں تلاش کرنی ہیں جو مساوات 6.10 کو مطمئن کرتے ہوں ۔\\
دو ) زمینی حال توانائی کی رتبہ دوم تخفیف مساوات 6.14 کی مدد سے  تعین کریں جیسا اپنے سوال 6.36 (الف ) میں دیکھا رتبہ اول تخفیف سفر ہوگی ۔جواب :\\
\[-m(3a^{2}eE_{ext}/2\hslash)^{2}.\]\\
(جزو ب)\\ اگر پروٹان کا برقی جست قطب میعارِاثر 
\(p\)
ہوتا تب  ھاٰےڈروجن کے  الیکٹرانکی مخفی توانائی درجذیل مقدار سے مضطرب ہوتی۔
\[H'=\frac{epcos\theta}{4\pi\epsilon r^{2}}\]
ایک ) زمینی حال طفعال موج کی رتبی اول تخفیف کو مساوات 6.10 حل کرکے تلاش کریں ۔\\
دو ) دیکھایں کہ رتبہ تک جوہر کا قل برقی جوعفت قطب میعارِاثر حیرت کی۔ بات ہے سفر ہوگا۔ \\
تین ) زمینی حال توانائی کی۔ رتبہ دوم تخفیف مساوات 6.14 سے تعین کریں رتبہ اول تخفیف کیا ہوگا ؟\\
\\

