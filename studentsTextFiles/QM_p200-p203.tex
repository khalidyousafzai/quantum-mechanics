\documentclass{book}
\usepackage{polyglossia}
\usepackage{amsmath}
\usepackage{enumitem}            
\setmainlanguage[numerals=maghrib]{arabic}   
\setotherlanguages{english}
\newfontfamily\arabicfont[Scale=1.0,Script=Arabic]{Urdu Typesetting} 
\newfontfamily\urdufont[Scale=1.25,WordSpace=60.0,Script=Arabic]{Urdu Typesetting}
%\setlength{\parskip}{5mm plus 4mm minus 3mm}
\begin{document}
\setcounter{chapter}{4}
\chapter{ یکساں ذرات } 
 

\section{ دو زراتی نظام} 

ایک زرہ کے لیے فلحال چکر کو نظر انداز کرتے ہوےٗ  
$ \psi ( r , t ) $
فضا ٗی مہدت r اور وقت t کا تفعال ہو گا۔ دو زراتی نظام کا حال پہلے زرے کے مخاطر 
$ ( r_1 ) $
دوسرے زرے کے مخاطر 
$ ( r_2 ) $
اور وقت کا طابع ہو گا۔ 
\begin{align}   
\psi ( r_1 , r_2 , t ) 
\end{align}
 ہمیشہ کی طرح یہ وقت کے لحاظ سے shrodinger مساوات 
\begin{align}
\iota \hbar \frac{ \partial \psi }{ \partial t } = H \psi
\end{align}
کے تحت ارتقا کرے گا۔ جہاں H مکمل  نظام کا Hamiltonian  ہے۔
\begin{equation}
H = - \frac{ \hbar }{ 2 m_1 } {v_1}^2  -  \frac{ \hbar }{ 2 m_2 } { v_2 }^2 +  v( r_1 , r_2 , t )
\end{equation}
زرہ ایک یا زرہ دو کے محددوں کے لحاظ سے  تفرقات لینے کو 
$ \Delta $
زیر نوشت میں ایک یا دو سے ظاہر کیا گیا ہے۔ زرہ ایک کا ہجم 
$ d^3 r_1 $
اور زرہ دو کا ہجم
$ d^3 r_2 $
پاےٗ جانے کا اہتمال درج ذیل ہو گا۔ 
\begin{align}
| \psi ( r_1 , r_2 , t ) |^2 { d^3 } { r_1 }  { d^3 } { r_2 }
\end{align}
ظاہر ہے کہ 
$ \psi $
کو درج ذیل کے لحاظ سے معمول پر لانا ہو گا۔ 
\begin{align}
\int | \psi ( r_1 , r_2 , t ) |^2 { d^3 } { r_1 } { d^3 } { r_2 } = 1
\end{align}

\newpage

غیر تابع وقت مخفی توانا ٗی کے لیے علیحدگی  متغیرات  سے حلوں کا مکمل سلسلہ حاصل ہوتا ہے ۔ 
\begin{align}
\psi ( r_1 , r_2 , t ) =  \psi ( r_1 , r_2 ) {e}^\frac{ - i E t }{h}  
\end{align}
جہاں فضا ٗی تفعال معاج 
$ \psi $
غیر تابع وقت shroudinger مساوات 
\begin{align}
-\frac{ \hbar }{ 2 m_1 }  {\nabla_1^2} { \psi } - \frac{ \hbar }{ 2 m_2 } \nabla_2^2 { \psi } + V \psi 
\end{align}
جس میں E پورے نظام کی قل توانا ٗی ہے ۔ 

سوال ۔5٫1:
عام طور پر  باہمی مخفی توانا ٗی انحصار صرف 2 زرات کے بیچ صمتیہ
$ r = r_1 - r_2 $
پر ہو گا ۔ ایسی صورت میں متغیرات 
$ r_1  $
اور 
$  r_2  $
کی جگہ نےٗ متغیرات  اور مرکز کمیت
$  R = \frac{ ( m_1 r_1 + m_2 r_2 ) }{ m_1 + m_2 } $
مساوات shroudinger ہوتی ہے ۔ 


(الف)۔  دکھا یٗں کہ 
$ r_1 = R + (\frac{ \mu}{m_1} )r , r_2 = R - (\frac{ \mu }{ m_2 } )r   $ 
اور 
$ \nabla_1 = ( \frac{ \mu }{ m_2 } )\nabla_R + \nabla_r , \nabla_2 = ( \frac{ \mu }{ m_1 } ) \nabla_R - \nabla_r $ 
جہاں 
\begin{align}
\mu = \frac{ m_1 m_2 }{ m_1 + m_2 }
\end{align}
نظام کی تشخیص شدہ کمیت ہے ۔
 
(ب)۔ دکھا ٗیں کہ غیر تابع وقت shroudinger  مساوات درج ذیل رعب اختیار کرتی ہے ۔
\[
- \frac{ \hbar^2 }{ 2 ( m_1 + m_2 ) } { \nabla_R }^2  \psi - \frac{ \hbar^2 }{ 2 \mu } { \nabla_r }^2 { \psi } + V(r) \psi = E \psi
\]

(ج)۔ متغیرات کو 
$ \psi ( R , r ) = { \psi_r }(R) { \psi_r } (r) $
لیتے ہوےٗ علیحدہ کریں ۔ آپ دیکھیں گے کہ 
$ \psi_r $
 ایک ذرہ کی shroudinger مساوات جہاں کمیت 
$ ( m_1 + m_2 ) $
مخفی توانا یٗ صفر ہو اور نظام کی توانا ٗی 
$ E_R $
کو مطمٰن کرتا ہے ۔ جبکہ 
$ \psi_r $
ایک ذرے کی shroudinger مساوات جہاں تخفیف شدہ کمیت ہو ۔  مخفی توانا ٗی  V(r) ہو ، کو مطمٗن کرتا ہے ۔ قل توانا ٗی اور ان کا مجموعہ 
$ E = E_R + E_r $
ہو گا ۔ اس سے ہمیں یہ معلوم ہوتا ہے  کہ مرکز ی کمیت ایک آزاد ذرہ کی طرح حرکت کرتا ہے جبکہ ذرہ ایک کے لحاظ سے ذرہ دو کی  نصبطی حرکت ایسے ہی ہو گی جیسا مخفی توانا ٗی V میں تخفیف شدہ کمیت کا ایک ذرہ  کرتی ہے classical mechanics میں بھی بلکل یہی تحلیل ہو گی    جو 2 اجسام مسلے کو محاصل ایک جسم مسلہ میں  تبدیل کرتی ہے ۔ 

\newpage 

سوال 5٫2 : یوں Hydrogen کے مرکزہ کی حرکت کو درست کرنے کے لیے ہم electron  کی کمیت کی جگہ تخفیف شدہ کمیت استعمال کریں گے 

(الف) ۔ hydrogen کی بندش کی توانا ٗی (مساوات 4٫77) جاننے کی خاطر 
$ \mu $
کی جگہ m استعمال کرنے سے  دو بمعنی ہندسوں تک فیصد خلل کتنا ہو گا۔  

(ب) ۔ hydrogen اور Dueterium کے لیے 
$ ( n=3 ) > ( n=2 ) $ 
 سرخ بالمر لکیروں کے بیچ تفعال معاج میں فرق تلاش کریں ۔ 

(ج) ۔ Positronium کی بندشی توانا ٗی تلاش کریں ۔ proton کی جگہ  positron رکھنے سے positronium پیدا ہو گا ۔ positron کی کمیت electron کی کمیت کے برابر ہو گا جبکہ اس کی علامت Electron کی علامت کے مخالف ہے ۔ 

(د) ۔ فرض کریں آپ muonic hydrogen  جس میں electron کی جگہ ایک muon کی موجودگی کی تصدیق کرنا جانتے ہوں۔ muon کا bar  electron کے bar  کے برابر ہے ۔ جبکہ یہ electron سے 206٫77 گنا زیادہ  کمیت رکھتا ہے ۔  آپ Lyman \alpha لکیر  
$ n = 2 $
تا 
$ n = 1 $
کے لیے کس طور معاج پر نظر رکھیں گے ۔ 

سوال 5٫3 : chlorine کے قدرتی دو ہم جا 
$ Cl^35 and Cl^37 $ 
پاےٗ جاتے ہیں ۔ دکھا ٗیں کہ HCL کی لرزشی طیف قریب قریب   جوڑیوں پر مشتمل ہو گا ۔ جن میں فرق 
$ \Delta_v = 7.51 x {10}^-4 v $
جہاں v خرجی photon کی تعدد ہے ۔  ( اشارہ : اس کو ایک Harmonium مرتعیش تصور کریں جہاں 
$ \omega = sqrt{ \frac{ k }{ mu} } $
ہو گا ۔ جہاں 
$ \mu $
تؒخفیف شدہ کمیت ( مساوات 5٫8 ) ہے ۔ جبکہ k دونوں ہمجا کے لیے ایک جیسا ہے ۔ 

\end{document}
