مثال 2.3
 
ہم دیکھتے ہیں کہ مثال 2.2 میں ابتدائی تفاعل موج (شکل 2.3) زمینی حال \عددی{  \psi_{1}  } کے ساتھ بہت قریبی مشابہت رکھتا ہے. یوں ہم توقع کریں گے کہ \عددی{ \left| c_{1} \right|^{2} }
 غالب جزروی ہو گا. یقیناً 
\begin{align*}
\left| c_{1} \right|^{2} = \left( \frac{8\sqrt{15}}{\pi^{3}} \right)^{2} = 0.998555 \cdots
\end{align*}
باقی تمام عددی سر مل کر فرق دیتے ہیں:
\begin{align*}
\sum_{n=1}^{\infty} \left| c_{n} \right|^{2} = \left( \frac{8\sqrt{15}}{\pi^{3}} \right)^{2} \sum_{n=1,3,5,...}^{\infty} \frac{1}{n^{2}} = 1.
\end{align*}
اس مثال میں توانائی کی توقعاتی قیمت
\begin{align*}
\langle H \rangle = \sum_{n=1,3,5,...}^{\infty} \left( \frac{8\sqrt{15}}{n^{3} \pi^{3}} \right)^{2} \frac{n^{2} \pi^{2} \hbar^{2}}{2ma^{2}} = \frac{480\hbar^{2}}{\pi^{4} ma^{2}} \sum_{n=1,3,5,...}^{\infty} \frac{1}{n^{4}} = \frac{5 \hbar^{2}}{ma^{2}}.
\end{align*}
جیسے ہم توقع کر سکتے تھے یہ \عددی{ E_{1} = \pi^{2} \hbar^{2}/2ma^{2}  } کے بہت قریب ہے- ہلکا سا زیادہ, حجان حل حالتوں کی شمول کی بنا. 


Problem 2.3 


دکھائیں کے \عددی{ E = 0 } یا \عددی{ E < 0 } کیلئے وقت کے غیر تابع شروڈنگر مساوات کا کوئی بھی قابل قبول حل لا متناہی چوکور کنویں کیلئے نہیں پایا جاتا ہے. یہ سوال 2.2 میں دیے گئے عمومی مسئلے کی مخصوص صورت ہے. لیکن اس بار شروڈنگر مساوات کو صریحاً حل کرتے ہوئے دکھائیں کہ آپ سرحدی شرائط پر پورا نہیں اتر سکتے ہیں.
 
Problem 2.4 

لامتناہی چکور کنواں کے \عددی{ n } وی حال کیلئے \عددی{ \langle x \rangle ,\, \langle x^{2} \rangle ,\, \langle p \rangle ,\, \langle p^{2} \rangle  } اور \عددی{ \sigma_{p} } تلاش کریں. تصدیق کریں کہ اصول غیر یقینیت پر یہ پورا اترتے ہیں. کونسا حال غیر یقینیت کی حد کے قریب ترین ہو گا.

Problem 2.5 

لامتناہی چکور کنواں میں ایک ذرے کی ابتدائی طفاعل موج اولین دو ساکن حالات کا جفت مرکب ہے 
\begin{align*}
\Psi(x,0) = A[\psi_{1}(x) + \psi_{2}(x)].
\end{align*}
\begin{enumerate}[a.]
\item \عددی{ \Psi(x,0) } کو معمول پر لائیں. یعنی \عددی{ A } تلاش کریں. آپ \عددی{ \psi_{1} } اور \عددی{ \psi_{2} } کی معیاری عمودیت بروئے کار لاتے ہوئے با آسانی ایسا کر سکتے ہیں. یاد رہے کہ \عددی{ t=0 } پہ \عددی{  \Psi  } کو معمول پر لانے کے بعد آپ یقین کر سکتے ہیں کہ یہ معمول شدہ ہی رہے گا. اگر آپ کو اس پر شک ہے تو اس کی صریحاً تصدیق کریں جز ب حل کرنے کے بعد. 
\item
\عددی{ \Psi(x,t) } اور \عددی{ \left| \Psi (x,t) \right|^{2} } تلاش کریں. \عددی{ \left| \Psi (x,t) \right|^{2} } کو وقت کے \عددی{ \sin } نما تفاعل کی صورت میں لکھیں جیسا کہ مثال 2.1 میں کیا گیا. نتائج کو سادہ صورت میں لکھنے کی خاطر امیگا لیں. 
\item 
\عددی{ \langle x \rangle  } تلاش کریں. آپ دیکھیں گے کہ یہ وقت کے ساتھ ارتعاش پزیر ہو گا. اس ارتعاش کی زاویائی تعدد کتنی ہو گی؟ ارتعاش کا حیطہ کیا ہو گا؟ اگر آپکا حیطہ \عددی{ a/2  } سے زیادہ ہو تب آپ کو جیل بھیجنے کی ضرورت ہو گی. 
\item 
\عددی{ \langle p \rangle  } تلاش کریں. اور اس پہ زیادہ وقت صرف نہ کریں. 
\item
اس ذرے کی توانائی کی پیمائش سے کون کون سی قیمتیں متوقع ہیں؟ اور ہر ایک قیمت کا احتمال کتنا ہو گا؟ \عددی{ H } کی توقعاتی قیمت تلاش کریں. اس کی قیمت کا موازنہ \عددی{ E_{1} } اور \عددی{ E_{2} } کے ساتھ کریں؟
\end{enumerate}

سوال 2.6

   
     
     
     
     
 
 
اگر چہ تفاعل موج جا مجموعی زاویائی مستقل کسی با معنی طبعی اہمیت کا حامل نہیں ہے (چونکہ یہ کسی بھی قابل پیمائش مقدار میں کٹ جاتا ہے)  مساوات 2.17 میں عددی سرو کے اضافی زاویائی مستقل اہمیت کے حامل ہیں. مثال کے طور پر ہم سوال 2.5 میں  \عددی{ \psi_{1} } اور \عددی{ \psi_{2} } کے اضافی زاویائی مستقل تبدیل کرتے ہیں. 
\begin{align*}
\Psi (x,t) = A[\psi{1} (x) + e^{i\phi}\psi_{2}(x)],
\end{align*}
جہاں \عددی{ \phi } کوئی مستقل ہے. \عددی{  \Psi(x,t), \left| \Psi (x,t) \right|^{2} } اور \عددی{ \langle x \rangle } تلاش کریں اور ان کا موازنہ پہلے حاصل شدہ نتائج کے ساتھ کریں. بالخصوص \عددی{ \phi = \pi/2 } اور  \عددی{ \phi = \pi } پر غور کریں. 

سوال 2.7 
لا متناہی چکور کنواں میں ایک ذرے کی ابتدائی تفاعل موج درج ذیل ہے,
\begin{align*}
\Psi (x,t) = \left\{ \begin{array}{lc}
Ax, & 0 \leq x \leq a/2 \\ A(a-x), & a/2 \leq x \leq a
\end{array} \right.
\end{align*}
\begin{enumerate}[a.]
\item 
\عددی{\Psi(x,0)} کا خاکہ کھینچیں اور مستقل \عددی{A} کی قیمت تلاش کریں.
\item  
\عددی{\Psi(x,t) } دریافت کریں.
\item  
توانائی کی پیمائش کا نتیجہ \عددی{E_{1}} ہونے کا احتمال کتنا ہو گا.
\item 
توانائی کی توقعاتی قیمت تلاش کریں.
\end{enumerate}
سوال 2.8 
ایک لامتناہی چکور کنواں جسکی چوڑائی \عددی{a} ہو, میں کمیت \عددی{m} کا ایک ذرہ کنویں کے بائیں حصے سے شروع ہوتا ہے اور یہ \عددی{t=0} پر بائیں نصف حصے کے کسی بھی نقطے پر ہو سکتا ہے.
\begin{enumerate}[a.]
\item
اس کی ابتدائی طفاعل موج \عددی{\Psi(x,0) } تلاش کریں. آپ فرض کریں کے یہ حقیقی ہو گا اور اسے معمول پر لانا نا بھولیے گا.
\item 
توانائی کئ پیمائش \عددی{\pi^{2}\hbar^{2}/2ma^{2}} دینے کا احتمال کیا ہو گا. 
\end{enumerate} 
سوال 2.9 
لمحہ \عددی{t=0}  پر مثال 2.2 کے تفاعل موج کیلئے \عددی{H} کی توقعاتی قیمت تکمل سے حاصل کریں. 
---
مثال 2.3 میں, مساوات 2.39 کی مدد سے حاصل نتیجے کے ساتھ موازنہ کریں. دھیان رہے کیونکہ \عددی{H} وقت جا تابع نہیں ہے لھذا \عددی{t=0} لینے سے نتیجے پر کوئی فرق نہیں ہو گا. 



