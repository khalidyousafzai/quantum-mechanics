\ابتدا{سوال}
% Question 3.25
لیجانڈر کثیر رکنیہ  وقفہ \عددی{-1 \leq x \leq 1}  پر تفاعل \عددی{1, \, x, \, x^{2}} اور  \عددی{x^{3}} کو گراہم شامیڈ کی ترکیب سے معیاری عمود بنائیں( سوال A. 4 دیکھیں) ۔عین ممکن ہے کہ آپ نتائج کو پہچان سکیں یہ معیاری عمودزنی کے علاوہ لیجانڈر کثیر رکنیہ ہیں (جدول 4.1)
\انتہا{سوال}
\ابتدا{سوال}
% Problem 3.26
ایک خلاف ہرمیشی یا منخرف ہرمیشی حامل اپنے ہرمیشی جوڑی دار کے منفی کے برابر ہوتا ہے 
         \begin{align}
         \hat{Q}^{\dagger} = -\hat{Q}
         \end{align}
\begin{enumerate}[a.]
\item  دکھائیں کہ ایک خلاف ہرمیشی حامل کی   توقعاتی قیمت خیالی ہو گی
\item دکھائیں کہ دو ہرمیشی حاملین کا تبادلکار خلاف ہرمیشی ہو گا۔ دو خلاف ہرمیشی حاملین کے تبادلکار کے بارے میں کیا کہہ سکتے ہیں؟ 
\end{enumerate}
\انتہا{سوال}
\ابتدا{سوال}
% Problem 3.27
ترتیبی پیمائشیں مشہود \عددی{A} کو ظاہر کرنے والا \عددی{\hat{A}} کے دو معمول شدہ امتیازی حالات \عددی{\psi_{1}}اور \عددی{\psi_{2}}اور بلترتیب امتیازی اقدار \عددی{a_{1}}اور \عددی{a_{2}}ہیں ۔ مشہود \عددی{B} کو ظاہر کرنے والے حامل \عددی{\hat{B}} کے دو معمول شدہ امتیازی حالات \عددی{\phi_{1}}اور \عددی{\phi_{2}} اور بلترتیب امتیازی اقدار \عددی{b_{1}}اور \عددی{b_{2}} ہیں۔ ان امتیازی حالات کا تعلق درج ذیل ہے 
\begin{align*}
\psi_{1} = ( 3\phi_{1} + 4\phi_{2})/5, \quad \psi_{2} = ( 4\phi_{1} - 3\phi_{2})/5. 
\end{align*}
\begin{enumerate}[a.]
\item  مشہود \عددی{  A} کی پیمائش \عددی{  a_{1} } قیمت دیتی ہے۔اس پیمائش کے فوراً بعد یہ نظام کس حال میں ہو گا؟ 
\item اب اگر \عددی{  B} کی پیمائش کی جائے تو کیا نتائج ممکن ہیں اور ان کے احتمال کیا ہوں گے؟ 
\item مشہود \عددی{   B} کی پیمائش کے فوراً بعد دوبارہ \عددی{  A} کی پیمائش کی جاتی ہے ۔نتیجہ \عددی{   a_{1}} حاصل کرنے کا احتمال کیا ہو گا؟ یہاں دھیان کریں کہ اگر میں آپ کو \عددی{  B} کی پیمائش کا نتیجہ بتاتا تو تب جواب بہت مختلف ہوتا۔ 
\end{enumerate}
\انتہا{سوال}
\ابتدا{سوال}
% Problem 3.28
لامتناہی چکور کنواں کے \عددی{n} ساکن حال کے معیار حرکت کو فضا موجی تفاعل \عددی{\Phi_{n}(p,t)} تلاش کریں۔ \عددی{| \Phi_{1}(p,t) |^{2}}اور \عددی{| \Phi_{2}(p,t) |^{2}} کو \عددی{p} کے تفاعل کے طور پر ترسیم کریں نکات \عددی{p = \pm n\pi\hbar/a}  پر خصوصی توجہ دیں۔ \عددی{\Phi_{n}(p,t)} کو استعمال کرتے ہوئے \عددی{p^{2}} کی توقعاتی قیمت کا حساب لگائیں۔ اپنے جواب کا سوال \حوالہ{  2.4 } کے ساتھ موازنہ کریں۔
\انتہا{سوال}
\ابتدا{سوال}
% Problem 3.29
درج ذیل تفاعل موج پر غور کریں
\begin{align*}
\Psi(x,0) = \left\{ \begin{array}{lc}
\frac{1}{\sqrt{2n\lambda}}e^{i2\pi x/\lambda}, & -n\lambda < x < n \lambda \\ 0, & \text{otherwise}
\end{array} \right.
\end{align*}
جہاں \عددی{n} کوئی مثبت عدد صحیح ہے
اگرچہ وقفہ \عددی{-n\lambda < x < n \lambda} پر یہ تفاعل خالص سائن نما ہے جس کی طولی موج \عددی{\lambda} ہے لیکن چونکہ یہ لامتناہی تک ارتعاش جاری نہیں رکھتا لہٰذا اس کی معیار حرکت کی قیمتیں ایک ساتھ پر مشتمل ہوں گی۔اس کی معیار حرکت و فضا تفاعل موج \عددی{\Phi(p,0)} تلاش کریں  \عددی{| \Psi(x,0)|^{2}}اور \عددی{| \Psi(p,0)|^{2}} ترسیم کریں اور چوٹی کے ایک جانب صفر اور دوسری جانب صفر کے بیچ چوڑائی \عددی{\omega_{x}}اور \عددی{\omega_{p}} تعین کریں۔ دیکھیں کہ \عددی{n \rightarrow \infty} کرنے سے ان چوڑایوں پر کیا اثر پڑتا ہے؟
 \عددی{\omega_{x}} اور \عددی{\omega_{p}} کو \عددی{\Delta x}اور \عددی{\Delta p} کی اندازاً قیمتیں لیتے ہوئے تصدیق کریں کہ عدم یقینیت کا اصول مطمئن ہوتا ہے۔ انتباہ: اگر آپ \عددی{\sigma_{p}} کا حساب کرنے کی کوشش کریں تو آپ کو ایک حیرانی کا سامنا ہو گا کیا آپ اس مسئلے کی وجہ بتا سکتے ہیں؟ 
\انتہا{سوال}
\ابتدا{سوال}
% Problem 3.30
درج ذیل فرض کریں
\begin{align*}
\Psi(x,0) = \frac{A}{x^{2}+a^{2}}. \quad (-\infty < x < \infty)
\end{align*}
جہاں \عددی{A} اور \عددی{a} مستقل ہیں
\begin{enumerate}[a.]
\item \عددی{  \Psi(x,0)}کو معمول پر لاتے ہوئے \عددی{A} تعین کریں ۔
\item لمحہ \عددی{ t=0 } پر \عددی{\langle x \rangle} \عددی{ \langle x^{2} \rangle}اور \عددی{ \sigma_{x} } تلاش کریں۔
\item معیار حرکت و فضا تفاعل موج \عددی{  \Phi(p,0)} تلاش کریں اور تصدیق کریں کہ یہ معمول شدہ ہے۔
\item \عددی{ \Phi(p,0) }استعمال کرتے ہوئے لمحہ \عددی{  t=0} پر \عددی{ \langle p \rangle } \عددی{  \langle p^{2} \rangle}اور \عددی{ \sigma_{p} } کا حساب کریں 
\item  اس حال کے لیے ہائزنبرگ عدم یقینیت کے اصول کو جانچیں 
\end{enumerate}
\انتہا{سوال}
\ابتدا{سوال}
%Problem 3.31
 درج ذیل کو مساوات \حوالہ{ 3.71 } کی مدد سے دکھائیں 
\begin{align}
\frac{d}{dt} \langle xp \rangle -2\langle T \rangle - \left\langle x \frac{dV}{dx} \right\rangle. 
\end{align}
جہاں \عددی{ T } حرکتی توانائی \عددی{ H=T+V } ساکن حال میں بایاں ہاتھ صفر ہو گا( ایسا کیوں ہے؟) لہذا درج ذیل ہو گا 
\begin{align}
s\langle T \rangle = \left\langle x \frac{dV}{dx} \right\rangle. 
\end{align}
اس کو مسئلہ ویریعل کہتے ہیں۔ ہارمونی مرتعش کے ساکن حالات کے لیے اس مسئلہ کو استعمال کرتے ہوئے ثابت کریں کہ \عددی{ \langle T \rangle = \langle V \rangle  } ہو گا اور تصدیق کریں کہ یہ سوال 2.11 اور 2.12 میں آپ کے حاصل کردہ نتائج کا ہم آہنگ ہے۔
\انتہا{سوال}
