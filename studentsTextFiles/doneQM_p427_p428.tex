
اس کے برعکس روشنی سے زیادہ تیز حرکت کرنے والے سببی اثر و وسوخ کے ناقبلِ قبول مضمرات ہو سکتے ہیں۔ خصوصی نظریہ اضافت میں ایسے جمودی چوکھٹ پائے جاتے ہیں جن میں اس طرح کا اشارہ وقت میں پیچھے جاسکے گا یعنی سبب سے پہلے اثر رونما ہوگا جس سے ناقابلِ قبول منتقی مسائل کھڑے ہوتے ہیں۔ مثلاً آپ اپنے نوزادہ دادا کو قتل کر سکتے ہیں۔ جو ظاہر ہے ایک بری بات ہے۔ اب سوال یہ کھڑا ہوتا ہے کہ آیہ روشنی سے تیز اثرات جن کیپیشاً گوئی کوانٹم میکانیات کرتی ہے اور جو ایسپیکٹ کے تجربہ میں کسف ہعتے ہیں ان معانوں میں سببی ہے یا یہ ساے کی حرکت کی طرح غیر حقیقی ہے جن پر فلسفیانہ اعترازات نہیں لگائے جا سکتے ہیں۔

آئیں تجربہ بل پر غور کریں کریں۔ کیا الیکٹران کی پیمائش کا پوزیٹران کی پیمائش پر اثر ہوگا  یقیناً ایسا ہوتا ہے ورنہ ہم مواد کے بیچ باہم رشتہ کی وضاحت پیش کرنے ساے قاصر ہوں گے۔ لیکن کیا الیکتران کی پیمائش پوزیٹران کی کسی مضصوص نتیجہ کا سبب ہے؟ الیکٹران کاشف پر بیٹھا شخص اپنی پیمائش کے ذریعہ پوزیٹران کاشف پر بیٹھے شخص کو اشارہ نہیں بھیج سکتا ہے چونکہ یہ اپنی پیمائش کے نتیجہ کو قابو نہیں کرتا یہ الیکٹران کو ہم میدان ہونے پر مجبور نہیں  کر سکتا ہے جیسا نقطہ \عددی{X} پر کیڑا کے ساے پر وہ شخص اثرانداز نہیں ہوسکتا، ہاں الیکٹران کاشف پر بیٹھا شخص فیصلہ کر سکتا ہے کہ وہ پیمائش کرے یا نہ کرے تاہم پوزیٹران کاشف پر بیٹھا شخص اپنی پیمائشی نتائج کو دیکھ کر یہ نہیں بتا سکتا کہ الیکٹران پر پیمائش کی گئی یانہیں دونوں کاشف کے نتائج پر علیحدہ علیحدہ غور کرنے سے مکمل بلاواستہ مواد دیکھنے کو ملتا ہے۔ صرف دونوں مواد کا ایک دوسرے کے ساتھ موازنہ کرنے سے ہمیں ان کے بیچ باہم رشتہ نظر آتا ہے کسی دوسرے جمودی چوکھٹ میں الیکٹران کی پیمائش سے قبل پوزیٹران کی پیمائش کی جائے گی لیکن اس کے باوجود اس سے کوئی منتقی تضاد پیدا نہیں ہوتا۔ دیکھا گیا باہم رشتہ اس پر منحصر نہیں کہ ہم کہیں الیکٹران کی پیمائش پوزیٹران کی پیمائش پر اثرانداز ہوتی ہے یا پوزیٹران کی پیمائش الیکٹران کی پیمائش پر اثرانداز ہوتی ہے۔ یہ ایک نہایت نازک اور خوبصورت اثر ہے جو بلا واستہ مواد کے بیچ باہم رشتہ کی صورت میں نظر آتا ہے۔

یوں ہمیں مختلف قسم کے اثرات کی بات کرنی ہوگی سببی قسم جو وصول کنندہ کی کسی طبعی خاصیت میں حقیقی تبدیلیاں پیدا کرتا ہو جنہیں صرف زیلی نظام پر تجرباتی پیمائش سے کشف کیا جا سکتا ہو اور آسمانی قسمپ جو توانائی یا معلومات کی ترسیل نہیں کرتا اور جس کے لیئے واحد ثبوت دو علیحدہ زیلی نظاموں کے مواد کے بیچ باہم رشتہ ہے۔ اس باہم رشتہ کو کسی بھی طرح کسی ایک زیلی نطام میں تجربات کے نتائج کو دیکھ کر کشف نہیں کیا جا سکتا ہے۔ سببی اثرات  رشنی کی رفتار سے تیز حرکت نہیں کر سکتے ہیں جبکہ آسمانی اثرات پر ایسی کوئی پابندی عائد نہیں۔ تفاعل نوج کی انہدام سے وابستہ اثرات مئخرالذکر قسم کی ہے جس کا روشنی سے تیز سفر کرنا حیران کن ضرور ہوسکتا ہے لیکن تباہ کن نہیں ہے۔

