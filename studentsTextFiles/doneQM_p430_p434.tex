
\حصہ{شروڈنگر کی بلّی}
کوانٹم میکانیات میں پیمائش کا عمل ایک شرارتی کردار ادا کرتا ہے جس میں عدم تعینیت غیر مکامیت تفاعل موج کا انہدام اور باقی تمام تصوراتی مشکلات رونما ہتی ہیں۔ پیمائش کی غیر موجودگی میں مساوات شروڈنگر کے تحت تفاعل موج قابلِ تعین طریقہ سے ارتقا کرتا ہے اور کوانٹم میکانیات کسی بھی سادہ نظریہ میدان کی طرح نظر آتا ہے جو کلاسیکی برقی حرکیات سے بہت سادہ ہوگا چونکہ دو میدان \عددی{E} اور \عددی{B} کی بجائے اس میں واحد ایک غیر سمتی \عددی{\psi} پایا جاتا ہے۔ یہ پیمائش کا عمل ہی ہے جو کوانٹم میکانیات میں عجیب و غریب کردار ادا کرتے ہوئے اس کو سمجھ سے باہر خواص سے نوازتا ہے۔ یہ پیمائش حقیقت میں ہے کیا؟ اسے گیگر طبعی عوامل سے کیا منفرد بناتا ہے اور ہم کس طرح جان سکتے ہیں کہ پیمائش کی گئی ہے؟

شعوڈنگر نے اپنے مشہر تضاد بلّی کے مفروضہ نے اس بنیادی سوال کو پیش کیا۔

ایک بلّی کو فولاد کے ایک بند ڈبے میں بند کیا جاتا ہے اس ڈبے میں ایک گائگر گنت کار اور کسی تاب کار مادہ کی اتنی چھوٹی مقدار رکھی جاتی ہے جس کا ایک گھنٹا میں صرف ایک جوہر کے تحلیل ہونے کا امکان ہو تاہم یہ بھی ممکن ہے کہ کوئی جوہر تحلیل نہ ہو تحلیل کی صورت میں گنت کار اس ڈبے میں ایک زہریلی گیس چھوڑتا ہے۔ ایک گھنٹا گزرنے کے بعد ہم کہہ سکتے ہیں کہ تحلیل نہ ہونے کی صورت میں یہ بلّی زندہ ہوگی۔ پہلی تحلیل اس کو زہر سے مار دیتی۔اس مکمل نظام کا تفاعل موج اس حقیقت کو ظاہر کرنے کے لیئے زندہ اور مردہ بلّی کے برابر حصوں پر مشتمل ہوگا۔

ایک گھنٹا کے بعد بلّی کا تفاعل موج درج ذیل روپ کا ہوگا
\begin{align}
	\psi=\frac{1}{\sqrt{2}}(\psi_{\text{\RL{زندہ}}}+\psi_{\text{\RL{مردہ}}})
\end{align}
یہ بلّی نہ تو زندہ اور نہ ہی مردہ ہے بلکہ پیمائش سے پہلے یہ ان دونوں کا ایک خطی جوڑ ہوگا یہاں کھڑکی سے اندر دیکھ کر بلّی کا حال جاننے کو پیمائش تصور کیا جائے گا۔ آپ کا دیکھنمے کا عمل بلّی کو زندہ یا مردہ ہونے پر مجبور کرتا ہے ایسی صورت میں اگر بلّی مردہ پائی جائے تو یقیناً اس کے زمہدار آپ ہی ہیں چونکہ آپ نے کھڑکی سے دیکھ کر اسے قتل کیا۔

شروڈنگر اس تمام کو ایک بکواس سے زیادہ نہیں سمجھتا تھا اور میرے خیال سے زیادہ تر ماہرِ طبیعیات ان  کے ساتھ متفق ہیں۔ کلاں بین اجسام کا دو مختلف حالات کی ایک خطی جوڑ کی صورت میں ہونے کا تصور بے معنی ہے۔ ایک الیکٹران تو ہم میدان اور خلاف میدان کے ایک خطی جوڑ کی صورت میں ہو سکتا ہے لیکن ایک بلّی زندہ اور مردہ حالات کے ایک خطی جوڑ کی صورت میں نہیں ہوسکتی ہے۔ اس کو کوانٹم میکانیات کی تقلید پسند تشریح کے ساتھ کس طرح ہم اہنگ بنایا جا سکتا ہے۔

شماریاتی مفہوم کے لحاظ سے مقبول ترین جواب یہ ہے کہ گنت کار کی گنتی پیمائش ہوگی نا کہ کھڑکی میں سے انسانی مشاہدہ پیمائش سے مراد وہ عمل ہے جو کلاں بین نظام پر اثر انداز ہو جو یہاں گنت کار ہے۔ پیمائش کا عمل اس لمحہ پر رونما ہوگا جب خردبین نظام جسے کوانٹم میکانیات کے قوانین بیان کرتا ہے کلاں بین نظام جسے کلاسیکی میکانیات کے قواعد بیان کرتے ہیں کے ساتھ اس طرح باہم عمل کرے جس سے دائمی تبدیلی رونما ہو۔ کلاں بین نظام ازخود منفرد حالات کی ایک خطی جوڑ کا مکین نہیں ہوسکتا ہے۔
\حصہ{کوانٹم زینو تضاد}
اس عجیب قصہ کی اہم ترین خاصیت تفاعل موج کا انہدام ہے۔ ایک پیمائش کے فوراً بعد دوسری پیمائش سے اسی نتیجہ کے حصول کی خاطر خالصتاً نظریاتی بنیادوں پر اسے متعارف کیا گیا تھا یقیناً اس دو رس  اصول موضوعہ کے قابلِ مشاہدہ اثرات بھی ہوں گے۔ مسرا اور سدرشان نے سن \num{1977} میں تفاعلی موج کی انہدام کا ایک ڈرامائی تجرباتی مظاہرہ تجویز کیا جسے انہوں نے کوانٹم زینو اثر کا نام دیا۔ ان کا تصور یہ تھا کہ ایک غیر مستحکم نظام مثلا ہیجان حال میں ایک جوہر کو بار بار پیمائشی عمل سے گزارا جائے۔ ہر ایک مشاہدہ تفاعل موج کو منہدم کر کے گھڑی کو دوبارہ صفر وسے چالو کرے گا اور یوں زیریں حال میں متوقے انتقال کو غیر معائنہ مدد تک روکا جاسکتا ہے۔

فرض کریں ایک نظام ہیجان حال \عددی{\psi_2} سے آغاز کرترا ہے اور زمینی حال \عددی{\psi_1} میں منتقلی کے لیئے اس کا قدرتی عرصہ حیات \عددی{\tau} ہے۔ عام طور پر \عددی{\tau} سے کافی کم وقتوں کے لیئے انتقالی احتمال وقت \عددی{t} کا راست متناسب ہوگا \حوالہء{مساوات \num{9.42}} دیکھیں چونکہ انتقالی شرح \عددی{1/\tau} ہے لحاظہ درج ذیل ہوگا 
\begin{align}
	P_{2\to1}=\frac{t}{\tau}
\end{align}
وقت \عددی{t} پر پیمائش کرنے کی صورت میں بالائی حال میں نظام ہونے کا احتمال درج ذیل ہوگا
\begin{align}
	P_2(t)=1-\frac{t}{\tau}
\end{align}
درض کریں ہم دیکھتے ہیں کے نظام بالائی حال میں ہی ہے ایسی صورت میں تفاعل موج واپس \عددی{\psi_2} پر منحدن ہوگا اور پورا عمل ایک بار نئے سرے سے دوبارہ شروع ہوگا۔ اگر ہم وقت \عددی{2t} پر دوسری پیمائش کریں تب بالائی حال میں نظام ہونے کا احتمال درج ذیل ہوگا 
\begin{align}
	\left(1-\frac{t}{\tau}\right)^2\approx1-\frac{2t}{\tau}
\end{align}
جو وہی ہے جو اس صورت ہوتا اگر ہم پہلی پیمائش کرتے ہی نہیں سادہ سوچ کے تحت ایسا ہی ہونا چاہیئے تھا۔ اگر ایسا ہی ہوتا تب نظام کا بار بار مشاہدہ کرنے سے کوئی فرق نہیں پڑتا اور نہ یی کوانٹم زینو اثر پیدا ہوتا تاہم بہت قلیل وقت کی صورت میں انتقالی احتمال وقت \عددی{t} کے بجائے \عددی{t^2} کا راست متانسب ہوگا \حوالہء{\num{9.39}} دیکھیں
\begin{align}
	P_{2\to1}=\alpha t^2
\end{align}
ایسی صورت میں دو پیمائشوں کے بعد بھی نظام کا بالائی حال میں ہونے کا احتمال درج ذیل ہوگا
\begin{align}
	\left(1-\alpha t^2\right)^2\approx 1-2\alpha t^2
\end{align}
جبکہ پہلی پیمائش نہ کرنے کی صورت میں اب احتمال درج ذیل ہوتا
\begin{align}
	1-\alpha(2t)^2\approx1-4\alpha t^2
\end{align}
آپ دیکھ سکتے ہیں کہ وقت \عددی{t} گزرنے کے بعد نظام کے مشاہدہ کی بنا زیریں حال میں منتقلی کا احتمال کم ہوا ہے۔

یقیناً \عددی{t=0} سے لیکر \عددی{t=T} تک \عددی{n} برابر وقفہ \عددی{T/n, 2T/n, 3T/n, \dots, T} پر نظام کا مشاہدہ کرنے کی وجہ سے اس دورانیہ کے آخر میں بھی نظام بالائی حال میں پائے جانے کا احتمال درج ذیل ہوگا
\begin{align}
	\left(1-\alpha(T/n)^2\right)^n\approx1-\frac{\alpha}{n}T^2
\end{align}
جو \عددی{n\to\infty} کی حد میں \عددی{1} تک پہنچتا ہے ایک غیر مستحکم نظام جس کا مسلسل مشاہدہ کیا جائے کبھی بھی تحویل نہیں ہوگا بعض مصنفین اس ماخوز سے اتفاق نہیں کرتے اور ان کے نزدیک یہ تفاعل موج کے انہدام غیر درست ہونے کا ثبوت ہے۔ تاہم ان کے سدلائل مشاہدہ کے مفہوم کی غلط تشریح پر مبنی ہے اگر بلبلا خانہ میں ایک ذرہ کی راہ کو مسلسل مشاہدہ کرار دے دیا جائے تب یہ بلکل درست ہوں گے چونکہ ایسی ذرات یقیناً تحویل ہوتے ہیں اور ان کا عرصہ حیات پرکاشف کا قابلِ پیمائش اثر نہیں پایا جاتا ہے تاہم ایسا ذرہ خانہ کے اندر جوہروں کے ساتھ خاد و خال  باہم عمل کرتا ہے جبکہ کوانٹم زینو اثر کے لیئے ضروری ہے کہ یکِ بعد دیگر پیمائشوں کے بیچ وقفہ اتنا کم ہو کہ نظام کو \عددی{t^2} خطہ میں پکڑا جائے۔

ہم دیکھتے ہیں کہ خود با خود انتقل کی صورت میں یہ تجربہ عملاً ممکن نہیں ہے۔ تاہم پیدا کردہ انتقال کی صورت میں نتائج کا نظریاتی پیشاً گوئی کے ساتھ مکمل اتفاق پایا جات ہے۔ بدقسمتی سے یہ تجربہ تفاعل موج کی انہدام کا ختمی ثبوت پیش نہیں کرسکتا ہے اس مشاہدہ کے دیگر وجوہات بھی دئے جاسکتے ہیں۔ 

میں نے اس کتاب میں ایک ہم اہہنگ اور بلا تضاد کہانی پیش کرنے کی کوشش کی ہے تفاعل موج \عددی{\psi} کسی ذرہ یا نظام کے حال کو ظاہر کرتا ہے۔ عمومی طور پر ای کذرہ کسی مخصوص حرکی خاصیت مثلاً مکام معیارِ حرکت توانائی زاویائی معیارِ حرکت وغیرہ کا حامل نہیں ہوتا اس وقت تک  جب پیمائشی عمل مداخلت نہ کرے کسی ایک تجربہ میں حاصل ایک مخصوص قیمت کا احتمال \عددی{\psi} کی شماریاتی مفہوم تعین کرتا ہے۔ پیمائشی عمل سے تفاعل موج منحدم ہوتا ہے جس کی بنا فوراً دوسری پیمائش لاظماً وہی نتیجہ دیگی۔ اگرچہ دیگر تشریحات مثلاً غیر مکامی درپردہ متغیر نظریات متعدد کائنات کا تصور بلا تضاد تاریخیں سگرہ نمونے وغیرہ بھی پائے جاتے ہیں لیکن میں یقین کرتا ہوں کہ یہ سب سے سادہ ہے جس سے عموماً ماہرِ طبیعیات اتفاق کرتے ہیں۔ یہ ہر تجربہ سے کامیابی سے ابھرا ہے تاہم یہ کہانی کا اختتام نہیں ہے ہمیں پیمائشی عمل کے بارے میں اور انہدام کے طریقے کار کے بارے میں بہت کچھ جاننا ہے عین ممکن ہے کہ آنے والے نسلیں زیادہ پیچیدا نظریہ جانتے ہوئے سوچتے ہوں کہ ہم اتنا سادہ کیسے ہوسکتے تھے۔

