\documentclass[leqno, b5paper]{khalid-urdu-book}
\begin{document}
\جزوحصہ{ہیجان حال کا عرصہ حیات}
مساوات \num{9.56} ہمارا بنیادی نتیجہ ہے جو تحرقی اخراج کی تحویلی شرح دیتی ہے۔ اب فرض کریں کسی طرح آپ بہت بڑی تعداد میں جوہر کو ہیجان حال منتقل کرتے ہیں۔ تحرقی اخراج کہ نتیجہ میں وقت کے ساتھ یہ تعداد گٹھے گی۔ بلخصوص وقتی دورانیہ \(dt\) میں جوہروں میں تعداد کی کمی \(Adt\) ہوگی۔
\begin{align}
	dN_b = -AN_bdt
\end{align}
جہاں ہم فرض کرتے ہیں کہ مزید نئے جوہر ہیجان انگیز نہیں کیئے جارہے ہیں۔ اس کو \(N_b(t)\) کے لیئے حل کرتے ہوئے درج ذیل حاصل ہوگا۔
\begin{align}
	N_b(t) = N_b(0)e^{-At}
\end{align}
ظاہر ہے کہ ہیجان حال میں تعداد قوت نمائی طور پر کم ہوگی جہاں وقتی مستقل درج ذیل ہوگا۔
\begin{align}
	\tau = \frac{1}{A}
\end{align}
جسی اس حال کا عرصہ حیات کہتے ہیں۔ ایک عرصہ حیات میں \(N_b(t)\) کی قیمت آغازی قیمت کی \(1/e \approx \num{0.368}\) ہوگی۔

میں اب تک فرض کرتا رہا ہوں کہ نظام میں صرف دو حالات پائے جاتے ہیں۔ تاہم سادہ علامتیت کے بنا ایسا کیا گیا تحرقی اخراج کا کلیہ مساوات \num{9.56} دیگر قابلِ رصوض سطح سے قطع نظر حال \(\psi_b \rightarrow \psi_a\) تحویلی شرح دیتی ہے سوال \num{9.15} دیکھیں۔ عمومی طور پر ایک ہیجان جوہر کے کئی مختلف انداز تنزل ہوں گے۔ یعنی \(\psi_b\) کا تنزل بہت ساری زیریں توانائی حالات \((\psi_{a1}, \psi_{a2}, \psi_{a3}, \dots)\) میں ہو سکتا ہے۔ ایسی صورت میں تمام تحویلی شرح جمع ہوکر درج ذیل عرصہ حیات دیں گی۔
\begin{align}
	\tau = \frac{1}{A_1+A_2+A_3+\dots}
\end{align}
\ابتدا{مثال}
فرض کریں ایک سپرنگ کے ساتھ باندھا ہوا بار \عددی{q} محور \عددی{x} پر ارتعاش کا پابند ہے۔ فرج کریں یہ حال \(\mid n \rangle\) مساوات \num{2.61} سے آغاز کر کے خود با خود اخراجہ تنزل کی بنا حال \(\mid n^\prime \rangle\) پہنچتا ہے۔ مساوات  \num{9.44} کے تحت درج ذیل ہوگا۔
\begin{align*}
	p = q\langle n\abs{x}n^\prime\rangle\hat{i}
\end{align*}
آپ نے سوال \num{3.33} میں \عددی{x} کے قالبی ارکان تلاش کئے۔
\begin{align*}
	\langle n\abs{x}n^\prime\rangle = \sqrt{\frac{\hbar}{2m\omega}}(\sqrt{n^\prime}\delta_{n.n^\prime-1}+\sqrt{n}\delta_{n^\prime.n-1})
\end{align*}
جہاں مرتعش کی قدرتی تعدد \(\omega\) ہے۔ مجھے تحرقی اخراج کے تعدد کے لیئے اس حرف کی ضرورت اب پیش نہیں آئے گی۔ چونک ہم اخراج کی بات کررہے ہیں لحاظہ \(n^\prime\) لاظمی طور پر \عددی{n} سے نیچے ہوگا۔ ہماری اس مقصد کی غرض سے تب درج ذیل ہوگا۔
\begin{align}
	p = q\sqrt{\frac{n\hbar}{2m\omega}}\delta_{n^\prime.n-1}\hat{i}
\end{align}
بظاہر تحویل سیڑھی پر صرف ایک قدم نیچے ممکن ہے اور اخراجی فوٹان کا تعدد درج ذیل ہے۔
\begin{align}
	\omega_0 = \frac{E_n-E_n^\prime}{\hbar} = \frac{(n+1/2)\hbar\omega - (n^\prime + 1/2)\hbar\omega}{\hbar} =(n-n^\prime)\omega = \omega
\end{align}
حیرت کی بات نہیں کہ نظام کلاسیکی ارتعاشی تعدد پر اخراج کرتا ہے۔ تحویلی شرح مساوات \num{9.56} درج ذیل ہوگا۔
\begin{align}
	A = \frac{nq^2\omega^2}{6\pi\epsilon_0mc^3}
\end{align}
اور \عددی{n}ویں ساکن حال کا عرصہ حیات درج ذیل ہوگا۔
\begin{align}
	\tau_n = \frac{6\pi\epsilon_0mc^3}{nq^2\omega^2}
\end{align}
چونکہ ہر ایک اخراجی فوٹان \(\hbar\omega\) توانائی ساتھ لے جاتا ہے لحاظہ اخراجی طاقت \(A\hbar\omega\) ہوگا۔
\begin{align*}
	P = \frac{q^2\omega^2}{6\pi\epsilon_0mc^3}(n\hbar\omega)
\end{align*}
یا \عددی{n}ویں حال میں مرتعش کی توانائی \(E = (n+1/2)\hbar\omega\) لیتے ہوئے درج ذیل ہوگا۔
\begin{align}
	P = \frac{q^2\omega^2}{6\pi\epsilon_0mc^3}(E-\frac{1}{2}\hbar\omega)
\end{align}
ابتدائی توانائی \عددی{E} کا کوانٹم مرتعش اوسطاً اتنی طاقت خارج کرے گا۔

موازنہ کی خاطر اسی طاقت کے کلاسیکی مرتعش کی اوسط اخراجی طقت تعین کرتے ہیں۔ کلاسیکی برقی حرکیات کے تحت مسرع بار \عددی{q} کا اخراجی طاقت کلیہ لارمر دیتا ہے۔
\begin{align}
	P = \frac{q^2a^2}{6\pi\epsilon_0c^3}
\end{align}
ہارمونی مرتعش \(x(t) = x_0\cos(\omega t)\) جس کا حیطہ \عددی{x_0} ہوگا میں مسرع \(a = -x_0\omega^2\cos(\omega t)\) ہوگا۔ پورے ایک چکر پر تب اوسط درج ذیل ہوگا۔
\begin{align*}
	P = \frac{q^2x^2_0\omega^4}{12\pi\epsilon_0c^3}
\end{align*}
لیکن اس مرتعش کی توانائی \(E = (1/2)m\omega^2x_0^2\) ہے لحاظہ \(x_0^2 = 2E/m\omega^2\) ہوگا۔ جس سے درج ذیل لکھا جا سکتا ہے۔
\begin{align}
	P = \frac{q^2\omega^2}{6\pi\epsilon_0mc^3}E
\end{align}
توانائی \عددی{E} کا کلاسیکی مرتعش اوسطاً اتنی طاقتی اخراج کرتا ہے۔ کلاسیکی حد \((\hbar\rightarrow0)\) میں کلاسیکی اور کوانٹم کلیات آپس میں متفق ہیں۔ البتہ زمینی حال کو کوانٹم کلیہ مساوات \num{9.65} تحفظ دیتا ہے۔ اگر \(E = (1/2)\hbar\omega\) ہو تب مرتعش طاقتی اخراج نہیں کرے گا۔
\انتہا{مثال}
\ابتدا{سوال}
ہیجان حال کی نصف حیات سے مراد وہ دورانیہ ہے جس میں بہت زیادہ تعداد کے جوہروں میں سے نصف تحویل کرتے ہوں۔ نصف حیات اور حال کے عرصیہ حیات کے بیچ رشتہ تلاش کریں۔
\انتہا{سوال}
\ابتدا{سوال}
ہائڈروجن کے چاروں \(n=2\) حالات کے لیئے عرصہ حیات کو سیکنڈوں میں تلاش کریں۔ اشارہ: آپ کو \(\langle \psi_{100}\abs{x}\psi_{200} \rangle, \langle \psi_{100}\abs{y}\psi_{211} \rangle\) وغیرہ وغیرہ۔ طرز کے قالبی ارکان کی قیمتیں تلاش کرنی ہوں گی۔ یاد رہے کہ \(x = r\sin\theta\cos\phi, y = r\sin\theta\sin\phi\) اور \(z = r\cos\theta\) ہوں گے۔ ان میں سے زیادہتر تکملات صفر کے برابر ہوں گے لحاظہ حساب شروع کرنے سے پہلے ین پر ایک گہری نظر ضرور ڈالیں۔

جواب: سوائے \(\psi_{200}\) جو لامتناہی ہے باقی تمام کے لیئے \(\num{1.60}\times10^{-9}\) سیکنڈز ہوگا۔
\انتہا{سوال}
\end{document}
