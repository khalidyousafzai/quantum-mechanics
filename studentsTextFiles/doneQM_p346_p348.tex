
\جزوحصہ{سائن نما اضطراب}
فرض کریں اضطراب میں تابعیت وقت سائن نما ہو 
\begin{align}
	H'(r, t) = V(r)\cos(\omega t)
\end{align}
تب درج ذیل ہوگا
\begin{align}
	H'_{ab}=V_{ab}\cos(\omega t)
\end{align}
جہاں \عددی{V_{ab}} درج ذیل ہے
\begin{align}
	V_{ab}\equiv\langle\psi_a\mid V\mid\psi_b\rangle
\end{align}
عملاً تقریباً ہر صورت میں وتری قالبی ارکان صفر ہوتے ہیں لحاظہ پہلے کی طرح یہاں بھی میں یہی فرض کروں گا۔ یہاں سے آگے چلتے ہوئے ہم صرف رتبہ اوّل تک متغیرات تلاش کریں گے لحاظہ زیرِ بالا میں تربہ کی نشاندہی نہیں کی جائے گی۔ رتبہ اوّل تک درج ذیل ہوگا \حوالہء{مساوات \num{9.17}} 
\begin{align}
	c_b(t) &\cong-\frac{i}{\hslash}V_{ba}\int_{0}^{t}\cos(\omega t')e^{i\omega_0 t'}\dif t' = -\frac{iV_{ba}}{2\hslash}\int_{0}^{t}\left[e^{i(\omega_0+\omega)t'}+e^{i(\omega_0-\omega)t'}\right]\dif t' \nonumber \\
	&=-\frac{V_{ba}}{2\hslash}\left[\frac{e^{i(\omega_0+\omega)t}-1}{\omega_0+\omega}+\frac{e^{i(\omega_0-\omega)t}-1}{\omega_0-\omega}\right]
\end{align}
یہی جواب ہے لیکن اس کے ساتھ کام کرنا ذرا دشوار ہوگا۔ انتقالی تعدد \عددی{\omega_0} کے بہت قریب جبری تعدد \عددی{\omega} پر توجہ رکھنے سے چکور کوسین میں دوسرا جزو غالب ہوگا جس سے چیزیں بہت آسان ہوجاتی ہیں۔ ہم درج ذیل فرض کرتے ہیں
\begin{align}
	\omega_0+\omega\gg\abs{\omega_0-\omega}
\end{align}
یہ کوئی بہت بڑی پابندی نہیں ہے چونکہ کسی دوسری تعدد پر انتقلا کا احتمال نہ ہونے کے برابر ہوگا۔ یوں پہلے جزو کو نظرانداز کرتے ہوئے درج ذیل لکھا جا سکتا ہے
\begin{align}
	c_b(t) &\cong-\frac{V_{ba}}{2\hslash}\frac{e^{i(\omega_0-\omega)t/2}}{\omega_0-\omega}\left[e^{i(\omega_0-\omega)t/2}-e^{-i(\omega_0-\omega)t/2}\right]\nonumber \\
	&=-i\frac{V_{ba}}{\hslash}\frac{\sin[(\omega_0-\omega)t/2]}{\omega_0-\omega}e^{i(\omega_0-\omega)t/2}
\end{align}
ایک ذرہ جو حال \عددی{\psi_a} سے آغاز کرے کا لمحہ\عددی{t} پر حال \عددی{\psi_b} میں پائے جانے کا احتمال درج ذیل ہوگا جس کو انتقالی احتمال کہتے ہیں 
\begin{align}
	P_{a\to b}(t)=\abs{c_b(t)}^2\cong\frac{\abs{V_{ab}^2}}{\hslash^2}\frac{\sin^2[(\omega_0-\omega)t/2]}{(\omega_0-\omega)^2}
\end{align}
وقت کے لحاظ سے انتقالی احتمال سائن نما ارتعاش کرتا ہے \حوالہء{شکل \num{9.1}}۔ یہ \عددی{\abs{V_{ab}}^2/\hslash^2(\omega_0-\omega)^2} کی زیادہ سے زیادہ قیمت تک پہنچ کر جو لاظمی طور پر ایک سے بہت کم ہے ورنہ کم اضطراب کا مفروضہ درست نہیں ہوگا یہ واپسصفر کو گرتا ہے۔ لمحات \عددی{t_n=2n\pi/\abs{\omega_0-\omega}} جہاں \عددی{n=1, 2, 3, \dots} ہیں پر ذرہ لاظماً نچلی حال میں ہوگا اگر آپ منتقلی کا احتمال بڑھانا چاہتے ہیں اضطراب کو لمبے عرصہ کے لیئے چالو نہ کریں۔ بہتر ہوگا کہ آپ وقت \عددی{\pi/\abs{\omega_0-\omega}} پر اضطراب کو روک کر نظام کو بالائی حال میں پانے کی اُمید کریں۔ \حوالہء{سوال \num{9.7}} میں آپ دیکھیں گے کہ دو حالات کے بیچ انتقال نظریہ اضطراب کی پیدا کرادہ  مسنوئی خاصیت  نہیں ہے بلکہ بلکل ٹھیک حال میں بھی ایسا ہوگا تاہم منتقلی کا تعدد کچھ مختلف ہوگا۔

جیسا میں ذکر کر چکا ہوں انتقال کی احتمال اس صورت زیادہ سے زیادہ ہوگا جب جبری تعدد قدرتی تعدد \عددی{\omega_0} کے قریب ہو۔ \حوالہء{شکل \num{9.2}} میں \عددی{\omega} کے لحاظ سے \عددی{P_{a\to b}} ترسیم کر کے اس حقیقت کو اجاگر کیا گیا ہے۔ چوٹی کی انچائی \عددی{(\abs{V_{ab}t/2\hbar})^2} جبکہ چوڑائی \عددی{4\pi/t} ہے یوں وقت گزرنے کے ساتھ ساتھ اسکی بلندی بڑھتی ہے اور چوڑائی گھٹتی ہے۔ بظاہر زیادہ سے زیادہ قیمت بغیر کسی حد کے بتدریج بڑھتی ہے تاہم ایک پر پہنچنے سے بہت پہلے اضطراب کا مفروضہ ناکرا ہو جاتا ہے۔ لحاظہ ہم بہت کم \عددی{t} کے لیئے اس نتیجہ پر یقین کر سکتے ہیں۔ \حوالہء{سوال \num{9.7}} میں آپ دیکھیں گے کہ بلکل ٹھیک ٹھیک نتیجہ کبھی بھی ایک سے ایک تجاوز نہیں کرتا ہے۔

\ابتدا{سوال}
پہلا جزو \حوالہء{مساوات \num{9.25}} میں \عددی{\cos(\omega t)} کے \عددی{e^{i\omega t}/2} سے جبکہ دوسرا \عددی{e^{-i\omega t}/2} سے آتا ہے یوں پہلے جزو کو  نظرانداز کرنا باضابطہ طور پر \عددی{H'=(V/2)e^{-i\omega t}} لکھنے کا معادل ہے یعنی ہم درج ذیل کہہ سکتے ہیں
\begin{align}
	H'_{ba}=\frac{V_{ba}}{2}e^{-i\omega t},&&H'_{ab}=\frac{V_{ab}}{2}e^{i\omega t}
\end{align}
ہیملٹنی قالب کو ہرمیشی بنانے کی خاطر مئاخر الذکر کی ضرورت پیش آتی ہے۔ آپ کہہ سکتے ہیں ہم \عددی{c_a(t)} کے لیئے \حوالہء{مساوات \num{9.25}} کی طرح کلیہ میں غالب جزو کو چنتے ہیں۔ اسکو گھومتی موج تخمین کہتے ہیں جناب رابی نے دیکھا کہ حساب کی آغاز میں گھومتی موج تخمین کرتے ہوئے \حوالہء{مساوات \num{9.13}} کو بغیر نظریہ اضطراب اور میدان کی زور کے بارے میں کچھ بھی فرض کیئے بغیر بلکل ٹھیک ٹھیک حل کیا جا سکتا ہے۔

(الف) عمومی ابتدائی معلومات \عددی{c_a(0)=1, c_b(0)=0} کے لیئے گھومتی موج تخمین \حوالہء{مساوات \num{9.29}} لیتے ہوئے \حوالہء{مساوات \num{9.13}} حل کریں۔ اپنے جوابات \عددی{c_a(t)} اور \عددی{c_b(t)} کو رابی تعدد 
\begin{align}
	\omega_r\equiv\frac{1}{2}\sqrt{(\omega-\omega_0)^2+(\abs{V_{ab}}/\hslash)^2}
\end{align}
کی صورت میں لکھیں۔

(ب) انتقالی احتمال \عددی{P_{a\to b}(t)} تعین کر کے دیکھائیں کہ یہ کبھی بھی ایک سے تجاوز نہیں کرتا۔ تصدیق کریں کہ \عددی{\abs{c_a(t)}^2+\abs{c_b(t)}^2=1} ہوگا۔

(ج) دیکھیں کہ کم اضطراب کی صورت میں \عددی{P_{a\to b}(t)} عین نظریہ اضطران کے نتیجہ \حوالہء{مساوات \num{9.28}} کے تحت ہوگا۔ سیاق و سباق کے لحاظ سے یہاں کم سے کیا مراد ہے اور \عددی{V} پر یہ کیا پابندی عاید کرتی ہے۔

(د) نظام پہلی بار اپنی ابتدائی حال میں کتنی دیر میں واپس آئے گا؟
\انتہا{سوال}
