\documentclass[leqno, b5paper]{khalid-urdu-book}
\begin{document}
\ابتدا{سوال}
متعدد سطحی نظام کے لیئے مساوات \num{9.1} اور مساوات \num{9.2} 
\begin{align}
	H_0\psi_n = E_n\psi_n, \langle \psi_n\mid\psi_m \rangle = \delta_{nm}
\end{align}
کو عمومیت دیتے ہوئے تابع وقت نظریہ اضطراب تشکیل دیں۔ لمحہ \(t=0\) پر ہم اس اضطراب \(H^\prime(t)\) چالو کرتے ہیں۔ یوں کل ہیملٹنی درج ذیل ہوگا۔
\begin{align}
	H = H_0 + H^\prime(t)
\end{align}
(الف) مساوات \num{9.6} کی تعمیمی صورت درج ذیل ہوگی۔
\begin{align}
	\psi(t) = \sum c_n(t)\psi_ne^{-iE_nt/\hbar}
\end{align}
دیکھائیں کہ درج ذیل ہوگا
\begin{align}
	c_m = -\frac{i}{\hbar}\sum_{n} c_nH^\prime_{mn}e^{i(E_m-E_n)t/\hbar}
\end{align}
جہاں \(H^\prime_{mn}\) درج ذیل ہے
\begin{align}
	H^\prime_{mn} \equiv \langle \psi_m\abs{H^\prime}\psi_n \rangle
\end{align}
(ب) اگر نظام حال \(\psi_N\) میں آغاز کریں تب دیکھائیں کہ رتبہ اوّل نظریہ اضطراب میں درج ذیل
\begin{align}
	c_N(t)\cong1-\frac{i}{\hbar}\int_{0}^{t}H^\prime_{NN}(t^\prime)dt^\prime
\end{align}
اور درج ذیل ہوگا
\begin{align}
	c_m(t)\cong-\frac{i}{\hbar}\int_{0}^{t}H^\prime_{mN}(t^\prime)e^{i(E_m-E_N)t^\prime/\hbar}dt^\prime \quad(m\neq N)
\end{align}
(ج) فرض کریں لمحہ \(t=0\) پر چالو اور بعد میں لمحہ \عددی{t} پر منقتع کرنے کے علاوہ \(H^\prime\) مستقل ہے۔ حال \عددی{N} سے حال \(M(M\neq N)\) میں تحویل کے احتمال کو \عددی{t} کا تفاعل لکھیں۔ جواب:
\begin{align}
	4\abs{H^\prime_{MN}}^2\frac{\sin^2[(E_N-E_M)t/2\hbar]}{(E_N-E_M)^2}
\end{align}
(د) فرض کریں \(H^\prime\) وقت کا سائن نما تفاعل \(H^\prime=V\cos(\omega t)\) ہے۔ عمومی مفروضے فرض کرتے ہوئے دیکھائیں کہ صرف توانائی \(E_M = E_N\pm\hbar\omega\) کے حالات میں تحویل ہوسکتی ہے اور انکا احتمال درج ذیل ہے۔
\begin{align}
	P_{N\rightarrow M} = \abs{V_{MN}}^2\frac{\sin^2[(E_N-E_M\pm\hbar\omega)t/2\hbar]}{(E_N-E_M\pm\hbar\omega)^2}
\end{align}
(و) فرض کریں ایک متعدد سطحی نظام پر غیر اتساکی برقناطیسی روشنی ڈالی جاتی ہے۔ حصہ 9.2.3 کو دیکھتے ہوئے دیکھائیں کہ دو سطحی نظام کے لیئے تحرقی اخراج کی تحویلی شرح وہی کلیہ مساوات \num{9.47} دیگا۔
\انتہا{سوال}
\ابتدا{سوال}
عددی سر \(c_m(t)\) کو رتبہ اوّل تک سوال \num{9.15} (ج) اور (د) کے لیئے تلاش کریں۔ معمولزنی شرط 
\begin{align}
	\sum_{m}\abs{c_m(t)}^2 = 1
\end{align}
کی تصدیق کر کے تزاد اگر موجود ہو پر تبصرہ کریں۔ فرض کریں آپ ابتدائی حال \(\psi_N\) میں رہنے کا احتمال جاننا چاہتے ہیں۔ کیا \(\abs{c_N(t)}^2\) یا \(1-\sum_{m\neq N}\abs{c_m(t)}^2\) کا استعمال بہتر ثابت ہوگا؟
\انتہا{سوال}
\ابتدا{سوال}
ایک لامتناہی چکور کنواں کہ \عددی{N}ویں حال میں وقت \(t=0\) پر ایک ذرہ آغاز کرتا ہے۔ وقتی طور پر کنواں کی تہ بلند ہو کر واپس اپنی جگہ نیچے بیٹھتی ہے جس کے تحت کنواں کے اندر مخفیہ یکساں ضرور لیکن تابع وقت ہوگی \(V_0(t)\) جہاں \(V_0(0) = V_0(T) = 0\) ہوگا۔

(الف) مساوات \num{9.82} استعمال کرتے ہوئے \(c_m(t)\) کی ٹھیک ٹھیک قیمت دریافت کریں اور دیکھائیں کہ تفاعل موج کی حیط زاویائی دور تبدیل ہوگا لیکن تحویل نہیں ہوگی۔ تفاعل \(V_0(t)\) کی صورت میں تبدیلی حیط، تبدیلی زاویائی دور \(\psi(T)\) تلاش کریں۔

(ب) اسی مسئلہ کو رتبہ اوّل نظریہ اضطراب سے حل کر کے دونوں نتائج کا موازنہ کریں۔

تبصرہ: ہر  اُس صورت میں جب مخفیہ کے ساتھ اضطراب \عددی{x} میں مستقل نا کے \عددی{t} میں جمع کرتا ہو یہی نتیجہ حاصل ہوگا۔ یہ صرف لامتناہی چکور کنواں کی خاصیت نہیں ہے۔ سوال \num{1.8} کے ساتھ موازنہ کریں۔
\انتہا{سوال}
\end{document}
