
\ابتدا{سوال}
ایک ہائڈروجن جوہر کو تابع وقت برقی میدان \عددی{E=E(t)\hat{k}} میں رکھا جاتا ہے۔ زمینی حال \عددی{n=1} اور چار گنا انحطاطی پہلا ہیجان حالات \عددی{n=2} کے بیچ اضطراب \عددی{H'=eEz} کے چاروں قالبی ارکان \عددی{H'_{ij}} کا حساب لگائیں۔ یہ بھی دیکھائیں کہ پانچوں حالات کے لیئے \عددی{H'_{ii}=0} ہوگا۔ تبصرہ محور \عددی{z} کے لحاظ سے طاق ہونے کو بروہِ کار لاتے ہوئے آپ کو صرف ایک تکمل حل کرنا ہوگا۔ اس روپ کے اضطراب زمیہنی حال سے \عددی{n=2} حالات میں سے صرف ایک تک رسائی دیتا ہے لحاظہ زیادہ بلند ہیجان حالات میں منتقلی کو نظرانداز کرتے ہوئے یہ نظام دو حالات تنظیم کے طور پر کام کرے گا۔
\انتہا{سوال}
\ابتدا{سوال}
غیر تابع وقت اضطراب کی صورت میں \عددی{c_a(0)=1} اور \عددی{c_b(0)=0} لیتے ہوئے \حوالہء{مساوات \num{9.13}} حل کریں۔ تصدیق کیجیئے گا کہ \عددی{\abs{c_a(t)}^2+\abs{c_b(t)}^2=1} ہے۔ تبصرہ: ظاہری طور پر یہ نظام خالص \عددی{\psi_a} اور کسی \عددی{\psi_b} کے بیچ ارتعاش کرتا ہے۔ کیا یہ میرے اس عمومی دعوے کی نفی نہیں کرتا کے غیر تابع وقت اضطراب کی صورت میں انتقال نہیں ہوگا؟ جی نہیں لیکن اس کی وجہ ذرا نازک ہے یہاں \عددی{\psi_a} اور \عددی{\psi_b} نہ کبھی ہیملٹنی کے امتیازی تفاعلات تھے اور نہ ہیں۔ توانائی کی پیمائش کبھی بھی \عددی{E_a} یا \عددی{E_b} نہیں دیگی۔ تابع وقت نظریہ اضطراب میں عمومی طور پر ہم کسی دورانیہ کے لیئے اضطراب چالو کر کے نطام پر نظر ڈالنے کی خاطر اضطراب ختم کرتے ہیں۔ صرف آغاز اور اختتام میں \عددی{\psi_a} اور \عددی{\psi_b} بلکل ٹھیک ہیملٹنی کے امتیازی حالات ہوں گے اور صرف انہی صورتوں میں ہم نظام میں انتقال کی بات کر سکتے ہیں۔ یوں موجودہ مسئلہ میں فرض کیجیئے گا کہ وقت \عددی{t=0} پر اضطراب چالو کیا جاتا ہے جسے وقت \عددی{t} پر منقتع کیا جاتا ہے۔ اس سے آپ کے حساب پر کوئی فرق نہیں پڑے گا تاہم نتائج کی معقول تشریح ممکن ہوگی۔
\انتہا{سوال}
\ابتدا{سوال}
فرض کریں اضطراب کی شکل و صورت وقت کے لحاظ سے \عددی{\delta} تفاعل ہے
\begin{align*}
	H'=U\delta(t)
\end{align*}
جہاں \عددی{U_{aa}=U_{bb}=0} ہے اور \عددی{U_{ab}=U^*_{ba}\equiv\alpha} لیں۔ اگر \عددی{c_a(-\infty)=1} اور \عددی{c_b(-\infty)=0} ہوں تب \عددی{c_a(t)} اور \عددی{c_b(t)} کیا ہوں گے اور کیا \عددی{\abs{c_a(t)}^2+\abs{c_b(t)}^2=1} ہوگا۔ انتقال ہونے کا احتمال \عددی{t\to\infty} کے لیئے \عددی{P_{a\to b}} کیا ہوگا۔ اشارہ: آپ ڈیلٹا تفاعل کو مستطیلوں کی تسلسل کی تحدیدی حد لے سکتے ہیں۔

جواب: \عددی{P_{a\to b}=\sin^2(\abs{\alpha}/\hslash)	} 
\انتہا{سوال}
\جزوحصہ{تابع وقت نظریہ اضطراب}
اب تک سب کچھ بلکل درست رہا ہے ہم نے اضطراب کی جسامت کے بارے میں کچھ فرض نہیں کیا تاہم کم \عددی{H'} کی صورت میں ہم \حوالہء{مساوات \num{9.13}} کو یکِ بعد دیگرِ تخمین سے حل کرسکتے ہیں۔ فرض کریں ذرہ زیریں حال
\begin{align}
	c_a(0)=1,&&c_b(0)=0
\end{align}
سے آغاز کرتا ہے۔ عند اضطراب کی صورت میں ذرہ ہمیشہ کے لیئے یہیں رہے گا۔

\موٹا{رتبہ صفر:}
\begin{align}
	c^{(0)}_a(t)=1,&&c_b^{(0)}(t)=0
\end{align}
میں تخمین کے رتبہ کو زیر، بالا میں کوسین میں لکھتا ہوں۔

ہم \حوالہء{مساوات \num{9.13}} کے دائیں ہاتھ رتبہ صفر کی قیمتیں پر کر کے رتبہ اوّل تخمین حاصل کرتے ہیں۔

\موٹا{رتبہ اوّل:}
\begin{align}
	\begin{matrix}
		\frac{\dif c_a^{(1)}}{\dif t}=0\Rightarrow c_a^{(1)}(t)=1; \\
		\frac{\dif c_b^{(1)}}{\dif t}=-\frac{i}{\hslash}H'_{ba}e^{i\omega_0t}\Rightarrow c_b^{(1)}=-\frac{i}{\hslash}\int_{0}^{t}H'_{ba}(t')e^{i\omega_0t'}\dif t'
	\end{matrix} 
\end{align}
اب ہم انہیں دائیں ہاتھ پُر کر کے رتبہ دوم تخمین حاصل کرتے ہیں۔

\موٹا{رتبہ دوم:}	
\begin{align}
	\begin{matrix}
		\frac{\dif c_a^{(2)}}{\dif t}=-\frac{i}{\hslash}H'_{ab}e^{-i\omega_0t}\left(-\frac{i}{\hslash}\right)\int_{0}^{t}H'{ba}(t')e^{i\omega_0t'}\dif t'\Rightarrow \\
		c_a^{(2)}(t)=1-\frac{1}{\hslash^2}\int_{0}^{t}H'_{ab}(t')e^{-i\omega_0t'}\left[\int_{0}^{t'}H'_{ba}(t'')e^{i\omega_0t''}\dif t''\right]\dif t'
	\end{matrix}
\end{align}
جہاں \عددی{c_b} تبدیل نہیں ہوا \عددی{(c^{(2)}_b(t)=c^{(1)}_b(t))}۔ دیہان رہے کہ \عددی{c^{(2)}_a(t)} میں صفر رتبی جز بھی پایا جات ہے دو رتبی تصحیح صرف تکملی حصہ ہوگا۔

اصولاً ہم اسی طرح چلتے ہوئے \عددی{n}ویں رتبی تخمین کو \حوالہء{مساوات \num{9.13}} کے دائیں ہاتھ میں پُر کر کے \عددی{n+1}ویں رتبہ کے لیئے حل کر سکتے ہیں۔ رتبہ صفر میں \عددی{H'} کا کوئی جز ضربی نہیں پایا جاتا ہے۔ رتبہ اوّل تصحیح میں \عددی{H'} کا ایک جز ضربی پایا جاتا ہے دو رتبی تصحیح میں \عددی{H'} کے دو جز ضربی پائے جاتے ہیں وغیرہ وغیرہ۔ رتبہ تخمین میں خلل \عددی{\abs{c^{(1)}_a(t)}^2+\abs{c^{(1)}_b(t)}^2\neq1} سے صاف ظاہر ہے بلکل درست عددی سروں کو یقیقناً \حوالہء{مساوات \num{9.5}} پر پورا اترنا ہوگا۔ ہاں \عددی{H'} کی طاقت \عددی{1} تک \عددی{\abs{c^{(1)}_a(t)}^2+\abs{c^{(1)}_b(t)}^2} ایک کے برابر ہے اور رتبہ اوّل تخمین سے صرف اتنی ہی توقع کی جاسکتی ہے زیادہ بلند رتبی تخمین کے لیئے بھی ایسا ہوگا۔

\ابتدا{سوال}
فرض کریں آپ \عددی{H'_{aa}=H'_{bb}=0} نہیں لیتے ہیں۔

(الف) اس صورت میں جب \عددی{c_a(0)=1, c_b(0)=0} ہو رتبہ اوّل نظریہ اضطراب سے \عددی{c_a(t)} اور \عددی{c_b(t)} حاصل کریں۔ دیکھائیں کہ \عددی{H'} کی طاقت ایک تک \عددی{\abs{c^{(1)}_a(t)}^2+\abs{c^{(1)}_b(t)}^2=1}۔

(ب) اس مسئلہ کو بہتر انداز سے نمٹا جاسکتا ہے درج ذیل لیکر
\begin{align}
	\dif_a\equiv e^{\frac{i}{\hslash}\int_{0}^{t}H'_{aa}(t')\dif t'}c_a,&&\dif_b\equiv e^{\frac{i}{\hslash}\int_{0}^{t}H'_{bb}(t')\dif t'}c_b
\end{align}
دیکھائیں کہ درج ذیل ہوگا 
\begin{align}
	\dot{\dif}_a=-\frac{i}{\hslash}e^{i\phi}H'_{ab}e^{-i\omega_0t}\dif_b;&&\dot{\dif}_b=-\frac{i}{\hslash}e^{-i\phi}H'_{ba}e^{i\omega_0t}\dif_a
\end{align}
جہاں درج ذیل ہے
\begin{align}
	\phi(t)\equiv\frac{1}{\hslash}\int_{0}^{t}[H'_{aa}(t')-H'_{bb}(t')]\dif t'
\end{align}
یوں \عددی{H'} کےساتھ اضافی جز ضرب \عددی{e^{i\phi}} منسلک ہونے کے علاوہ \عددی{\dif_a} اور \عددی{\dif_b} کی مساواتیں ساخت کے لحاظ سے \حوالہء{مساوات \num{9.13}} کے متماثل ہیں۔

(ج) رربہ اوّل نظریہ اضطراب سے جز (ب) کی ترکیب استعمال کرتے ہوئے \عددی{c_a(t)} اور \عددی{c_b(t)} حاصل کریں۔ اپنے جواب کا جز (الف) کے ساتھ موازنہ کریں دونوں میں فرق پر تبصرہ کریں۔
\انتہا{سوال}
\ابتدا{سوال}
عمومی صورت \عددی{c_a(0)=a, c_b(0)=b} کے لیئے نظریہ اضطراب سے \حوالہء{مساوات \num{9.13}} کو رتبہ دوم تک حل کریں۔
\انتہا{سوال}
\ابتدا{سوال}
غیر تابع وقت اضطراب \حوالہء{سوال \num{9.2}} کے لیئے \عددی{c_a(t)} اور \عددی{c_b(t)} کو رتبہ دوم تک حاصل کریں۔ اپنے جواب کا بلکل ٹھیک نتیجہ کے ساتھ موازنہ کریں۔
\انتہا{سوال}

