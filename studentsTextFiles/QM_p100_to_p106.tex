\باب{گروپ اکبر}
%%%%%%%%%%%%%%%%%%%%%%%%%%%%%%%%%%%%%%%%%%%%%%%%%%%%%%%%%%%%%%%%%%%
%		Page 100-106
%		Contributors:
%			Mirza Akbar Ali
%			Ayesha Jamshaid
%			Fahad Khan
%
%%%%%%%%%%%%%%%%%%%%%%%%%%%%%%%%%%%%%%%%%%%%%%%%%%%%%%%%%%%%%%%%%%%
% 			Section 3.3 Eigenfunctions of 		    
%           Hermitian Operators
%%%%%%%%%%%%%%%%%%%%%%%%%%%%%%%%%%%%%%%%%%%%%%%%%%%%%%%%%%%%%%%%%%%
\حصہ{ہرمیشی عامل کے امتیازی تفاعل}
یوں ہماری توجہ ہرمیشی عملین کے امتیازی تفاعل کی طرف ہوتی ہے. طبی طور پر قابل مشاہدہ کے قابل معلوم حالات. ان کے دو اقسام ہیں. اگر غیر مسلسل ہو, یعنی امتیازی اقدار الگ الگ ہوں تب امتیازی تفاعل حلبرٹ فضاء میں پائے جائیں گے. اور یہ طبی طور پر قابل حصول حالات دیتے ہیں. اگر طیف استمراری ہو, یعنی امتیازی اقدار پوری ساتھ کو بھرتے ہوں تب امتیازی تفاعل معمول پر لانے کے قابل نہیں ہوں گے. اور یہ کسی بھی ممکنہ تفاعل موج کو ظاہر نہیں کر سکتے. اگرچہ انکی خطی جوڑ, جن میں امتیازی اقدار کی ایک سات موجود ہو, معمول پر لانے کے قابل ہو سکتے ہیں. چند عملین کا صرف غیر مسلسل طیف ہو گا. مثال کے طور پر ہارمونی مرتعش کی ہیملٹونی چند کا صرف استمراری انحراف ہو گا. مثال کے طور پر آزاد ذرہ کی ہیملٹونی. اور بعض کا کچھ حصہ غیر مسلسل اور کچھ حصہ استمراری ہو گا. مثال کے طور پر چکور کنواں کا ہیملٹونی. ان میں سے غیر مسلسل صورت نبھانا زیادہ آسان ہے چونکہ ان کی متعلقہ اندرونی ضرب لازماً موجود ہوں گی. در حقیقت یہ متناہی آبادی نظریہ سے بہت مشابہت رکھتا ہے. ہرمیشی قالب کے امتیازی سمتیات. میں پہلے غیر مسلسل صورت کو دیکھوں گا اور بعد میں استمراری کو. 
\جزوحصہ{غیر مسلسل طیف}
ریاضیاتی طور پر ہرمیشی عامل کےمعمول پر لانے کے قابل امتیازی تفاعل کی دو اہم خصوصیات ہیں. 
مسئلہ 1
ان کے امتیازی اقدار حقیقی ہوتے ہیں. 
ثبوت 
فرض کریں
\begin{align*}
\hat{Q}f = qf 
\end{align*}
یعنی 
\عددی{  f(x)} , \عددی{ \hat{Q}  } کا امتیازی تفاعل ہے اور اس کی امتیازی قدر \عددی{  q } ہے. اور 
\begin{align*}
\langle f | \hat{Q}  f \rangle = \langle \hat{Q} f | f  \rangle
\end{align*}
\عددی{  \hat{Q}} ہرمیشی ہے, تب
\begin{align*}
q\langle f \left\vert f \right.  \rangle = q^{*} \langle f \left\vert f \right. \rangle
\end{align*}
چونکہ \عددی{ q  } ایک عدد ہے لہذا یہ تکمل سے باہر لے جایا جا سکتا ہے. اور چونکہ اندرونی ضرب میں پہلا تفاعل مخلوط جوڑی دار ہے ( مساوات \حوالہ{3.6} ) لہذا دائیں طرف \عددی{ q  }  بھی ہو گا. لیکن \عددی{   \langle f | f \rangle }  صفر نہیں ہو سکتا. قوانین کے تحت \عددی{   f(x)=0 } امتیازی تفاعل نہیں ہو سکتا. لہذا \عددی{  q=q^{*} } حقیقی ہو گا. یہ باعث اطمینان ہے. قابل معلوم حال میں ایک ذرہ کی ایک قابل مشاہدہ کی پیمائش ایک حقیقی عدد دے گی. 
مسئلہ 2
انفرادی امتیازی اقدار کے متعلقہ امتیازی طفاعل عمودی ہوں گے. 
ثبوت 
فرض کریں 
\begin{align*}
\hat{Q}f = qf,  \quad \text{اور } \hat{Q}g=q^{'}g, 
\end{align*}
جہاں \عددی{  \hat{Q}} ہرمیشی ہے. تب \عددی{   \langle f | \hat{Q}g \rangle = \langle \hat{Q}f | g \rangle , } لہذا
\begin{align*}
q^{'} \langle f | g \rangle = q^{*} \langle f | g \rangle
\end{align*}
یہاں بھی چونکہ ہم نے فرض کیا ہے کہ امتیازی طفاعل ہلبرٹ فضاء میں پائے جاتے ہیں لہذا انکا اندرونی ضرب موجود ہو گا. اب مسئلہ 1 کے تحت \عددی{   q} حقیقی ہے. لہذا اگر \عددی{   q{'} \neq q} تب \عددی{\langle f|g \rangle = 0 } ہو گا. 
یہی وجہ ہے کہ لامتناہی چکور کنواں یا مثال کے طور پر ہارمونی مرتعش کے امتیازی حال عمودی ہیں. یہ ایسے ہہیملٹونی کے امتیازی طفاعل ہیں جن کے امتیازی اقدار منفرد ہیں. لیکن یہ خاصیت صرف انہی کیلئے نہیں اور نہ ہی ہیملٹونی کیلئے مخصوص ہے. بلکہ یہ کسی بھی قابل مشاہدہ کے قابل معلوم حالات کیلئے درست ہو گا. بدقسمتی سے مسئلہ 2 ہمیں انحطاطی حالات \عددی{  q^{'}=q} کے بارے میں کوئی معلومات فراہم نہیں کرتا. اگر دو یا دو سے زیادہ امتیازی حالات ایک امتیازی قدر رکھتے ہوں تب انکا ہر خطی جوڑ بھی امتیازی حال ہو گا جسکا امتیازی قدر یہی ہو گا. (سوال 3.7a   )  اور ہم گرام شمد عمودیت (سوال A4) کی ترکیب استعمال کرتے ہوئے ہر ایک انحطاطی ذیلی وقفہ میں عمودی امتیازی تفاعل تیار کر سکتے ہیں. اصولی طور پر ہر صورت ایسا کرنا ممکن ہو گا لیکن شکر اللہ کا کہ ہمیں تقریباً کبھی بھی ایسا کرنے کی ضرورت پیر نہیں آئے گی. یوں انحطاط کی موجودگی میں بھی ہم عمودی امتیازی طفاعل منتخب کر سکتے ہیں. اور کوانٹم میکانیات کے ضوابط طہ کرتے ہوئے ہم فرض کرتے ہیں کہ ہم ایسا کر چکے ہیں. اس طرح ہم فوریئر کی ترکیب استعمال کر سکتے ہیں. جو اساس طفاعل کی معیاری عمودیت پر مبنی ہے. ایک متناہی آبادی سمتی فضاء میں ہرمیشی قالب کے امتیازی اقدار ایک تیسری بنیادی خاصیت رکھتے ہیں. یہ فضاء کو احاطہ کرتے ہیں. یعنی ہر سمتیہ کو ان کا خطی جوڑ لکھا جا سکتا ہے. بد قسمتی سے یہ ثبوت لا متناہی آبادی فضاء کیلئے نہیں ہے. لیکن یہ خاصیت کوانٹم میکانیات کی اندرونی ہم آہنگی کیلئے لازم ہے. لہذا ڈیراک کی طرح ہم اسے ایک مسلمہ لیتے ہیں. بلکہ یہ کہنا زیادہ درست ہو گا کہ یہ ان ہرمیشی عملین پر ایک شرط مسلط ہو گی جو قابل مشاہدہ کو ظاہر کر سکتے ہیں. مسلمہ قابل مشاہدہ کے امتیازی تفاعل مکمل ہوں گے. ہلبرٹ فضاء میں ہر تفاعل کو انکا خطی جوڑ لکھا جا سکتا ہے.
% Problem 3.7
\ابتدا{سوال}
\begin{enumerate}[a.]
\item  فرض کریں کہ حامل \عددی{ \hat{Q}} کے دو امتیازی تفاعل \عددی{ f(x) } اور  \عددی{ g(x) } ہیں اور ان دونوں کا امتیازی قدر \عددی{ q } ہے. دکھائیں کہ \عددی{ f } اور \عددی{ g } کا ہر خطی جوڑ ازخود \عددی{ \hat{Q} } کا امتیازی تفاعل ہے جس کی امتیازی قدر \عددی{q  } ہے
\item تصدیق کریں کہ \عددی{f(x) = exp(x)} اور \عددی{ g(x) = exp(-x) } - حامل \عددی{ d^{2}/dx^{2} } کے امتیازی تفاعل ہیں اور ان کے امتیازی اقدار ایک دوسرے جیسے ہیں. \عددی{  f} اور \عددی{ g} کے ایسے دو خطی جوڑ تیار کریں جو وقفہ \عددی{ (-1,1) } پر عمودی امتیازی تفاعل ہوں 
\end{enumerate}
\انتہا{سوال}
\ابتدا{سوال}
سوال 3.8
\begin{enumerate}[a.]
\item تصدیق کریں کہ مثال \حوالہ{ 3.1} میں ہرمیشی حامل کے امتیازی اقدار حقیقی ہیں. دکھائیں کہ منفرد امتیازی اقدار کے امتیازی تفاعل عمودی ہیں 
\item یہی کچھ سوال \حوالہ{ 3.6} کے حامل کے لیے کریں 
\end{enumerate}
\انتہا{سوال}
\جزوحصہ{استمراری طیف}
ہرمیشی حامل کے طیف کا استمراری ہونے کی صورت میں عین ممکن ہے کہ ان کا اندرونی ضرب موجود نہ ہو لہٰذا مسئلہ 1 اور 2  کے ثبوت درست نہیں ہوں گے لہٰذا امتیازی تفاعل معمول پر لانے کے قابل نہیں ہوں گے. اس کے باوجود ایک لحاظ سے یہ تین لازم خصوصیات یعنی حقیقی ہونا، عموددیت اور مکمل ہونا اب بھی کارآمد ہے. اس صورت کو ایک مخصوص مثال کی مدد سے سمجھنا بہتر ہو گا



%Example 3.2

معیار حرکت حامل کے امتیازی تفاعل اور امتیازی اقدار تلاش کریں
: حل
فرض کریں کہ \عددی{p } امتیازی قدر اور \عددی{f_{p}(x) } امتیازی تفاعل ہو
\begin{align}
\frac{\hbar}{i} \frac{d}{dx} f_{p}(x) = pf_{p}(x)
\end{align}
اس کا عمومی حل درج ذیل ہو گا
\begin{align*}
f_{p}(x) = Ae^{ipx/\hbar}
\end{align*}
چونکہ \عددی{ p} کی کسی بھی مخلوط قیمت کے لیے یہ قابل تکمل مربع نہیں ہے لہذا معیار حرکت حامل کا ہلبرٹ فضا میں کوئی امتیازی تفاعل نہیں ہو گا. اس کے باوجود اگر ہم حقیقی امتیازی اقدار تک اپنے آپ کو محدود رکھیں تو ہمیں متبادل معیاری عموددیت حاصل ہوتی ہے. یہاں سوال  \حوالہ{ 2.24} (الف) اور  \حوالہ{ 2.26} دیکھیں
\begin{align}
\int_{-\infty}^{\infty} f_{p^{'}}^{*}(x)f_{p}(x) dx = |A|^{2}\int_{-\infty}^{\infty}e^{i(p-p^{'})x/\hbar} dx = |A|^{2}2\pi\hbar\delta(p-p^{'})
\end{align}
اگر ہم \عددی{ A=1/\sqrt{2\pi\hbar}} لیں تب
\begin{align}
f_{p}(x) = \frac{1}{\sqrt{2\pi\hbar}}e^{ipx/\hbar}
\end{align}ہو گا. لہذا
\begin{align}
\langle f_{p}^{'} | f_{p} \rangle = \delta(p-p^{'})
\end{align}
جو حقیقی معیاری عموددیت مساوات \حوالہ{ 3.10} کی یاد دلاتا ہے. یہاں اشاریہ استمراری تغیرات ہیں اور کرونیکر ڈیلٹا کی جگہ ڈیراک ڈیلٹا پایا جاتا ہے لیکن ان کے علاوہ اس کی صورت وہی ہے. میں مساوات \حوالہ{ 3.33} کو ڈیراک معیاری عموددیت کہوں گا. سب سے اہم بات یہ ہے کہ یہ امتیازی تفاعل مکمل ہیں اور ان کے مجموعہ ( مساوات \حوالہ{ 3.11}) کی جگہ اب تکمل استعمال ہوتا ہے کسی بھی قابل تکمل مربع تفاعل \عددی{ } کو درج ذیل روپ میں لکھا جا سکتا ہے
\begin{align}
f(x) = \int_{-\infty}^{\infty} c(p)f_{p}(x)dp = \frac{1}{\sqrt{2\pi\hbar}}\int_{-\infty}^{\infty}c(p)e^{ipx/\hbar}dp. 
\end{align}
پھیلاؤ کا عددی سر جو اب ایک تفاعل ہے \عددی{ c(p)} کو فوررئیر کی ترکیب سے حاصل کیا جا سکتا ہے
\begin{align}
\langle f_{p^{'}} | f \rangle = \int_{-\infty}^{\infty} c(p) \langle f_{p^{'}} | f \rangle dp = \int_{\infty}^{\infty}c(p)\delta (p-p^{'}) dp = c(p^{'}).
\end{align}
چونکہ مساوات \حوالہ{ 3.34} درحقیقت فوررئیر تبادل ہے لہذا آپ انہیں مسئلہ پلانشرال ( مساوات \حوالہ{ 2.102}) بھی حاصل کر سکتے ہیں.
معیار حرکت کے امتیازی تفاعل ( مساوات \حوالہ{ 3.32}) \عددی{ \sin} نما ہیں جن کی طولی موج درج ذیل ہے
\begin{align}
\lambda = \frac{2\pi\hbar}{p} 
\end{align}
یہ وہ ڈی براگلی کلیہ (مساوات \حوالہ{ 1.39} ) ہے جس کا ثبوت موزوں وقت پر پیش کرنے کا وعدہ میں نے کیا تھا. یہ ڈی براگلی کے خیالات سے زیادہ پراسرار ہے چونکہ حقیقت میں ایسا کوئی ذرہ نہیں پایا جاتا جس کا معیار حرکت قابل معلوم ہو. ہاں ہم تنگ ساتھ کی معیار حرکت کا ایسے موجی اکٹھ بنا سکتے ہیں جو معمول پر لانے کے قابل ہو اور ڈی براگلی کا تعلق اسی پر لاگو ہو گا
ہم مثال \حوالہ{ 3.2} سے کیا مطلب لیں؟ اگرچہ \عددی{\hat{p} } کا کوئی بھی امتیازی تفاعل ہلبرٹ فضا میں نہیں رہتا لیکن ان کا ایک مخصوص کنبہ جن کی امتیازی اقدار حقیقی ہوں قریبی مضافات رہتے ہیں اور یہ بظاہر معمول پر لانے کے قابل ہیں. یہ طبعی طور پر ممکنہ حالات کو ظاہر نہیں کرتے لیکن اس کے باوجود کارآمد ثابت ہوتے ہیں (جیسا یک بودی بکھراؤ پر غور کے دوران ہم نے دیکھا)
مثال 3.3
حامل مقام کے امتیازی اقدار اور امتیازی تفاعل تلاش کریں
:حل
فرض کریں کہ \عددی{y } امتیازی قدر اور \عددی{ g_{y}(x)} امتیازی تفاعل ہے
\begin{align}
xg_{y}(x) = yg_{y}(x) 
\end{align}
یہاں کسی بھی ایک امتیازی تفاعل کے لیے \عددی{ y} ایک مقررہ عدد ہے جبکہ \عددی{x } استمراری تغیر ہے.\عددی{ x} کا کون سا ایسا تفاعل ہے کہ جس کی یہ خاصیت ہے کہ اس کو \عددی{ x} سے ضرب دینا اس کے مترادف ہے کہ اس کو \عددی{ y} سے ضرب دیا گیا ہو؟ ظاہر ہے کہ ماسوائےنکتہ \عددی{ x=y} کے یہ صفر ہی ہو گا درحقیقت یہ ڈیراک ڈیلٹا تفاعل ہے
\begin{align*}
g_{y}(x) = A\delta(x-y)
\end{align*}
اس بار؟؟؟ اقدار کو لازماً حقیقی ہونا ہو گا : امتیازی تفاعل قابل تکمل مربع نہیں ہیں لیکن اب بھی یہ ڈیراک معیاری عموددیت پر پورا اترتے ہیں
\begin{align}
\int_{-\infty}^{\infty}g_{y^{'}}^{*}g_{y}(x) dx = |A|^{2}\int_{-\infty}^{\infty}\delta(x-y^{'})\delta(x-y) dx = |A|^{2} \delta (y-y^{'})
\end{align}

\عددی{  }

اگر ہم \عددی{A=1 } لیں تا کہ
\begin{align}
g_{y}(x) = \delta (x-y)
\end{align}
درج ذیل ہو گا
\begin{align}
\langle g_{y^{'}} | g_{y} \rangle = \delta (y-y^{'})
\end{align}
یہ آئینی تفاعل بھی مکمل ہیں
\begin{align}
f(x) = \int_{-\infty}^{\infty} c(y)g_{y}(x) dy = \int_{-\infty}^{\infty} c(y)\delta(x-y)dy,
\end{align}
جہاں درج ذیل ہو گا
\begin{align}
c(y) = f(y)
\end{align}
جس کا حصول اس مثال میں بہت سادہ تھا آپ اس کو فوررئیر کی ترکیب سے بھی حاصل کر سکتے ہیں
اگر ایک ہرمیشی حامل کا طیف استمراری ہو تاکہ اس کے امتیازی اقدار کو استمراری متغیر کو  \عددی{ } یا \عددی{ } سے نام دیا جائے جیسا کہ اس مثال میں کیا گیا تب اس کے امتیازی تفاعل معمول پر لانے کے قابل نہیں ہوں گے، یہ ہلبرٹ فضا میں نہیں پائے جائیں گے اور یہ ممکنہ کسی بھی طبعی حالت کو ظاہر نہیں کرتے ہیں ہاں جن امتیازی تفاعل کے امتیازی اقدار حقیقی ہوں وہ ڈیراک معیاری عموددیت پر پورا اترتے ہیں اور مکمل ہیں جہاں مجموعہ کی جگہ اب تکمل استعمال ہو گا. خوش قسمتی سے ہمیں صرف اتنا ہی چاہیے تھا
% Question 3.9
\ابتدا{سوال}
\begin{enumerate}[a.]
\item  ہارمونی مرتعش کے علاوہ باب \حوالہ{ 2} سے ایک ایسا ہیملٹنی بتائیں جس کا طیف صرف غیر مسلسلہ ہو 
\item آذاد ذرہ کے علاوہ باب \حوالہ{ 2}  سے ایسے ہیملٹنی بتائیں جس کا طیف صرف استمراری ہے 
\item متناہی چکور کنواں کے علاوہ باب \حوالہ{ 2}  سے ایسا ہیملٹنی بتائیں جس کے طیف کا کچھ حصہ غیر مسلسل اور کچھ استمراری ہے 
\end{enumerate}
\انتہا{سوال}
%Question 3.10
\ابتدا{سوال}
کیا لا متناہی چکور کنواں کا زمینی حال معیار حرکت کا  امتیازی تفاعل ہے؟ اگر ایسا ہے تو اس کا معیار حرکت کیا ہو گا اور اگر ایسا نہیں ہے تو کیوں نہیں ہے؟
\انتہا{سوال}
