
\ابتدا{سوال}
ایک ذرہ جس کی کمیت \عددی{m} اور توانائی \عددی{E} ہو درج ذیل مخفیہ پر بائیں سے آمدی ہے
\begin{align*}
	V(x)=
	\begin{cases}
		0, & (x<-a). \\
		-V_0, & (-a\leq z\leq0). \\
		\infty, & (x>0).
	\end{cases}
\end{align*}
(الف) آمدی موج \عددی{Ae^{ikx}} جہاں \عددی{k=\sqrt{2mE}/\hslash} کی صورت میں منعکس موج تلاش کریں۔

جواب:
\begin{align*}
	Ae^{-2ika}\left[\frac{k-ik'\cot(k'a)}{k+ik'\cot(k'a)}\right]e^{-ikx}, && \text{\RL{جہاں}} k'=\sqrt{2m(E+V_0)}/\hslash 
\end{align*}
(ب) تصدیق کریں کہ منعکس موج کا حیطہ وہی ہے جو آمدی موج کا ہے۔

(ج) بہت گہرا کنواں \عددی{E\ll V_0} کے لیئے ینتقلات حیط \عددی{\delta} \حوالہء{مساوات \num{11.40}} تلاش کریں۔

جواب: \عددی{\delta=-ka}
\انتہا{سوال}
\ابتدا{سوال}
سخت کرہ بکھراؤ کے لیئے جزوی موج حیطی انتقال \عددی{\delta_l} کیا ہوں گے \حوالہء{مثال \num{11.3}}؟
\انتہا{سوال}
\ابتدا{سوال}
ایک ڈیلٹا تفاعل خول \حوالہء{سوال \num{11.4}} سے \عددی{S} موج \عددی{l=0} جزوی موج انتقال حیط \عددی{\delta_0(k)} تلاش کریں۔ ایسا کرتے ہوئے فرض کریں کہ \عددی{r\to\infty} پر رداسی تفاعل موج \عددی{u(r)} صفر کو پہنچے گا۔

جواب:
\begin{align*}
	-\cot^{-1}\left[\cot(ka)+\frac{ka}{\beta\sin^2(ka)}\right], &&\text{\RL{جہاں}} \beta\equiv\frac{2m\alpha a}{\hslash^2}
\end{align*}
\انتہا{سوال}
\حصہ{بارن تخمین}
\جزوحصہ{مساوات شروڈنگر کی تکملی روپ}
غیر تابع وقت شروڈنگر مساوات
\begin{align}
	-\frac{\hslash^2}{2m}\nabla\psi+V\psi=E\psi
\end{align}
کو مختصراً
\begin{align}
	(\nabla^2+k^2)\psi=Q
\end{align}
لکھا جا سکتا ہے جہاں درج ذیل ہوں گے
\begin{align}
	k\equiv\frac{\sqrt{2mE}}{\hslash} & \text{\RL{اور}} Q\equiv\frac{2m}{\hslash^2}V\psi
\end{align}
اس کا روپ سرسری طور پر مساوات ہلم ہولٹز کی طرح ہے۔ البتہ غیر متجانس جز \عددی{Q} از خود \عددی{\psi} کا تابع ہے۔

فرض کریں ہم ایک تفاعل \عددی{G(r)} دریافت کر پائیں جو ڈیلٹا تفاعلی منبع کے لیئے مساوات ہلم ہولٹز کو متمعن کرتا ہو
\begin{align}
	(\nabla^2+k^2)G(r)=\delta^3(r)
\end{align}
ایسی صورت میں ہم \عددی{\psi} کو بطور ایک تکمل لکھ سکتے ہیں
\begin{align}
	\psi(r)=\int G(r-r_0)Q(r_0)\dif^3r_0
\end{align}
ہم با آسانی دیکھا سکتے ہیں کہ یہ \حوالہء{مساوات \num{11.50}} روپ کی شروڈنگر مساوات کو متمعن کرتا ہے
\begin{align*}
	(\nabla^2+k^2)\psi(r) &= \int\left[(\nabla^2+k^2)G(r-r_0)\right]Q(r_0)\dif^3r_0 \\
	&= \int\delta^3(r-r_0)Q(r_0)\dif^3r_0 = Q(r)
\end{align*}
تفاعل \عددی{G(r)} کو مساوات ہلم ہولٹز کا تفاعل گرین کہتے ہیں۔ عمومی طور پر ایک خطی تفرقی مساوات کا تفاعل گرین ایک ڈیلٹا تفاعلی منبع کو ردِ عمل ظاہر کرتا ہے۔

ہمارا پہلا کام \عددی{G(r)} کے لیئے \حوالہء{مساوات \num{11.52}} کا حل تلاش کرنا ہے۔ ایسا کرنے کا آسان ترین طریقہ یہ ہے کہ ہم فوریر بدل لیں جو تفرقی مساوات کو ایک الجبرائی مساوات میں تبدیل کرتا ہے۔ درج ذیل لیں
\begin{align}
	G(r)=\frac{1}{(2\pi)^{3/2}}\int e^{is\cdot r}g(s)\dif^3s
\end{align}
تب 
\begin{align*}
	(\nabla^2+k^2)G(r) = \frac{1}{(2\pi)^{3/2}}\int\left[(\nabla^2+k^2)e^{is\cdot r}\right]g(s)\dif^3s
\end{align*}
ہوگا تاہم
\begin{align}
	\nabla^2e^{is\cdot r} = -s^2 e^{is\cdot r}
\end{align}
اور \حوالہء{مساوات \num{2.144}} دیکھیں
\begin{align}
	\delta^3(r)=\frac{1}{(2\pi)^3}\int e^{is\cdot r}\dif^3s
\end{align}
لحاظہ \حوالہء{مساوات \num{11.52}} درج ذیل کہے گی
\begin{align*}
	\frac{1}{(2\pi)^{3/2}}\int(-s^2+k^2)e^{is\cdot r}g(s)\dif^3s = \frac{1}{(2\pi)^3}\int e^{is\cdot r}\dif^3s
\end{align*}
یوں درج ذیل ہوگا 
\begin{align}
	g(s) = \frac{1}{(2\pi)^{3/2}(k^2-s^2)}
\end{align}
اس کو واپس \حوالہء{مساوات \num{11.54}} میںپُر کع کے درج ذیل ملتا ہے
\begin{align}
	G(r) = \frac{1}{(2\pi)^3}\int e^{is\cdot r}\frac{1}{(k^2-s^2)}\dif^3s
\end{align}
اب \عددی{s} تکمل کے نقطع نظر سے \عددی{r} غیر متغیر ہے ہم کروی محدد \عددی{(s, \theta, \phi)} کو یوں چنتے ہیں کہ \عددی{r} کتبی محور پر پایا جاتا ہو \حوالہء{شکل \num{11.8}}۔ یوں \عددی{s\cdot r = sr\cos\theta} ہوگا متغیر \عددی{\phi} کا تکمل \عددی{2\pi} ہوگا جبکہ \عددی{\theta} تکمل درج ذیل ہوگا
 
\begin{align}
	\int_{0}^{\pi}e^{isr\cos\theta}\sin\theta\dif\theta = -\frac{e^{isr\cos\theta}}{isr}\bigg|^\pi_{0} = \frac{2\sin(sr)}{sr}
\end{align}
یوں درج ذیل ہوگا
\begin{align}
	G(r) = \frac{1}{(2\pi^2)}\frac{2}{r}\int_{0}^{\infty}\frac{s \sin(sr)}{k^2-s^2}\dif s = \frac{1}{4\pi^2r}\int_{-\infty}^{\infty}\frac{s \sin(sr)}{k^2-s^2}\dif s
\end{align}
باقی تکمل اتنا آسان نہیں ہے۔ قوت نمائی علامتیت استعمال کرکے نصب نما کو اجزائے ضربی کی روپ میں لکھنا مددگا ثابت ہوتا ہے
\begin{align}
	G(r) &= \frac{i}{8\pi^2r}\left\{\int_{-\infty}^{\infty}\frac{se^{isr}}{(s-k)(s+k)}\dif s-\int_{-\infty}^{\infty}\frac{se^{-isr}}{(s-k)(s+k)}\dif s\right\}\nonumber \\
	&= \frac{i}{8\pi^2r}(I_1-I_2)
\end{align}
اگر \عددی{z_0} خطِ ارتفاہ کے اندر پایا جاتا ہو تب کوشی کلیہ تکمل 
\begin{align}
	\oint\frac{f(z)}{(z-z_0)}\dif z = 2\pi if(z_0)
\end{align}
استعملا کرتے ہوئے ان تکملات کی قیمت تلاش کی جا سکتی ہے دیگر صورت تکمل صفر ہوگا۔ یہاں حقیقی محور جو \عددی{\pm k} پر قطبی نادر نکات کے بلکل اوپر سے گزرتا ہے کے کے ساتھ ساتھ تکمل لیا جا رہا ہے۔ ہمیں قطبین کے اطراف سے گزرنا ہوگا میں \عددی{-k} پر بلائی جانب سے \عددی{+k} پر زیریں جانب سے گزروں گا \حوالہء{شکل \num{11.9}}۔ آپ کوئی نیا راستہ منتخب کر سکتے ہیں مثلاً آپ ہر قطب کے گرد سات مرتبہ چکر کاٹ کر راہ منتخب کر سکتے ہیں جس سے آپ کو ایک مختلف تفاعل گرین حاصل ہوگا لیکن میں کچھ ہی دیر میں دیکھاؤں گا کہ یہ تمام قابلِ قبول ہوں گے۔

\حوالہء{مساوات \num{11.61}} میں ہر ایک تکمل کے لیئے ہمیں خط استوا کو اس طرح بند کرنا ہوگا  کہ لامتناہی پر نصف دائرہ تکمل کی قیمت میں کوئی حصہ نہ ڈالے۔ تکمل \عددی{I_1} کی صورت میں اگر \عددی{s} کا خیالی جز بہت بڑا اور مثبت ہو تب جز ضربی \عددی{e^{isr}} صفر کو پہنچے گا اس تکمل کے لیئے ہم بالا نصف دائرہ لیتے ہیں \حوالہء{شکل \num{11.10} (الف)}۔ اب خط ارتفا صرف \عددی{s=+k} پر پائے جانے والا نادر نقطع کو گھیرتا ہے لحاظہ درج ذیل ہوگا
\begin{align}
	I_1 = \oint\left[\frac{se^{isr}}{s+k}\right]\frac{1}{s-k}\dif s = 2\pi i\left[\frac{se^{isr}}{s+k}\right]\bigg|_{s=k} = i\pi e^{ikr}
\end{align}
تکمل \عددی{I_2} کی صورت میں جب \عددی{s} کا خیالی جز بہت بڑی منفی مقدار ہو تب جز ضربی \عددی{e^{-isr}} صفر کو پہنچتا ہے لحاظہ ہم زیریں نصف دائراہ لیتے ہیں \حوالہء{شکل \num{11.10} (ب)}۔ اس مرتبہ خطِ ارتفا \عددی{s=-k} پر پائے جانے والے نادر نقطہ جو کو گھیرتا ہے اور یہ گھڑی وار ہے لحاظہ اس کے ساتھ اضافی منفی علامت ہوگا
\begin{align}
	I_2 = -\oint\left[\frac{se^{-isr}}{s-k}\right]\frac{1}{s+k}\dif s = -2\pi i\left[\frac{se^{-isr}}{s-k}\right]\bigg|_{s=-k} = -i\pi e^{ikr}
\end{align}
ماخوذ:
\begin{align}
	G(r) = \frac{i}{8\pi^2r}\left[\left(i\pi e^{ikr}\right)-\left(-i\pi e^{ikr}\right)\right] = -\frac{e^{ikr}}{4\pi r}
\end{align}
یہ \حوالہء{مساوات \num{11.52}} کا حل اور مساوات ہلم ہولٹز کا تفاعل گرین ہے اگر آپ کہیں ریاضیاتی تجزیہ میں گم ہوگئے ہوں تب بلاواسطہ تفرق کی مدد سے نتیجہ کی تصدیق کی جیئے گا \حوالہء{سوال \num{11.8}} دیکھیں۔ بلکہ یہ مساوات ہلم ہولٹز کا ایک تفاعل گرین ہے چونکہ ہم \عددی{G(r)} کے ساتھ ایسا کوئی بھی تفاعل \عددی{G_0(r)} جمع کر سکتے ہیں جو متجانز ہلم ہولٹز مساوات کو متمعن کرتا ہو
\begin{align}
	(\nabla^2+k^2)G_0(r) = 0
\end{align}
صاف ظاہر ہے کہ \حوالہء{مساوات \num{11.52}} کو \عددی{(G+G_0)} بھی متمعن کرتا ہے۔ اس ابہام کی وجہ قطبین کے قریب سے گزرتے ہوئے راہ کی بنا ہے راہ کی ایک مختلف انتخاب ایک مختلف تفاعل \عددی{G_0(r)} کے مترادف ہے۔

\حوالہء{مساوات \num{11.53}} کو دوبارہ دیکھتے ہوئے مساوات شروڈنگر کا عمومی حل درج ذیل روپ کا ہوگا
\begin{align}
	\psi(r) = \psi_0(r)-\frac{m}{2\pi\hslash^2}\int\frac{e^{ik\abs{r-r_0}}}{\abs{r-r_0}}V(r_0)\psi(r_0)\dif^3r_0
\end{align}
جہاں \عددی{\psi_0} آزاد ذرہ مساوات شروڈنگر کو متمعن کرتا ہے
\begin{align}
	(\nabla^2+k^2)\psi_0 = 0
\end{align}
\حوالہء{مساوات \num{11.67}} شروڈنگر مساوات کی تکملی روپ ہے جو زیادہ معروف تفرقی روپ کی مکمل طور پر معدل ہے۔ پہلی نظر میں ایسا معلوم ہوتا ہے کہ یہ کسی بھی مخفیہ کے لیئے مساوات شروڈنگر کا سری حل ہے جو ماننے والی بات نہیں ہے۔ دھوکہ مت کھائیں۔ دائیں ہاتھ تکمل کی علامت کے اندر \عددی{\psi} پایا جات ہے جسے جاننے بغیر آپ تکمل حاصل کر کے حل نہیں جان سکتے ہیں  تاہم تکملی روپ انتہائی طاقتور ثابت ہوتا ہے اور جیسا ہم اگلے حصہ میں دیکھیں گے یہ بلخصوص بکھراؤ مسائل کے لیئے نہایت موضوع ہے۔

\ابتدا{سوال}
\حوالہء{مساوات \num{11.65}} کو \حوالہء{مساوات \num{11.52}} میں پُر کر کے دیکھیں کہ یہ اسے متمعن کرتا ہے۔ اشارہ: \عددی{\nabla^2(1/r) = -4\pi\delta^3(r)}۔
\انتہا{سوال}
\ابتدا{سوال}
دیکھائیں کہ \عددی{V} اور \عددی{E} کی مناسب قیمتوں کے لیئے مساوات شروڈنگر کی تکملی روپ کو ہائڈروجن کا زمینی حال \حوالہء{مساوات \num{4.80}} متمعن کرتا ہے۔ دیہان رہے کہ \عددی{E} منفی ہے لحاظہ \عددی{k=i\kappa} ہوگا جہاں \عددی{\kappa\equiv\sqrt{-2mE}/\hslash} ہوگا۔
\انتہا{سوال}

