\documentclass{book}
\usepackage{fontspec}
\usepackage{makeidx}
\usepackage{amsmath}                                                                         %\tfrac for fractions in text
\usepackage{amssymb}    
\usepackage{gensymb}  
\usepackage{amsthm}      						%theorem environment. started using in the maths book
\usepackage{mathtools}
\usepackage{multicol}
\usepackage{commath}									%differentiation symbols
\usepackage{polyglossia}    
\setmainlanguage[numerals=maghrib]{arabic}     %for english numbers use numerals=maghrib, for arabic numerals=arabicdigits
\setotherlanguages{english}

\newfontfamily\arabicfont[Scale=1.0,Script=Arabic]{Jameel Noori Nastaleeq} 
\setmonofont{DejaVu Sans Mono}                                                                  %had to add this and the next line to get going after ubuntu upgrade
\let\arabicfontt\ttfamily                                                                                  %had to add this and the above line to get going after ubuntu upgrade
\newfontfamily\urduTechTermsfont[Scale=1.0,Script=Arabic]{AA Sameer Sagar Nastaleeq Bold}
\newfontfamily\urdufont[WordSpace=1.0,Script=Arabic]{Jameel Noori Nastaleeq}
\newfontfamily\urdufontBig[Scale=1.25,WordSpace=1.0,Script=Arabic]{Jameel Noori Nastaleeq}
\newfontfamily\urdufontItalic[Scale=1.25,WordSpace=1.0,Script=Arabic]{Jameel Noori Nastaleeq Italic}
\setlength{\parskip}{5mm plus 4mm minus 3mm}
\begin{document}
حصہ معیار حرکت حال
\(\psi\)
میں پائے جانے والے زرے کے مقام 
x
کی توقعاتی قیمت درج ذیل ہوگی
\[<x>=\int_{-\infty}^{+\infty}x|\psi (x,t)|^{2}\dif{x}\]
اس کا حقیقتا مطلب کیا ہے ؟ اس کا یہ پر گز مطلب نہیں ہے کہ اگر آپ ایک ہی ذرے کا مقام جاننے کے لیے بار بار پیمائش کریں تو آپ کو
\(\int x|\psi|^{2}\dif{x}\)
یہ تکمل نتائج کی اوست قیمت دے گا . جس کا جواب تر معین ہوگا۔ تفال موج کو اس قیمت پر بیٹھنے پر مجبور کرے گا جو پیمائش سے حاصل ہوگی اس کے بعد اگر جلد دوسری پیمائش کو جائے توہی نتیجہ حامل ہوگا . حقیقت میں
\(<x>\)
ان تمام پیمائش کی اوست ہوگا جو ایسے تمام ذرات پر کیے جائیں جو تمام کے تمام حال
\(\psi\)
میں پائے جائے ہوں یوں یا تو آپ پر ایک پیمائش کے بعد کسی طریقہ سے ابتدائی حال
\(\psi\)
میں لائیں گے یا آپکو متعدد کو حال
\(\psi\)
میں لا کر تمام کی پیمائش کی جائے گی . بہت سارے ذرات جو ایک ہی حال میں ہوں کو ہم زرات کا سگرا کیئے ہیں . یوں
\(<x>\)
ان تمام نتائج کی اوست قیمت ہوگی  میں اس کی تصوراتی شکل یوں پیش کرتا ہوں کہ ایک الماری میں کر قطار پر شیشے کی بوتلیں ہیں اور ہر بوتل میں ایک زرہ پایا جاتا ہے . تمام زرات ایک جیسے ہی حال
\(\psi\)
میں پائے جاتے ہیں . جو بوتل کے وسط کے لحاظ سے تقال ہوگا . ہر بوتل کے ساتھ ایک شاگرد کو کھڑا کیا جاتا ہے اور ہر شاگرد کے ہاتھ میں ایک فیتا ہوتا ہے جب اشاره دیا جائے تو عام طلبہ اپنے اپنے زره کا مقام ناپتے ہیں . اگر ہم ان نتائج کا ترسیم بنائیں تو یہ
\(|\psi ^{2}|\)
کے برابر ہوگا۔ اور اگر ہم اس کی اوسط جانتا چاہیں تووہ
\(<x>\)
کے برابر ہوگا۔ اگر چہ چونکہ ہم متناہی تعداد کے زرات پر تجربه کریں گے لحا ظہ ہم یہ توقع نہیں کریں گے کہ جوابات بالکل عین یہی حاصل ہوں لیکن جب بوتلوں کی تعداد بڑھائی جائے تو جوابات نظر یاتی جوابات کے برابر ہوں گے . مختصراً توقعاتی قیمت زرات کے سگراپر کیے جانے والے تجربات سے حاصل توقعاتی قیمت کی اوسط قیمت ہوگی نہ کہ کسی ایک زرے پر باربار نتائج سے حاصل اوسط قیمت چونکہ
\(\psi\)
وقتی اور مقام دونوں کی تابع ہے لہذا وقت گزرنے کا ساتھ ساتھ
\(\psi\)
تبدیل ہوگا جس کو بنا پر
\(<x>\)
بھی تبدیل ہوگا . اور ہم جانت چاہیں گے کہ یہ کس رفتار سے حرکت کرے گا۔
مساوات 1.25 اور 1.28 سے درج ذیل لکھی جاسکتا ہے \\
\[\frac{d\textlangle x \textrangle}{\dif{t}}=\int x\frac{\partial}{\partial{t}}|\psi|^{2}\dif{x}=\frac{i\hbar}{2m}\int x\frac{\partial}{\partial{x}}\big (\psi *\frac{\partial{\psi}}{\partial{x}}-\frac{\partial{\psi *}}{\partial{x}}\psi \big)\dif{x}\]

ہم تکمل کی مدد سے اس فقرے کی ساده صورت حاصل کرتے ہیں \\
\[\frac{d\textlangle x \textrangle}{\dif{t}}=\frac{-i\hbar}{2m}\int\big (\psi *\frac{\partial{\psi}}{\partial{x}}-\frac{\partial{\psi *}}{\partial{x}}\psi \big)\dif{x}\]
میں نے یہاں
\(\frac{\partial{x}}{\partial{x}}=1\)
استعمال کیا اور سرحدی جزکو اس بنا رد کی کہ جمع اور منفی لامتناہی پر
\(\psi\)
کی قیمت 
0
ہوگی۔ ایک بار دوبارہ تکمل کی مدد سے جرا حاصل ہو گا۔

\[\frac{d\textlangle x \textrangle}{\dif{t}}=\frac{-i\hbar}{m}\int\big (\psi *\frac{\partial{\psi}}{\partial{x}} \big)\dif{x}\]
اس نتیجے سے ہم کیا مطلب حاصل کر سکتے ہیں . یہ
\(<x>\)
کو توقعاتی قیمت کی رفتار ہے نہ کرنے کی رفتار ہے ابھی تک ہم جو کچھ دیکھ چکے ہیں اس سے ہم زدے کی سمتی رفتار معلوم نہیں کر سکتے۔ کوانٹم مکانیات میں زدے کی سمتی رفتار کا مفہوم واضع نہیں ہے . اگر یک ذرے کا مقام پمائش سے قبل غیر معین ہو تب اس کی معین رفتار بھی ممکن نہیں ہو سکتی ہے۔ ہم یک مخصوص قیمت کے نتیجہ حاصل کرنے کے احتمال کے بارے میں بات کر سکتے ہیں . ہم
\(\psi\)
جانتے ہوئے کثافت احتمال کی بناوٹ باب ۳ میں دیکھیں گے۔ اب کے لیے صرف اتن کافی ہے کہ ہم جان سیکس کے سمتی رفتار کی توقعاتی قیمت زرے کے مقام کی توقعاتی قیمت کا تفرق ہوگا۔
\[\textlangle v \textrangle=\frac{d\textlangle x \textrangle}{\dif{t}}\]
مساوات 1.31 کی مدد سے ہم
\(\psi\)
سے سمتی رفتار حامل کر سکتے ہیں۔ روایتی طور پر معیار حرکت
\(p=mv\)
کے ساتھ ہم کام کرتے ہیں .
\[\textlangle v \textrangle=m\frac{d\textlangle x \textrangle}{\dif{t}}=-i\hbar\int\big (\psi *\frac{\partial{\psi}}{\partial{x}} \big)\dif{x}\]
یہ میں
\(<x>\)
اور
\(<P>\)
کو بہتر طریقے سے لکھوں 
\[\textlangle x \textrangle =\int \psi *(x)\psi \dif{x}\]
\[\textlangle p \textrangle=\int \psi *\big (\frac{\hbar}{i}\frac{\partial}{\partial{x}}\big )\psi \dif{x}\]
کوانٹم مکانیات میں عامل
x
مقام کو ظاہر کرتا ہے اور عامل
\(\frac{\hbar}{i}\frac{d}{\dif{x}}\)
معیار حرکت کو ظاہر کرتا ھے ۔کی  کسی بھی توقعاتی قیمت کی حصول کی خاطر ہم موضوع عامل کو
\(\psi *\)
اور
\(\psi\)
کے بیچ لکھ کر تکمل لیتے ہیں یہاں تک بہت اچھا دیا لیکن مقام اور معیار حرکت کے علاوہ  دیگر ہر قیمت تغیرات کے بارے میں ہم کیا کرسکتے ہیں. حقیقت یہ ہے کہ ہر کلا سیے کی تغیرات کو مقام اور معیار حرکت کی صورت میں لکھا جاسکتا ہے۔ مثال کے طور پر حرکی توانائی کو
\[T=\frac{1}{2}mv^{2}=\frac{p^{2}}{2m}\]
اور زاویائی معیار حرکت کو
\[\text{\textbf{L}}=r\times mv=r\times p\]
لکھا جاسکت ہے جہاں زاویای معیار حرکت یک بودی مسائل میں نہیں پایا جا سکتا کسی بحر مقدار مثلاً
\(Q(x,p)\)
کو توقع تو قیمت کو حاصل کرنے کے لیے ہم ہر
P
کی جگہ
\(\frac{\hbar}{i}\frac{d}{\dif{x}}\)
پر کرتے ہیں اور اس کے بعد Q کو
\(\psi *\)
اور
\(\psi\)
کے بیچ تلورکر تکمل حاصل کرتے ہیں .
\[\textlangle Q(x,p) \textrangle=\int \psi *Q\big (x,\frac{\hbar}{i}\frac{\partial}{\partial{x}}\big )\psi \dif{x}\]
مثال کے طور پر حرکی توانائی کی تو تعائی قیمت درج ذیل ہو گی۔
\[\textlangle T \textrangle=\frac{-\hbar ^{2}}{2m}\int \psi *\frac{\partial ^{2}\psi}{\partial{x^{2}}}\dif{x}\]
مساوات 1:36 کسی بھی حرکی مقدار کو توقعاتی قیمت کے حصول کی مساوات ہے .
مساوات 1.35 اور 1.34  اس کی دو مخصوص مورتیں ہیں میں نے کوشش کی ہے کہ مساوات 
1.34\\
%
%QM 18-20
اصول لاتیکن:\\
فر ض کریں آپ ایک نے ایک لمبی رسی پکڑی ہے اور آپ اس کے اک سرکو اوپر نیچے ہم آہنگی کے ساتھ ہلا کر موج پیدا کرتے ہیں جیسا شکل 1.7 میں دکھایا گیا ہے . اب اگر آپ سے کوئی سوال کرے کہ یہ موج ٹھیک ٹھیک کس نقطے پر پایا جاتا ہے آپ اس کا جواب دینے سے قاصر ہونگے۔ چونکہ موج فیک ایک نقطے پر نہیں پائی جاتی مگر یہ ایک میٹر کی لمبائی پر پائی جاتی ہے . اگر آپ سے پوچھا جائے کہ اس کا طول موج کتنا ہے تو آپ اس کا تقریباً تقریباً صحیح جواب دے سکتے ہیں کہ اس کو طولی موج تقریباً ایک میٹر ہے اس کے برعکس اگر آپ اسی تو ایک جھٹکا دیں تو اس پر ایک نوکیلی موج پیدا ہوگی اب گرآپ سے پوچھا جائے اس موج کو طولی مرج کتنی ہے تو آپ جواب دینے سے قاصر ہونگے۔ مگر آپ سے اس کے برعکس یہ پوچھا جائے کہ یہ موج کس نقطے پر صحیح صحیح واقع ہے تو آپ اس کا جواب دے سکتے ہیں۔ پہلی شکل میں موج کا مقام پوچھا بے معنی سوال ہوگا۔ جبکہ دوسری شکل میں طولی موج جاننا بے معنی مقصد ہو گا۔ ہم ان دونوں کے بیچ کے حالات بھی پیدا کر سکتے ہیں . جن میں موج کا مقام بہتر بتانا ممکن ہو یا اس کا طولی موج بیتر بتانا نسبتا بہتر ہو . لیکن آپ دیکھ سکتے ہیں کہ موج کا طولی موج بہترے بہتر جانتے ہوئے موج کا مقام نہیں بتا سکتے یا پھر مقام جانتے ہوئے موج کا طولی موج کم سے کم جانتے ہیں۔ فورئیر تجریہ کا ایک مسئلہ اس کو مضبوط بنیادوں پر کھڑا کرتا ہے . فلحال میں صرف کیفی دلائل پیش کررہا ہوں . یہ حقیقت کسی بھی موجی مزہر کے لیے بہتر ہوگا . بالخصوص کوانٹم مکانیات کے موج تفال کے لیے . اب
\(\psi\) 
کا طولی موج اور اس کی معیار حرکت کا تعلق  ڈ برو گلی
کا کلیہ
\[p=\frac{h}{\lambda}=\frac{2\pi\hbar}{\lambda}\]
پیش کرتا ہے .
ہوں طولی موج میں پھیلاؤ معیار حرکت میں پھیلاؤ کے مترادف ہوتا ہے . اب ہماری عمومی مشایده یه ہوگا کہ کسی زرے کا مقام ٹھیک ٹھیک جانتے ہوئے ہم اس کے معیار حرکت تو تم سے کم ٹھیک جان سکتے ہیں۔ اس حقیقت کو درج ذیل تو جاتا ہے
\[\sigma_{x}\sigma_{p}\ge\frac{\hbar}{2}\]
جہاں
\(\sigma _{x} \) 
متریل x اور
\(\sigma _{p} \) 
معیار حرکت کا معیاری انحراف ہے۔ یہ ہائزن برگ کا مشہور اصول لاتیکن ہے . جسے ہم باب 3 میں ثابت کریں گے۔ \\
یہاں اس بات کی تسلی کر لیں کہ آپ کو اصول لا تیکن کا مطلب سمجھ آگیا ہو  کہ مقام کی ٹھیک ٹھیک پیمائش کی طرح معیار حرکت کی بیمائش بالکل ٹھیک ٹھیک جواب دے گی۔ یہاں پر پھیلاؤ سے مراد یہ ہوگا که بالکل ایک جیسے تیار کرده نظام پر پیمائش بالکل ایک جیسے نہیں ہونگے اگر آپ چاہیں تو اس حال پیدا کر سکتے ہیں کہ مقام کی بار بار پیمائش قریبی قریبی نتائج دے گا جوکہ
\(\psi\) 
تو ایک نوک کے قریب کرنے سے ہوگا۔ لیکن ایسا کرنے سے معیار حرکت اس حال میں بيت مختلف نتائج دے گا اس طرح آپ ایسا حال بنا سکتے ہیں جس میں تمام زروں کا معیار حرکت بالکل ایک جیسا ہو . یہ آپ
\(\phi\) 
کولمبی موج بنا کر کر سکتے ہیں لیکن اس طرح زروں کے مقام کی بیمائش کے نتائج بہت مخلتف ہوں گے۔ اور ہاں اور اگر آپ غصے کی حالت میں ہوں تو ایسی حالت بنا سکتے ہیں جس میں نہ تو مقام نہ تو معیار حرکت ٹھیک ٹھیک ہوں . مساوات 1.40 عدم مساوات ہے اس میں
\(\sigma _{x} \) 
اور 
\(\sigma _{p} \) 
پر کوئی حد مقرر نہیں کہ وہ کتنے لمبے ہو سکتے ہیں۔ آپ
\(\psi\) 
کو ایک لمبی بلدار لکیر بنائیں جس میں بہت سارے ابہار اور گڑھے پائے جاتے ہوں اور اس کی کوئی تو اتر نہ پایا جاتا ہو تو ایس صورت میں
\(\sigma _{x} \) 
اور
\(\sigma _{p} \) 
میں بہت ساری قیمین پائی جاتی ہوں گی .
سوال  1.9
ایک زرہ جس کی کمیت
m
ہے روج ذیل حال میں پایا جاتا ہے 
\[\psi (x,t)=Ae^{-a[(mx^{2}/\hbar)+it]}\]
جہاں
A
اور
a
مثبت حقیقی مستقل ہیں \\
ا)
A
طلاش کریں .\\
ب)
کس توانائی کے لیے
\(\phi\)
شرو دنگر مساوات پر پورا ترے گا۔\\
ج) 
\(x\)،
\(x^{2}\)،
\(p\)
اور
\(p^{2}\)
کی توقعاتی قیمین معلوم کریں۔\\
\(\sigma _{x} \) ,\(\sigma _{p} \) 
طلاش کریں . کیا ان کا حاصل ضرب اصول لا ٹینک پر پورا اترتا ہے ؟\\
سوال 
1.10\\
مستقل
\(\pi\)
کے عددی بھیلاؤ کے اولین
25
ہند سے
\(3,1,4,5,1,9\)
اگر آپ اس گرو سے بلا منصوبہ ایک ہندسہ منتخب کریں تو 
0-9
آنے کا احتمال کیا ہوگا؟\\
ب) کسی ہند سے کے آنے کا احتمال سب سے زیادہ ہے ؟
وسطانیہ ہندسہ کونسا ہے ؟
اوسط قیمت کیا ہوگی۔؟\\
 ج) اس تقسیم کا معیاری انہراس کیا ہوگا۔؟\\
سوال 
1.11\\
گاڑی کی رفتار پیمن کی ٹوٹی ہوئی سوئی آزادرانہ طور پر حرکت کرتی ہے . ہر جھٹکے کے بعد یہ ا عتراف سے ٹکڑا کر واپس آتی ہے جو 0 تا
\(\pi\)
زاویے کے بیچ ہوگا۔
اس کی کثافت احتمال
\(\rho (\theta)\)
کیا ہوگا۔ اشاره:
\(\rho (\theta)\dif{\theta}\)
وہ احتمال ہوگا جب سوئی زاویہ 
\(\theta\)
اور
\(\theta+\dif{\theta}\)
کے بیچ رکے .
\(\theta\)
کے لحاظ سے
\(\rho(\theta)\)
کو
\(\frac{-\pi}{2}\)
سے لے کر
\(\frac{3\pi}{2}\)
تک ترسیم کریں . اس وقفے میں جہاں
\(\rho\)
کی قیمت صفر ہوگی وہ شامل نہیں ہوگا۔ آپ تصدیق کر لیں کہ کل احتمال 1 کے برابر ہوگا۔\\
ب)
اس تقسیم کے لیے دریافت کریں \\
\[\textlangle \theta \textrangle, \textlangle \theta ^{2} \textrangle, \sigma\]
ج). اس طرح
\(\textlangle\sin{\theta}\textrangle\)
کی اوسط
اور
\(\textlangle\cos ^{2}\theta\textrangle\)
کی اوسط کا حساب لگائیں\\
\end{document}


