
\باب{ونزل، کرامرز، برلوان تخمین}
ونزل، کرامرز، برلوان ترکیب سے غیر تابع وقت شروڈنگر مساوات کی یک بُعدی تخمینی حل حاصل کیئے جا سکتے ہیں اسی بنیادی تصور کا اطلاق کئی دیگر تفرقی مساوات پر اور بالخصوص تین ابعاد میں مساوات شروڈنگر کی رداسی حصے پر کیا جاسکتا ہے۔ یہ بالخصوص مکید حال توانائیوں اور مخف رکاوٹ سے گزرنے کی سرنگ زنی شرح کے حساب میں مفید ثابت ہوتا ہے۔
اس کا بنیادی تصور درج ذیل ہے: فرض کریں ای کذرہ جس کی توانائی \عددی{E} ہو اک ایسے خطہ میں حرکت کرتا ہے جہاں مخفیہ \عددی{V(x)} ایک مستقل ہو۔ تفاعل موج \عددی{E>V} کی صورت میں درج ذیل روپ کا ہوگا
\begin{align*}
	\psi(x)=Ae^{\pm ikx}, && k\equiv\sqrt{2m(E-V)}/\hslash \text{\RL{جہاں}}
\end{align*}
دائیں رخ حرکت کرتے ہوئے ذرہ کے لیئے مثبت علامت جبکہ بائیں رخ کے لیئے منفی علامت استعمال ہوگا یقیناً ان دونوں کا خطی جوڑ ہمیں عمومی حل دیگا۔ یہ تفاعل موج ارتعاشی ہے جس کا طولِ موج \عددی{\lambda=2\pi/k} اٹل ہے اور اس کا حیطہ \عددی{A} غیر تغیر ہے۔ اب فرض کریں کہ \عددی{V(x)} مستقل نہیں ہے بلکہ \عددی{\lambda} کے لحاظ سے بہت آہستہ تبدیل ہوتا ہے تاکہ کئی مکمل طول امواج پر مخفیہ کو مستقل تصور کیا جاسکتا ہو۔ ایسی صورت میں ہم کہہ سکتے ہیں کہ \عددی{\psi} عملاً سائن نما ہوگا تاہم اس کا طولِ موج اور حیطہ \عددی{x} کے ساتھ ساتھ آہستہ آہستہ تبدیل ہوںگے۔ یہی ونزل، کرامرز، برلوان تخمین کی بنیاد ہے۔ درحقیقت یہ \عددی{x} پر دو مختلف طرز کے تابعیت کی بات کرتا ہے تیز ارتعاشات جنہیں طولِ موج اور حیطہ میں آہستہ آہستہ تبدیلی ترمیم کرتا ہو۔

اسی طرح \عددی{E<V} جہاں \عددی{V} ایک مستقل ہے کی صورت میں \عددی{\psi} قوت نمائی ہوگا۔
\begin{align*}
	\psi(x)=Ae^{\pm\kappa x},&& \kappa\equiv\sqrt{2m(V-E)}/\hslash \text{\RL{جہاں}}
\end{align*}
اور اگر \عددی{V(x)} ایک مستقل نہ ہو بلکہ \عددی{1/\kappa} کے لحاظ سے آہستہ آہستہ تبدیل ہوتا ہو تب حل عملاً قوت نمائی ہوںگے البتہ \عددی{A} اور \عددی{\kappa} اب \عددی{x} کے آہستہ آہستہ تبدیل ہوتے تفاعلات ہوںگے۔ یہ نظریہ کلاسیکی نقطہ واپسی جہاں \عددی{E\approx V} ہو کی قریبی پڑوس میں ناکامی کا شکار ہوگا چونکہ یہاں \عددی{\lambda} یا \عددی{1/\kappa} لامتناہی تک بڑھتا ہے اور ہم یہ نہیں کہہ سکتے ہیں کہ \عددی{V(x)} آہستہ آہستہ تبدیل ہوتا ہے۔ جیسا آپ دیکھیں گے اس تخمین میں نقات واپسی سے نمٹنا دشوار ترین  ہوگا اگرچہ آخری نتائج بہت سادہ ہوںگے۔
\حصہ{کلاسیکی خطہ}
مساوات شروڈنگر 
\begin{align*}
	-\frac{\hslash^2}{2m}\frac{\dif^2\psi}{\dif x^2}+V(x)\psi=E\psi
\end{align*}
کو درج ذیل روپ میں لکھا جاسکتا ہے
\begin{align}
	\frac{\dif^2\psi}{\dif x^2}=-\frac{p^2}{\hslash^2}\psi
\end{align}
جہاں 
\begin{align}
	p(x)\equiv\sqrt{2m[E-V(x)]}
\end{align}
اس ذرے کے معیارِ حرکت کا کلاسیکی کلیہ ہے جس کی کل توانائی \عددی{E} اور مخفی توانائی \عددی{V(x)} ہو۔ فل حال میں فرض کرتا ہوں کہ \عددی{E>V(x)} ہےلحاظہ \عددی{V(x)} حقیقی ہوگا اس خطہ کو ہم کلاسیکی خطہ کہتے ہیں کلاسیکی طور پر ذرہ \عددی{x} کے ساتھ پر رہنے کا پابند ہوگا \حوالہء{شکل \num{8.1}}۔ عمومی طور پر \عددی{\psi} ایک مخلوط تفاعل ہوگا جس کو حیطہ \عددی{A(x)} اور حیط \عددی{\phi(x)} جہاں دونوں حقیقی ہیں کی صورت میں لکھا جا سکتا ہے 	
\begin{align}
	\psi(x)=A(x)e^{i\phi(x)}
\end{align}
ہم \عددی{x} کے لحاظ سے تفرق کو قوت نمائی میں چھوٹی لکیر سے ظاہر کرتے ہوئے درج ذیل لکھ سکتے ہیں
\begin{align*}
	\frac{\dif\psi}{\dif x}=(A'+iA\phi')e^{i\phi}
\end{align*}
اور 
\begin{align}
	\frac{\dif^2\psi}{\dif x^2}=[A''+2iA'\phi'+iA\phi''-A(\phi')^2]e^{i\phi}
\end{align}
اس کو \حوالہء{مساوات \num{8.1}} میں پُر کرتے ہیں
\begin{align}
	A''+2iA'\phi'+iA\phi''-A(\phi')^2=-\frac{p^2}{\hslash^2}A
\end{align}
دونوں ہاتھ کی حقیقی اجزا کو ایک دوسرے کے برابر رکھ کر ایک حقیقی مساوات حاصل ہوگ جبکہ دونوں ہاتھ کے خیالی اجزا کو ایک دوسرے کے برابر رکھ کر دوسرا حقیقی مساوات حاصل ہوگا 
\begin{align}
	A''-A(\phi')^2=-\frac{p^2}{\hslash^2}A,&& \text{\RL{یا}}&& A''=A\left[(\phi')^2-\frac{p^2}{\hslash^2}\right]
\end{align}
اور 
\begin{align}
	2A'\phi'+A\phi''=0,&&\text{\RL{یا}}&&\left(A^2\phi'\right)'=0
\end{align}
\حوالہء{مساوات \num{8.6} اور \num{8.7}} ہر لحاظ سے اصل شروڈنگر مساوات کے معادل ہیں ان میں سے دوسرے کو با آسانی حل کیا جاسکتا ہے
\begin{align}
	A^2\phi'=C^2,&&\text{\RL{یا}}&& A=\frac{C}{\sqrt{\phi'}}
\end{align}
جہاں \عددی{C} ایک حقیقی مستقل ہوگا۔ ان میں سے پہلی \حوالہء{مساوات \num{8.6}} کو عموماً حل کرنا ممکن نہیں ہوگا یہی ہمیں تخمین کی ضرورت پیش آتی ہے ہم فرض کرتے ہیں کہ حیطہ \عددی{A} بہت آہستہ آہستہ تبدیل ہوتا ہے لحاظہ جزو \عددی{A''} قابلِ نظرانداز ہوگا۔ بلکہ یہ کہنا زیادہ درست ہوگا کہ ہم فرض کرتے ہیں کہ \عددی{(\phi')^2} اور \عددی{p^2/\hslash^2} دونوں سے \عددی{A''/A} بہت کم ہے۔ ایسی صورت میں ہم \حوالہء{مساوات \num{8.6}} کے بائیں ہاتھ کو نظرانداز کر کے درج ذیل حاصل کرتے ہیں
\begin{align*}
	(\phi')^2=\frac{p^2}{\hslash^2},&&\text{\RL{یا}}&&\frac{\dif\phi}{\dif x}=\pm\frac{p}{\hslash}
\end{align*}
جس کے تحت درج ذیل ہوگا
\begin{align}
	\phi(x)=\pm\frac{1}{\hslash}\int p(x)\dif x
\end{align}
میں فل حال اسکو ایک غیر قطعی تکمل لکھتا ہوں کسی بھی مستقل کو \عددی{C} میں زن کیا جاسکتا ہے جس کے تحت یہ مخلوط ہوسکتا ہے اس طرح درج ذیل ہوگا
\begin{align}
	\boxed{\psi(x)\cong\frac{C}{\sqrt{p(x)}}e^{\pm\frac{i}{\hslash}\int p(x)\dif x}}
\end{align}
اور تخمینی عمومی حل انکا خطی جوڑ ہوگا جہاں ایک جزو میں مثبت اور دوسرے میں منفی علامت استعمال ہوگی۔

آپ دیکھ سکتے ہیں کہ درج ذیل ہوگا
\begin{align}
	\abs{\psi(x)}^2\cong\frac{\abs{C}^2}{p(x)}
\end{align}
جس کے تحت نقطہ \عددی{x} پر ذرہ پایا جانے کا احتمال اس نقطہ پر ذرے کے کلاسیکی معیارِ حرکت لحاظہ سمتی رفتار کا بلعکس متناصب ہوگا۔ ہم یہی توقع رکھتے ہیں چونکہ جس مکام پر ذرہ کی رفتار تیز ہو وہاں اسے پانے کا احتمال کم سے کم ہوگا۔ درحقیقت بعض اوقات تفرقی مساوات میں جزو \عددی{A''} کو نظرانداز کرنے کی بجائے اس نیم کلاسیکی مشاہدہ سے آغاز کرتے ہوئے ونزل، کرامرز، برلوان تخمین اغز کیا جاتا ہے۔ مواخر الذکر طریقہ ریاضیاتی طور پر زیادہ صاف ہے لیکن اوّل الذکر بہتر عقلی وقجہ پیش کرتا ہے۔

\ابتدا{مثال}
\موٹا{دو انتصابی دیواروں والا مخفیہ کنواں۔} فرض کرٰن ہمارے پاس ایک لامتناہی چکور کنواں ہو جس کی تہہ غیر ہموار ہو \حوالہء{شکل \num{8.2}}۔
\begin{align}
	V(x)=
	\begin{cases}
		\text{\RL{کچھ مخصوص تفاعل}}, & 0<x<a \text{\RL{اگر}}, \\
		\infty, & \text{\RL{دیگر صورت}}
	\end{cases}
\end{align}
کنواں کے اندر ہر جگہ \عددی{E>V(x)} فرج کرتے ہوئے درج ذیل ہوگا
\begin{align*}
	\psi(x)\cong\frac{1}{\sqrt{p(x)}}\left[C_+e^{i\phi(x)}+C_-e^{-i\phi(x)}\right]
\end{align*}
جس کو درج ذیل لکھا جاسکتا ہے
\begin{align}
	\psi(x)\cong\frac{1}{\sqrt{p(x)}}[C_1\sin\phi(x)+C_2\cos\phi(x)]
\end{align}
جہاں درج ذیل ہوگا
\begin{align}
	\phi(x)=\frac{1}{\hslash}\int_{0}^{x}p(x')\dif x'
\end{align}
جیسا ہم ذکر کر چکے ہیں ہم تکمل کی زیریں حد اپنی مرضی کا منتخب کرسکتے ہیں یہاں یہی کیا گیا۔ اب \عددی{x=0} پر \عددی{\psi(x)} لاظماً صفر ہوگا لحاظہ چونکہ \عددی{\psi(0)=0} ہے \عددی{C_2=0} ہوگا۔ ساتھ ہی \عددی{x=a} پر بھی \عددی{\psi(x)} صفر ہوگا لحاظہ درج ذیل ہوگا
\begin{align}
	\phi(a)=n\pi&&(n=1, 2, 3,\dots)
\end{align}
ماخوذ 
\begin{align}
	\boxed{\int_{0}^{a}p(x)\dif x=n\pi\hslash}
\end{align}
کوانٹازنی کی درج بالا شرط تخمینی اجازتی توانائیاں تعین کرتا ہے۔

مثلاً اگر کنویں کی تہہ ہموار ہو \عددی{V(x)=0} تب \عددی{p(x)=\sqrt{2mE}} ایک مستقل ہوگا اور \حوالہء{مساوات \num{8.16}} کے تحت \عددی{pa=n\pi\hslash} یا 
\begin{align*}
	E_n=\frac{n^2\pi^2\hslash^2}{2ma^2}
\end{align*}
جو لامتناہی چکور کنواں کی توانائیوں کا پرانا کلیہ ہے \حوالہء{مساوات \num{2.27}}۔ یہاں ونزل، کرامرز، برلوان تخمین ہمیں بلکل ٹھیک ٹھیک جواب فراہم کرتا ہے چونکہ اصل تفاعل موج کا حیطہ مستقل ہے لحاظہ \عددی{A''} کو نطرانداز کرنے سے کوئی اثر نہیں پڑا۔
\انتہا{مثال}
\ابتدا{سوال}
ونزل، کرامرز، برلوان تخمین استعمال کرتے ہوئے ایسے لامتناہی چکور کنواں کی اجزاتی توانائیاں \عددی{E_n} تلاش کریں جس کی آدھی تہہ میں \عددی{V_0}  بلندی کی سیڑھی پائی جاتی ہو \حوالہء{شکل \num{6.3}} 
\begin{align*}
	V(x)=
	\begin{cases}
		V_0, & 0<x<a/2 \text{\RL{اگر}} \\
		0, & a/2<x<a \text{\RL{اگر}} \\
		\infty, & \text{\RL{دیگر صورت}}
	\end{cases}
\end{align*}
اپنے جواب کو \عددی{V_0} اور \عددی{E_n^0\equiv(n\pi\hslash)^2/2ma^2} کی صورت میں لکھیں جہاں بغیر سیڑھی لامتناہی چکور کنواں کے \عددی{n}ویں اجازتی توانائی \عددی{E_n^0} ہے۔ فرض کریں \عددی{E_1^0>V_0} تاہم یہ فرض نہ کریں کہ \عددی{E_n\gg V_0} ہوگا۔ اپنے جواب کا موازنہ \حوالہء{مثال \num{6.1}} میں رتبہ اوّل ںطریہ اضطراب سے حاصل جواب کے ساتھ کریں۔ آپ دیکھیں گے کہ بہت چھوٹی \عددی{V_0} جہاں نظریہ اضطراب کارآمد ہوگا یا بہت بڑی \عددی{n} جہاں ونزل، کرامرز، برلوان تخمین کارآمد ہوگی کی صورت میں جوابات ایک دوسرے جیسے ہوںگے۔
\انتہا{سوال}
\ابتدا{سوال}
ونزل، کرامرز، برلوان کلیہ \حوالہء{مساوات \num{8.10}} کو \عددی{\hslash} کی طاقتی پھیلاو سے اغز کیا جاسکتا ہے۔ آزاد ذرہ کی تفاعل موج \عددی{\psi=A\exp(\pm ipx/\hslash)} سے حوصلہ افزائی حاصل کر کے ہم درج ذیل لکھتے ہیں
\begin{align*}
	\psi(x)=e^{if(x)/\hslash}
\end{align*}
جہاں \عددی{f(x)} کوئی مخلوط تفاعل ہے۔ دیہان رہے کہ کسی بھی غیر صفر تفاعل کو اس طرح لکھا جاسکاتا ہے لحاظہ ایسا کرنے سے ہم عمومیت نہیں کھوتے۔

(الف) اس کو \حوالہء{مساوات \num{8.1}} روپ کی مساوات شروڈنگر میں پُر کر کے درج ذیل دیکھائیں
\begin{align*}
	i\hslash f''-(f')^2+p^2=0
\end{align*}
(ب) تفاعل \عددی{f(x)} کو \عددی{\hslash} کی طاقتی تسلسل کی صورت 
\begin{align*}
	f(x)=f_0(x)+\hslash f_1(x)+\hslash^2f_2(x)+\dots
\end{align*}
میں لکھ کر \عددی{\hslash} کی ایک جیسی طاقتوں کو اکھٹا کر کے درج ذیل دیکھائیں
\begin{align*}
	(f'_0)^2=p^2,\quad if''_0=2f'_0f'_1,\quad if''_1=2f'_0f'_2+(f'_1)^2,\quad\text{\RL{وغیرہ وغیرہ}}
\end{align*}
(ج) انہیں \عددی{f_0(x)} اور \عددی{f_1(x)} کے لیئے حل کر کے دیکھائیں کہ \عددی{\hslash} کی اوّل رتبہ تک آپ \حوالہء{مساوات \num{8.10}} دوبارہ حاصل کرتے ہیں۔

تبصرہ: منفی عددی کی لوگردم کی تعریف \عددی{\ln(-z)=\ln(z)+in\pi} ہے جہاں \عددی{n} ایک طاق عدد صحیح ہوگا۔ اگر آپ اس کلیہ سے ناواقف ہوں تب دونوں اطراف کو قوت نما میں منتقل کر کے دیکھیں۔
\انتہا{سوال}

