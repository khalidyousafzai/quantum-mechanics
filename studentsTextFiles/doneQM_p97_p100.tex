
 چونکہ ہرمیشی عاملین کی توقعاتی قیمت حقیقی ہوتی ہے لحاظہ یہ کوانٹم میکانیات میں قدرتی طور پر رونما ہوتے ہیں۔
\begin{align}
	\text{\RL{قابل مشاہدہ کو ہرمشی عاملین ظاہر کرتے ہیں}}
\end{align}
آئیں اس کی تصدیق کریں۔ مثلاً کیا معیاری حرکت کا عامل ہرمیشی ہے؟ 
\begin{align}
	\langle f\mid\hat{p}g\rangle = \int_{-\infty}^{\infty}f^{*}\frac{\hbar}{i} \frac{dg}{dx}dx = \frac{\hbar}{i}f^{*}g\mid^\infty_{-\infty} + \int_{-\infty}^{\infty}(\frac{\hbar}{i}\frac{df}{dx})^{*}gdx = \langle \hat{p}f\mid g \rangle 
\end{align}
میں نے تکمل بالحصص استعمال کیا ہے اور چونکہ \عددی{f(x)} اور \عددی{g(x)} قابلِ تکمل مربع ہیں لحاظہ\(\pm\infty\) پر یہ صفر تک پہنچیں گے۔ لحاظہ تکمل میں سرحدی اجزاء کو رد کیا گیا ہے۔ آپ نی دیکھا ہوگا کہ تکمل بالحصص کے بنا منفی کی علامت کو \عددی{i} کا مخلوط جوڑی دار سے حاصل منفی کی علامت ختم کرتی ہے۔ علامت \(d/dx\) جس میں \عددی{i} نہیں پایا جاتا غیر ہرمیشی ہے اور یہ کسی بھی قابلِ مشاہدہ کو ظاہر نہیں کرتا۔

\ابتدا{سوال}
دیکھائیں کہ ہلبرٹ فضاء میں تمام تفاعل \عددی{h} جن کے لیئے \(\langle h\mid\hat{Q}h \rangle = \langle \hat{Q}h\mid h \rangle\) ہو تب تمام \عددی{f} اور \عددی{g} کے لیئے \(\langle f\mid\hat{Q}g \rangle = \langle \hat{Q}f\mid g \rangle\) ہوگا۔ مساوات \num{3.16} اور \num{3.17} میں ہرمیشی کی تعریفات معادل ہیں۔ اشارہ پہلے \(h = f+g\) لیں اور بعد میں \(h = f+ig\) لیں۔
\انتہا{سوال}
\ابتدا{سوال}

(الف) دیکھائیں کہ دو ہرمیشی عاملین کا مجموعہ ازخود ہرمیشی ہوگا۔

(ب) فرض کریں \(\hat{Q}\) ہرمیشی ہے اور \(\alpha\) ایک مخلوط عدد ہے۔ \(\alpha\) پر کیا شرائط مسلط کرنے سے \(\alpha\hat{Q}\) بھی ہرمیشی ہوگا؟

(ج) دو ہرمیشی عاملین کا حاصل ضرب کب ہرمیشی ہوگا؟

(د) دیکھائیں کہ ہامل مقام \((\hat{x} = x)\) اور ہیملٹونی عامل \((\hat{H} = -(\hbar^{2}/2m)d^{2}/dx^{2}+V(x))\) ہرمیشی ہے۔
\انتہا{سوال}
\ابتدا{سوال}
عامل \(\hat{Q}\) کا ہرمیشی جوڑی دار یا شریک عامل \(\hat{Q^\dagger}\) درج ذیل کو مطمئن کرتا ہے۔
\begin{align}
	\langle f\mid\hat{Q}g\rangle = \langle \hat{Q}^{\dagger}f\mid g\rangle (\text{for all f and g})
\end{align}
یوں ہرمیشی عامل اپنے ہر میشی جوڑی دار کے برابر ہوگا \(\hat{Q} = \hat{Q^\dagger}\)۔

(الف) \عددی{x, i} اور \(d/dx\) کے ہرمیشی جوڑی دار تلاش کریں۔

(ب) ہارمونی مرتعش کے عامل رفت \(a_+\) مساوات \num{2.47} کا ہارمیشی جوڑی دار تیار کریں۔

(ج) دیکھائیں کہ \((\hat{Q}\hat{R})^\dagger = \hat{R}^\dagger \hat{Q^\dagger}\) ہوگا۔
\انتہا{سوال}
\جزوحصہ{قابلِ معلوم حالات}
کوانٹم میکانیات کی ناقابلِ معلومیت کی بنا عام طور پر بلکل یکساں تیار کردہ کہ صدرہ جو تمام \(\psi\) حال میں ہوں کی قابلِ مشاہدہ \عددی{Q} پیمائش سے ایک جیسے نتائج حاصل نہیں ہوں گے۔ سوال: کیاایسا ممکن ہوگا کہ ہم کوئی ایسا حال تیار کریں جہاں پہ \عددی{Q} کی ہر پیمائش کوئی مخصوص قیمت جسے ہم \عددی{q} کہہ سکتے ہیں دیگا؟ اس کو ٰپ قابلَ مشاہدہ \عددی{Q} کی قابلَ معلوم حال کہہ سکتے ہو۔ ہم ایسی ایک مثال دیکھ چکے ہیں: ہیملٹونی کی ساکن حالات قابلِ معلوم ہے۔ ساکن حال\(\psi_n\) میں ایک ذرہ کی قُل توانائی کی پیمائش ہر صورت مطابقتی اجازتی توانائی \(E_n\) دیگا۔

قابلِ معلوم حال میں \عددی{Q} کی معیاری انحراف صفر ہوگی جسے درج ذیل لکھا جاسکتا ہے
\begin{align}
	\sigma^{2} = \langle (\hat{Q}-\langle Q \rangle)^{2} \rangle = \langle \psi\mid(\hat{Q}-q)^{2}\psi \rangle = \langle (\hat{Q}-q)\psi\mid(\hat{Q}-q)\psi \rangle = 0
\end{align}
اب اگر ہر پیمائش \عددی{q} دے تب ظاہر ہے کہ اوسط قیمت بھی \عددی{q} ہوگی \(\langle Q \rangle = q\)۔ چونکہ \(\hat{Q}\) ہرمیشی ہے لحاظہ \(\hat{Q}-q\) بھی ہرمیشی عامل ہوگا۔ میں نے اندرونی ضرب میں اس حقیقت کو استعمال کرتے ہوئے ایک جز ضربی کو بائیں منتقل کیا لیکن ایسا واحد تفاعل جس کا خود کے ساتھ اندرونی ضرب صفر ہے لحاظہ درج ذیل ہوگا
\begin{align}
	\hat{Q}\psi = q\psi
\end{align}
یہ عامل کیونکہ امتیازی قدر مساوات یا آئگنی قدر مساوات ہے۔ \(\hat{Q}\)  کا ایک امتیازی تفاعل \(\psi\) ہے جس کی مطابقتی آئگنی قدر \(\hat{Q}\) ہے۔ یوں درج ذیل ہوگا
\begin{align}
	\text{Determinate states are eigenfunctions of}\hat{Q}
\end{align}
ایسے حال پر \عددی{Q} کی پیمائش لاظماً امتیازی قدر \عددی{q} دیگی۔

دیہان رہے کہ آئگنی قدر ایک عدد ہے ناکہ کوئی عامل یا تفاعل۔ ایک آئگنی تفاعل کو ایک مستقل سے ضرب دینے سے دوبارہ ایک آئگنی تفاعل حاصل ہوگا جسکی آئگنی قیمت وہی ہوگی۔ امتیازی تفاعل کی تعریف کے رو سے صفر ایک ئگنی تفاعل نہیں ہے۔ اگر ایسا ہوتا تب کسی بھی عامل \(\hat{Q}\) اور تمام \عددی{q} کے لیئے \(\hat{Q}0=q0=0\) ہوتا اور ہر عدد ایک آئگنی قدر ہوتا۔ ہاں آئگنی قدر کی قیمت صفر ہوسکتی ہے ایک عامل کی تمام امتیازی اقدار کو اکٹھا کرنے سے اسکا طف حاصل ہوگا۔ بعض اوقات دو یا دو سے ذیادہ خطی غیر تابع امتیازی تفاعل کی امتیازی قیمت ایک دوسرے جیسی ہوگی ایسی صورت میں ہم کہتے ہیں کہ طف انحطاطی ہے۔

مثال کے طور پر قل توانائی کے قابلِ معلوم حالات ہیملٹونی کے امتیازی تفاعل ہوںگے۔ 
\begin{align}
	\hat{H}\psi = E\psi
\end{align}
جو عین وقت کا غیر تابع شروڈنگر مساوات ہے۔ ایسی سیاق و سباق میں ہم امتیازی قدر کے لیئے حرف \عددی{E} استعمال کرتے ہیں اور امتیازی تفاعل کے لیئے \(\psi\) اس کے ساتھ جز \(\exp(-iEt/\hbar)\) جوڑ کا \(\psi\) حاصل ہوگا جو اگر آپ چاہیں اب بھی \عددی{H} کا امتیازی تفاعل ہے۔
\ابتدا{مثال}
درج ذیل عامل پر غور کریں جہاں دوابعاد میں \(\phi\) قطبی معدد کا ایک متغیر ہے
\begin{align}
	\hat{Q} \equiv i\frac{d}{d\phi}
\end{align}
یہ عامل سوال \num{2.46} میں کارآمد ثابت ہوسکتا تھا کیا \(\hat{Q}\) ہرمیشی ہے؟ اسکے امتیازی تفاعل اور امتیازی اقدار تلاش کریں۔

حل: یہاں ہم متناہی وقفہ \(0\leq\phi\leq 2\pi\) پر تفاعل \(f(\phi)\) کے ساتھ کام کر رہے ہیں جہاں \(\phi\) اور \(\phi+2\pi\) ایک ہی طبی نقطہ کو طاہر کرتے ہیں لحاظہ درج ذیل ہوگا 
\begin{align}
	f(\phi + 2\pi) = f(\phi)
\end{align}
تکمل بالحصص استعمال کرتے ہوئے درج ذیل ہوگا 
\begin{align*}
	\langle f\mid\hat{Q}g \rangle = \int_{0}^{2\pi}f{*}(i\frac{dg}{d\phi})d\phi = if{*}g\mid^{2\pi}_{0} - \int_{0}^{2\pi}i(\frac{df^*}{d\phi})gd\phi = \langle \hat{Q}f\mid g \rangle
\end{align*}
لحاظہ \(\hat{Q}\) ہرمیشی ہے یہاں مساوات \num{3.26} کی بنا سرحدی جز خارج ہوگا۔ امتیازی اقدار مساوات  
\begin{align}
	i\frac{d}{d\phi}f(\phi) = qf(\phi)
\end{align}
کا عمومی حل درج ذیل ہوگا
\begin{align}
	f(\phi) = Ae^{-iq\phi}
\end{align}
مساوات \num{3.26} \عددی{q} کی ممکنہ قیمتوں کو درج ذیل پر رہنے کا پابند بناتی ہے۔
\begin{align}
	e^{-iq2\pi} = 1 \Rightarrow q = 0, \pm1, \pm2, \dots
\end{align}
اس عامل کا طف تمام عدد صحیح پر مشتمل ہوگا اور یہ غیر انحطاطی ہے۔
\انتہا{مثال}
