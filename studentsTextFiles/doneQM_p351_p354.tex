
کوانٹم برقی حرقیات اس کتاب کے دائرہ کار سے باہر ہے تاہم آئنسٹائن کی ایک خوبصورت دلیل ان تینوں انجزاب تحرقی اخراج اور خود با خود اخراج کا تعلق پیش کرتا ہے۔ آئنسٹائن نے خود با خود اخراج کی وجہ زمینی حال برقناطیسی میدان کا اضطراب پیش نہیں کی تاہم انکے نتائج ہمیں خود با خود اخراج کا حساب کرنے کا مجاز بناتی ہے جس سے ہیجان جوہری حال کی قدرتی عرصہ حیات تلاش کی جا سکتے ہے۔ ایسا کرنے سے پہلے ہر طرف سے غیر یک رنگی، غیر تقطیب شدہ، غیر اتساکی برقناطیسی امواج کی آمد سے جوہر کے رد عمل پر بات کرتے ہیں۔ حراری شعاع میں جوہر رکھنے سے ایسی صورت حال پیدا ہوگی۔
\جزوحصہ{غیر اتساکی اضطراب}
برقناطیسی موج کی کثافت توانائی درج ذیل ہے۔ جہاں \(E_0\) ہمیشہ کی طرح برقی میدان کا حیطہ ہوگا۔
\begin{align}
	u = \frac{\epsilon_0}{2}E^2_0
\end{align}
یوں حیرانی کی بات نہیں کہ تحویلی احتمال مساوات \num{9.36} میدان کی کثافت توانائی کا راست متناسب ہے۔
\begin{align}
	P_{b\rightarrow a }(t) = \frac{2u}{\epsilon_0\hbar^2}\abs{p}^2 \frac{\sin^2[(\omega_0-\omega)t/2]}{(\omega_0-\omega)^2}
\end{align}
تاہم یہ نتیجہ واحد ایک تعدد \(\omega\) پر یکرنگی موج کے لیئے درست ہوگا۔ کئی عملی استعمال میں نظام پر ایک بری تعددی پٹی کی برقناطیسی امواج کی روشنی ڈالی جائے گی ایسی صورت میںٍ \(u\rightarrow\rho(\omega)d\omega\) ہوگا جہاں \(\rho(\omega)d\omega\) تعدی ساتھ \(d\omega\) میں کثافت توانائی ہے اور تحویلی احتمال درج ذیل تکمل کا روپ اختیار کرے گا
\begin{align}
	P_{b\rightarrow a}(t) = \frac{2}{\epsilon_0\hbar^2}\abs{p}^2\int_{0}^{\infty}\rho(\omega){\frac{\sin^2[(\omega_0-\omega)t/2]}{(\omega_0-\omega)^2}}d\omega
\end{align}
کنگی کوسین میں جزو کی چوٹی \(\omega_0\) پر پائی جاتی ہے شکل\num{9.2} جبکہ عام طور پر \(\rho(\omega)\) کافی چوڑا ہوگا لحاظہ ہم \(\rho\omega\) کی جگہ \(\rho(\omega_0)\) لکھ کر اسے تکمل کے باہر منتقل کر سکتے ہیں۔
\begin{align}
	P_{b\rightarrow a}(t) \cong \frac{2\abs{p}^2}{\epsilon_0\hbar^2}\rho(\omega_0)\int_{0}^{\infty}\frac{\sin^2[(\omega_0-\omega)t/2]}{(\omega_0-\omega)^2}d\omega
\end{align}
متغیرات تبدیل کر کے \(x\equiv(\omega_0-\omega)t/2\) لکھ کر تکمل کے حدوں کو \(x = \pm\infty\) تک وصعت دے کر چونکہ باہر تکمل صفر ہی ہے اور قطعی تکمل کو ھدول سے دیکھ کر 
\begin{align}
	\int_{-\infty}^{\infty}\frac{\sin^2x}{x^2}dx = \pi
\end{align}
درج ذیل حاصل ہوتا ہے 
\begin{align}
	P_{b\rightarrow a}(t)\cong\frac{\pi\abs{p}^2}{\epsilon_0\hbar^2}\rho(\omega_0)t
\end{align}
اس بار تحویلی احتمال وقت \عددی{t} کا راست متناسب ہے۔ آپ نے دیکھا کہ یکرنگی اضطراب کے برعکس غیر اتساکی تعدد کی وصعت پلٹیں کھاتا ہوا احتمال نہیں دیتا ہے۔ بلخصوص تحویلی شرع \((R\equiv dP/dt)\) ایک مستقل ہوگا:
\begin{align}
	R_{b\rightarrow a} = \frac{\pi}{\epsilon_0\hbar^2}\abs{p}^2\rho(\omega_0)
\end{align}
اب تک ہم فرض کرتے رہے ہیں کہ اضطرابی موج \عددی{y} رخ سے آمدی شکل\num{9.3} اور \عددی{z} رخ تکتیب شدہ ہے۔ لیکن ہم اُس صورت میں دلچسپی رکھتے ہیں جب جوہر پر شعاع ہر رخ سے آمدی ہو اور اس میں ہر ممکنہ تکتیب پائی جاتی ہو۔ میدان کی توانائی \((\rho(\omega))\) ان مختلف انداز میں برابر تقسیم ہوگی۔ ہمیں \(\abs{p}^2\) کی جگہ \(\abs{p.\hat{n}}^2\) کی اوسط قیمت درکار ہوگی جہاں مساوات\num{9.33} کو عمومیت دیتے ہوئے درج ذیل ہوگا۔ 
\begin{align}
	p \equiv q \langle \psi_b\abs{r}\psi_a \rangle
\end{align}
اور اوسط تمام تکتیب اور تمام آمدی رخ پر لیا جائے گا۔ اوسط درج ذیل طریقہ سے حاصل کیا جا سکتا ہے۔ کروی محدد منتخب کرکے حرکت کے رخ کو \عددی{z} محور پر رکھیں تاکہ تکتیب \عددی{xy} سطح میں ہو اور مستقل سمتیہ \(\rho\) سطح \عددی{yz} میں پایا جاتا ہو شکل 9.5۔
\begin{align}
	p.\hat{n} = p\cos\theta
\end{align}  
تب 
\begin{align*}
	\abs{p.\hat{n}}^2_{ave} = \frac{1}{4\pi}\int\abs{p}^2\cos^2\theta\sin\theta d\theta d\phi
\end{align*}
اور درج ذیل ہوگا 
\begin{align}
	\abs{p.\hat{n}}^2_{ave} = \frac{\abs{p}^2}{4\pi}(-\frac{\cos^3\theta}{3})\mid^\pi_{0}(2\pi) = \frac{1}{3}\abs{p}^2
\end{align}
ماخوذ ہر جانب سے آمدی، غیر تکتیبی، غیر اتساکی شعاع کے زیرِ اثر حال \عددی{b} سے حال \عددی{a} میں تحرقی اخراج کا تحویلی شرع درج ذیل ہوگا۔
\begin{align}
	R_{b\rightarrow a} = \frac{\pi}{3\epsilon_0\hbar^2}\abs{p}^2\rho(\omega_0)
\end{align}
جہاں دو حالات کے بیچ برقی جفت کتب معیارِ اثر کا قالبی رکن \عددی{p} ہوگا مساوات\num{9.44} اور \(\omega_0 = (E_b-E_a)/\hbar\) پر فی اکائی تعدد میدان میں کثافتِ توانائی \(\rho(\omega_0)\) ہوگی۔

