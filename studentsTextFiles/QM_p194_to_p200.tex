\ابتدا{سوال}
فرض کریں کہ ہم جانتے ہیں کہ دو عدد \( 1/2 \) چکر ذرات یکتا تنظیم  \حوالہ {  4.178 } میں پائے جاتے ہیں ۔مان لیں کہ اکائی سمتیا \عددی{S_{a}^{(1)}  } کے رخ ذرہ 1 کے چکری زاویائی معیار حرکت کا جز \عددی{  \hat{a} } ہے اسی طرح مان لیں کہ اکائی سمتیا \عددی{  S_{b}^{(2)}\ }  کے رخ ذرہ 2 کے چکری زاویائی معیار حرکت کا جز \عددی{  \hat{b} } ہے۔ درج ذیل دکھائیں جہاں \عددی{   \hat{a}} اور \عددی{  \hat{b} } کے بیچ زاویہ \عددی{  \theta } ہے 
\begin{align}
    \langle S_{a}^{(1)}S_{b}^{(2)}\rangle=-\frac{\hslash^{2}}{4}\cos\theta
\end{align}
\انتہا{سوال}
\ابتدا{سوال}
\begin{enumerate}[a.]
\item     کلیبش گورڈن عددی سروں کو \( s_1=1/2\)\( s_2=any thing\) کچھ بھی لیتے ہوئے حاصل کریں۔ آپ درج ذیل میں \عددی{ A  } اور \عددی{  B } عددی سروں کی وہ قیمت تلاش کرنا چاہتے ہیں جن کے لیے \عددی{  |sm\rangle } کا امتیازی حال ویکٹر \عددی{   S^2} ہو گا 
\begin{align*}
    |sm \rangle=A|\frac{1}{2}\frac{1}{2}\rangle|S_2(m-\frac{1}{2})\rangle+B|\frac{1}{2}(-\frac{1}{2})\rangle|S_2(m+\frac{1}{2})\rangle
\end{align*} 
مساوات 4.179 تا مساوات 4.182 کی ترکیب استعمال کریں ۔ اگر آپ یہ جاننے سے قاصر ہوں کہ \عددی{  S_{x}^{(2)} } مثلاً ویکٹر \عددی{  |s_{2} m_{2}\rangle } پر کیا کرتا ہے تو مساوات 4.136 سے رجوع کریں اور مساوات 4.147  سے قبل جملہ دوبارہ پڑھیں۔ جواب:
\begin{align*}
    A=\sqrt{\frac{s_2\pm m+1/2}{2s_2+1}}; 
     B=\pm \sqrt{\frac{s_2\mp m+1/2}{2s_2+1}} 
\end{align*}
جہاں  \( s=s_2\pm1/2 \)  علامتیں تعین کرتی ہیں۔
\item اس عمومی نتیجے کی تصدیق جدول 4.8 میں تین یا چار درجہ دیکھ کر کریں۔
\end{enumerate}
\انتہا{سوال}
\ابتدا{سوال}
ہمیشہ کی طرح \عددی{  S_z } کی امتیازی حالات کو اساس لیتے ہوئے 3/2چکر کے ذرے کے لیے قالب \عددی{  S_x } تلاش کریں۔ امتیازی مساوات حل کرتے ہوئے \عددی{ S_x  } کی امتیازی اقدار معلوم کریں۔
\انتہا{سوال}
\ابتدا{سوال}
مساوات 4.145 اور 4.147 میں 1/2 چکر سوال 4.31 میں ایک چکر اور سوال 4.52 میں 3/2 چکر کے قالبوں کی بات کی گئی۔ ان نتائج کو عمومیت دیتے ہوئے اختیاری \عددی{  s } چکر کے لیے چکری قالب تلاش کریں۔
جواب:
\begin{align*}
    \kvec S_z&=\hslash \begin{pmatrix}
    s&0&0&\cdots&0\\0&s-1&0&\cdots&0\\0&0&s-2&\cdots&0\\\vdots&\vdots&\vdots&\cdots&\vdots\\0&0&0&\cdots&-s
    \end{pmatrix} \\ 
    \kvec S_x&=\frac{\hslash}{2}\begin{pmatrix}0&b_s&0&0&\cdots&0&0\\b_s&0&b_{s-1}&0&\cdots&0&0\\0&b_{s-1}&0&b_{s-2}&\cdots&0&0\\0&0&b_{s-2}&0&\cdots&0&0\\\vdots&\vdots&\vdots&\vdots&\cdots&\vdots&\vdots\\0&0&0&0&\cdots&0&b_{-s+1}\\0&0&0&0&\cdots&b_{-s+1}&0\end{pmatrix} \\
    \kvec S_y&=\frac{\hslash}{2}\begin{pmatrix}0&\iota b_s&0&0&\cdots&0&0\\\iota b_s&0&-\iota b_{s-1}&0&\cdots&0&0\\0&\iota b{s-1}&0&-\iota b{s-2}&\cdots&0&0\\0&0&\iota b_{s-2}&0&\cdots&0&0\\\vdots&\vdots&\vdots&\vdots&\cdots&\vdots&\vdots\\0&0&0&0&\cdots&0&-\iota b_{-s+1}\\0&0&0&0&\cdots&\iota b_{-s+1}&0
    \end{pmatrix}
\end{align*} 
جہاں \(b_j=\sqrt{(s+j)(s+1-j)}\) ہو گا۔
\انتہا{سوال}
\ابتدا{سوال}
کروی ہارمونیات کے لیے،؟؟؟؟ ضربی جز درج ذیل طریقے سے حاصل کریں۔ ہم حصہ 4.1.2 سے درج ذیل جانتے ہیں 
\begin{align*} 
    Y_{l}^{m}=B_{    l}^{m}e^{\iota m\phi}P_{l}    ^{m}(\cos\theta)
\end{align*} 
آپ کو جز \عددی{  B_{l}^{m} } تعین کرنا ہو گا (جس کی قیمت تلاش کیے بغیر میں نے ذکر مساوات 4.32 میں کیا)۔ مساوات 4.120 ، 4.121 اور 4.130 استعمال کرتے ہوئے \عددی{  B_{l}^{m+1} } کی صورت میں \عددی{ B_{l}^{m}  } کا کلیہ توالی دریافت کریں۔ اس کو \عددی{  m } کے ریاضی ماخول کی ترکیب سے حل کرتے ہوئے \عددی{  B_{l}^{m} } کو مجموعی مستقل \عددی{ C(l)  } تک حل کریں۔آخر میں سوال 4.22 کا نتیجہ استعمال کرتے ہوئے اس مستقل کا بھی کچھ کریں۔ شریک لیجانڈر تفاعل کے تفرک کا درج ذیل کلیہ مددگار ثابت ہو سکتا ہے:
\begin{align}
    (1-x^{2})\frac{ d P_{l}^{m}} {dx}=\sqrt{1-x^{2}}  P_{l}^{m+1}-mxP_{l}^{m}
\end{align}
\انتہا{سوال}
\ابتدا{سوال}
ہائیڈروجن جوہر میں ایک الیکٹران درج ذیل چکر اور فضائی حال کے ملاپ میں پایا جاتا ہے 
\begin{align*}
    R_{21}(\sqrt{1/3}Y_{1}^{0}\chi + \sqrt{2/3}Y_{1}^{1}\chi-)
\end{align*} 
\begin{enumerate}[a.]
\item مداری زاویائی معیار حرکت کے مربع \عددی{ (L^{2})  } کی پیمائش سے کیا قیمتیں حاصل ہو سکتی ہیں؟ ہر قیمت کا انفرادی احتمال کیا ہو گا؟ 
\item یہی کچھ معیاری \(z\) زاویائی معیار حرکت کے \عددی{ (L_z) } جز کے لیے معلوم کریں۔
\item یہی کچھ چکری زاویائی معیار حرکت کے مربع سکیئر\((S^2)\) کے لیے معلوم کریں۔
\item یہی کچھ چکری زاویائی معیار\(z\) کے \عددی{  (S_z) } جز کے لیے کریں۔ کل زاویائی معیار حرکت کو \(\kvec{J}=\kvec{L}+\kvec{S}\) لیں ۔
\item آپ \عددی{  J^2 } کی پیمائش کرتے ہیں آپ کیا قیمتیں حاصل کرتے ہیں ان کا انفرادی احتمال کیا ہو گا
\item یہی کچھ \عددی{  J_z } کے لیے معلوم کریں۔
\item  آپ ذرے کے مقام کی پیمائش کرتے ہیں، اس کی \عددی{   r,\theta,\phi }  پر پائے جانے کی کثافت احتمال کیا ہو گا؟ 
\item آپ چکر کے \عددی{ z  } جز اور منبع سے فاصلہ کی پیمائش کرتے ہیں (یاد رہے کہ یہ ہم آہنگ مشہودات ہیں) ایک ذرے کا رداس  \عددی{  r } پر اور ہم میدان ہونے کا کثافت احتمال کیا ہو گا؟ 
\end{enumerate}
\انتہا{سوال}
\ابتدا{سوال}
\begin{enumerate}[a.]
\item  دکھائیں کہ ایک تفاعل \عددی{  f(\phi) } جس کو؟؟؟؟؟ تسلسل میں پھیلایا جا سکتا ہے، کے لیے درج ذیل ہو گا 
\begin{align*}
    f(\phi+\varphi)\equiv e^{\frac{\iota L_z\varphi}{\hslash}}f(\phi)
\end{align*} 
(جہاں \عددی{  \varphi } اختیاری زاویہ ہے) ۔اسی کی بنا \عددی{  L_z/\hbar } کو \عددی{  z } کے گرد گھومنے کا پیداکار کہتے ہیں۔ اشارہ: مساوات 4.129 استعمال کریں اور سوال 3.39 سے مدد لیں۔ زیادہ عمومی \عددی{  \kvec{L}.\hat{n}/\hslash } ہو گا جو \عددی{ \hat{n}  } کے رخ گھومنے کا پیداکار ہے یعنی \عددی  exp(\iota \kvec{L}.\hat{n}\varphi/\hslash)  کے گرد دائیں ہاتھ سے     زاویہ { \varphi  } گھومنے کا اثر پیدا کرتا ہے۔ چکر کی صورت میں گھومنے کا پیداکار  \عددی\( \kvec{S}\cdot\hat{n}/\hbar \) ہو گا بالخصوص  \(1/2\)  چکر کے لیے
\begin{align}
    \chi'=e^{\iota(\sigma.\hat{n})\varphi/2}\chi
\end{align} ہمیں چکر کاروں کے گھومنے کے بارے میں بتاتی ہے۔
\item محور \عددی{  x-axis } کے لحاظ سے 180 ڈگری گھومنے کو ظاہر کرنے والا \عددی{  (2\times2) } قالب تیار کریں اور دکھائیں کہ یہ ہماری توقعات کے عین مطابق ہمہ میدان \عددی{ (\chi_+)  } کو خلاف میدان \عددی{  (\chi_-) } میں تبدیل کرتا ہے 
\item محور \عددی{  y-axis } کے لحاظ سے 90 ڈگری گھومنے والا قالب تیار کریں اور دیکھیں کہ \عددی{ (\chi_+)  } پر اس کا اثر کیا ہو گا؟ 
\item محور \عددی{  z-axis } کے لحاظ سے 360 زاویہ گھومنے کو ظاہر کرنے والا قالب تیار کریں۔ کیا جواب آپ کی توقعات کے مطابق ہے؟ ایسا نہ ہونے کی صورت میں اس کی مضمرات پر تبصرہ کریں۔
\item درج ذیل دکھائیں 
\begin{align} e^{\iota(\sigma.\hat{n})\varphi/2}=\cos{(\varphi/2)}+\iota(\hat{n}.\sigma)\sin{(\varphi/2)}
\end{align}
\end{enumerate}
\انتہا{سوال}
\ابتدا{سوال}
زاویائی معیار حرکت کے بنیادی تبادلی رشتے (مساوات 4.99) امتیازی اقدار کے عدد صحیح قیمتوں کے ساتھ ساتھ نصف عدد صحیح قیمتوں کی بھی اجازت دیتے ہیں۔ جبکہ مداری زاویائی معیار حرکت کی صرف عدد صحیح قیمتیں پائی جاتی ہیں۔ یوں ہم توقع کریں گے کہ \عددی{ \kvec{L}=\kvec{r}\times\kvec{p}  } کے روپ میں کوئی اضافی شرط ضرور نصف عددی قیمتوں کو خارج کرتا ہو گا۔ ہم \عددی{  a } کو کوئی ایسا مستقل لیتے ہیں جسکا بود لمبائی ہو مثلاً ہائیڈروجن پر بات کرتے ہوئے رداس بوہر درج ذیل حاملین متعارف کرتے ہیں 
\begin{align*}
    q_1=\frac{1}{\sqrt{2}}[x+(a^2/\hslash)p_y] ; p_1\equiv\frac{1}{\sqrt{2}}[p_x-(\hslash/a^2)y];
\end{align*}
\begin{align*}
    q_2\equiv\frac{1}{\sqrt{2}}[x-(a^2/\hslash)p_y];p_2\equiv\frac{1}{\sqrt{2}}[p_x+(\hslash/a^2)y].
\end{align*}
\begin{enumerate}[a.]
\item  تصدیق کریں کہ \عددی{  [q_1,q_2]=[p_1,p_2]=0;[q_1,p_1]=[q_2,p_2]=\iota\hslash } یوں مقام اور معیار حرکت کی باضابطہ تبادلی رشتوں کو  \عددی{  q's } اور  \عددی{  p's } مطمئین کرتے ہیں اور اشاریہ  \عددی{  1 } کے حاملین اشاریہ \عددی{  2 } کے حاملین کے ہم آہنگ ہیں 
\item  درج ذیل دکھائیں 
\begin{align*}
    L_z=\frac{\hslash}{2a^2}(q_1^2-q_2^2)+\frac{a^2}{2\hslash}(q_1^2-q_2^2)
\end{align*}
\item تصدیق کریں کہ ایک ایسا ہارمونی مرتعش جس کی کمیت  
\(m=\hslash/a^2\) 
ہو اور تعدد \عددی{  \omega=1 }  ہو کہ ہر ایک ہیملٹنی \عددی{  H } کے لیے \(L_z=H_1-H_2\) گا۔ 
\item ہم جانتے ہیں کہ ہارمونی مرتعش کے ہیملٹنی کی   امتیازی اقدار\((n+1/2)\hslash\omega\)ہیں جہاں\(n=0,1,2,3,\cdots\) ہو گا (حصہ \حوالہ {  } کے الجبرائی نظریہ میں ہیملٹنی کی روپ اور باضابطہ تبادلی رشتوں سے یہ اخذ کیا گیا) اس کو استعمال کرتے ہوئے یہ اخذ کریں کہ \عددی{  L_z } کے امتیازی اقدار لازماً عدد ہوں گے ۔
\end{enumerate}
\انتہا{سوال}
\ابتدا{سوال}
عمومی حال مساوات 4.139 می\(1/2\) چکر کے \عددی{  S_z } اور \عددی{  S_y } کی کم سے کم عدم یقینیت کا شرط معلوم کریں یعنی \(\sigma_{S_{x}}\sigma_{S_{y}}\geq(\hslash/2)|\langle S_z\rangle|\) میں مساوات کی صورت میں تلاش کریں۔ جواب: عمومیت کھوئے بغیر \عددی{  a } کو حقیقی منتخب کر سکتے ہیں تب عدم یقینیت کی کم سے کم قیمت اس صورت میں حاصل ہو گی \عددی{  b } خالف حقیقی یا خالف خیالی ہو۔
\انتہا{سوال}
\ابتدا{سوال}
کلاسیکی برقی حرکیات میں ایک ذرہ جس کا؟؟؟؟ \عددی{  q } ہو اور جو مقناطیسی میدان \عددی{  \kvec{E} } اور  \عددی{  \kvec{B} } میں سمتی رفتار  \عددی{ \kvec{v}  } کے ساتھ حرکت کرتا ہو، پر قوت عمل کرتا ہے جو لورینز قوت کی مساوات دیتی ہے \begin{align}
    \kvec{F}=q(\kvec{E}+\kvec{v}\times\kvec{B})
\end{align}
اس قوت کو کسی بھی غیر سمتی مخفی توانائی تفاعل کی ڈھلوان کی صورت میں لکھا جا سکتا ہے لہذا مساوات شروڈنگر اپنی اصلی روپ میں (مساوات 1.1) اس کو قبول نہیں کر سکتی ہے تاہم اس کی نفیس روپ 
\begin{align}
    \iota\hslash\frac{\partial\psi}{\partial t}=H\psi
\end{align}
کوئی مسئلہ نہیں کھڑا کرتی ہے۔ کلاسیکی ہیملٹنی درج ذیل ہو گا 
\begin{align}
    H=\frac{1}{2m}(\kvec{p}-q\kvec{A})^2+q\varphi
\end{align}
 جہاں \عددی{ \kvec{A}  }  سمتی مخفی قوہ \(\kvec{B}=\nabla \times \kvec{A}\) اور \عددی{  \varphi } غیر سمتی مخفی قوہ \((\kvec{E}= -\nabla\varphi-\partial\kvec{A}/\partial t)\) ہیں لہٰذا شروڈنگر مساوات میں باضابطہ متبادل \((\kvec{p}\rightarrow((\hslash/\iota)\nabla)\) درج ذیل لکھا جا سکتا ہے۔
\begin{align}
    \iota\hslash\frac{\partial\psi}{\partial t}=[\frac{1}{2m}(\frac{\hslash}{\iota}\nabla-q\kvec{A})^2+q\varphi]\psi
\end{align}
\begin{enumerate}[a.]
\item درج ذیل دکھائیں 
  \begin{align}
    \frac{d\langle r \rangle }{dt}=\frac{1}{m}\langle(\kvec{p}-q\kvec{A})\rangle
\end{align}
\item ہمیشہ کی طرح مساوات 1.32 دیکھیں۔ ہم \عددی{  d\langle \kvec{r}\rangle/dt }  کو \عددی{ \langle \kvec{v} \rangle  } لیتے ہیں ۔ درج ذیل دکھائیں 
\begin{align}
    m\frac{d\langle v \rangle}{dt}=q\langle\kvec{E}\rangle+\frac{q}{2m}\langle(\kvec{p}\times\kvec{B}-\kvec{B}\times\kvec{p})\rangle-\frac{q^2}{m} \langle (\kvec{A}\times\kvec{B})\rangle
\end{align}
\item    بالخصوص موجی اکٹھ کے حجم پر یکساں \عددی{  \kvec{E} } اور \عددی{ \kvec{B}  } میدانوں کی صورت میں درج ذیل دکھائیں 
  \begin{align}
    m\frac{d\langle \kvec{v}\rangle}{dt}=q(\kvec{E}+\langle\kvec{v}\rangle\times\kvec{B}),
\end{align}
  اس طرح \عددی{ \langle\kvec{v\rangle}  } کی توقعاتی قیمت عین لورینز قوت کی مساوات کے تحت حرکت کرے گی جیسا ہم مسئلہ؟؟؟؟؟ کے تحت کرتے ہیں۔
\end{enumerate}
\انتہا{سوال}
\ابتدا{سوال}
( پس منظر جاننے کے لیے سوال 4.59 پر نظر ڈالیں) درج ذیل فرض کریں جہاں \عددی{  B_0 } اور  \عددی{  K } مستقلات ہیں 
\begin{align*}
    \kvec{A}=\frac{\kvec{B_0}}{2}(x_{\hat{j}}-y_{\hat{i}})
\end{align*}
;
\begin{align*}
    \varphi=Kz^2
\end{align*} 
\begin{enumerate}[a.]
\item  میدان \عددی{  E } اور \عددی{  B } تلاش کریں 
\item ان میدانوں میں جن کی کمیت \عددی{  m } اور بار \عددی{ q  } ہوں کے ساکن حالات کی اجازتی توانائیاں تلاش کریں۔ جواب
\begin{align}
    E(n_1,n_2)=(n_1+\frac{1}{2})\hslash\omega_1+(n_2+\frac{1}{2})\hslash\omega,       (n_1,n_2=0,1,2,3,\cdots)
\end{align}
جہاں \(\omega_1=q\kvec{B_0}/m\) اور   \( \omega_2\equiv\sqrt{2q\kvec{K}\m}  \) ہو گا۔ تبصرہ: \( \kvec{K} = 0 \) کی صورت میں یہ سائیکلوٹران حرکت کا کوانٹم مماثل ہو گا۔ کلاسیکی سائیکلوٹران تعدد \عددی{  \omega_1 } ہو گا اور یہ \عددی{  z } رخ میں آزاد ذرہ ہے۔ اجازتی توانائیاں \((n_1+\frac{1}{2})\hslash\omega\) ہوں گی جنہیں لانڈاؤ سطحیں کہتے ہیں۔
\end{enumerate}
\انتہا{سوال}
\ابتدا{سوال}
( پس منظر جاننے کی خاطر سوال 4.59 پر نظر ڈالیں) کلاسیکی برقی حرکیات میں مخفی قوہ \عددی{ \kvec{A}  } اور \عددی{  \varphi } یکتا طور پر تعین نہیں کیے جا سکتے ہیں، طبی مقداریں میدان \عددی{  \kvec{E} } اور \عددی{ \kvec{B}  } ہیں 
\begin{enumerate}[a.]
\item  دکھائیں کہ مخفی قوہ 
\begin{align}
    \varphi'\equiv\varphi-\frac{\partial\Lambda}{\partial t},  \kvec{A}'\equiv\kvec{A}+\nabla\Lambda
\end{align}
(جہاں مقام اور وقت کا \عددی{  \Lambda } ایک اختیاری حقیقی تفاعل ہے) بھی وہی میدان \عددی{  \varphi } اور \عددی{  \kvec{A} } دیتے ہیں۔ مساوات 4.210 گیج تبادلہ کہلاتی ہے جبکہ ہم کہتے ہیں کہ یہ نظریہ گیج غیر متغیر ہے۔
\item کوانٹم میکانیات میں مخفی قوہ کا کردار زیادہ براہ راست پایا جاتا ہے اور ہم جاننا چاہیں گے کہ ایا یہ نظریہ گیج متغیر رہتا ہے یا نہیں؟ دکھائیں کہ 
\begin{align}
    \Psi'\equiv e^{\iota q \Lambda/\hslash}\Psi
\end{align}
شروڈنگر مساوات ( مساوات 4.20) کو گیج تبادلہ مخفی قوہ  \عددی{  \varphi' } اور \عددی{ \kvec{A}  } لیتے ہوئے مطمئن کرتا ہے۔ چونکہ \عددی{ \Psi  } اور \عددی{ \Psi'  } میں صرف زاویائی جز کا فرق پایا جاتا ہے لہٰذا یہ ایک ہی طبی حال کو ظاہر کرتے ہیں اور یوں یہ نظریہ گیج غیر متغیر ہو گا۔ مزید معلومات کے لیے حصہ 10.2.3 سے رجوع کیجئے گا۔
\end{enumerate}
\انتہا{سوال}



