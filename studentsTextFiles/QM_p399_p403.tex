\documentclass[leqno, b5paper]{khalid-urdu-book}
\begin{document}
\حصہ{جزوی موج تجزیہ}
\جزوحصہ{اصول و ضوابط}
ہم نے باب 4 میں دیکھا کہ کروی تشاکلی مخفیہ \عددی{V(r)} کے لیئے مساوات شروڈنگر قابلِ علیحدگی حلوں
\begin{align}
	\psi(r, \theta, \phi) = R(r)Y^m_l(\theta, \phi)
\end{align}
کا حامل ہوگا جہاں \(Y_l^m\) کروی ہارمونی مساوات \num{4.32} ہے اور \(u(r) = rR(r)\) رداسی مساوات مساوات \num{4.37} 
\begin{align}
	-\frac{\hbar^2}{2m}\frac{d^2u}{dr^2}+\left[V(r)+\frac{\hbar^2}{2m}\frac{l(l+1)}{r^2}\right]u = Eu
\end{align}
کو متمعن کرتا ہے بہت بڑی \عددی{r} کی صورت میں مخفیہ صفر کو پہنچتا ہے اور مرکز گریز حصہ قابلِ نظرابداز ہوگا۔ لحآظہ درج ذیل لکھا جاسکتا ہے۔
\begin{align*}
	\frac{d^2u}{dr^2} \approx-k^2u
\end{align*}
اس کا عمومی حل درج ذیل ہے
\begin{align*}
	u(r) = Ce^{ikr}+De^{-ikr}
\end{align*}
پہلا جز رخصتی کروی موج کو اور دوسرا جز آمدی موج کو ظاہر کرتا ہے پھر ہے کہ موج  بکھرائو کے لیئے ہم \(D=0\) چاہتے ہیں۔ یوں بہت بڑی \عددی{r} کی صورت میں درج ذیل ہوگا
\begin{align*}
	R(r)\sim\frac{e^{ikr}}{r}
\end{align*}
جہ ہم گزشتہ حصہ میں طبعی وجوہات سے اغز کر چکے ہیں مساوات  \num{11.12}۔

یہ بہت بڑی \عددی{r} کے لیئے تھا یا یہ کہنا زیادہ درست ہوگا کہ \(kr>>1\) کے لیئے تھا جسی بصریات میں خطہ اشاعی کہیں گے۔ یک بُعدی نظریہ بکھرائو کی طرح ہم یہاں فرض کرتے ہیں کہ مخفیہ مکامی ہے جس سے  ہمارا مراد یہ ہوگا کہ کسی متناہی بکھرائو خطہ کے باہر یہ تقریباً صفر ہوگا شکل \num{11.6}۔ درمیانی خطہ میں جہاں \عددی{V} کو رد کیا جا سکتا ہے لیکن مرکز گریز جز کو نظرانداز نہیں کیا جا سکتا رداسی مساوات درج ذیل روپ اختیار کرتی ہے۔  
\begin{align}
	\frac{d^2u}{dr^2}-\frac{l(l+1)}{r^2}u = -k^2u
\end{align}
جس کا عمومی حل مساوات \num{4.45} کروی بیسل تفاعلات کا خطی جوڑ ہوگا
\begin{align}
	u(r) = Arj_l(kr)+Brn_l(kr)
\end{align}
لیکن نہ ہی \عددی{j_l} جو سائن تفاعل کی طرح ہے اور نہ ہی \عددی{n_l} جو متعمم کوسائن کی طرح ہے کسی رخصتی یا آمدی موج کو ظاہر نہیں کرتے ہیں۔ ہمیں یہاں \(e^{ikr}\) اور \(e^{-ikr}\) طرز کے خطی جوڑ درکار ہوں گے جنہیں کروی ہینکل تفاعلات کہتے ہیں
\begin{align}
	h^{(1)}_l(x)\equiv j_l(x)+in_l(x);\quad h^{(2)}_l(x)\equiv j_l(x)-in_l(x)
\end{align}
جدول \num{11.1} میں چند ابتدائی کروی ہینکل تفاعلات پیش کیئے گئے ہیں۔
\begin{table}[h!]
\centering
\caption{کروی ہینکل تفاعلات $h_l^{(1)}(x)$ اور $h_l^{(2)}(x)$}
\label{table:1}
\begin{tabular}{|c c c|}
\hline
$h_0^{(1)} = -i\frac{e^{ix}}{x}$ & & $h_0^{(2)} = i\frac{e^{-ix}}{x}$ \\
$h_1^{(1)} = \left(-\frac{i}{x^2}-\frac{1}{x}\right)e^{ix}$ & & $h_1^{(2)} = \left(\frac{i}{x^2}-\frac{1}{x}\right)e^{-ix}$ \\
$h_2^{(1)} = \left(-\frac{3i}{x^3}-\frac{3}{x^2}+\frac{i}{x}\right)e^{ix}$ & & $h_2^{(2)} = \left(\frac{3i}{x^3}-\frac{3}{x^2}+\frac{i}{x}\right)e^{-ix}$\\
 & $\begin{matrix}
 	h_l^{(1)}\rightarrow\frac{1}{x}(-i)^{l+1}e^{ix} \\
 	h_2^{(2)}\rightarrow\frac{1}{x}(i)^{l+1}e^{-ix}
 \end{matrix}
	\Bigg\}x>>1\text{\RL{کے لیئے}}$ & \\
\hline
\end{tabular}
\end{table}
بڑی \عددی{r} کی صورت میں \(h_l^{(1)}(kr)\) جسے ہینکل تفاعک کا پہلا قسم کہتے ہیں \(e^{ikr}/r\) کے لحاظ سے تبدیل ہوتا ہے جبکہ \(h_l^{(2)}(kr)\) ہینکل تفاعل کی دوسری قسم \(e^{-ikr}/r\) کے لحاظ سے تبدیل ہوگا۔ یوں رخصتی امواج کے لیئے ہمیں کروی ہینکل تفاعلات کی پہلی قسم درکار ہوگی:
\begin{align}
	R(r)\sim h^{(1)}_l(kr)
\end{align}
اس طرح خطہ بکھرائو کے باہر جہاں \(V(r) = 0\) ہوگا بلکل ٹھیک تفاعل موج درج ذیل ہوگا
\begin{align}
	\psi(r, \theta, \phi) = A\left\{e^{ikz}+\sum_{l, m}C_{l, m}h^{(1)}_l(kr)Y^m_l(\theta, \phi)\right\}
\end{align}
اس کا پہلا جز آمدی مستوی موج ہے جبکہ مجموعہ جس کے عددی سر \عددی{C_{l, m}} ہے موج بکھرائو کو ظاہر کرتا ہے۔ چونکہ ہم فرض کر چکے ہیں کہ مخفیہ کروی تشاکلی ہے لحاظہ تفاعل موج \(\phi\) کا تابع نہیں ہوسکتا ہے۔ یوں صرف وہ اجزاء باقی رہیں گے جن میں \(m=0\) ہو یاد رہے \(Y_l^m\sim e^{im\phi}\) اب مساوات \num{4.27} اور \num{4.32} سے درج ذیل ہوگا
\begin{align}
	Y^0_l(\theta, \phi) = \sqrt{\frac{2l+1}{4\pi}}P_l(\cos\theta)
\end{align}
جہاں \عددی{l} ویں لیژانڈر کثیر رکنی کو \عددی{P_l} کو ظاہر کرتا ہے۔ روایتی طور پر \(C_{l, 0}\equiv i^{l+1}k\sqrt{4\pi(2l+1)}a_l\) لکھ کر عددی سروں کی تعریف یوں کی جاتی ہے:
\begin{align}
	\psi(r, \theta) = A\left\{e^{ikz}+k\sum_{l=0}^{\infty}i^{l+1}(2l+1)a_lh_l^{(1)}(kr)P_l(\cos\theta)\right\}
\end{align}
آپ کچھ ہی دیر میں دیکھیں گے کہ یہ مخصوص علامتیت کیوں بہتر ہے \عددی{a_l} کو \عددی{l} واں حیطہ جزوی موج کہتے ہیں۔

اب بہت بڑی \عددی{r} کی صورت میں ہینکل تفاعل \((-i)^{l+1}e^{ikr}/kr\) جدول \num{11.1} کے لحاظ سے تبدیل ہوگا لحاظہ درج ذیل ہوگا 
\begin{align}
	\psi(r, \theta)\approx A\left\{e^{ikz}+f(\theta)\frac{e^{(ikr)}}{r}\right\}
\end{align}
جہاں \(f(\theta)\) درج ذیل ہے
\begin{align}
	f(\theta) = \sum_{l=0}^{\infty}(2l+1)a_lP_l(\cos\theta)
\end{align}
یہ مساوات \num{11.12} میں میں پیش کی گئی عمومی ساخت کے اصول موضوعہ کی تصدیق کرتا ہے اور ہمیں دیکھاتا ہے کہ جزوی موج حیطوں \عددی{a_l} کی صورت میں حیطہ بکھرائو \(f(\theta)\) کس طرح حاصل ہوگا تفریقی عمودی تراش درج ذیل ہوگا
\begin{align}
	D(\theta) = \abs{f(\theta)}^2 = \sum_{l}\sum_{l^\prime}(2l+1)(2l^\prime+1)a^*_la_{l^\prime}P_l(\cos\theta)P_{l^\prime}(\cos\theta)
\end{align}
اور کل عمودی تراش درج ذیل ہوگا
\begin{align}
	\sigma=4\pi\sum_{l=0}^{\infty}(2l+1)\abs{a_l}^2
\end{align}
زاویائی تکمل کو حل کرنے کے لیئے میں نے لیژانڈر کثیررکنیوں کی عمودیت مساوات \num{4.34} استعمال کی۔
\جزوحصہ{لایاعمل}
زیرِ غور مخفیہ کے لیئے جزوی موج حیطوں \عددی{a_l} کا تعین کرنا باقی ہے۔ اندرونی خطہ جہاں \عددی{V(r)} غیر صفر ہے میں مساوات شروڈنگر کو حل کر کے اسے بیرونی حل مساوات \num{11.23} کے ساتھ مناسب سرحدی شرائط استعمال کرتے ہوئے ملانے سے ایسا کیا جاسکتا ہے۔ مثلا صرف اتنا ہے کہ میں نے بکھرائو موج کے لیئے کروی محدد جبکہ آمدی موج کے لیئے کارتیسی محدد استعمال کیئے ہیں۔ ہمیں تفاعل موج کو ایک جیسی علامتوں میں لکھنا ہوگا۔

یقیناً \(V=0\) کے لیئے مساوات شروڈنگر کو \(e^{ikz}\) متمعن کرتا ہے۔ ساتھ ہی میں دلائل پیشکر چکا ہوں کہ \(V=0\) کے لیئے مساوات شروڈنگر کا عمومی حل درج ذیل روپ کا ہوگا
\begin{align*}
	\sum_{l, m}\left[A_{l, m}j_l(kr)+B_{l, m}n_l(kr)\right]Y_l^m(\theta, \phi)
\end{align*}
یوں بلخصوص \(e^{ikz}\) کو اس طرح بیان کرنا ممکن ہونا چاہیئے اب مبدہ پر \(e^{ikz}\) متناہی ہے لحاظہ نیومن تفاعلات کی اجازت نہیں ہوگی \(r=0\) پر \(n_l(kr)\) بے قابو بڑھتے ہیں اور چونکہ \(z=r\cos\theta\) میں کوئی \(\phi\) نہیں پایا جاتا ہے لحاظہ صرف \(m=0\) اجزاء ہوں گے۔ مستوی موج کی کروی امواج کی صورت میں سریحاً پھیلائو کلیہ ریلے دیتی ہے۔
\begin{align}
	e^{ikz} = \sum_{l=0}^{\infty}i^l(2l+1)j_l(kr)P_l(\cos\theta)
\end{align}
اس کو استعمال کرتے ہوئے بیرونی خطہ میں تفاعل موج کو صرف \عددی{r} اور \(\theta\) کی صورت میں پیش کیا جا سکتا ہے
\begin{align}
	\psi(r, \theta) = A\sum_{l=0}^{\infty}i^l(2l+1)\left[j_l(kr)+ika_lh_l^{(1)}(kr)\right]P_l(\cos\theta)
\end{align}
\end{document}
