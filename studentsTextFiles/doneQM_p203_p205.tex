
\جزوحصہ{بوزان اور فرمیون}
فرض کریں ذرہ ایک یک ذرہ حال \عددی{\psi_a(r)} اور ذرہ دو حال \عددی{\psi_b(r)} میں پایا جاتا ہے۔ یاد رہے کہ یہاں میں چکر کو نظرانداز کر رہا ہوں ایسی صورت میں \عددی{\psi(r_1, r_2)} سادہ حاصلِ ضرب ہوگا
\begin{align}
	\psi(r_1, r_2)=\psi_a(r_1)\psi_b(r_2)
\end{align}
ایسا کہتے ہوئے ہم یہ فرض کر رہے ہیں کہ ہم ان ذرات کو علیحدہ علیحدہ پہچان سکتے ہیں ورنہ یہ کہنا کہ ذرہ ایک حال \عددی{\psi_a} میں اور ذرہ دو حال \عددی{\psi_b} میں ہے پیمانی ہوتا اور ہم بغیر جانے کے کونسا ذرہ ایک اور کونسا ذرہ دو ہے یہ کہتے کہ ایک ذرہ \عددی{\psi_a} میں اور دوسرا ذرہ \عددی{\psi_b} میں پایا جاتا ہے۔ کلاسیکی میکانیتا میں یہ ایک بیوقفانہ اعتراض ہوتا۔ اصولاً ایک ذرے کو سرخ رنگ اور دوسرے کو نیلا رنگ دیکر آپ انہیں ہر وقت پہچان سکتے ہیں۔ کوانٹم میکانیات میں صورتِ حال بنیادی طور پر مختلف ہے۔ آپ کسی الیکٹران کو سرخ رنگ نہیں دے سکتے  اور نہ ہی اس پر کوئی پرچی چسپاں کر سکتے ہیں حقیقت یہ ہے کہ تمام الیکٹران بلکل یکساں ہوتے ہیں جبکہ کلاسیکی اشیاء اتنی یکسانیت کبھی نہیں رکھ سکتے ہیں۔ ایسا نہیں ہے کہ ہم الیکٹرانوں کو پہچاننے ساے قاصر ہیں بلکہ حقیقت یہ ہے کہ یہ الیکٹران اور وہ الیٹران کونٹم میکانیات مین بے معنی ہیں ہم صرف ایک الیکٹران کی بات کر سکتے ہیں۔ اصولی طور پر غیر ممیہز ذرات کی موجودگی کو کوانٹم میکانیات خوش اسلوبی سے سموتی ہے۔ ہم ایک ایسا غیر مشرود	تفاعل ومج تیار کرتے ہیں جہ اس کیبات نہیں کرتا کہ کون ذرہ کس  حال میں ہپے ایسا دو طریقوں سے کیا جاسکتا ہے۔
\begin{align}
	\psi\pm(r_1, r_2)=A[\psi_a(r_1)\psi_b(r_2)\pm\psi_b(r_1)\psi_a(r_2)]
\end{align}
یوں یہ ذرہ دو اقسام کے یکساں ذرات کا حامل ہوگا بوزان جن کے لیئے ہم مثبت علامت استعمال کرتے ہیں اور فرمیون جن کے لیئے ہم منفی علامت استعمال کرتے ہیں۔ بوزان کی مثال فوٹان اور میزون ہے جبکہ فرمیون کی مثال پروٹان اور ایلکٹران ہے ایسے ہے کہ
\begin{align}
	\begin{cases}
		\text{\RL{عدد صحیح چکر کے تمام ذرات بوزان جبکہ}} \\
		\text{\RL{نصف عدد صحیح چکر کے تمام ذرات فرمیون ہوں گے}}
	\end{cases}
\end{align}
چکر اور شماریات کے مابین یہ تعلق جیسا ہم دیکھیں گے فرمیونز اور بوزانز کی شماریاتی خواس ایک دوسرے سے بہت مختلف ہتے ہیں کو اضافی کوانٹم میکانیات میں ثابت کیا جا سکتا ہے۔ غیر اضافی نظریہ میں اس کو ایک مسلمہ لیا جاتا ہے۔ 

اس سے بلخصوص اب یہ اخز کر سکتے ہیں کہ دو یکساں فرمیونز مثلاً سو الیکٹران ایک ہی حال کے مکین نہیں ہوسکتے ہیں۔ اگر \عددی{\psi_a=\psi_b} ہو تب
\begin{align*}
	\psi_{-}(r_1, r_2)=A[\psi_a(r_1)\psi_a(r_2)-\psi_a(r_1)\psi_a(r_2)]=0
\end{align*}
کی بنا کوئی موج تفاعل نہیں ہوگا۔ یہ مشہور نتیجہ پولی کا اخراجی اصول کہلاتا ہے۔ یہ کوئی عجیب مفروضہ نہیں ہے جو صرف الیکتران پر لاگو ہوتا ہے بلکہ یہ دو ذراتی تفاعلی امواج کی تیاری کے قواعد کا ایک نتیجہ ہے جسکا اطلاق تمام یکساں فرمیونز پر ہوگا۔

میں نے دلائل پیش کرنے کے نقطہ نظر سے یہ فرض کیا تھا کہ ایک ذرہ حال \عددی{\psi_a} میں اور دوسرا حال \عددی{\psi_b} میں پایا جاتا ہے لیکن اس مسئلہ کو زیادہ عمومی اور زیادہ نفیس طریقے سے وضح کیا جاسکتا ہے۔ ہم عامل مبادلہ \عددی{P} متعارف کرتے ہیں جو دو ذرات کا باہمی مبادلہ کرتا ہے
\begin{align}
	Pf(r_1, r_2)=f(r_2, r_1)
\end{align}
صاف ظاہر ہے کہ \عددی{P^2=1} ہوگا لحاظہ تصدیق کیجیئے گا کہ \عددی{P} کے امتیازی اقدار \عددی{\pm1} ہوں گے۔ اب اگر دو ذرات یکساں ہوں تب لاظمی ہے کہ ہیملٹونی ان کے ساتھ ایک جیسا رویہ برتھے گا \عددی{m_1=m_2} اور \عددی{V(r_1, r_2)=V(r_2, r_1)} اس طرح \عددی{P} اور \عددی{H} ہم اہنگ مشود ہوںگے
\begin{align}
	[P, H]=0
\end{align}
لحاظہ ہم دونوں کے بیک وقت امتیازی حالات کے تفاعلوں کا مکمل سلسلہ معلوم کر سکتے ہیں۔ دوسرے لفظوں میں ہم زیرِ مبادلہ 
\begin{align}
	\psi(r_1, r_2)=\pm\psi(r_2, r_1)
\end{align}
مساوات شروڈنگر کے ایسے حل تلاش کرسکتے ہیں جو یا تشاکلی امتیازی قدر \عددی{+1} یا غیر تشاکلی امتیازی قدر \عددی{-1} ہوں۔ مزید ایک نطام جو اس حال سے آغاز کرے اس یحال میں برقرار رہتا ہے یکساں ذرات کا ایک نیا قائدہ جس کو میں ضرورت تشاکل کہتا ہوں کے تحت تفاعل موج کو \حوالہء{مساوات \num{5.14}} پر صرف پورا اُترنے کی ضرورت نہیں بلکہ اس پر لاظم ہے کہ وہ اس مساوات کو متمعن کرتا ہو۔ یہاں بوزون کے لیئے مثبت علامت اور فرمیونز کے لیئے منفی علامت استعمال ہوگا۔ یہ اییک عمومی فکرہ ہے جس کی \حوالہء{مساوات \num{5.10}} ایک مخصوص صورت ہے۔

