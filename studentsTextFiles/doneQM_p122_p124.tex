

انگریزی میں قوسین کو  کٹ  کہتے ہیں۔ڈیراک نے اندرونی ضرب \عددی{ \langle \alpha | \beta \rangle} میں   کٹ  کی علامت کو دو ٹکڑوں میں تقسیم کر کے پہلے حصہ کو برا،    \عددی{\langle \alpha |}،  اور دوسرے حصے کو کٹ،   \عددی{| \beta \rangle} کا نام دیا۔ ان میں سے موخر الذکر ایک سمتیہ ہے، لیکن  اول الذکر کیا ہے؟ یہ اس لحاظ سے سمتیات کا ایک خطی تفاعل ہے کہ اس کے دائیں جانب ایک سمتیہ جوڑنے سے ایک( مخلوط)  عدد حاصل ہوتا ہے جو اندرونی ضرب ہو گا۔ ایک  عامل کے دائیں جانب ایک سمتیہ جوڑنے سے ایک نیا سمتیہ حاصل ہوتا ہے جبکہ ایک براہ کے دائیں جانب ایک سمتیہ جوڑنے سے ایک عدد حاصل ہوتا ہے۔ ایک تفاعلی فضا میں براہ کو تکمل کرنے کی ہدایت تصور کیا جا سکتا ہے 
\begin{align*}
\langle f | = \int f^{*} [ \cdots ] dx
\end{align*}
جہاں چکور قوسین میں وہ تفاعل پر کیا جائے گا جو براہ کے دائیں ہاتھ کٹ میں موجود ہو گا۔ ایک متناہی آبادی سمتی فضا میں جہاں سمتیات کو قطاروں 
\begin{align}
| \alpha \rangle = \begin{pmatrix}
a_{1} \\ a_{2} \\ \vdots \\ a_{n} 
\end{pmatrix}
\end{align}
کی صورت میں بیان کیا گیا ہو ان کے مطابقتی براہ ایک صف سمتیہ ہو گا
\begin{align}
langle \alpha | = ( a_{1}^{*}a_{2}^{*} \cdots a_{n}^{*})
\end{align}
تمام براہ کو اکٹھا کرنے سے ایک دوسرا سمتی فضا حاصل ہو گا جس کو دوہری فضا کہتے ہیں۔
براہ کی ایک علیحدہ وجود کا تصور ہمیں طاقتور اور خوبصورت علامتیت کا موقع فراہم کرتی ہے( اگرچہ اس کتاب میں ان سے فائدہ نہیں اٹھایا جائے گا) ۔ مثال کے طور پر اگر سمتیہ \عددی{| \alpha \rangle} معمول پر لانے کے قابل سمتیہ ہو تب عامل 
\begin{align}
\hat{P} \equiv | \alpha \rangle \langle \alpha | 
\end{align}
کسی بھی سمتیہ کا وہ حصہ جو سمتیہ \عددی{       | \alpha \rangle} کے ساتھ ساتھ پایا جاتا ہو، کو منتخب کرتا ہے 
\begin{align*}
\hat{P} | \beta \rangle = \langle \alpha | \beta \rangle | \alpha \rangle ;
\end{align*}
ہم اس کو سمتیہ \عددی{| \alpha \rangle} کے احاطہ کیے گئے یک بعدی ذیلی فضا پر عامل تظلیل کہتے ہیں اگر \عددی{ | e_{n} \rangle } غیر مسلسل معیاری عمودی اساس
\begin{align}
\langle e_{m} | e_{n} \rangle = \delta_{mn},
\end{align}
ہو تب درج ذیل ہو گا 
\begin{align}
\sum_{n} | e_{n} \langle \rangle e_{n} | = 1 
\end{align}
جو عامل مماثل ہے۔ چونکہ کسی بھی سمتیہ سمتیہ \عددی{| \alpha \rangle} پر عمل کرتے ہوئے یہ عامل اساس\عددی{|e_{n} \rangle} 
میں سمتیہ \عددی{| \alpha \rangle} کی پھیلاؤ پھر سے حاصل کرتا ہے 
\begin{align}
\sum_{n} | e_{n} \rangle \langle e_{n} | \alpha \rangle = | \alpha \rangle 
\end{align}
اسی طرح اگر \عددی{| e_{z} \rangle} ایک ڈیراک معیاری عمودی شدہ استمراری اساس ہو 
\begin{align}
\langle e_{z} | e_{z^{'}} = \delta ( z-z^{'})
\end{align}
تب درج ذیل ہو گا 
\begin{align}
\int | e_{z} \rangle \langle e_{z} | dz = 1
\end{align}
مکمل ہونے کو بیان کرنے کا بہترین طریقہ مساوات \حوالہء{   3.91 } اور مساوات \حوالہء{   3.94 } دیتی ہے۔
% Problem 3.21 
\ابتدا{سوال}
دکھائیں کہ عاملین تظلیل کے لیے\عددی{\hat{P}^{2} = \hat{P}}ہو گا۔ \عددی{\hat{P}} کے امتیازی اقدار تعین کریں اور اس کے امتیازی سمتیات کے خواص  بیان کریں۔
\انتہا{سوال}
% Problem 3.22
\ابتدا{سوال}
 ایک معیاری عمودی اساس \عددی{| 1 \rangle }، \عددی{| 2 \rangle } ، \عددی{| 3 \rangle } کا احاطہ کیا گیا ایک تین آبادی فضا پر غور کریں۔کٹ \عددی{| \alpha \rangle } اور \عددی{| \beta \rangle } درج ذیل ہیں 
\begin{align*}
| \alpha \rangle = i | 1 \rangle -2|2\rangle -i|3\rangle , \quad | \beta \rangle = i|1\rangle +2|3\rangle 
\end{align*}
\begin{enumerate}[a.]
\item  \عددی{ \langle \alpha |}   \عددی { \langle \beta |} کو دوہری اساس \عددی{\langle 1 |} ،\عددی{ \langle 2 |}، \عددی{ \langle 3 |} کی صورت میں تیار کریں 
\item  \عددی{ \langle \alpha | \beta \rangle} اور \عددی{  \langle \beta | \alpha \rangle} تلاش کرتے ہوئے \عددی{ \langle \beta | \alpha \rangle = \langle \alpha | \beta \rangle^{*}}  کی تصدیق کریں
\item ان اساس میں عامل \عددی{  \hat{A} \equiv | \alpha \rangle \langle \beta | } کے قالب کے نو ارکان تلاش کر کے قالب \عددی{ \kvec{A}} تیار کریں کیا یہ ہرمیشی ہیں؟
\end{enumerate} 
\انتہا{سوال}
% Problem 2.23
\ابتدا{سوال}
کسی دو سطحی نظام کا ہیملٹنی درج ذیل ہے 
\begin{align*}
\hat{H} = \epsilon ( | 1 \rangle \langle 1 | - |2\rangle \langle 2 | + | 1 \rangle \langle 2 | + | 2 \rangle \langle 1 | )
\end{align*}
جہاں \عددی{ | 1 \rangle , | 2 \rangle } معیاری عمودی اساس ہے اور \عددی{ \epsilon} ایک عدد ہے جس کے آباد توانائی کے آباد ہیں۔ اس کے امتیازی اقدار اور \عددی{| 1 \rangle } اور \عددی{| 2 \rangle } کے خطی جوڑ کی صورت میں معمول شدہ امتیازی تفاعل تلاش کریں۔ اس اساس کے لحاظ سے \عددی{\hat{H}} کو کون سا قالب  \عددی{ \kvec{H}} ظاہر کرے گا؟ 
\انتہا{سوال}
%Problem 2.24
\ابتدا{سوال}
 فرض کریں کہ ایک عامل \عددی{\hat{Q}} معیاری عمودی امتیازی تفاعلات  کا ایک مکمل سلسلہ پایا جاتا ہے 
\begin{align*}
\hat{Q}|e_{n} \rangle = q_{n} | e_{n} \rangle \quad (n = 1,2,3,\cdots )
\end{align*}
دکھائیں کہ \عددی{\hat{Q}} کو اس کے طیفی  اجزاء کی صورت میں لکھا جا سکتا ہے 
\begin{align*}
\hat{Q} = \sum_{n} q_{n} | e_{n} \rangle \langle e_{n} |
\end{align*}
اشارہ: تمام ممکنہ سمتیات پر ایک عامل کے عمل سے ایک عامل کو جانچا جاتا ہے لہٰذا کسی بھی سمتیہ \عددی{| \alpha \rangle} کے لیے آپ کو درج ذیل دکھانا ہو گا 
\begin{align*}
\hat{Q} | \alpha \rangle = \left\{ \sum_{n} q_{n} | e_{n} \rangle \langle e_{n} | \right\} | \alpha \rangle 
\end{align*}
\انتہا{سوال}

