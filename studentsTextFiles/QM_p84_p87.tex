\documentclass[leqno, b5paper]{khalid-urdu-book}
\begin{document}
\ابتدا{سوال}

(الف) دیکھائیں کہ لامتناہی چکور کنواں میں ایک ذرہ کی تفاعل موج کوانٹائی تجدیدی وقت \(T = 4ma^2/\pi\hbar\) کے بعد دوبارہ اپنی اصل روپ میں واپس آتا ہے۔ ئعنی کسی بھی حال کے لیئے نہ صرف ساکن حال کے لیئے نہ صرف ساکن حال کے لیئے \(\psi(x, T) = \psi(x, 0)\)۔

(ب) دیواروں سے ٹکرا کر دائیں سے بائیں اور بائیں سے دائیں حرکت کرتے ہوئے ایک ذرہ جس کی توانائی \عددی{E} ہو کا کلاسیکی تجدیدی وقت کیا ہوگا؟

(ج) کس توانائی کے لیئے یہ دو تجدیدی اوقات ایک دوسرے کے برابر ہیں؟  
\انتہا{سوال}
\ابتدا{سوال}
ایک ذرہ جس کی کمیت \عددی{m} ہو درج ذیل مخفی کوہ میں پایا جاتا ہے
\begin{align*}
	V(x)=
	\begin{cases}
		\infty & (x<0) \\
		-32\hbar^2/ma^2 & (0\leq x\leq a) \\
		0 & (x>a) 
	\end{cases}
\end{align*}

(الف) اس کے مقید حلوں کی تعداد کیا ہوگی؟

(ب) مقید حال میں سب سے زیادہ توانائی کی صورت میں کنواں کے باہر یعنی \((x>a)\) پر ذرہ پائے جانے کا احتمال کیا ہوگا؟

جواب: \num{0.542} اگرچہ یہ کنواں میں مقید ہے لیکن کنواں سے باہر اور کنواں کے اندر اسکی موجودگی کا امکان ایک جیسا ہے۔  
\انتہا{سوال}
\ابتدا{سوال}
ایک ذرہ جس کی کمیت \عددی{m} ہے ہارمونی مرتعش کی مخفی کوہ میں درج ذیل حال سے ابتدا کرتا ہے۔
\begin{align*}
	\psi(x, 0) = A(1-2\sqrt{\frac{m\omega}{\hbar}}x)^2e^{-\frac{m\omega}{2\hbar}x^2}
\end{align*}
جہاں \عددی{A} کوئی مستقل ہے۔

(الف) توانائی کی توقعاتی قیمت کیا ہے؟

(ب) مستقبل کے لمحہ \عددی{T} پر تفاعل موج درج ذیل ہوگا۔
\begin{align*}
	\psi(x, T) = B(1+2\sqrt{\frac{m\omega}{\hbar}}x)^2e^{-\frac{m\omega}{2\hbar}x^2}
\end{align*}
جہاں \عددی{B} کوئی مستقل ہے۔ لمحہ \عددی{T} کی کم سے کم ممکنہ قیمت کیا ہوگی؟
\انتہا{سوال}
\ابتدا{سوال}
درج ذیل نصف ہارمونی مرتعش کی اازتی توانائیاں تلاش کریں
\begin{align*}
	V(x)=
	\begin{cases}
		(1/2)m\omega^2x^2 & x>0 \text{\RL{کے لیئے}} \\
		\infty & x<0 \text{\RL{کے لیئے}}
	\end{cases}
\end{align*}
مثلاً ایک ایسا سپرنگ جس کو کھینچا تو جاسکتا ہے لیکن اسے دبایا نہیں جا سکتا۔ اشارہ: اس کو حل کرنے کے لیئے آپ کو ایک بار اچھی طرح سوچنا پڑے گا جبکہ حقیقی احساب بہت کم درکار ہوگی۔
\انتہا{سوال}
\ابتدا{سوال}
آپ نے سوال \num{2.22} میں ساکن گوسی ذرہ پرندہ موج کا تجزیہ کیا۔ اب ابتدائی تفاعل موج
\begin{align*}
	\psi(x, 0) = Ae^{-ax^2}e^{ilx}
\end{align*}
جہاں \عددی{l} ایک حقیقی مستقل ہے سے شروع کرتے ہوئے متحرک گوسی پرندہ موج کے لیئے وہی مسئلہ دوبارہ حل کریں۔
\انتہا{سوال}
\ابتدا{سوال}
مدہ پر درج ذیل لامتناہی چکور کنواں جس کے وسط پر ڈیلٹا تفاعل رکاوٹ ہو کے لیئے وقت کا غیر تابع شروڈنگر مساوات حل کریں
\begin{align*}
	V(x)=
	\begin{cases}
		\alpha\delta(x) & -a<x<+a \text{\RL{کے لیئے}} \\
		\infty & \abs{x}\geq a \text{\RL{کے لیئے}}
	\end{cases}
\end{align*}
جفت اور طاق تفاعل امواج کو علیحدہ علیحدہ حل کریں۔ انہیں معمول پر لانے کی ضرورت نہیں ہے۔ اجازتی توانائیوں کو اگر ضرورت ہو تو ترسیمی طور پر تلاش کریں۔ انکا موازنہ ڈیلٹا تفاعل کی غیر موجودگی میں متابقتی توانائیوں کے ساتھ کریں۔ طاق حالوں پر ڈیلٹا تفاعل کا کوئی اثر نہ ہونے پر تبصرہ کریں۔ تحدیدیی صورت \(\alpha\rightarrow 0\) اور \(\alpha\rightarrow\infty\) پر تبصرہ کریں۔
\انتہا{سوال}
\ابتدا{سوال}
ایسے دو یا دو سے زیادہ وقت کے غیر تابع شرودنگر مساوات کے حل جنکی توانائی \عددی{E} ایک دوسرے جیسی ہو کو انحطاطی حال کہتے ہیں۔ مثال کے طور پر آزاد ذرہ حالات دوہری انحطاطی ہیں ان میں سے ایک حل دائیں رخ اور دوسرا بائیں رخ حرکت کو ظاہر کرتا ہے۔ لیکن ہم نے ایسے کوئی انحطاطی حل نہیں دیکھے جو معمول پر لانے کے قابل ہوں اور یہ محظ ایک اتفاق نہیں ہے۔ درج ذیل مسئلہ ثابت کریں: ایک بُعد میں کوئی مقید انحطاطی حال نہیں پائے جاتے ہیں۔ اشارہ: فرض کریں \(\psi_1\) اور \(\psi_2\) ایسے دو حل ہوں جن کی توانائی \عددی{E} ایک دوسرے جیسی ہو۔ حل \(\psi_1\) کی شروڈنگر مساوات کو \(\psi_2\) سے ضرب دیں اور اس سے \(\psi_2\) کی شروڈنگر مساوات کو \(\psi_1\) سے ضرب دے کر منفی کریں یوں دیکھائیں کہ \(\psi_2d\psi_1/dx - \psi_1d\psi_2/dx\) ایک مستقل ہوگا۔ اب \(\pm\infty\) پر معمول پر لانے کے قابل ہر حل \(\psi\rightarrow0\) ہوگا اس حقیقت کو استعمال کرتے ہوئے دیکھائیں کہ یہ مستقل در حقیقت صفر ہوگا۔ اس سے آپ یہ نتیجہ  اغذ کر سکتے ہیں کہ \(\psi_2\) دراصل \(\psi_1\) کا مضرب ہوگا لحاظہ یہ دو حل الگ الگ نہپیں ہوسکتے ہیں۔
\انتہا{سوال}
\ابتدا{سوال}
فرض کریں کمیت \عددی{m} کا ایک موتی ایک دائری چھلہ پر بے رگڑ حرکت کرتا ہے چھلے کا محیط \عددی{L} ہے۔ یہ ایک آزاد ذرہ کی طرح ہے لیکن یہاں \(\psi(x+L) = \psi(x)\) ہوگا۔ اس کے ساکن حال تلاش کریں اور انہیں معمول پر لائیں اور انکی مطابقتی اجازتی توانائیاں تلاش کریں۔ آپ دیکھیں گے کہ ہر توانائی \عددی{E_n} کے لیئے دو اپس میں غیر تابع حل پائے جائیں گے جن میں سے ایک گھڑی وار اور دوسرا خلافِ گھڑی حرکت کے لیئے ہوگا جنہیں آپ \(\psi_n^+(x)\) اور \(\psi_n^-(x)\) کہہ سکتے ہیں۔ سوال \num{2.45} کے مسئلہ کو مدِ نظر رکھتے ہوئے آپ اس انحطاط کے بارے میں کیا کہیں گے؟ یہ مسئلہ یہاں کارآمد کیوں نہیں ہے؟
\انتہا{سوال}

\end{document}
