\documentclass[leqno, b5paper]{khalid-urdu-book}
\begin{document}
\ابتدا{مثال}
کوانٹم سخت کرہ بکھرائو۔ درج ذیل فرض کریں
\begin{align}
	V(r)=
	\begin{cases}
		\infty, & r\leq a \text{\RL{کے لیئے}} \\
		0, & r>a \text{\RL{کے لیئے}}
	\end{cases}
\end{align}
سرحدی شرط تب درج ذیل ہوگا
\begin{align}
	\psi(a, \theta) = 0
\end{align}
یوں تمام \عددی{\theta} کے لیئے
\begin{align}
	\sum_{l=0}^{\infty}i^l(2l+1)\left[j_l(ka)+ika_lh_l^{(1)(ka)}\right]P_l(\cos\theta) = 0
\end{align}
ہوگا۔ جس سے درج ذیل حاصل ہوتا ہے \حوالہء{سوال\num{11.3}} 
\begin{align}
	a_l = i\frac{j_l(ka)}{kh_l^(1)(ka)}
\end{align}
بلخصوص کل عمودی تراش درج ذیل ہوگا
\begin{align}
	\sigma=\frac{4\pi}{k^2}\sum_{l=0}^{\infty}(2l+1)\abs{\frac{j_l(ka)}{h_l^{(1)}(ka)}}^2
\end{align}
یہ بلکل درست جواب ہے۔ لیکن اس کو دیکھ کر کچھ زیادہ نہیں کہا جاسکتا ہے آئیں کم توانائی بکھرائو \عددی{ka\ll1} کی تحدید صورت پر غور کریں \عددی{k=2\pi/\lambda} کی بنا یہ کہتا ہے کہ دوری عرصہ کرہ کے رداس سے بہت بڑا ہے۔ \حوالہء{جدول \num{4.4}} سے مدد لیتے ہوئے ہم دیکھتے ہیں کہ چھوٹی \عددی{z} کے لیئے \عددی{n_l(z)} کی مقدار \عددی{j_l(z)} سے بہت زیادہ ہوگی لحاظہ 
\begin{align}
	\frac{j_l(z)}{h_l^{(1)}(z)} &= \frac{j_l(z)}{j_l(z)+in_l(z)}\approx-i\frac{j_l(z)}{n_l(z)}\nonumber \\
	&\approx-i\frac{2^ll!z^l/(2l+1)!}{-(2l)!z^{-l-1}/2^ll!} = \frac{i}{2l+1}\left[\frac{2^ll!}{(2l)!}\right]^2z^{2l+1}
\end{align}
اور درج ذیل ہوگا 
\begin{align*}
	\sigma\approx\frac{4\pi}{k^2}\sum_{l=0}^{\infty}\frac{1}{2l+1}\left[\frac{2^ll!}{(2l)!}\right]^4(ka)^{4l+2}
\end{align*}
چونکہ ہم \عددی{ka\ll1} فرض کر رہے ہیں لحاظہ بلند طاقتیں قابلِ نظرانداز ہوں گی۔ کم توانائی تخمین میں \عددی{l=0} جز بکھرائو میں غالب ہوگا۔ یوں کلاسیکی صورت کے لیئے تفریقی عمودی تراش \عددی{\theta} کا تابع نہیں ہوگا۔ ظاہر ہے کہ کم تواانائی سخت کرہ بکھرائو کے لیئے درج ذیل ہوگا 
\begin{align}
	\sigma\approx4\pi a^2
\end{align}
حیرانی کی بات ہے کہ بکھراؤ عمودی تراش کی قیمت جو میٹرائی عمودی تراش کے چار گنا ہے۔ درحقیقت \عددی{\sigma} کی قیمت کرہ کی کل سطحی رقبہ کے برابر ہے۔ لمبی طولِ موج بکھراؤ کی ایک خاصیت بڑی معاصر جسامت ہے جو بصریات میں بھی ہوگا۔ ایک لحاظ سے یہ امواج کرہ کو چھوتے ہوئے اس کے اُپر سے گزرتے ہیں ناکہ کلاسیکی ذرات کی طرح جنہیں صرف سیدھا دیکھتے ہوئے عمودی تراش نظر آتا ہے۔
\انتہا{مثال}
\ابتدا{سوال}
\حوالہء{مساوات \num{11.32}} سے آغاز کرتے ہوئے \حوالہء{مساوات \num{11.33}} ثابت کریں۔ اشارہ: لیژانڈر کثیررکنی کی عمودیت بروئےکار لاتے ہوئے دیکھائیں کہ \عددی{l} کی مختلف قیمتوں والے عددی سر لاظماً صفر ہوں گے۔
\انتہا{سوال}
\ابتدا{سوال}
کروی ڈیلٹا تفاعل خول:
\begin{align*}
	V(r) = \alpha\delta(r-a)
\end{align*}
سے کم توانائی بکھراؤ کیصور پر غور کریں جہاں \عددی{\alpha} اور \عددی{a} مستقلات ہیں۔ حیطہ بکھراؤ \عددی{f(\theta)} تفریقی عمودی تراش \عددی{D(\theta)} اور کل عمودی تراش \عددی{\sigma} کا حساب کریں۔ ان میں \عددی{ka\ll1} فرض کریں لحاظہ صرف \عددی{l=0} جز خاطرخاہ حصہ ڈالیں گے۔ چیزوں کو آسان بنانے کی خاطر آغاز سے ہی \عددی{l\neq0} والے تمام اجزاء کو نظرانداز کریں۔ یہاں \عددی{a_0} تعین کرنا اصل مسئلہ ہے۔ اپنے جواب کو بے بُعدی مقدار \عددی{\beta\equiv2ma\alpha/\hslash^2} کی صورت میں پیش کریں۔

جواب: \عددی{\sigma=4\pi a^2\beta^2/(1+\beta)^2}	
\انتہا{سوال}
\حصہ{ینتقلات حیط}
پہلے نصف لکیر \عددی{x<0} پر مکامی مخفیہ \عددی{V(x)} سے یک بُعدی بکھراؤ کے مسئلے پر غور کرتے ہیں \حوالہء{شکل \num{11.7}} میں \عددی{x=0} پر اینٹون کی ایک دیوار کھڑی کرتا ہوں تاکہ بائیں سے آمدی موج 
\begin{align}
	\psi_i(x) = Ae^{ikx}&&(x<-a)
\end{align}
مکمل طور پر منعکس ہوگا
\begin{align}
	\psi_r(x) = Be^{-ikx}&&(x<-a)
\end{align}
باہم عمل خطہ \عددی{(-a<x<0)} میں جو کچھ بھی ہو احتمال کی بقا کی بنا منعکد موج کا حیطہ لاظماً آمدی موج کے حیطہ کے برابر ہوگا۔ تاہم ضروری نہیں کہ اس کا حیط وہی ہو اگر ماسوائے \عددی{x=0} پر دیوار کے کوئی مخفیہ نہیں پایا جاتا ہو تب چونکہ مبدہ پر آمدی جمع منعکس کل تفاعل موج صفر ہوگا 
\begin{align}
	\psi_0(x) = A\left(e^{ikx}-e^{-ikx}\right)&&(V(x)=0)
\end{align}
لحاظہ \عددی{B=-A} ہوگا۔ غیر صفر مخفیہ کی صورت میں \عددی{x<-a} کے لیئے تفاعل موج درج ذیل روپ اختیار کرتا ہے
\begin{align}
	\psi(x) = A\left(e^{ikx}-e^{i(2\delta-kx)}\right)&&(V(x)\neq0)
\end{align}
نظریہ بکھراؤ کی پوری کہانی کسی مخصوص مخفیہ کے لیئے \عددی{k} لحاظہ توانائی \عددی{E=\hslash^2k^2/2m} کی صورت میں ینتقل حیط کے حساب کا دوسرا نام ہے۔ ہم خطہ بکھراؤ \عددی{(-a<x<0)} میں مساوات زروڈنگر کو حل کر کے مناصب سرحدی شرائط مسلط کر کے ایسا کرتے ہیں \حوالہء{سوال \num{11.5}} دیکھیں۔ مخلوط حیطہ \عددی{B} کی بجائے ینتقل حیط کے ساتھ کرنے کا فائدہ یہ ہے کہ یہ طبیعات پر روشنی ڈالتا ہے۔ احتمال کی بقا کی بدولت مخفیہ منعکس موج کی صرف حیط تبدیل کرسکتا ہے اور ایک مخلوط مقدار جو دو حقیقی اعدات پر مشتمل ہوتا ہے کی بجائے ایک حقیقی مقدار کے ساتھ کام کرتے ہوئے ریاضی آسان ہوتی ہے۔

آئیں اب تین بُعدی صورت پر دوبارہ ڈالیں۔ آمدی مستوی موج \عددی{(Ae^{ikz})} کا \عددی{z} رخ میں کوئی زاویائی معیارِ حرکت نہیں پایا جاتا کلیہ ریلے میں \عددی{m\neq0} والا کوئی جز نہیں پایا جاتا۔ تاہم اس میں کل زیاویائی معیارِ حرکت \عددی{(l=0, 1, 2, \dots)} کی تمام قیمتیں شامل ہیں۔ چونکہ کروی تشاکلی مخفیہ زاویائی معیارِ حرکت کی بقا کرتا ہے لحاظہ ہر ایک جزوی موج جسے کسی ایک خصوصی \عددی{l} سے نام دیا جاتا ہے انفرادی طور پر بکھرے گی اور اس کے حیطہ میں کوئی تبدیلی رونما نہیں ہوگی تاہم اس کا حیطہ تبدیل ہوسکتا ہے۔ مخفیہ بلکل نہ ہونے کی صورت میں \عددی{\psi_0=Ae^{ikz}} ہوگا لحاظہ \عددی{l}ویں جزوی موج درج ذیل ہوگی \حوالہء{مساوات \num{11.28}}
\begin{align}
	\psi_0^{(l)} = Ai^l(2l+1)j_l(kr)P_l(\cos\theta)&&(V(r)=0)
\end{align}
لیکن \حوالہء{مساوات \num{11.19}} اور \حوالہء{جدول \num{11.1}} کے تحت درج ذیل ہوگا
\begin{align}
	j_l(x) = \frac{1}{2}\left[h^{(1)}(x)+h_l^{(2)}(x)\right]\approx\frac{1}{2x}\left[(-i)^{l+1}e^{ix}+i^{l+1}e^{-ix}\right]&&(x\gg1)
\end{align}
لحاظہ بڑی \عددی{r} کی صورت میں درج ذیل ہوگا
\begin{align}
	\psi_0^{(l)}\approx A\frac{(2l+1)}{2ikr}\left[e^{ikr}-(-1)^le^{-ikr}\right]P_l(\cos\theta)&&(V(r)=0)
\end{align}
چکور کوسین میں دوسرا جز آمدی کروی موج کو ظاہر کرتا ہے مخفیہ بکھراؤ متعارف کرمے نے یہ تبدیل نہیں ہوگا۔ پہلا جز رخصتی موج ہے جو ینتقل حیط \عددی{\delta_l} لیتا ہے
\begin{align}
	\psi^{(1)}\approx A\frac{(2l+1)}{2ikr}\left[e^{i(kr+2\delta_1)}-(-1)^le^{-ikr}\right]P_l(\cos\theta)&&(V(r)\neq0)
\end{align}
آپ \عددی{e^{ikz}} میں \عددی{h_l^{(2)}} جز کی بنا اس کو کروی مرتکز موج تصور کر سکتے ہیں جس میں \عددی{2\delta_l} ینتقل حیط پایا جاتا ہے اور جو \عددی{e^{ikz}} میں \عددی{h_l^{(1)}} حصہ کے ساتھ بکھرے موج کی بدولت رخصتی کرویہ موج کے طور پر اُبھرتا ہے۔

\حوالہء{حصہ 11.2.1} میں پورے نظریہ کو جزوی تفاعل حیطوں \عددی{a_l} کی صورت میں پیش کیا گیا یہاں اس کو ینتقل حیط \عددی{\delta_l} کی صورت میں پیش کیا گیا۔ ان دونوں کے بیچ ضرور کوئی تعلق پایا جاتا ہوگا۔ یقیناً \حوالہء{مساوات \num{11.23}} کی بڑی \عددی{r} کی صورت میں متقاربی روپ 
\begin{align}
	\psi^{(1)}\approx A\left\{\frac{(2l+1)}{2ikr}\left[e^{ikr}-(-1)^le^{-ikr}\right]+\frac{(2l+1)}{r}a_le^{ikr}\right\}P_l(\cos\theta)
\end{align}
کا \عددی{\delta_l} کی صورت میں عمومی کی صورت \حوالہء{مساوات \num{1.44}} کے ساتھ موازنہ کرنے ساے درج ذیل حاصل ہوگا
\begin{align}
	a_l=\frac{1}{2ik}\left(e^{2i\delta_l}-1\right)=\frac{1}{k}e^{i\delta_l}\sin(\delta_l)
\end{align}
اس طرح بلخصوص \حوالہء{مساوات \num{11.25}} 
\begin{align}
	f(\theta) = \frac{1}{k}\sum_{l=0}^{\infty}(2l+1)e^{i\delta_l}\sin(\delta_l)P_l(\cos\theta)
\end{align}
اور درج ذیل ہوگا \حوالہء{مساوات \num{11.27}} 
\begin{align}
	\sigma=\frac{4\pi}{k^2}\sum_{l=0}^{\infty}(2l+1)\sin^2(\delta_l)
\end{align}
اب بھی جزوی موج حیطوں کی بجائے ینتقلات حیط کے ساتھ کام کرنا بہتر ثابت ہوتا ہے چونکہ ان سے طبعی معلومات باآسانی حاصل ہوتی ہے اور ریاضی کی نقطہ نظر سے ان کے ساتھ کام کرنا آسان ہوتا ہے ینتقلی حیط زاویائی معیارِ حرکت کی بقا کو استعمال کرتے ہوئے مخلوط مقدار \عددی{a_l} جو دو حقیقی اعدات پر مشتمل ہوتا ہے کی بجائے ایک حقیقی عدد \عددی{\delta_l} استعمال کرتا ہے۔
\end{document}
