\documentclass[leqno, b5paper]{khalid-urdu-book}
\begin{document}
\حصہ{خود با خود اخراج}
\جزوحصہ{آئنسٹائن \عددی{A} اور \عددی{B} عددی سر}
فرض کریں ایک برتن میں زیریں حال \(\psi_a\) میں \(N_a\) اور بالائی حال \(\psi_b\) میں \(N_b\) جوہر پائے جاتے ہوں۔ خود با خود اخراجی شرح \عددی{A} لیتے ہوئے اکائی وقت میں بالائی حال کو \(N_bA\) ذرات خود با خود اخراج کے عمل سے چوڑیں گے۔ جیسا ہم مساوات \num{9.47} میں دیکھ چکے ہیں تحرقی اخراج کی تحویلی شرح برقناطیسی میدان کی کثافت توانائی کے راست متناسب ہوگا \(B_{ab}\rho(\omega_0)\)۔ یوں بالائی حال کو تحرقی اخراج کی بنا اکائی وقت میں \(N_bB_{ba}\rho(\omega_0)\) ذرات چوڑیں گے۔ اسی طرح انجزابی ریٹ \(\rho(\omega_0)\) کا راست متناسب ہے جسے ہم \(B_{ab}\rho(\omega_0)\) کہتے ہیں۔ اس طرح اکائی وقت میں \(N_aB_{ab}\rho(\omega_0)\) ذرات بالائی حال میں شامل ہوں گے تمام کو ملا کر درج ذیل ہوگا۔
\begin{align}
	\frac{dN_b}{dt} = -N_bA-N_bB_{ba}\rho(\omega_0)+N_aB_{ab}\rho(\omega_0)
\end{align}
فرض کریں پائے جانے والے میدان کے ساتھ یہ جوہر حراری توازن میں ہوں یوں ہر ایک سطح میں ذرات کی تعداد مستقل ہوگی اور \(dN_b/dt = 0\) ہوگا۔ جس سے درج ذیل حاصل ہوتا ہے۔
\begin{align}
	\rho(\omega_0) = \frac{A}{(N_a/N_b)B_{ab}-B_{ba}}
\end{align}
ہم بنیادی شماریاتی میکانیات سے جانتے ہیں کہ درجہ حرارت \عددی{T} پر حراری توازن میں توانائی \عددی{E} ذرات کی تعداد بولٹزمان جز ضربی \(\exp(-E/k_BT)\) کے راست متناسب ہوگا لحاظہ
\begin{align}
	\frac{N_a}{N_b} = \frac{e^{-E_a/k_{B}T}}{e^{-E_b/k_BT}} = e^{\hbar\omega_0/k_BT}
\end{align}
اور درج ذیل ہوں گے
\begin{align}
	\rho(\omega_0) = \frac{A}{e^{\hbar\omega_0/k_BT}B_{ab}-B_{ba}}
\end{align}
لیکن پلانک کا سیاہ جسمی کلیہ مساوات \num{5.113} ہمیں حراری شعاع کی کثافت توانائی دیتی ہے۔
\begin{align}
	\rho(\omega) = \frac{\hbar}{\pi^2c^3}\frac{\omega^3}{e^{\hbar\omega/k_BT}-1}
\end{align}
ان دونوں ریاضی جملوں کو موازنہ کرنے سے درج ذیل 
\begin{align}
	B_{ab} = B_{ba}
\end{align}
اور درج ذیل حاصل ہوگا
\begin{align}
	A = \frac{\omega^3_0\hbar}{\pi^2c^3}B_{ba}
\end{align}
مساوات \num{9.53} اُس بات کی تصدیق کرتی ہے جو ہم پہلے سے جانتے ہیں تحرقی اخراج کی تحویلی شرح وہی ہے جو انجزاب کی ہے۔ لیکن سن \num{1917} میں یہ ایک حیرت کن نتیجہ تھا جس میں آئنسٹائن کو اس بات پر مجبور کیا کہ وہ کلیہ پلانک حاصل کرنے کی خاطر تحرقی اخراج ایجاد کرے تاہم ہماری دلچسپی یہاں پر مساوات \num{9.54} ہے جو ہمیں تحرقی اخراجی شرح \((B_{ba}\rho(\omega_0))\) جہ ہم پہلے سے جانتے ہیں کی صورت میں خود با خود اخراجی شرح \عددی{A} دیتی ہے۔ جسے ہم جاننا چاہتے ہیں مساوات \num{9.47} کی مدد سے درج ذیل لکھا جا سکتا ہے۔
\begin{align}
	B_{ba} = \frac{\pi}{3\epsilon_0\hbar^2}\abs{p}^2
\end{align}
لحاظہ خود با خود اخراجی شرح درج ذیل ہوگا
\begin{align}
	A = \frac{\omega^3_0\abs{p}^2}{3\pi\epsilon_0\hbar c^3}
\end{align}
\ابتدا{سوال}
نیچے رختحویل میں خود با خود اخراج اور حراری تحرقی اخراج وہ تحرقی اخراج جو سیاہ جسم شعاع کی بنا ہو میں مقابلہ ہوگا۔ دیکھائیں کہ رہائشی درجہ حرارت \(T = \SI{300}{\kelvin}\) پر \(\SI{5e12}{\hertz}\) سے بہت کم تعددوں پر حراری تحرقی اخراج غالب ہوگا جبکہ \(\SI{5e12}{\hertz}\) سے بہت زیادہ تعدد پر خود با خود اخراج غالب ہوگا۔ دیکھائی دینے والی روشنی کے لیئے کونسا غالب ہوگا؟
\انتہا{سوال}
\ابتدا{سوال}
برقناطیسی میدان کا زمینی حال کثافت توانائی \(\rho_0(\omega)\) جانتے ہوئے خود با خود اخراجی اشارہ درحقیقت تحرقی اخراج مساوات \num{9.47} ہوگا۔ لحاظہ آئنسٹائن عددی سر \عددی{A} اور \عددی{B} جانے بغیر آپ خود با خود اخراجی شرح مساوات \num{9.56} اخز کر سکتے ہیں۔ اگرچہ ایسا کرنے کے لیئے کوانٹم برقی حرقیات بروحِ کار لانی ہوگی تاہم اگر آپ یہ ماننے پر آمادہ ہوجائیں کہ زمینی حال کی ہر ایک انداز میں صرف ایک فوٹان پایا جاتا ہے تب اس کو اخز کرنا بہت آسان ہوگا۔

(الف) مساوات \num{5.111} کی جگی \(N_\omega = d_k\) پُر کرکے \(\rho_0(\omega)\) حاصل کریں۔ بہت زیادہ تعدد پر اس کلیہ کو ناکارا ہونا ہوگا ورنہ کل خلائی توانائی لامتناہی ہوگی۔ تاہم یہ کہانی کسی دوسرے دن کے لیئے چھوڑتے ہیں۔

(ب) اپنے نتیجہ کے ساتھ مساوات \num{9.47} استعمال کرکے خود با خود اخراجی شرح حاصل کریں۔ مساوات \num{9.56} کے ساتھ موازنہ کریں۔
\انتہا{سوال}
\end{document}
