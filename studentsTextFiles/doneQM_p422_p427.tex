
\ابتدا{سوال}
\موٹا{یولیدہ حالات}۔ یولیدہ حالات کی ایک کلاسیکی مثال یکتا چکر تنظیم \حوالہء{مساوات \num{12.1}} ہے۔ اس دو ذرہ حال کو دو یک ذری حالات کا مجموعہ نہیں لکھا جا سکتا ہے لحاظہ جس کے بارے میں بات کرتے ہوئے کسی ایک ذرے کے علیحدہ حال کی بات نہیں کی جاسکتی ہے۔ آپ گمان کر سکتے ہیں کہ شائد ہماری علامتیت کی بنا ہے اور عین ممکن ہے کہ یک ذرہ حلات کا کوئی خطی جوڑ اس نظام کو کھول سکے درج ذیل مسئلے کا ثبوت پیش کریں۔

دو سطحی ایک نظام \عددی{\mid\psi_a\rangle} اور \عددی{\mid\psi_b\rangle} پر غور کریں جہاں \عددی{\langle\psi_i\mid\psi_j\rangle=\delta_{ij}} ہو۔ مثلاً \عددی{\mid\psi_a\rangle} ہم میدان اور \عددی{\mid\psi_b\rangle} خلاف میدان کو ظاہر کرسکتا ہے۔ دو ذری حال 
\begin{align*}
	\alpha\mid\phi_a(1)\rangle\mid\phi_b(2)\rangle+\beta\mid\phi_b(1)\rangle\mid\phi_a(2)\rangle
\end{align*}
جہاں \عددی{\alpha\neq0} اور \عددی{\beta\neq0} ہیں کو کسی بھی یک ذری حالات \عددی{\mid\psi_r\rangle} اور \عددی{\mid\psi_s\rangle} کا حاصل ضرب
\begin{align*}
	\abs{\psi_r(1)\rangle}\psi_s(2)\rangle
\end{align*}
نہیں لکھا جا سکتا ہے۔

اشارہ: \عددی{\mid\psi_s\rangle} اور \عددی{\mid\psi_r\rangle} کو \عددی{\mid\psi_a\rangle} اور \عددی{\mid\psi_b\rangle} کے خطی جوڑ لکھیں۔
\انتہا{سوال}
\حصہ{مسئلہ بل}
آئنسٹائن، پوڈولسکی اور روزن کا کوانٹم میکنیات کی درستگی پر کوئی شق نہیں تھا البتہ انکا دعوہ کے طبعی حقیقت کو بیان کرنے کے لیئے یہ ایک نہ مکمل نظریہ ہے کسی بھی نظام کا حال پوری طرح جاننے کی خاطر \عددی{\psi} کے ساتھ ساتھ ایک اور مقدار \عددی{\lambda} درکار ہوگی۔ چونکہ فل حال ہم نہیں جانتے کہ \عددی{\lambda} کو کس طرح ناپا یا حساب کے ذریعہ معلوم کیا جائے۔ لحاظہ ہم اسے درپردہ متغیر کہتے ہیں۔ تاریخی طورپر کئی درپردہ متغیر نظریات پیش کئے گئے جو پیچیدہ ہونے کے ساتھ ساتھ نامعقول ثابت ہوئے بہر حال سن \num{1964} تک اس پر کام کرنے کی وجہ نظر آتی تھی تاہم اس سال جناب بل نے ثابت کیا کہ درپردہ متغیر نظریہ اور کوانٹم میکانیات ساتھ ساتھ نہیں چل سکتے ہیں۔

بل نے آئنساٹائن، پڈولسکی اور روزن بوہم تجربہ کو عمومی بنانے کی بات کی الیکٹران اور پوزیٹران کاشف کو ایک ہی رخ رکھنے کی بجائے بل نے انہیں علیحدہ علیحدہ زاویوں پر رکھنے کی اجازت دی۔ پہلا کاشف اکائی سمتیہ \عددی{a}  کے رخ الیکٹران چکر کا جز ناپتا ہے جبکہ دوسرا \عددی{b} کے رخ پوزیٹران کے چکر کا حصہ ناپتا ہے \حوالہء{شکل \num{12.2}}۔ ہم اپنی آسانی کے لیئے چکر کو \عددی{\hslash/2} کی اکائیوں میں ناپتے ہیں یوں کاشف کے رخ ہم میدان کی قیمت \عددی{+1} اور خلاف میدان کی قیمت \عددی{-1} ناپی جائے گی۔ کئی \عددی{\pi^0} تنزل کے نتائج درج ذیل جدول میں پیش کئے گئے نتائج کی طرح ہوسکتے ہیں۔
\begin{table}[h!]
\begin{center}
\begin{tabular}{|c c c|}
\hline
الیکٹران & پوزیٹران & ضرب \\
$+1$ & $-1$ & $-1$ \\
$+1$ & $+1$ & $+1$ \\
$-1$ & $+1$ & $-1$ \\
$+1$ & $-1$ & $-1$ \\
$-1$ & $-1$ & $+1$ \\
$\vdots$ & $\vdots$ & $\vdots$ \\
\hline
\end{tabular}
\end{center}
\end{table}
کاشف کے رخوں کی کسی ایک جوڑی کے لیئے بل نے چکر کے حاصلِ ضرب کی اوسط قیمت تلاش کی جسے ہم \عددی{P(a,b)} لکھتے ہیں۔ متوازی کاشفوں کی صورت میں \عددی{b=a} ہوگا جو ہمیں اصل آئنسٹائن، پڈلسکی، روزن اور بوہم تجربہ کے نتائج دیگا ایسی صورت میں ایک ہم میدان اور دوسرا خلاف میدان ہوگا لحاظہ ان کا حاصل ضرب ہر صورت \عددی{-1} ہوگا اور یوں اوسط کی قیمت بھی یہی ہوگی
\begin{align}
	P(a, a) = -1
\end{align}
اسی طرح اگر کاشف زد متوازی ہوں تب \عددی{b=-a} اور ہر حاصل ضرب \عددی{+1} لحآظہ درج ذیل ہوگا
\begin{align}
	P(a, -a) = +1
\end{align}
اختیاری سمت بندی کے لیئے کواانٹم میکانیات درج ذیل پیشاً گوئی کرتی ہے
\begin{align}
	P(a, b) = -a\cdot b
\end{align}
\حوالہء{سوال \num{4.50}} دیکھیں۔ بل نے دریافت کیا کہ یہ نتیجہ کسی بھی درپردہ متغیر نظریہ کا ہم اہنگ نہیں ہوسکتا ہے۔

اسکا دلیل حیرت کن حد تک سادہ ہے فرض کریں الیکٹران پوزیٹران نظام کے مکمل حال کو کوئی درپردہ متغیر یا متغیرات \عددی{\lambda} ظاہر کرتا ہے۔ ایک پائیون تنزل سے دوسرے پائیون تنزل تک \عددی{\lambda} کی تبدیلی کو نہ ہم سمجھتے اور نہ ہی قابو کرتے ہیں۔ ساتھ ہی فرض کرتے ہیں کہ الیکٹران کی پیمائش پر پوزیٹران کاشف کی سمت بندی \عددی{b} کا کوئی اثر نہیں پایا جاتا ہے یاد رہے کہ تجربہ کرنے والا الیکٹران کی پیمائش کے بعد پوزیٹران کاشف کا رخ منتخب کرسکتا ہے۔ ایسی صورت میں چونکہ پوزیٹران کاشف کا رخ منتخب کرنے سے پہلے ہی الیکٹران کی پیمائش کی جا چکی ہوگی لحاظہ اس پر بھی کی سمت کا کوئی اثر نہیں ہوسکتا ہے۔ یہ اصول مقامیت کا مفروضہ ہے یوں الیکٹران کی پیمائش کوئی تفاعل \عددی{A(a, \lambda)} اور پوزیٹران کی پیمائش کوئی دوسرا تفاعل \عددی{B(b, \lambda)} دیگا۔ ان تفاعلات کی قیمتیں صرف \عددی{\pm1} ہوسکتی ہیں
\begin{align}
	A(a, \lambda) = \pm1; && B(b, \lambda) = \pm1
\end{align}
جب کاشف متوازی ہوں تب تمام \عددی{\lambda} کے لیئے درج ذیل ہوگا 
\begin{align}
	A(a, \lambda) = -B(a, \lambda)
\end{align}
اب پیمائشوں کی حاصل ضرب کی اوسط قیمت درج ذیل ہوگی جہاں \عددی{\rho(\lambda)} درپردہ متغیر کی کثافت احتمال ہے
\begin{align}
	P(a, b) = \int\rho(\lambda)A(a, \lambda)B(b, \lambda)\dif\lambda
\end{align}
کسی بھی کثافت کا احتمال کے لیئے یہ غیر منفی ہوگا اور معمولزنی شرط \عددی{\int\rho(\lambda)\dif\lambda=1} کو متمعن کرے گا تاہم اس کے علاوہ ہم \عددی{\rho(\lambda)} کے بارے میں کچھ بھی فرض نہیں کرتے ہیں درپردہ متغیر کے مختلف نظریات \عددی{\rho} کے لیئے کافی مختلف تفاعلات پیش کر سکتے ہیں۔ \حوالہء{مساوات \num{12.6}} کو استعمال کرتے ہوئے ہم \عددی{B} کو خارج کر سکتے ہیں۔
\begin{align}
	P(a, b) = -\int\rho(\lambda)A(a, \lambda)A(b, \lambda)\dif\lambda
\end{align}
اگر \عددی{c} کوئی تیسرا اکائی سمتیہ ہو ت بدرج ذیل ہوگا
\begin{align}
	P(a, b)-P(a, c) = -\int\rho(\lambda)\left[A(a, \lambda)A(b, \lambda)-A(a, \lambda)A(c, \lambda)\right]\dif\lambda
\end{align}
اور چونکہ \عددی{[A(b, \lambda)]^2=1} ہے لحاظہپ درج ذیل ہوگا 
\begin{align}
	P(a, b)-P(a, c) =-\int\rho(\lambda)\left[1-A(b, \lambda)A(c, \lambda)\right]A(a, \lambda)A(b, \lambda)\dif\lambda
\end{align}
تاہم \حوالہء{مساوات \num{12.5}} کے تحت \عددی{-1\leq[A(a, \lambda)A(b, \lambda)]\leq+1} مزید \عددی{\rho(\lambda)[1-A(b, \lambda)A(c, \lambda)]\geq0} لحاظہ 
\begin{align}
	\abs{P(a, b)-P(a,c)}\leq\int\rho(\lambda)\left[1-A(b, \lambda)A(c, \lambda)\right]\dif\lambda
\end{align}
یا مختصراً درج ذیل ہوگا
\begin{align}
	\abs{P(a, b)-P(a, c)}\leq1+P(b, c)
\end{align}
یہ مشہور بل عدم مساوات ہے۔ \حوالہء{مساوات \num{12.5} اور \num{12.6}} کے علاوہ کوئی شرط عائد نہیں کی گئی ہے ہم نے درپردہ متغیرات کی تعداد یا خاصیت یا تقسیم \عددی{\rho} کے بارے میں کچھ بھی فرض نہیں کیا لحاظہ یہ عدم مساوات ہر مکامی درپردہ متغیر نظریہ کے لیئے کارامد ہوگا۔

لیکن ہم بہت آسانی ساے دیکھا ساکتے ہیں کہ کوانٹم میکانیات کی پیشاً گوئی \حوالہء{مساوات \num{12.4}} اور بل عدم مساوات ہم اہن نہیں ہیں۔ فرض کریں تینوں اکائی سمتیات ایک مستوی میں پائے جاتے ہوں اور \عددی{a} اور \عددی{b} کے ساتھ \عددی{c} کا زاویہ \عددی{45^\circ} ہو \حوالہء{شکل \num{12.3}} ایسی صورت میں کوانٹم میکانیات کہتی ہے کہ 
\begin{align*}
	P(a, b) = 0, && P(a, c) = P(b, c) = -\num{0.707}
\end{align*}
جبکہ بل عدم مساوات کہتی ہے کہ
\begin{align*}
	\num{0.707}\nleq1-\num{0.707} = \num{0.293}
\end{align*}
جہ ایک دوسرے کے غیر ہم اہنگ نتائج ہیں یوں بل کی ترمیم سے آئنسٹائن، پڈولسکی اور روزن تضاد ایک ایسی بات ثآبت کرتا ہے جو اس کے مصنفین تصور بھی نہیں کر سکتے تھے۔ اگر وہ درست ہوں تب نہ صرف کوانٹم میاکانیا نہ مکمل ہے بلکہ یہ مکلمل طور پر غلط ہے اس کے بر عکس اگر کوانٹم میکانیا درست ہے تب کوئی درپردہ متغیر نظریہ ہمیں اس غیر مکامیت سے نجات نہیں دو سکتی جسے آئنسٹائن مضائقہ خیز سمجھتا تھا۔	 مزید اب ہم بہت سادی تجربہ سے اس مسئلے کو دفنا سکتے ہیں۔

بل عدم مساوات کو پرکھنے کے لیئے ساٹھ اور ستر کی دیہائیوں میں کئی تجربات سرانجام دئے گئے جن میں ایسمیکٹ، گرینگیئر اور روجر کا کام قابلِ فخر ہے ہمیں یہاں انکے تجربہ کی تفصیل سے دلچسپی نہیں ہے۔ انہوں نے پائیون تمزل کی بجائے دو فوٹان جوہری انتقال استعمال کیا یہ خدشہ دور کرنمے کے لیئے کہ الیکٹران کاشف کی سمت بندی کو کسی طرح پوزیٹران کاشف جان پائے گا  فوٹان کی راوانگی کے بعد دونوں کی سمت بندی کی گئی۔نتائج کوانٹم میکانیات کی پیشاً گوئی کی عین مطابق تھے اور بل عدم مساوات کے غیر ہم اہنگ تھے۔

ستم ظریفی کی بات ہے کہ کوانٹم میکانیات کی تجرباتی تصدیق نے سائنسی برادری کو ہلا کر رکھ دیا۔ لیکن اس کی وجہ حقیقت پسند سوچ کا غلط ثابت ہونا نہیں تھا عموماً سائنسدان کب کے اس حقیقت کو مان چکے تھے اور جو ابھی بھی مانتے تھے انکے لیئے غیر مکامی درپردہ متغیر نظریات کا راستہ ابھی کھلا ہے چونکہ مثلا بل اطلاق ان پر نہیں ہوتا ہے۔ اصل سدمہ اس بات کا تھا کہ قدرت ازخود بنیادی طور پر غیر مکامی ہے۔ تفاعل موج کی فوراً انہدام کی صورت میں غیر مکامیت یا متماثل ذرات کے لیئے ضرورت تشاکلیت ہمیشہ تقلید پسند نظریہ کی خاصیت رہی ہے۔ تاہم ایسپیکٹ کے تجربہ سے قبل اُمید کی جاسکتی تھی کہ کوانٹم غیر مکامیت کسی طرح قائد و ضوابط کی غیر طبعی پیداوار تھی جس کے قابلِ کشف اثرات نہیں ہوسکتے ہیں اس اُمید کو بھول جائیں ہمیں فاصلہ پر یکدم عمل کے تصور کو دوبارہ دیکھنا ہوگا۔

ماہر طبیعیات روشنی سے زیادہ تیز رفتار اثر و وسوخ کو کیوں برداشت نہیں کر سکتے ہیں؟ آخر کئی چیزیں روشنی سے زیادہ تیز رفتار سے حرکت کرتی ہے۔ ایک موم بتی کے سامنے چلتے ہوئے کیڑے کا سامنے دیوار پر ساے کی رفتار دیوار تک فاصلے کے راست متناسب ہوگی اصولاً آپ اس فاصلہ کو اتنا بڑھا سکتے ہیں کہ سایہ کی رفتار روشنی سے زیادہ ہو \حوالہء{شکل \num{12.4}}۔ تاہم دیوار پر کسی ایک نقطہ سے دوسرے نقطہ تک سایہ نہ کوئی توانائی منتقل کرسکتا ہے اور نہ ہی کوئی خبر پہنچا سکتا ہے۔ نقطہ \عددی{X} پر ایک شخص ایسا کوئی عمل نہیں کر سکتا جو یہاں سے گزرتے ہوئے ساے کے ذریعہ نقطہ \عددی{Y} پر اثر انداز ہو۔

