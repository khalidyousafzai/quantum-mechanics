


\حصہ{ڈیراک علامتیت}
دو ابعاد میں ایک سادہ سمتیہ \عددی{\kvec{A}}پر غور کریں (شکل \حوالہ{3.3} الف) ۔ آپ اس سمتیہ کو کس طرح بیان کریں گے؟ سب سے آسان طریقہ یہ ہو گا کہ آپ \عددی{x } اور \عددی{y} محدد کا ایک کارتیسی نطام قائم کریں اور اس پر سمتیہ \عددی{\kvec{A}} کے اجزاء \عددی{ A_{x} = \hat{i}\cdot \kvec{A} } ، \عددی{    A_{y} = \hat{j}\cdot \kvec{A} }بیان کریں (شکل \حوالہ{3.3} ب) ۔اب عین ممکن ہے کہ آپ کی بہن نے ایک مختلف کارتیسی نظام قائم کیا ہو جس کے محدد \عددی{x^{'}} اور \عددی{y^{'}} ہوں اور وہ سمتیہ \عددی{\kvec{A}} کے اجزاء \عددی{A_{x}^{'} = \hat{i}^{'} \cdot \kvec{A}} ، \عددی{A_{y}^{'} = \hat{j}^{'} \cdot \kvec{A}} پیش کرے گی  ( شکل\حوالہ{3.3} ج). حقیقت میں آپ دونوں ایک ہی سمتیہ کو دو مختلف اساس \عددی{( \{ \hat{i} , \hat{j} \} ) } اور \عددی{( \{ \hat{i}^{'} , \hat{j}^{'} \} ) } کی صورت میں بیان کر رہے ہیں۔ سمتیہ ازخود فضا میں پایا جاتا ہے اور کسی کے بھی قائم کردہ محددی نظام کا تابع نہیں ہے۔ یہی کچھ میکانی سمتیات میں ایک نظام کے حال کے بارے میں درست ہو گا۔ اس کو سمتیہ \عددی{| \& (t) \rangle } سے ظاہر کیا جا سکتا ہے جو ہلبرٹ فضا میں رہتا ہے لیکن ہم اس کو کسی بھی مختلف اساس کے لحاظ سے بھی بیان کر سکتے ہیں۔ درحقیقت امتیازی تفاعل مقام کی اساس میں \عددی{| \& \rangle} کی پھیلاؤ کا عددی سر موجی تفاعل \عددی{\Psi(x,t)} ہو گا:
\begin{align}
\Psi(x,t) = \langle x | \&(t) \rangle . 
\end{align}
جہاں \عددی{\hat{x}} کے امتیازی تفاعل جس کی امتیازی قیمت \عددی{x} ہے کو ویکٹر \عددی{| x \rangle} ظاہر کرتا ہے جب کہ امتیازی تفاعل معیار حرکت کی اساس میں
\عددی{| \& \rangle } کی پھیلاؤ مقام و معیار حرکت موجی تفاعل  
\عددی{\Phi(p,t)} ہے:
\begin{align}
\Phi(p,t) = \langle p | \& (t) \rangle
\end{align}
جہاں \عددی{\hat{p}} کا امتیازی تفاعل جس کی امتیازی قیمت \عددی{    p} ہے کو ویکٹر \عددی{| p \rangle } ظاہر کرتا ہے۔ ہم ان کی بجائے \عددی{| \& \rangle} کے پھیلاؤ کو امتیازی تفاعل  توانائی کی اساس کی صورت میں بھی کر سکتے ہیں جہاں اپنی آسانی کے لیے ہم غیر مسلسل طیف فرض کر رہے ہیں :
\begin{align}
c_{n} (t) = \langle n | \& (t) \rangle
\end{align}
جہاں \عددی{\hat{H}} کے \عددی{nth}امتیازی تفاعل کو ویکٹر  \عددی{| n \rangle } ظاہر کرتا ہے مساوات \حوالہ{ 3.46 }لیکن یہ تمام ایک ہی حالت کو ظاہر کرتے ہیں تفاعل
\عددی{\Psi} اور \عددی{\Phi} اور عددی سروں کا سلسلہ 
\عددی{\{ c_{n} \} } ایک ہی معلومات رکھتے ہیں۔ یہ تینوں ایک ہی سمتیہ مساوات کو ظاہر کرنے کے تین مختلف طریقے ہیں۔
\begin{align}
\Psi (x,t) &= \int \Psi(y,t) \delta (x-y) dy = \int \Phi (p,t) \frac{1}{2\pi\hbar} e^{ipx/\hbar} dp \nonumber \\ 
&= \sum c_{n} e^{-iE_{n}t/\hbar} \psi_{n}(x)
\end{align}
قابل مشاہدہ کو ظاہر کرنے والے حاملین خطی تبادلہ ہوتے ہیں جو ایک سمتیہ کا تبادلہ دوسری سمتیہ میں کرتے ہیں 
\begin{align}
| \beta \rangle = \hat{Q}|| \alpha \rangle
\end{align}
بالکل سمتیات کی طرح جنہیں ایک مخصوص اساس \عددی{\{ |e_{n} \rangle \} } کے لحاظ سے ان کے اجزاء 
\begin{align}
| \alpha \rangle = \sum_{n} a_{n} | e_{n} \rangle. \text{ with } a_{n} = \langle e_{n} | \alpha \rangle : | \beta \rangle = \sum_{n} b_{n} | e_{n} \rangle , \text{ with } b_{n} \langle e_{n} \beta \rangle
\end{align}
سے ظاہر کیا جاتا ہے۔ حاملین کو کسی مخصوص اساس کے لحاظ سے ان کے قالب کے ارکان 
\begin{align}
\langle e_{m} | \hat{Q} | e_{n} \equiv Q_{mn}
\end{align}
سے ظاہر کیا جاتا ہے 
اس علامت کو استعمال کرتے ہوئے مساوات
\حوالہ{ 3.79 } درج ذیل روپ اختیار کرتی ہے 
\begin{align}
\sum_{n} b_{n} | e_{n} \rangle = \sum_{n} a_{n} \hat{Q} | e_{n} \rangle
\end{align}
یا ویکٹر \عددی{| e_{m} \rangle} کے ساتھ اندرونی ضرب لیتے ہوئے 
\begin{align}
\sum_{n} b_{n} \langle e_{m} | e_{n} \rangle = \sum_{n} a_{n} \langle e_{m} | \hat{Q} | e_{n} \rangle
\end{align}
لہٰذا درج ذیل ہو گا 
\begin{align}
b_{m} = \sum_{n} Q_{mn}a_{n}
\end{align}
یوں اجزاء کے تبادلہ کے بارے میں قالبی ارکان معلومات فراہم کرتی ہے ۔
بعد میں ہمیں ایسے نظاموں سے واسطہ ہو گا جن کے خطی غیر تابع حالات کی تعداد متناہی عدد \عددی{N} ہو گی۔ ایسی صورت میں ویکٹر \عددی{| \& (t) \rangle} \عددی{N} ابعادی سمتی فضا میں رہتا ہے جسے کسی اساس کے لحاظ سے \عددی{N} اجزاء کی قطار سے ظاہر کیا جا سکتا ہے جبکہ حاملین کو \عددی{N \times N} سادہ قالب کی صورت لیتے ہیں۔ یہ سادہ ترین کوانٹائ نظام ہیں جن میں لامتناہی آبادی سمتی فضا کی باریکیاں نہیں پائی جاتی ہیں ان میں سب سے آسان دو حالت نظام ہے جس پر درج ذیل مثال میں غور کیا گیا ہے۔
% Example 3.8
تصور کریں کہ ایک نظام میں صرف دو خطی غیر تابع حال ممکن ہیں 
\begin{align*}
|1\rangle = \begin{pmatrix} 1 \\ 0 \end{pmatrix} \text{ and } |2\rangle = \begin{pmatrix} 0 \\ 1  \end{pmatrix}
\end{align*}
سب سے زیادہ عمومی حال ان کا معمول شدہ خطی جوڑ ہو گا
\begin{align*}
| \& \rangle = a|1\rangle +b|2\rangle = \begin{pmatrix} a \\ b  \end{pmatrix} \quad \text{ with } |a|^{2} + |b|^{2} = 1
\end{align*}
ہیملٹنی کو ایک ہرمیشی قالب کے روپ میں لکھا جا سکتا ہے۔ فرض کریں کہ اس کی روپ درج ذیل ہے 
\begin{align*}
H = \begin{pmatrix} h & g \\ g & h  \end{pmatrix}
\end{align*}
جہاں\عددی{g} اور \عددی{h} حقیقی مستقل ہیں۔ اگر \عددی{t=0} پر یہ نظام حال \عددی{|1\rangle} سے شروع ہو تب وقت \عددی{t} پر اس کا حال کیا ہو گا؟ 
حل: وقت کی تابع شروڈنگر مساوات درج ذیل کہتی ہے 
\begin{align}
i \hbar \frac{d}{dt} | \& \rangle = H | \&  \rangle
\end{align}
ہمیشہ کی طرح ہم وقت کی غیر تابع شروڈنگر
\begin{align}
H | \& \rangle = E | \& \rangle 
\end{align}
 کی حل سے ابتداء کرتے ہیں یعنی ہم \عددی{H} کی امتیازی سمتیات اور امتیازی اقدار تلاش کرتے ہیں۔ امتیازی اقدار کی قیمت امتیازی مساوات تعین کرتی ہے 
\begin{align*}
det \begin{pmatrix} h-E & g \\ g & h-E \end{pmatrix} = (h-E)^{2} - g^{2} = 0 \Rightarrow h-E = \mp g \Rightarrow E_{\pm} = h \pm g
\end{align*}
آپ دیکھ سکتے ہیں کہ اجازتی  توانائیاں \عددی{( h+g)} اور \عددی{(h-g)} ہیں۔ امتیازی سمتیات کا تعین کرنے کی خاطر ہم درج ذیل لکھتے ہیں 
\begin{align*}
\begin{pmatrix} h & g \\ g & h  \end{pmatrix} \begin{pmatrix}
 \alpha \\ \beta \end{pmatrix} = (h\pm g) \begin{pmatrix} \alpha \\ \beta  \end{pmatrix} \Rightarrow h\alpha + g\beta = (h\pm g) \alpha \Rightarrow \beta = \pm \alpha
\end{align*}
لہٰذا معمول شدہ امتیازی سمتیات درج ذیل ہوں گے 
\begin{align*}
| \& _{\pm} \rangle = \frac{1}{\sqrt{2}} \begin{pmatrix} 1 \\ \pm 1 \end{pmatrix}
\end{align*}
اس کے بعد ابتدائی حال کو ہم ہیملٹنی کے امتیازی سمتیات کے خطی جوڑ کی صورت میں لکھتے ہیں 
\begin{align*}
| \& (0) \rangle = \begin{pmatrix} 1 \\ 0  \end{pmatrix} = \frac{1}{\sqrt{2}} ( | \&_{+} \rangle + | \&_{-} \rangle )
\end{align*}
آخر میں ہم اس کے ساتھ معیاری تابع وقت جز \عددی{e^{-iE_{n}t/\hbar}} منسلک کرتے ہیں 
\begin{align*}
| \& (t) \rangle &= \frac{1}{\sqrt{2}} [ e^{-i(h+g)t/\hbar} | \&_{+} \rangle + e^{-i(h-g)t/\hbar} | \&_{-} \rangle ] \\
&= \frac{1}{2} e^{-iht/\hbar} \left[ e^{-igt/\hbar} \begin{pmatrix} 1 \\ 1 \end{pmatrix} + e^{igt/\hbar} \begin{pmatrix} 1 \\ -1  \end{pmatrix} \right] \\
&= \frac{1}{2} e^{-iht/\hbar} \begin{pmatrix} e^{-igt/\hbar} + e^{igt/\hbar} \\ e^{-igt/\hbar} - e^{igt/\hbar} \end{pmatrix} = e^{-iht/\hbar} \begin{pmatrix} \cos (gt/\hbar) \\ -i\sin (gt/\hbar) \end{pmatrix}
\end{align*}
اگر آپ کو اس نتیجے پر شک ہو تو آپ اس کی جانچ پڑتال کر سکتے ہیں۔کیا یہ وقت کی تابع شروڈنگر مساوات کو مطمئن کرتا ہے؟ کیا یہ \عددی{t=0} پر ابتدائی حال کے موافق ہے؟ 
یہ دیگر چیزوں کے ساتھ نیوٹرینو کی ارتعاش کا ایک سادہ نمونہ ہے جہاں \عددی{|1\rangle} الیکٹران نیوٹرینو کو ظاہر کرتا ہے \عددی{|2\rangle} میون نیوٹرینو کو ظاہر کرتا ہے۔ اگر ہیملٹنی کی قالب میں خلاف وتر جز \عددی{(g)} غیر معدوم ہو تب وقت گزرنے کے ساتھ بار بار الیکٹران نیوٹرینو تبدیل ہو کر میون نیوٹرینو میں اور میون نیوٹرینو واپس الیکٹران نیوٹرینو میں تبدیل ہوتا رہے گا۔
