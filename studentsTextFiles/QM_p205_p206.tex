\documentclass{book}
\usepackage{amsmath}
\usepackage{polyglossia}
\setmainlanguage[numerals=maghrib]{arabic}
\setotherlanguages{english}
\newfontfamily\arabicfont[Scale=1.0,Script=Arabic]{Jameel Noori Nastaleeq}
\newfontfamily\urdufont[Scale=1.25,Script=Arabic]{Jameel Noori Nastaleeq}
\begin{document}
%$$ page 205 $$
\paragraph*{مثال 5.1}
فرض کریں ایک لا متناہی چکور کنواں میں کمیت M کے باہم غیر متعمل دو ذرات جو ایک دوسرے کے اندر سے گزر سکتے ہیں پاۓ جاتے ہیں۔ آپکو فکر کرنے کی ضرورت نہیں کہ عملا کیسے کیا جا سکتا ہے۔ یک ذرہ حالات درج ذیل ہوں گے۔ جہاں $ K=\frac{(\pi)^2 (\hbar)^2}{2m(a)^2 }
$
ہے۔
\begin{align}
 \Psi_{n} (x)=\sqrt{\frac{2}{a}}sin(\frac{n (\Pi)}{a}x), \quad E_{n}=n^2 K 
\end{align}
یہ ذرات قابل ممیز ہونے کی صورت میں جہاں ذرہ $ 1 $ حال   $ n_{1} $ میں اور ذرہ $ 2 $ حال $ n_{2} $  میں ہو مرکب تفاعل موج سادہ حاصل ضرب ہوگا۔
\begin{align}
 \Psi_{n_{1} n_{2}} (x_{1},x_{2})=\Psi_{n_{1}}(x_{1})\Psi_{n_{2}}(x_{2}), \quad E_{n_{1} n_{2}}= ((n_{1})^2+(n_{2})^2)K. 
\end{align}
%$$ page 206 $$
مثال کے طور پر زمینی حال
\begin{align}
 \Psi_{11}=\frac{2}{a}sin(\frac{\pi x_{1}}{a}) sin(\frac{\pi x_{2}}{a}), \quad E_{11}=2K; 
\end{align}
پہلا حجان حال دوچند انحطاطی 
\begin{align}
 \Psi_{12}=\frac{2}{a}sin(\frac{\pi x_{1}}{a}) sin(\frac{2\pi x_{2}}{a}), \quad E_{12}=5K, \\
\Psi_{21}=\frac{2}{a}sin(\frac{2\pi x_{1}}{a}) sin(\frac{\pi x_{2}}{a}), \quad E_{21}=5K; 
\end{align}
ہوگا وغیرہ وغیرہ۔ دونوں ذرات یکساں بوزان ہونے کی صورت میں زمینی حال تبدیل نہیں ہوگا ۔ تاہم پہلا حجان حال جسکی توانائی اب بھی 5K ہوگی غیر انحطاطی ہوگا۔
\begin{align}
\frac{\sqrt{2}}{a}[sin(\frac{\pi x_{1}}{a})sin(\frac{2\pi x_{2}}{a})+ sin(\frac{2 \pi x_{1}}{a})sin(\frac{\pi x_{2}}{a})]
\end{align} 
اور اگر ذرات یکساں فرمیون ہوں تب کوئی حال بھی 2K توانائی کا نہیں ہوگا۔ جبکہ زمینی حال جسکی توانائی 5K ہوگی۔ درج ذیل ہوگا۔
\begin{align}
\frac{\sqrt{2}}{a}[sin(\frac{\pi x_{1}}{a}) sin(\frac{2 \pi x_{2}}{a})- sin(\frac{2 \pi x_{1}}{a}) sin(\frac{\pi x_{2}}{a})], 
\end{align}
\section*{*سوال 5.4}
\paragraph*{(جزوالف)}
اگر $ \Psi_{b} $ اور  $ \Psi_{a} $ عمودی ہوں اور دونوں معمول شدہ ہوں تب مساوات 5.10 میں مستقل 'A' کیا ہوگا؟ 
\paragraph*{(جزوب)}
اگر  $ \Psi_{a} = \Psi_{b} $ ہوں اور یہ معمول شدہ ہوں تب 'A' کیا ہوگا؟ (یہ صورت صرف بوزون کیلۓ ممکن ہے۔)
\section*{سوال 5.5} 
\paragraph*{(جزو الف)}
لامتناہی چکور کنواں میں باہم غیر متعمل دو یکساں ذرات کا ہملتنی لکھیں۔ تصدیق کیجیۓ کہ مثال 5.1 میں دیا گیا فرمیون کا زمینی حال 'H' کا مناسب امتیازی قدر والا امتیازی تفاعل ہوگا۔ 
\paragraph*{(جزوب)}
مثال 5.1 میں دیۓ گۓ حجان حالات سے اگلے دو حالات تفاعل موج اور توانائیاں تینوں صورتوں میں قابل ممیز یکساں موزوں، یکساں فرمیون حاصل کریں۔

\end{document}
\end{document}
