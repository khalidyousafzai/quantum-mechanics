}
	فرض کریں مادہ کے ایک ٹکڑا میں $N$ جوہر پائے جاتے ہوں اور ہر جوہر اپنے حصہ کے $q$ آزاد الیکٹرون دیتا ہو۔ عملاً کسی بھی کلاں بینی جسامت کے چیز کے لیئے $N$ کی قیمت بہت بڑی ہوگی جو ایوگادرو عدد میں گنی جائے گی جبکہ $q$ ایک چھوٹا عدد مثلاً 1 یا 2 ہوگا۔ اگر ایلکٹرون بوزان یا قابلِ ممیز ذرات ہوتے تب وہ زمینی حال $\psi_{111}$ میں سکونیت اختیار کرتے حقیقتاً الیکٹروں یکساں فرمیونز ہیں جن پر پالی اصول منات کا اطلاق ہوتا ہے لحاظہ کسی بھی حل کی مکین صرف دو الیکٹرون ہو سکتے ہیں۔ یہ $k$ فضا میں ایک کرہ کا ایک ثمن رداس $k_F$ تک بھرے گی جس کو اس حقیقت سے تعین کیا جا سکتا ہے کہ الیکٹران کی ہر ایک جوڑی کو \(\frac{\pi^{3}}{V}\) حجم درکار ہوگا مساوات  \num{5.40}: 
	\begin{align*}
		\frac{1}{8}(\frac{4}{3} \pi k^{3}_F) =  \frac{Nq}{2}(\frac{\pi^3}{V})
	\end{align*}
یوں
\begin{align}
	k_F =(3\rho\pi^{2})^{\frac{1}{3}}
\end{align}
جہاں
\begin{align}
	\rho \equiv \frac{Nq}{V}
\end{align}
آزاد الیکٹران کثافت ہے(آزاد حجم میں الیکٹرانوں کی تعداد)۔

$k$ فضا میں مکین اور غیر مکین حالات کی سرحد کو \موٹا{فرمی سطح} کہتے ہیں (اسی کی بنا زیرنوشت میں $F$ لکھا گیا)۔
اس سطح پر طاقتی توانائی کو \موٹا{فرمی توانائی} $E_F$ کہتے ہیں۔آزاد الیکٹران گیس کے لیئے درج ذیل ہو گا۔
\begin{align}
	E_F = \frac{h^{2}}{2m}(3\rho\pi^{2})^{\frac{2}{3}}
\end{align}
الیکٹران گیس کی کل توانائی کو درج ذیل طریقہ سے حل کیا جا سکتا ہے. ایک خول جس کی موٹائی $dk$ شکل \num{5.4} ہو کا حجم
\begin{align*}
	\frac{1}{8}(4\pi k^{2})dk
\end{align*}
لحاظہ اس خول میں الیکٹرون حالات کی تعداد درج ذیل ہوگی
\begin{align*}
	\frac{2[(\frac{1}{2})\pi k^{2}dk]}{\frac{\pi^{3}}{V}} = \frac{V}{\pi^{2}}k^{2}dk
\end{align*}
ان میں سی ہر ایک حال کی توانائی \(\frac{\hbar^{2}k^{2}}{2m}\) مساوات \num{5.39} لحاظہ خول کی توانائی
\begin{align}
	dE = \frac{\hbar^{2}k^2}{2m} \frac{V}{\pi^{2}}k^{2}dk
\end{align}
اور کل توانائی درج ذیل ہوگی
\begin{align}
	E_{tot}=\frac{\hbar^{2}V}{2\pi^{2}m}\int_{0}^{k_F}k^{4}dk = \frac{\hbar^{2}k^{5}_F V}{10\pi^{2}m} = \frac{\hbar^{2}(3\pi^{2}Nq)^{\frac{5}{3}}}{10\pi^{2}m}V^{\frac{-2}{3}}
\end{align}
کوانٹم میکانی توانائی کا کردار کچھ ایسا ہی ہے جیسا سادہ گیس میں اندرونی حراری توانائی $U$ کا ہوتا ہے۔ بل خصوص یہ دیواروں پر ایک دباؤ پیدا کرتا ہے اور اگر ڈبے کے حجم میں $dV$ کا اضافہ ہو تب کل توانائی میں درج ذیل کمی رونما ہوگی
\begin{align*}
	dE_{tot} = -\frac{2}{3}\frac{\hbar^2(3\pi^{2}Nq)^{\frac{5}{3}}}{10\pi^{2}m}V^{\frac{5}{3}}dV = -\frac{2}{3}E_{tot}\frac{dV}{V}
\end{align*}
جو بیرون پر کوانٹم دباؤ $P$ کا کیا ہوا کام \(dW = PdV\) نظر آتا ہے
\begin{align}
	P = \frac{2}{3}\frac{E_{tot}}{V} = \frac{2}{3}\frac{\hbar^{2}k^{5}_F}{10\pi^{2}m} = \frac{(3\pi^{2})^{\frac{2}{3}}\hbar^{2}}{5m}\rho^{\frac{5}{3}}
\end{align}
یہ اس سوال کا جزوی جواب ہے کہ ایک ٹھنڈا ٹھوس شہ اندر کی طرف منہدن کیوں نہیں ہو جاتا۔ ایک اندرونی کوانٹم میکانی دباؤ توازن برقرار رکھتی ہے جس کا الیکٹرون کے  باہمی دفع جنہیں ہم نظر انداز کر چکے ہیں یا حراری حرکت جس کو ہم خارج کر چکے ہیں کے ساتھ کوئی تعلق نہیں ہے۔ بلکہ جو یکساں فرمیان کی ضرورت خلاف تشاکلیت سے پیدا ہوتا ہے۔ اس کو بعض اوقات انحطاطی دباؤ کہتے ہیں اگر چہ مناتی دباؤ بہتر اصطلاح ہو گی۔

\ابتدا{سوال}
ایک آزاد الیکٹرون کی اوسط توانائی \(\frac{E_{tot}}{Nq}\) کو فرمی توانائی کے قصر کی صورت میں لکھیں۔

جواب: \(\frac{3}{5}E_F\)
\انتہا{سوال}
\ابتدا{سوال}
تانبا کی کثافت $\SI{8.96}{\gram \per \centi\meter\cubed}$ ہے جبکہ اس کا جوہری وزن $\SI{63.5}{\gram \per \mole}$ ہے۔

(الف) مساوات $\num{5.43}$استعمال کرتے ہوئے $q = 1$ لیتے ہوئے تانبے کی فرمی توانائی کا حساب لگا کر نتیجہ کو الیکٹرون ولٹ کی صورت میں لکھیں۔

(ب) الیکٹران کی مطابقتی سمتی رفتار کیا ہوگی؟اشارہ: \(E_F = (\frac{1}{2})mv^{2}\) لیں۔ کیا تانباے میں الیکٹرون کو غیر اضافی تصور کرنا خطرے سے باہر ہو گا؟

(ج) تانبہ کے لیئے کس درجہ حرارت پر امتیازی حراری توانائی $k_{B}T$ جہاں $k_B$ بولٹزمن مستقل اور $T$ کیلون حرارت ہے فرمی توانائی کے برابر ہوگا؟ تبصرہ: اس کو فرمی حرارت کہتے ہیں۔ جب تک حقیقی حرارت فرمی حرارت سے کفی کم ہو مادہ کو ٹھنڈہ  تصور کیا جا سکتا ہے اور اس میں الیکٹرون نچلے ترین قابلِ پہنچ حال میں ہوں گے۔ چونکہ تانبے $\SI{1356}{\kelvin}$ پر گلتا ہے لحاظہ ٹھوس تانبہ ہر صورت ٹھنڈہ ہوگا۔
 
(د) الیکٹران گیس نمونہ میں تانبہ کے لیئے انحطاطی دباؤ مساوات $\num{5.46}$ کا حساب لگائیں۔
\انتہا{سوال}
\ابتدا{سوال}
کسی جسم پر دباؤ میں معمولی کمی اور نتیجتاً حجم میں نصبتی اظافہ کے تناسب کو جسم مقیاس کہتے ہیں۔
\begin{align*}
	B = -V\frac{dP}{dV}
\end{align*}
دیکھائیں کہ آزاد الیکٹران نمونہ میں\(B = \frac{5}{3}P\) ہوگا اور سوال$\num{5.16}(\text{\RL{د}})$ کا نتیجہ استعمال کرتے ہوئے تانباہ کے لیئے جسیم مقیاس کی اندازاً قیمت تلاش کریں۔ تبصرہ: تجربہ سے حاصل قیمت $\SI{13.4e10}{\newton \per \meter \squared}$ ہے مکمل درست جواب کی توقع نہ کریں چونکہ ہم نے الیکٹران مرکزہ اور الیکٹران الیکٹران قوتوں کو نظرانداز کیا ہے! حقیقت میں یہ ایک حرین کن نتیجہ ہے کہ حساب سے حاصل نتیجہ حقیقت کے اتنا قریب ہے۔ 
\انتہا{سوال}
\جزوحصہ{سخت پٹی}
ہم آزاد الیکٹران نمونہ میں منظم فاصلوں پر ساکن مثبت بار کے مرکزہ کی الیکٹرانوں پر قوت کو شامل کر کے بہتر نمونہ حاصل کرتے ہیں۔ ٹھوس اجسام کا رویہ نمایاں حد تک  اس حقیقت پر مبنی ہے کہ اس کا مخفیہ دوری ہوتا ہے۔ مخفیہ کی حقیقی شکل و صورت مادہ کی تفصیلی رویہ میں کردار ادا کرتی ہے۔ یہ عمل دیکھنے کی خاطر میں سادہ ترین نمونہ تیار کرتا ہوں جس سے یک بُعدی ڈیراک کنگھی کہتے ہیں اور جو ایک جتنے برابر فاصلوں پر نوکیلی ڈیلٹا تفاعلوں پر مشتمل ہوتا ہے شکل \num{5.5} ۔ لیکن اس سے پہلے میں ایک طاقتور مسئلہ پیش کرتا ہوں جو دوری مخفیہ کے مسائل کا حل نہایت سادہ بناتا ہے۔

دوری مخفیہ سے مراد ایسا مخفیہ ہے جو کسی مستقل فاصلہ $a$ کے بعد اپنے آپ کو دہراتا ہے۔ 
\begin{align}
	V(x+a) = V(x)
\end{align}
مسئلہ بلوخ کہتا ہے کہ دوری مخفیہ کے لیئے مساوات شروڈنگر،
\begin{align}
	-\frac{\hbar^{2}}{2m}\frac{d^{2}\psi}{dx^{2}} +V(x)\psi = E\psi
\end{align}
کے حل سے مراد وہ تفاعل لیا جا سکتا ہے جو درج ذیل شرط کو مطمئن کرتا ہو
\begin{align}
	\psi(x+a) = e^{iKa}\psi(x)
\end{align}
جہاں $K$ ایک مستقل ہے۔ یہاں مستقل سے مراد ایسا تفاعل ہے جو $x$ کا تابع نہیں ہے اگرچہ یہ $E$ کا تابع ہو سکتا ہے۔

\موٹا{ثبوت}: مان لیں کے $D$ ایک ہٹاؤ عامل ہے:
\begin{align}
	Df(x) = f(x+a)
\end{align}
دوری مخفیہ مساوات \num{5.47} کی صورت میں $D$ ہیملٹنی کا قابلِ تبادل ہو گا:
\begin{align}
	[D, H] = 0
\end{align}
لحاظہ ہم $H$ کے ایسے امتیازی تفاعلات چھنڈ سکتے ہیں جو بیک وقت $D$ کے امتیازی تفاعلات بھی ہوں:\(D\psi = \lambda\psi\) یا
\begin{align}
	\psi(x+a) = \lambda\psi(x)
\end{align}
یہاں $\lambda$ کسی صورت صفر نہیں ہو سکتا اگر یہ صفر ہو تب چونکہ مساوات \num{5.52} تمام $x$ کے لیئے مطمئن ہوگا لحاظہ ہمیں \(\psi(x) = 0\) ملے گا جو قابلِ قبول امتیازی تفاعل نہیں ہے۔ کسی بھی غیر مخلوط عدد کی طرح اس کو قوتِ نمائی روپ میں لکھا جا سکتا ہے: 
\begin{align}
	\lambda = e^{iKa}
\end{align}
جہاں $K$ ایک مستقل ہوگا۔

اس مقام پر مساوات \num{5.53} امتیازی قدر $\lambda$ لکھنے کا ایک انوکھا طریقہ ہے لیکن ہم جلد دیکھیں گے کہ $K$ حقیقی ہے اور یوں اگرچہ $\psi(x)$ ازخود غیر دوری ہے\(\abs{\psi(x)}^{2}\) جو درج ذیل ہے۔
\begin{align}
	\abs{\psi(x+a)}^{2} = \abs{\psi(x)}^{2}
\end{align}
دوری ہوگا جیسا کہ ہم توقع کرتے ہیں۔

اب ظاہر ہے کہ کوئی بھی حقیقی ٹھوس جسم ہمیشہ کے لیئے چلتا نہیں جائے گا بلکہ کہیں نہ کہیں اس کی سرحد پائی جائے گی جو $V(x)$ کی دوریت کو ختم کرتے ہوئے مسئلہ بلوخ کو ناکارہ بنا دے گی۔ تاہم کسی بھی کلابین سطح کے  قلم میں کئی ایوگادرو عدد کے برابر جوہر پائے جائیں گے اور ہم فرض کر سکتے ہیں کہ تھوس جسم کی سطح سے بہت دور الیکٹران پر سطحی اثر قابلِ نظر انداز ہوگا۔ ہم مسئلہ بلوخ پر پورا اترنے کی خاطر $x$ کو ایک دائرے پر رکھتے ہیں تاکہ اس کی دم بہت بڑی تعداد \(N\approx10^{23}\) دوری فاصلوں کے بعد اس کے سر پر پایا جاتا ہو باضابطہ طور پر ہم درج ذیل سرحدی شرط مسلط کرتے ہیں   
\begin{align}
	\psi(x+Na) = \psi(x)
\end{align}
یوں مساوات \num{5.49} کے تحت درج ذیل ہوگا
\begin{align*}
	e^{iNKa}\psi(x) = \psi(x)
\end{align*}
لحاظہ \(e^{iNKa} = 1\) یا \(NKa = 2\pi n\) ہوگا جس کے تحت درج ذیل ہوگا 
\begin{align}
	K = \frac{2\pi n}{Na}, (n = 0, \pm1, \pm2, \dots)
\end{align}
یہاں $K$ لازماً حقیقی ہوگا مسئلہ بلوخ کی عفادیت یہ ہے کہ ہمیں صرف ایک خانہ مثلاً \((0\leq x<a)\) کے وقفہ پر مسئلہ شروڈنگر حل کرنا ہوگا مساوات \num{5.49} کی بار بار اطلاق سے ہر جگہ کے حالات  حاصل ہوںگے۔

اب فرض کریں کے مخفیہ در حقیقت نوکیلی ڈیلٹا تفاعلات ڈیراک کنگھی پر مشتمل ہو:
\begin{align}
	V(x) = \alpha\sum_{j=0}^{N-1}\delta(x-ja)
\end{align}
شکل\num{5.5} میں آپ تصور کریں گے  کہ محور $x$ کو یوں دائروی شکل مین گھومایا گیا ہے کہ $N$ویں نوکیلی تفاعل درحقیقت نقطہ \(x= -a\)  پر پایا جاتا ہے۔ اگر چہ یہ حقیقت پسند نمونہ نہیں ہے لیکن یاد رہے ہمیں دوریت سے دلچسپی ہے۔ کلاسیکی طور پر دہراتا ہوا مستطیلی مخفیہ استعمال کیا گیا جو اب بھی بہت سے مسنیفین کا پسندیدہ مخفیہ ہے خطہ \((0<x<a)\) میں مخفیہ صفر ہوگا لحاظہ 
\begin{align*}
	-\frac{\hbar^{2}}{2m}\frac{d^{2}\psi}{dx^{2}} = E\psi,
\end{align*}
یا
\begin{align*}
	\frac{d^{2}\psi}{dx^{2}} = -k^{2}\psi,
\end{align*}
ہوگا۔

جہاں ہمیشہ کہ طرح درج ذیل ہوگا 
\begin{align}
	k = \frac{\sqrt{2mE}}{\hbar},
\end{align}
اس کا عمومی حل درج ذیل ہے 
\begin{align}
	\psi(x) = A\sin(kx) + B\cos(kx), (0<x<a).
\end{align}
مسئلہ بلوخ کے تحت مبدا کے بلکل بائیں ہاتھ پہلے خانہ میں تفاعل موج درج ذیل ہوگا 
\begin{align}
	\psi(x) = e^{-iKa}[A\sin k(x+a) + B\cos k(x+a)], (-a<x<0). 
\end{align}
نقطہ\(x=0\) پر $\psi$ لازماً استماری ہوگا لحاظہ 
\begin{align}
	B = e^{-iKa}[A\sin(ka) + B\cos(ka)];
\end{align}
اس کے تفرق میں ڈیلٹا تفاعل کی زور کے براہراست متناسب عدم استمرار پائے جائے گی مساوات\num{2.125} جس میں $\alpha$ کی علامت اُلٹ ہوگی چونکہ یہاں کنواں کی بجائے نوکیلی تفاعل پایا جاتا ہے
\begin{align}
	kA - e^{-iKa}k[A\cos(ka) - B\sin(ka)] = \frac{2m\alpha}{\hbar^{2}}B
\end{align}
مساوات \num{5.61} کو \(A\sin(ka)\) کے لیئے حل کرتے ہوئے درج ذیل حاصل ہوگا 
\begin{align}
	A\sin(ka) = [e^{iKa}-\cos(ka)]B
\end{align}
اس کو مساوات \num{5.62} میں پُر کرتے ہوئے اور \عددی{k_B} کو منسوخ کرتے ہوئے 
\begin{align*}
	[e^{iKa}-\cos(ka)][1-e^{-iKa}\cos(ka)] + e^{-iKa}\sin^{2}(ka) = \frac{2m\alpha}{\hbar^{2}k}\sin(ka)
\end{align*}
حاصل ہوگا۔

جس سے درج ذیل سادہ روپ حاصل ہوتا ہے
\begin{align}
	\cos(ka) = \cos(ka) + \frac{m\alpha}{\hbar^{2}k}\sin(ka)
\end{align}
یہ ایک بنیادی نتیجہ ہے جس سے باقی سب کچھ اخز ہوتا ہے۔ کرونیگ پینی مخفیہ ہاشیہ \num{18} دیکھیں کے لیئے کلیہ زیادہ پیچیدہ ہوگا لیکن جو خدوخال ہم دیکھنے جا رہے ہیں وہی اس میں بھی پائے جاتے ہیں۔

 مساوات \num{5.64} $k$ کی ممکنات قیمتیں لحاظہ اجازتی توانائیاں تعین کرتی ہیں۔ علامتیت کو سادہ بنانے کی نقطہ نظر سے ہم درج ذیل لکھتے ہیں 
\begin{align}
	z \equiv ka, \text{and} \beta \equiv \frac{m\alpha a}{\hbar^{2}}
\end{align}
جس سے مساوات \num{5.64} کا دائیاں ہاتھ درج ذیل روپ اختیار کرتا ہے
\begin{align}
	f(z) \equiv \cos(z) + \beta\frac{\sin(z)}{z}
\end{align}
مستقل $\beta$ بُعدی ہے جو ڈیلٹا تفاعل کی زور کی ناپ ہے شکل \num{5.6} میں میں نے \(\beta = 10\) کے لیئے \عددی{f(z)} کو ترسیم کیا ہے۔ یہاں دیکھنے کی اہم بات یہ ہے کے\عددی{f(z)} ساتھ\((-1, +1)\) سے باہر بھٹکتا ہے اور چونکہ \(\abs{\cos(Ka)}\) کی قیمت کسی صورت ایک سے تجاز نہیں کر سکتی ہے لحاظہ ایسی خطوں میں مساوات \num{5.64} کا حل نہیں پایا جائے گا۔ یہ درز ممنوع توانائیوں کو ظاہر کرتی ہے انکے بیچ اجازتی توانائیوں کی پٹیاں پائی جاتی ہیں مساوات \num{5.56} کے تحت \(Ka = \frac{2\pi n}{N}\) ہے جہاں \عددی{N} ایک بہت بڑا عدد ہے لحاظہ \عددی{n} کوئی بھی عدد صحیح ہو سکتا ہے۔ یوں کسی ایک پٹی میں تقریباً ہر توانائی اجازتی ہوگی۔ آپ تصور میں شکل \num{5.6} پر \(\cos(\frac{2\pi n}{N})\) قیمت کے فاصلون پر \(+1(n = 0)\) سے لے کر نیچے \(-1(n = \frac{N}{2})\) تک اور واپس تقریباً \(+1(n = N-1)\) تک جہاں بلوخ جزو ضربی \(e^{iKa}\) دوبارہ چکر شروع کرتا پے لحاظہ \عددی{n} کو مزید بڑھانے سے کوئی نیا حل حاصل نہیں ہو گا لکیریں کھینچ کر دیکھ سکتے ہیں۔ ان لیکیروں میں ہر ایک کا \عددی{f(z)} کے ساتھ تقاطع ایک اجازتی توانائی دیگا۔ ظاہر ہے کہ ہر پٹی میں \عددی{N} حالات پائے جاتے ہیں جو ایک دوسرے کے اتنے قریب ہیں  کہ کسی بھی نقطہ نظر سے انہیں ایک مسلسل خطہ تصور کیا جا سکتا ہے شکل \num{5.7}۔

 ہم نے ابھی تک اپنے مخ فیہ میں ایک الیکٹران رکھا ہے۔ حقیقت میں \عددی{N_q} الیکٹران ہوںگے جہاں ہر ایک جوہر \عددی{q} تعداد کے آزاد الیکٹران مہیہ کرے گا۔ پالی اصولِ منات کے بنا صرف دو الیکٹران کسی ایک فضائی حال کے مکین ہو سکتے ہیں۔ یوں  \(q = 1\) کی صورت میں یہ زمینی حال میں پہلی پٹی کو آدھا  بھریں گے اگر  \(q = 2\) ہو تب یہ پہلی پٹی کو مکمل کریں گے اگر \(q = 3\) ہو یہ دوسری پٹی کو آدھا بھریں گے وغیرہ وغیرہ. تین ابعاد میں اور زیادہ حقیقی مخفیہ کی صورت میں پٹیوں کی ساخت زیادہ پیچیدہ ہوسکتی ہے لیکن اجازتی پٹیاں جنکے بیچ ممنوع درز پائے جاتے ہوں تب بھی ہوگا۔ دوری مخفیہ کی نشانی بھی پٹی ہے۔
 
 اب اگر ایک پٹی مکمل طور پر بھری ہوئی ہو ممنوع خطہ سے گزرتے ہوئے اگلی پٹی تک چھلانگ کے لیئے ایک الیکٹران کو نصبتاً زیادہ توانائی درکار ہوگی ایسا مادہ برقی طور پر غیر موئثل ہوگا۔ اس کے برعکس اگر ایک پٹی پوری طرح بھری ہوئی نہیں ہے تب ایک الیکتران کو بہت معمولی توانائی درکار ہوگی کہ وہ ہیجان ہوسکے  اس طرح کا مادہ عموماً موئثل ہوگا۔ ایک غیر موئثل میں بڑے یا کم \عددی{q} کے چند جوہر کی ملاوٹ سے اگلی بلند پٹی میں چند اظافی الیکٹران رکھ دیئے جاتے ہیں پہلے سے مکمل پُر پٹی میں خول پیدا کیئے جاتے ہیں۔ ان دونوں صورتوں میں ایک کمزور برقی رو گزر سکتا ہے اور ایسے اشیاء نیم موئثل کہلاتے ہیں۔ آزاد الیکٹران نمونہ میں تمام ٹھوس اجسام کو لازماً بہت اچھا موئثل ہونا چاہیئے تھا چونکہ انکے اجازتی توانائیوں کے طیف میں کوئی بڑا وقفہ نہیں پایا جاتا ہے۔ قدرت میں پائے جانے والے ٹھوس اجسام کی برقی موصلیت میں اتنا زیادہ فرق صرف نظریہ پٹی کی مدد سے سمجھا ھا سکتا ہے۔  

