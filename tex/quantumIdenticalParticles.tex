%sec  5.1.1 to just above example 5.1 missing (about 2 pages in total)
%included p213-p215 start of chapter till prob 5.3
%included example 5.1 to prob 5.5

%===================
%p213-p215 unedited. start of chapter till prob 5.3
\باب{متماثل ذرات}\شناخت{باب_متماثل_ذرات}
\section{ دو   زراتی نظام} 

ایک زرہ کے لیے فلحال چکر کو نظر انداز کرتے ہوےٗ  
$ \psi ( r , t ) $
فضا ٗی مہدت r اور وقت t کا تفعال ہو گا۔ دو زراتی نظام کا حال پہلے زرے کے مخاطر 
$ ( r_1 ) $
دوسرے زرے کے مخاطر 
$ ( r_2 ) $
اور وقت کا طابع ہو گا۔ 
\begin{align}   
\psi ( r_1 , r_2 , t ) 
\end{align}
 ہمیشہ کی طرح یہ وقت کے لحاظ سے shrodinger مساوات 
\begin{align}
\iota \hbar \frac{ \partial \psi }{ \partial t } = H \psi
\end{align}
کے تحت ارتقا کرے گا۔ جہاں H مکمل  نظام کا Hamiltonian  ہے۔
\begin{equation}
H = - \frac{ \hbar }{ 2 m_1 } {v_1}^2  -  \frac{ \hbar }{ 2 m_2 } { v_2 }^2 +  v( r_1 , r_2 , t )
\end{equation}
زرہ ایک یا زرہ دو کے محددوں کے لحاظ سے  تفرقات لینے کو 
$ \Delta $
زیر نوشت میں ایک یا دو سے ظاہر کیا گیا ہے۔ زرہ ایک کا ہجم 
$ d^3 r_1 $
اور زرہ دو کا ہجم
$ d^3 r_2 $
پاےٗ جانے کا اہتمال درج ذیل ہو گا۔ 
\begin{align}
| \psi ( r_1 , r_2 , t ) |^2 { d^3 } { r_1 }  { d^3 } { r_2 }
\end{align}
ظاہر ہے کہ 
$ \psi $
کو درج ذیل کے لحاظ سے معمول پر لانا ہو گا۔ 
\begin{align}
\int | \psi ( r_1 , r_2 , t ) |^2 { d^3 } { r_1 } { d^3 } { r_2 } = 1
\end{align}

\newpage

غیر تابع وقت مخفی توانا ٗی کے لیے علیحدگی  متغیرات  سے حلوں کا مکمل سلسلہ حاصل ہوتا ہے ۔ 
\begin{align}
\psi ( r_1 , r_2 , t ) =  \psi ( r_1 , r_2 ) {e}^\frac{ - i E t }{h}  
\end{align}
جہاں فضا ٗی تفعال معاج 
$ \psi $
غیر تابع وقت shroudinger مساوات 
\begin{align}
-\frac{ \hbar }{ 2 m_1 }  {\nabla_1^2} { \psi } - \frac{ \hbar }{ 2 m_2 } \nabla_2^2 { \psi } + V \psi 
\end{align}
جس میں E پورے نظام کی قل توانا ٗی ہے ۔ 

سوال ۔5٫1:
عام طور پر  باہمی مخفی توانا ٗی انحصار صرف 2 زرات کے بیچ صمتیہ
$ r = r_1 - r_2 $
پر ہو گا ۔ ایسی صورت میں متغیرات 
$ r_1  $
اور 
$  r_2  $
کی جگہ نےٗ متغیرات  اور مرکز کمیت
$  R = \frac{ ( m_1 r_1 + m_2 r_2 ) }{ m_1 + m_2 } $
مساوات shroudinger ہوتی ہے ۔ 


(الف)۔  دکھا یٗں کہ 
$ r_1 = R + (\frac{ \mu}{m_1} )r , r_2 = R - (\frac{ \mu }{ m_2 } )r   $ 
اور 
$ \nabla_1 = ( \frac{ \mu }{ m_2 } )\nabla_R + \nabla_r , \nabla_2 = ( \frac{ \mu }{ m_1 } ) \nabla_R - \nabla_r $ 
جہاں 
\begin{align}
\mu = \frac{ m_1 m_2 }{ m_1 + m_2 }
\end{align}
نظام کی تشخیص شدہ کمیت ہے ۔
 
(ب)۔ دکھا ٗیں کہ غیر تابع وقت shroudinger  مساوات درج ذیل رعب اختیار کرتی ہے ۔
\[
- \frac{ \hbar^2 }{ 2 ( m_1 + m_2 ) } { \nabla_R }^2  \psi - \frac{ \hbar^2 }{ 2 \mu } { \nabla_r }^2 { \psi } + V(r) \psi = E \psi
\]

(ج)۔ متغیرات کو 
$ \psi ( R , r ) = { \psi_r }(R) { \psi_r } (r) $
لیتے ہوےٗ علیحدہ کریں ۔ آپ دیکھیں گے کہ 
$ \psi_r $
 ایک ذرہ کی shroudinger مساوات جہاں کمیت 
$ ( m_1 + m_2 ) $
مخفی توانا یٗ صفر ہو اور نظام کی توانا ٗی 
$ E_R $
کو مطمٰن کرتا ہے ۔ جبکہ 
$ \psi_r $
ایک ذرے کی shroudinger مساوات جہاں تخفیف شدہ کمیت ہو ۔  مخفی توانا ٗی  V(r) ہو ، کو مطمٗن کرتا ہے ۔ قل توانا ٗی اور ان کا مجموعہ 
$ E = E_R + E_r $
ہو گا ۔ اس سے ہمیں یہ معلوم ہوتا ہے  کہ مرکز ی کمیت ایک آزاد ذرہ کی طرح حرکت کرتا ہے جبکہ ذرہ ایک کے لحاظ سے ذرہ دو کی  نصبطی حرکت ایسے ہی ہو گی جیسا مخفی توانا ٗی V میں تخفیف شدہ کمیت کا ایک ذرہ  کرتی ہے classical mechanics میں بھی بلکل یہی تحلیل ہو گی    جو 2 اجسام مسلے کو محاصل ایک جسم مسلہ میں  تبدیل کرتی ہے ۔ 

\newpage 

سوال 5٫2 : یوں Hydrogen کے مرکزہ کی حرکت کو درست کرنے کے لیے ہم electron  کی کمیت کی جگہ تخفیف شدہ کمیت استعمال کریں گے 

(الف) ۔ hydrogen کی بندش کی توانا ٗی (مساوات 4٫77) جاننے کی خاطر 
$ \mu $
کی جگہ m استعمال کرنے سے  دو بمعنی ہندسوں تک فیصد خلل کتنا ہو گا۔  

(ب) ۔ hydrogen اور Dueterium کے لیے 
$ ( n=3 ) > ( n=2 ) $ 
 سرخ بالمر لکیروں کے بیچ تفعال معاج میں فرق تلاش کریں ۔ 

(ج) ۔ Positronium کی بندشی توانا ٗی تلاش کریں ۔ proton کی جگہ  positron رکھنے سے positronium پیدا ہو گا ۔ positron کی کمیت electron کی کمیت کے برابر ہو گا جبکہ اس کی علامت Electron کی علامت کے مخالف ہے ۔ 

(د) ۔ فرض کریں آپ muonic hydrogen  جس میں electron کی جگہ ایک muon کی موجودگی کی تصدیق کرنا جانتے ہوں۔ muon کا bar  electron کے bar  کے برابر ہے ۔ جبکہ یہ electron سے 206٫77 گنا زیادہ  کمیت رکھتا ہے ۔  آپ Lyman  \عددی{\alpha}  لکیر  
$ n = 2 $
تا 
$ n = 1 $
کے لیے کس طور معاج پر نظر رکھیں گے ۔ 

سوال 5٫3 : chlorine کے قدرتی دو ہم جا 
$ Cl^35 and Cl^37 $ 
پاےٗ جاتے ہیں ۔ دکھا ٗیں کہ HCL کی لرزشی طیف قریب قریب   جوڑیوں پر مشتمل ہو گا ۔ جن میں فرق 
$ \Delta_v = 7.51 x {10}^-4 v $
جہاں v خرجی photon کی تعدد ہے ۔  ( اشارہ : اس کو ایک Harmonium مرتعیش تصور کریں جہاں 
$ \omega = sqrt{ \frac{ k }{ mu} } $
ہو گا ۔ جہاں 
$ \mu $
تؒخفیف شدہ کمیت ( مساوات 5٫8 ) ہے ۔ جبکہ k دونوں ہمجا کے لیے ایک جیسا ہے ۔ 

%the above is p213-p215
%==============================================
%sec  5.1.1's start to just above example 5.1 missing (about 2 pages in total)

%===========================
%what follows is example 5.1 to prob 5.5


%$$ page 205 $$
\paragraph*{مثال 5.1}
فرض کریں ایک لا متناہی چکور کنواں میں کمیت M کے باہم غیر متعمل دو ذرات جو ایک دوسرے کے اندر سے گزر سکتے ہیں پاۓ جاتے ہیں۔ آپکو فکر کرنے کی ضرورت نہیں کہ عملا کیسے کیا جا سکتا ہے۔ یک ذرہ حالات درج ذیل ہوں گے۔ جہاں $ K=\frac{(\pi)^2 (\hbar)^2}{2m(a)^2 }
$
ہے۔
\begin{align}
 \Psi_{n} (x)=\sqrt{\frac{2}{a}}sin(\frac{n (\Pi)}{a}x), \quad E_{n}=n^2 K 
\end{align}
یہ ذرات قابل ممیز ہونے کی صورت میں جہاں ذرہ $ 1 $ حال   $ n_{1} $ میں اور ذرہ $ 2 $ حال $ n_{2} $  میں ہو مرکب تفاعل موج سادہ حاصل ضرب ہوگا۔
\begin{align}
 \Psi_{n_{1} n_{2}} (x_{1},x_{2})=\Psi_{n_{1}}(x_{1})\Psi_{n_{2}}(x_{2}), \quad E_{n_{1} n_{2}}= ((n_{1})^2+(n_{2})^2)K. 
\end{align}
%$$ page 206 $$
مثال کے طور پر زمینی حال
\begin{align}
 \Psi_{11}=\frac{2}{a}sin(\frac{\pi x_{1}}{a}) sin(\frac{\pi x_{2}}{a}), \quad E_{11}=2K; 
\end{align}
پہلا حجان حال دوچند انحطاطی 
\begin{align}
 \Psi_{12}=\frac{2}{a}sin(\frac{\pi x_{1}}{a}) sin(\frac{2\pi x_{2}}{a}), \quad E_{12}=5K, \\
\Psi_{21}=\frac{2}{a}sin(\frac{2\pi x_{1}}{a}) sin(\frac{\pi x_{2}}{a}), \quad E_{21}=5K; 
\end{align}
ہوگا وغیرہ وغیرہ۔ دونوں ذرات یکساں بوزان ہونے کی صورت میں زمینی حال تبدیل نہیں ہوگا ۔ تاہم پہلا حجان حال جسکی توانائی اب بھی 5K ہوگی غیر انحطاطی ہوگا۔
\begin{align}
\frac{\sqrt{2}}{a}[sin(\frac{\pi x_{1}}{a})sin(\frac{2\pi x_{2}}{a})+ sin(\frac{2 \pi x_{1}}{a})sin(\frac{\pi x_{2}}{a})]
\end{align} 
اور اگر ذرات یکساں فرمیون ہوں تب کوئی حال بھی 2K توانائی کا نہیں ہوگا۔ جبکہ زمینی حال جسکی توانائی 5K ہوگی۔ درج ذیل ہوگا۔
\begin{align}
\frac{\sqrt{2}}{a}[sin(\frac{\pi x_{1}}{a}) sin(\frac{2 \pi x_{2}}{a})- sin(\frac{2 \pi x_{1}}{a}) sin(\frac{\pi x_{2}}{a})], 
\end{align}
\section*{*سوال 5.4}
\paragraph*{(جزوالف)}
اگر $ \Psi_{b} $ اور  $ \Psi_{a} $ عمودی ہوں اور دونوں معمول شدہ ہوں تب مساوات 5.10 میں مستقل 'A' کیا ہوگا؟ 
\paragraph*{(جزوب)}
اگر  $ \Psi_{a} = \Psi_{b} $ ہوں اور یہ معمول شدہ ہوں تب 'A' کیا ہوگا؟ (یہ صورت صرف بوزون کیلۓ ممکن ہے۔)
\section*{سوال 5.5} 
\paragraph*{(جزو الف)}
لامتناہی چکور کنواں میں باہم غیر متعمل دو یکساں ذرات کا ہملتنی لکھیں۔ تصدیق کیجیۓ کہ مثال 5.1 میں دیا گیا فرمیون کا زمینی حال 'H' کا مناسب امتیازی قدر والا امتیازی تفاعل ہوگا۔ 
\paragraph*{(جزوب)}
مثال 5.1 میں دیۓ گۓ حجان حالات سے اگلے دو حالات تفاعل موج اور توانائیاں تینوں صورتوں میں قابل ممیز یکساں موزوں، یکساں فرمیون حاصل کریں۔

%the above is example 5.1 to prob 5.5

%missing sec 5.1.2 till start of prob 5.6

%below is prob 5.6 to prob 5.11
%========================

%page 210
\paragraph*{سوال 5.6}
لامتناہی چکور کنواں میں دو باہم غیر متعامل ذرات جن میں سے ہر ایک کی کمیت M ہے پاۓ جاتے ہیں۔ ان میں سے ایک حال $ \Psi_{n} $ مساوات 2.28 اور دوسرا حال $  \Psi _{l} $ ۔ $ l\neq n $  میں ہے۔ $ (x_{1} - x_{2})^2 $ کا حساب اس صورت لگائیں کہ (الف) یہ غیر قابل ممیز ہوں۔ (ب) یہ یکساں بوزون ہوں اور (ج) یہ یکساں فرمیونز ہوں۔
\paragraph*{سوال 5.7}
فرض کریں آپ کے پاس تین ذرات ہیں جن میں سے ایک حال $ \Psi_{a} $ دوسرا حال $ \Psi_{b} $ اور تیسرا حال $ \Psi_{c} $ میں پاۓ جاتے ہیں۔حالات $ \Psi_{a} $ ، $ \Psi_{b} $ اور $ \Psi_{c} $ کو معیاری عمودی تصور کرتے ہوۓ مساوات 5.15، 5.16 اور 5.17 کی طرز پر تین ذرہ حالات تیار کریں جو (الف) قابل ممیز ذرات کو (ب) یکساں بوزون کو اور (ج) یکساں فرمیونز کوظاہر کرتے ہوں۔ یاد رہے کہ کسی بھی دو ذرات کی جوڑی کے باہمی مبادلہ کے لحاظ سے (ب) کو مکمل طور پر تشاقلی ہوتا ہوگا۔ جبکہ (ج)کو مکمل طور پر خلاف تشاقلی ہونا ہوگا۔تبصرہ: مکمل طور پر خلاف تشاقل تفاعل امواج تیار کرنے کا ایک بہترین طریقہ پایا جاتا ہے۔ سلیٹر نقطہ تیار کریں جس کی پہلی صف  $ \Psi_{a}(x_{1}) $ ، $ \Psi_{b}(x_{1}) $ ، $ \Psi_{c}(x_{1}) $ وغیرہ پر مشتمل ہو۔ اس کی دوسری صف $ \Psi_{a}(x_{2}) $ ، $ \Psi_{b}(x_{2}) $ , $ \Psi_{c}(x_{2}) $ وغیرہ پر مشتمل ہوگی اور اسی طرح اس کے بقایا صف ہوں گے۔ یہ نقطہ کسی بھی تعداد کے ذرات کیلۓ کارآمد ہوگا۔
\section*{حصہ 5.2 جوہر}
ایک مادل جوہر جس کا جوہری عدد Z ہو ایک بھاری مرکزہ جس کا بار Ze ہو اور جس کی کمیت M اور بار e کے Z الیکٹران گھیرتے ہوں پر مشتمل ہوگا۔
%page 211
\begin{align}
H=\sum_{j=1}^{z} { -\frac{h^2 \vartriangle^2 _{j} }{2m}-(\frac{1}{4\Pi\epsilon_{0}})  \frac{Ze^2}{r_{j}} } + \frac{1}{2}(\frac{1}{4\Pi\epsilon_{0}}) \sum_{j\neq 1}^{z} \frac{e^2}{|r_{j} - r_{k} |}. 
\end{align}
ہریہ قوسین میں بند جزو مرکزہ کے برقی میدان میں j الیکٹران کی حرکی توانائی جمع مخفی توانائی کو ظاہر کرتا ہے۔ دوسرا جزو جو ماسواۓ $ j=k $ تمام j اور k مجموعہ پر ہے۔ الیکٹانز میں باہمی قوت دفاع کی بنا مخفی توانائی کو ظاہر کرتاہے۔ جہاں $ \frac{1}{2}
$ اس حقیقت کو درست کرتا ہے کہ مجموعہ لیتے ہوۓ ہر جوڑی کو دو بار گنا جاتا ہے۔ ہمیں تفاعل موج $ \Psi(r_{1} , r_{2}, ...r_{z}) $ کیلۓ درج ذیل شروڈنگر مساوات حل کرنی ہوگی:
\begin{align}
 H\Psi=E\Psi
\end{align}
چونکہ الیکٹران یکساں فرمیون ہیں لہذا تمام حل قابل قبول نہیں ہوںگے۔ صرف وہ حل قابل قبول ہوں گے جن کا مکمل حال ، مقام اور چکر 
\begin{align}
 \Psi(r_{1},r_{2},...,r_{z}) \chi(s_{1},s_{2},...,s_{z}), 
\end{align}
کسی بھی دو الیکٹران کے باہمی مبادلہ کے لحاظ سے خلاف تشاقل ہو۔ بالخصوص کوئی بھی دو الیکٹران ایک ہی حال کے مکین نہیں ہو سکتے ہیں۔ بدقسمتی سے ماسواۓ سادہ ترین صورت $ z=1 $  ہائیڈروجن کیلۓ مساوات 5.24 میں دی گئی ہملتنی کی شروڈنگر مساوات ٹھیک حل نہیں کی جاسکتی ہے۔ کم از کم آج تک کوئی بھی ایسا نہیں کر پایا ہے۔ عملا ہمیں پیچیدہ تخمینی تراکیب استعمال کر نے ہوں گے۔ ان میں سے چند ایک تراکیب پر اگلے بابوں میں غور کیا جاۓ گا۔ ابھی میں الیکٹران کی قوت دفاع کو مکمل طور پر نظر انداز کرتے ہوۓ حلوں کا کیفی تجزیہ پیش کرنا چاہوں گا۔ حصہ 5.2.1 میں ہم ہلیم کی زمینی حال اور ہجان  حالات پر غور کریں گے۔ جبکہ حصہ 5.2.2 میں ہم بالا جواہر کے زمینی حالات پر غور کریں گے۔
\paragraph*{سوال 5.8}
فرض کریں مساوات 5.24 میں دی گئی ہملتنی کے لیۓ آپ شروڈنگر مساوات 5.25 کا حل $ \Psi(r_{1} , r_{2}, r_{3},...r_{z}) $ حصل کر پائیں۔ آپ اس سے ایک ایسا مکمل تشاقل تفاعل ایک مکمل خلاف تشاقل تفاعل کس طرح بنا پائیں گے جو شروڈنگر مساواتکو کسی توانائی کیلۓ مطمئن کرتا ہو۔
%page 212 
\section*{جز حصہ 5.2.1 ہلیم} 
ہائیڈروجن کے بعد سب سے زیادہ جوہر ہلیم $ Z=2 $ ہے۔ اس کا حملتنی
\begin{align}
H= { -\frac{h^2 \vartriangle^2 _{1}}{2m} -\frac{1}{4\Pi\epsilon_{0}} \frac{2e^2}{r_{1}}} + { -\frac{h^2 \vartriangle^2 _{2}}{2m}-\frac{1}{4 \pi \epsilon_{0}} \frac{2e^2}{r_{2}}}+ \frac{1}{4 \pi \epsilon_{0}}\frac{e^2}{|r_{1} -r_{2}|},
\end{align}
بار Ze کے مرکزہ کے دو ہائیڈروجن نما ہملتنی الیکٹران 1 اور دوسرا الیکٹران 2 کے ساتھ دو الیکٹران کے بیچ توانائی دفاع پر مشتمل ہو گا۔ یہ آخری جزو ہماری پریشانیوں کا سبب بنتا ہے۔ اس کو نظرانداز کرتے ہوۓ مساوات شروڈنگر قابل علیحدگی ہوگا۔ اور اس کے حلوں کو نصف بوہر رداس مساوات 4.72 اور چار گنا بوہر توانائیوں مساوات 4.70 کے وجہ نہ سمجھنے کی صورت میں سوال 4.16 پر دوبارہ نظر ڈالیں کہ ہائیڈروجن تفاعلات موج کے حاصل ضرب 
$$ \Psi(r_{1}, r_{2})= \Psi_{nlm} (r_{1}) \Psi_{n^{`} l^{`} m^{`}} (r_{2}), \quad [5.28] $$
کی صورت میں لکھا جا سکتا ہے۔ کل توانائی درج ذیل ہوگی جہاں $ E_{n}=-13.6/n^2 eV $ ہوگا۔
$$ E= 4(E_{n} +E_{n^{`}}), \quad [5.29] $$
بالخصوص زمینی حال درج ذیل ہوگا۔
\begin{align}
\Psi_{0}(r_{1}, r_{2})=\Psi_{100}(r_{1}) \Psi_{100}(r_{2})=\frac{8e^-2(r_{1} + r_{2})/a}{\pi a^3},
\end{align}
مساوات 4.80 دیکھیں اور اس طرح کی توانائی درج ذیل ہوگی۔
$$ E_{0}=8(-13.6eV)=-109 eV . \quad [5.31] $$
چونکہ $ \psi_{0} $ تشاقل تفاعل ہے لہذا چکر حال کو خلاف تشاقل ہونا ہوگا اور یوں ہلیم کے زمینی حال کا تنظیم یکتا ہوگا۔ جس میں چکر ایک دوسرے کے مخالف صف بند ہوں گے۔ حقیقت میں ہلیم کا زمینی حال یقینا یکتا ہے۔ لیکن اس کی توانائی تجرباتی طور پر $ -78.975eV $ حاصل ہوتی ہے۔ جو مساوات 5.31 سے کافی مختلف ہے۔ یہ حیرت کی بات نہیں ہے کہ ہم نے الیکٹران کی توانائی دفاع کو مکمل طور پر نظرانداز کیا جو چھوٹی مقدار نہیں ہے۔ یہ ایک مثبت مقدار ہے۔ مساوات 5.27 دیکھیں۔ جس کو شامل کرتے ہوۓ کل توانائی  -109 کی بجاۓ -79eV ہوگی۔ سوال 5.11 دیکھیں۔ ہلیم ہجان حالات 
$$ \Psi_{nlm} \Psi_{100} . \quad [5.32] $$
ہائیڈروجن زمینی حال میں ایک الیکٹران اور داسرا ہجان حال پر مشتمل ہوگا۔ دانوں الیکٹران کو ہجان حالات میں لے جاتے ہی ایک فورا زمینی حال میں واپس گر کر توانائی خارج کرتا ہے جو دوسرے الیکٹران کو جوہر سے باہر  پھینکتا ہے۔ $ (E>0) $ ۔ یوں ایک آزاد الیکٹران اور ہلیم بارداریہ $ ( He^+ ) $ حصل ہوگا۔ یہ باذات خود ایک
%page 213
دلچسپ نظام ہے جس پر ہم یہاں بات نہیں کر رہے ہیں۔ سوال 5.9 دیکھیں۔ ہم ہمیشہ کی طرح تشاقل اور خلاف تشاقل حالات تیار کر سکتے ہیں۔ مساوات 5.10;  اول الفکر خلاف تشاقل چکر تنظیم یکتا کے ساتھ جاۓ گا۔ جنہیں پیراہلیم کہتے ہیں۔ جبکہ مؤخر ذکر کو تشاقل چکر تنظیم سہتا درکار ہوگی اور انہیں اورتھوہلیم کہتے ہیں۔ زمینی حال لازما پیراہلیم ہوگا جبکہ ہجان حالات دونوں روپ میں پاۓ جاتے ہیں۔ جیسا ہم نے حصہ 5.1.2 میں دریافت کیا۔ تشاقل فضائی حال الیکٹرانز کو قریب لاتا ہے۔ جس کی بنا ہم توقع کرتے ہیں کہ پیراہلیم کی باہم متعامل توانائی زیادہ ہوگی۔ یقینا تجربات سے تصدیق ہوتی ہے کہ اورتھوہلیم کے لحاظ سے پیراہلیم حالات کی توانائی زیادہ ہے۔ شکل 5.2 دیکھیں۔
%figure
%page 214 
\section*{سوال 5.9}
\paragraph*{جزو الف}
فرض کریں کہ آپ ہلیم ایٹم کے دونوں الیکٹرانز کو $ n=2 $ حال میں رکھتے ہیں۔ خارج الیکٹران کی توانائی کیا ہوگی۔
\paragraph*{جزوب}
ہلیم بارداریہ $ He^+ $ کی تیف پر مقداری تجزیہ کریں۔ 
\paragraph*{سوال 5.10}
ہلیم کی توانائیوں کی سطح پر درج ذیل صورت میں کیفی تجزیہ کریں۔ (الف) اگر الیکٹران یکساں بوزون ہوتے۔ (ب) اگر الیکاتران قابل ممیز ہوتے۔ جبکہ ان کی کمیت اور بار نہ ہوتا۔ فرض کریں کہ الیکٹران کا چکر اب بھی $ \frac{1}{2} $ 
ہے اور ان کی تنظیم چکر یکتا اور سہتا ہے۔
\section*{سوال 5.11}
\paragraph*{(جزو الف)}
مساوات 5.30 میں دی گئی حال $ \Psi_{0} $  کیلۓ  $ ((\frac{1}{|r_{1} - r_{2} |))} $ کا حساب لگائیں۔ اشارہ: کری محدداستعمال کرتے ہوۓ قطبی محور کو $ r_{1} $ پر رکھتے ہوۓ تا کہ
\begin{align}
|r_{1}-r_{2}|= \sqrt{(r_{1})^2+ (r_{2})^2 -2r_{1} r_{2}\cos\theta_{2}}.
\end{align}
ہو۔ پہلے 
$ d^3 r_{2} $
  کا تکمل حل کریں۔ زاویہ 
  $ \theta_{2} $
   کے لحاظ سے تکمل آسان ہے۔ بس اتنا یاد رکھیں کہ آپ کو مثبت جزو لینا ہوگا۔ آپ کو  
   $ r_{2} $
    تکمل دو ٹکڑوں میں تقسیم کرنا ہوگا۔ پہلا صفر سے 
    $ r_{1} $ 
    تک اور دوسرا 
     $ r_{1} $ سے $ \infty $ تک۔ جواب: $\frac{5}{4a} $ ۔ 
\paragraph*{جزو ب}
جزو الف کا نتیجہ استعمال کرتے ہوۓ ہلیم کی زمینی حال میں الیکٹران کا باہمی متعامل توانائی کا اندازہ لگائیں۔ اپنے جواب کو الیکٹران وولٹ کی صورت میں پیش کریں۔ اور اس کو $ E_{0} $ مساوات 5.31 کے ساتھ جمع کر کے زمینی حال توانائی کی بہتر تخمیم حصل کریں۔ اس کا موازنہ تجرباتی قیمت کے ساتھ کریں۔ دھیان رہے کہ اب بھی آپ تخمینی تفاعل موج کے ساتھ کام کر رہے ہیں۔ لہذا آپ کا جواب ٹھیک تجرباتی جواب نہیں ہوگا۔  
 
 %the above is prob 5.6 to prob 5.11
 %missing from here till the end of the chapter
 
