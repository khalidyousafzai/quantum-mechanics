%sec  5.1.1 to just above example 5.1 missing (about 2 pages in total)
%included p213-p215 start of chapter till prob 5.3
%included example 5.1 to prob 5.5
%missing sec 5.1.2 till start of prob 5.6 and prob 5.23 till end
%===================
%p213-p215 unedited. start of chapter till prob 5.3
\باب{متماثل ذرات}\شناخت{باب_متماثل_ذرات}
\section{ دو   زراتی نظام} 

ایک زرہ کے لیے فلحال چکر کو نظر انداز کرتے ہوےٗ  
$ \psi ( r , t ) $
فضا ٗی مہدت r اور وقت t کا تفعال ہو گا۔ دو زراتی نظام کا حال پہلے زرے کے مخاطر 
$ ( r_1 ) $
دوسرے زرے کے مخاطر 
$ ( r_2 ) $
اور وقت کا طابع ہو گا۔ 
\begin{align}   
\psi ( r_1 , r_2 , t ) 
\end{align}
 ہمیشہ کی طرح یہ وقت کے لحاظ سے shrodinger مساوات 
\begin{align}
\iota \hbar \frac{ \partial \psi }{ \partial t } = H \psi
\end{align}
کے تحت ارتقا کرے گا۔ جہاں H مکمل  نظام کا Hamiltonian  ہے۔
\begin{equation}
H = - \frac{ \hbar }{ 2 m_1 } {v_1}^2  -  \frac{ \hbar }{ 2 m_2 } { v_2 }^2 +  v( r_1 , r_2 , t )
\end{equation}
زرہ ایک یا زرہ دو کے محددوں کے لحاظ سے  تفرقات لینے کو 
$ \Delta $
زیر نوشت میں ایک یا دو سے ظاہر کیا گیا ہے۔ زرہ ایک کا ہجم 
$ d^3 r_1 $
اور زرہ دو کا ہجم
$ d^3 r_2 $
پاےٗ جانے کا اہتمال درج ذیل ہو گا۔ 
\begin{align}
| \psi ( r_1 , r_2 , t ) |^2 { d^3 } { r_1 }  { d^3 } { r_2 }
\end{align}
ظاہر ہے کہ 
$ \psi $
کو درج ذیل کے لحاظ سے معمول پر لانا ہو گا۔ 
\begin{align}
\int | \psi ( r_1 , r_2 , t ) |^2 { d^3 } { r_1 } { d^3 } { r_2 } = 1
\end{align}

\newpage

غیر تابع وقت مخفی توانا ٗی کے لیے علیحدگی  متغیرات  سے حلوں کا مکمل سلسلہ حاصل ہوتا ہے ۔ 
\begin{align}
\psi ( r_1 , r_2 , t ) =  \psi ( r_1 , r_2 ) {e}^\frac{ - i E t }{h}  
\end{align}
جہاں فضا ٗی تفعال معاج 
$ \psi $
غیر تابع وقت shroudinger مساوات 
\begin{align}
-\frac{ \hbar }{ 2 m_1 }  {\nabla_1^2} { \psi } - \frac{ \hbar }{ 2 m_2 } \nabla_2^2 { \psi } + V \psi 
\end{align}
جس میں E پورے نظام کی قل توانا ٗی ہے ۔ 

سوال ۔5٫1:
عام طور پر  باہمی مخفی توانا ٗی انحصار صرف 2 زرات کے بیچ صمتیہ
$ r = r_1 - r_2 $
پر ہو گا ۔ ایسی صورت میں متغیرات 
$ r_1  $
اور 
$  r_2  $
کی جگہ نےٗ متغیرات  اور مرکز کمیت
$  R = \frac{ ( m_1 r_1 + m_2 r_2 ) }{ m_1 + m_2 } $
مساوات shroudinger ہوتی ہے ۔ 


(الف)۔  دکھا یٗں کہ 
$ r_1 = R + (\frac{ \mu}{m_1} )r , r_2 = R - (\frac{ \mu }{ m_2 } )r   $ 
اور 
$ \nabla_1 = ( \frac{ \mu }{ m_2 } )\nabla_R + \nabla_r , \nabla_2 = ( \frac{ \mu }{ m_1 } ) \nabla_R - \nabla_r $ 
جہاں 
\begin{align}
\mu = \frac{ m_1 m_2 }{ m_1 + m_2 }
\end{align}
نظام کی تشخیص شدہ کمیت ہے ۔
 
(ب)۔ دکھا ٗیں کہ غیر تابع وقت shroudinger  مساوات درج ذیل رعب اختیار کرتی ہے ۔
\[
- \frac{ \hbar^2 }{ 2 ( m_1 + m_2 ) } { \nabla_R }^2  \psi - \frac{ \hbar^2 }{ 2 \mu } { \nabla_r }^2 { \psi } + V(r) \psi = E \psi
\]

(ج)۔ متغیرات کو 
$ \psi ( R , r ) = { \psi_r }(R) { \psi_r } (r) $
لیتے ہوےٗ علیحدہ کریں ۔ آپ دیکھیں گے کہ 
$ \psi_r $
 ایک ذرہ کی shroudinger مساوات جہاں کمیت 
$ ( m_1 + m_2 ) $
مخفی توانا یٗ صفر ہو اور نظام کی توانا ٗی 
$ E_R $
کو مطمٰن کرتا ہے ۔ جبکہ 
$ \psi_r $
ایک ذرے کی shroudinger مساوات جہاں تخفیف شدہ کمیت ہو ۔  مخفی توانا ٗی  V(r) ہو ، کو مطمٗن کرتا ہے ۔ قل توانا ٗی اور ان کا مجموعہ 
$ E = E_R + E_r $
ہو گا ۔ اس سے ہمیں یہ معلوم ہوتا ہے  کہ مرکز ی کمیت ایک آزاد ذرہ کی طرح حرکت کرتا ہے جبکہ ذرہ ایک کے لحاظ سے ذرہ دو کی  نصبطی حرکت ایسے ہی ہو گی جیسا مخفی توانا ٗی V میں تخفیف شدہ کمیت کا ایک ذرہ  کرتی ہے classical mechanics میں بھی بلکل یہی تحلیل ہو گی    جو 2 اجسام مسلے کو محاصل ایک جسم مسلہ میں  تبدیل کرتی ہے ۔ 

\newpage 

سوال 5٫2 : یوں Hydrogen کے مرکزہ کی حرکت کو درست کرنے کے لیے ہم electron  کی کمیت کی جگہ تخفیف شدہ کمیت استعمال کریں گے 

(الف) ۔ hydrogen کی بندش کی توانا ٗی (مساوات 4٫77) جاننے کی خاطر 
$ \mu $
کی جگہ m استعمال کرنے سے  دو بمعنی ہندسوں تک فیصد خلل کتنا ہو گا۔  

(ب) ۔ hydrogen اور Dueterium کے لیے 
$ ( n=3 ) > ( n=2 ) $ 
 سرخ بالمر لکیروں کے بیچ تفعال معاج میں فرق تلاش کریں ۔ 

(ج) ۔ Positronium کی بندشی توانا ٗی تلاش کریں ۔ proton کی جگہ  positron رکھنے سے positronium پیدا ہو گا ۔ positron کی کمیت electron کی کمیت کے برابر ہو گا جبکہ اس کی علامت Electron کی علامت کے مخالف ہے ۔ 

(د) ۔ فرض کریں آپ muonic hydrogen  جس میں electron کی جگہ ایک muon کی موجودگی کی تصدیق کرنا جانتے ہوں۔ muon کا bar  electron کے bar  کے برابر ہے ۔ جبکہ یہ electron سے 206٫77 گنا زیادہ  کمیت رکھتا ہے ۔  آپ Lyman  \عددی{\alpha}  لکیر  
$ n = 2 $
تا 
$ n = 1 $
کے لیے کس طور معاج پر نظر رکھیں گے ۔ 

سوال 5٫3 : chlorine کے قدرتی دو ہم جا 
$ Cl^35 and Cl^37 $ 
پاےٗ جاتے ہیں ۔ دکھا ٗیں کہ HCL کی لرزشی طیف قریب قریب   جوڑیوں پر مشتمل ہو گا ۔ جن میں فرق 
$ \Delta_v = 7.51 x {10}^-4 v $
جہاں v خرجی photon کی تعدد ہے ۔  ( اشارہ : اس کو ایک Harmonium مرتعیش تصور کریں جہاں 
$ \omega = sqrt{ \frac{ k }{ mu} } $
ہو گا ۔ جہاں 
$ \mu $
تؒخفیف شدہ کمیت ( مساوات 5٫8 ) ہے ۔ جبکہ k دونوں ہمجا کے لیے ایک جیسا ہے ۔ 

%the above is p213-p215
%==============================================
%sec  5.1.1's start to just above example 5.1 missing (about 2 pages in total)

%===========================
%what follows is example 5.1 to prob 5.5


%$$ page 205 $$
\paragraph*{مثال 5.1}
فرض کریں ایک لا متناہی چکور کنواں میں کمیت M کے باہم غیر متعمل دو ذرات جو ایک دوسرے کے اندر سے گزر سکتے ہیں پاۓ جاتے ہیں۔ آپکو فکر کرنے کی ضرورت نہیں کہ عملا کیسے کیا جا سکتا ہے۔ یک ذرہ حالات درج ذیل ہوں گے۔ جہاں $ K=\frac{(\pi)^2 (\hbar)^2}{2m(a)^2 }
$
ہے۔
\begin{align}
 \Psi_{n} (x)=\sqrt{\frac{2}{a}}sin(\frac{n (\Pi)}{a}x), \quad E_{n}=n^2 K 
\end{align}
یہ ذرات قابل ممیز ہونے کی صورت میں جہاں ذرہ $ 1 $ حال   $ n_{1} $ میں اور ذرہ $ 2 $ حال $ n_{2} $  میں ہو مرکب تفاعل موج سادہ حاصل ضرب ہوگا۔
\begin{align}
 \Psi_{n_{1} n_{2}} (x_{1},x_{2})=\Psi_{n_{1}}(x_{1})\Psi_{n_{2}}(x_{2}), \quad E_{n_{1} n_{2}}= ((n_{1})^2+(n_{2})^2)K. 
\end{align}
%$$ page 206 $$
مثال کے طور پر زمینی حال
\begin{align}
 \Psi_{11}=\frac{2}{a}sin(\frac{\pi x_{1}}{a}) sin(\frac{\pi x_{2}}{a}), \quad E_{11}=2K; 
\end{align}
پہلا حجان حال دوچند انحطاطی 
\begin{align}
 \Psi_{12}=\frac{2}{a}sin(\frac{\pi x_{1}}{a}) sin(\frac{2\pi x_{2}}{a}), \quad E_{12}=5K, \\
\Psi_{21}=\frac{2}{a}sin(\frac{2\pi x_{1}}{a}) sin(\frac{\pi x_{2}}{a}), \quad E_{21}=5K; 
\end{align}
ہوگا وغیرہ وغیرہ۔ دونوں ذرات یکساں بوزان ہونے کی صورت میں زمینی حال تبدیل نہیں ہوگا ۔ تاہم پہلا حجان حال جسکی توانائی اب بھی 5K ہوگی غیر انحطاطی ہوگا۔
\begin{align}
\frac{\sqrt{2}}{a}[sin(\frac{\pi x_{1}}{a})sin(\frac{2\pi x_{2}}{a})+ sin(\frac{2 \pi x_{1}}{a})sin(\frac{\pi x_{2}}{a})]
\end{align} 
اور اگر ذرات یکساں فرمیون ہوں تب کوئی حال بھی 2K توانائی کا نہیں ہوگا۔ جبکہ زمینی حال جسکی توانائی 5K ہوگی۔ درج ذیل ہوگا۔
\begin{align}
\frac{\sqrt{2}}{a}[sin(\frac{\pi x_{1}}{a}) sin(\frac{2 \pi x_{2}}{a})- sin(\frac{2 \pi x_{1}}{a}) sin(\frac{\pi x_{2}}{a})], 
\end{align}
\section*{*سوال 5.4}
\paragraph*{(جزوالف)}
اگر $ \Psi_{b} $ اور  $ \Psi_{a} $ عمودی ہوں اور دونوں معمول شدہ ہوں تب مساوات 5.10 میں مستقل 'A' کیا ہوگا؟ 
\paragraph*{(جزوب)}
اگر  $ \Psi_{a} = \Psi_{b} $ ہوں اور یہ معمول شدہ ہوں تب 'A' کیا ہوگا؟ (یہ صورت صرف بوزون کیلۓ ممکن ہے۔)
\section*{سوال 5.5} 
\paragraph*{(جزو الف)}
لامتناہی چکور کنواں میں باہم غیر متعمل دو یکساں ذرات کا ہملتنی لکھیں۔ تصدیق کیجیۓ کہ مثال 5.1 میں دیا گیا فرمیون کا زمینی حال 'H' کا مناسب امتیازی قدر والا امتیازی تفاعل ہوگا۔ 
\paragraph*{(جزوب)}
مثال 5.1 میں دیۓ گۓ حجان حالات سے اگلے دو حالات تفاعل موج اور توانائیاں تینوں صورتوں میں قابل ممیز یکساں موزوں، یکساں فرمیون حاصل کریں۔

%the above is example 5.1 to prob 5.5

%missing sec 5.1.2 till start of prob 5.6

%below is prob 5.6 to prob 5.23 
%========================

%page 210
\paragraph*{سوال 5.6}
لامتناہی چکور کنواں میں دو باہم غیر متعامل ذرات جن میں سے ہر ایک کی کمیت M ہے پاۓ جاتے ہیں۔ ان میں سے ایک حال $ \Psi_{n} $ مساوات 2.28 اور دوسرا حال $  \Psi _{l} $ ۔ $ l\neq n $  میں ہے۔ $ (x_{1} - x_{2})^2 $ کا حساب اس صورت لگائیں کہ (الف) یہ غیر قابل ممیز ہوں۔ (ب) یہ یکساں بوزون ہوں اور (ج) یہ یکساں فرمیونز ہوں۔
\paragraph*{سوال 5.7}
فرض کریں آپ کے پاس تین ذرات ہیں جن میں سے ایک حال $ \Psi_{a} $ دوسرا حال $ \Psi_{b} $ اور تیسرا حال $ \Psi_{c} $ میں پاۓ جاتے ہیں۔حالات $ \Psi_{a} $ ، $ \Psi_{b} $ اور $ \Psi_{c} $ کو معیاری عمودی تصور کرتے ہوۓ مساوات 5.15، 5.16 اور 5.17 کی طرز پر تین ذرہ حالات تیار کریں جو (الف) قابل ممیز ذرات کو (ب) یکساں بوزون کو اور (ج) یکساں فرمیونز کوظاہر کرتے ہوں۔ یاد رہے کہ کسی بھی دو ذرات کی جوڑی کے باہمی مبادلہ کے لحاظ سے (ب) کو مکمل طور پر تشاقلی ہوتا ہوگا۔ جبکہ (ج)کو مکمل طور پر خلاف تشاقلی ہونا ہوگا۔تبصرہ: مکمل طور پر خلاف تشاقل تفاعل امواج تیار کرنے کا ایک بہترین طریقہ پایا جاتا ہے۔ سلیٹر نقطہ تیار کریں جس کی پہلی صف  $ \Psi_{a}(x_{1}) $ ، $ \Psi_{b}(x_{1}) $ ، $ \Psi_{c}(x_{1}) $ وغیرہ پر مشتمل ہو۔ اس کی دوسری صف $ \Psi_{a}(x_{2}) $ ، $ \Psi_{b}(x_{2}) $ , $ \Psi_{c}(x_{2}) $ وغیرہ پر مشتمل ہوگی اور اسی طرح اس کے بقایا صف ہوں گے۔ یہ نقطہ کسی بھی تعداد کے ذرات کیلۓ کارآمد ہوگا۔
\section*{حصہ 5.2 جوہر}
ایک مادل جوہر جس کا جوہری عدد Z ہو ایک بھاری مرکزہ جس کا بار Ze ہو اور جس کی کمیت M اور بار e کے Z الیکٹران گھیرتے ہوں پر مشتمل ہوگا۔
%page 211
\begin{align}
H=\sum_{j=1}^{z} { -\frac{h^2 \vartriangle^2 _{j} }{2m}-(\frac{1}{4\Pi\epsilon_{0}})  \frac{Ze^2}{r_{j}} } + \frac{1}{2}(\frac{1}{4\Pi\epsilon_{0}}) \sum_{j\neq 1}^{z} \frac{e^2}{|r_{j} - r_{k} |}. 
\end{align}
ہریہ قوسین میں بند جزو مرکزہ کے برقی میدان میں j الیکٹران کی حرکی توانائی جمع مخفی توانائی کو ظاہر کرتا ہے۔ دوسرا جزو جو ماسواۓ $ j=k $ تمام j اور k مجموعہ پر ہے۔ الیکٹانز میں باہمی قوت دفاع کی بنا مخفی توانائی کو ظاہر کرتاہے۔ جہاں $ \frac{1}{2}
$ اس حقیقت کو درست کرتا ہے کہ مجموعہ لیتے ہوۓ ہر جوڑی کو دو بار گنا جاتا ہے۔ ہمیں تفاعل موج $ \Psi(r_{1} , r_{2}, ...r_{z}) $ کیلۓ درج ذیل شروڈنگر مساوات حل کرنی ہوگی:
\begin{align}
 H\Psi=E\Psi
\end{align}
چونکہ الیکٹران یکساں فرمیون ہیں لہذا تمام حل قابل قبول نہیں ہوںگے۔ صرف وہ حل قابل قبول ہوں گے جن کا مکمل حال ، مقام اور چکر 
\begin{align}
 \Psi(r_{1},r_{2},...,r_{z}) \chi(s_{1},s_{2},...,s_{z}), 
\end{align}
کسی بھی دو الیکٹران کے باہمی مبادلہ کے لحاظ سے خلاف تشاقل ہو۔ بالخصوص کوئی بھی دو الیکٹران ایک ہی حال کے مکین نہیں ہو سکتے ہیں۔ بدقسمتی سے ماسواۓ سادہ ترین صورت $ z=1 $  ہائیڈروجن کیلۓ مساوات 5.24 میں دی گئی ہملتنی کی شروڈنگر مساوات ٹھیک حل نہیں کی جاسکتی ہے۔ کم از کم آج تک کوئی بھی ایسا نہیں کر پایا ہے۔ عملا ہمیں پیچیدہ تخمینی تراکیب استعمال کر نے ہوں گے۔ ان میں سے چند ایک تراکیب پر اگلے بابوں میں غور کیا جاۓ گا۔ ابھی میں الیکٹران کی قوت دفاع کو مکمل طور پر نظر انداز کرتے ہوۓ حلوں کا کیفی تجزیہ پیش کرنا چاہوں گا۔ حصہ 5.2.1 میں ہم ہلیم کی زمینی حال اور ہجان  حالات پر غور کریں گے۔ جبکہ حصہ 5.2.2 میں ہم بالا جواہر کے زمینی حالات پر غور کریں گے۔
\paragraph*{سوال 5.8}
فرض کریں مساوات 5.24 میں دی گئی ہملتنی کے لیۓ آپ شروڈنگر مساوات 5.25 کا حل $ \Psi(r_{1} , r_{2}, r_{3},...r_{z}) $ حصل کر پائیں۔ آپ اس سے ایک ایسا مکمل تشاقل تفاعل ایک مکمل خلاف تشاقل تفاعل کس طرح بنا پائیں گے جو شروڈنگر مساواتکو کسی توانائی کیلۓ مطمئن کرتا ہو۔
%page 212 
\section*{جز حصہ 5.2.1 ہلیم} 
ہائیڈروجن کے بعد سب سے زیادہ جوہر ہلیم $ Z=2 $ ہے۔ اس کا حملتنی
\begin{align}
H= { -\frac{h^2 \vartriangle^2 _{1}}{2m} -\frac{1}{4\Pi\epsilon_{0}} \frac{2e^2}{r_{1}}} + { -\frac{h^2 \vartriangle^2 _{2}}{2m}-\frac{1}{4 \pi \epsilon_{0}} \frac{2e^2}{r_{2}}}+ \frac{1}{4 \pi \epsilon_{0}}\frac{e^2}{|r_{1} -r_{2}|},
\end{align}
بار Ze کے مرکزہ کے دو ہائیڈروجن نما ہملتنی الیکٹران 1 اور دوسرا الیکٹران 2 کے ساتھ دو الیکٹران کے بیچ توانائی دفاع پر مشتمل ہو گا۔ یہ آخری جزو ہماری پریشانیوں کا سبب بنتا ہے۔ اس کو نظرانداز کرتے ہوۓ مساوات شروڈنگر قابل علیحدگی ہوگا۔ اور اس کے حلوں کو نصف بوہر رداس مساوات 4.72 اور چار گنا بوہر توانائیوں مساوات 4.70 کے وجہ نہ سمجھنے کی صورت میں سوال 4.16 پر دوبارہ نظر ڈالیں کہ ہائیڈروجن تفاعلات موج کے حاصل ضرب 
$$ \Psi(r_{1}, r_{2})= \Psi_{nlm} (r_{1}) \Psi_{n^{`} l^{`} m^{`}} (r_{2}), \quad [5.28] $$
کی صورت میں لکھا جا سکتا ہے۔ کل توانائی درج ذیل ہوگی جہاں $ E_{n}=-13.6/n^2 eV $ ہوگا۔
$$ E= 4(E_{n} +E_{n^{`}}), \quad [5.29] $$
بالخصوص زمینی حال درج ذیل ہوگا۔
\begin{align}
\Psi_{0}(r_{1}, r_{2})=\Psi_{100}(r_{1}) \Psi_{100}(r_{2})=\frac{8e^-2(r_{1} + r_{2})/a}{\pi a^3},
\end{align}
مساوات 4.80 دیکھیں اور اس طرح کی توانائی درج ذیل ہوگی۔
$$ E_{0}=8(-13.6eV)=-109 eV . \quad [5.31] $$
چونکہ $ \psi_{0} $ تشاقل تفاعل ہے لہذا چکر حال کو خلاف تشاقل ہونا ہوگا اور یوں ہلیم کے زمینی حال کا تنظیم یکتا ہوگا۔ جس میں چکر ایک دوسرے کے مخالف صف بند ہوں گے۔ حقیقت میں ہلیم کا زمینی حال یقینا یکتا ہے۔ لیکن اس کی توانائی تجرباتی طور پر $ -78.975eV $ حاصل ہوتی ہے۔ جو مساوات 5.31 سے کافی مختلف ہے۔ یہ حیرت کی بات نہیں ہے کہ ہم نے الیکٹران کی توانائی دفاع کو مکمل طور پر نظرانداز کیا جو چھوٹی مقدار نہیں ہے۔ یہ ایک مثبت مقدار ہے۔ مساوات 5.27 دیکھیں۔ جس کو شامل کرتے ہوۓ کل توانائی  -109 کی بجاۓ -79eV ہوگی۔ سوال 5.11 دیکھیں۔ ہلیم ہجان حالات 
$$ \Psi_{nlm} \Psi_{100} . \quad [5.32] $$
ہائیڈروجن زمینی حال میں ایک الیکٹران اور داسرا ہجان حال پر مشتمل ہوگا۔ دانوں الیکٹران کو ہجان حالات میں لے جاتے ہی ایک فورا زمینی حال میں واپس گر کر توانائی خارج کرتا ہے جو دوسرے الیکٹران کو جوہر سے باہر  پھینکتا ہے۔ $ (E>0) $ ۔ یوں ایک آزاد الیکٹران اور ہلیم بارداریہ $ ( He^+ ) $ حصل ہوگا۔ یہ باذات خود ایک
%page 213
دلچسپ نظام ہے جس پر ہم یہاں بات نہیں کر رہے ہیں۔ سوال 5.9 دیکھیں۔ ہم ہمیشہ کی طرح تشاقل اور خلاف تشاقل حالات تیار کر سکتے ہیں۔ مساوات 5.10;  اول الفکر خلاف تشاقل چکر تنظیم یکتا کے ساتھ جاۓ گا۔ جنہیں پیراہلیم کہتے ہیں۔ جبکہ مؤخر ذکر کو تشاقل چکر تنظیم سہتا درکار ہوگی اور انہیں اورتھوہلیم کہتے ہیں۔ زمینی حال لازما پیراہلیم ہوگا جبکہ ہجان حالات دونوں روپ میں پاۓ جاتے ہیں۔ جیسا ہم نے حصہ 5.1.2 میں دریافت کیا۔ تشاقل فضائی حال الیکٹرانز کو قریب لاتا ہے۔ جس کی بنا ہم توقع کرتے ہیں کہ پیراہلیم کی باہم متعامل توانائی زیادہ ہوگی۔ یقینا تجربات سے تصدیق ہوتی ہے کہ اورتھوہلیم کے لحاظ سے پیراہلیم حالات کی توانائی زیادہ ہے۔ شکل 5.2 دیکھیں۔
%figure
%page 214 
\section*{سوال 5.9}
\paragraph*{جزو الف}
فرض کریں کہ آپ ہلیم ایٹم کے دونوں الیکٹرانز کو $ n=2 $ حال میں رکھتے ہیں۔ خارج الیکٹران کی توانائی کیا ہوگی۔
\paragraph*{جزوب}
ہلیم بارداریہ $ He^+ $ کی تیف پر مقداری تجزیہ کریں۔ 
\paragraph*{سوال 5.10}
ہلیم کی توانائیوں کی سطح پر درج ذیل صورت میں کیفی تجزیہ کریں۔ (الف) اگر الیکٹران یکساں بوزون ہوتے۔ (ب) اگر الیکاتران قابل ممیز ہوتے۔ جبکہ ان کی کمیت اور بار نہ ہوتا۔ فرض کریں کہ الیکٹران کا چکر اب بھی $ \frac{1}{2} $ 
ہے اور ان کی تنظیم چکر یکتا اور سہتا ہے۔
\section*{سوال 5.11}
\paragraph*{(جزو الف)}
مساوات 5.30 میں دی گئی حال $ \Psi_{0} $  کیلۓ  $ ((\frac{1}{|r_{1} - r_{2} |))} $ کا حساب لگائیں۔ اشارہ: کری محدداستعمال کرتے ہوۓ قطبی محور کو $ r_{1} $ پر رکھتے ہوۓ تا کہ
\begin{align}
|r_{1}-r_{2}|= \sqrt{(r_{1})^2+ (r_{2})^2 -2r_{1} r_{2}\cos\theta_{2}}.
\end{align}
ہو۔ پہلے 
$ d^3 r_{2} $
  کا تکمل حل کریں۔ زاویہ 
  $ \theta_{2} $
   کے لحاظ سے تکمل آسان ہے۔ بس اتنا یاد رکھیں کہ آپ کو مثبت جزو لینا ہوگا۔ آپ کو  
   $ r_{2} $
    تکمل دو ٹکڑوں میں تقسیم کرنا ہوگا۔ پہلا صفر سے 
    $ r_{1} $ 
    تک اور دوسرا 
     $ r_{1} $ سے $ \infty $ تک۔ جواب: $\frac{5}{4a} $ ۔ 
\paragraph*{جزو ب}
جزو الف کا نتیجہ استعمال کرتے ہوۓ ہلیم کی زمینی حال میں الیکٹران کا باہمی متعامل توانائی کا اندازہ لگائیں۔ اپنے جواب کو الیکٹران وولٹ کی صورت میں پیش کریں۔ اور اس کو $ E_{0} $ مساوات 5.31 کے ساتھ جمع کر کے زمینی حال توانائی کی بہتر تخمیم حصل کریں۔ اس کا موازنہ تجرباتی قیمت کے ساتھ کریں۔ دھیان رہے کہ اب بھی آپ تخمینی تفاعل موج کے ساتھ کام کر رہے ہیں۔ لہذا آپ کا جواب ٹھیک تجرباتی جواب نہیں ہوگا۔  



\جزوحصہ{دوری جدول}
بھاری جوہروں کے زمینی حال الیکٹرانی تنظیم اسی طرح جوڑ کر حاصل کی جاتی ہے۔ پہلی تخمین کی حد میں انکی باہمی توانائی دفع کو مکمل طور پر نظرانداز کرتے ہوئے بار \عددی{Z_e} کے مرکزہ کے کولمب مخفیہ میں یک ذرہ ہائڈروجن حالات \عددی{(n, l, m)} جنہیں مدارچے کہتے ہیں کہ انفرادی الیکٹران مکین ہوں گے۔ اگر الیکٹران بوزان یا قابلِ ممیز ذرات ہوتے تب یہ زمینی حال \عددی{(1, 0, 0)} گر جاتے اور کیمیا انتی دلچسپ نہ ہوتی۔ حقیقت میں الیکٹران یکساں فرمیان ہے جن پر پولی اصول منات لاگو ہتا ہے لحاظہ کسی ایک مدارچہ میں صرف دو الیکٹران رہ سکتے ہیں ایک ہم میدان اور ایک خلاف میدان بلکہ یہ کہنا زیادہ درست کہ یکتا تنظیم میں الیکٹران رہ سکتے ہیں۔ کسی بھی \عددی{n} کی قیمت کے لیئے \عددی{n^2} ہائڈروجینی تفاعلات موج پائے جاتے ہیں جن میں سے ہر ایک کی توانائی \عددی{E_n} ہوگی یوں \عددی{n=1} خول میں دو الیکٹرانوں کی جگہ \عددی{n=2} خول میں آٹھ \عددی{n=3} میں اٹھارہ اور \عددی{n}ویں خول میں \عددی{2n^2} الیکٹرانوں کی جگہ ہوگی۔ کیفی طور پر بات کرتے ہئے دوری جدول کے اُفکی صف انفرادی خول کو بھرنے کے مترادف ہے اگرثہ یہ پوری کہانی نہیں ہے چونکہ ایسا ہونے کی صورت میں انکی لمبائیاں \عددی{2, 8, 18, 32, 50,} وغیرہ ہوتی نا کہ \عددی{2, 8, 8, 18, 18} وغیرہ ہم جلد دیکھیں گے کہ الیکٹرانوں کی باہمی توانائی دفع اس شمار کو کس طرح خراب کرتا ہے۔

ہیلیم کا \عددی{n=1} خول مکمل طور پر بھرا ہوگا لحاظہ اگلا جوہر لیتھیم \عددی{Z=3} کو ایک الیکٹران \عددی{n=2} خول میں رکھنا ہوگا۔ اب \عددی{n=2} کی صورت میں \عددی{l=0} یا \عددی{l=1} ہوسکتا ہے۔ تیسا الیکٹران ان میں سے کس ایک کا انتخاب کرے گا؟ چونکہ بوہر توانائی \عددی{n} پر منحصر ہوتی ہے نا کہ \عددی{l} پر لھاظہ الیکٹران کا باہمی عمل نہ ہونے کی صورت میں ان دونوں کی توانائی ایک دوسرے جیسی ہوگی۔ تاہم درج ذیل وجہ کی بنا الیکٹران کی توانائی دفا
 \(l\) 
کی کم سے کم قیمت کی طرف داری کرتی ہے۔  زاویائی معیارے حرکت الیکٹران کو بےرونی روح دھکیلنے کی کوشش کرتا ہے اور الیکٹران جتنا مرکزا سے دور ہوتا ہے اتنا ہی یہ مرکزا بہتر چھپاتا ہے۔ ہم کہہ سکتے ہیں کہ اندرونی الیکٹران کو مرکزا کا پورا 
\(Ze\)
نظر آتا ہے جب کہ بےرونی الیکٹران کو مشکل سے 
\(e\)
 سے زیادہ موثر  نظر آتا ہے ۔یوں کسی بھی ایک ہول میں کم سے کم توانائی کا حال یعنی دوسرے لفظوں میں سب سے سخت مقید الیکٹران
 \( l=0\) 
 ہوگا ۔ اور بڑھتے
  \(l\) 
 کے ساتھ توانائی بڑھے گی اس طرح لتیم میں تیسرا الیکٹران مدارجہ
  \((2,0,0)\)
 کا مقید ہوگا۔ اگلا جوہر بیریلیم جس کا
  \(Z=4\) 
 ہے اسی حال میں ہوگا لیکن اس کا چکر مخالف رخ ہوگا لیکن بوران
  \(Z=5\) 
 کو
 \( l=1\) 
 استعمال کرنا ہوگا۔ 
 %==================
 اسی طرح چلتے ہوئے ہم  نیین
 \( Z=10\) 
 تک پہنچتے ہیں جہاں
  \(n=2\) 
 ہول مکمل بھرا ہوگا اور ہم دوری جدول کی اگلی صف کو پہنچ کر
  \(n= 3\) 
 ہول کو بھرنا شروع کرتے ہیں ۔ آغاز میں دو جوہر سوڈیم اور میگنیشیم ہیں جنکا
 \( l=0\) 
 ہے اور اس کے بعد الیمینیم سے آرگان تک چھ ایسے جوہر ہیں جن کے لیے 
 \(L=1\)
  ہوگا۔ آرگان کے بعد ہم توقع کرتے ہیں کہ دس ایسے جوہر پائے جائنگے جن کے لیے 
  \(n=3\) 
  اور 
  \(l=2\) 
  ہوگا البتہ یہاں پہنچ کر اندرونی الیکٹران مرکزا کو اتنی خوش اسلوبی کے ساتھ پردہ کرتے ہیں کہ یہ اگلے ہول کو بھی ڈنگتا ہے لہذا پوٹیشیم
\(( Z=19)\) 
  اور کیلشیم
  \((Z=20)\)
  ،
  \((n=3),(l=2)\)
 کی بجائے
 \( (n=4),( L=0 )\)
 منتخب کرتے ہیں  ۔ اس کے بعد ہم نیچے اتر کر سکینڈیم سے زنک تک کے جوہر اٹھاتے ہیں جن کے لیے
 \( n=3\)
  اور
\( l=2\)
   ہوگا ۔ اس کے بعد گیلیم سے کرپٹان تک
\( n=4، l=1\)
    ہوگا جس کے آخر میں ہم دوبارہ قبل از وقت اگلی صف
\( n=5\) 
    کو چھلانگ لگاتے ہیں اور بعد میں واپس اتر کر
\( n= 4\)  
    ہول کے ۔ وہ مدارجے جن کے لیے
\( l=2، l=3\) 
    ہوں پر کرتے ہیں ۔ یہاں جوہری حالات کی قدیم نام جنہیں تمام ماہر کیمیات اور تبیات کے زیادہ تر ماہرین استعمال کرتے ہیں پر تبصرہ کرنا ضروری ہوگا اس کی وجہ شاید صرف انیسویں صدی کے تیز پیمائی کاروں کو معلوم ہوگا کہ
 \(l=0\) 
    کو
  \(s\)
      کہتے ہیں 
 \(l=1\)
       کو 
 \(p\)
        کہتے ہیں،
\( l=2\) 
        کو
   \(d\)
          کہتے ہیں اور
   \(l=3\)
            کو
\( f\)
  کہتے ہیں۔ میرے خیال سے اس کے بعد وہ سیدھی راس پر آ گئے اور انہوں نے عروف تہجی کے تحت
\( (g,h,i,,k,l)\)
   وغیرہ نام دینا شروع کیا۔ انہوں نے ہماری ناک میں دم کرنے کی خاطر 
  \(j\)
    کو نظر انداز کیا ۔ کسی ایک الیکٹران کے حال کو
    \( (n,l)\) 
    کی جوڑی ظاہر کرتی ہے جہاں عدد n حال کو اور حرف
    \(l\)
      مدارجی زاویائی معار حرکت کو ظاہر کرتا ہے ۔ کوانٹم عدد
   \(m\) 
      کا زکر نہیں کیا جاتا لیکن قوت نما میں حال کے مقین الیکٹرانوں کی تعداد لکھی جاتی ہے ۔ یوں درج ذیل تنظیم 
\[(1s)^{2}(2s)^{2}(2p)^{2} \]
کہتی ہے کہ مدارجہ
\((1,0,0)\)
 میں 2 الیکٹران، 
مدارجہ
\(( 2,0,0)\) 
میں 2 جبکہ مدارجے
\(( 2,1,1)\)
، 
\((2,1,0)\) 
اور
 \((2,1,-1)\) 
کے کسی ملاپ میں 2 الیکٹران پائے جاتے ہیں ۔ یہ درحقیقت کاربن کا زمینی حل ہے ۔ \\
اس مثال میں 2 الیکٹران ایسے پائے جاتے ہیں جن کے مدارجی زاویائی معارے ہرکت کوانٹم  عدد ایک ہے لہذا مدارچی زاوییائ معیار حرکت کوانٹم عدد ایک ہے لہذا کل مدارچی زاوییائ معیار حرکت کوانٹم نمبر 
\(l\)
کسی ایک ذرہ کی جبک
\(L\)
کل قیمت کو ظاہر کرتا ہے ۔ ایک، دو یا صفر ہو  سکتا ہے۔ جبکہ
 \((1s)\) 
کے دو الیکڑان ایک دوسرے کے ساتھ یکتا حال میں بندھے ہیں اور ان کا کل چکر صفر ہوگا۔ یہی کچھ
\(( 2S)\)
 کے دو الیکڑانوں کے لئے بھی ہوگا لیکن
 \((2p)\)
 کے دو الیکڑان یا تو یکتا نظام اور یا سہتا نظام میں ہوں گے۔ یوں کل چکر کوانٹم عدد S کل کو ظاہر کرنے کے لئے بڑا حرف استعمال ہوگا۔ جس کی قیمت ایک یا صفر ہو سکتی ہے۔ ظاہر ہے میزان کل مدارچی جمع چکر J کی قیمت تین، دو، ایک یا صفر ہو سکتی ہے۔ کسی ایک جوہر کے لئے ان کل قیمتوں کو ہن قواعد ( سوال 5.1 دیکھیں)  سے حاصل کیا جا سکتا ہے ۔نتیجہ کو درجہ ذیل روپ میں لکھا جا سکتا ہے. 
 \[^{2s+1}L_{J}\]
  جہاں J اور S اعداد جبکہ L ایک حرف ہو گا اور چونکہ ہم کل کی بات کر رہے ہیں لہذا یہ بڑا حرف ہو گا کاربن کا زمینی حالِ 3D ہے جس کا کل چکّر ایک ہے جس کی بنا 3 لکھا گیا ہے کل مدارچی زاویای معیار حرکت ایک ہے لہذا
  \(1p\) 
  لکھا گیا ہے اورمیزان  کل زاوییائ معیار حرکت صفر ہے لہذا صفر لکھا گیا ہے۔ جدول  5.1 میں دوری جدول کے ابتدائی چار صفحوں کے لئے انفرادی  تنظیم اور کل زاوییائ معیار حرکت مساوات 5.34 کی روپ میں پیش کئے گئے ہیں ۔ \\
  سوال 5.12 \\
  جز الف: دوری جدول کے ابتدائی  دو صفحوں  کے لئے نییوون تک مساوات 5.33 کی روپ میں تنظیم الیکڑان پیش کر کے ان کی تصدیق جدول 5.1 کے ساتھ کریں ۔ \\
  جز ب :ابتدائی  چار عناصر  کے لئے مساوات  5.34 کی روپ میں ان کا مطابقتی کل زاوییائ معیار حرکت تلاش کریں ۔بوران، کاربن اور نایڑوجن کے لئے تمام ممکنات پیش کریں۔\\
   سوال 5.13\\ 
   جز الف: ہن کا پہلا قاعدہ کہتا ہے کہ باقی چیزیں ایک جیسا ہونے کے لیے صورت میں وہ حال جس کا کل چکری زیادہ سے زیادہ ہوگی کم سے کم توانائی ہو گی۔ ہیلیم کے ھجان حالات کے لیے یہ کیا پیشنگوئی کرتا ہے۔\\
    جز ب: ہن کا دوسرا قاعدہ کہتا ہے کہ کسی ایک چکر کی صورت میں مجموعی طور پر خلاف تشاکلییت پر پورا اترتا ہو۔ وہ حال جس کی مدارچی زاوییائ معیار حرکتLl زیادہ سے زیادہ ہو گی توانائی کم سے کم ہو گی ۔ کاربن کے لئے 2=L کیوں نہیں ہوگا؟ اشارہ سیڑھی کا بالائی سر
    \((M_{L}=L)\)
      تشاکلی ہے۔\\
      جز ج: ہن کا تیسرا قاعدہ کہتا ہے کہ اگر ایک ذیلی خول 
   \((n,l)\)
       نصف سے زیادہ بھرا نا ہو تب کم سے کم توانائی کی سطح کے لئے 
    \(J=\abs{L-S}\)
        ہو گا۔ اگر یہ نصف سے زیادہ بھرا ہو تب
\( J=L+S\)
 کی توانائی کم سے کم ہوگی۔ اس حقیقت کو استعمال کرتے ہوئے سوال 5.12 ب میں بوران کے مسائلہ سے شک دور کرے۔\\ 
جز د: قواعد ہن کے ساتھ یہ حقیقت استعمال کرتے ہوئے کہ تشاکلی چکری حال کے ساتھ خلاف تشاکلی موزہ حال کے ساتھ خلاف تشاکل چکر حال استعمال ہوگا ۔سوال 5.12 ب میں کاربن اور نایڑوجن میں درپیش مشکلات سے چھٹکارا حاصل کریں ۔ اشارہ کسی بھی حال کی تشاکلی جاننے کی خاطر سیڑھی کے بالائی سر سے آغاز کریں ۔\\
 سوال 5.14\\
  دوری جدول کے چھٹے صف میں عنصر چار ساٹھ ڈسپروسییم کا زمینی حال
  \(^{5}I_{8}\) 
  ہے۔ اس کے کل چکر کل مدارچے اور میزان کل زاوییائ معیار حرکت کوانٹم کل حالات کیا ہوں گے۔ ڈسپروسییم کے الیکڑان  کی تنظیم کا خاکہ کیا ہو سکتا ہے۔\\
حصہ
 5.3\\
 
ٹھوس حال میں ہر جوہر کے بیرونی ڈیلے مقید گرفتی الیکٹرانوں میں سے چند ایک علیحدہ ہو کر کسی مخصوص موروثی مرکزا کے کولوم میدان سے آزاد، تمام قلمی جال کے مخفیا کے زیرِ اثر حرکت کرنا شروع کرتے ہے اس حصہ میں ہم تو بہت سادے نمونوں پے غور کرے گے۔ پہلا نمونہ الیکٹرون گیس نظریہ ہے جو سمرفيل نے پیش کیا اس نمونے میں سرحد کے اثرات کے علاوہ باقی تمام قوتوں کو نظرانداز کیا جاتا ہے اور الیکٹرانوں کو لامتناہی چاکور کنواں کے تین آبادی مماثل کی طرح ڈبے میں آزاد ذرات تصویر کیا جاتا ہے۔ دوسرا نمونہ بلخ نظریہ کہلایا جاتا ہے الیکٹرون کی بہمی دفاع کو نظر انداز کرتے ہوئے باقاعدگی سے ایک جیتنے فاصلے پر مثبت بار کے مرکزہ کو دوری مخفیہ سے ظاہر کرتا ہے، یہ نمونے ٹھوس اجسام کی کوانٹم نظریے کی طرف پہلے لڑکھڑاتے قدم ہیں۔ اس کے باوجود یہ پولی حصولمنات کا جموت میں گہرا کردار اور موصال، غیر موصل اور نیم موصل کی حیرت کن برقی خواص پر روشنی ڈالنے میں مدد دیتی ہے۔\\
جز حصہ
 5.3.1\\
 
آزاد الیکٹرون گیس\\
، فرض کرے ایک ٹھوس جسم مستطیل چکل کا ہے جس کے اصلا
 \(l_{x}\)
 ،
 \(l_{y}\)
  اور
 \(l_{z}\)
    ہے  اور فرض کرے کے اِس کے اندر الیکٹرون پر کوئی قوت اثر انداز نہیں ہوسکی ما سوائے ناقابلِ گزر دیواروں کے۔
\begin{equation}
V(x,y,z)=
\begin{cases}
0 & 0<x<l_{x}, \quad 0<y<l_{y}, \quad 0<z<l_{z}\\
\infty & otherwise\\
\end{cases}
\end{equation}
شرودنگر مساوات
\[\frac{-\hbar^{2}}{2m}\nabla^{2}\psi=E\psi\]
\[\psi(x,y,z)=X(x)Y(y)Z(z)\]
\[\frac{-\hbar^{2}}{2m}\frac{\dif^{2}X}{\dif{x}^{2}}=E_{x}X ; \frac{-\hbar^{2}}{2m}\frac{\dif^{2}Y}{\dif{y}^{2}}=E_{y}Y ; \frac{-\hbar^{2}}{2m}\frac{\dif^{2}Z}{\dif{z}^{2}}=E_{z}Z\]
اور
\[E=E_{x}+E_{y}+E_{z}\]
درج ذیل لیتے ہوئے،
\[k_{x}\equiv \frac{\sqrt{2mE_{x}}}{\hbar}, k_{y}\equiv\frac{\sqrt{2mE_{y}}}{\hbar}, k_{z}\equiv \frac{\sqrt{2mE_{z}}}{\hbar}\]
 ہم عمومی حل حاصل کرتے ہے۔
\begin{equation}
X(x)=A_{x}\sin{(K_{x}x)}+B_{x}\cos{(K_{x}x)} \quad Y(y)=A_{y}\sin{(K_{y}y)}+B_{y}\cos{(K_{y}y)}\\
Z(z)=A_{z}\sin{(K_{z}z)}+B_{z}\cos{(K_{z}z)}\\
\end{equation}
سرحدی شرائط کے تحد 
\[X(0)=Y(0)=Z(0), B_{x}=B_{y}=B_{z}=0, X(l_{x})=Y(l_{y})=Z(l_{z})=0\]
ہوگا۔ لہٰذا درج ذیل ہوگا۔
\[k_{x}l_{x}=n_{x}\pi, k_{y}l_{y}=n_{y}\pi, k_{z}l_{z}=n_{z}\pi\]
جہاں ہر n ایک مثبت عدد صحیح ہے۔
\[n_{x}=1,2,3,\dotsc \quad n_{y}=1,2,3,\dotsc \quad n_{z}=1,2,3,\dotsc\]
معمول شدہ تفلات مٌوج درج ذیل ہونگے۔
\[\psi_{n_{x}n_{y}n_{z}}=\sqrt{\frac{8}{l_{x}l_{y}l_{z}}}\sin{\big(\frac{n_{x}\pi}{l_{x}} x\big)}\sin{\big(\frac{n_{y}\pi}{l_{y}} y\big)}\sin{\big(\frac{n_{z}\pi}{l_{z}} z\big)}\]
اور اجازاتی توانائياں درج ذیل ہونگی۔
\[E_{n_{x}n_{y}n_{z}}=\frac{\hbar^{2}\pi^{2}}{2m}\big(\frac{n_{x}^{2}}{l_{x}^{2}}+\frac{n_{y}^{2}}{l_{y}^{2}}+\frac{n_{z}^{2}}{l_{z}^{2}}\big )=\frac{\hbar^{2}k^{2}}{2m}\]
جہاں سمتیاں موج، 
\(k\equiv (k_{x},k_{y},k_{z})\)
 کی مطلق قیمت K ہوگی ۔
اگر آپ ایک تین آبادی فضا کا تصویر کرے جس کے محور
\(k_{x}، k_{y} ،k_{z}\)
 ہو اور جس پر
\(k_{x}=(\pi/l_{x})(2\pi/l_{x})(3\pi/l_{x})\dotsc\)
اور
\(k_{y}=(\pi/l_{y})(2\pi/l_{y})(3\pi/l_{y})\dotsc\)
اور
\(k_{z}=(\pi/l_{z})(2\pi/l_{z})(3\pi/l_{z})\dotsc\)
پر سیدھے سطوت پاے جاتے ہو تب ہر انفرادی نقطہ تکاتے ایک منفرد یک ذرا ساکن حال دیگا۔\\
اس جال میں ہر ایک خانہ لہٰذا ہر ایک حال کی فضا میں درج ذیل حجم گہیرے گا، جہاں  پورے جسم کا حجم ہے۔
\[\frac{\pi^{3}}{l_{x}l_{y}l_{z}}=\frac{\pi^{3}}{V}\]


 %==============
 
	فرض کریں مادہ کے ایک ٹکڑا میں $N$ جوہر پائے جاتے ہوں اور ہر جوہر اپنے حصہ کے $q$ آزاد الیکٹرون دیتا ہو۔ عملاً کسی بھی کلاں بینی جسامت کے چیز کے لیئے $N$ کی قیمت بہت بڑی ہوگی جو ایوگادرو عدد میں گنی جائے گی جبکہ $q$ ایک چھوٹا عدد مثلاً 1 یا 2 ہوگا۔ اگر ایلکٹرون بوزان یا قابلِ ممیز ذرات ہوتے تب وہ زمینی حال $\psi_{111}$ میں سکونیت اختیار کرتے حقیقتاً الیکٹروں یکساں فرمیونز ہیں جن پر پالی اصول منات کا اطلاق ہوتا ہے لحاظہ کسی بھی حل کی مکین صرف دو الیکٹرون ہو سکتے ہیں۔ یہ $k$ فضا میں ایک کرہ کا ایک ثمن رداس $k_F$ تک بھرے گی جس کو اس حقیقت سے تعین کیا جا سکتا ہے کہ الیکٹران کی ہر ایک جوڑی کو \(\frac{\pi^{3}}{V}\) حجم درکار ہوگا مساوات  \num{5.40}: 
	\begin{align*}
		\frac{1}{8}(\frac{4}{3} \pi k^{3}_F) =  \frac{Nq}{2}(\frac{\pi^3}{V})
	\end{align*}
یوں
\begin{align}
	k_F =(3\rho\pi^{2})^{\frac{1}{3}}
\end{align}
جہاں
\begin{align}
	\rho \equiv \frac{Nq}{V}
\end{align}
آزاد الیکٹران کثافت ہے(آزاد حجم میں الیکٹرانوں کی تعداد)۔

$k$ فضا میں مکین اور غیر مکین حالات کی سرحد کو \موٹا{فرمی سطح} کہتے ہیں (اسی کی بنا زیرنوشت میں $F$ لکھا گیا)۔
اس سطح پر طاقتی توانائی کو \موٹا{فرمی توانائی} $E_F$ کہتے ہیں۔آزاد الیکٹران گیس کے لیئے درج ذیل ہو گا۔
\begin{align}
	E_F = \frac{h^{2}}{2m}(3\rho\pi^{2})^{\frac{2}{3}}
\end{align}
الیکٹران گیس کی کل توانائی کو درج ذیل طریقہ سے حل کیا جا سکتا ہے. ایک خول جس کی موٹائی $dk$ شکل \num{5.4} ہو کا حجم
\begin{align*}
	\frac{1}{8}(4\pi k^{2})dk
\end{align*}
لحاظہ اس خول میں الیکٹرون حالات کی تعداد درج ذیل ہوگی
\begin{align*}
	\frac{2[(\frac{1}{2})\pi k^{2}dk]}{\frac{\pi^{3}}{V}} = \frac{V}{\pi^{2}}k^{2}dk
\end{align*}
ان میں سی ہر ایک حال کی توانائی \(\frac{\hbar^{2}k^{2}}{2m}\) مساوات \num{5.39} لحاظہ خول کی توانائی
\begin{align}
	dE = \frac{\hbar^{2}k^2}{2m} \frac{V}{\pi^{2}}k^{2}dk
\end{align}
اور کل توانائی درج ذیل ہوگی
\begin{align}
	E_{tot}=\frac{\hbar^{2}V}{2\pi^{2}m}\int_{0}^{k_F}k^{4}dk = \frac{\hbar^{2}k^{5}_F V}{10\pi^{2}m} = \frac{\hbar^{2}(3\pi^{2}Nq)^{\frac{5}{3}}}{10\pi^{2}m}V^{\frac{-2}{3}}
\end{align}
کوانٹم میکانی توانائی کا کردار کچھ ایسا ہی ہے جیسا سادہ گیس میں اندرونی حراری توانائی $U$ کا ہوتا ہے۔ بل خصوص یہ دیواروں پر ایک دباؤ پیدا کرتا ہے اور اگر ڈبے کے حجم میں $dV$ کا اضافہ ہو تب کل توانائی میں درج ذیل کمی رونما ہوگی
\begin{align*}
	dE_{tot} = -\frac{2}{3}\frac{\hbar^2(3\pi^{2}Nq)^{\frac{5}{3}}}{10\pi^{2}m}V^{\frac{5}{3}}dV = -\frac{2}{3}E_{tot}\frac{dV}{V}
\end{align*}
جو بیرون پر کوانٹم دباؤ $P$ کا کیا ہوا کام \(dW = PdV\) نظر آتا ہے
\begin{align}
	P = \frac{2}{3}\frac{E_{tot}}{V} = \frac{2}{3}\frac{\hbar^{2}k^{5}_F}{10\pi^{2}m} = \frac{(3\pi^{2})^{\frac{2}{3}}\hbar^{2}}{5m}\rho^{\frac{5}{3}}
\end{align}
یہ اس سوال کا جزوی جواب ہے کہ ایک ٹھنڈا ٹھوس شہ اندر کی طرف منہدن کیوں نہیں ہو جاتا۔ ایک اندرونی کوانٹم میکانی دباؤ توازن برقرار رکھتی ہے جس کا الیکٹرون کے  باہمی دفع جنہیں ہم نظر انداز کر چکے ہیں یا حراری حرکت جس کو ہم خارج کر چکے ہیں کے ساتھ کوئی تعلق نہیں ہے۔ بلکہ جو یکساں فرمیان کی ضرورت خلاف تشاکلیت سے پیدا ہوتا ہے۔ اس کو بعض اوقات انحطاطی دباؤ کہتے ہیں اگر چہ مناتی دباؤ بہتر اصطلاح ہو گی۔

\ابتدا{سوال}
ایک آزاد الیکٹرون کی اوسط توانائی \(\frac{E_{tot}}{Nq}\) کو فرمی توانائی کے قصر کی صورت میں لکھیں۔

جواب: \(\frac{3}{5}E_F\)
\انتہا{سوال}
\ابتدا{سوال}
تانبا کی کثافت $\SI{8.96}{\gram \per \centi\meter\cubed}$ ہے جبکہ اس کا جوہری وزن $\SI{63.5}{\gram \per \mole}$ ہے۔

(الف) مساوات $\num{5.43}$استعمال کرتے ہوئے $q = 1$ لیتے ہوئے تانبے کی فرمی توانائی کا حساب لگا کر نتیجہ کو الیکٹرون ولٹ کی صورت میں لکھیں۔

(ب) الیکٹران کی مطابقتی سمتی رفتار کیا ہوگی؟اشارہ: \(E_F = (\frac{1}{2})mv^{2}\) لیں۔ کیا تانباے میں الیکٹرون کو غیر اضافی تصور کرنا خطرے سے باہر ہو گا؟

(ج) تانبہ کے لیئے کس درجہ حرارت پر امتیازی حراری توانائی $k_{B}T$ جہاں $k_B$ بولٹزمن مستقل اور $T$ کیلون حرارت ہے فرمی توانائی کے برابر ہوگا؟ تبصرہ: اس کو فرمی حرارت کہتے ہیں۔ جب تک حقیقی حرارت فرمی حرارت سے کفی کم ہو مادہ کو ٹھنڈہ  تصور کیا جا سکتا ہے اور اس میں الیکٹرون نچلے ترین قابلِ پہنچ حال میں ہوں گے۔ چونکہ تانبے $\SI{1356}{\kelvin}$ پر گلتا ہے لحاظہ ٹھوس تانبہ ہر صورت ٹھنڈہ ہوگا۔
 
(د) الیکٹران گیس نمونہ میں تانبہ کے لیئے انحطاطی دباؤ مساوات $\num{5.46}$ کا حساب لگائیں۔
\انتہا{سوال}
\ابتدا{سوال}
کسی جسم پر دباؤ میں معمولی کمی اور نتیجتاً حجم میں نصبتی اظافہ کے تناسب کو جسم مقیاس کہتے ہیں۔
\begin{align*}
	B = -V\frac{dP}{dV}
\end{align*}
دیکھائیں کہ آزاد الیکٹران نمونہ میں\(B = \frac{5}{3}P\) ہوگا اور سوال$\num{5.16}(\text{\RL{د}})$ کا نتیجہ استعمال کرتے ہوئے تانباہ کے لیئے جسیم مقیاس کی اندازاً قیمت تلاش کریں۔ تبصرہ: تجربہ سے حاصل قیمت $\SI{13.4e10}{\newton \per \meter \squared}$ ہے مکمل درست جواب کی توقع نہ کریں چونکہ ہم نے الیکٹران مرکزہ اور الیکٹران الیکٹران قوتوں کو نظرانداز کیا ہے! حقیقت میں یہ ایک حرین کن نتیجہ ہے کہ حساب سے حاصل نتیجہ حقیقت کے اتنا قریب ہے۔ 
\انتہا{سوال}
\جزوحصہ{سخت پٹی}
ہم آزاد الیکٹران نمونہ میں منظم فاصلوں پر ساکن مثبت بار کے مرکزہ کی الیکٹرانوں پر قوت کو شامل کر کے بہتر نمونہ حاصل کرتے ہیں۔ ٹھوس اجسام کا رویہ نمایاں حد تک  اس حقیقت پر مبنی ہے کہ اس کا مخفیہ دوری ہوتا ہے۔ مخفیہ کی حقیقی شکل و صورت مادہ کی تفصیلی رویہ میں کردار ادا کرتی ہے۔ یہ عمل دیکھنے کی خاطر میں سادہ ترین نمونہ تیار کرتا ہوں جس سے یک بُعدی ڈیراک کنگھی کہتے ہیں اور جو ایک جتنے برابر فاصلوں پر نوکیلی ڈیلٹا تفاعلوں پر مشتمل ہوتا ہے شکل \num{5.5} ۔ لیکن اس سے پہلے میں ایک طاقتور مسئلہ پیش کرتا ہوں جو دوری مخفیہ کے مسائل کا حل نہایت سادہ بناتا ہے۔

دوری مخفیہ سے مراد ایسا مخفیہ ہے جو کسی مستقل فاصلہ $a$ کے بعد اپنے آپ کو دہراتا ہے۔ 
\begin{align}
	V(x+a) = V(x)
\end{align}
مسئلہ بلوخ کہتا ہے کہ دوری مخفیہ کے لیئے مساوات شروڈنگر،
\begin{align}
	-\frac{\hbar^{2}}{2m}\frac{d^{2}\psi}{dx^{2}} +V(x)\psi = E\psi
\end{align}
کے حل سے مراد وہ تفاعل لیا جا سکتا ہے جو درج ذیل شرط کو مطمئن کرتا ہو
\begin{align}
	\psi(x+a) = e^{iKa}\psi(x)
\end{align}
جہاں $K$ ایک مستقل ہے۔ یہاں مستقل سے مراد ایسا تفاعل ہے جو $x$ کا تابع نہیں ہے اگرچہ یہ $E$ کا تابع ہو سکتا ہے۔

\موٹا{ثبوت}: مان لیں کے $D$ ایک ہٹاؤ عامل ہے:
\begin{align}
	Df(x) = f(x+a)
\end{align}
دوری مخفیہ مساوات \num{5.47} کی صورت میں $D$ ہیملٹنی کا قابلِ تبادل ہو گا:
\begin{align}
	[D, H] = 0
\end{align}
لحاظہ ہم $H$ کے ایسے امتیازی تفاعلات چھنڈ سکتے ہیں جو بیک وقت $D$ کے امتیازی تفاعلات بھی ہوں:\(D\psi = \lambda\psi\) یا
\begin{align}
	\psi(x+a) = \lambda\psi(x)
\end{align}
یہاں $\lambda$ کسی صورت صفر نہیں ہو سکتا اگر یہ صفر ہو تب چونکہ مساوات \num{5.52} تمام $x$ کے لیئے مطمئن ہوگا لحاظہ ہمیں \(\psi(x) = 0\) ملے گا جو قابلِ قبول امتیازی تفاعل نہیں ہے۔ کسی بھی غیر مخلوط عدد کی طرح اس کو قوتِ نمائی روپ میں لکھا جا سکتا ہے: 
\begin{align}
	\lambda = e^{iKa}
\end{align}
جہاں $K$ ایک مستقل ہوگا۔

اس مقام پر مساوات \num{5.53} امتیازی قدر $\lambda$ لکھنے کا ایک انوکھا طریقہ ہے لیکن ہم جلد دیکھیں گے کہ $K$ حقیقی ہے اور یوں اگرچہ $\psi(x)$ ازخود غیر دوری ہے\(\abs{\psi(x)}^{2}\) جو درج ذیل ہے۔
\begin{align}
	\abs{\psi(x+a)}^{2} = \abs{\psi(x)}^{2}
\end{align}
دوری ہوگا جیسا کہ ہم توقع کرتے ہیں۔

اب ظاہر ہے کہ کوئی بھی حقیقی ٹھوس جسم ہمیشہ کے لیئے چلتا نہیں جائے گا بلکہ کہیں نہ کہیں اس کی سرحد پائی جائے گی جو $V(x)$ کی دوریت کو ختم کرتے ہوئے مسئلہ بلوخ کو ناکارہ بنا دے گی۔ تاہم کسی بھی کلابین سطح کے  قلم میں کئی ایوگادرو عدد کے برابر جوہر پائے جائیں گے اور ہم فرض کر سکتے ہیں کہ تھوس جسم کی سطح سے بہت دور الیکٹران پر سطحی اثر قابلِ نظر انداز ہوگا۔ ہم مسئلہ بلوخ پر پورا اترنے کی خاطر $x$ کو ایک دائرے پر رکھتے ہیں تاکہ اس کی دم بہت بڑی تعداد \(N\approx10^{23}\) دوری فاصلوں کے بعد اس کے سر پر پایا جاتا ہو باضابطہ طور پر ہم درج ذیل سرحدی شرط مسلط کرتے ہیں   
\begin{align}
	\psi(x+Na) = \psi(x)
\end{align}
یوں مساوات \num{5.49} کے تحت درج ذیل ہوگا
\begin{align*}
	e^{iNKa}\psi(x) = \psi(x)
\end{align*}
لحاظہ \(e^{iNKa} = 1\) یا \(NKa = 2\pi n\) ہوگا جس کے تحت درج ذیل ہوگا 
\begin{align}
	K = \frac{2\pi n}{Na}, (n = 0, \pm1, \pm2, \dots)
\end{align}
یہاں $K$ لازماً حقیقی ہوگا مسئلہ بلوخ کی عفادیت یہ ہے کہ ہمیں صرف ایک خانہ مثلاً \((0\leq x<a)\) کے وقفہ پر مسئلہ شروڈنگر حل کرنا ہوگا مساوات \num{5.49} کی بار بار اطلاق سے ہر جگہ کے حالات  حاصل ہوںگے۔

اب فرض کریں کے مخفیہ در حقیقت نوکیلی ڈیلٹا تفاعلات ڈیراک کنگھی پر مشتمل ہو:
\begin{align}
	V(x) = \alpha\sum_{j=0}^{N-1}\delta(x-ja)
\end{align}
شکل\num{5.5} میں آپ تصور کریں گے  کہ محور $x$ کو یوں دائروی شکل مین گھومایا گیا ہے کہ $N$ویں نوکیلی تفاعل درحقیقت نقطہ \(x= -a\)  پر پایا جاتا ہے۔ اگر چہ یہ حقیقت پسند نمونہ نہیں ہے لیکن یاد رہے ہمیں دوریت سے دلچسپی ہے۔ کلاسیکی طور پر دہراتا ہوا مستطیلی مخفیہ استعمال کیا گیا جو اب بھی بہت سے مسنیفین کا پسندیدہ مخفیہ ہے خطہ \((0<x<a)\) میں مخفیہ صفر ہوگا لحاظہ 
\begin{align*}
	-\frac{\hbar^{2}}{2m}\frac{d^{2}\psi}{dx^{2}} = E\psi,
\end{align*}
یا
\begin{align*}
	\frac{d^{2}\psi}{dx^{2}} = -k^{2}\psi,
\end{align*}
ہوگا۔

جہاں ہمیشہ کہ طرح درج ذیل ہوگا 
\begin{align}
	k = \frac{\sqrt{2mE}}{\hbar},
\end{align}
اس کا عمومی حل درج ذیل ہے 
\begin{align}
	\psi(x) = A\sin(kx) + B\cos(kx), (0<x<a).
\end{align}
مسئلہ بلوخ کے تحت مبدا کے بلکل بائیں ہاتھ پہلے خانہ میں تفاعل موج درج ذیل ہوگا 
\begin{align}
	\psi(x) = e^{-iKa}[A\sin k(x+a) + B\cos k(x+a)], (-a<x<0). 
\end{align}
نقطہ\(x=0\) پر $\psi$ لازماً استماری ہوگا لحاظہ 
\begin{align}
	B = e^{-iKa}[A\sin(ka) + B\cos(ka)];
\end{align}
اس کے تفرق میں ڈیلٹا تفاعل کی زور کے براہراست متناسب عدم استمرار پائے جائے گی مساوات\num{2.125} جس میں $\alpha$ کی علامت اُلٹ ہوگی چونکہ یہاں کنواں کی بجائے نوکیلی تفاعل پایا جاتا ہے
\begin{align}
	kA - e^{-iKa}k[A\cos(ka) - B\sin(ka)] = \frac{2m\alpha}{\hbar^{2}}B
\end{align}
مساوات \num{5.61} کو \(A\sin(ka)\) کے لیئے حل کرتے ہوئے درج ذیل حاصل ہوگا 
\begin{align}
	A\sin(ka) = [e^{iKa}-\cos(ka)]B
\end{align}
اس کو مساوات \num{5.62} میں پُر کرتے ہوئے اور \عددی{k_B} کو منسوخ کرتے ہوئے 
\begin{align*}
	[e^{iKa}-\cos(ka)][1-e^{-iKa}\cos(ka)] + e^{-iKa}\sin^{2}(ka) = \frac{2m\alpha}{\hbar^{2}k}\sin(ka)
\end{align*}
حاصل ہوگا۔

جس سے درج ذیل سادہ روپ حاصل ہوتا ہے
\begin{align}
	\cos(ka) = \cos(ka) + \frac{m\alpha}{\hbar^{2}k}\sin(ka)
\end{align}
یہ ایک بنیادی نتیجہ ہے جس سے باقی سب کچھ اخز ہوتا ہے۔ کرونیگ پینی مخفیہ ہاشیہ \num{18} دیکھیں کے لیئے کلیہ زیادہ پیچیدہ ہوگا لیکن جو خدوخال ہم دیکھنے جا رہے ہیں وہی اس میں بھی پائے جاتے ہیں۔

 مساوات \num{5.64} $k$ کی ممکنات قیمتیں لحاظہ اجازتی توانائیاں تعین کرتی ہیں۔ علامتیت کو سادہ بنانے کی نقطہ نظر سے ہم درج ذیل لکھتے ہیں 
\begin{align}
	z \equiv ka, \text{and} \beta \equiv \frac{m\alpha a}{\hbar^{2}}
\end{align}
جس سے مساوات \num{5.64} کا دائیاں ہاتھ درج ذیل روپ اختیار کرتا ہے
\begin{align}
	f(z) \equiv \cos(z) + \beta\frac{\sin(z)}{z}
\end{align}
مستقل $\beta$ بُعدی ہے جو ڈیلٹا تفاعل کی زور کی ناپ ہے شکل \num{5.6} میں میں نے \(\beta = 10\) کے لیئے \عددی{f(z)} کو ترسیم کیا ہے۔ یہاں دیکھنے کی اہم بات یہ ہے کے\عددی{f(z)} ساتھ\((-1, +1)\) سے باہر بھٹکتا ہے اور چونکہ \(\abs{\cos(Ka)}\) کی قیمت کسی صورت ایک سے تجاز نہیں کر سکتی ہے لحاظہ ایسی خطوں میں مساوات \num{5.64} کا حل نہیں پایا جائے گا۔ یہ درز ممنوع توانائیوں کو ظاہر کرتی ہے انکے بیچ اجازتی توانائیوں کی پٹیاں پائی جاتی ہیں مساوات \num{5.56} کے تحت \(Ka = \frac{2\pi n}{N}\) ہے جہاں \عددی{N} ایک بہت بڑا عدد ہے لحاظہ \عددی{n} کوئی بھی عدد صحیح ہو سکتا ہے۔ یوں کسی ایک پٹی میں تقریباً ہر توانائی اجازتی ہوگی۔ آپ تصور میں شکل \num{5.6} پر \(\cos(\frac{2\pi n}{N})\) قیمت کے فاصلون پر \(+1(n = 0)\) سے لے کر نیچے \(-1(n = \frac{N}{2})\) تک اور واپس تقریباً \(+1(n = N-1)\) تک جہاں بلوخ جزو ضربی \(e^{iKa}\) دوبارہ چکر شروع کرتا پے لحاظہ \عددی{n} کو مزید بڑھانے سے کوئی نیا حل حاصل نہیں ہو گا لکیریں کھینچ کر دیکھ سکتے ہیں۔ ان لیکیروں میں ہر ایک کا \عددی{f(z)} کے ساتھ تقاطع ایک اجازتی توانائی دیگا۔ ظاہر ہے کہ ہر پٹی میں \عددی{N} حالات پائے جاتے ہیں جو ایک دوسرے کے اتنے قریب ہیں  کہ کسی بھی نقطہ نظر سے انہیں ایک مسلسل خطہ تصور کیا جا سکتا ہے شکل \num{5.7}۔

 ہم نے ابھی تک اپنے مخ فیہ میں ایک الیکٹران رکھا ہے۔ حقیقت میں \عددی{N_q} الیکٹران ہوںگے جہاں ہر ایک جوہر \عددی{q} تعداد کے آزاد الیکٹران مہیہ کرے گا۔ پالی اصولِ منات کے بنا صرف دو الیکٹران کسی ایک فضائی حال کے مکین ہو سکتے ہیں۔ یوں  \(q = 1\) کی صورت میں یہ زمینی حال میں پہلی پٹی کو آدھا  بھریں گے اگر  \(q = 2\) ہو تب یہ پہلی پٹی کو مکمل کریں گے اگر \(q = 3\) ہو یہ دوسری پٹی کو آدھا بھریں گے وغیرہ وغیرہ. تین ابعاد میں اور زیادہ حقیقی مخفیہ کی صورت میں پٹیوں کی ساخت زیادہ پیچیدہ ہوسکتی ہے لیکن اجازتی پٹیاں جنکے بیچ ممنوع درز پائے جاتے ہوں تب بھی ہوگا۔ دوری مخفیہ کی نشانی بھی پٹی ہے۔
 
 اب اگر ایک پٹی مکمل طور پر بھری ہوئی ہو ممنوع خطہ سے گزرتے ہوئے اگلی پٹی تک چھلانگ کے لیئے ایک الیکٹران کو نصبتاً زیادہ توانائی درکار ہوگی ایسا مادہ برقی طور پر غیر موئثل ہوگا۔ اس کے برعکس اگر ایک پٹی پوری طرح بھری ہوئی نہیں ہے تب ایک الیکتران کو بہت معمولی توانائی درکار ہوگی کہ وہ ہیجان ہوسکے  اس طرح کا مادہ عموماً موئثل ہوگا۔ ایک غیر موئثل میں بڑے یا کم \عددی{q} کے چند جوہر کی ملاوٹ سے اگلی بلند پٹی میں چند اظافی الیکٹران رکھ دیئے جاتے ہیں پہلے سے مکمل پُر پٹی میں خول پیدا کیئے جاتے ہیں۔ ان دونوں صورتوں میں ایک کمزور برقی رو گزر سکتا ہے اور ایسے اشیاء نیم موئثل کہلاتے ہیں۔ آزاد الیکٹران نمونہ میں تمام ٹھوس اجسام کو لازماً بہت اچھا موئثل ہونا چاہیئے تھا چونکہ انکے اجازتی توانائیوں کے طیف میں کوئی بڑا وقفہ نہیں پایا جاتا ہے۔ قدرت میں پائے جانے والے ٹھوس اجسام کی برقی موصلیت میں اتنا زیادہ فرق صرف نظریہ پٹی کی مدد سے سمجھا ھا سکتا ہے۔  
%==========================

\ابتدا{سوال}

(الف) مساوات \num{5.59} اور مساوات \num{5.63} استعمال کرتے ہوئے دیکھائیں کہ دوری ڈیلٹا تفاعل مخفیہ میں ایک ذرے کی تفاعل موج درج ذیل روپ میں لکھی جا سکتی ہے 
\begin{align*}
	\psi(x) = C[\sin(kx)+e^{-iKa}\sin k(a-x)], (0\leq x\leq a).
\end{align*} 
معمولزنی مستقل \عددی{C} تعین کرنے کی ضرورت نہیں ہے۔

(ب)  البتہ پٹی کے بالائی سر پر جہاں \عددی{z} $\pi$ کا عدد صحیح مضرب ہوگا شکل \num{5.6} (الف) سے \(\psi(x) = 0\) حاصل ہوگا ایسی صورت میں درست تفاعل موج تلاش کریں دیکھیئے گا کہ ہر ایک ڈیلٹا تفاعل پر \(\psi\) کو کیا ہوتا ہے؟
\انتہا{سوال}
\ابتدا{سوال}
پہلی اجازتی پٹی کے نچھلے نقطہ پر \(\beta = 10\) کی صورت میں توانائی کی قیمت تین با معنی ہندسوں تک تلاش کریں۔ دلائل پیش کرتے ہوئے آپ فرض کر سکتے ہیں کہ\(\frac{\alpha}{a} = \SI{1}{\electronvolt}\) ہو گا۔
\انتہا{سوال}
\ابتدا{سوال}
فرض کریں ہم ڈیلٹا تفاعل سولن کے بجائے ڈیلٹا تفاعل کنواں پر غور کر رہے ہیں یعنی مساوات \num{5.57} میں \(\alpha\) کی علامت تبدیل کریں۔ ایسی صورت میں شکل \num{5.6} اور \num{5.7} کی طرح کے شکال بنائیں۔ مثبت توانائی حلوں کے لیئے آپ کو کوئی نیا حساب کرنے کی ضرورت نہیں ہے بس مساوات \num{5.66} میں موضوع تبدیلیاں لائیں لیکن منفی توانائی حلوں کے لیئے آپ کو کام کرنا ہوگا اور انہیں ترسیم پر شامل کرنا مت بھولیئے گا جو اب \(-z\) تک وسیع ہوگا۔ پہلی اجازتی پٹی میں اب کتنے حالات ہونگے؟
\انتہا{سوال}
\ابتدا{سوال}
دیکھائیں کہ مساوات \num{5.64} میں حاصل زیادہ تر توانائیاں دوہری انحطاطی ہے۔ کن صورتوں میں ایسا نہیں ہے؟ اشارہ: \((N=1, 2, 3, 4, \dots)\) لیتے ہوئے دیکھیئے گا کیا ہوتا ہے۔ ایسی ہر صورت میں \(\cos(Ka)\) کی کیا ممکنا قیمتیں ہوں گی؟
\انتہا{سوال}
\حصہ{کوانٹم شماریاتی میکانیات}
مطلق صفر حرارت پر ایک طبی نظام اپنے کم سے کم اجازتی توانائی تنظیم کا مکین ہوگا۔ درجہ حرارت بڑھاتے ہوئے بلا منصوبہ حراری سرگرمیوں کے بنا ہیجانی حالات ابھرنے شروع ہونگے جس سے درج ذیل سوال پیدا ہوتا ہے: اگر \عددی{T} درجہ حرارت پر حراری توازن میں ایک بڑی تعداد \عددی{N} کے ذرات پائے جاتے ہوں تب اسکا کیا احتمال ہے کہ ایک ذرہ جس کو بلا منصوبہ منتخب کیا گیا ہو کی مخصوص توانائی \عددی{E_j}؟ ہوگی دیہان رہے کہ اس احتمال کا کوانٹم عدم تعین کے ساتھ کوئی تعلق نہیں ہے بلکل یہی سوال کلاسیکی شماریاتی میکانیات میں بھی کھڑا ہوتا ہے۔ ہمیں احتمالی جواب اس لیئے منظور ہوگا کہ جن ذرات کی ہم بات کر رہے ہیں انکی تعداد اتنی بڑی ہوگی کہ یہ کسی صورت ممکن نہیں ہوگا کہ ہم ہر ایک پر علیحدہ علیحدہ نظر رکھ سکیں چاہے یہ قابلِ تعین ہو یا نہ ہوں۔

شماریاتی میکانیات کا بنیادی مفروضہ یہ ہے کہ حراری توازن میں ہر وہ منفرد حال جس کی ایک جیسی کل توانائی \عددی{E} ہو ایک جتنا معتمل ہوگا۔ بلا واسطہ حراری حرکتوں کی بنا مستقل طور پر توانائی ایک ذرہ سے دوسرا ذرہ ایک روپ حرکی، گردشی، گھومتی وغیرہ سے دوسری روپ میں منتقل ہوگی لیکن بیرونی مداخلت کی عدم موجودگی میں بقاء توانائی کی بنا کل مقررہ ہوگا۔ یہاں مفروضہ یہ ہے  کہ توانائی کی لگاتار نئی تقسیم کسی مخصوص حال کو ترجیح نہیں دیتا ہے۔ یہ ایک گہرا مفروضہ ہے جو سوچنے کے قابل ہے درجہ حرارت \عددی{T} حراری توازن میں ایک نظام کی کل توانائی کی بس پیمائش ہے۔ ان منفرد حالات کی گنتی میں کوانٹم میکانیات ایک نئی پیچیدگی پیدا کرتی ہے لیکن چونکہ حالات غیر مسلسل ہیں لحاظہ یہ کلاسیکی نظریہ سے زیادہ آسان ہے اور اسکا فیصلہ کن انحصار اس بات پر ہوگا کہ یہ ذرات قابلِ ممیز، یکساں بوزان یا یکساں فرمیون ہیں۔ ان کے دلائل نسبتاً سیدھے لیکن ریاضی کافی گہری ہے لحاظہ میں ایک انتہائی سادھا مثال سے شروع کروں گا تاکہ آپ بنیادی حقائق سمجھ سکیں۔
\جزوحصہ{ایک مثال} 
فرض کریں ہمارے پاس یک بعدی لامتناہی چکور کنواں حصہ\num{2.2} میں کمیت \عددی{m} کے صرف تین باہم غیر متعمل ذرات پائے جاتے ہیں۔ ان کی کل توانائی درج ذیل ہوگی ماساوات \num{2.27} دیکھیں
\begin{align}
	E = E_A + E_B + E_C = \frac{\pi^2 \hbar ^2}{2ma^2}(n^2_A + n^2_B + n^2_C)
\end{align}
جہاں \(n_A\)، \(n_B\) اور \(n_C\) مثبت عدد صحیح ہوں گے۔ اب تبصرہ جاری رکھنے کی خاطر فرض کریں کہ\(E=363(\frac{\pi ^2 \hbar ^2}{2ma^2})\) یعنی درج ذیل
\begin{align}
	n^2_A + n^2_B + n^2_C = 363.
\end{align}  
جیسے آپ تصدیق کرسکتے ہیں ہمارے پاس تین مثبت عدد صحیح اعداد کے تیرہ ایسے ملاپ  پائے جاتے ہیں جن کے مربعوں کا مجموعہ \num{363} ہوگا: تینوں اعداد گیارہ ہوسکتے ہیں دو اعداد تیرہ  اور ایک پانچ جو تین مرتب اجتماعات میں ہوگا ایک عدد اُٗنّیس اور دو ایک یہاں نھی تین مرتب اجتماعات میں یا ایک عدد سترہ ایک ساٹھ اعر ایک پانچ چھ مرتب اجتماعات میں ہوسکتے ہیں۔ یوں \(n_A, n_B, n_C\)  درج ذیل میں سے ایک ہوگا:
\begin{align*}
	(11, 11, 11)\\
	(13, 13, 5), (13, 5, 13), (5, 13, 13)\\
	(1, 1, 19), (1, 19, 1), (19, 1, 1)\\
	(5, 7, 17), (5, 17, 7), (7, 5, 17), (7, 17, 5), (17, 5, 7), (17, 7, 5).
\end{align*}
اگر یہ ذرات قابلِ ممیز ہوں تب ان میں سے ہر ایک کسی ایک منفرد کوانٹم حال کو ظاہر کرے گا اور شماریاتی میکانیات کے بنیادی مفرضہ کے تحت حراری توزن میں یہ سب برابر محتمل ہوں گے۔ لیکن میں اس میں دلچسپی نہیں رکھتا ہوں کہ کونسا ذرہ کس یک ذرہ حال میں پایا جاتا ہے بلکہ میں یہ جاننا چاہتا ہوں کہ ہر ایک حال میں کل کتنے ذرات پائے جاتے ہیں حال\(\psi_n\) کی تعداد مکین \(N_n\)۔ ہم اس دن ذرہ حال کے تمام تعدادِ مکین کے اجتماع کو تنظیم کہتے ہیں۔ اگر تینوں حال \(\psi_{11}\)  میں ہوں تب تنظیم درج ذیل ہوگا
\begin{align}
	(0, 0, 0, 0, 0, 0, 0, 0, 0, 0, 3, 0, 0, 0, 0, 0, 0, 0, \dots)
\end{align} 
یعنی \(N_{11}=3\) باقی تمام صفر اگر دو حال \(\psi_{13}\) میں اور ایک \(\psi_5\) میں ہو تب تنظیم درج ذیل ہوگا
\begin{align}
	(0, 0, 0, 0, 1, 0, 0, 0, 0, 0, 0, 0, 2, 0, 0, 0, 0, \dots)
\end{align}  
یعنی \(N_5=1, N_{13}=2\) باقی تمام صفر اگر دو \(\psi_1\) میں ایک \(\psi_{19}\) میں تب تنظیم درج ذیل ہوگا
\begin{align}
	(2, 0, 0, 0, 0, 0, 0, 0, 0, 0, 0, 0, 0, 0, 0, 0, 0, 0, 1, 0, \dots)
\end{align} 
یعنی\(N_1 = 2, N_{19} = 1\) باقی تمام صفر اور اگر ایک ذرہ \(\psi_5\) میں ایک \(\psi_7\) میں اور ایک \(\psi_{17}\) میں تب تنظیم درج ذیل ہوگا 
\begin{align}
	(0, 0, 0, 0, 1, 0, 1, 0, 0, 0, 0, 0, 0, 0, 0, 0, 1, 0, 0, \dots)
\end{align} 
یعنی \(N_5 = N_7 = N_{17} = 1, \text{\RL{باقی تمام صفر}}\) ان تمام میں آخری تنظیم زیادہ سے زیادہ محتمل ہوگی چونکہ اسکو چھ مختلف طریقوں سے حاصل کیا جاسکتا ہے جبکہ درمیانی دو کو تین طریقوں سے اور پہلی کو صرف ایک طریقہ سے حاصل کیا جاسکتا ہے۔

میں اب دوبارہ اپنے اصل سوال پر آتا ہوں کہ بلا واسطہ تین ذرات منتخب کرتے ہوئے کوئی مخصوص اجازتی توانائی \عددی{E_n} حاصل کرنے کا احتمال \عددی{P_n} کیا ہوگا؟ توانائی \عددی{E_1} صرف اس صورت حاصل ہوگا جب ذرہ تیسری تنظیم مساوات \num{5.71} میں ہو اس تنظیم میں نظام ہونے کا اتفاق تیرہ میں سے تین ہے اور اس تنظیم میں \عددی{E_1} کے حصول کا احتمال \(\frac{2}{3}\) لحاظہ\(P_1 =(\frac{3}{13})\times (\frac{2}{3})= \frac{2}{13}\)۔ آپ \عددی{E_5} کو تنظیم دو مساوات \num{5.70} تیرہ میں سے تین کا امکان جس کا احتمال \(\frac{1}{3}\) 	یا تنظیم چار مساوات \num{5.72} تیرہ میں سے چھ امکان اور احتمال \(\frac{1}{3}\) لحاظہ\(P_5 = (\frac{3}{13})\times(\frac{1}{3}) + (\frac{6}{13})\times(\frac{1}{3}) = \frac{3}{13}\)۔ آپ \عددی{E_7} کو صرف چار سے حاصل کرسکتے ہیں لحاظہ\(P_7 = (\frac{6}{13})\times(\frac{1}{3}) = \frac{2}{13}\)۔ اسی طرح \عددی{E_{11}} صرف پہلی تنظیم سے مساوات \num{5.69} سے تیرہ میں سے ایک امکان اور احتمال ایک کے ساتھ حاصل ہوگا لحاظہ\(P_{11} = (\frac{1}{13})\) ہوگا۔ اسی طرح \(P_{13} = (\frac{3}{13})\times(\frac{2}{3}) =\frac{2}{13}\)، \(P_{17} = (\frac{6}{13})\times(\frac{1}{3}) = \frac{2}{13}\) اور \(P_{19} = (\frac{3}{13})\times(\frac{1}{3}) = \frac{1}{13}\) ہوگا۔ انکی تصدیق درج ذیل سے ہوگی 
\begin{align*}
	P_1 + P_5 + P_7 + P_{11} + P_{13} + P_{17} + P_{19} = \frac{2}{13} + \frac{3}{13} + \frac{2}{13} + \frac{1}{13} + \frac{2}{13} + \frac{2}{13} + \frac{1}{13} = 1.
\end{align*} 

یہ قابلِ ممیز ذرات کے لیئے تھا۔ اس کی بجائے اگر ذرات یکساں فرمیان ہوتے اپنی آسانی کے لیئے چکر کع نظراندا کرتے ہوئے یا اگر آپ چاہیں تو یہ تصور کرتے ہوئے کہ تمام ایک جیسے چکر حال میں ہیں ضرورت خلاف تشاکلیت کی بنا پہلی تین تنظیم جو دو یا اس سے بھی برا تین ذرات کع ایک ہی حال میں ڈالتے ہیں خارجل امکان ہوں گے لحاظہ چوتھی تنظیم میں صرف ایک حال ہوگا سوال \num{5.22} الف دیکھیں۔ یکساں فرمیونز کے لیئے \(P_5 = P_7 = P_{17} = \frac{1}{3}\) ہوگا اور اب بھی احتمالات کا مجموعہ ایک ہے اس کے برعکس اگر ذرات یکساں بوزان ہوتے تب ضرورت تشاکلیت ہر تنظیم میں صرف ایک حال کی اجازت دیتا سوال \num{5.22} ب دیکھیں۔ لحاظہ\(P_1 = (\frac{1}{4})\times(\frac{2}{3}) = \frac{1}{6}\)، \(P_5 = (\frac{1}{4})\times(\frac{1}{3}) + (\frac{1}{4})\times(\frac{1}{3}) = \frac{1}{6}\)، \(P_7 = (\frac{1}{4})\times(\frac{1}{3}) = \frac{1}{12}\)، \(P_{11} = (\frac{1}{4})\times(1) = \frac{1}{4}\)،\(P_{13} = (\frac{1}{4})\times(\frac{2}{3}) = \frac{1}{6}\)، \(P_{17} = (\frac{1}{4})\times(\frac{1}{3}) = \frac{1}{12}\) اور \(P_{19} = (\frac{1}{4})\times(\frac{1}{3}) = \frac{1}{12}\) ہوتا۔ ہمیشہ کی طرح احتمالات کا مجموعہ ایک ہے۔

اس مثال کا مقصد آپ کو یہ دیکھانا تھا کہ ذرات کی قسم پر حالات کی شمار کس طرح منحصر ہے۔ ایک لحاظ سے ایک حقیقی صورتِحال سے جہاں \عددی{N} ایک بہت بڑا عدد ہوگا سے یہ مثال زیادہ پیچیدہ تھا۔ چونکہ \عددی{N} کی قیمت بڑھانے سے زیادہ محتمل تقسیم جو قابلِ ممیز ذرات کے لیئے اس مثال میں \(N_5 = N_7 = N_{17} = 1\) ہے پائے جانے کا امکان اتنا زیادہ ہوجائے گا کہ کسی بھی شماریاتی نقطہ نظر سے باقی تمام امکانات کو رد کیا جا سکتا ہے۔ توازن کی صورت میں انفرادی ذرہ توانائیوں کی تقسیم درحقیقت انکی زیادہ سے زیادہ محتمل تنظیم میں تقسیم ہے۔ اگر یہ \(N = 3\)  کے لیئے درست ہوتا جوکہ یہ نہیں ہے ہم قابلِ ممیز ذرات کے لیئے \(N = 3\) کی صور میں اخذ کرتے \(P_5 = P_7 = P_{17} = \frac{1}{3}\) میں حصہ 5.4.3 میں اس نقطہ پر دوبارہ آؤں گا لیکن اس سے پہلے گنتی کی ترکیب کو عمومیت دیتے ہیں۔

\ابتدا{سوال}

(الف) حال \(\psi_5\) میں ایک حال \(\psi_7\) میں ایک اور حال \(\psi_{17}\) میں ایک یکساں تین فرمیون کا مکمل خلاف تشاکل تفاعل موج \(\psi(x_A, x_B, x_C)\) تیار کریں۔

(ب) تین یکساں بوزان کے لیئے مکمل تشاکل تفاعل موج\(\psi(x_A, x_B, x_C)\) درج ذیل صورتوں میں تیار کریں (ا) تینوں حال \(\psi_{11}\) میں ہوں، (ب) اگر دو \(\psi_1\) اور ایک \(\psi_{19}\) میں ہو، (ج) اگر ایک حال \(\psi_5\) ایک حال \(\psi_7\) اور ایک حال \(\psi_{17}\) میں ہو۔ 
\انتہا{سوال}
\ابتدا{سوال}
فرض کریں یک بُعدی حارمونی ارتعاشی مخفیہ میں آپ کے پاس تین باہم غیر متعمل ذرات ہیں جو حراری توازن میں پائے جاتے ہیں جن کی کل توانائی\(E = (\frac{9}{2})\hbar\omega\) ہے۔

(الف) اگر یہ تمام ایک جیسی کمیت کے قابلِ ممہز ذرات ہوں تب انکی کتنی عدد مکین تنظیمات ہوں گے اور ہر ایک کے لیئے کتنے منفرد تین ذرہ حالات ہوں گے؟ سب سے زیادہ محتمل تنظیم کیا ہوگی؟ اگر آپ ایک ذرہ بلا منصوبہ منتخب کریں اور اسکی توانائی کی پیمائش کریں تب کیا قیمتیں متوقع ہوں گی؟ اور ہر ایک کا احتمال کیا ہوگا؟ سب سے زیادہ محتمل توانائی کیا ہوگی؟

(ب) یہی کچھ یکساں فرمیونز کے لیئے کریں چکر کو نظر انداز کریں جیسا ہم نے حصہ 5.4.1 میں کیا۔

(ج) یہی کچھ یکساں بوزان کے لیئے کریں چکر کو نظرانداز کریں۔ 
\انتہا{سوال}
%==============
%the above is prob 5.23 
%missing from sec 5.4.2 (p245)  till the end of the chapter
