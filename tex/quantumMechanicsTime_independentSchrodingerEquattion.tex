\باب{غیر تابع وقت شروڈنگر مساوات}

باب اول میں ہم نے تفاعل موج پر بات کی جہاں اس کا استعمال کرتے ہوئے دلچسپی کے مختلف مقداروں کا حساب کیا گیا۔ اب وقت آن پہنچا ہے کہ ہم کسی مخصوص خفی توانائی \( V (x،t) \) کیلیئے شروڈنگر مساوات

\begin{align}\label{مساوات_شروڈنگر_تابع_وقت}
i \hslash \frac{\partial \Psi}{\partial t} = - \frac{\hslash^{2}}{2 m} \frac{\partial^{2} \Psi}{\partial x^{2}} + V\Psi
\end{align}
  حل کرتے ہوئے  \( \Psi (x,t) \)   حاصل کرنا سیکھیں۔  اس باب میں (بلکہ کتاب کے بیشتر حصے میں) ہم فرض کرتے ہیں  کہ  \( V \) وقت \عددی{t} کا تابع نہیں ہے۔ ایسی صورت میں مساوات شروڈنگر کو \اصطلاح{علیحدگی متغیرات}\فرہنگ{علیحدگی متغیرات}\حاشیہب{separation of variables}\فرہنگ{variables!separation of} کے طریقے سے حل کیا جا سکتا ہے، جو ماہر طبیعیات کا پسندیدہ طریقہ ہے۔ ہم ایسے حل تلاش کرتے ہیں جنہیں حاصل ضرب
\begin{align}
\Psi (x,t) = \psi (x)  \varphi (t)
\end{align}

 کی صورت میں لکھنا ممکن ہو جہاں \عددی{\psi} صرف \عددی{x} اور \عددی{\varphi} صرف \عددی{t}  کا تفاعل ہے۔ ظاہری طور پر حل پر ایسی شرط مسلط کرنا درست قدم نظر نہیں آتا ہے لیکن حقیقت میں یوں حاصل کردہ حل بہت کار آمد ثابت ہوتے ہیں۔ مزید (جیسا کہ علیحدگی متغیرات کیلئے عموماً ہوتا ہے)  ہم علیحدگی متغیرات سے حاصل حلوں کو یوں آپس میں جوڑ سکتے ہیں کہ ان سے عمومی حل حاصل کرنا ممکن ہو۔ قابل علیحدگی حلوں کیلئے درج ذیل ہو گا 
\begin{align*}
\frac{\partial \Psi }{\partial t} = \psi \frac{\dif \varphi}{\dif t}, \quad \frac{\partial^{2} \Psi}{\partial x^{2}} = \frac{\dif^{\,2} \Psi}{\dif x^{2}} \varphi
\end{align*}
جو سادہ تفرقی مساوات ہیں۔  ان کی مدد سے مساوات شروڈنگر درج ذیل روپ اختیار کرتی ہے۔
\begin{align*}
i \hslash \psi \frac{\dif \varphi}{\dif t} = - \frac{\hslash^{2}}{2m} \frac{\dif^{\,2} \psi}{\dif x^{2}} \varphi + V\psi\varphi
\end{align*}
دونوں اطراف کو \عددی{\psi\varphi} سے تقسیم کرتے ہیں۔
\begin{align}\label{مساوات_شروڈنگر_علاحدہ_الف}
i\hslash\frac{1}{\varphi} \frac{\dif \varphi}{\dif t} = - \frac{\hslash^{2}}{2m} \frac{1}{\psi} \frac{\dif^{\,2} \psi}{\dif x^{2}} + V
\end{align}
اب بائیں ہاتھ تفاعل صرف \عددی{t} کا تابع ہے جبکہ دایاں ہاتھ تفاعل صرف \عددی{x} کا تابع ہے۔ یاد رہے اگر \عددی{V} از خود \عددی{x} اور  \عددی{t} دونوں پر منحصر ہو تب ایسا نہیں ہو گا۔  صرف \عددی{t} تبدیل ہونے سے دایاں ہاتھ کسی صورت تبدیل نہیں ہو سکتا ہے جبکہ بایاں ہاتھ اور دایاں ہاتھ لازمی طور پر ایک دوسرے کے برابر ہیں لحاضہ \عددی{t} تبدیل کرنے سے بایاں ہاتھ بھی تبدیل نہیں ہو گا۔اسی طرح صرف \عددی{x} تبدیل کرنے سے بایاں ہاتھ تبدیل نہیں ہو سکتا ہے اور چونکہ دونوں اطراف لازماً ایک دوسرے کے برابر ہیں لہٰذا \عددی{x} تبدیل کرنے سے دایاں ہاتھ بھی تبدیل نہیں ہو گا۔  ہم کہہ سکتے ہیں کہ دونوں اطراف ایک مستقل کے برابر ہوں گے۔ (یہاں تسلی کر لیں کہ آپ کو یہ دلائل سمجھ آ گئے ہیں۔) اس مستقل کو ہم علیحدگی مستقل کہتے ہیں جس کو ہم 
\عددی{E} سے ظاہر کرتے ہیں۔ یو مساوات \حوالہ{مساوات_شروڈنگر_علاحدہ_الف} درج ذیل لکھی جا سکتی ہے۔ 
\begin{align}\label{مساوات_شروڈنگر_علیحدہ_اول}
i\hslash \frac{1}{\varphi} \frac{\dif \varphi}{\dif t} &= E\nonumber\\
\frac{\dif \varphi }{\dif t} &= - \frac{i E}{\hslash} \varphi &&\text{\RL{یا}}
\end{align}
اور
\begin{align}\label{مساوات_شروڈنگر_علیحدہ_دوم}
- \frac{\hslash^{2}}{2m} \frac{1}{\psi} \frac{\dif^{\,2} \psi}{\dif x^{2}} + V &= E\nonumber\\
- \frac{\hslash^{2}}{2m} \frac{\dif^{\,2} \psi}{\dif x^{2}} +V\psi &= E\psi&&\text{\RL{یا}}
\end{align}
علیحدگی متغیرات نے ایک جزوی تفرقی مساوات کو دو سادہ تفرقی مساوات (مساوات \حوالہ{مساوات_شروڈنگر_علیحدہ_اول} اور \حوالہ{مساوات_شروڈنگر_علیحدہ_اول}) میں علیحدہ کیا۔ ان میں سے پہلی  (مساوات \حوالہ{مساوات_شروڈنگر_علیحدہ_اول}) کو حل کرنا بہت آسان ہے۔ دونوں اطراف کو \عددی{\dif t} سے ضرب دیتے ہوئے تکمل لیں۔ یوں عمومی حل \عددی{C e^{-iEt/\hslash}} حاصل ہو گا۔ چونکہ ہم   حاصل ضرب\عددی{\psi\varphi } میں دلچسپی رکھتے ہیں لہذا ہم مستقل  \عددی{C} کو  \عددی{\psi}  میں ضم کر سکتے ہیں۔ یوں مساوات \حوالہ{مساوات_شروڈنگر_علیحدہ_اول} کا حل درج ذیل لکھا جا سکتا ہے۔ 
\begin{align}
\varphi(t) = e^{-iEt/\hslash}
\end{align}
دوسری  (مساوات \حوالہ{مساوات_شروڈنگر_علیحدہ_دوم})  کو \اصطلاح{غیر تابع وقت شروڈنگر مساوات}\فرہنگ{شروڈنگر!غیر تابع وقت}\حاشیہب{time-independent Schrodinger equation}\فرہنگ{Schrodinger!time-independent} کہتے ہیں۔ پوری طرح مخفی توانائی \عددی{V} جانے بغیر ہم آگے نہیں بڑھ سکتے ہیں۔

اس باب کے بقیہ حصے میں ہم  مختلف سادہ خفی توانائی کیلئےغیر تابع وقت شروڈنگر مساوات حل کریں گے۔ ایسا کرنے سے پہلے آپ پوچھ سکتے ہیں کہ علیحدگی متغیرات کی کیا خاص بات ہے؟ بہرحال تابع وقت شروڈنگر مساوات کے زیادہ تر حل \عددی{\psi (x) \varphi (t)}  کی صورت میں نہیں لکھے جا سکتے۔ میں اس کے تین جوابات دیتا ہوں۔ ان میں سے دو طبعی اور ایک ریاضیاتی ہو گا۔ 

\عددی{(1}\quad 
یہ \اصطلاح{ساکن حالات}\فرہنگ{ساکن!حالات} ہیں۔ اگرچہ طفاعل موج ازخود 
\begin{align}
\Psi (x,t) = \psi (x) e^{-iEt/\hslash}
\end{align}
وقت \عددی{t} کا تابع ہے،  \ترچھا{کثافت احتمال}
\begin{align}
\left| \Psi (x,t) \right|^{2} = \Psi^{*}\Psi = \psi^{*}e^{+iEt/\hslash} \psi e^{-iEt/\hslash} = \left| \psi (x) \right|^{2}
\end{align}
وقت کا تابع نہیں ہے؛ تابعیت وقت کٹ جاتی ہے۔ یہی کچھ کسی بھی حرکی متغیر کی توقعاتی قیمت کے حساب میں ہو گا۔ مساوات \حوالہ{مساوات_تفاعل_موج_توقعاتی_قیمت_حصول} تخفیف کے بعد درج زیل صورت اختیار کرتی ہے۔ 
\begin{align}
\langle Q(x,p) \rangle = \int \psi^{*} Q \left( x, \frac{\hslash}{i} \frac{\dif}{\dif x} \right) \psi \dif x
\end{align}
ہر توقعاتی قیمت \ترچھا{وقت} میں \ترچھا{مستقل} ہو گی؛ یہاں تک کہ ہم  \عددی{\varphi (t) } کو رد کر کے  \عددی{\Psi} کی
 جگہ  \عددی{\psi} استعمال کر کے وہی نتائج حاصل کر سکتے ہیں۔ اگرچہ بعض اوقات  \عددی{\psi}  کو ہی تفاعل موج پکارا جاتا ہے، لیکن ایسا کرنا حقیقتاً غلط ہے جس سے مسئلے کھڑے ہو سکتے ہیں۔ یہ ضروری ہے کہ آپ یاد رکھیں کہ اصل تفاعل موج ہر صورت تابع وقت ہو گا۔ بالخصوص  \( \langle x \rangle \) مستقل  ہو گا لہٰذا (مساوات \حوالہ{مساوات_تفاعل_موج_تعریف_معیار_حرکت} کے تحت)   \( \langle p \rangle = 0 \)  ہو گا۔ ساکن حال میں کبھی بھی کچھ نہیں ہوتا ہے۔ 

\عددی{(2} \quad
یہ \ترچھا{غیر مبہم کل توانائی} کے حالات ہوں گے۔ کلاسیکی میکانیات میں کل توانائی (حرکی جمع خفی) کو \اصطلاح{ہیملٹنی}\فرہنگ{ہیملٹنی}\حاشیہب{Hamiltonian}\فرہنگ{Hamiltonian} کہتے ہیں جس کو \عددی{H} سے ظاہر کیا جاتا ہے۔ 
\begin{align}
H(x,p) = \frac{p^{2}}{2m} + V(x)
\end{align}
اس کا مطابقتی ہیملٹنی عامل، قواعدو ظوابط کے تحت  \عددی{p \rightarrow (\hslash/i)(\partial/\partial x)}  پر کر کے درج ذیل حاصل ہو گا۔ 
\begin{align}
\hat{H} = - \frac{\hslash^{2}}{2m} \frac{\partial^{2}}{\partial x^{2}} + V(x) 
\end{align}
یوں غیر تابع وقت شروڈنگر مساوات \حوالہ{مساوات_شروڈنگر_علیحدہ_دوم} درج ذیل روپ اختیار کریگی 
\begin{align}
\hat{H} \psi = E\psi
\end{align}

جس کے کل توانائی کی توقعاتی قیمت درج ذیل ہو گی۔ 
\begin{align}
\langle H \rangle = \int \psi^{*} \hat{H}\psi \dif x = E \int \left| \psi \right|^{2} \dif x = E \int \left| \Psi \right|^{2} \dif x = E
\end{align}
آپ دیکھ سکتے ہیں کہ \عددی{\psi} کی معمول زنی \عددی{\psi} کی معمول زنی کے مترادف ہے۔ مزید درج ذیل 
\begin{align*}
\hat{H}^{2} \psi = \hat{H} (\hat{H}\psi ) = \hat{H} ( E\psi ) = E (\hat{H} \psi ) = E^{2} \psi 
\end{align*}
کی بنا درج ذیل ہوگا۔
\begin{align*}
\langle H^{2} \rangle = \int \psi^{*} \hat{H}^{2} \psi \dif x = E^{2} \int \left| \psi \right|^{2} \dif x = E^{2}
\end{align*}
یوں\عددی{H} کی تغیریت درج ذیل ہو گی۔ 
\begin{align}
\sigma^{2}_{H} = \langle H^{2} \rangle - \langle H \rangle^{2} =E^{2} - E^{2} = 0
\end{align}
یاد رہے کہ  \عددی{\sigma = 0}  کی صورت میں تمام ارکان کی قیمت ایک دوسری جیسی ہو گی (تقسیم کا پھیلاؤ صفر ہو گا)۔ نتیجتاً قابل علیحدگی حل کی ایک خاصیت یہ ہو ہے کہ کل توانائی کی ہر پیمائش یقیناً ایک ہی قیمت \عددی{E} دے گی۔ (اسی کی بنا  علیحدگی مستقل کو \عددی{E} سے ظاہر کیا گیا۔) 

\عددی{(3}\quad
عمومی حل قابل علیحدگی حلوں کا \اصطلاح{خطی جوڑ}\فرہنگ{خطی جوڑ}\حاشیہب{linear combination}\فرہنگ{linear!combination} ہو گا۔ جیسا ہم جلد دیکھیں گے،  غیر تابع وقت شروڈنگر مساوات  (مساوات \حوالہ{مساوات_شروڈنگر_علیحدہ_دوم})   لامتناعی تعداد کے حل \عددی{(\psi_{1}(x),\, \psi_{2}(x),\, \psi_{3}(x), \cdots)}  دے گا جہاں ہر ایک حل کے ساتھ ایک علحدگی مستقل \عددی{(E_1,E_2,E_3,\cdots)} منسلک ہو گا لہٰذا  ہر \اصطلاح{اجازتی توانائی}\فرہنگ{توانائی!اجازتی}\حاشیہب{allowed energy}\فرہنگ{energy!allowed} کا ایک منفرد تفاعل موج پایا جائے گا۔ 
\begin{align*}
\Psi_{1} (x,t) = \psi_{1}(x)e^{-iE_{1}t/\hslash} , \quad \Psi_{2} (x,t) = \psi_{2}(x)e^{-iE_{2}t/\hslash}, \, \cdots 
\end{align*}
اب (جیسا کہ آپ خود تصدیق کر سکتے ہیں)  تابع وقت شروڈنگر مساوات  (مساوات \حوالہ{مساوات_شروڈنگر_تابع_وقت}) کی ایک خاصیت یہ ہے کہ اس کے حلوں کا ہر خطی جوڑ ازخود ایک حل ہو گا۔ ایک بار قابل علیحدگی حل تلاش کرنے کے بعد ہم زیادہ عمومی حل درج ذیل روپ میں تیار کر سکتے ہیں۔
\begin{align}\label{مساوات_شروڈنگر_خطی_جوڑ_عمومی_حل}
\Psi (x,t) = \sum_{n=1}^{\infty} c_{n} \psi_{n}(x)e^{-iE_{n}t/\hslash}
\end{align}
حقیقتاً  تابع وقت شروڈنگر مساوات کا ہر حل درج بالا روپ میں لکھا جا سکتا ہے۔ ایسا کرنے کی خاطر ہمیں وہ مخصوص مستقل 
\عددی{(c_{1},\, c_{2}, \, \cdots )}  تلاش کرنے ہوں گے جن کو استعمال کرتے ہوئے درج بالا حل (مساوات \حوالہ{مساوات_شروڈنگر_خطی_جوڑ_عمومی_حل}) ابتدائی شرائط مطمئن کرتا ہو۔ آپ آنے والے حصوں میں دیکھیں گے کہ ہم کس طرح یہ سب کچھ کر پائیں گے۔ باب \حوالہء{3} میں  ہم اس کو زیادہ مضبوط بنیادوں پر کھڑا کر پائیں گے۔ بنیادی نقطہ یہ ہے کہ ایک بار  غیر تابع وقت شروڈنگر مساوات حل کرنے کے بعد آپ کے مسائل ختم ہو جاتے ہیں۔ یہاں سے تابع وقت شروڈنگر مساوات کا عمومی حل حاصل کرنا آسان کام ہے۔ 

گذشتہ چار صفحات میں ہم بہت کچھ کہا جا چکا ہے۔ میں ان کو مختصراً اور مختلف نقطہ نظر سے دوبارہ پیش کرتا ہوں۔ زیر غور عمومی مسئلہ کا غیر تابع وقت خفی توانائی  \عددی{V(x)}  اور ابتدائی تفاعل موج  \عددی{\Psi (x,0)} دیے  گئے ہوں گے۔ آپ کو مستقبل کے تمام \عددی{t}  کیلئے  \عددی{ \Psi (x,t)} تلاش کرنا ہوگا۔ ایسا کرنے کی خاطر آپ تابع وقت شروڈنگر مساوات  (مساوات \حوالہ{مساوات_شروڈنگر_تابع_وقت}) حل کریں گے۔ پہلی قدم میں  آپ غیر تابع وقت شروڈنگر مساوات  (مساوات \حوالہ{مساوات_شروڈنگر_علیحدہ_دوم})  حل کر کے  لا متناہی تعداد کے حلوں  کا سلسلہ   \عددی{(\psi_{1}(x),\, \psi_{2}(x),\, \psi_{3}(x), \cdots)} حاصل کریں گے جہاں ہر ایک
 کی منفرد توانائی  \عددی{(E_{1}, \, E_{2}, \, E_{3}, \, \cdots)} ہو گی۔ ٹھیک ٹھیک \عددی{\Psi (x,0)} پر بیٹھنے کی خاطر آپ ان حلوں کا  خطی جوڑ لیں گے۔
\begin{align}
\Psi (x,0) = \sum_{n=1}^{\infty} c_{n} \psi_{n}(x)
\end{align}
یہاں کمال کی  بات یہ ہے کہ کسی بھی ابتدائی حال کے لئے آپ ہر صورت  مستقل \عددی{c_{1}, \,c_{2}, \,c_{3}, \cdots} دریافت کر پائیں گے۔ تفاعل موج \عددی{\Psi(x,t)}  تیار کرنے کی خاطر آپ ہر جزو کے ساتھ مختص تابعیت وقت \عددی{e^{-iE_nt/\hslash}} چسپاں کریں گے۔ 
\begin{align}\label{مساوات_شروڈنگر_عمومی_حل_مجموعہ}
\Psi (x,t) = \sum_{n=1}^{\infty} c_{n} \psi_{n}(x)e^{-iE_{n}t/\hslash} = \sum_{n=0}^{\infty} c_{n} \Psi_{n} (x,t)
\end{align}
چونکہ قابل علیحدگی حل
\begin{align}
\Psi_{n} (x,t) = \psi_{n}(x) e^{-iE_{n}t/\hslash}
\end{align}
کے تمام احتمال اور توقعاتی قیمتیں غیر تابع وقت ہوں گی لہٰذا یہ  از خود ساکن حالات ہوں گے، تا ہم عمومی حل  (مساوات \حوالہ{مساوات_شروڈنگر_عمومی_حل_مجموعہ}) یہ خاصیت نہیں رکھتا ہے؛  انفرادی ساکن حالات کی توانائیاں ایک دوسرے سے مختلف ہونے کی بنا \عددی{\left| \Psi \right|^{2}} کا حساب کرتے ہوئے  قوت نمائی ایک دوسرے کو حذف نہیں کرتی ہیں۔ 


\ابتدا{مثال}
فرض کریں  ایک ذرہ ابتدائی طور پر دو ساکن حالات کا خطی جوڑ ہو: 
\begin{align}
\Psi (x,0) = c_{1} \psi_{1}(x) + c_{2} \psi_{2}(x) 
\end{align}
(چیزوں کو سادہ رکھنے کی خاطر میں فرض کرتا ہوں کے مستقل  \عددی{c_{n}} اور حالات  \عددی{\psi_{n} (x)} حقیقی ہیں۔) مستقبل وقت t کیلئے تفاعل موج  \عددی{\Psi (x,t)} کیا ہو گا ؟ کثافت احتمال تلاش کریں اور ذرے کی حرکت  بیان کریں۔ 

حل:\quad
اسکا پہلا حصہ آسان ہے
\begin{align*}
\Psi (x,t) = c_{1} \psi_{1}(x)e^{-iE_{1}t/\hslash} + c_{2} \psi_{2}(x)e^{-iE_{2}t/\hslash}
\end{align*}
جہاں  \عددی{E_{1}} اور \عددی{E_{2}}  بالترتیب تفاعل  \عددی{\psi_{1}} اور  \عددی{\psi_{2}} کی مطابقتی توانائیاں ہیں۔ یوں درج ذیل ہو گا۔ 
\begin{align*}
\left| \Psi (x,t) \right|^{2} &= \left( c_{1} \psi_{1} e^{iE_{1}t/\hslash} + c_{2} \psi_{2} e^{iE_{2}t/\hslash} \right) \left( c_{1} \psi_{1} e^{-iE_{1}t/\hslash} + c_{2} \psi_{2} e^{-iE_{2}t/\hslash} \right) \\
&= c_{1}^{2} \psi_{1}^{2} + c_{2}^{2} \psi_{2}^{2} + 2c_{1}c_{2}\psi_{1}\psi_{2} \cos [ ( E_{2} - E_{1})t/\hslash]
\end{align*}
(میں نے نتیجہ کی سادہ صورت حاصل کرنے کی خاطر کلیہ یولر \عددی{e^{i\theta}=\cos\theta+i\sin\theta} استعمال کیا۔)  ظاہری طور پر  کثافت احتمال زاویائی تعدد \عددی{(\tfrac{E_2-E_1}{\hslash})} سے سائن نما ارتعاش کرتا ہے لہٰذا یہ ہرگز ساکن حال نہیں ہو گا۔ لیکن دھیان رہے کہ (ایک دوسرے سے مختلف) تونائیوں کے تفاعلات کے خطی جوڑ نء حرکت پیدا کیا۔ 
\انتہا{مثال}
%===========================
%%%%%%%%%%%%%%%
%%%%%%%%%%%%%%%%%%%%%%%
% Missing region here
%=========================

\ابتدا{مثال} 
ہم نے دیکھا  کہ مثال \حوالہء{2.2} میں ابتدائی تفاعل موج (شکل \حوالہء{2.3}) زمینی حال \عددی{\psi_{1}} (شکل \حوالہء{2.2}) کے ساتھ قریبی مشابہت رکھتا ہے۔ یوں ہم توقع کرتے گے کہ \عددی{\left| c_{1} \right|^{2}}  غالب ہو گا۔ یقیناً ایسا ہی ہے۔
\begin{align*}
\left| c_{1} \right|^{2} = \left( \frac{8\sqrt{15}}{\pi^{3}} \right)^{2} = 0.998555 \cdots
\end{align*}
باقی تمام عددی سر مل کر فرق دیتے ہیں:
\begin{align*}
\sum_{n=1}^{\infty} \left| c_{n} \right|^{2} = \left( \frac{8\sqrt{15}}{\pi^{3}} \right)^{2} \sum_{n=1,3,5,...}^{\infty} \frac{1}{n^{6}} = 1
\end{align*}
اس مثال میں توانائی کی توقعاتی قیمت ہماری توقعات کے عین مطابق درج ذیل ہے۔
\begin{align*}
\langle H \rangle = \sum_{n=1,3,5,...}^{\infty} \left( \frac{8\sqrt{15}}{n^{3} \pi^{3}} \right)^{2} \frac{n^{2} \pi^{2} \hbar^{2}}{2ma^{2}} = \frac{480\hbar^{2}}{\pi^{4} ma^{2}} \sum_{n=1,3,5,...}^{\infty} \frac{1}{n^{4}} = \frac{5 \hbar^{2}}{ma^{2}}
\end{align*}
 یہ \عددی{ E_{1} = \pi^{2} \hbar^{2}/2ma^{2}  } کے بہت قریب،  حجان حل حالتوں کی شمول کی بنا معمولی زیادہ ہے۔ 
\انتہا{مثال}
%===========
\ابتدا{سوال}\شناخت{سوال_شروڈنگر_حل_ناقابل_قبول}
دکھائیں کہ لا متناہی چوکور کنواں کے لئے  \عددی{ E = 0 } یا \عددی{ E < 0 } کی صورت میں  غیر تابع وقت شروڈنگر مساوات کا کوئی بھی قابل قبول حل  نہیں پایا جاتا ہے۔ (یہ سوال \حوالہء{2.2} میں دیے گئے عمومی مسئلے کی ایک خصوصی صورت ہے، لیکن اس بار شروڈنگر مساوات کو صریحاً حل کرتے ہوئے دکھائیں کہ آپ سرحدی شرائط پر پورا نہیں اتر سکتے ہیں۔)
 \انتہا{سوال}
%==============
\ابتدا{سوال}
لامتناہی چکور کنواں کے \عددی{ n } وی ساکن حال کیلئے \عددی{\langle x \rangle}، \عددی{\langle x^2 \rangle} ، \عددی{\langle p \rangle}، \عددی{\langle p^2 \rangle}، \عددی{\sigma_x}  اور \عددی{\sigma_p}  تلاش کریں۔ تصدیق کریں کہ اصول غیر یقینیت مطمئن ہوتا ہے۔ کونسا حال غیر یقینیت کی حد کے قریب ترین ہو گا؟
\انتہا{سوال}
%=============
\ابتدا{سوال}\شناخت{سوال_شروڈنگر_لامتناہی_کنواں_برابر_حصے}
لامتناہی چکور کنواں میں ایک ذرے کا ابتدائی تفاعل موج اولین دو ساکن حالات کے برابر حصوں کا مرکب ہے۔ 
\begin{align*}
\Psi(x,0) = A[\psi_{1}(x) + \psi_{2}(x)]
\end{align*}
\begin{enumerate}[a.]
\item 
\عددی{ \Psi(x,0) } کو معمول پر لائیں۔ (یعنی \عددی{ A } تلاش کریں۔ آپ \عددی{ \psi_{1} } اور \عددی{ \psi_{2} } کی معیاری عمودیت بروئے کار لاتے ہوئے با آسانی ایسا کر سکتے ہیں۔ یاد رہے کہ \عددی{ t=0 } پر \عددی{  \Psi  } کو معمول پر لانے کے بعد آپ یقین رکھ سکتے  ہیں کہ یہ معمول شدہ ہی رہے گا۔ اگر آپ کو شک ہے، جزو-ب کا نتیجہ حاصل کرنے کے بعد  اس کی صریحاً تصدیق کریں۔) 
\item
\عددی{ \Psi(x,t) } اور \عددی{ \left| \Psi (x,t) \right|^{2} } تلاش کریں۔ موخر الذکر  کو وقت کے سائن نما تفاعل کی صورت میں لکھیں، جیسا مثال \حوالہء{2.1} میں کیا گیا۔ نتائج کو سادہ صورت میں لکھنے کی خاطر \عددی{\omega\equiv\tfrac{\pi^2\hslash}{2ma^2}} لیں۔ 
\item 
\عددی{ \langle x \rangle  } تلاش کریں۔ آپ دیکھیں گے کہ یہ وقت کے ساتھ ارتعاش کرتا ہے۔ اس ارتعاش کی زاویائی تعدد کتنی ہو گی؟ ارتعاش کا حیطہ کیا ہو گا؟ (اگر آپکا حیطہ \عددی{ \tfrac{a}{2}} سے زیادہ ہو تب آپ کو جیل بھیجنے کی ضرورت ہو گی۔) 
\item 
\عددی{ \langle p \rangle  } تلاش کریں (اور اس پہ زیادہ وقت صرف نہ کریں)۔ 
\item
اس ذرے کی توانائی کی پیمائش سے کون کون سی قیمتیں متوقع ہیں؟ اور ہر ایک قیمت کا احتمال کتنا ہو گا؟ \عددی{ H } کی توقعاتی قیمت تلاش کریں۔ اس کی قیمت کا موازنہ \عددی{ E_{1} } اور \عددی{ E_{2} } کے ساتھ کریں؟
\end{enumerate}
\انتہا{سوال}
%================
\ابتدا{سوال}
اگر چہ تفاعل موج کا مجموعی زاویائی مستقل کسی با معنی طبعی اہمیت کا حامل نہیں ہے (چونکہ یہ کسی بھی قابل پیمائش مقدار میں کٹ جاتا ہے) لیکن  مساوات \حوالہ{مساوات_شروڈنگر_عمومی_حل_مجموعہ} میں عددی سروں کے اضافی زاویائی مستقل اہمیت کے حامل ہیں۔ مثال کے طور پر ہم سوال \حوالہ{سوال_شروڈنگر_لامتناہی_کنواں_برابر_حصے} میں  \عددی{ \psi_{1} } اور \عددی{ \psi_{2} } کے اضافی زاویائی مستقل تبدیل کرتے ہیں:
\begin{align*}
\Psi (x,0) = A[\psi_{1} (x) + e^{i\phi}\psi_{2}(x)]
\end{align*}
جہاں \عددی{ \phi } کوئی مستقل ہے۔ \عددی{  \Psi(x,t) }، \عددی{  \ \left| \Psi (x,t) \right|^{2} } اور \عددی{ \langle x \rangle } تلاش کر کے ان کا موازنہ پہلے حاصل شدہ نتائج کے ساتھ کریں۔ بالخصوص \عددی{ \phi = \pi/2 } اور  \عددی{ \phi = \pi } کی صورتوں پر غور کریں۔ 
\انتہا{سوال}
%===========
\ابتدا{سوال} 
لا متناہی چکور کنواں میں ایک ذرے کا ابتدائی تفاعل موج درج ذیل ہے۔
\begin{align*}
\Psi (x,0) = 
\begin{cases}
Ax, & 0 \leq x \leq a/2 \\ 
A(a-x), & a/2 \leq x \leq a
\end{cases}
\end{align*}
\begin{enumerate}[a.]
\item 
\عددی{\Psi(x,0)} کا خاکہ کھینچیں اور مستقل \عددی{A} کی قیمت تلاش کریں۔
\item  
\عددی{\Psi(x,t) } تلاش کریں۔
\item  
توانائی کی پیمائش کا نتیجہ \عددی{E_{1}} ہونے کا احتمال کتنا ہو گا؟
\item 
توانائی کی توقعاتی قیمت تلاش کریں۔
\end{enumerate}
\انتہا{سوال}
%===========
\ابتدا{سوال}
ایک لامتناہی چکور کنواں، جسکی چوڑائی \عددی{a} ہے، میں کمیت \عددی{m} کا ایک ذرہ کنویں کے بائیں حصے سے ابتدا ہوتا ہے اور یہ \عددی{t=0} پر بائیں نصف حصے کے کسی بھی نقطے پر ہو سکتا ہے۔
\begin{enumerate}[a.]
\item
اس کی ابتدائی تفاعل موج \عددی{\Psi(x,0) } تلاش کریں۔ (فرض کریں کے یہ حقیقی ہے اور اسے معمول پر لانا نا بھولیے گا۔)
\item 
پیمائش توانائی کا نتیجہ \عددی{\pi^{2}\hbar^{2}/2ma^{2}} ہونے کا احتمال کیا ہو گا؟ 
\end{enumerate} 
\انتہا{سوال}
%==========
\ابتدا{سوال}
لمحہ \عددی{t=0}  پر مثال \حوالہء{2.2} کے تفاعل موج کیلئے \عددی{H} کی توقعاتی قیمت تکمل کے ذریعہ حاصل کریں۔ 
\begin{align*}
\langle H \rangle=\int \Psi(x,0)^*\hat{H}\, \Psi(x,0)\dif x
\end{align*}
مثال \حوالہ{سوال_شروڈنگر_حل_ناقابل_قبول} میں مساوات \حوالہء{2.39} کی مدد سے حاصل کردہ نتیجے کے ساتھ موازنہ کریں۔ دھیان رہے کیونکہ \عددی{H}  غیر تابع وقت ہے لہٰذا \عددی{t=0} لینے سے نتیجے پر کوئی اثر نہیں ہو گا۔ 
\انتہا{سوال}


