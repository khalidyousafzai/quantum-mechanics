\باب{غیر تابع وقت شروڈنگر مساوات}
\حصہ{ساکن حالات}
باب اول میں ہم نے تفاعل موج پر بات کی جہاں اس کا استعمال کرتے ہوئے دلچسپی کے مختلف مقداروں کا حساب کیا گیا۔ اب وقت آن پہنچا ہے کہ ہم کسی مخصوص خفی توانائی \( V (x,t) \) کی لئے شروڈنگر مساوات

\begin{align}\label{مساوات_شروڈنگر_تابع_وقت}
i \hslash \frac{\partial \Psi}{\partial t} = - \frac{\hslash^{2}}{2 m} \frac{\partial^{2} \Psi}{\partial x^{2}} + V\Psi
\end{align}
  حل کرتے ہوئے  \( \Psi (x,t) \)   حاصل کرنا سیکھیں۔  اس باب میں (بلکہ کتاب کے بیشتر حصے میں) ہم فرض کرتے ہیں  کہ  \( V \) وقت \عددی{t} کا تابع نہیں ہے۔ ایسی صورت میں مساوات شروڈنگر کو \اصطلاح{علیحدگی متغیرات}\فرہنگ{علیحدگی متغیرات}\حاشیہب{separation of variables}\فرہنگ{variables!separation of} کے طریقے سے حل کیا جا سکتا ہے، جو ماہر طبیعیات کا پسندیدہ طریقہ ہے۔ ہم ایسے حل تلاش کرتے ہیں جنہیں حاصل ضرب
\begin{align}
\Psi (x,t) = \psi (x)  \varphi (t)
\end{align}

 کی صورت میں لکھنا ممکن ہو جہاں \عددی{\psi} صرف \عددی{x} اور \عددی{\varphi} صرف \عددی{t}  کا تفاعل ہے۔ ظاہری طور پر حل پر ایسی شرط مسلط کرنا درست قدم نظر نہیں آتا ہے لیکن حقیقت میں یوں حاصل کردہ حل بہت کار آمد ثابت ہوتے ہیں۔ مزید (جیسا کہ علیحدگی متغیرات کیلئے عموماً ہوتا ہے)  ہم علیحدگی متغیرات سے حاصل حلوں کو یوں آپس میں جوڑ سکتے ہیں کہ ان سے عمومی حل حاصل کرنا ممکن ہو۔ قابل علیحدگی حلوں کیلئے درج ذیل ہو گا 
\begin{align*}
\frac{\partial \Psi }{\partial t} = \psi \frac{\dif \varphi}{\dif t}, \quad \frac{\partial^{2} \Psi}{\partial x^{2}} = \frac{\dif^{\,2} \Psi}{\dif x^{2}} \varphi
\end{align*}
جو سادہ تفرقی مساوات ہیں۔  ان کی مدد سے مساوات شروڈنگر درج ذیل روپ اختیار کرتی ہے۔
\begin{align*}
i \hslash \psi \frac{\dif \varphi}{\dif t} = - \frac{\hslash^{2}}{2m} \frac{\dif^{\,2} \psi}{\dif x^{2}} \varphi + V\psi\varphi
\end{align*}
دونوں اطراف کو \عددی{\psi\varphi} سے تقسیم کرتے ہیں۔
\begin{align}\label{مساوات_شروڈنگر_علاحدہ_الف}
i\hslash\frac{1}{\varphi} \frac{\dif \varphi}{\dif t} = - \frac{\hslash^{2}}{2m} \frac{1}{\psi} \frac{\dif^{\,2} \psi}{\dif x^{2}} + V
\end{align}
اب بائیں ہاتھ تفاعل صرف \عددی{t} کا تابع ہے جبکہ دایاں ہاتھ تفاعل صرف \عددی{x} کا تابع ہے۔ یاد رہے اگر \عددی{V} از خود \عددی{x} اور  \عددی{t} دونوں پر منحصر ہو تب ایسا نہیں ہو گا۔  صرف \عددی{t} تبدیل ہونے سے دایاں ہاتھ کسی صورت تبدیل نہیں ہو سکتا ہے جبکہ بایاں ہاتھ اور دایاں ہاتھ لازمی طور پر ایک دوسرے کے برابر ہیں لحاضہ \عددی{t} تبدیل کرنے سے بایاں ہاتھ بھی تبدیل نہیں ہو گا۔اسی طرح صرف \عددی{x} تبدیل کرنے سے بایاں ہاتھ تبدیل نہیں ہو سکتا ہے اور چونکہ دونوں اطراف لازماً ایک دوسرے کے برابر ہیں لہٰذا \عددی{x} تبدیل کرنے سے دایاں ہاتھ بھی تبدیل نہیں ہو گا۔  ہم کہہ سکتے ہیں کہ دونوں اطراف ایک مستقل کے برابر ہوں گے۔ (یہاں تسلی کر لیں کہ آپ کو یہ دلائل سمجھ آ گئے ہیں۔) اس مستقل کو ہم علیحدگی مستقل کہتے ہیں جس کو ہم 
\عددی{E} سے ظاہر کرتے ہیں۔ یو مساوات \حوالہ{مساوات_شروڈنگر_علاحدہ_الف} درج ذیل لکھی جا سکتی ہے۔ 
\begin{align}\label{مساوات_شروڈنگر_علیحدہ_اول}
i\hslash \frac{1}{\varphi} \frac{\dif \varphi}{\dif t} &= E\nonumber\\
\frac{\dif \varphi }{\dif t} &= - \frac{i E}{\hslash} \varphi &&\text{\RL{یا}}
\end{align}
اور
\begin{align}\label{مساوات_شروڈنگر_علیحدہ_دوم}
- \frac{\hslash^{2}}{2m} \frac{1}{\psi} \frac{\dif^{\,2} \psi}{\dif x^{2}} + V &= E\nonumber\\
- \frac{\hslash^{2}}{2m} \frac{\dif^{\,2} \psi}{\dif x^{2}} +V\psi &= E\psi&&\text{\RL{یا}}
\end{align}
علیحدگی متغیرات نے ایک جزوی تفرقی مساوات کو دو سادہ تفرقی مساوات (مساوات \حوالہ{مساوات_شروڈنگر_علیحدہ_اول} اور \حوالہ{مساوات_شروڈنگر_علیحدہ_اول}) میں علیحدہ کیا۔ ان میں سے پہلی  (مساوات \حوالہ{مساوات_شروڈنگر_علیحدہ_اول}) کو حل کرنا بہت آسان ہے۔ دونوں اطراف کو \عددی{\dif t} سے ضرب دیتے ہوئے تکمل لیں۔ یوں عمومی حل \عددی{C e^{-iEt/\hslash}} حاصل ہو گا۔ چونکہ ہم   حاصل ضرب\عددی{\psi\varphi } میں دلچسپی رکھتے ہیں لہذا ہم مستقل  \عددی{C} کو  \عددی{\psi}  میں ضم کر سکتے ہیں۔ یوں مساوات \حوالہ{مساوات_شروڈنگر_علیحدہ_اول} کا حل درج ذیل لکھا جا سکتا ہے۔ 
\begin{align}
\varphi(t) = e^{-iEt/\hslash}
\end{align}
دوسری  (مساوات \حوالہ{مساوات_شروڈنگر_علیحدہ_دوم})  کو \اصطلاح{غیر تابع وقت شروڈنگر مساوات}\فرہنگ{شروڈنگر!غیر تابع وقت}\حاشیہب{time-independent Schrodinger equation}\فرہنگ{Schrodinger!time-independent} کہتے ہیں۔ پوری طرح مخفی توانائی \عددی{V} جانے بغیر ہم آگے نہیں بڑھ سکتے ہیں۔

اس باب کے بقیہ حصے میں ہم  مختلف سادہ خفی توانائی کیلئے غیر تابع وقت شروڈنگر مساوات حل کریں گے۔ ایسا کرنے سے پہلے آپ پوچھ سکتے ہیں کہ علیحدگی متغیرات کی کیا خاص بات ہے؟ بہرحال تابع وقت شروڈنگر مساوات کے زیادہ تر حل \عددی{\psi (x) \varphi (t)}  کی صورت میں نہیں لکھے جا سکتے۔ میں اس کے تین جوابات دیتا ہوں۔ ان میں سے دو طبعی اور ایک ریاضیاتی ہو گا۔ 

\عددی{(1}\quad 
یہ \اصطلاح{ساکن حالات}\فرہنگ{ساکن!حالات} ہیں۔ اگرچہ تفاعل موج ازخود 
\begin{align}\label{مساوات_شروڈنگر_غیر_تابع_اور_تابع}
\Psi (x,t) = \psi (x) e^{-iEt/\hslash}
\end{align}
وقت \عددی{t} کا تابع ہے،  \ترچھا{کثافت احتمال}
\begin{align}
\left| \Psi (x,t) \right|^{2} = \Psi^{*}\Psi = \psi^{*}e^{+iEt/\hslash} \psi e^{-iEt/\hslash} = \left| \psi (x) \right|^{2}
\end{align}
وقت کا تابع نہیں ہے؛ تابعیت وقت کٹ جاتی ہے۔ یہی کچھ کسی بھی حرکی متغیر کی توقعاتی قیمت کے حساب میں ہو گا۔ مساوات \حوالہ{مساوات_تفاعل_موج_توقعاتی_قیمت_حصول} تخفیف کے بعد درج زیل صورت اختیار کرتی ہے۔ 
\begin{align}
\langle Q(x,p) \rangle = \int \psi^{*} Q \left( x, \frac{\hslash}{i} \frac{\dif}{\dif x} \right) \psi \dif x
\end{align}
ہر توقعاتی قیمت \ترچھا{وقت} میں \ترچھا{مستقل} ہو گی؛ یہاں تک کہ ہم  \عددی{\varphi (t) } کو رد کر کے  \عددی{\Psi} کی
 جگہ  \عددی{\psi} استعمال کر کے وہی نتائج حاصل کر سکتے ہیں۔ اگرچہ بعض اوقات  \عددی{\psi}  کو ہی تفاعل موج پکارا جاتا ہے، لیکن ایسا کرنا حقیقتاً غلط ہے جس سے مسئلے کھڑے ہو سکتے ہیں۔ یہ ضروری ہے کہ آپ یاد رکھیں کہ اصل تفاعل موج ہر صورت تابع وقت ہو گا۔ بالخصوص  \( \langle x \rangle \) مستقل  ہو گا لہٰذا (مساوات \حوالہ{مساوات_تفاعل_موج_تعریف_معیار_حرکت} کے تحت)   \( \langle p \rangle = 0 \)  ہو گا۔ ساکن حال میں کبھی بھی کچھ نہیں ہوتا ہے۔ 

\عددی{(2} \quad
یہ \ترچھا{غیر مبہم کل توانائی} کے حالات ہوں گے۔ کلاسیکی میکانیات میں کل توانائی (حرکی جمع خفی) کو \اصطلاح{ہیملٹنی}\فرہنگ{ہیملٹنی}\حاشیہب{Hamiltonian}\فرہنگ{Hamiltonian} کہتے ہیں جس کو \عددی{H} سے ظاہر کیا جاتا ہے۔ 
\begin{align}
H(x,p) = \frac{p^{2}}{2m} + V(x)
\end{align}
اس کا مطابقتی ہیملٹنی عامل، قواعدو ظوابط کے تحت  \عددی{p \rightarrow (\hslash/i)(\partial/\partial x)}  پر کر کے درج ذیل حاصل ہو گا۔ 
\begin{align}
\hat{H} = - \frac{\hslash^{2}}{2m} \frac{\partial^{2}}{\partial x^{2}} + V(x) 
\end{align}
یوں غیر تابع وقت شروڈنگر مساوات \حوالہ{مساوات_شروڈنگر_علیحدہ_دوم} درج ذیل روپ اختیار کریگی 
\begin{align}
\hat{H} \psi = E\psi
\end{align}

جس کے کل توانائی کی توقعاتی قیمت درج ذیل ہو گی۔ 
\begin{align}
\langle H \rangle = \int \psi^{*} \hat{H}\psi \dif x = E \int \left| \psi \right|^{2} \dif x = E \int \left| \Psi \right|^{2} \dif x = E
\end{align}
آپ دیکھ سکتے ہیں کہ \عددی{\psi} کی معمول زنی \عددی{\psi} کی معمول زنی کے مترادف ہے۔ مزید درج ذیل 
\begin{align*}
\hat{H}^{2} \psi = \hat{H} (\hat{H}\psi ) = \hat{H} ( E\psi ) = E (\hat{H} \psi ) = E^{2} \psi 
\end{align*}
کی بنا درج ذیل ہوگا۔
\begin{align*}
\langle H^{2} \rangle = \int \psi^{*} \hat{H}^{2} \psi \dif x = E^{2} \int \left| \psi \right|^{2} \dif x = E^{2}
\end{align*}
یوں\عددی{H} کی تغیریت درج ذیل ہو گی۔ 
\begin{align}
\sigma^{2}_{H} = \langle H^{2} \rangle - \langle H \rangle^{2} =E^{2} - E^{2} = 0
\end{align}
یاد رہے کہ  \عددی{\sigma = 0}  کی صورت میں تمام ارکان کی قیمت ایک دوسری جیسی ہو گی (تقسیم کا پھیلاؤ صفر ہو گا)۔ نتیجتاً قابل علیحدگی حل کی ایک خاصیت یہ ہو ہے کہ کل توانائی کی ہر پیمائش یقیناً ایک ہی قیمت \عددی{E} دے گی۔ (اسی کی بنا  علیحدگی مستقل کو \عددی{E} سے ظاہر کیا گیا۔) 

\عددی{(3}\quad
عمومی حل قابل علیحدگی حلوں کا \اصطلاح{خطی جوڑ}\فرہنگ{خطی جوڑ}\حاشیہب{linear combination}\فرہنگ{linear!combination} ہو گا۔ جیسا ہم جلد دیکھیں گے،  غیر تابع وقت شروڈنگر مساوات  (مساوات \حوالہ{مساوات_شروڈنگر_علیحدہ_دوم})   لامتناعی تعداد کے حل \عددی{(\psi_{1}(x),\, \psi_{2}(x),\, \psi_{3}(x), \cdots)}  دے گا جہاں ہر ایک حل کے ساتھ ایک علحدگی مستقل \عددی{(E_1,E_2,E_3,\cdots)} منسلک ہو گا لہٰذا  ہر \اصطلاح{اجازتی توانائی}\فرہنگ{توانائی!اجازتی}\حاشیہب{allowed energy}\فرہنگ{energy!allowed} کا ایک منفرد تفاعل موج پایا جائے گا۔ 
\begin{align*}
\Psi_{1} (x,t) = \psi_{1}(x)e^{-iE_{1}t/\hslash} , \quad \Psi_{2} (x,t) = \psi_{2}(x)e^{-iE_{2}t/\hslash}, \, \cdots 
\end{align*}
اب (جیسا کہ آپ خود تصدیق کر سکتے ہیں)  تابع وقت شروڈنگر مساوات  (مساوات \حوالہ{مساوات_شروڈنگر_تابع_وقت}) کی ایک خاصیت یہ ہے کہ اس کے حلوں کا ہر خطی جوڑ ازخود ایک حل ہو گا۔ ایک بار قابل علیحدگی حل تلاش کرنے کے بعد ہم زیادہ عمومی حل درج ذیل روپ میں تیار کر سکتے ہیں۔
\begin{align}\label{مساوات_شروڈنگر_خطی_جوڑ_عمومی_حل}
\Psi (x,t) = \sum_{n=1}^{\infty} c_{n} \psi_{n}(x)e^{-iE_{n}t/\hslash}
\end{align}
حقیقتاً  تابع وقت شروڈنگر مساوات کا ہر حل درج بالا روپ میں لکھا جا سکتا ہے۔ ایسا کرنے کی خاطر ہمیں وہ مخصوص مستقل 
\عددی{(c_{1},\, c_{2}, \, \cdots )}  تلاش کرنے ہوں گے جن کو استعمال کرتے ہوئے درج بالا حل (مساوات \حوالہ{مساوات_شروڈنگر_خطی_جوڑ_عمومی_حل}) ابتدائی شرائط مطمئن کرتا ہو۔ آپ آنے والے حصوں میں دیکھیں گے کہ ہم کس طرح یہ سب کچھ کر پائیں گے۔ باب \حوالہء{3} میں  ہم اس کو زیادہ مضبوط بنیادوں پر کھڑا کر پائیں گے۔ بنیادی نقطہ یہ ہے کہ ایک بار  غیر تابع وقت شروڈنگر مساوات حل کرنے کے بعد آپ کے مسائل ختم ہو جاتے ہیں۔ یہاں سے تابع وقت شروڈنگر مساوات کا عمومی حل حاصل کرنا آسان کام ہے۔ 

گذشتہ چار صفحات میں ہم بہت کچھ کہا جا چکا ہے۔ میں ان کو مختصراً اور مختلف نقطہ نظر سے دوبارہ پیش کرتا ہوں۔ زیر غور عمومی مسئلہ کا غیر تابع وقت خفی توانائی  \عددی{V(x)}  اور ابتدائی تفاعل موج  \عددی{\Psi (x,0)} دیے  گئے ہوں گے۔ آپ کو مستقبل کے تمام \عددی{t}  کیلئے  \عددی{ \Psi (x,t)} تلاش کرنا ہوگا۔ ایسا کرنے کی خاطر آپ تابع وقت شروڈنگر مساوات  (مساوات \حوالہ{مساوات_شروڈنگر_تابع_وقت}) حل کریں گے۔ پہلی قدم میں  آپ غیر تابع وقت شروڈنگر مساوات  (مساوات \حوالہ{مساوات_شروڈنگر_علیحدہ_دوم})  حل کر کے  لا متناہی تعداد کے حلوں  کا سلسلہ   \عددی{(\psi_{1}(x),\, \psi_{2}(x),\, \psi_{3}(x), \cdots)} حاصل کریں گے جہاں ہر ایک
 کی منفرد توانائی  \عددی{(E_{1}, \, E_{2}, \, E_{3}, \, \cdots)} ہو گی۔ ٹھیک ٹھیک \عددی{\Psi (x,0)} پر بیٹھنے کی خاطر آپ ان حلوں کا  خطی جوڑ لیں گے۔
\begin{align}
\Psi (x,0) = \sum_{n=1}^{\infty} c_{n} \psi_{n}(x)
\end{align}
یہاں کمال کی  بات یہ ہے کہ کسی بھی ابتدائی حال کے لئے آپ ہر صورت  مستقل \عددی{c_{1}, \,c_{2}, \,c_{3}, \cdots} دریافت کر پائیں گے۔ تفاعل موج \عددی{\Psi(x,t)}  تیار کرنے کی خاطر آپ ہر جزو کے ساتھ مختص تابعیت وقت \عددی{e^{-iE_nt/\hslash}} چسپاں کریں گے۔ 
\begin{align}\label{مساوات_شروڈنگر_عمومی_حل_مجموعہ}
\Psi (x,t) = \sum_{n=1}^{\infty} c_{n} \psi_{n}(x)e^{-iE_{n}t/\hslash} = \sum_{n=0}^{\infty} c_{n} \Psi_{n} (x,t)
\end{align}
چونکہ قابل علیحدگی حل
\begin{align}\label{مساوات_شروڈنگر_تمام_عمومی_حل}
\Psi_{n} (x,t) = \psi_{n}(x) e^{-iE_{n}t/\hslash}
\end{align}
کے تمام احتمال اور توقعاتی قیمتیں غیر تابع وقت ہوں گی لہٰذا یہ  از خود ساکن حالات ہوں گے، تا ہم عمومی حل  (مساوات \حوالہ{مساوات_شروڈنگر_عمومی_حل_مجموعہ}) یہ خاصیت نہیں رکھتا ہے؛  انفرادی ساکن حالات کی توانائیاں ایک دوسرے سے مختلف ہونے کی بنا \عددی{\left| \Psi \right|^{2}} کا حساب کرتے ہوئے  قوت نمائی ایک دوسرے کو حذف نہیں کرتی ہیں۔ 


\ابتدا{مثال}
فرض کریں  ایک ذرہ ابتدائی طور پر دو ساکن حالات کا خطی جوڑ ہو: 
\begin{align*}
\Psi (x,0) = c_{1} \psi_{1}(x) + c_{2} \psi_{2}(x) 
\end{align*}
(چیزوں کو سادہ رکھنے کی خاطر میں فرض کرتا ہوں کے مستقل  \عددی{c_{n}} اور حالات  \عددی{\psi_{n} (x)} حقیقی ہیں۔) مستقبل وقت t کیلئے تفاعل موج  \عددی{\Psi (x,t)} کیا ہو گا ؟ کثافت احتمال تلاش کریں اور ذرے کی حرکت  بیان کریں۔ 

حل:\quad
اس کا پہلا حصہ آسان ہے
\begin{align*}
\Psi (x,t) = c_{1} \psi_{1}(x)e^{-iE_{1}t/\hslash} + c_{2} \psi_{2}(x)e^{-iE_{2}t/\hslash}
\end{align*}
جہاں  \عددی{E_{1}} اور \عددی{E_{2}}  بالترتیب تفاعل  \عددی{\psi_{1}} اور  \عددی{\psi_{2}} کی مطابقتی توانائیاں ہیں۔ یوں درج ذیل ہو گا۔ 
\begin{align*}
\left| \Psi (x,t) \right|^{2} &= \left( c_{1} \psi_{1} e^{iE_{1}t/\hslash} + c_{2} \psi_{2} e^{iE_{2}t/\hslash} \right) \left( c_{1} \psi_{1} e^{-iE_{1}t/\hslash} + c_{2} \psi_{2} e^{-iE_{2}t/\hslash} \right) \\
&= c_{1}^{2} \psi_{1}^{2} + c_{2}^{2} \psi_{2}^{2} + 2c_{1}c_{2}\psi_{1}\psi_{2} \cos [ ( E_{2} - E_{1})t/\hslash]
\end{align*}
(میں نے نتیجہ کی سادہ صورت حاصل کرنے کی خاطر کلیہ یولر \عددی{e^{i\theta}=\cos\theta+i\sin\theta} استعمال کیا۔)  ظاہری طور پر  کثافت احتمال زاویائی تعدد \عددی{(\tfrac{E_2-E_1}{\hslash})} سے سائن نما ارتعاش کرتا ہے لہٰذا یہ ہرگز ساکن حال نہیں ہو گا۔ لیکن دھیان رہے کہ (ایک دوسرے سے مختلف) تونائیوں کے تفاعلات کے خطی جوڑ نے حرکت پیدا کیا۔ 
\انتہا{مثال}
%===========================
\ابتدا{سوال}
درج ذیل تین مسائل کا ثبوت پیش کریں۔
\begin{enumerate}[a.]
\item
قابل علیحدگی حلوں کے لئے علیحدگی مستقل \عددی{E} لازماً \ترچھا{حقیقی} ہو گا۔ \ترچھا{اشارہ:} مساوات \حوالہ{مساوات_شروڈنگر_غیر_تابع_اور_تابع} میں \عددی{E} کو \عددی{E_0+i\Gamma} لکھ کر (جہاں \عددی{E} اور \عددی{\Gamma} حقیقی ہیں)،  دکھائیں کہ تمام \عددی{t} کے لئے مساوات \حوالہء{1.20} اس صورت کارآمد ہو گا جب \عددی{\Gamma} صفر ہو۔ 
\item
غیر تابع وقت تفاعل موج \عددی{\psi(x)}  ہر موقع پر حقیقی لیا جا سکتا ہے (جبکہ تفاعل موج \عددی{\Psi(x,t)} لازماً مخلوط ہوتا ہے)۔ اس کا  ہرگز یہ مطلب نہیں ہے کہ غیر تابع شروڈنگر مساوات کا ہر حل حقیقی ہو گا؛ بلکہ غیر حقیقی حل پائے جانے کی صورت میں اس حل کو ہمیشہ،  ساکن حالات کا (اتنی ہی توانائی کا) خطی جوڑ لکھنا ممکن ہو گا۔  یوں بہتر ہو گا کہ آپ صرف حقیقی \عددی{\psi} ہی استعمال کریں۔ \ترچھا{اشارہ:} اگر کسی مخصوص \عددی{E} کے لئے  \عددی{\psi{(x)}} مساوات \حوالہ{مساوات_شروڈنگر_علیحدہ_دوم} کو مطمئن کرتا ہو تب اس کا مخلوط خطی جوڑ  بھی اس مساوات کو مطمئن کرے گا اور یوں ان کے خطی جوڑ \عددی{(\psi+\psi^*)} اور \عددی{i(\psi-\psi^*)} بھی اس مساوات کو مطمئن کریں گے۔
\item
اگر \عددی{V(x)} \اصطلاح{جفت تفاعل}\فرہنگ{جفت!تفاعل} ہو یعنی \عددی{V(-x)=V(x)} تب \عددی{\psi(x)} کو ہمیشہ جفت یا طاق لیا سکتے ہو۔ \ترچھا{اشارہ:} اگر کسی مخصوص \عددی{E} کے لئے  \عددی{\psi{(x)}} مساوات \حوالہ{مساوات_شروڈنگر_علیحدہ_دوم} کو مطمئن کرتا ہو تب \عددی{\psi(-x)}  بھی اس مساوات کو مطمئن کرے گا اور یوں ان کے جفت اور طاق خطی جوڑ \عددی{\psi(x)\pm\psi(-x)}  بھی اس مساوات کو مطمئن کریں گے۔ 
\end{enumerate}
\انتہا{سوال}
%====================
\ابتدا{سوال}\شناخت{سوال_شروڈنگر_کم_سے_کم_توانائی}
دکھائیں کہ غیر تابع وقت شروڈنگر مساوات کے ہر اس حل کے لئے، جس کو معمول پر لایا جا سکتا ہو، \عددی{E} کی قیمت لازماً \عددی{V(x)} کی کم سے کم قیمت سے زیادہ ہو گی۔ اس کا کلاسیکی مماثل کیا ہو گا؟ \ترچھا{اشارہ:}  مساوات \حوالہ{مساوات_شروڈنگر_علیحدہ_دوم} کو درج ذیل روپ میں لکھ کر
\begin{align*}
\frac{\dif^{\,2} \psi}{\dif x^2}=\frac{2m}{\hslash^2}[V(x)-E]\psi
\end{align*}
دکھائیں کہ \عددی{E<V_{\text{کمتر}}} کی صورت میں \عددی{\psi} اور اس کے دو گنّا تفرق کی علامتیں  لازماً ایک دوسری جیسی ہوں گی؛ اب دلیل پیش کریں  کہ ایسا تفاعل معمول پر لانے کے قابل نہیں ہو گا۔
\انتہا{سوال}
%====================
%%%%%%%%%%%%%%%%%%%%%
%%%%%%%%%%%%%%%
\حصہ {لامتناہی چکور کنواں}
  درج ذیل فرض کریں (شکل  \حوالہء{2.1})۔
\begin{align}
V(x)=
\begin{cases}
0& 0\le x\le a\\
\infty &\text{\RL{دیگر صورت}}
\end{cases}
\end{align}
اس مخفی توانائی میں ایک ذرہ مکمل آزاد ہو گا، ماسوائے  دونوں سروں یعنی \عددی{x=0}\عددی{x=a} پر، جہاں ایک لامتناہی قوت اس کو فرار  ہونے سے روکتی ہے۔ اس کا کلاسیکی نمونہ ایک کنواں  میں ایک لامتناہی لچکدار  گیند ہو سکتا ہے جو ہمیشہ کے لئے دیواروں سے ٹکرا کر دائیں سے بائیں اور بائیں سے دائیں  حرکت کرتا رہتا ہو۔ (اگرچہ یہ ایک فرضی مخفی توانائی ہے، آپ اس کو اہمیت دیں ۔ اگرچہ یہ بہت سادہ نظر آتا ہے البتہ اس کی سادگی کی بنا ہی یہ بہت ساری معلومات فراہم کرنے کے قابل ہے۔ ہم اس سے بار بار رجوع کریں گے۔)

کنواں سے باہر \عددی{\psi (x)=0} ہو گا  (لہٰذا یہاں ذرہ پایا جانے  کا احتمال صفر ہو گا)۔ کنواں کے اندر، جہاں \عددی{V=0} ہے،  غیر تابع وقت  شروڈنگر مساوات  (مساوات \حوالہ{مساوات_شروڈنگر_علیحدہ_دوم}) درج ذیل روپ اختیار کرتی ہے۔
\begin{align}
-\frac{\hslash^{2}}{2m}\frac{\dif^{\,2}\psi}{\dif x^{2}}&=E\psi
\end{align} 
یا 
\begin{align}\label{مساوات_شروڈنگر_کلاسیکی_ہارمونی_مرتعش}
\frac{\dif^{\,2}\psi}{\dif x^{2}}&=-k^{2}\psi, && k\equiv \frac{\sqrt{2mF}}{\hslash}
\end{align}
(اس کو یوں لکھتے ہوئے  میں  خاموشی سے فرض کرتا ہوں کہ \عددی{E\ge 0} ہو گا۔ ہم  سوال \حوالہ{سوال_شروڈنگر_کم_سے_کم_توانائی} سے جانتے ہیں کہ \عددی{E< 0 } سے بات نہیں بنے گی۔)   مساوات \حوالہ{مساوات_شروڈنگر_کلاسیکی_ہارمونی_مرتعش}  کلاسیکی \اصطلاح{سادہ ہارمونی مرتعش}\فرہنگ{ہارمونی!مرتعش}\فرہنگ{مرتعش!ہارمونی}\حاشیہب{simple harmonic oscillator}\فرہنگ{harmonic!oscillator} کی مساوات ہے جس کا عمومی حل درج ذیل  ہو گا
\begin{align}
\psi(x)=A\sin kx+B\cos kx
\end{align}
 جہاں \عددی{A }  اور  \عددی{B} اختیاری مستقل ہیں۔ ان مستقلات کو مسئلہ  کے \اصطلاح{سرحدی شرائط}\فرہنگ{سرحدی شرائط}\حاشیہب{boundary conditions}\فرہنگ{boundary conditions} تعین کرتے ہیں۔ \عددی{\psi (x)} کے موزوں سرحدی شرائط کیا ہونگے؟ عموماً \عددی{\psi} اور \عددی{\tfrac{\dif \psi}{\dif x}} \ترچھا{دونوں استمراری} ہونگے، لیکن جہاں مخفیہ لامتناہی کو پہنچتا ہو وہاں  صرف اول الذکر کا اطلاق ہو گا۔ (میں حصہ  \حوالہء{2.5}میں ان سرحدی شرائط کو ثابت کروں گا اور \عددی{V=\infty}   کی صورت حال کو بھی دیکھوں گا ۔ فی الحال مجھ پر یقین کرتے ہوئے میری کہی ہوئی بات مان لیں۔) 

تفاعل \عددی{\psi(x)} کے استمرار کی بنا درج ذیل ہو گا
\begin{align}
\psi(0)=\psi(a)=0
\end{align} 
تا کہ کنواں کے باہر اور کنواں کے اندر حل ایک دوسرے کے ساتھ جڑ سکیں۔ یہ ہمیں  \عددی{A}اور  \عددی{B}کے بارے میں کیا معلومات فراہم کرتی ہے؟ چونکہ
\begin{align*}
\psi(0)=A\sin 0+B\cos 0=B
\end{align*}
ہے لہٰذا \عددی{B=0}  اور درج ذیل ہو گا۔
\begin{align}
\psi(x)=A\sin kx
\end{align}
یوں  \عددی{\psi(a)=A\sin ka} کی بنا یا \عددی{A=0}   ہو گا (ایسی صورت میں ہمیں غیر اہم حل \عددی{\psi(x)=0}  ملتا ہے جو معمول پر لانے کے قابل نہیں ہے)  یا \عددی{\sin ka =0}  ہو گا جس کے تحت درج ذیل ہو گا۔
\begin{align}
ka&=0,\pm\pi,\pm2\pi,\pm3\pi,\cdots
\end{align} 
اب \عددی{ k =0}  (بھی \عددی{ \psi(x)=0}  دیتا ہے جس) میں ہم دلچسپی نہیں رکھتے اور  \عددی{ \sin(-\theta)=-\sin(\theta)}  کی بنا \عددی{k} کی منفی قیمتیں کوئی نیا حل نہیں دیتی ہیں  لہٰذا ہم منفی کی علامت کو  \عددی{A} میں ضم کر سکتے ہیں۔ یوں منفرد حل درج ذیل ہوں گے۔ 
\begin{align}
k_{n}=\frac{n\pi}{a},&& n=1,2,3,\cdots
\end{align}

دلچسپ بات یہ ہے کہ \عددی{x=a} پر سرحدی شرط مستقل   \عددی{A} تعین نہیں کرتا ہے بلکہ اس کی بجائے مستقل \عددی{k}تعین کرتے ہوئے    \عددی{E} کی اجازتی قیمتیں تعین کرتا ہے:
\begin{align}
E_{n}=\frac{\hslash^2 k^{2}_{n}}{2m}=\frac{n^{2}\pi^{2}\hslash^{2}}{2ma^{2}}
\end{align} 
کلاسیکی صورت کے برعکس لامتناہی چکور کنواں میں کوانٹم ذرہ ہر ایک توانائی کا حامل نہیں ہو سکتا ہے بلکہ اس کی توانائی کی قیمت کو درج بالا مخصوص \اصطلاح{اجازتی}\فرہنگ{اجازتی!توانائیاں}\حاشیہب{allowed}\فرہنگ{allowed!energies} قیمتوں  میں سے ہونا ہو گا۔ مستقل  \عددی{A} کی قیمت حاصل کرنے کے لئے\عددی{ \psi}  کو معمول پر لانا ہو گا: 
\begin{align*}
\int_{0}^{a}\abs{A}^{2}\sin^{2}(kx)\dif{x}=\abs{A}^{2}\frac{a}{2}=1,\quad \implies\quad \abs{A}^{2}=\frac{2}{a}
\end{align*} 
یہ  \عددی{A} کی صرف \ترچھا{مقدار} دیتی ہے ہے، تاہم مثبت حقیقی جذر \عددی{A=\sqrt{2/a}}  منتخب کرنا بہتر ہو گا (کیونکہ  \عددی{A} کا زاویہ کوئی طبعی معنی نہیں رکھتا ہے)۔  اس طرح کنواں کے اندر شروڈنگر مساوات کے حل درج ذیل ہوں گے۔
\begin{align}
\psi_{n}(x)=\sqrt{\frac{2}{a}}\sin\big(\frac{n\pi}{a}x\big)
\end{align}
میرے قول کو پورا کرتے ہوئے، (ہر مثبت عدد صحیح \عددی{n} کے عوض  ایک حل دے کر)  غیر تابع وقت شروڈنگر مساوات نے حلوں کا ایک لامتناہی سلسلہ دیا ہے۔  ان میں سے اولین چند کو  شکل \حوالہء{2.2}میں ترسیم کیا گیا ہے جو لمبائی \عددی{a} کے دھاگے پر ساکن امواج کی طرح نظر آتے ہیں۔ تفاعل \عددی{\psi_{1}} جو \اصطلاح{زمینی حال}\فرہنگ{حال!زمینی}\حاشیہب{ground state}\فرہنگ{state!ground} کہلاتا ہے     کی توانائی کم سے کم ہے۔باقی حالات جن کی توانائیاں \عددی{n^{2}}    کے براہ راست بڑھتی ہیں  \اصطلاح{ہیجان حالات}\فرہنگ{حال!ہیجان}\حاشیہب{excited states}\فرہنگ{state!excited} کہلاتے ہیں ۔ تفاعلات \عددی{\psi_n(x)}   چند اہم اور دلچسپ خواص رکھتے ہیں:
\begin{enumerate}
\item
 کنواں کے وسط کے لحاض سے یہ تفاعلات باری باری جفت اور طاق ہیں۔ \عددی{\psi_{1}}      جفت ہے، \عددی{\psi_{2}}      طاق ہے، \عددی{\psi_{3}}  جفت ہے، وغیرہ وغیرہ۔
\item
توانائی بڑھاتے ہوئے ہر اگلے حال کے \اصطلاح{عقدوں}\فرہنگ{عقدہ}\حاشیہب{nodes}\فرہنگ{node} (عبور صفر) کی تعداد میں ایک \عددی{(1)} کا اضافہ ہو گا۔  (چونکہ آخری نقاط کے صفر کو نہیں گنا جاتا ہے لہٰذا) \عددی{\psi_{1}} میں کوئی عقدہ نہیں پایا جاتا ہے،  \عددی{\psi_{2}}  میں ایک پایا جاتا ہے، \عددی{\psi_{3}} میں دو پائے جاتے ہیں، وغیرہ وغیرہ۔
\item
 یہ تمام درج ذیل نقطہ نظر سے باہمی \اصطلاح{عمودی}\فرہنگ{عمودی}\حاشیہب{orthogonal}\فرہنگ{orthogonal} ہیں جہاں  
\عددی{m\ne n} ہے۔
\begin{align}
\int\psi_{m}(x)^*\psi_{n}(x)\dif{x}=0
\end{align}
\ترچھا{ثبوت:}\quad
\begin{align*}
\int&\psi_{m}(x)^*\psi_{n}(x)\dif{x}=\frac{2}{a}\int_{0}^{a}\sin\big(\frac{m\pi}{a}x\big)\sin\big(\frac{n\pi}{a}x\big)\dif{x}\\
&=\frac{1}{a}\int_{0}^{a}\big[\cos\big(\frac{m-n}{a}\pi x\big)-\cos\big(\frac{m+n}{a}\pi x\big)\big]\dif{x}\\
&=\big\{\left.\frac{1}{(m-n)\pi}\sin\big(\frac{m-n}{a}\pi x\big)-\frac{1}{(m+n)\pi}\sin\big(\frac{m+n}{a}\pi x\big)\big\}\right\vert_{0}^{a}\\
&=\frac{1}{\pi}\big\{\frac{\sin[(m-n)\pi]}{(m-n)}-\frac{\sin[(m+n)\pi]}{(m+n)}\big\}=0
\end{align*} 
دھیان رہے کہ  \عددی{m= n}   کی صورت میں درج بالا دلیل درست نہیں ہو گا؛ (کیا آپ بتا سکتے ہیں کہ ایسی صورت میں دلیل کیوں ناقابل قبول ہو گا۔) ایسی صورت میں معمول پر لانے کا عمل ہمیں بتاتا ہے کہ تکمل کی قیمت  \عددی{1} ہے۔ درحقیقت ہم عمودیت اور معمول زنی کو ایک فقرے میں سمیٹ سکتے ہیں:\حاشیہد{یہاں تمام \عددی{\psi} حقیقی ہیں لہٰذا \عددی{\psi_m} پر \عددی{^*} ڈالنے کی ضرورت نہیں ہے، لیکن مستقل کی استعمال کے نقطہ نظر سے ایسا کرنا ایک اچھی عادت ہے۔}
\begin{align}
\int\psi_{m}(x)^*\psi_{n}(x)\dif{x}=\delta_{mn}
\end{align}
جہاں \عددی{\delta_{mn}}    \اصطلاح{کرونیکر ڈیلٹا}\فرہنگ{ڈیلٹا!کرونیکر}\حاشیہب{Kronecker delta}\فرہنگ{delta!Kronecker} کہلاتا ہے ہیں جس کی تعریف درج ذیل ہے۔ 
\begin{align}
\delta_{mn}=
\begin{cases}
0&  m\neq n\\
1 &  m=n
\end{cases}
\end{align} 
ہم کہتے ہیں کہ مذکورہ بالا (تمام) \عددی{\psi}  \اصطلاح{معیاری عمودی}\فرہنگ{معیار عمودی}\فرہنگ{عمودی!معیاری}\حاشیہب{orthonormal}\فرہنگ{orthonormal} ہیں۔ 
\item
 یہ \اصطلاح{مکمل}\فرہنگ{مکمل}\حاشیہب{complete}\فرہنگ{complete} ہیں، جس سے مراد ہے کہ کسی  بھی دوسرے تفاعل \عددی{f(x)} کو ان کا خطی جوڑ لکھا جا سکتا ہے:
\begin{align}\label{مساوات_شروڈنگر_کوئی_تفاعل}
f(x)=\sum_{n=1}^{\infty}c_{n}\psi_{n}(x)=\sqrt{\frac{2}{a}}\sum_{n=1}^{\infty}c_{n}\sin\big(\frac{n\pi}{a}x\big)
\end{align}
 میں تفاعلات \عددی{\sin\tfrac{n\pi x}{a}}   کی مکملیت کو یہاں ثابت نہیں کروں گا، البتہ  اعلٰی علم الاحصاء کے ساتھ واقفیت کی صورت میں آپ مساوات \حوالہ{مساوات_شروڈنگر_کوئی_تفاعل} کو \عددی{f(x)} کا \اصطلاح{فوریئر تسلسل}\فرہنگ{تسلسل!فوریئر}\حاشیہب{Fourier series}\فرہنگ{series!Fourier} پہچان پائیں گے۔ یہ حقیقت، کہ ہر تفاعل کو فوریئر تسلسل کی صورت میں پھیلا کر لکھا جا سکتا ہے، بعض اوقات \اصطلاح{مسئلہ ڈرشلے}\فرہنگ{مسئلہ!ڈرشلے}\حاشیہب{Dirichlet's theorem}\فرہنگ{theorem!Dirichlet's} کہلاتا ہے۔\حاشیہد{تفاعل \عددی{f(x)} میں متناہی تعداد کی عدم استمرار (چھلانگ) پائے جا سکتی ہیں۔}

کسی بھی دیے گئے تفاعل \عددی{f(x)} کے لئے عددی سروں \عددی{c_{n}}  کو \عددی{\{ \psi_n\}}  کی معیاری عمودیت کی مدد سے حاصل کیا جاتا ہے۔ مساوات \حوالہ{مساوات_شروڈنگر_کوئی_تفاعل} کے دونوں اطراف کو \عددی{\psi_{m}(x)} سے ضرب دے کر تکمل لیں:
 \begin{align}
\int \psi_{m}(x)^*f(x)\dif{x}=\sum_{n=1}^{\infty}c_{n}\int\psi_{m}(x)^*\psi_{n}(x)\dif{x}=\sum_{n=1}^{\infty}c_{n}\delta_{mn}=c_{m}
\end{align}
(آپ دیکھ سکتے ہیں کہ کرونیکر ڈیلٹا مجموعے میں تمام اجزاء کو ختم کر دیتا ہے ماسوائے اس جزو کو جس کے لئے  \عددی{n=m} ہو۔) یوں تفاعل   \عددی{f(x)} کے پھیلاو  کے \عددی{n} ویں جزو کا عددی سر درج ذیل ہو گا۔\حاشیہد{آپ یہاں نقلی متغیر کو \عددی{m} یا \عددی{n} یا کوئی تیسرا حرف لے سکتے ہیں (بس اتنا خیال رکھیں کہ مساوات کی دونوں اطراف ایک ہی حرف استعمال کریں)، اور ہاں یاد رہے کہ یہ حرف "کسی مثبت عدد صحیح" کو ظاہر کرتا ہے۔}
\begin{align}
c_{n}=\int\psi_{n}(x)^*f(x)\dif{x}
\end{align}
\end{enumerate}

درج بالا   چار خواص انتہائی طاقتور ہیں جو صرف لامتناہی چکور کنواں کے لیے مخصوص نہیں ہیں ۔ پہلا خواص ہر اس صورت میں کارآمد ہو گا جب مخفیہ  تشاکلی ہو؛  دوسرا، مخفیہ  کی شکل و صورت سے قطع نظر، ایک عالمگیر خواص ہے۔ عمودیت بھی کافی  عمومی خاصیت ہے، جس کا ثبوت میں باب  \حوالہء{ 3} میں پیش کروں گا۔  ان تمام مخفیہ کے لئے جن کو آپ کا (ممکنہ) سامنا ہو سکتا ہے کے لئے مکملیت  کارآمد ہو گی، لیکن اس کا ثبوت کافی لمبا اور  پیچیدہ ہے؛ جس کی بنا عموماً ماہر طبیعیات یہ ثبوت دیکھے بغیر، اس کو مان لیتے ہیں۔

 لا متناہی چکور کنواں کے ساکن حال (مساوات  \حوالہ{مساوات_شروڈنگر_تمام_عمومی_حل}) درج ذیل ہوں گے۔ 
 \begin{align}
\Psi_{n}(x,t)=\sqrt{\frac{2}{a}}\sin\big(\frac{n\pi}{a}x\big)e^{-i(n^{2}\pi^{2}\hslash/2ma^{2})t}
\end{align}
 میں نے دعوی  کیا (مساوات  \حوالہ{مساوات_شروڈنگر_عمومی_حل_مجموعہ}) کہ تابع وقت شروڈنگر مساوات کا عمومی ترین حل، ساکن حالات کا خطی جوڑ ہو گا۔
\begin{align}\label{مساوات_شروڈنگر_ساکن_حالات_کا_مجموعہ}
\Psi(x,t)=\sum_{n=1}^{\infty}c_n\sqrt{\frac{2}{a}}\sin\big(\frac{n\pi}{a}x\big)e^{-i(n^{2}\pi^{2}\hslash/2ma^{2})t}
\end{align}  
 (اگر آپ کو اس حل پر شق ہو تو اس کی تصدیق ضرور کیجیے گا۔) مجھے صرف اتنا دکھانا ہو گا کہ کسی بھی ابتدائی تفاعل موج \عددی{\psi(x,0)}   پر اس حل کو   بٹھانے کے لیے  موزوں  عددی سر \عددی{c_{n}} درکار ہوں گے:
\begin{align*}
\Psi(x,0)=\sum_{n=1}^{\infty}c_{n}\psi_{n}(x)
\end{align*}
تفاعلات \عددی{\psi }  کی مکملیت (جس کی تصدیق یہاں مسئلہ ڈرشلے کرتی ہے) اس کی ضمانت دیتی ہے کہ میں ہر \عددی{\psi(x,0)}   کو ہر صورت یوں بیان کر سکتا ہوں، اور ان کی معیاری عمودیت کی بنا   \عددی{c_{n}} کو فوریئر تسلسل سے حاصل کیا جا سکتا ہے: 
\begin{align}\label{مساوات_شروڈنگر_عددی_سروں}
c_{n}=\sqrt{\frac{2}{a}}\int_{0}^{a}\sin\big(\frac{n\pi}{a}x\big)\Psi(x,0)\dif{x}
\end{align}
آپ نے دیکھا: دی گئی ابتدائی تفاعل موج \عددی{\Psi(x,0)} کے لئے ہم سب سے پہلے پھیلاو کے عددی سروں  \عددی{c_{n}} کو مساوات \حوالہ{مساوات_شروڈنگر_عددی_سروں} سے حاصل کرتے ہیں۔   اس کے بعد انہیں
 مساوات  \حوالہ{مساوات_شروڈنگر_ساکن_حالات_کا_مجموعہ} میں پر کر \عددی{\Psi(x,t)} حاصل کرتے ہیں ۔ تفاعل موج جانتے ہوئے  دلچسپی کی کسی بھی حرکی مقدار   کا حساب، باب  \حوالہ{باب_تفاعل_موج} میں مستعمل تراکیب استعمال کرتے ہوئے، کیا جا سکتا ہے ۔ یہی ترکیب کسی بھی مخفیہ کے لیے کارآمد ہو گا؛  صرف  \عددی{\psi}  کی  قیمتیں اور اجازتی توانائیاں یہاں سے مختلف  ہوں گی۔
%=====================
\ابتدا{مثال}
لا متناہی چکور کنواں میں ایک ذرے کا ابتدائی تفاعل موج درج ذیل ہے جہاں \عددی{A} ایک مستقل ہے ( شکل  \حوالہء{شکل 2.3})۔
\begin{align*}
\Psi(x,0)=Ax(a-x),&& (0\le x\le  a)
\end{align*}
 کنواں سے باہر \عددی{\psi=0}  ہے ۔ \عددی{\Psi(x,t)}  تلاش کریں۔ 

حل:\quad
ہم پہلے \عددی{\Psi(x,0)}  کو معمول پر لاتے ہوئے 
\begin{align*}
1=\int_{0}^{a}\abs{\Psi(x,0)}^{2}\dif{x}=\abs{A}^{2}\int_{0}^{a}x^{2}(a-x)^{2}\dif{x}=\abs{A}^{2}\frac{a^{5}}{30}
\end{align*}
\عددی{A} تعین کرتے ہیں: 
\begin{align*}
A=\sqrt{\frac{30}{a^{5}}}
\end{align*}
مساوات  \حوالہ{مساوات_شروڈنگر_عددی_سروں} کے تحت \عددی{n} واں عددی سر درج ذیل ہو گا۔
\begin{align*}
c_{n}&=\sqrt{\frac{2}{a}}\int_{0}^{a}\sin\big(\frac{n\pi}{a}x\big)\sqrt{\frac{30}{a^{5}}}x(a-x)\dif{x}\\
&=\frac{2\sqrt{15}}{a^{3}}\Big[a\int_{0}^{a}x\sin\big(\frac{n\pi}{a}x\big)\dif{x}-\int_{0}^{a}x^{2}\sin\big(\frac{n\pi}{a}x\big)\dif x\Big]\\
&=\frac{2\sqrt{15}}{a^{3}}\Big\{a \left .\Big[\big(\frac{a}{n\pi}\big)^{2}\sin\big(\frac{n\pi}{a}x\big)-\frac{ax}{n\pi}\cos\big(\frac{n\pi}{a}x\big)\Big]\right\vert_{0}^{a}\.\\
&\quad-\left .\Big[2\big(\frac{a}{n\pi}\big)^{2}x\sin\big(\frac{n\pi}{a}x\big)-\frac{(n\pi x/a)^{2}-2}{(n\pi/a)^{3}}\cos\big(\frac{n\pi}{a}x\big)\Big]\right\vert_{0}^{a}\Big\}\\
&=\frac{2\sqrt{15}}{a^{3}}\Big[-\frac{a^{3}}{n\pi}\cos(n\pi)+a^{3}\frac{(n\pi)^{2}-2}{(n\pi)^{3}}\cos(n\pi)+a^{3}\frac{2}{(n\pi)^{3}}\cos(0)\Big]\\
&=\frac{4\sqrt{15}}{(n\pi)^{3}}[\cos(0)-\cos(n\pi)]\\
&=\begin{cases}
0 & \text{\RL{جفت $n$}}\\
8\sqrt{15}/(n\pi)^{3}& \text{\RL{طاق $n$}}
\end{cases}
\end{align*}
یوں درج ذیل ہو گا (مساوات \حوالہ{مساوات_شروڈنگر_ساکن_حالات_کا_مجموعہ})۔
 \begin{align*}
\Psi(x,t)=\sqrt{\frac{30}{a}}\big(\frac{2}{\pi}\big)^3\sum_{n=1,3,5\cdots}\frac{1}{n^3}\sin\big(\frac{n\pi}{a}x\big)e^{-in^{2}\pi^{2}\hslash t/2ma^{2}}
\end{align*}
\انتہا{مثال}
%%%%%%%%%%%%%%%%%%%%%%%
%%%%AM HERE To enter what lies between ex 2.2 and ex 2.3
% Missing region here
%=========================

\ابتدا{مثال}\شناخت{مثال_شروڈنگر_عددی_سر_توقعاتی}
ہم نے دیکھا  کہ مثال \حوالہء{2.2} میں ابتدائی تفاعل موج (شکل \حوالہء{2.3}) زمینی حال \عددی{\psi_{1}} (شکل \حوالہء{2.2}) کے ساتھ قریبی مشابہت رکھتا ہے۔ یوں ہم توقع کرتے گے کہ \عددی{\left| c_{1} \right|^{2}}  غالب ہو گا۔ یقیناً ایسا ہی ہے۔
\begin{align*}
\left| c_{1} \right|^{2} = \left( \frac{8\sqrt{15}}{\pi^{3}} \right)^{2} = 0.998555 \cdots
\end{align*}
باقی تمام عددی سر مل کر فرق دیتے ہیں:
\begin{align*}
\sum_{n=1}^{\infty} \left| c_{n} \right|^{2} = \left( \frac{8\sqrt{15}}{\pi^{3}} \right)^{2} \sum_{n=1,3,5,...}^{\infty} \frac{1}{n^{6}} = 1
\end{align*}
اس مثال میں توانائی کی توقعاتی قیمت ہماری توقعات کے عین مطابق درج ذیل ہے۔
\begin{align*}
\langle H \rangle = \sum_{n=1,3,5,...}^{\infty} \left( \frac{8\sqrt{15}}{n^{3} \pi^{3}} \right)^{2} \frac{n^{2} \pi^{2} \hslash^{2}}{2ma^{2}} = \frac{480\hslash^{2}}{\pi^{4} ma^{2}} \sum_{n=1,3,5,...}^{\infty} \frac{1}{n^{4}} = \frac{5 \hslash^{2}}{ma^{2}}
\end{align*}
 یہ \عددی{ E_{1} = \pi^{2} \hslash^{2}/2ma^{2}  } کے بہت قریب،  حجان حل حالتوں کی شمول کی بنا معمولی زیادہ ہے۔ 
\انتہا{مثال}
%===========
\ابتدا{سوال}\شناخت{سوال_شروڈنگر_حل_ناقابل_قبول}
دکھائیں کہ لا متناہی چوکور کنواں کے لئے  \عددی{ E = 0 } یا \عددی{ E < 0 } کی صورت میں  غیر تابع وقت شروڈنگر مساوات کا کوئی بھی قابل قبول حل  نہیں پایا جاتا ہے۔ (یہ سوال \حوالہء{2.2} میں دیے گئے عمومی مسئلے کی ایک خصوصی صورت ہے، لیکن اس بار شروڈنگر مساوات کو صریحاً حل کرتے ہوئے دکھائیں کہ آپ سرحدی شرائط پر پورا نہیں اتر سکتے ہیں۔)
 \انتہا{سوال}
%==============
\ابتدا{سوال}
لامتناہی چکور کنواں کے \عددی{ n } وی ساکن حال کیلئے \عددی{\langle x \rangle}، \عددی{\langle x^2 \rangle} ، \عددی{\langle p \rangle}، \عددی{\langle p^2 \rangle}، \عددی{\sigma_x}  اور \عددی{\sigma_p}  تلاش کریں۔ تصدیق کریں کہ اصول غیر یقینیت مطمئن ہوتا ہے۔ کونسا حال غیر یقینیت کی حد کے قریب ترین ہو گا؟
\انتہا{سوال}
%=============
\ابتدا{سوال}\شناخت{سوال_شروڈنگر_لامتناہی_کنواں_برابر_حصے}
لامتناہی چکور کنواں میں ایک ذرے کا ابتدائی تفاعل موج اولین دو ساکن حالات کے برابر حصوں کا مرکب ہے۔ 
\begin{align*}
\Psi(x,0) = A[\psi_{1}(x) + \psi_{2}(x)]
\end{align*}
\begin{enumerate}[a.]
\item 
\عددی{ \Psi(x,0) } کو معمول پر لائیں۔ (یعنی \عددی{ A } تلاش کریں۔ آپ \عددی{ \psi_{1} } اور \عددی{ \psi_{2} } کی معیاری عمودیت بروئے کار لاتے ہوئے با آسانی ایسا کر سکتے ہیں۔ یاد رہے کہ \عددی{ t=0 } پر \عددی{  \Psi  } کو معمول پر لانے کے بعد آپ یقین رکھ سکتے  ہیں کہ یہ معمول شدہ ہی رہے گا۔ اگر آپ کو شک ہے، جزو-ب کا نتیجہ حاصل کرنے کے بعد  اس کی صریحاً تصدیق کریں۔) 
\item
\عددی{ \Psi(x,t) } اور \عددی{ \left| \Psi (x,t) \right|^{2} } تلاش کریں۔ موخر الذکر  کو وقت کے سائن نما تفاعل کی صورت میں لکھیں، جیسا مثال \حوالہء{2.1} میں کیا گیا۔ نتائج کو سادہ صورت میں لکھنے کی خاطر \عددی{\omega\equiv\tfrac{\pi^2\hslash}{2ma^2}} لیں۔ 
\item 
\عددی{ \langle x \rangle  } تلاش کریں۔ آپ دیکھیں گے کہ یہ وقت کے ساتھ ارتعاش کرتا ہے۔ اس ارتعاش کی زاویائی تعدد کتنی ہو گی؟ ارتعاش کا حیطہ کیا ہو گا؟ (اگر آپکا حیطہ \عددی{ \tfrac{a}{2}} سے زیادہ ہو تب آپ کو جیل بھیجنے کی ضرورت ہو گی۔) 
\item 
\عددی{ \langle p \rangle  } تلاش کریں (اور اس پہ زیادہ وقت صرف نہ کریں)۔ 
\item
اس ذرے کی توانائی کی پیمائش سے کون کون سی قیمتیں متوقع ہیں؟ اور ہر ایک قیمت کا احتمال کتنا ہو گا؟ \عددی{ H } کی توقعاتی قیمت تلاش کریں۔ اس کی قیمت کا موازنہ \عددی{ E_{1} } اور \عددی{ E_{2} } کے ساتھ کریں؟
\end{enumerate}
\انتہا{سوال}
%================
\ابتدا{سوال}
اگر چہ تفاعل موج کا مجموعی زاویائی مستقل کسی با معنی طبعی اہمیت کا حامل نہیں ہے (چونکہ یہ کسی بھی قابل پیمائش مقدار میں کٹ جاتا ہے) لیکن  مساوات \حوالہ{مساوات_شروڈنگر_عمومی_حل_مجموعہ} میں عددی سروں کے اضافی زاویائی مستقل اہمیت کے حامل ہیں۔ مثال کے طور پر ہم سوال \حوالہ{سوال_شروڈنگر_لامتناہی_کنواں_برابر_حصے} میں  \عددی{ \psi_{1} } اور \عددی{ \psi_{2} } کے اضافی زاویائی مستقل تبدیل کرتے ہیں:
\begin{align*}
\Psi (x,0) = A[\psi_{1} (x) + e^{i\phi}\psi_{2}(x)]
\end{align*}
جہاں \عددی{ \phi } کوئی مستقل ہے۔ \عددی{  \Psi(x,t) }، \عددی{  \ \left| \Psi (x,t) \right|^{2} } اور \عددی{ \langle x \rangle } تلاش کر کے ان کا موازنہ پہلے حاصل شدہ نتائج کے ساتھ کریں۔ بالخصوص \عددی{ \phi = \pi/2 } اور  \عددی{ \phi = \pi } کی صورتوں پر غور کریں۔ 
\انتہا{سوال}
%===========
\ابتدا{سوال} 
لا متناہی چکور کنواں میں ایک ذرے کا ابتدائی تفاعل موج درج ذیل ہے۔
\begin{align*}
\Psi (x,0) = 
\begin{cases}
Ax, & 0 \leq x \leq a/2 \\ 
A(a-x), & a/2 \leq x \leq a
\end{cases}
\end{align*}
\begin{enumerate}[a.]
\item 
\عددی{\Psi(x,0)} کا خاکہ کھینچیں اور مستقل \عددی{A} کی قیمت تلاش کریں۔
\item  
\عددی{\Psi(x,t) } تلاش کریں۔
\item  
توانائی کی پیمائش کا نتیجہ \عددی{E_{1}} ہونے کا احتمال کتنا ہو گا؟
\item 
توانائی کی توقعاتی قیمت تلاش کریں۔
\end{enumerate}
\انتہا{سوال}
%===========
\ابتدا{سوال}
ایک لامتناہی چکور کنواں، جسکی چوڑائی \عددی{a} ہے، میں کمیت \عددی{m} کا ایک ذرہ کنویں کے بائیں حصے سے ابتدا ہوتا ہے اور یہ \عددی{t=0} پر بائیں نصف حصے کے کسی بھی نقطے پر ہو سکتا ہے۔
\begin{enumerate}[a.]
\item
اس کی ابتدائی تفاعل موج \عددی{\Psi(x,0) } تلاش کریں۔ (فرض کریں کے یہ حقیقی ہے اور اسے معمول پر لانا نا بھولیے گا۔)
\item 
پیمائش توانائی کا نتیجہ \عددی{\pi^{2}\hslash^{2}/2ma^{2}} ہونے کا احتمال کیا ہو گا؟ 
\end{enumerate} 
\انتہا{سوال}
%==========
\ابتدا{سوال}
لمحہ \عددی{t=0}  پر مثال \حوالہء{2.2} کے تفاعل موج کیلئے \عددی{H} کی توقعاتی قیمت تکمل کے ذریعہ حاصل کریں۔ 
\begin{align*}
\langle H \rangle=\int \Psi(x,0)^*\hat{H}\, \Psi(x,0)\dif x
\end{align*}
مثال \حوالہ{مثال_شروڈنگر_عددی_سر_توقعاتی} میں مساوات \حوالہء{2.39} کی مدد سے حاصل کردہ نتیجے کے ساتھ موازنہ کریں۔ دھیان رہے کیونکہ \عددی{H}  غیر تابع وقت ہے لہٰذا \عددی{t=0} لینے سے نتیجے پر کوئی اثر نہیں ہو گا۔ 
\انتہا{سوال}
%=========================

\حصہ{ہارمونی مرتعش}
کلاسیکی ہارمونی مرتعش ایک لچک دار اسپرنگ جس کا مقیاس لچک \عددی{k} ہو  اور  کمیت \عددی{m}  پر مشتمل ہوتا ہے ۔ کمیت کی حرکت \اصطلاح{قانون ہک}\فرہنگ{قانون!ہک}\حاشیہب{Hooke's law}\فرہنگ{law!Hooke} 
\begin{align*}
F=-kx=m\frac{\dif{^{2}x}}{\dif{t^{2}}}
\end{align*}
کے تحت ہو گی جہاں رگڑ کو نظر انداز کیا گیا ہے ۔ اس کا حل
\begin{align*}
x(t)=A\sin(\omega t)+B\cos(\omega t)
\end{align*}
ہو گا جہاں
\begin{align}\label{مساوات_شروڈنگر_زاویائی_تعدد}
\omega\equiv \sqrt{\frac{k}{m}}
\end{align}
ارتعاش کا (زاویائی) تعدد ہے ۔ مخفی توانائی
\begin{align*}
V(x)=\frac{1}{2}Kx^{2}
\end{align*}
ہو گی جس کی ترسیم قطع مکافی ہے۔ 

حقیقت میں کامل ہارمونی مرتعش نہیں پایا جاتا ہے۔ اگر آپ اسپرنگ کو زیادہ  کھینچیں تو وہ  ٹوٹ جائے گا اور قانون ہک اس سے بہت پہلے غیر کارآمد ہو چکا ہو گا۔ تاہم عملاً کوئی بھی مخفی قوہ، مقامی کم سے کم نقطہ کی پڑوس میں تخمیناً قطع مکافی  ہو گا (شکل \حوالہء{2.4})۔مخفی قوہ \عددی{V(x)} کے کم سے کم نقطہ \عددی{x_0} کے لحاظ سے \عددی{V(x)} کو \اصطلاح{ٹیلر تسلسل}\فرہنگ{تسلسل!ٹیلر}\حاشیہب{Taylor series}\فرہنگ{series!Taylor} کے لحاظ سے پھیلا کر
\begin{align*}
V(x)=V( x_{0})+V'(x_{0})(x-x_{0})+\frac{1}{2}V''(x_{0})(x-x_{0})^{2}+\cdots
\end{align*}
اس سے \عددی{V(x_0)} منفی کر کے (ہم \عددی{V(x)} سے کوئی بھی مستقل بغیر خطر و فکر منفی کر سکتے ہیں کیونکہ ایسا کرنے  سے قوت تبدیل نہیں ہو گا)  اور یہ جانتے ہوئے کہ \عددی{V'(x_0)=0} ہو گا (چونکہ \عددی{x_0} کم سے کم نقطہ ہے)، ہم تسلسل کے بلند رتبی ارکان  رد کرتے ہوئے (جو \عددی{(x-x_0)} کی قیمت کم ہونے کی صورت میں قابل نظرانداز ہونگے) درج ذیل حاصل کرتے ہیں
\begin{align*}
V(x)\cong\frac{1}{2}V''(x_{0})(x-x_{0})^{2}
\end{align*}
جو نقطہ \عددی{x_0} پر ایک ایسی سادہ  ہارمونی ارتعاش  بیان کرتا ہے  جس کا موثر مقیاس لچک \عددی{k=V''(x_0)} ہو۔ یہی وہ وجہ ہے جس کی بنا سادہ ہارمونی مرتعش اتنا اہم ہے: تقریباً ہر وہ ارتعاشی حرکت جس کا حیطہ کم ہو تخمیناً سادہ ہارمونی ہو گا۔

کوانٹم میکانیات میں ہمیں مخفی قوہ
\begin{align*}
V(x)=\frac{1}{2}m\omega ^{2}x^{2}
\end{align*}
کے لیے شروڈنگر  مساوات حل کرنی ہو گی (جہاں روایتی طور پر مقیاس لچک کی جگہ کلاسیکی تعدد (مساوات \حوالہ{مساوات_شروڈنگر_زاویائی_تعدد}) استعمال کی جاتی ہے)۔  جیسا کہ ہم دیکھ چکے ہیں، اتنا کافی ہو گا کہ  ہم  غیر تابع وقت شروڈنگر مساوات
\begin{align}\label{مساوات_شروڈنگر_مخفی_قوہ_الف}
\frac{-\hslash ^{2}}{2m}\frac{\dif{^{2}\psi}}{\dif{x^{2}}}+\frac{1}{2}m\omega ^{2}x^{2}\psi=E\psi
\end{align}
حل کریں۔ اس مسئلے کو حل کرنے کے لیے دو بالکل مختلف طریقے اپنائے جاتے ہیں۔ پہلی میں  تفرقی مساوات کو "طاقت کے بل بوتے پر"   \اصطلاح{طاقتی تسلسل}\فرہنگ{تسلسل!طاقتی}\حاشیہب{power series}\فرہنگ{series!power} کے ذریعہ حل کرنے کی ترکیب استعمال کی جاتی ہے، جو دیگر مخفی قوہ کے لیے بھی کارآمد ثابت ہوتا ہے  (اور جسے استعمال کرتے ہوئے ہم باب \حوالہء{4} میں کولمب مخفی قوہ کے لیے حل تلاش کریں گے)۔ دوسری ترکیب ایک شیطانی الجبرائی تکنیک ہے جس میں \اصطلاح{عاملین سیڑھی} استعمال ہوتے ہیں ۔ میں آپ کی واقفیت پہلے الجبرائی تکنیک کے ساتھ پیدا کرتا ہوں جو زیادہ سادہ، زیادہ دلچسپ (اور جلد حل دیتا) ہے۔ اگر آپ طاقتی تسلسل کی ترکیب یہاں استعمال نہ کرنا چاہیں تو آپ ایسا کر سکتے ہیں لیکن کہیں نہ کہیں  آپکو یہ ترکیب سیکھنی ہو گی۔

\حصہ{الجبرائی ترکیب}
ہم مساوات \حوالہ{مساوات_شروڈنگر_مخفی_قوہ_الف} کو زیادہ معنی خیز روپ میں لکھ کر ابتدا کرتے ہیں
\begin{align}
\frac{1}{2m}[p^{2}+(m\omega x)^{2}]\psi=E\psi
\end{align}
جہاں \عددی{p\equiv \frac{\hslash}{i}\frac{d}{\dif{x}}} معیار حرکت کا عامل ہے۔ بنیادی طور پر  ہیملٹنی
\begin{align}
H=\frac{1}{2m}[p^{2}+(m\omega x)^{2}]
\end{align}
کو  کو \ترچھا{اجزائے ضربی} لکھنے کی ضرورت ہے۔اگر یہ عداد ہوتے تب ہم یوں لکھ سکتے تھے۔
\begin{align*}
u^{2}+v^{2}=(iu+v)(-iu+v)
\end{align*}
البتہ  یہاں بات اتنی سادہ نہیں ہے چونکہ \عددی{p} اور \عددی{x} \ترچھا{عاملین} ہیں اور  عاملین عموماً  \اصطلاح{قابل تبادل} نہیں ہوتے ہیں (یعنی آپ \عددی{xp} سے مراد \عددی{px} نہیں لے سکتے ہیں)۔ اس کے باوجود یہ ہمیں درج ذیل مقداروں پر غور کرنے پر آمادہ  کرتا ہے
\begin{align}
a\pm\equiv \frac{1}{\sqrt{2\hslash m\omega}}(\mp ip+m\omega x)
\end{align}
(جہاں قوسین کے باہر جزو ضربی لگانے سے آخری نتیجہ خوبصورت نظر آئے گا)۔

آئیں دیکھیں حاصل ضرب \عددی{a_{-}a_{+}}  کیا ہو گا؟
\begin{align*}
a_{-}a_{+}&=\frac{1}{2\hslash m\omega}(ip+m\omega x)(-ip+m\omega x)\\
&=\frac{1}{2\hslash m\omega}[p^{2}+(m\omega x)^{2}-im\omega(xp-px)]
\end{align*}
اس میں متوقع اضافی جزو \عددی{(xp-px)} پایا جاتا ہے جس کو ہم \عددی{x} اور \عددی{p} کا  \اصطلاح{تبادل کار}\فرہنگ{تبادل کار}\حاشیہب{commutator}\فرہنگ{commutator} کہتے ہیں اور جو ان کی آپس میں قابل تبادل نہ  ہونے کی پیمائش ہے۔ عمومی طور پر عامل \عددی{A} اور عامل \عددی{B} کا تبادل کار (جسے چکور قوسین میں لکھا ہے) درج ذیل ہو گا۔
\begin{align}
[A,B]\equiv AB-BA
\end{align}
اس علامتیت کے تحت درج ذیل ہو گا۔
\begin{align}\label{مساوات_شروڈنگر_سیڑھی_حاصل_ضرب}
a_{-}a_{+}=\frac{1}{2\hslash m\omega}[p^{2}+(m\omega x)^{2}]-\frac{i}{2\hslash}[x,p]
\end{align}

ہمیں \عددی{x} اور  عددی{p} کا تبادل کار دریافت کرنا ہو گا۔ \ترچھا{انتباہ:} عاملین پر ذہنی  کام کرنا  عموماً غلطی کا سبب بنتا ہے۔ بہتر ہو گا کہ  عاملین پرکھنے کے لیے آپ انہیں تفاعل \عددی{f(x)} عمل کرنے کے لئے پیش کریں۔ آخر میں اس پرکھی تفاعل کو رد کر کے آپ صرف عاملین پر مبنی مساوات حاصل کر سکتے ہیں۔ موجودہ صورت میں درج ذیل ہو گا۔
\begin{align}
[x,p]f(x)=\big[x\frac{\hslash}{i}\frac{d}{\dif{x}}(f)-\frac{\hslash}{i}\frac{d}{\dif{x}}(xf)\big ]=\frac{\hslash}{i}\big (x\frac{\dif{f}}{\dif{x}}-x\frac{\dif{f}}{\dif{x}}-f\big )=-i\hslash f(x)
\end{align}
پرکھی تفاعل (جو اپنا کام کر چکا) کو رد کرتے ہوئے درج ذیل ہو گا۔
\begin{align}
[x,p]=i\hslash
\end{align}
یہ خوبصورت نتیجہ جو بار بار سامنے آتا ہے \اصطلاح{باضابطہ تبادلی رشتہ}\فرہنگ{تبادلی!باضابطہ رشتہ}\حاشیہب{canonical commutation relation}\فرہنگ{commutation!canonical relation} کہلاتا ہے۔

اسے کے استعمال سے مساوات \حوالہ{مساوات_شروڈنگر_سیڑھی_حاصل_ضرب} درج ذیل روپ  
\begin{align}
a_{-}a_{+}=\frac{1}{\hslash\omega}H+\frac{1}{2}
\end{align}
یا
\begin{align}
H=\hslash\omega\big (a_{-}a_{+}-\frac{1}{2} \big )
\end{align}
اختیار کرتی ہے۔ آپ نے دیکھا کہ ہیملٹنی کو ٹھیک اجزائے ضربی کی صورت میں نہیں لکھا جا سکتا  اور دائیں ہاتھ اضافی \عددی{-\tfrac{1}{2}} ہو گا۔ یاد رہے گا یہاں \عددی{a_+} اور \عددی{a_-} کی ترتیب بہت اہم ہے۔ اگر آپ \عددی{a_+} کو بائیں طرف رکھیں تو درج ذیل حاصل ہو گا۔
\begin{align}
a_{+}a_{-}=\frac{1}{\hslash\omega}H-\frac{1}{2}
\end{align}
بالخصوص درج ذیل ہو گا۔
\begin{align}
[a_{-},a_{+}]=1
\end{align}
یوں ہیملٹنی کو درج ذیل بھی لکھا جا سکتا ہے۔
\begin{align}
H=\hslash\omega\big (a_{+}a_{-}+\frac{1}{2}\big )
\end{align}

%KKKKKKKKKKKKKKK (the above is eq 2.56 on page 56 of my book)
