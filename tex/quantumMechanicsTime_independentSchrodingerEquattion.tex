% complete from beginning to eq 2.137 (p86)
\باب{غیر تابع وقت شروڈنگر مساوات}\شناخت{باب_غیر_تابع_وقت_شروڈنگر_مساوات}
\حصہ{ساکن حالات}
باب اول میں ہم نے تفاعل موج پر بات کی جہاں اس کا استعمال کرتے ہوئے دلچسپی کے مختلف مقداروں کا حساب کیا گیا۔ اب وقت آن پہنچا ہے کہ ہم کسی مخصوص مخفی توانائی \عددی{ V (x,t) } کی لئے شروڈنگر مساوات

\begin{align}\label{مساوات_شروڈنگر_تابع_وقت}
i \hslash \frac{\partial \Psi}{\partial t} = - \frac{\hslash^{2}}{2 m} \frac{\partial^{2} \Psi}{\partial x^{2}} + V\Psi
\end{align}
  حل کرتے ہوئے  \عددی{ \Psi (x,t) }   حاصل کرنا سیکھیں۔  اس باب میں (بلکہ کتاب کے بیشتر حصے میں) ہم فرض کرتے ہیں  کہ  \عددی{ V } وقت \عددی{t} کا تابع نہیں ہے۔ ایسی صورت میں مساوات شروڈنگر کو \اصطلاح{علیحدگی متغیرات}\فرہنگ{علیحدگی متغیرات}\حاشیہب{separation of variables}\فرہنگ{variables!separation of} کے طریقے سے حل کیا جا سکتا ہے، جو ماہر طبیعیات کا پسندیدہ طریقہ ہے۔ ہم ایسے حل تلاش کرتے ہیں جنہیں حاصل ضرب
\begin{align}
\Psi (x,t) = \psi (x)  \varphi (t)
\end{align}

 کی صورت میں لکھنا ممکن ہو جہاں \عددی{\psi} صرف \عددی{x} اور \عددی{\varphi} صرف \عددی{t}  کا تفاعل ہے۔ ظاہری طور پر حل پر ایسی شرط مسلط کرنا درست قدم نظر نہیں آتا ہے لیکن حقیقت میں یوں حاصل کردہ حل بہت کار آمد ثابت ہوتے ہیں۔ مزید (جیسا کہ علیحدگی متغیرات کیلئے عموماً ہوتا ہے)  ہم علیحدگی متغیرات سے حاصل حلوں کو یوں آپس میں جوڑ سکتے ہیں کہ ان سے عمومی حل حاصل کرنا ممکن ہو۔ قابل علیحدگی حلوں کیلئے درج ذیل ہو گا 
\begin{align*}
\frac{\partial \Psi }{\partial t} = \psi \frac{\dif \varphi}{\dif t}, \quad \frac{\partial^{2} \Psi}{\partial x^{2}} = \frac{\dif^{\,2} \Psi}{\dif x^{2}} \varphi
\end{align*}
جو سادہ تفرقی مساوات ہیں۔  ان کی مدد سے مساوات شروڈنگر درج ذیل روپ اختیار کرتی ہے۔
\begin{align*}
i \hslash \psi \frac{\dif \varphi}{\dif t} = - \frac{\hslash^{2}}{2m} \frac{\dif^{\,2} \psi}{\dif x^{2}} \varphi + V\psi\varphi
\end{align*}
دونوں اطراف کو \عددی{\psi\varphi} سے تقسیم کرتے ہیں۔
\begin{align}\label{مساوات_شروڈنگر_علاحدہ_الف}
i\hslash\frac{1}{\varphi} \frac{\dif \varphi}{\dif t} = - \frac{\hslash^{2}}{2m} \frac{1}{\psi} \frac{\dif^{\,2} \psi}{\dif x^{2}} + V
\end{align}
اب بائیں ہاتھ تفاعل صرف \عددی{t} کا تابع ہے جبکہ دایاں ہاتھ تفاعل صرف \عددی{x} کا تابع ہے۔ یاد رہے اگر \عددی{V} از خود \عددی{x} اور  \عددی{t} دونوں پر منحصر ہو تب ایسا نہیں ہو گا۔  صرف \عددی{t} تبدیل ہونے سے دایاں ہاتھ کسی صورت تبدیل نہیں ہو سکتا ہے جبکہ بایاں ہاتھ اور دایاں ہاتھ لازمی طور پر ایک دوسرے کے برابر ہیں لہٰذا \عددی{t} تبدیل کرنے سے بایاں ہاتھ بھی تبدیل نہیں ہو گا۔اسی طرح صرف \عددی{x} تبدیل کرنے سے بایاں ہاتھ تبدیل نہیں ہو سکتا ہے اور چونکہ دونوں اطراف لازماً ایک دوسرے کے برابر ہیں لہٰذا \عددی{x} تبدیل کرنے سے دایاں ہاتھ بھی تبدیل نہیں ہو گا۔  ہم کہہ سکتے ہیں کہ دونوں اطراف ایک مستقل کے برابر ہوں گے۔ (یہاں تسلی کر لیں کہ آپ کو یہ دلائل سمجھ آ گئے ہیں۔) اس مستقل کو ہم علیحدگی مستقل کہتے ہیں جس کو ہم 
\عددی{E} سے ظاہر کرتے ہیں۔ یو مساوات \حوالہ{مساوات_شروڈنگر_علاحدہ_الف} درج ذیل لکھی جا سکتی ہے۔ 
\begin{align}\label{مساوات_شروڈنگر_علیحدہ_اول}
i\hslash \frac{1}{\varphi} \frac{\dif \varphi}{\dif t} &= E\nonumber\\
\frac{\dif \varphi }{\dif t} &= - \frac{i E}{\hslash} \varphi &&\text{\RL{یا}}
\end{align}
اور
\begin{align}\label{مساوات_شروڈنگر_علیحدہ_دوم}
- \frac{\hslash^{2}}{2m} \frac{1}{\psi} \frac{\dif^{\,2} \psi}{\dif x^{2}} + V &= E\nonumber\\
- \frac{\hslash^{2}}{2m} \frac{\dif^{\,2} \psi}{\dif x^{2}} +V\psi &= E\psi&&\text{\RL{یا}}
\end{align}
علیحدگی متغیرات نے ایک جزوی تفرقی مساوات کو دو سادہ تفرقی مساوات (مساوات \حوالہ{مساوات_شروڈنگر_علیحدہ_اول} اور \حوالہ{مساوات_شروڈنگر_علیحدہ_اول}) میں علیحدہ کیا۔ ان میں سے پہلی  (مساوات \حوالہ{مساوات_شروڈنگر_علیحدہ_اول}) کو حل کرنا بہت آسان ہے۔ دونوں اطراف کو \عددی{\dif t} سے ضرب دیتے ہوئے تکمل لیں۔ یوں عمومی حل \عددی{C e^{-iEt/\hslash}} حاصل ہو گا۔ چونکہ ہم   حاصل ضرب\عددی{\psi\varphi } میں دلچسپی رکھتے ہیں لہٰذا ہم مستقل  \عددی{C} کو  \عددی{\psi}  میں ضم کر سکتے ہیں۔ یوں مساوات \حوالہ{مساوات_شروڈنگر_علیحدہ_اول} کا حل درج ذیل لکھا جا سکتا ہے۔ 
\begin{align}
\varphi(t) = e^{-iEt/\hslash}
\end{align}
دوسری  (مساوات \حوالہ{مساوات_شروڈنگر_علیحدہ_دوم})  کو \اصطلاح{غیر تابع وقت شروڈنگر مساوات}\فرہنگ{شروڈنگر!غیر تابع وقت}\حاشیہب{time-independent Schrodinger align}\فرہنگ{Schrodinger!time-independent} کہتے ہیں۔ پوری طرح مخفی توانائی \عددی{V} جانے بغیر ہم آگے نہیں بڑھ سکتے ہیں۔

اس باب کے باقی حصے میں ہم  مختلف سادہ خفی توانائی کیلئے غیر تابع وقت شروڈنگر مساوات حل کریں گے۔ ایسا کرنے سے پہلے آپ پوچھ سکتے ہیں کہ علیحدگی متغیرات کی کیا خاص بات ہے؟ بہرحال تابع وقت شروڈنگر مساوات کے زیادہ تر حل \عددی{\psi (x) \varphi (t)}  کی صورت میں نہیں لکھے جا سکتے۔ میں اس کے تین جوابات دیتا ہوں۔ ان میں سے دو طبعی اور ایک ریاضیاتی ہو گا۔ 

\عددی{(1}\quad 
یہ \اصطلاح{ساکن حالات}\فرہنگ{ساکن!حالات} ہیں۔ اگرچہ تفاعل موج ازخود 
\begin{align}\label{مساوات_شروڈنگر_غیر_تابع_اور_تابع}
\Psi (x,t) = \psi (x) e^{-iEt/\hslash}
\end{align}
وقت \عددی{t} کا تابع ہے،  \ترچھا{کثافت احتمال}
\begin{align}
\left| \Psi (x,t) \right|^{2} = \Psi^{*}\Psi = \psi^{*}e^{+iEt/\hslash} \psi e^{-iEt/\hslash} = \left| \psi (x) \right|^{2}
\end{align}
وقت کا تابع نہیں ہے؛ تابعیت وقت کٹ جاتی ہے۔ یہی کچھ کسی بھی حرکی متغیر کی توقعاتی قیمت کے حساب میں ہو گا۔ مساوات \حوالہ{مساوات_تفاعل_موج_توقعاتی_قیمت_حصول} تخفیف کے بعد درج ذیل  صورت اختیار کرتی ہے۔ 
\begin{align}
\langle Q(x,p) \rangle = \int \psi^{*} Q \left( x, \frac{\hslash}{i} \frac{\dif}{\dif x} \right) \psi \dif x
\end{align}
ہر توقعاتی قیمت \ترچھا{وقت} میں \ترچھا{مستقل} ہو گی؛ یہاں تک کہ ہم  \عددی{\varphi (t) } کو رد کر کے  \عددی{\Psi} کی
 جگہ  \عددی{\psi} استعمال کر کے وہی نتائج حاصل کر سکتے ہیں۔ اگرچہ بعض اوقات  \عددی{\psi}  کو ہی تفاعل موج پکارا جاتا ہے، لیکن ایسا کرنا حقیقتاً غلط ہے جس سے مسئلے کھڑے ہو سکتے ہیں۔ یہ ضروری ہے کہ آپ یاد رکھیں کہ اصل تفاعل موج ہر صورت تابع وقت ہو گا۔ بالخصوص  \عددی{ \langle x \rangle } مستقل  ہو گا لہٰذا (مساوات \حوالہ{مساوات_تفاعل_موج_تعریف_معیار_حرکت} کے تحت)   \عددی{ \langle p \rangle = 0 }  ہو گا۔ ساکن حال میں کبھی بھی کچھ نہیں ہوتا ہے۔ 

\عددی{(2} \quad
یہ \ترچھا{غیر مبہم کل توانائی} کے حالات ہوں گے۔ کلاسیکی میکانیات میں کل توانائی (حرکی جمع خفی) کو \اصطلاح{ہیملٹنی}\فرہنگ{ہیملٹنی}\حاشیہب{Hamiltonian}\فرہنگ{Hamiltonian} کہتے ہیں جس کو \عددی{H} سے ظاہر کیا جاتا ہے۔ 
\begin{align}
H(x,p) = \frac{p^{2}}{2m} + V(x)
\end{align}
اس کا مطابقتی ہیملٹنی عامل، قواعد و ضوابط کے تحت  \عددی{p \rightarrow (\hslash/i)(\partial/\partial x)}  پر کر کے درج ذیل حاصل ہو گا۔ 
\begin{align}
\hat{H} = - \frac{\hslash^{2}}{2m} \frac{\partial^{2}}{\partial x^{2}} + V(x) 
\end{align}
یوں غیر تابع وقت شروڈنگر مساوات \حوالہ{مساوات_شروڈنگر_علیحدہ_دوم} درج ذیل روپ اختیار کریگی 
\begin{align}
\hat{H} \psi = E\psi
\end{align}

جس کے کل توانائی کی توقعاتی قیمت درج ذیل ہو گی۔ 
\begin{align}
\langle H \rangle = \int \psi^{*} \hat{H}\psi \dif x = E \int \left| \psi \right|^{2} \dif x = E \int \left| \Psi \right|^{2} \dif x = E
\end{align}
آپ دیکھ سکتے ہیں کہ \عددی{\psi} کی معمول زنی \عددی{\psi} کی معمول زنی کے مترادف ہے۔ مزید درج ذیل 
\begin{align*}
\hat{H}^{2} \psi = \hat{H} (\hat{H}\psi ) = \hat{H} ( E\psi ) = E (\hat{H} \psi ) = E^{2} \psi 
\end{align*}
کی بنا درج ذیل ہو گا۔
\begin{align*}
\langle H^{2} \rangle = \int \psi^{*} \hat{H}^{2} \psi \dif x = E^{2} \int \left| \psi \right|^{2} \dif x = E^{2}
\end{align*}
یوں\عددی{H} کی تغیریت درج ذیل ہو گی۔ 
\begin{align}
\sigma^{2}_{H} = \langle H^{2} \rangle - \langle H \rangle^{2} =E^{2} - E^{2} = 0
\end{align}
یاد رہے کہ  \عددی{\sigma = 0}  کی صورت میں تمام ارکان کی قیمت ایک دوسری جیسی ہو گی (تقسیم کا پھیلاؤ صفر ہو گا)۔ نتیجتاً قابل علیحدگی حل کی ایک خاصیت یہ ہو ہے کہ کل توانائی کی ہر پیمائش یقیناً ایک ہی قیمت \عددی{E} دے گی۔ (اسی کی بنا  علیحدگی مستقل کو \عددی{E} سے ظاہر کیا گیا۔) 

\عددی{(3}\quad
عمومی حل قابل علیحدگی حلوں کا \اصطلاح{خطی جوڑ}\فرہنگ{خطی جوڑ}\حاشیہب{linear combination}\فرہنگ{linear!combination} ہو گا۔ جیسا ہم جلد دیکھیں گے،  غیر تابع وقت شروڈنگر مساوات  (مساوات \حوالہ{مساوات_شروڈنگر_علیحدہ_دوم})   لامتناہی تعداد کے حل \عددی{(\psi_{1}(x),\, \psi_{2}(x),\, \psi_{3}(x), \cdots)}  دے گا جہاں ہر ایک حل کے ساتھ ایک علیحدگی مستقل \عددی{(E_1,E_2,E_3,\cdots)} منسلک ہو گا لہٰذا  ہر \اصطلاح{اجازتی توانائی}\فرہنگ{توانائی!اجازتی}\حاشیہب{allowed energy}\فرہنگ{energy!allowed} کا ایک منفرد تفاعل موج پایا جائے گا۔ 
\begin{align*}
\Psi_{1} (x,t) = \psi_{1}(x)e^{-iE_{1}t/\hslash} , \quad \Psi_{2} (x,t) = \psi_{2}(x)e^{-iE_{2}t/\hslash}, \, \cdots 
\end{align*}
اب (جیسا کہ آپ خود تصدیق کر سکتے ہیں)  تابع وقت شروڈنگر مساوات  (مساوات \حوالہ{مساوات_شروڈنگر_تابع_وقت}) کی ایک خاصیت یہ ہے کہ اس کے حلوں کا ہر خطی جوڑ ازخود ایک حل ہو گا۔ ایک بار قابل علیحدگی حل تلاش کرنے کے بعد ہم زیادہ عمومی حل درج ذیل روپ میں تیار کر سکتے ہیں۔
\begin{align}\label{مساوات_شروڈنگر_خطی_جوڑ_عمومی_حل}
\Psi (x,t) = \sum_{n=1}^{\infty} c_{n} \psi_{n}(x)e^{-iE_{n}t/\hslash}
\end{align}
حقیقتاً  تابع وقت شروڈنگر مساوات کا ہر حل درج بالا روپ میں لکھا جا سکتا ہے۔ ایسا کرنے کی خاطر ہمیں وہ مخصوص مستقل 
\عددی{(c_{1},\, c_{2}, \, \cdots )}  تلاش کرنے ہوں گے جن کو استعمال کرتے ہوئے درج بالا حل (مساوات \حوالہ{مساوات_شروڈنگر_خطی_جوڑ_عمومی_حل}) ابتدائی شرائط مطمئن کرتا ہو۔ آپ آنے والے حصوں میں دیکھیں گے کہ ہم کس طرح یہ سب کچھ کر پائیں گے۔ باب \حوالہ{باب_قواعد_و_ضوابط} میں  ہم اس کو زیادہ مضبوط بنیادوں پر کھڑا کر پائیں گے۔ بنیادی نقطہ یہ ہے کہ ایک بار  غیر تابع وقت شروڈنگر مساوات حل کرنے کے بعد آپ کے مسائل ختم ہو جاتے ہیں۔ یہاں سے تابع وقت شروڈنگر مساوات کا عمومی حل حاصل کرنا آسان کام ہے۔ 

گزشتہ چار صفحات میں ہم بہت کچھ کہا جا چکا ہے۔ میں ان کو مختصراً اور مختلف نقطہ نظر سے دوبارہ پیش کرتا ہوں۔ زیر غور عمومی مسئلہ کا غیر تابع وقت خفی توانائی  \عددی{V(x)}  اور ابتدائی تفاعل موج  \عددی{\Psi (x,0)} دیے  گئے ہوں گے۔ آپ کو مستقبل کے تمام \عددی{t}  کیلئے  \عددی{ \Psi (x,t)} تلاش کرنا ہو گا۔ ایسا کرنے کی خاطر آپ تابع وقت شروڈنگر مساوات  (مساوات \حوالہ{مساوات_شروڈنگر_تابع_وقت}) حل کریں گے۔ پہلی قدم میں  آپ غیر تابع وقت شروڈنگر مساوات  (مساوات \حوالہ{مساوات_شروڈنگر_علیحدہ_دوم})  حل کر کے  لا متناہی تعداد کے حلوں  کا سلسلہ   \عددی{(\psi_{1}(x),\, \psi_{2}(x),\, \psi_{3}(x), \cdots)} حاصل کریں گے جہاں ہر ایک
 کی منفرد توانائی  \عددی{(E_{1}, \, E_{2}, \, E_{3}, \, \cdots)} ہو گی۔ ٹھیک ٹھیک \عددی{\Psi (x,0)} پر بیٹھنے کی خاطر آپ ان حلوں کا  خطی جوڑ لیں گے۔
\begin{align}\label{مساوات_شروڈنگر_ساکن_حالات_کا_خطی_جوڑ}
\Psi (x,0) = \sum_{n=1}^{\infty} c_{n} \psi_{n}(x)
\end{align}
یہاں کمال کی  بات یہ ہے کہ کسی بھی ابتدائی حال کے لئے آپ ہر صورت  مستقل \عددی{c_{1}, \,c_{2}, \,c_{3}, \cdots} دریافت کر پائیں گے۔ تفاعل موج \عددی{\Psi(x,t)}  تیار کرنے کی خاطر آپ ہر جزو کے ساتھ مختص تابعیت وقت \عددی{e^{-iE_nt/\hslash}} چسپاں کریں گے۔ 
\begin{align}\label{مساوات_شروڈنگر_عمومی_حل_مجموعہ}
\Psi (x,t) = \sum_{n=1}^{\infty} c_{n} \psi_{n}(x)e^{-iE_{n}t/\hslash} = \sum_{n=0}^{\infty} c_{n} \Psi_{n} (x,t)
\end{align}
چونکہ قابل علیحدگی حل
\begin{align}\label{مساوات_شروڈنگر_تمام_عمومی_حل}
\Psi_{n} (x,t) = \psi_{n}(x) e^{-iE_{n}t/\hslash}
\end{align}
کے تمام احتمال اور توقعاتی قیمتیں غیر تابع وقت ہوں گی لہٰذا یہ  از خود ساکن حالات ہوں گے، تا ہم عمومی حل  (مساوات \حوالہ{مساوات_شروڈنگر_عمومی_حل_مجموعہ}) یہ خاصیت نہیں رکھتا ہے؛  انفرادی ساکن حالات کی توانائیاں ایک دوسرے سے مختلف ہونے کی بنا \عددی{\left| \Psi \right|^{2}} کا حساب کرتے ہوئے  قوت نمائی ایک دوسرے کو حذف نہیں کرتی ہیں۔ 


\ابتدا{مثال}
فرض کریں  ایک ذرہ ابتدائی طور پر دو ساکن حالات کا خطی جوڑ ہو: 
\begin{align*}
\Psi (x,0) = c_{1} \psi_{1}(x) + c_{2} \psi_{2}(x) 
\end{align*}
(چیزوں کو سادہ رکھنے کی خاطر میں فرض کرتا ہوں کے مستقل  \عددی{c_{n}} اور حالات  \عددی{\psi_{n} (x)} حقیقی ہیں۔) مستقبل وقت t کیلئے تفاعل موج  \عددی{\Psi (x,t)} کیا ہو گا ؟ کثافت احتمال تلاش کریں اور ذرے کی حرکت  بیان کریں۔ 

حل:\quad
اس کا پہلا حصہ آسان ہے
\begin{align*}
\Psi (x,t) = c_{1} \psi_{1}(x)e^{-iE_{1}t/\hslash} + c_{2} \psi_{2}(x)e^{-iE_{2}t/\hslash}
\end{align*}
جہاں  \عددی{E_{1}} اور \عددی{E_{2}}  بالترتیب تفاعل  \عددی{\psi_{1}} اور  \عددی{\psi_{2}} کی مطابقتی توانائیاں ہیں۔ یوں درج ذیل ہو گا۔ 
\begin{align*}
\left| \Psi (x,t) \right|^{2} &= \left( c_{1} \psi_{1} e^{iE_{1}t/\hslash} + c_{2} \psi_{2} e^{iE_{2}t/\hslash} \right) \left( c_{1} \psi_{1} e^{-iE_{1}t/\hslash} + c_{2} \psi_{2} e^{-iE_{2}t/\hslash} \right) \\
&= c_{1}^{2} \psi_{1}^{2} + c_{2}^{2} \psi_{2}^{2} + 2c_{1}c_{2}\psi_{1}\psi_{2} \cos [ ( E_{2} - E_{1})t/\hslash]
\end{align*}
(میں نے نتیجہ کی سادہ صورت حاصل کرنے کی خاطر کلیہ یولر \عددی{e^{i\theta}=\cos\theta+i\sin\theta} استعمال کیا۔)  ظاہری طور پر  کثافت احتمال زاویائی تعدد \عددی{(\tfrac{E_2-E_1}{\hslash})} سے سائن نما ارتعاش کرتا ہے لہٰذا یہ ہرگز ساکن حال نہیں ہو گا۔ لیکن دھیان رہے کہ (ایک دوسرے سے مختلف) تونائیوں کے تفاعلات کے خطی جوڑ نے حرکت پیدا کیا۔ 
\انتہا{مثال}
%===========================
\ابتدا{سوال}
درج ذیل تین مسائل کا ثبوت پیش کریں۔
\begin{enumerate}[a.]
\item
قابل علیحدگی حلوں کے لئے علیحدگی مستقل \عددی{E} لازماً \ترچھا{حقیقی} ہو گا۔ \ترچھا{اشارہ:} مساوات \حوالہ{مساوات_شروڈنگر_غیر_تابع_اور_تابع} میں \عددی{E} کو \عددی{E_0+i\Gamma} لکھ کر (جہاں \عددی{E} اور \عددی{\Gamma} حقیقی ہیں)،  دکھائیں کہ تمام \عددی{t} کے لئے مساوات \حوالہء{1.20} اس صورت کارآمد ہو گا جب \عددی{\Gamma} صفر ہو۔ 
\item
غیر تابع وقت تفاعل موج \عددی{\psi(x)}  ہر موقع پر حقیقی لیا جا سکتا ہے (جبکہ تفاعل موج \عددی{\Psi(x,t)} لازماً مخلوط ہوتا ہے)۔ اس کا  ہرگز یہ مطلب نہیں ہے کہ غیر تابع شروڈنگر مساوات کا ہر حل حقیقی ہو گا؛ بلکہ غیر حقیقی حل پائے جانے کی صورت میں اس حل کو ہمیشہ،  ساکن حالات کا (اتنی ہی توانائی کا) خطی جوڑ لکھنا ممکن ہو گا۔  یوں بہتر ہو گا کہ آپ صرف حقیقی \عددی{\psi} ہی استعمال کریں۔ \ترچھا{اشارہ:} اگر کسی مخصوص \عددی{E} کے لئے  \عددی{\psi{(x)}} مساوات \حوالہ{مساوات_شروڈنگر_علیحدہ_دوم} کو مطمئن کرتا ہو تب اس کا مخلوط خطی جوڑ  بھی اس مساوات کو مطمئن کرے گا اور یوں ان کے خطی جوڑ \عددی{(\psi+\psi^*)} اور \عددی{i(\psi-\psi^*)} بھی اس مساوات کو مطمئن کریں گے۔
\item
اگر \عددی{V(x)} \اصطلاح{جفت تفاعل}\فرہنگ{جفت!تفاعل} ہو یعنی \عددی{V(-x)=V(x)} تب \عددی{\psi(x)} کو ہمیشہ جفت یا طاق لیا سکتے ہو۔ \ترچھا{اشارہ:} اگر کسی مخصوص \عددی{E} کے لئے  \عددی{\psi{(x)}} مساوات \حوالہ{مساوات_شروڈنگر_علیحدہ_دوم} کو مطمئن کرتا ہو تب \عددی{\psi(-x)}  بھی اس مساوات کو مطمئن کرے گا اور یوں ان کے جفت اور طاق خطی جوڑ \عددی{\psi(x)\pm\psi(-x)}  بھی اس مساوات کو مطمئن کریں گے۔ 
\end{enumerate}
\انتہا{سوال}
%====================
\ابتدا{سوال}\شناخت{سوال_شروڈنگر_کم_سے_کم_توانائی}
دکھائیں کہ غیر تابع وقت شروڈنگر مساوات کے ہر اس حل کے لئے، جس کو معمول پر لایا جا سکتا ہو، \عددی{E} کی قیمت لازماً \عددی{V(x)} کی کم سے کم قیمت سے زیادہ ہو گی۔ اس کا کلاسیکی مماثل کیا ہو گا؟ \ترچھا{اشارہ:}  مساوات \حوالہ{مساوات_شروڈنگر_علیحدہ_دوم} کو درج ذیل روپ میں لکھ کر
\begin{align*}
\frac{\dif^{\,2} \psi}{\dif x^2}=\frac{2m}{\hslash^2}[V(x)-E]\psi
\end{align*}
دکھائیں کہ \عددی{E<V_{\text{کمتر}}} کی صورت میں \عددی{\psi} اور اس کے دو گنّا تفرق کی علامتیں  لازماً ایک دوسری جیسی ہوں گی؛ اب دلیل پیش کریں  کہ ایسا تفاعل معمول پر لانے کے قابل نہیں ہو گا۔
\انتہا{سوال}
%====================
%%%%%%%%%%%%%%%%%%%%%
%%%%%%%%%%%%%%%
\حصہ {لامتناہی چکور  کنواں}
  درج ذیل فرض کریں (شکل  \حوالہء{2.1})۔
\begin{align}\label{مساوات_شروڈنگر_لامتناہی_چکور}
V(x)=
\begin{cases}
0& 0\le x\le a\\
\infty &\text{\RL{دیگر صورت}}
\end{cases}
\end{align}
اس مخفی توانائی میں ایک ذرہ مکمل آزاد ہو گا، ماسوائے  دونوں سروں یعنی \عددی{x=0}\عددی{x=a} پر، جہاں ایک لامتناہی قوت اس کو فرار  ہونے سے روکتی ہے۔ اس کا کلاسیکی نمونہ ایک کنواں  میں ایک لامتناہی لچکدار  گیند ہو سکتا ہے جو ہمیشہ کے لئے دیواروں سے ٹکرا کر دائیں سے بائیں اور بائیں سے دائیں  حرکت کرتا رہتا ہو۔ (اگرچہ یہ ایک فرضی مخفی توانائی ہے، آپ اس کو اہمیت دیں۔ اگرچہ یہ بہت سادہ نظر آتا ہے البتہ اس کی سادگی کی بنا ہی یہ بہت ساری معلومات فراہم کرنے کے قابل ہے۔ ہم اس سے بار بار رجوع کریں گے۔)

کنواں سے باہر \عددی{\psi (x)=0} ہو گا  (لہٰذا یہاں ذرہ پایا جانے  کا احتمال صفر ہو گا)۔ کنواں کے اندر، جہاں \عددی{V=0} ہے،  غیر تابع وقت  شروڈنگر مساوات  (مساوات \حوالہ{مساوات_شروڈنگر_علیحدہ_دوم}) درج ذیل روپ اختیار کرتی ہے۔
\begin{align}
-\frac{\hslash^{2}}{2m}\frac{\dif^{\,2}\psi}{\dif x^{2}}&=E\psi
\end{align} 
یا 
\begin{align}\label{مساوات_شروڈنگر_کلاسیکی_ہارمونی_مرتعش}
\frac{\dif^{\,2}\psi}{\dif x^{2}}&=-k^{\,2}\psi, && k\equiv \frac{\sqrt{2mF}}{\hslash}
\end{align}
(اس کو یوں لکھتے ہوئے  میں  خاموشی سے فرض کرتا ہوں کہ \عددی{E\ge 0} ہو گا۔ ہم  سوال \حوالہ{سوال_شروڈنگر_کم_سے_کم_توانائی} سے جانتے ہیں کہ \عددی{E< 0 } سے بات نہیں بنے گی۔)   مساوات \حوالہ{مساوات_شروڈنگر_کلاسیکی_ہارمونی_مرتعش}  کلاسیکی \اصطلاح{سادہ ہارمونی مرتعش}\فرہنگ{ہارمونی!مرتعش}\فرہنگ{مرتعش!ہارمونی}\حاشیہب{simple harmonic oscillator}\فرہنگ{harmonic!oscillator} کی مساوات ہے جس کا عمومی حل درج ذیل  ہو گا
\begin{align}
\psi(x)=A\sin kx+B\cos kx
\end{align}
 جہاں \عددی{A }  اور  \عددی{B} اختیاری مستقل ہیں۔ ان مستقلات کو مسئلہ  کے \اصطلاح{سرحدی شرائط}\فرہنگ{سرحدی شرائط}\حاشیہب{boundary conditions}\فرہنگ{boundary conditions} تعین کرتے ہیں۔ \عددی{\psi (x)} کے موزوں سرحدی شرائط کیا ہونگے؟ عموماً \عددی{\psi} اور \عددی{\tfrac{\dif \psi}{\dif x}} \ترچھا{دونوں استمراری} ہونگے، لیکن جہاں مخفیہ لامتناہی کو پہنچتا ہو وہاں  صرف اول الذکر کا اطلاق ہو گا۔ (میں حصہ  \حوالہء{2.5}میں ان سرحدی شرائط کو ثابت کروں گا اور \عددی{V=\infty}   کی صورت حال کو بھی دیکھوں گا۔ فی الحال مجھ پر یقین کرتے ہوئے میری کہی ہوئی بات مان لیں۔) 

تفاعل \عددی{\psi(x)} کے استمرار کی بنا درج ذیل ہو گا
\begin{align}
\psi(0)=\psi(a)=0
\end{align} 
تا کہ کنواں کے باہر اور کنواں کے اندر حل ایک دوسرے کے ساتھ جڑ سکیں۔ یہ ہمیں  \عددی{A}اور  \عددی{B}کے بارے میں کیا معلومات فراہم کرتی ہے؟ چونکہ
\begin{align*}
\psi(0)=A\sin 0+B\cos 0=B
\end{align*}
ہے لہٰذا \عددی{B=0}  اور درج ذیل ہو گا۔
\begin{align}
\psi(x)=A\sin kx
\end{align}
یوں  \عددی{\psi(a)=A\sin ka} کی بنا یا \عددی{A=0}   ہو گا (ایسی صورت میں ہمیں غیر اہم حل \عددی{\psi(x)=0}  ملتا ہے جو معمول پر لانے کے قابل نہیں ہے)  یا \عددی{\sin ka =0}  ہو گا جس کے تحت درج ذیل ہو گا۔
\begin{align}
ka&=0,\pm\pi,\pm2\pi,\pm3\pi,\cdots
\end{align} 
اب \عددی{ k =0}  (بھی \عددی{ \psi(x)=0}  دیتا ہے جس) میں ہم دلچسپی نہیں رکھتے اور  \عددی{ \sin(-\theta)=-\sin(\theta)}  کی بنا \عددی{k} کی منفی قیمتیں کوئی نیا حل نہیں دیتی ہیں  لہٰذا ہم منفی کی علامت کو  \عددی{A} میں ضم کر سکتے ہیں۔ یوں منفرد حل درج ذیل ہوں گے۔ 
\begin{align}
k_{n}=\frac{n\pi}{a},&& n=1,2,3,\cdots
\end{align}

دلچسپ بات یہ ہے کہ \عددی{x=a} پر سرحدی شرط مستقل   \عددی{A} تعین نہیں کرتا ہے بلکہ اس کی بجائے مستقل \عددی{k}تعین کرتے ہوئے    \عددی{E} کی اجازتی قیمتیں تعین کرتا ہے:
\begin{align}\label{مساوات_شروڈنگر_لامتناہی_چکور_کنواں_توانائیاں}
E_{n}=\frac{\hslash^2 k^{2}_{n}}{2m}=\frac{n^{2}\pi^{2}\hslash^{2}}{2ma^{2}}
\end{align} 
کلاسیکی صورت کے برعکس لامتناہی چکور  کنواں میں کوانٹم ذرہ ہر ایک توانائی کا حامل نہیں ہو سکتا ہے بلکہ اس کی توانائی کی قیمت کو درج بالا مخصوص \اصطلاح{اجازتی}\فرہنگ{اجازتی!توانائیاں}\حاشیہب{allowed}\فرہنگ{allowed!energies} قیمتوں  میں سے ہونا ہو گا۔ مستقل  \عددی{A} کی قیمت حاصل کرنے کے لئے\عددی{ \psi}  کو معمول پر لانا ہو گا: 
\begin{align*}
\int_{0}^{a}\abs{A}^{2}\sin^{2}(kx)\dif{x}=\abs{A}^{2}\frac{a}{2}=1,\quad \implies\quad \abs{A}^{2}=\frac{2}{a}
\end{align*} 
یہ  \عددی{A} کی صرف \ترچھا{مقدار} دیتی ہے ہے، تاہم مثبت حقیقی جذر \عددی{A=\sqrt{2/a}}  منتخب کرنا بہتر ہو گا (کیونکہ  \عددی{A} کا زاویہ کوئی طبعی معنی نہیں رکھتا ہے)۔  اس طرح کنواں کے اندر شروڈنگر مساوات کے حل درج ذیل ہوں گے۔
\begin{align}\label{مساوات_شروڈنگر_میری_سائے}
\psi_{n}(x)=\sqrt{\frac{2}{a}}\sin\big(\frac{n\pi}{a}x\big)
\end{align}
میرے قول کو پورا کرتے ہوئے، (ہر مثبت عدد صحیح \عددی{n} کے عوض  ایک حل دے کر)  غیر تابع وقت شروڈنگر مساوات نے حلوں کا ایک لامتناہی سلسلہ دیا ہے۔  ان میں سے اولین چند کو  شکل \حوالہء{2.2}میں ترسیم کیا گیا ہے جو لمبائی \عددی{a} کے دھاگے پر ساکن امواج کی طرح نظر آتے ہیں۔ تفاعل \عددی{\psi_{1}} جو \اصطلاح{زمینی حال}\فرہنگ{حال!زمینی}\حاشیہب{ground state}\فرہنگ{state!ground} کہلاتا ہے     کی توانائی کم سے کم ہے۔باقی حالات جن کی توانائیاں \عددی{n^{2}}    کے براہ راست بڑھتی ہیں  \اصطلاح{ہیجان حالات}\فرہنگ{حال!ہیجان}\حاشیہب{excited states}\فرہنگ{state!excited} کہلاتے ہیں۔ تفاعلات \عددی{\psi_n(x)}   چند اہم اور دلچسپ خواص رکھتے ہیں:
\begin{enumerate}
\item
 کنواں کے وسط کے لحاض سے یہ تفاعلات باری باری جفت اور طاق ہیں۔ \عددی{\psi_{1}}      جفت ہے، \عددی{\psi_{2}}      طاق ہے، \عددی{\psi_{3}}  جفت ہے، وغیرہ وغیرہ۔
\item
توانائی بڑھاتے ہوئے ہر اگلے حال کے \اصطلاح{عقدوں}\فرہنگ{عقدہ}\حاشیہب{nodes}\فرہنگ{node} (عبور صفر) کی تعداد میں ایک \عددی{(1)} کا اضافہ ہو گا۔  (چونکہ آخری نقاط کے صفر کو نہیں گنا جاتا ہے لہٰذا) \عددی{\psi_{1}} میں کوئی عقدہ نہیں پایا جاتا ہے،  \عددی{\psi_{2}}  میں ایک پایا جاتا ہے، \عددی{\psi_{3}} میں دو پائے جاتے ہیں، وغیرہ وغیرہ۔
\item
 یہ تمام درج ذیل نقطہ نظر سے باہمی \اصطلاح{عمودی}\فرہنگ{عمودی}\حاشیہب{orthogonal}\فرہنگ{orthogonal} ہیں جہاں  
\عددی{m\ne n} ہے۔
\begin{align}
\int\psi_{m}(x)^*\psi_{n}(x)\dif{x}=0
\end{align}
\ترچھا{ثبوت:}\quad
\begin{align*}
\int&\psi_{m}(x)^*\psi_{n}(x)\dif{x}=\frac{2}{a}\int_{0}^{a}\sin\big(\frac{m\pi}{a}x\big)\sin\big(\frac{n\pi}{a}x\big)\dif{x}\\
&=\frac{1}{a}\int_{0}^{a}\big[\cos\big(\frac{m-n}{a}\pi x\big)-\cos\big(\frac{m+n}{a}\pi x\big)\big]\dif{x}\\
&=\big\{\left.\frac{1}{(m-n)\pi}\sin\big(\frac{m-n}{a}\pi x\big)-\frac{1}{(m+n)\pi}\sin\big(\frac{m+n}{a}\pi x\big)\big\}\right\vert_{0}^{a}\\
&=\frac{1}{\pi}\big\{\frac{\sin[(m-n)\pi]}{(m-n)}-\frac{\sin[(m+n)\pi]}{(m+n)}\big\}=0
\end{align*} 
دھیان رہے کہ  \عددی{m= n}   کی صورت میں درج بالا دلیل درست نہیں ہو گا؛ (کیا آپ بتا سکتے ہیں کہ ایسی صورت میں دلیل کیوں ناقابل قبول ہو گا۔) ایسی صورت میں معمول پر لانے کا عمل ہمیں بتاتا ہے کہ تکمل کی قیمت  \عددی{1} ہے۔ درحقیقت، عمودیت اور معمول زنی کو ایک فقرے میں سمویا جا  سکتا ہے:\حاشیہد{یہاں تمام \عددی{\psi} حقیقی ہیں لہٰذا \عددی{\psi_m} پر \عددی{^*} ڈالنے کی ضرورت نہیں ہے، لیکن مستقل کی استعمال کے نقطہ نظر سے ایسا کرنا ایک اچھی عادت ہے۔}
\begin{align}
\int\psi_{m}(x)^*\psi_{n}(x)\dif{x}=\delta_{mn}
\end{align}
جہاں \عددی{\delta_{mn}}    \اصطلاح{کرونیکر ڈیلٹا}\فرہنگ{ڈیلٹا!کرونیکر}\حاشیہب{Kronecker delta}\فرہنگ{delta!Kronecker} کہلاتا ہے ہیں جس کی تعریف درج ذیل ہے۔ 
\begin{align}
\delta_{mn}=
\begin{cases}
0&  m\neq n\\
1 &  m=n
\end{cases}
\end{align} 
ہم کہتے ہیں کہ مذکورہ بالا (تمام) \عددی{\psi}  \اصطلاح{معیاری عمودی}\فرہنگ{معیار عمودی}\فرہنگ{عمودی!معیاری}\حاشیہب{orthonormal}\فرہنگ{orthonormal} ہیں۔ 
\item
 یہ \اصطلاح{مکمل}\فرہنگ{مکمل}\حاشیہب{complete}\فرہنگ{complete} ہیں، جس سے مراد ہے کہ کسی  بھی دوسرے تفاعل \عددی{f(x)} کو ان کا خطی جوڑ لکھا جا سکتا ہے:
\begin{align}\label{مساوات_شروڈنگر_کوئی_تفاعل}
f(x)=\sum_{n=1}^{\infty}c_{n}\psi_{n}(x)=\sqrt{\frac{2}{a}}\sum_{n=1}^{\infty}c_{n}\sin\big(\frac{n\pi}{a}x\big)
\end{align}
 میں تفاعلات \عددی{\sin\tfrac{n\pi x}{a}}   کی مکملیت کو یہاں ثابت نہیں کروں گا، البتہ  اعلٰی علم الاحصاء کے ساتھ واقفیت کی صورت میں آپ مساوات \حوالہ{مساوات_شروڈنگر_کوئی_تفاعل} کو \عددی{f(x)} کا \اصطلاح{فوریئر تسلسل}\فرہنگ{تسلسل!فوریئر}\حاشیہب{Fourier series}\فرہنگ{series!Fourier} پہچان پائیں گے۔ یہ حقیقت، کہ ہر تفاعل کو فوریئر تسلسل کی صورت میں پھیلا کر لکھا جا سکتا ہے، بعض اوقات \اصطلاح{مسئلہ ڈرشلے}\فرہنگ{مسئلہ!ڈرشلے}\حاشیہب{Dirichlet's theorem}\فرہنگ{theorem!Dirichlet's} کہلاتا ہے۔\حاشیہد{تفاعل \عددی{f(x)} میں متناہی تعداد کی عدم استمرار (چھلانگ) پائے جا سکتی ہیں۔}

کسی بھی دیے گئے تفاعل \عددی{f(x)} کے لئے عددی سروں \عددی{c_{n}}  کو \عددی{\{ \psi_n\}}  کی معیاری عمودیت کی مدد سے حاصل کیا جاتا ہے۔ مساوات \حوالہ{مساوات_شروڈنگر_کوئی_تفاعل} کے دونوں اطراف کو \عددی{\psi_{m}(x)} سے ضرب دے کر تکمل لیں:
 \begin{align}
\int \psi_{m}(x)^*f(x)\dif{x}=\sum_{n=1}^{\infty}c_{n}\int\psi_{m}(x)^*\psi_{n}(x)\dif{x}=\sum_{n=1}^{\infty}c_{n}\delta_{mn}=c_{m}
\end{align}
(آپ دیکھ سکتے ہیں کہ کرونیکر ڈیلٹا مجموعے میں تمام اجزاء کو ختم کر دیتا ہے ماسوائے اس جزو کو جس کے لئے  \عددی{n=m} ہو۔) یوں تفاعل   \عددی{f(x)} کے پھیلاو  کے \عددی{n} ویں جزو کا عددی سر درج ذیل ہو گا۔\حاشیہد{آپ یہاں نقلی متغیر کو \عددی{m} یا \عددی{n} یا کوئی تیسرا حرف لے سکتے ہیں (بس اتنا خیال رکھیں کہ مساوات کی دونوں اطراف ایک ہی حرف استعمال کریں)، اور ہاں یاد رہے کہ یہ حرف "کسی مثبت عدد صحیح" کو ظاہر کرتا ہے۔}
\begin{align}\label{مساوات_شروڈنگر_عددی_سر}
c_{n}=\int\psi_{n}(x)^*f(x)\dif{x}
\end{align}
\end{enumerate}

درج بالا   چار خواص انتہائی طاقتور ہیں جو صرف لامتناہی چکور  کنواں کے لیے مخصوص نہیں ہیں۔ پہلا خواص ہر اس صورت میں کارآمد ہو گا جب مخفیہ  تشاکلی ہو؛  دوسرا، مخفیہ  کی شکل و صورت سے قطع نظر، ایک عالمگیر خواص ہے۔ عمودیت بھی کافی  عمومی خاصیت ہے، جس کا ثبوت میں باب  \حوالہ{باب_قواعد_و_ضوابط} میں پیش کروں گا۔  ان تمام مخفیہ کے لئے جن کو آپ کا (ممکنہ) سامنا ہو سکتا ہے کے لئے مکملیت  کارآمد ہو گی، لیکن اس کا ثبوت کافی لمبا اور  پیچیدہ ہے؛ جس کی بنا عموماً ماہر طبیعیات یہ ثبوت دیکھے بغیر، اس کو مان لیتے ہیں۔

 لا متناہی چکور  کنواں کے ساکن حال (مساوات  \حوالہ{مساوات_شروڈنگر_تمام_عمومی_حل}) درج ذیل ہوں گے۔ 
 \begin{align}
\Psi_{n}(x,t)=\sqrt{\frac{2}{a}}\sin\big(\frac{n\pi}{a}x\big)e^{-i(n^{2}\pi^{2}\hslash/2ma^{2})t}
\end{align}
 میں نے دعوی  کیا (مساوات  \حوالہ{مساوات_شروڈنگر_عمومی_حل_مجموعہ}) کہ تابع وقت شروڈنگر مساوات کا عمومی ترین حل، ساکن حالات کا خطی جوڑ ہو گا۔
\begin{align}\label{مساوات_شروڈنگر_ساکن_حالات_کا_مجموعہ}
\Psi(x,t)=\sum_{n=1}^{\infty}c_n\sqrt{\frac{2}{a}}\sin\big(\frac{n\pi}{a}x\big)e^{-i(n^{2}\pi^{2}\hslash/2ma^{2})t}
\end{align}  
 (اگر آپ کو اس حل پر شق ہو تو اس کی تصدیق ضرور کیجیے گا۔) مجھے صرف اتنا دکھانا ہو گا کہ کسی بھی ابتدائی تفاعل موج \عددی{\psi(x,0)}   پر اس حل کو   بٹھانے کے لیے  موزوں  عددی سر \عددی{c_{n}} درکار ہوں گے:
\begin{align*}
\Psi(x,0)=\sum_{n=1}^{\infty}c_{n}\psi_{n}(x)
\end{align*}
تفاعلات \عددی{\psi }  کی مکملیت (جس کی تصدیق یہاں مسئلہ ڈرشلے کرتی ہے) اس کی ضمانت دیتی ہے کہ میں ہر \عددی{\psi(x,0)}   کو ہر صورت یوں بیان کر سکتا ہوں، اور ان کی معیاری عمودیت کی بنا   \عددی{c_{n}} کو فوریئر تسلسل سے حاصل کیا جا سکتا ہے: 
\begin{align}\label{مساوات_شروڈنگر_عددی_سروں}
c_{n}=\sqrt{\frac{2}{a}}\int_{0}^{a}\sin\big(\frac{n\pi}{a}x\big)\Psi(x,0)\dif{x}
\end{align}
آپ نے دیکھا: دی گئی ابتدائی تفاعل موج \عددی{\Psi(x,0)} کے لئے ہم سب سے پہلے پھیلاو کے عددی سروں  \عددی{c_{n}} کو مساوات \حوالہ{مساوات_شروڈنگر_عددی_سروں} سے حاصل کرتے ہیں۔   اس کے بعد انہیں
 مساوات  \حوالہ{مساوات_شروڈنگر_ساکن_حالات_کا_مجموعہ} میں پر کر \عددی{\Psi(x,t)} حاصل کرتے ہیں۔ تفاعل موج جانتے ہوئے  دلچسپی کی کسی بھی حرکی مقدار   کا حساب، باب  \حوالہ{باب_تفاعل_موج} میں مستعمل تراکیب استعمال کرتے ہوئے، کیا جا سکتا ہے۔ یہی ترکیب کسی بھی مخفیہ کے لیے کارآمد ہو گا؛  صرف  \عددی{\psi}  کی  قیمتیں اور اجازتی توانائیاں یہاں سے مختلف  ہوں گی۔
%=====================
\ابتدا{مثال}
لا متناہی چکور  کنواں میں ایک ذرے کا ابتدائی تفاعل موج درج ذیل ہے جہاں \عددی{A} ایک مستقل ہے ( شکل  \حوالہء{شکل 2.3})۔
\begin{align*}
\Psi(x,0)=Ax(a-x),&& (0\le x\le  a)
\end{align*}
 کنواں سے باہر \عددی{\psi=0}  ہے۔ \عددی{\Psi(x,t)}  تلاش کریں۔ 

حل:\quad
ہم پہلے \عددی{\Psi(x,0)}  کو معمول پر لاتے ہوئے 
\begin{align*}
1=\int_{0}^{a}\abs{\Psi(x,0)}^{2}\dif{x}=\abs{A}^{2}\int_{0}^{a}x^{2}(a-x)^{2}\dif{x}=\abs{A}^{2}\frac{a^{5}}{30}
\end{align*}
\عددی{A} تعین کرتے ہیں: 
\begin{align*}
A=\sqrt{\frac{30}{a^{5}}}
\end{align*}
مساوات  \حوالہ{مساوات_شروڈنگر_عددی_سروں} کے تحت \عددی{n} واں عددی سر درج ذیل ہو گا۔
\begin{align*}
c_{n}&=\sqrt{\frac{2}{a}}\int_{0}^{a}\sin\big(\frac{n\pi}{a}x\big)\sqrt{\frac{30}{a^{5}}}x(a-x)\dif{x}\\
&=\frac{2\sqrt{15}}{a^{3}}\Big[a\int_{0}^{a}x\sin\big(\frac{n\pi}{a}x\big)\dif{x}-\int_{0}^{a}x^{2}\sin\big(\frac{n\pi}{a}x\big)\dif x\Big]\\
&=\frac{2\sqrt{15}}{a^{3}}\Big\{a \left .\Big[\big(\frac{a}{n\pi}\big)^{2}\sin\big(\frac{n\pi}{a}x\big)-\frac{ax}{n\pi}\cos\big(\frac{n\pi}{a}x\big)\Big]\right\vert_{0}^{a}\.\\
&\quad-\left .\Big[2\big(\frac{a}{n\pi}\big)^{2}x\sin\big(\frac{n\pi}{a}x\big)-\frac{(n\pi x/a)^{2}-2}{(n\pi/a)^{3}}\cos\big(\frac{n\pi}{a}x\big)\Big]\right\vert_{0}^{a}\Big\}\\
&=\frac{2\sqrt{15}}{a^{3}}\Big[-\frac{a^{3}}{n\pi}\cos(n\pi)+a^{3}\frac{(n\pi)^{2}-2}{(n\pi)^{3}}\cos(n\pi)+a^{3}\frac{2}{(n\pi)^{3}}\cos(0)\Big]\\
&=\frac{4\sqrt{15}}{(n\pi)^{3}}[\cos(0)-\cos(n\pi)]\\
&=\begin{cases}
0 & \text{\RL{جفت $n$}}\\
8\sqrt{15}/(n\pi)^{3}& \text{\RL{طاق $n$}}
\end{cases}
\end{align*}
یوں درج ذیل ہو گا (مساوات \حوالہ{مساوات_شروڈنگر_ساکن_حالات_کا_مجموعہ})۔
 \begin{align*}
\Psi(x,t)=\sqrt{\frac{30}{a}}\big(\frac{2}{\pi}\big)^3\sum_{n=1,3,5\cdots}\frac{1}{n^3}\sin\big(\frac{n\pi}{a}x\big)e^{-in^{2}\pi^{2}\hslash t/2ma^{2}}
\end{align*}
\انتہا{مثال}
%
غیر محتاط بات چیت میں ہم کہتے ہیں کہ  \عددی{\Psi} میں \عددی{\psi_n} کی مقدار کو \عددی{c_n} ظاہر کرتا ہے۔  بعض اوقات ہم  کہتے ہیں کہ \عددی{n} ویں ساکن حال میں ایک ذرہ  پائے جانے کا احتمال \عددی{\abs{c_n}^2} ہے جو درست نہیں چونکہ ذرہ حال \عددی{\Psi} میں  نا کہ حال \عددی{\psi_n} میں پایا جاتا ہے؛ مزید تجربہ گاہ میں آپ کسی ایک ذرہ کو کسی ایک مخصوص حال میں نہیں دیکھ پاتے بلکہ آپ کسی  \ترچھا{مشہود} کی پیمائش کرتے ہو جس کا جواب ایک \ترچھا{عدد} کی صورت میں سامنے آتا ہے۔ جیسا آپ باب \حوالہ{باب_قواعد_و_ضوابط} میں دیکھیں گے، توانائی کی پیمائش سے \عددی{E_n} قیمت حاصل ہونے کا احتمال \عددی{\abs{c_n}^2} ہو گا۔ (کوئی بھی پیمائش، "اجازتی" قیمتوں میں سے کوئی ایک دے گی، اسی لئے انہیں اجازتی قیمتیں کہتے ہیں،  اور کوئی مخصوص قیمت \عددی{E_n} حاصل ہونے کا احتمال \عددی{\abs{c_n}^2} ہو گا۔)

یقیناً ان تمام احتمالات کا مجموعہ \عددی{1} ہو گا
\begin{align}
\sum_{n=1}^{\infty}\abs{c_n}^2=1
\end{align}
جس کا ثبوت \عددی{\Psi} کی عمود زنی سے حاصل ہو گا (چونکہ تمام \عددی{c_n} غیر تابع وقت ہیں لہٰذا میں \عددی{t=0} پر ثبوت پیش کرتا ہوں۔ آپ باآسانی اس ثبوت کو عمومیت دے کر کسی بھی \عددی{t} کے لئے ثبوت پیش کر سکتے ہیں)۔
\begin{align*}
1&=\int\abs{\Psi(x,0)}^2\dif x=\int \big(\sum_{m=1}^{\infty}c_m\psi_m(x)\big)^{\!*}\big(\sum_{n=1}^{\infty}c_n\psi_n(x)\big)\dif x\\
&=\sum_{m=1}^{\infty}\sum_{n=1}^{\infty}c_m^*c_n\int\psi_m(x)^*\psi_n(x)\dif x\\
&=\sum_{n=1}^{\infty}\sum_{m=1}^{\infty}c_m^*c_n\delta_{mn}=\sum_{n=1}^{\infty}\abs{c_n}^2
\end{align*}
(یہاں بھی \عددی{m} پر مجموعہ لینے میں کرونیکر ڈیلٹا جزو \عددی{m=n} کو چنتا ہے۔)

مزید، توانائی کی توقعاتی قیمت لازماً درج ذیل ہو گی  
\begin{align}
\langle H\rangle=\sum_{n=1}^{\infty} \abs{c_n}^2 E_n
\end{align} 
 جس کی بلا واسطہ تصدیق کی جا سکتی ہے: غیر تابع وقت شروڈنگر مساوات کہتی ہے
\begin{align}
H\psi_n=E_n\psi_n
\end{align}
لہٰذا درج ذیل ہو گا۔
\begin{align*}
\langle H \rangle &=\int \Psi^*H\Psi\dif x=\int \big(\sum c_m\psi_m\big)^* H\big(\sum c_n\psi_n\big)\dif x\\
&=\sum\sum c_m^*c_nE_n\int \psi_m^*\psi_n\dif x=\sum\abs{c_n}^2 E_n
\end{align*}
دھیان رہے کہ کسی ایک مخصوص توانائی کے حصول  کا احتمال غیر تابع وقت ہو گا اور یوں \عددی{H} کی توقعاتی قیمت بھی غیر تابع وقت ہو گی۔ کوانٹم میکانیات میں  \اصطلاح{بقا توانائی}\فرہنگ{بقا!توانائی}\حاشیہب{conservation of energy}\فرہنگ{energy!conservation} کی یہ ایک مثال ہے۔

\ابتدا{مثال}\شناخت{مثال_شروڈنگر_عددی_سر_توقعاتی}
ہم نے دیکھا  کہ مثال \حوالہء{2.2} میں ابتدائی تفاعل موج (شکل \حوالہء{2.3}) زمینی حال \عددی{\psi_{1}} (شکل \حوالہء{2.2}) کے ساتھ قریبی مشابہت رکھتا ہے۔ یوں ہم توقع کرتے گے کہ \عددی{\left| c_{1} \right|^{2}}  غالب ہو گا۔ یقیناً ایسا ہی ہے۔
\begin{align*}
\left| c_{1} \right|^{2} = \left( \frac{8\sqrt{15}}{\pi^{3}} \right)^{2} = 0.998555 \cdots
\end{align*}
باقی تمام عددی سر مل کر فرق دیتے ہیں:
\begin{align*}
\sum_{n=1}^{\infty} \left| c_{n} \right|^{2} = \left( \frac{8\sqrt{15}}{\pi^{3}} \right)^{2} \sum_{n=1,3,5,...}^{\infty} \frac{1}{n^{6}} = 1
\end{align*}
اس مثال میں توانائی کی توقعاتی قیمت ہماری توقعات کے عین مطابق درج ذیل ہے۔
\begin{align*}
\langle H \rangle = \sum_{n=1,3,5,...}^{\infty} \left( \frac{8\sqrt{15}}{n^{3} \pi^{3}} \right)^{2} \frac{n^{2} \pi^{2} \hslash^{2}}{2ma^{2}} = \frac{480\hslash^{2}}{\pi^{4} ma^{2}} \sum_{n=1,3,5,...}^{\infty} \frac{1}{n^{4}} = \frac{5 \hslash^{2}}{ma^{2}}
\end{align*}
 یہ \عددی{ E_{1} = \pi^{2} \hslash^{2}/2ma^{2}  } کے بہت قریب،  ہیجان حل حالتوں کی شمول کی بنا معمولی زیادہ ہے۔ 
\انتہا{مثال}
%===========
\ابتدا{سوال}\شناخت{سوال_شروڈنگر_حل_ناقابل_قبول}
دکھائیں کہ لا متناہی چکور  کنواں کے لئے  \عددی{ E = 0 } یا \عددی{ E < 0 } کی صورت میں  غیر تابع وقت شروڈنگر مساوات کا کوئی بھی قابل قبول حل  نہیں پایا جاتا ہے۔ (یہ سوال \حوالہء{2.2} میں دیے گئے عمومی مسئلے کی ایک خصوصی صورت ہے، لیکن اس بار شروڈنگر مساوات کو صریحاً حل کرتے ہوئے دکھائیں کہ آپ سرحدی شرائط پر پورا نہیں اتر سکتے ہیں۔)
 \انتہا{سوال}
%==============
\ابتدا{سوال}
لامتناہی چکور  کنواں کے \عددی{ n } وی ساکن حال کیلئے \عددی{\langle x \rangle}، \عددی{\langle x^2 \rangle} ، \عددی{\langle p \rangle}، \عددی{\langle p^2 \rangle}، \عددی{\sigma_x}  اور \عددی{\sigma_p}  تلاش کریں۔ تصدیق کریں کہ اصول غیر یقینیت مطمئن ہوتا ہے۔ کونسا حال غیر یقینیت کی حد کے قریب ترین ہو گا؟
\انتہا{سوال}
%=============
\ابتدا{سوال}\شناخت{سوال_شروڈنگر_لامتناہی_کنواں_برابر_حصے}
لامتناہی چکور  کنواں میں ایک ذرے کا ابتدائی تفاعل موج اولین دو ساکن حالات کے برابر حصوں کا مرکب ہے۔ 
\begin{align*}
\Psi(x,0) = A[\psi_{1}(x) + \psi_{2}(x)]
\end{align*}
\begin{enumerate}[a.]
\item 
\عددی{ \Psi(x,0) } کو معمول پر لائیں۔ (یعنی \عددی{ A } تلاش کریں۔ آپ \عددی{ \psi_{1} } اور \عددی{ \psi_{2} } کی معیاری عمودیت بروئے کار لاتے ہوئے با آسانی ایسا کر سکتے ہیں۔ یاد رہے کہ \عددی{ t=0 } پر \عددی{  \Psi  } کو معمول پر لانے کے بعد آپ یقین رکھ سکتے  ہیں کہ یہ معمول شدہ ہی رہے گا۔ اگر آپ کو شک ہے، جزو-ب کا نتیجہ حاصل کرنے کے بعد  اس کی صریحاً تصدیق کریں۔) 
\item
\عددی{ \Psi(x,t) } اور \عددی{ \left| \Psi (x,t) \right|^{2} } تلاش کریں۔ موخر الذکر  کو وقت کے سائن نما تفاعل کی صورت میں لکھیں، جیسا مثال \حوالہء{2.1} میں کیا گیا۔ نتائج کو سادہ صورت میں لکھنے کی خاطر \عددی{\omega\equiv\tfrac{\pi^2\hslash}{2ma^2}} لیں۔ 
\item 
\عددی{ \langle x \rangle  } تلاش کریں۔ آپ دیکھیں گے کہ یہ وقت کے ساتھ ارتعاش کرتا ہے۔ اس ارتعاش کی زاویائی تعدد کتنی ہو گی؟ ارتعاش کا حیطہ کیا ہو گا؟ (اگر آپ کا حیطہ \عددی{ \tfrac{a}{2}} سے زیادہ ہو تب آپ کو جیل بھیجنے کی ضرورت ہو گی۔) 
\item 
\عددی{ \langle p \rangle  } تلاش کریں (اور اس پہ زیادہ وقت صرف نہ کریں)۔ 
\item
اس ذرے کی توانائی کی پیمائش سے کون کون سی قیمتیں متوقع ہیں؟ اور ہر ایک قیمت کا احتمال کتنا ہو گا؟ \عددی{ H } کی توقعاتی قیمت تلاش کریں۔ اس کی قیمت کا موازنہ \عددی{ E_{1} } اور \عددی{ E_{2} } کے ساتھ کریں؟
\end{enumerate}
\انتہا{سوال}
%================
\ابتدا{سوال}
اگرچہ تفاعل موج کا مجموعی زاویائی مستقل کسی با معنی طبعی اہمیت کا حامل نہیں ہے (چونکہ یہ کسی بھی قابل پیمائش مقدار میں کٹ جاتا ہے) لیکن  مساوات \حوالہ{مساوات_شروڈنگر_عمومی_حل_مجموعہ} میں عددی سروں کے اضافی زاویائی مستقل اہمیت کے حامل ہیں۔ مثال کے طور پر ہم سوال \حوالہ{سوال_شروڈنگر_لامتناہی_کنواں_برابر_حصے} میں  \عددی{ \psi_{1} } اور \عددی{ \psi_{2} } کے اضافی زاویائی مستقل تبدیل کرتے ہیں:
\begin{align*}
\Psi (x,0) = A[\psi_{1} (x) + e^{i\phi}\psi_{2}(x)]
\end{align*}
جہاں \عددی{ \phi } کوئی مستقل ہے۔ \عددی{  \Psi(x,t) }، \عددی{  \ \left| \Psi (x,t) \right|^{2} } اور \عددی{ \langle x \rangle } تلاش کر کے ان کا موازنہ پہلے حاصل شدہ نتائج کے ساتھ کریں۔ بالخصوص \عددی{ \phi = \pi/2 } اور  \عددی{ \phi = \pi } کی صورتوں پر غور کریں۔ 
\انتہا{سوال}
%===========
\ابتدا{سوال} 
لا متناہی چکور  کنواں میں ایک ذرے کا ابتدائی تفاعل موج درج ذیل ہے۔
\begin{align*}
\Psi (x,0) = 
\begin{cases}
Ax, & 0 \leq x \leq a/2 \\ 
A(a-x), & a/2 \leq x \leq a
\end{cases}
\end{align*}
\begin{enumerate}[a.]
\item 
\عددی{\Psi(x,0)} کا خاکہ کھینچیں اور مستقل \عددی{A} کی قیمت تلاش کریں۔
\item  
\عددی{\Psi(x,t) } تلاش کریں۔
\item  
توانائی کی پیمائش کا نتیجہ \عددی{E_{1}} ہونے کا احتمال کتنا ہو گا؟
\item 
توانائی کی توقعاتی قیمت تلاش کریں۔
\end{enumerate}
\انتہا{سوال}
%===========
\ابتدا{سوال}
ایک لامتناہی چکور  کنواں، جس کی چوڑائی \عددی{a} ہے، میں کمیت \عددی{m} کا ایک ذرہ کنواں کے بائیں حصے سے ابتدا ہوتا ہے اور یہ \عددی{t=0} پر بائیں نصف حصے کے کسی بھی نقطے پر ہو سکتا ہے۔
\begin{enumerate}[a.]
\item
اس کی ابتدائی تفاعل موج \عددی{\Psi(x,0) } تلاش کریں۔ (فرض کریں کے یہ حقیقی ہے اور اسے معمول پر لانا نا بھولیے گا۔)
\item 
پیمائش توانائی کا نتیجہ \عددی{\pi^{2}\hslash^{2}/2ma^{2}} ہونے کا احتمال کیا ہو گا؟ 
\end{enumerate} 
\انتہا{سوال}
%==========
\ابتدا{سوال}
لمحہ \عددی{t=0}  پر مثال \حوالہء{2.2} کے تفاعل موج کیلئے \عددی{H} کی توقعاتی قیمت تکمل کے ذریعہ حاصل کریں۔ 
\begin{align*}
\langle H \rangle=\int \Psi(x,0)^*\hat{H}\, \Psi(x,0)\dif x
\end{align*}
مثال \حوالہ{مثال_شروڈنگر_عددی_سر_توقعاتی} میں مساوات \حوالہء{2.39} کی مدد سے حاصل کردہ نتیجے کے ساتھ موازنہ کریں۔ دھیان رہے کیونکہ \عددی{H}  غیر تابع وقت ہے لہٰذا \عددی{t=0} لینے سے نتیجے پر کوئی اثر نہیں ہو گا۔ 
\انتہا{سوال}
%=========================

\حصہ{ہارمونی مرتعش}
کلاسیکی ہارمونی مرتعش ایک لچک دار اسپرنگ جس کا مقیاس لچک \عددی{k} ہو  اور  کمیت \عددی{m}  پر مشتمل ہوتا ہے۔ کمیت کی حرکت \اصطلاح{قانون ہک}\فرہنگ{قانون!ہک}\حاشیہب{Hooke's law}\فرہنگ{law!Hooke} 
\begin{align*}
F=-kx=m\frac{\dif{^{2}x}}{\dif{t^{2}}}
\end{align*}
کے تحت ہو گی جہاں رگڑ کو نظر انداز کیا گیا ہے۔ اس کا حل
\begin{align*}
x(t)=A\sin(\omega t)+B\cos(\omega t)
\end{align*}
ہو گا جہاں
\begin{align}\label{مساوات_شروڈنگر_زاویائی_تعدد}
\omega\equiv \sqrt{\frac{k}{m}}
\end{align}
ارتعاش کا (زاویائی) تعدد ہے۔ مخفی توانائی
\begin{align}
V(x)=\frac{1}{2}kx^{2}
\end{align}
ہو گی جس کی ترسیم قطع مکافی ہے۔ 

حقیقت میں کامل ہارمونی مرتعش نہیں پایا جاتا ہے۔ اگر آپ اسپرنگ کو زیادہ  کھینچیں تو وہ  ٹوٹ جائے گا اور قانون ہک اس سے بہت پہلے غیر کارآمد ہو چکا ہو گا۔ تاہم عملاً کوئی بھی مخفیہ، مقامی کم سے کم نقطہ کی پڑوس میں تخمیناً قطع مکافی  ہو گا (شکل \حوالہء{2.4})۔مخفی توانائی \عددی{V(x)} کے کم سے کم نقطہ \عددی{x_0} کے لحاظ سے \عددی{V(x)} کو \اصطلاح{ٹیلر تسلسل}\فرہنگ{تسلسل!ٹیلر}\حاشیہب{Taylor series}\فرہنگ{series!Taylor} کے لحاظ سے پھیلا کر
\begin{align*}
V(x)=V( x_{0})+V'(x_{0})(x-x_{0})+\frac{1}{2}V''(x_{0})(x-x_{0})^{2}+\cdots
\end{align*}
اس سے \عددی{V(x_0)} منفی کر کے (ہم \عددی{V(x)} سے کوئی بھی مستقل بغیر خطر و فکر منفی کر سکتے ہیں کیونکہ ایسا کرنے  سے قوت تبدیل نہیں ہو گا)  اور یہ جانتے ہوئے کہ \عددی{V'(x_0)=0} ہو گا (چونکہ \عددی{x_0} کم سے کم نقطہ ہے)، ہم تسلسل کے بلند رتبی ارکان  رد کرتے ہوئے (جو \عددی{(x-x_0)} کی قیمت کم ہونے کی صورت میں قابل نظرانداز ہونگے) درج ذیل حاصل کرتے ہیں
\begin{align*}
V(x)\cong\frac{1}{2}V''(x_{0})(x-x_{0})^{2}
\end{align*}
جو نقطہ \عددی{x_0} پر ایک ایسی سادہ  ہارمونی ارتعاش  بیان کرتا ہے  جس کا موثر مقیاس لچک \عددی{k=V''(x_0)} ہو۔ یہی وہ وجہ ہے جس کی بنا سادہ ہارمونی مرتعش اتنا اہم ہے: تقریباً ہر وہ ارتعاشی حرکت جس کا حیطہ کم ہو تخمیناً سادہ ہارمونی ہو گا۔

کوانٹم میکانیات میں ہمیں مخفیہ
\begin{align}\label{مساوات_شروڈنگر_مخفیہ_ہارمونی}
V(x)=\frac{1}{2}m\omega ^{2}x^{2}
\end{align}
کے لیے شروڈنگر  مساوات حل کرنی ہو گی (جہاں روایتی طور پر مقیاس لچک کی جگہ کلاسیکی تعدد (مساوات \حوالہ{مساوات_شروڈنگر_زاویائی_تعدد}) استعمال کی جاتی ہے)۔  جیسا کہ ہم دیکھ چکے ہیں، اتنا کافی ہو گا کہ  ہم  غیر تابع وقت شروڈنگر مساوات
\begin{align}\label{مساوات_شروڈنگر_مخفی_قوہ_الف}
\frac{-\hslash ^{2}}{2m}\frac{\dif{^{2}\psi}}{\dif{x^{2}}}+\frac{1}{2}m\omega ^{2}x^{2}\psi=E\psi
\end{align}
حل کریں۔ اس مسئلے کو حل کرنے کے لیے دو بالکل مختلف طریقے اپنائے جاتے ہیں۔ پہلی میں  تفرقی مساوات کو "طاقت کے بل بوتے پر"   \اصطلاح{طاقتی تسلسل}\فرہنگ{تسلسل!طاقتی}\حاشیہب{power series}\فرہنگ{series!power} کے ذریعہ حل کرنے کی ترکیب استعمال کی جاتی ہے، جو دیگر مخفیہ کے لیے بھی کارآمد ثابت ہوتا ہے  (اور جسے استعمال کرتے ہوئے ہم باب \حوالہ{باب_تین_ابعادی_کوانٹم_میکانیات} میں کولمب مخفیہ کے لیے حل تلاش کریں گے)۔ دوسری ترکیب ایک شیطانی الجبرائی تکنیک ہے جس میں \اصطلاح{عاملین سیڑھی} استعمال ہوتے ہیں۔ میں آپ کی واقفیت پہلے الجبرائی تکنیک کے ساتھ پیدا کرتا ہوں جو زیادہ سادہ، زیادہ دلچسپ (اور جلد حل دیتا) ہے۔ اگر آپ طاقتی تسلسل کی ترکیب یہاں استعمال نہ کرنا چاہیں تو آپ ایسا کر سکتے ہیں لیکن کہیں نہ کہیں  آپکو یہ ترکیب سیکھنی ہو گی۔

\جزوحصہ{الجبرائی ترکیب}
ہم مساوات \حوالہ{مساوات_شروڈنگر_مخفی_قوہ_الف} کو زیادہ معنی خیز روپ میں لکھ کر ابتدا کرتے ہیں
\begin{align}
\frac{1}{2m}[p^{2}+(m\omega x)^{2}]\psi=E\psi
\end{align}
جہاں \عددی{p\equiv \frac{\hslash}{i}\frac{d}{\dif{x}}} معیار حرکت کا عامل ہے۔ بنیادی طور پر  ہیملٹنی
\begin{align}
H=\frac{1}{2m}[p^{2}+(m\omega x)^{2}]
\end{align}
کو  کو \ترچھا{اجزائے ضربی} لکھنے کی ضرورت ہے۔اگر یہ عداد ہوتے تب ہم یوں لکھ سکتے تھے۔
\begin{align*}
u^{2}+v^{2}=(iu+v)(-iu+v)
\end{align*}
البتہ  یہاں بات اتنی سادہ نہیں ہے چونکہ \عددی{p} اور \عددی{x} \ترچھا{عاملین} ہیں اور  عاملین عموماً  \اصطلاح{قابل تبادل} نہیں ہوتے ہیں (یعنی آپ \عددی{xp} سے مراد \عددی{px} نہیں لے سکتے ہیں)۔ اس کے باوجود یہ ہمیں درج ذیل مقداروں پر غور کرنے پر آمادہ  کرتا ہے
\begin{align}\label{مساوات_شروڈنگر_تعریفات_سیڑھی}
a\pm\equiv \frac{1}{\sqrt{2\hslash m\omega}}(\mp ip+m\omega x)
\end{align}
(جہاں قوسین کے باہر جزو ضربی لگانے سے آخری نتیجہ خوبصورت نظر آئے گا)۔

آئیں دیکھیں حاصل ضرب \عددی{a_{-}a_{+}}  کیا ہو گا؟
\begin{align*}
a_{-}a_{+}&=\frac{1}{2\hslash m\omega}(ip+m\omega x)(-ip+m\omega x)\\
&=\frac{1}{2\hslash m\omega}[p^{2}+(m\omega x)^{2}-im\omega(xp-px)]
\end{align*}
اس میں متوقع اضافی جزو \عددی{(xp-px)} پایا جاتا ہے جس کو ہم \عددی{x} اور \عددی{p} کا  \اصطلاح{تبادل کار}\فرہنگ{تبادل کار}\حاشیہب{commutator}\فرہنگ{commutator} کہتے ہیں اور جو ان کی آپس میں قابل تبادل نہ  ہونے کی پیمائش ہے۔ عمومی طور پر عامل \عددی{A} اور عامل \عددی{B} کا تبادل کار (جسے چکور  قوسین میں لکھا ہے) درج ذیل ہو گا۔
\begin{align}
[A,B]\equiv AB-BA
\end{align}
اس علامتیت کے تحت درج ذیل ہو گا۔
\begin{align}\label{مساوات_شروڈنگر_سیڑھی_حاصل_ضرب}
a_{-}a_{+}=\frac{1}{2\hslash m\omega}[p^{2}+(m\omega x)^{2}]-\frac{i}{2\hslash}[x,p]
\end{align}

ہمیں \عددی{x} اور  عددی{p} کا تبادل کار دریافت کرنا ہو گا۔ \ترچھا{انتباہ:} عاملین پر ذہنی  کام کرنا  عموماً غلطی کا سبب بنتا ہے۔ بہتر ہو گا کہ  عاملین پرکھنے کے لیے آپ انہیں تفاعل \عددی{f(x)} عمل کرنے کے لئے پیش کریں۔ آخر میں اس پرکھی تفاعل کو رد کر کے آپ صرف عاملین پر مبنی مساوات حاصل کر سکتے ہیں۔ موجودہ صورت میں درج ذیل ہو گا۔
\begin{align}
[x,p]f(x)=\big[x\frac{\hslash}{i}\frac{d}{\dif{x}}(f)-\frac{\hslash}{i}\frac{d}{\dif{x}}(xf)\big ]=\frac{\hslash}{i}\big (x\frac{\dif{f}}{\dif{x}}-x\frac{\dif{f}}{\dif{x}}-f\big )=-i\hslash f(x)
\end{align}
پرکھی تفاعل (جو اپنا کام کر چکا) کو رد کرتے ہوئے درج ذیل ہو گا۔
\begin{align}
[x,p]=i\hslash
\end{align}
یہ خوبصورت نتیجہ جو بار بار سامنے آتا ہے \اصطلاح{باضابطہ تبادلی رشتہ}\فرہنگ{تبادلی!باضابطہ رشتہ}\حاشیہب{canonical commutation relation}\فرہنگ{commutation!canonical relation} کہلاتا ہے۔

اسے کے استعمال سے مساوات \حوالہ{مساوات_شروڈنگر_سیڑھی_حاصل_ضرب} درج ذیل روپ  
\begin{align}
a_{-}a_{+}=\frac{1}{\hslash\omega}H+\frac{1}{2}
\end{align}
یا
\begin{align}
H=\hslash\omega\big (a_{-}a_{+}-\frac{1}{2} \big )
\end{align}
اختیار کرتی ہے۔ آپ نے دیکھا کہ ہیملٹنی کو ٹھیک اجزائے ضربی کی صورت میں نہیں لکھا جا سکتا  اور دائیں ہاتھ اضافی \عددی{-\tfrac{1}{2}} ہو گا۔ یاد رہے گا یہاں \عددی{a_+} اور \عددی{a_-} کی ترتیب بہت اہم ہے۔ اگر آپ \عددی{a_+} کو بائیں طرف رکھیں تو درج ذیل حاصل ہو گا۔
\begin{align}
a_{+}a_{-}=\frac{1}{\hslash\omega}H-\frac{1}{2}
\end{align}
بالخصوص درج ذیل ہو گا۔
\begin{align}\label{مساوات_شروڈنگر_سیڑھی_ضرب_برابر_اکائی}
[a_{-},a_{+}]=1
\end{align}
یوں ہیملٹنی کو درج ذیل بھی لکھا جا سکتا ہے۔
\begin{align}
H=\hslash\omega\big (a_{+}a_{-}+\frac{1}{2}\big )
\end{align}


ہارمونی مرتعش کی شروڈنگر مساوات کو \عددی{a_{\pm}} کی صورت میں  درج ذیل لکھا جا سکتا ہے۔
\begin{align}\label{مساوات_شروڈنگر_عوامل_روپ}
 \hslash \omega \big(a_{\pm} a_{\mp} \,\pm \frac{1}{2}\big)=E\psi  		
\end{align}
(اس  طرح کی مساوات میں آپ بالائی علامتیں ایک ساتھ پڑھتے ہو یا زیریں علامتیں ایک ساتھ پڑھتے ہو۔)

 ہم ایک اہم موڑ پر ہیں۔ میں دعویٰ کرتا ہوں  اگر    توانائی \عددی{E} کی شروڈنگر  مساوات کو \عددی{ \psi } مطمئن کرتا ہو 
 \عددی{ ( H \psi = E \psi ) } تب  توانائی \عددی{(E+\hslash \omega)} کی شروڈنگر  مساوات کو \عددی{a_+ \psi }  مطمئن کرے گا:  \عددی{  H ( a_+ \psi ) = ( E + \hslash \omega) ( a_+ \psi ) }

ثبوت:
\begin{align*}
H ( a_+ \psi ) &= \hslash \omega ( a_+ a_- + \frac{1}{2}) (a_+ \psi) = \hslash \omega (a_+ a_- a_+ +  \frac{1}{2}  a_+) \psi
\\
  &= \hslash \omega a_+ ( a_- a_+ + \tfrac{1}{2} ) \psi  = a_+\big[ \hslash \omega ( a_+ a_- + 1 + \tfrac{1}{2} ) \psi \big ]\\
  &= a_+ ( H + \hslash \omega ) \psi = a_+ ( E + \hslash \omega ) \psi = ( E + \hslash \omega ) ( a_+ \psi )
\end{align*}  
(میں نے دوسری لکیر میں مساوات \حوالہ{مساوات_شروڈنگر_سیڑھی_ضرب_برابر_اکائی} استعمال کرتے ہوئے  \عددی{ a_- a_+ }
 کی جگہ \عددی{ a_+ a_- + 1}   استعمال کیا ہے۔ دھیان  رہے اگرچہ \عددی{ a_+}  اور \عددی{  a_- }  کی ترتیب اہمیت کا حامل ہے،  
\عددی{ a \pm }   اور کسی بھی مستقل، مثلاً \عددی{ \hslash }، \عددی{ \omega} اور E،   کی ترتیب اہم نہیں ہے۔ ایک عامل ہر مستقل کے ساتھ قابل تبادل ہو گا۔)

 اسی طرح حل \عددی{ a_- \psi }  کی  توانائی \عددی{ ( E - \hslash \omega) } ہو گی۔
\begin{align*}
H(a_- \psi ) &= \hslash \omega ( a_- a_+   - \tfrac{1}{2}) (a_- \psi)    = \hslash \omega a_- \,( a_+ a_-  - \tfrac{1}{2} ) \psi\\
&= a_- \big[ \hslash \omega ( a_- a_+  - 1 - \tfrac{1}{2} ) \psi \big] = a_- ( H - \hslash \omega )\psi = a_- ( E - \hslash \omega) \psi \\	
&=( E - \hslash \omega ) ( a_- \psi )
\end{align*}
یوں ہم نے ایک ایسی خودکار ترکیب دریافت کر لی ہے  جس سے، کسی ایک حل کو جانتے ہوئے،  بالائی اور زیریں توانائی کے نئے حل دریافت کیے جا سکتے ہیں۔ چونکہ \عددی{ a \pm } کے ذریعے ہم توانائی میں اوپر چڑھ  یا نیچے اتر سکتے ہیں لہٰذا انہیں ہم \اصطلاح{عاملین سیڑھی}\فرہنگ{سیڑھی!عاملین}\حاشیہب{ladder operators}\فرہنگ{ladder!operators} پکارتے ہیں: \عددی{ a_+ } \اصطلاح{عامل رفعت}\فرہنگ{عامل!رفعت}\حاشیہب{raising operator}\فرہنگ{operator!raising} اور  \عددی{ a_- } \اصطلاح{عامل تقلیل}\فرہنگ{عامل!تقلیل}\حاشیہب{lowering operator}\فرہنگ{operator!lowering} ہے۔ حالات کی "سیڑھی" کو شکل \حوالہء{2.5} میں دکھایا گیا ہے۔

ذرا رکیے! عامل تقلیل کے بار بار استعمال سے آخرکار  ایسا حل حاصل ہو گا جس کی توانائی صفر سے کم ہو گی ( جو سوال \حوالہء{2.2} میں پیش عمومی مسئلہ کے تحت  نا ممکن ہے۔)  نئے حالات  حاصل کرنے کی خودکار ترکیب  کسی نہ کسی نقطہ پر  لازماً ناکامی کا شکار ہو گی۔  ایسا کیوں کر ہو گا؟ ہم جانتے ہیں کہ  \عددی{ a_- \psi }  شروڈنگر   مساوات کا ایک نیا حل ہو گا، تاہم اس کی ضمانت نہیں دی جا  سکتی  ہے کہ یہ معمول پر لانے کے قابل بھی ہو گا؛  یہ صفر ہو سکتا ہے  یا اس کا مربعی تکمل لامتناہی ہو سکتا ہے۔ عملاً اول الذکر ہو گا: سیڑھی کے سب سے نچلے پایہ (جس کو ہم  \عددی{ \psi_0 } کہتے ہیں) پر درج ذیل ہو گا۔  
\begin{align}
 a_- \psi_0 = 0 
\end{align}  
اس کو استعمال کرتے ہوئے ہم \عددی{ \psi_0 (x) } تعین کر سکتے ہیں:
\begin{align*}
\frac{1}{\sqrt{2 \hslash m \omega}} ( \hslash \frac{\dif }{\dif x} + m \omega x ) \psi_0 = 0 
\end{align*}
سے تفرقی مساوات
\begin{align*}
\frac{ \dif  \psi_0}{\dif x } = -\frac{m \omega }{ \hslash } x \psi_0
\end{align*}
لکھی جا سکتی ہے جسے باآسانی حل کیا جا سکتا ہے: 
\begin{align*}
\int{\frac{ \dif \psi_0 }{ \psi_0 } } &= -\frac{ m \omega }{ \hslash } \int{ x \dif x } \implies \ln \psi_0 = 
-\frac{m \omega }{ 2 \hslash } x^2 +C  
\end{align*} 
(\عددی{C} مستقل ہے۔) لہٰذا درج ذیل ہو گا۔
\begin{align*}
\psi_0 (x) = A e^{\frac{ - m \omega }{ 2 \hslash } x^2}
\end{align*}
ہم اس کو یہیں معمول پر لاتے ہیں: 
\begin{align*}
1 = \abs{A}^2 \int_{ - \infty}^{ \infty } e^{- m \omega x^2/ \hslash}  \dif x = \abs{A}^2  \sqrt{ \frac{ \pi \hslash }{ m \omega }}
\end{align*}
لہٰذا \عددی{A^2 = \sqrt{ \frac{ m \omega }{ \pi \hslash }}} اور درج ذیل ہو گا۔
\\
\begin{align}\label{مساوات_شروڈنگر_معمول_شدہ_حال_صفر}
\psi_0 (x) = \big(\frac{ m \omega }{ \pi  \hslash}\big)^{1/4} e^{-\frac{ m \omega }{ 2 \hslash} x^2}
\end{align}
اس حال کی توانائی دریافت کرنے کی خاطر ہم اس کو (مساوات \حوالہ{مساوات_شروڈنگر_عوامل_روپ} روپ کی) شروڈنگر  مساوات میں پر کر کے
\begin{align*}
\hslash \omega (a_+ a_- + \tfrac{1}{2} )\psi_0 = E_0 \psi_0 
\end{align*}
 یہ جانتے ہوئے کہ \عددی{ a_- \psi_0 = 0 } ہو گا درج ذیل حاصل کرتے ہیں۔
\begin{align}
 E_0 = \frac{1}{2} \hslash \omega 
\end{align}
سیڑھی کے نچلا پایہ (جو کوانٹم مرتعش کا زمینی حال ہے)  پر پیر رکھ کر،   بار بار عامل رفعت استعمال کر کے ہیجان حالات دریافت کیے جا سکتے ہیں\حاشیہد{ہارمونی مرتعش کی صورت میں روایتی طور پر، عمومی طریقہ کار سے ہٹ کر، حالات کی شمار \عددی{n=1} کی بجائے \عددی{n=0} سے شروع کی جاتی ہے۔ظاہر ہے ایسی صورت میں مساوات \حوالہ{مساوات_شروڈنگر_عمومی_حل_مجموعہ} طرز کی مساواتوں میں مجموعہ کی زیریں حد کو بھی  تبدیل کیا جائے گا۔} جہاں ہر قدم پر توانائی میں \عددی{ \hslash \omega } کا اضافہ ہو گا۔
\begin{align}\label{مساوات_شروڈنگر_ہارمونی_حالات}
\psi_n (x) &= A_n ( a_+)^n \psi_0(x) , && E_n = ( n + \tfrac{1}{2}) \hslash \omega
\end{align}
یہاں \عددی{ A_n } مستقل معمول زنی ہے۔ یوں \عددی{ \psi_0 } پر عامل رفعت بار بار استعمال کرتے ہوئے ہم (اصولاً)  ہارمونی مرتعش کے تمام ساکن حالات دریافت کر سکتے ہیں۔ صریحاً ایسا کیے بغیر ہم تمام  اجازتی توانائیاں تعین کر پائے ہیں۔


\ابتدا{مثال}\شناخت{مثال_شروڈنگر_ہارمونی_پہلا_ہیجان_حال}
ہارمونی مرتعش کا پہلا ہیجان حال تلاش کریں۔ 

حل:\quad
 ہم مساوات \حوالہ{مساوات_شروڈنگر_ہارمونی_حالات} استعمال کرتے ہیں۔ 
\begin{gather}
\begin{aligned}\label{مساوات_شروڈنگر_ہارمونی_ہیجان_حالات}
\psi_1 (x) &= A_1 a_+ \psi_0 = \frac{ A_1 }{ \sqrt{ 2 \hslash m \omega } } \big( -\hslash \frac{\dif}{\dif x} + m \omega x\big) \big( \frac{ m \omega}{ \pi \hslash} \big)^{1/4}    e^{-\frac{  m \omega }{ 2 \hslash } x^2}\\
 &= A_1 \big( \frac{ m \omega }{ \pi \hslash}\big)^{1/4} \sqrt{ \frac{ 2 m \omega }{ \hslash} } x e^{-\frac{ m \omega }{ 2 \hslash } x^2}
\end{aligned}
\end{gather}
ہم اس کو قلم و کاغذ کے ساتھ  معمول پر لاتے ہیں۔ 
\begin{align*}
\int \abs{\psi_1}^2 \dif x = \abs{A_1 }^2 \sqrt{ \frac{ m \omega}{ \pi \hslash } } \big( \frac{ 2 m \omega }{ \hslash} \big ) \int_{ - \infty }^{ \infty } x^2  e^{-\frac{ m \omega }{ \hslash } x^2}  \dif x  =  \abs{A_1}^2
\end{align*} 
جیسا آپ دیکھ سکتے ہیں \عددی{ A_1 = 1 } ہو گا۔
 
اگرچہ میں پچاس مرتبہ عامل رفعت استعمال کر کے \عددی{ \psi_50 }  حاصل نہیں کرنا چاہوں گا، اصولی طور پر،   معمول زنی کے علاوہ،  مساوات \حوالہ{مساوات_شروڈنگر_ہارمونی_حالات} اپنا کام خوش اسلوبی سے کرتی ہے۔
\انتہا{مثال}
%=================

  آپ الجبرائی  طریقے سے ہیجان حالات کو معمول پر بھی لا سکتے ہیں لیکن اس کے لیے بہت محتاط چلنا ہو گا لہٰذا دھیان رکھیے  گا۔ ہم جانتے ہیں کہ 
\عددی{ a \pm \psi_n } اور \عددی{  \psi_{n\pm1} }  ایک دوسرے کے راست متناسب ہیں۔ 
\begin{align} 
a_+ \psi_n&= c_n  \psi_{n+1}  , &&  a_- \psi_n = d_n \psi_{n-1}
\end{align} 
 تناسبی مستقل \عددی{ c_n } اور  \عددی{ d_n } کیا ہوں گے؟  پہلے  جان لیں کہ کسی بھی تفاعلات \عددی{ f (x) } اور \عددی{ g(x) } کے لیے درج ذیل ہو گا۔(ظاہر ہے کہ تکملات کا موجود ہونا لازمی ہے، جس کا مطلب ہے کہ \عددی{\pm\infty} پر \عددی{f(x)} اور \عددی{g(x)} کو لازماً صفر پہنچنا  ہو گا۔)
\begin{align}\label{مساوات_شروڈنگر_سیڑھی_متبادل}
\int_{ - \infty }^{ \infty } f^*( a_{\pm} g ) \dif x = \int_{ - \infty }^{ \infty } ( a_{\mp} f)^* g \dif x
\end{align}
(خطی الجبرا کی زبان میں \عددی{ a \mp } اور \عددی{ a \pm } ایک دوسرے کے  \اصطلاح{ہرمشی جوڑی دار}\فرہنگ{ہرمشی!جوڑی دار}\حاشیہب{Hermitian conjugate}\فرہنگ{Hermitian!conjugate} ہیں۔)

ثبوت :
\begin{align*}
\int_{ - \infty}^{ \infty } f^*( a_{\pm}g ) \dif x = \frac{1}{ \sqrt{ 2 \hslash m \omega } } \int_{ - \infty }^{ \infty } f^*\big( \mp \hslash \frac{\dif}{\dif x} + m \omega x \big) g \dif x
\end{align*} 
تکمل بالحصص  کے ذریعے \عددی{ \int f^*( \tfrac{ \dif g }{ \dif x }) \dif x }
  سے \عددی{ -\int ( \tfrac{ \dif f }{ \dif x })^* g \dif x } حاصل ہو گا  (جہاں \عددی{\pm\infty} پر \عددی{f(x)} اور \عددی{g(x)} کی قیمتیں صفر تک پہنچنے کی بنا  سرحدی اجزاء صفر ہوں گے) لہٰذا 
\begin{align*}
\int_{ - \infty }^{ \infty } f^*( a_{\pm} g) \dif x &= \frac{1}{ \sqrt{ 2 \hslash m \omega } } \int_{ - \infty }^{ \infty } \big[ \big( \pm \hslash \frac{\dif}{\dif x} + m \omega x \big) f\big]^* g \dif x \\
&= \int_{ - \infty }^{ \infty } (a_\mp f )^* g \dif x
\end{align*}  
اور بالخصوص درج ذیل ہو گا۔
\begin{align*}
\int_{ - \infty }^{ \infty } ( a_\pm \psi_n )^*( a_\pm \psi_n ) \dif x = \int_{ - \infty }^{ \infty } ( a_\mp a_\pm \psi_n )^* \psi_n  \dif x
\end{align*}
 مساوات \حوالہ{مساوات_شروڈنگر_عوامل_روپ} اور مساوات \حوالہ{مساوات_شروڈنگر_ہارمونی_حالات} استعمال کرتے ہوئے 
\begin{align}\label{مساوات_شروڈنگر_سیڑھی_رفعت_تقلیل}
a_+ a_-  \psi_n &= n \psi_n , && a_- a_+ \psi_n = ( n+1 ) \psi_n 
\end{align}
ہو گا لہٰذا درج ذیل ہوں  گے۔ 
\begin{align*}
\int_{ - \infty }^{ \infty } ( a_+ \psi_n )^*( a_+ \psi_n ) \dif x &= \abs{c_n }^2 \int_{ - \infty }^{ \infty } \abs{ \psi_{n+1} }^2  \dif x  = ( n+1 ) \int_{ - \infty }^{ \infty } \abs{ \psi_n }^2 \dif x
\\
\int_{ - \infty }^{ \infty } ( a_- \psi_n )^*( a_- \psi_n ) \dif x &= \abs{ d_n }^2 \int_{ - \infty }^{ \infty } \abs{ \psi_{n-1} }^2  \dif x  =  n \int_{ - \infty }^{ \infty } \abs{ \psi_n }^2 \dif x
\end{align*}
چونکہ  \عددی{ \psi_n } اور  \عددی{ \psi_{ n \pm 1 } } معمول شدہ ہیں ، لہٰذا \عددی{ | c_n |^2 = n+1 } اور
\عددی{ | d_n |^2 = n } ہوں گے۔ یوں درج ذیل ہو گا۔
\begin{align}
a_+ \psi_n &= \sqrt{n+1} \,\psi_{n+1} ,&& a_- \psi_n = \sqrt{n} \,\psi_{n-1}
\end{align}
اس طرح درج ذیل ہوں گے۔
\begin{align*}
\psi_1 &= a_+ \psi_0 , \quad \psi_2 =\frac{1}{\sqrt{2}} a_+ \psi_1 = \frac{1}{ \sqrt{ 2 } } (a_+)^2 \psi_0,
\\
\psi_3 &= \frac{ 1 }{ \sqrt{ 3 } } a_+ \psi_2 = \frac{ 1 }{ \sqrt{3 \cdot 2} } ( a_+ )^3 \psi_0 ,\quad  \psi_4 = \frac{1}{ \sqrt{4} } a_+ \psi_3 =\frac{1}{ \sqrt{4 \cdot 3 \cdot 2} } ( a_+)^4 \psi_0,
\end{align*}
دیگر تفاعلات بھی اسی طرح حاصل کیے جا سکتے ہیں۔صاف ظاہر ہے کہ درج ذیل ہو گا۔ 
\begin{align}\label{مساوات_شروڈنگر_ہارمونی_ساکن_حالات}
\psi_n = \frac{1}{ \sqrt{n!} } ( a_+ )^n \psi_0
\end{align}
اس کے تحت مساوات \حوالہ{مساوات_شروڈنگر_ہارمونی_حالات} میں مستقل معمول  زنی \عددی{ A_n = \frac{1}{ \sqrt{ n! } } } ہو گا۔   (بالخصوص \عددی{ A_1 = 1 } ہو گا جو مثال \حوالہ{مثال_شروڈنگر_ہارمونی_پہلا_ہیجان_حال} میں ہمارے نتیجے کی تصدیق کرتا ہے۔)
 
لا متناہی  چکور  کنواں کے ساکن حالات کی طرح ہارمونی مرتعش کے ساکن حالات ایک دوسرے کے  عمودی ہیں۔ 
\begin{align}
\int_{ - \infty }^{ \infty } \psi_m^*\psi_n \dif x = \delta_{mn}
\end{align}
ہم ایک بار مساوات \حوالہ{مساوات_شروڈنگر_سیڑھی_رفعت_تقلیل} اور دو بار مساوات \حوالہ{مساوات_شروڈنگر_سیڑھی_متبادل} استعمال کر کے  پہلے  \عددی{ a_+ }  اور بعد میں \عددی{ a_-  } اپنی جگہ سے ہلا کر اس کا ثبوت پیش کر سکتے ہیں۔
\begin{align*}
\int_{ - \infty }^{ \infty } \psi_m^*(a_+ a_- )\psi_n \dif x &= n \int_{ - \infty }^{ \infty } \psi_m^* \psi_n \dif x
\\
&=\int_{ - \infty }^{ \infty } ( a_- \psi_m  )^*( a_- \psi_n ) \dif x = \int_{ - \infty }^{ \infty } ( a_+ a_- \psi_m)^* \psi_n \dif x\\
&= m \int_{ - \infty }^{ \infty } \psi_m^* \psi_n \dif x
\end{align*}
جب تک  \عددی{ m = n } نہ ہو  \عددی{ \int \psi_m^* \psi_n \dif x  } لازماً صفر ہو گا۔ معیاری عمودی ہونے کا مطلب  ہے کہ ہم 
\عددی{ \psi ( x , 0 ) } کو  ساکن حالات  کا خطی جوڑ (مساوات \حوالہ{مساوات_شروڈنگر_ساکن_حالات_کا_خطی_جوڑ}) لکھ کر خطی جوڑ  کے عددی سر  مساوات \حوالہ{مساوات_شروڈنگر_عددی_سر} سے حاصل کر سکتے ہیں اور   پیمائش سے توانائی کی قیمت  \عددی{ E_n } حاصل ہونے کا  احتمال \عددی{\abs{c_n}^2} ہو گا۔

\ابتدا{مثال}\شناخت{مثال_شروڈنگر_ہارمونی_مرتعش_مخفی_توقعاتی}
ہارمونی مرتعش کے \عددی{n} ویں حال کی مخفی توانائی کی توقعاتی قیمت تلاش کریں۔

حل:
\begin{align*}
\langle V  \rangle =\big\langle\frac{1}{2}m\omega ^{2}x^{2}\big\rangle=\frac{1}{2}m\omega ^{2}\int_{-\infty}^{\infty}\psi_{n}^{*}x^{2}\psi_{n}\dif{x}
\end{align*}
اس قسم کے تکملات جن میں \عددی{x} یا \عددی{p} کے طاقت پائے جاتے ہوں کے حصول کے لیے یہ ایک بہترین طریقہ کار ہے: متغیرات \عددی{x} اور \عددی{p} کو مساوات \حوالہ{مساوات_شروڈنگر_تعریفات_سیڑھی} میں پیش کی گئی تعریفات استعمال کرتے ہوئے  عاملین رفعت اور تقلیل کی روپ میں لکھیں:
\begin{align}
x&=\sqrt{\frac{\hslash}{2m\omega}}(a_++a_-); && p=i\sqrt{\frac{\hslash m \omega}{2}}(a_+-a_-)
\end{align}
اس مثال میں ہم \عددی{x^2} میں دلچسپی رکھتے ہیں:
\begin{align*}
x^{2}=\frac{\hslash}{2m\omega}[(a_{+})^{2}+(a_{+}a_{-})+(a_{-}a_{+})+(a_{-})^{2}] \\
\end{align*}
لہٰذا درج ذیل ہو گا۔
\begin{align*}
\langle V \rangle =\frac{\hslash\omega}{4}\int \psi_{n}^{*}\big[(a_{+})^{2}+(a_{+}a_{-})+(a_{-}a_{+})+(a_{-})^{2}\big ]\psi_{n}\dif{x}
\end{align*}
اب (ماسوائے معمول زنی کے) \عددی{(a_{+})^{2}\psi_{n}} تفاعل \عددی{\psi_{n+2}} کو ظاہر کرتا ہے  جو \عددی{\psi_n} کو عمودی ہے۔ یہی کچھ \عددی{(a_{-})^{2}\psi_{n}} کے بارے میں بھی کہا جا سکتا ہے جو \عددی{\psi_{n-2}} کا راست متناسب ہے۔ یوں یہ اجزاء خارج ہو جاتے ہیں، اور ہم مساوات \حوالہ{مساوات_شروڈنگر_سیڑھی_رفعت_تقلیل}  استعمال کر کے باقی دو کی قیمتیں حاصل کر سکتے ہیں:
\begin{align*}
\langle V \rangle =\frac{\hslash\omega}{4}(n+n+1)=\frac{1}{2}\hslash\omega\big (n+\frac{1}{2} \big )
\end{align*}
جیسا آپ نے دیکھا مخفی توانائی کی توقعاتی قیمت کل توانائی کی بالکل نصف ہے (باقی نصف حصہ یقیناً حرکی توانائی ہے)۔ جیسا ہم بعد میں دیکھیں گے یہ ہارمونی مرتعش کی ایک مخصوص خاصیت  ہے۔
\انتہا{مثال}
%%%
\ابتدا{سوال}\شناخت{سوال_شروڈنگر_عمودیت_اور_تیار}
\begin{enumerate}[a.]
\item
\عددی{\psi_{2}(x)} تیار کریں۔
\item
\عددی{\psi_{0}},\عددی{\psi_{1}}\عددی{\psi_{2}} کا خاکہ  کھینچیں۔
\item
\عددی{\psi_{0}},\عددی{\psi_{1}}\عددی{\psi_{2}} کی عمودیت کی تصدیق  تکمل لے کر صریحاً کریں۔ اشارہ:  تفاعلات کی جفت پن اور طاق پن کو بروئے کار لاتے ہوئے حقیقتاً صرف ایک تکمل حل کرنا ہو گا۔
\end{enumerate}
\انتہا{سوال}
%
\ابتدا{سوال}
\begin{enumerate}[a.]
\item
حالات \عددی{\psi_{0}} (مساوات \حوالہ{مساوات_شروڈنگر_معمول_شدہ_حال_صفر})   اور \عددی{\psi_{1}} (مساوات \حوالہ{مساوات_شروڈنگر_ہارمونی_ہیجان_حالات}) کے لئے صریح تکملات لے کر  \عددی{\langle x \rangle}، \عددی{ \langle p \rangle }، \عددی{ \langle x^{2} \rangle }، اور \عددی{\langle p^{2} \rangle } کی قیمتیں  دریافت کریں۔ \ترچھا{تبصرہ}: ہارمونی مرتعش کے مسائل میں متغیر \عددی{\xi\equiv\sqrt{m\omega/\hslash}x} اور مستقل 
\عددی{\alpha\equiv (m\omega/\pi\hslash)^{1/4}} متعارف کرتے ہوئے مسئلہ سادہ صورت اختیار کرتا ہے۔
\item
عدم یقینیت  کے حصول کو ان حالات کے لئے پرکھیں۔
\item
ان حالات کے لیے اوسط حرکی توانائی  \عددی{\langle T \rangle}اور اوسط مخفی توانائی \عددی{\langle V \rangle} کی قیمتیں حاصل کریں۔ (آپکو نیا تکمل حل کرنے کی اجازت نہیں ہے!) کیا ان کا مجموعہ آپ کی توقع کے مطابق ہے؟
\end{enumerate}
\انتہا{سوال}
%
\ابتدا{سوال}\شناخت{سوال_شروڈنگر_تصدیق_کریں}
ہارمونی مرتعش کے \عددی{n} ویں ساکن حال کے لئے مثال \حوالہ{مثال_شروڈنگر_ہارمونی_مرتعش_مخفی_توقعاتی}
کی ترکیب استعمال کرتے ہوئے \عددی{\langle x \rangle}، \عددی{ \langle p \rangle }، \عددی{ \langle x^{2} \rangle }،  \عددی{\langle p^{2} \rangle } اور \عددی{\langle T \rangle} تلاش کریں۔ تصدیق کریں کہ اصول عدم یقینیت مطمئن ہوتا ہے۔
\انتہا{سوال}
%
\ابتدا{سوال}
ہارمونی مرتعش مخفی قوہ میں ایک ذرہ درج ذیل حال سے ابتداء کرتا ہے۔
\begin{align*}
\Psi(x,0)=A[3\psi_{0}(x)+4\psi_{1}(x)]
\end{align*}
\begin{enumerate}[a.]
\item
\عددی{A} تلاش کریں۔
\item
\عددی{\Psi(x,t)} اور \عددی{\abs{\Psi(x,t)}^{2}} تیار کریں۔
\item
\عددی{\langle x \rangle} اور \عددی{\langle p \rangle} تلاش کریں۔ ان کے کلاسیکی تعدد پر ارتعاش پذیر ہونے پر حیران مت ہوں: اگر میں \عددی{\psi_{1}(x)} کی بجائے \عددی{\psi_{2}(x)} دیتا تب جواب کیا ہوتا؟  تصدیق کریں کہ اس تفاعل موج کے لیے مسئلہ اہرنفسٹ (مساوات \حوالہ{مساوات_تفاعل_موج_مخفی_توانائی_سے_معیار_حرکت})  مطمئن ہوتا ہے؟
\item
اس ذرے کی توانائی کی پیمائش میں کون کون سی قیمتیں متوقع ہیں اور ان کا احتمال کیا ہوں گے؟
\end{enumerate}
\انتہا{سوال}
%
\ابتدا{سوال}
ہارمونی مرتعش کے زمینی حال میں ایک ذرہ کلاسیکی تعدد \عددی{\omega} پر ارتعاش پذیر ہے۔ ایک دم مقیاس لچک \عددی{4} گنا ہو جاتا ہے لہٰذا \عددی{\omega^{\prime}=2\omega} ہو گا جبکہ ابتدائی تفاعل موج تبدیل نہیں ہو گا (یقیناً ہیملٹنی تبدیل ہونے کے بنا \عددی{\Psi}
اب مختلف انداز سے  ارتقا پائے گا)۔ اس کا احتمال کتنا ہے کہ توانائی کی پیمائش اب بھی \عددی{\hslash \omega/2}  قیمت دے؟  پیمائشی نتیجہ \عددی{\hslash\omega} حاصل ہونے کا احتمال کیا ہو گا؟
\انتہا{سوال}
%%%%%%%%%%%%%%%%
\جزوحصہ{تحلیلی ترکیب}\شناخت{حصہ_شروڈنگر_تحلیلی_ترکیب}
ہم اب ہارمونی مرتعش کی شروڈنگر مساوات کو دوبارہ لوٹ کر
\begin{align}\label{مساوات_شروڈنگر_تحلیلی_حل_الف}
-\frac{\hslash^{2}}{2m}\frac{\dif^{\,2}\psi}{\dif x^{2}}+\frac{1}{2}m\omega^{2}x^{2}\psi=E\psi
\end{align}
اور اس تو تسلسل کی ترکیب سے بلا واسطہ  حل کرتے ہیں۔ درج ذیل غیر بعدی متغیر متعارف کرنے سے چیزیں کچھ صاف نظر آتی ہیں۔
\begin{align}
\xi=\sqrt{\frac{m\omega}{\hslash}}x
\end{align}
شروڈنگر مساوات اب درج ذیل روپ اختیار کرتی ہے۔
\begin{align}\label{مساوات_شروڈنگر_تحلیلی_شروڈنگر}
\frac{\dif^{\,2}\psi}{\dif \xi^{2}}=(\xi^{2}-K)\psi
\end{align}
جہاں \عددی{K} توانائی ہے جس کی اکائی \عددی{\tfrac{1}{2}\hslash\omega} ہے۔
\begin{align}\label{مساوات_شروڈنگر_تحلیلی_مستقل}
K\equiv \frac{2E}{\hslash\omega}
\end{align}
ہم نے مساوات \حوالہ{مساوات_شروڈنگر_تحلیلی_شروڈنگر} کو حل کرنا ہو گا۔ ایسا کرتے ہوئے ہمیں \عددی{K} اور (یوں \عددی{E})  کی "اجازتی" قیمتیں بھی حاصل ہوں گی۔ 

ہم  اس صورت سے شروع کرتے ہیں جہاں \عددی{\xi} کی قیمت (یعنی \عددی{x} کی قیمت) بہت بڑی ہو۔ ایسی صورت میں
\عددی{\xi^{2}} کی قیمت \عددی{K} کی قیمت سے بہت زیادہ ہو گی لہٰذا مساوات \حوالہ{مساوات_شروڈنگر_تحلیلی_شروڈنگر} درج ذیل روپ اختیار کرے گی
\begin{align}
\frac{\dif^{\,2}\psi}{\dif \xi^{2}}\approx \xi^{2}\psi
\end{align}
جس کا تخمینی حل درج ذیل ہے (اس کی تصدیق کیجیے گا)۔ 
\begin{align}
\psi(\xi)\approx Ae^{-\xi^{2}/2}+Be^{+\xi^{2}/2}
\end{align}
 اس میں \عددی{B} کا جزو معمول پر لانے کے قابل نہیں ہے  (چونکہ \عددی{\abs{x}\to \infty}  کرنے سے اس کی قیمت بے قابو بڑھتی ہے)۔ طبی طور پر قابل قبول حل درج ذیل متقارب صورت کا ہو گا۔
\begin{align}
\psi(\xi)&\to(\quad )e^{-\xi^{2}/2}&&\text{\RL{($\xi$ کی بڑی قیمت کے لئے)}}
\end{align}
اس سے ہمیں خیال آتا ہے کہ ہمیں قوت نما حصہ کو "چھیلنا"  چاہیے،
\begin{align}\label{مساوات_شروڈنگر_متقارب_الف}
\psi(\xi)=h(\xi)e^{-\xi^{2}/2}
\end{align}
اور توقع کرنی چاہیے کہ جو کچھ باقی رہ جائے، \عددی{h(\xi)}، اس کی  صورت \عددی{\psi(\xi)} سے سادہ ہو۔\حاشیہد{اگرچہ ہم نے مساوات \حوالہ{مساوات_شروڈنگر_متقارب_الف} لکھتے ہوئے تخمین سے کام لیا، اس کے بعد باقی تمام بالکل ٹھیک ٹھیک ہے۔ تفرقی مساوات کے طاقتی تسلسل حل میں متقاربی جزو کا چھیلنا  عموماً پہلا قدم ہوتا ہے۔} ہم مساوات \حوالہ{مساوات_شروڈنگر_متقارب_الف} کے تفرقات
\begin{align*}
\frac{\dif{\psi}}{\dif{\xi}}=\big(\frac{\dif{h}}{\dif{\xi}}-\xi h\big )e^{-\xi^{2}/2}
\end{align*}
اور
\begin{align*}
\frac{\dif^{\,2}{\psi}}{\dif{\xi^{\,2}}}=\big (\frac{\dif^{\,2}{h}}{\dif{\xi^{\,2}}}-2\xi\frac{\dif{h}}{\dif{\xi}}+(\xi^{2}-1)h\big )e^{-\xi^{2}/2}
\end{align*}
 لیتے ہیں لہٰذا شروڈنگر مساوات (مساوات \حوالہ{مساوات_شروڈنگر_تحلیلی_شروڈنگر}) درج ذیل صورت اختیار کرتی ہے۔
\begin{align}\label{مساوات_شروڈنگر_فروبنیوس_الف}
\frac{\dif^{\,2}{h}}{\dif{\xi^{\,2}}}-2\xi\frac{\dif{h}}{\dif{\xi}}+(K-1)h=0
\end{align}
ہم \اصطلاح{ترکیب فروبنیوس}\فرہنگ{فروبنیوس!ترکیب}\حاشیہب{Frobenius method}\فرہنگ{Frobenius!method} استعمال کرتے ہوئے مساوات \حوالہ{مساوات_شروڈنگر_فروبنیوس_الف} کا حل \عددی{\xi} کے طاقتی تسلسل کی صورت میں حاصل کرتے ہیں۔ 
\begin{align}
h(\xi)=a_{0}+a_{1}\xi+a_{2}\xi^{2}+\cdots = \sum_{j=0}^{\infty}a_{j}\xi^{j}
\end{align}
 اس تسلسل کے جزو در جزو تفرقات
\begin{align*}
\frac{\dif{h}}{\dif{\xi}}=a_{1}+2a_{2}\xi+3a_{3}\xi^{2}+\cdots =\sum_{j=0}^{\infty}ja_{j}\xi^{j-1}
\end{align*}
اور
\begin{align*}
\frac{\dif^{\,2}{h}}{\dif{\xi^{\,2}}}=2a_{2}+2\cdot 3a_{3}\xi+3\cdot 4a_{4}\xi^{2}+\cdots =\sum_{j=0}^{\infty}(j+1)(j+2)a_{j+2}\xi^{j}
\end{align*}
لیتے ہیں۔ انہیں مساوات \حوالہ{مساوات_شروڈنگر_فروبنیوس_الف} میں پر کر کہ درج ذیل حاصل ہو گا۔
\begin{align}
\sum_{j=0}^{\infty}[(j+1)(j+2)a_{j+2}-2ja_{j}+(K-1)a_{j}]\xi^{j}=0
\end{align}
طاقتی تسلسل پھیلاو کے یکتائی  کی بنا \عددی{\xi}  کے ہر طاقت کا عددی سر  صفر ہو گا:
\begin{align*}
(j+1)(j+2)a_{j+2}-2ja_{j}+(K-1)a_{j}=0
\end{align*}
لہٰذا درج ذیل ہو گا۔
\begin{align}\label{مساوات_شروڈنگر_کلیہ_توالی_الف}
a_{j+2}=\frac{(2j+1-K)}{(j+1)(j+2)}a_{j}
\end{align}
یہ \اصطلاح{کلیہ توالی}\فرہنگ{توالی!کلیہ}\حاشیہب{recursion formula}\فرہنگ{recursion!formula}  شروڈنگر مساوات کا مکمل مبدل ہے جو \عددی{a_{0}} سے ابتداء کرتے ہوئے تمام جفت عددی سر
\begin{align*}
a_{2}=\frac{(1-K)}{2}a_{0}, \quad a_{4}=\frac{(5-K)}{12}a_{2}=\frac{(5-K)(1-K)}{24}a_{0},\cdots
\end{align*}
اور \عددی{a_{1}} سے شروع کر کے تمام طاق عددی سر پیدا کرتا ہے۔
\begin{align*}
a_{3}=\frac{(3-K)}{6}a_{1},\quad a_{5}=\frac{(7-K)}{20}a_{3}=\frac{(7-K)(3-K)}{120}a_{1},\cdots
\end{align*}
ہم مکمل حل کو درج ذیل لکھتے ہیں
\begin{align}\label{مساوات_شروڈنگر_کلیہ_توالی_ب}
h(\xi)=h_{\text{جفت}}(\xi)+h_{\text{طاق}}(\xi)
\end{align}
جہاں 
\begin{align*}
h_{\text{جفت}}(\xi)=a_{0}+a_{2}\xi^{2}+a_{4}\xi^{4}+\cdots
\end{align*}
متغیر \عددی{\xi} کا جفت تفاعل ہے جو از خود \عددی{a_{0}} پر منحصر ہے اور 
\begin{align*}
h_{\text{طاق}}(\xi)=a_{1}\xi+a_{3}\xi^{3}+a_{5}\xi^{5}+\cdots
\end{align*}
طاق تفاعل ہے جو \عددی{a_{1}} پر منحصر ہے۔  مساوات \حوالہ{مساوات_شروڈنگر_کلیہ_توالی_الف} دو اختیاری مستقلات \عددی{a_{0}}
اور \عددی{a_{1}} کی صورت میں \عددی{\xi} تعین کرتی ہے، جیسا ہم دو درجی تفرقی مساوات کے حل سے توقع کرتے ہیں۔

 البتہ اس طرح حاصل حلوں میں سے کئی معمول پر لانے کے قابل نہیں ہوں گے۔اس کی وجہ یہ ہے کہ \عددی{j} کی بہت بڑی قیمت کے لئے کلیہ توالی (تخمیناً) درج ذیل روپ اختیار کرتا ہے
\begin{align*}
a_{j+2}\approx\frac{2}{j}a_{j}
\end{align*}
جس کا تخمینی حل
\begin{align*}
a_{j}\approx\frac{C}{(j/2)!}
\end{align*}
ہو گا جہاں \عددی{C}  ایک مستقل ہے اور اس سے (بڑی \عددی{\xi}  کے لیے جہاں بڑی طاقتیں  غالب ہوں گی) درج ذیل حاصل ہو گا،
\begin{align*}
h(\xi)\approx C\sum\frac{1}{(j/2)!}\xi^{j}\approx C\sum\frac{1}{j!}\xi^{2j}\approx Ce^{\xi^{2}}
\end{align*}
اور اب اگر \عددی{h}  کی  قیمت \عددی{e^{\xi^{2}}} کے لحاظ سے بڑھے تب \عددی{\psi} (جس کو ہم حاصل کرنا چاہتے ہیں) 
 \عددی{e^{\xi^{2}/2}} (مساوات \حوالہ{مساوات_شروڈنگر_متقارب_الف}) کے لحاظ سے بڑھے گا جو وہی متقاربی روپ ہے جو ہم نہیں چاہتے۔ اس مشکل سے نکلنے  کا ایک ہی طریقہ ہے۔ معمول پر لانے کے قابل حل کے لئے لازم ہے کہ اس کا طاقتی تسلسل اختتام پذیر ہو۔ لازمی طور پر  \عددی{j} کی ایک ایسی بلند ترین قیمت، \عددی{n}،  پائی جائے گی جو \عددی{a_{n+2}=0} دیتی ہو (یوں \عددی{h_{\text{جفت}}}  تسلسل یا \عددی{h_{\text{طاق}}}  تسلسل اختتام پذیر  ہو گا؛ جبکہ  دوسرا لازماً ابتداء سے ہی صفر ہو گا؛ جفت \عددی{n} کی صورت میں \عددی{a_1=0} ہو گا جبکہ   طاق \عددی{n} کی صورت میں \عددی{a_0=0} ہو گا)۔  یوں قابل قبول طبعی حل کے لیے مساوات \حوالہ{مساوات_شروڈنگر_کلیہ_توالی_الف} کے تحت درج ذیل ہو گا
\begin{align*}
K=2n+1
\end{align*}
 جہاں \عددی{n} کوئی غیر منفی عدد صحیح ہو گا، یعنی ہم کہنا چاہتے ہیں کہ (مساوات \حوالہ{مساوات_شروڈنگر_تحلیلی_مستقل} کو دیکھیے)  توانائی ہر صورت درج ذیل ہو گی۔
\begin{align}
E_{n}&=(n+\tfrac{1}{2} )\hslash\omega && n=0,1,2\cdots
\end{align}
یوں ہم  ایک  مختلف طریقہ کار سے مساوات \حوالہ{مساوات_شروڈنگر_ہارمونی_حالات} میں الجبرائی طریقہ سے حاصل کردہ بنیادی کوانٹازنی شرط دوبارہ حاصل کرتے ہیں۔ ابتدائی طور پر یہ حیرانی کی بات نظر آتی ہے کہ توانائی کی کوانٹازنی، شروڈنگر مساوات کے طاقتی تسلسل حل  کے ایک تکنیکی نقطہ سے حاصل ہوتی ہے۔ آئیں اسے ایک مختلف نقطہ نظر سے دیکھتے ہیں۔ یقیناً  \عددی{E} کے کسی بھی قیمت کے لئے  مساوات \حوالہ{مساوات_شروڈنگر_تحلیلی_حل_الف} کے حل ممکن ہیں (درحقیقت ہر \عددی{E} کے لیے اس کے دو خطی  غیر تابع حل پائے جاتے ہیں)۔ تاہم ان میں سے زیادہ تر حل،  بڑی \عددی{x} پر، بے قابو قوت نمائی بڑھتے ہیں جس کی بنا یہ معمول پر لانے کے قابل نہیں رہتے۔ مثال کے طور پر فرض کریں ہم \عددی{E} کی کسی ایک اجازتی قیمت  سے معمولی کم   قیمت (مثلاً \عددی{0.49\hslash \omega}) لے کر حل کو ترسیم کرتے ہیں (شکل \حوالہء{2.6a})؛ اس کی دم لامتناہی کی طرف بڑھے گی۔ اب \عددی{E} کی قیمت کسی ایک اجازتی قیمت سے معمولی زیادہ  (مثلاً
\عددی{0.51\hslash\omega})  تصور کر کے حل ترسیم کرتے ہیں؛ اب حل کی دم دوسری سمت میں لامتناہی کی طرف بڑھے گی (شکل \حوالہء{2.6b})۔  اگر  ہم اس مقدار معلوم کی قیمت \عددی{0.49} اور \عددی{0.51} کے بیچ چھوٹے  چھوٹے قدم لے کر تبدیل کریں تو ہر مرتبہ \عددی{0.50} سے گزرتے ہوئے  حل کی دم دوسری  طرف لامتناہی کی طرف بڑھے گی۔ ٹھیک \عددی{0.50} پر اس کی دم صفر  کو پہنچ کر معمول زنی کے قابل حل دے گی۔

 کلیہ توالی \عددی{K} کی اجازتی قیمتوں کے لیے  درج ذیل روپ اختیار کرتی ہے۔
\begin{align}\label{مساوات_شروڈنگر_کلیہ_توالی_اجازتی_الف}
a_{j+2}=\frac{-2(n-j)}{(j+1)(j+2)}a_{j}
\end{align}
اگر \عددی{n=0} ہو تب تسلسل میں ایک جزو پایا جائے گا (ہمیں \عددی{a_{1}=0} لینا ہو گا تا کہ \عددی{h_{\text{طاق}}} خارج ہوں، اور مساوات \حوالہ{مساوات_شروڈنگر_کلیہ_توالی_اجازتی_الف} میں \عددی{j=0} سے \عددی{a_{2}=0} حاصل ہوتا ہے):
\begin{align*}
h_{0}(\xi)=a_{0}
\end{align*}
لہٰذا
\begin{align*}
\psi_{0}(\xi)=a_{0}e^{-\xi^{2}/2}
\end{align*}
(جو ماسوائے معمول زنی،  مساوات \حوالہ{مساوات_شروڈنگر_معمول_شدہ_حال_صفر} دوبارہ دیتی ہے)۔ اسی طرح ہم \عددی{n=1} کے لیے 
\عددی{a_{0}=0} لیں گے\حاشیہد{دھیان رہے کہ \عددی{n} کی ہر ایک قیمت کے لئے عددی سروں \عددی{a_j} کا ایک منفرد سلسلہ  پایا جاتا ہے۔}، اور مساوات \حوالہ{مساوات_شروڈنگر_کلیہ_توالی_اجازتی_الف} میں \عددی{j=1} سے \عددی{a_{3}=0}
حاصل ہو گا، لہٰذا 
\begin{align*}
h_{1}(\xi)=a_{1}(\xi)
\end{align*}
اور
\begin{align*}
\psi_{1}(\xi)=a_{1}\xi e^{-\xi^{\,2}/2}
\end{align*}
ہو گا (جو مساوات \حوالہ{مساوات_شروڈنگر_ہارمونی_ہیجان_حالات} کی تصدیق کرتی ہے)۔ ہم \عددی{n=2} کے لیے \عددی{j= 0}  لے کر
\عددی{a_{2}=-2a_{0}} اور \عددی{j=2} لے کر \عددی{a_{4}=0} حاصل کرتے ہیں۔ یوں
\begin{align*}
h_{2}(\xi)=a_{0}(1-2\xi^{2})
\end{align*}
اور
\begin{align*}
\psi_{2}(\xi)=a_{0}(1-2\xi^{2})e^{-\xi^{2}/2}
\end{align*}
ہوں گے، وغیرہ  وغیرہ۔ (سوال \حوالہ{سوال_شروڈنگر_عمودیت_اور_تیار} کے ساتھ موازنہ کریں جہاں یہ آخری نتیجہ الجبرائی ترکیب سے حاصل کیا گیا۔) 
\begin{table}
\caption{ابتدائی چند ہرمائٹ کثیر رکنیاں \عددی{H_n(\xi)}}
\label{جدول_شروڈنگر_ہرمشی_کثیر_رکنیاں}
\centering
\begin{tabular}{l}
$H_0=1$\\
$H_1=2\xi$\\
$H_2=4\xi^2-2$\\
$H_3=8\xi^3-12\xi$\\
$H_4=16\xi^4-48\xi^2+12$\\
$H_5=32\xi^5-160\xi^3+120\xi$
\end{tabular}
\end{table}
 عمومی طور پر \عددی{h_{n}(\xi)} متغیر \عددی{\xi} کا \عددی{n} درجی کثیر رکنی ہو گا، جو جفت عدد صحیح \عددی{n} کی صورت میں جفت طاقتوں کا اور طاق عدد صحیح \عددی{n} کی صورت میں طاق طاقتوں کا کثیر رکنی ہو گا۔  جزو ضربی \عددی{a_{0}} اور \عددی{a_{1}}
کے علاوہ یہ عین \اصطلاح{ہرمائٹ کثیر رکنی}\فرہنگ{کثیر رکنی!ہرمائٹ}\حاشیہب{Hermite polynomials}\فرہنگ{polynomial!Hermite} \عددی{H_{n}(\xi)} ہیں\حاشیہد{ہرمائٹ کثیر رکنیوں پر سوال \حوالہ{سوال_شروڈنگر_ہرمائٹ_کثیر_رکنیاں} میں مزید غور کیا گیا ہے۔}۔ جدول \حوالہ{جدول_شروڈنگر_ہرمشی_کثیر_رکنیاں} میں اس کے چند ابتدائی ارکان پیش کیے  گئے ہیں۔ روایتی طور پر اختیاری جزو ضربی  یوں منتخب کیا جاتا ہے کہ \عددی{\xi} کے بلند تر طاقت کا عددی سر \عددی{2^{n}} ہو۔ اس روایت کے تحت ہارمونی مرتعش کے معمول شدہ\حاشیہد{میں یہاں معمول زنی مستقلات حاصل نہیں کروں گا۔} ساکن حالات درج ذیل ہوں گے
\begin{align}
\psi_{n}(x)=\big (\frac{m\omega}{\pi\hslash}\big )^{1/4}\frac{1}{\sqrt{2^{n}n!}}H_{n}(\xi)e^{-\xi^{2}/2}
\end{align}
جو (یقیناً) مساوات \حوالہ{مساوات_شروڈنگر_ہارمونی_ساکن_حالات} میں الجبرائی طریقے سے حاصل نتائج کے  متماثل ہیں۔

 شکل \حوالہء{2.7 (a)} میں  چند ابتدائی \عددی{n} کے لیے \عددی{\psi_{n}(x)} ترسیم کیے گئے ہیں۔ کوانٹم مرتعش حیران کن حد تک کلاسیکی  مرتعش سے مختلف ہے۔ نہ صرف اس کی توانائیاں کوانٹاشدہ ہیں بلکہ اس کی  موضعی تقسیم  کے بھی عجیب خواص پائے جاتے ہیں۔ مثلاً  کلاسیکی طور پر اجازتی سعت کے باہر (یعنی توانائی کے کلاسیکی حیطہ سے زیادہ \عددی{x} پر) ذرہ پایا جانے کا احتمال غیر صفر ہے (سوال \حوالہ{سوال_شروڈنگر_ہارمونی_مرتعش_حیطہ} دیکھیں) اور تمام طاق حالات میں عین وسط پر ذرہ پائے جانے کا احتمال صفر ہے۔ کلاسیکی اور کوانٹائی صورتوں میں مشابہت صرف \عددی{n} کی بڑی قیمتوں پر  پائی جاتی ہے۔ میں نے شکل \حوالہء{2.7b} میں   کلاسیکی موضعی تقسیم کو \عددی{n=100} کے  کوانٹائی موضعی تقسیم پر ترسیم کیا ہے۔ انہیں ہموار کرنے سے یہ ایک دوسرے پر اچھی طرح بیٹھتے ہیں (البتہ  کلاسیکی صورت میں ہم ایک ارتعاش میں وقت کے لحاظ سے مقام کی تقسیم  کی بات کرتے ہیں جبکہ کوانٹائی صورت میں ہم  یکساں تیار کردہ حالات کے ایک سگرا کی تقسیم کی بات کرتے ہیں)۔\حاشیہد{کلاسیکی تقسیم کو ایک جیسی توانائی کے متعدد مرتعشات، جن کے نقاط آغاز بلا منصوبہ ہوں، کا سگرا تصور کرتے ہوئے یہ مماثل زیادہ بہتر ہو گا۔}
%%%%

\ابتدا{سوال}\شناخت{سوال_شروڈنگر_ہارمونی_مرتعش_حیطہ}
ہارمونی مرتعش کے زمینی حال میں کلاسیکی اجازتی خطہ کے باہر ایک ذرہ کی موجودگی کا احتمال (تین با معنی ہندسوں تک) تلاش کریں۔
\ترچھا{اشارہ:}  کلاسیکی طور پر ایک مرتعش کی توانائی \عددی{E=(1/2)ka^{2}=(1/2)m\omega^{2}a^{2}} ہو گی جہاں  \عددی{a} حیطہ  ہے۔ یوں  توانائی \عددی{E} کے مرتعش کا "کلاسیکی اجازتی خطا" \عددی{-\sqrt{2E/m\omega^{2}}} تا \عددی{+\sqrt{2E/m\omega^{2}}} ہو گا۔ تکمل کی قیمت "عمومی تقسیم" یا "تفاعل خلل" کی جدول سے دیکھیں۔
\انتہا{سوال}
%
\ابتدا{سوال}
کلیہ توالی (مساوات \حوالہ{مساوات_شروڈنگر_کلیہ_توالی_اجازتی_الف}) استعمال کر کے \عددی{H_{5}(\xi)} اور \عددی{H_{6}(\xi)}
تلاش کریں۔ مجموعی مستقل تعین کرنے کی خاطر  \عددی{\xi} کی بلند تر طاقت کا عددی سر روایت کے تحت \عددی{2^{n}} لیں۔
\انتہا{سوال}
%
\ابتدا{سوال}\شناخت{سوال_شروڈنگر_ہرمائٹ_کثیر_رکنیاں} 
اس سوال میں ہم ہرمائٹ کثیر رکنی کے چند اہم مسائل، جن کا ثبوت پیش نہیں کیا جائے گا، پر غور کرتے ہیں۔
\begin{enumerate}[a.]
\item
\اصطلاح{کلیہ روڈریگیس}\فرہنگ{کلیہ!روڈریگیس}\حاشیہب{Rodrigues formula}\فرہنگ{Rodrigues!formula}  درج ذیل کہتا ہے۔
\begin{align}
H_{n}(\xi)=(-1)^{n}e^{\xi^{\,2}}\frac{\dif^{\,n} }{\dif \xi^{\,n}}e^{-\xi^{2}}
\end{align}
اس کو استعمال کر کے \عددی{H_{3}} اور \عددی{H_{4}} اخذ کریں۔
\item
درج ذیل کلیہ توالی گزشتہ دو ہرمائٹ کثیر رکنیوں کی صورت میں \عددی{H_{n+1}} دیتا ہے۔
\begin{align}
H_{n+1}(\xi)=2\xi H_{n}(\xi)-2nH_{n-1}(\xi)
\end{align}
اس کو جزو-ا کے نتائج کے ساتھ استعمال کر کے \عددی{H_{5}} اور \عددی{H_{6}} تلاش کریں۔
\item
اگر آپ \عددی{n} رتبی کثیر رکنی کا تفرق لیں تو آپکو \عددی{n-1} رتبی کثیر رکنی حاصل ہو گی۔ ہرمائٹ کثیر رکنیوں کے لیے درج ذیل ہو گا
\begin{align}
\frac{\dif{H_{n}}}{\dif{\xi}}=2nH_{n-1}(\xi)
\end{align}
جس کی تصدیق ہرمائٹ کثیر رکنی \عددی{H_{5}} اور \عددی{H_{6}} کے لئے  کریں۔ 
\item
\اصطلاح{پیداکار تفاعل}\فرہنگ{پیداکار!تفاعل}\حاشیہب{generating function}\فرہنگ{generating!function} \عددی{e^{-z^{2}+2z\xi}} کا \عددی{z=0} پر \عددی{n} واں تفرق \عددی{H_{n}(\xi)} ہو گا، یا دوسرے لفظوں میں،  درج ذیل تفاعل کے ٹیلر پھیلاو میں یہ
\عددی{z^{n}/n!} کا عددی سر ہو گا۔
\begin{align}
e^{-z^{2}+2z\xi}=\sum_{n=0}^{\infty}\frac{z^{n}}{n!}H_{n}(\xi)
\end{align}
اس کو  استعمال کر کے\عددی{H_0}، \عددی{H_1}  اور \عددی{H_{2}}  دوبارہ اخذ کریں۔
\end{enumerate}
\انتہا{سوال}
%=================================

\حصہ{آزاد ذرہ}
ہم اب آزاد ذرہ (جس کے لیے پر جگہ \عددی{V(x)=0}  ہو گا) پر غور کرتے ہیں جس سادہ ترین صورت ہونی چاہیے تھی۔ کلاسیکی طور پر اس سے مراد  مستقل سمتی رفتار ہو گی، لیکن کوانٹم میکانیات میں یہ مسئلہ حیران کن حد تک پیچیدہ اور پراسرار ثابت ہوتا ہے۔  غیر تابع وقت شروڈنگر مساوات  ذیل 
\begin{align}\label{مساوات_شروڈنگر_آزاد_ذرہ_الف}
-\frac{\hslash^{2}}{2m}\frac{\dif^{\,2}{\psi}}{\dif{x^{2}}}=E\psi
\end{align}
یا ذیل ہے۔
\begin{align}
\frac{\dif^{\,2}{\psi}}{\dif{x^{2}}}&=-k^{\,2}\psi&& k\equiv\frac{\sqrt{2mE}}{\hslash}
\end{align}
یہاں تک یہ لامتناہی چکور  کنواں (مساوات \حوالہ{مساوات_شروڈنگر_کلاسیکی_ہارمونی_مرتعش}) کی مانند ہے  جہاں (بھی) مخفی قوہ صفر ہے؛ البتہ اس بار،  میں عمومی مساوات کو قوت نما (نا کہ سائن اور کوسائن) کی صورت میں لكهنا چاہوں گا، جس کی وجہ آپ پر جلد عیاں ہو گی۔
\begin{align}
\psi(x)=Ae^{ikx}+Be^{-ikx}
\end{align}
لامتناہی چکور  کنواں کے برعکس، یہاں  کوئی سرحدی شرائط نہیں پائے جاتے ہیں جو \عددی{k} (اور یوں \عددی{E})  کی ممکنہ قیمتوں پر کسی قسم کی  پابندی عائد کرتے ہوں؛ لہٰذا آزاد ذرہ کسی بھی (مثبت) توانائی کا حامل ہو سکتا ہے۔ اس کے ساتھ  تابعیت وقت \عددی{e^{-iEt/\hslash}}
جوڑتے ہوئے ذیل حاصل ہو گا۔
\begin{align}\label{مساوات_شروڈنگر_آزاد_ذرہ_حرکت}
\Psi(x,t)=Ae^{ik(x-\frac{\hslash k}{2m}t)}+Be^{-ik(x+\frac{\hslash k}{2m}t)}
\end{align}
ایسا کوئی بھی تفاعل جو  \عددی{x} اور \عددی{t} متغیرات کی  مخصوص جوڑ \عددی{(x\pm vt)}  کا تابع ہو  (جہاں \عددی{v} مستقل ہے)، 
 غیر تغیر شکل و صورت کی ایسی موج کو ظاہر کرے گا جو \عددی{v} رفتار سے  \عددی{\mp x} رخ حرکت کرتی ہے۔ اس موج پر ایک اٹل نقطہ (مثلاً کم سے کم یا زیادہ سے زیادہ قیمت کا نقطہ) تفاعل کے \اصطلاح{دلیل}\فرہنگ{دلیل}\حاشیہب{argument}\فرہنگ{argument} کی ایک اٹل قیمت کا یوں مطابقتی ہو گا کہ درج ذیل ہو۔
\begin{align*}
x=\mp vt+\text{مستقل}\quad \text{یا}\quad  x\pm vt=\text{مستقل}
\end{align*}
چونکہ موج پر تمام نقاط ایک جیسی سمتی رفتار سے حرکت کرتے ہیں لہٰذا موج کی شکل و صورت حرکت کے ساتھ تبدیل نہیں ہو گی۔ یوں مساوات \حوالہ{مساوات_شروڈنگر_آزاد_ذرہ_حرکت} کا پہلا جزو دائیں رخ حرکت کرتی موج کو ظاہر کرتا ہے جبکہ اس کا دوسرا جزو   بائیں رخ حرکت کرتی (اتنی ہی توانائی کی) موج کو ظاہر کرتا ہے۔ چونکہ ان میں فرق صرف \عددی{k} کی علامت کا ہے لہٰذا انہیں درج ذیل بھی لکھا جا سکتا ہے 
\begin{align}
\Psi_{k}(x,t)=Ae^{i(kx-\frac{\hslash k^{2}}{2m}t)}
\end{align}
جہاں \عددی{k} کی قیمت منفی لینے سے بائیں رخ حرکت کرتی موج حاصل ہو گی۔
\begin{align}
k\equiv \pm\frac{\sqrt{2mE}}{\hslash},\quad
\begin{cases}
k>0\Rightarrow \text{\RL{دائیں رخ حرکت}}\\
k<0\Rightarrow \text{\RL{بائیں رخ حرکت}}
\end{cases}
\end{align}
صاف ظاہر ہے کہ آزاد ذرے کے "ساکن حالات" حرکت کرتی امواج کو ظاہر کرتے ہیں، جن کی طول موج \عددی{\lambda=2\pi/\abs{k}} ہو گا،  اور کلیہ ڈی بروگ لی (مساوات \حوالہ{مساوات_تفاعل_موج_ڈی_بروگلی_معیار_حرکت}) کے تحت ان کا معیار حرکت درج ذیل ہو گا۔
\begin{align}
p=\hslash k
\end{align}
ان امواج کی رفتار (یعنی \عددی{t} کا عددی سر تقسیم  \عددی{x} کا عددی سر) درج ذیل ہو گا۔
\begin{align}
v_{\text{کوانٹائی}}=\frac{\hslash\abs{k}}{2m}=\sqrt{\frac{E}{2m}}
\end{align}
اس کے برعکس ایک آزاد ذرہ جس کی  توانائی \عددی{E} ہو (جو خالصتاً حرکی ہو گی چونکہ \عددی{V=0} ہے) کی کلاسیکی رفتار \عددی{E=(1/2)mv^{2}} سے حاصل کی جا سکتی ہے۔
\begin{align}\label{مساوات_شروڈنگر_کلاسیکی_کوانٹائی_رفتار}
v_{\text{کلاسیکی}}=\sqrt{\frac{2E}{m}}=2v_{\text{کوانٹائی}}
\end{align}
ظاہری طور پر کوانٹم میکانی تفاعل موج اس ذرے کی نصف رفتار سے حرکت کرتا ہے جس کو یہ ظاہر کرتا ہے۔ اس تضاد پر ہم کچھ دیر میں غور کریں گے۔ اس سے پہلے ایک زیادہ  سنگین  مسئلہ پر غور کرنا ضروری ہے۔ درج ذیل کے تحت یہ تفاعل موج معمول پر لانے کے قابل نہیں ہے۔ 
\begin{align}
\int_{-\infty}^{+\infty}\Psi_{k}^{*}\Psi_{k}\dif{x}=\abs{A}^{2}\int_{-\infty}^{+\infty}\dif{x}=\abs{A}^{2}(\infty)
\end{align}
یوں آزاد ذرے کی صورت میں قابل علیحدگی حل طبعی طور پر قابل قبول حالات کو ظاہر نہیں کرتے ہیں۔ ایک آزاد ذرہ ساکن حال میں نہیں پایا جا سکتا ہے؛ دوسرے لفظوں میں،  غیر مبہم توانائی کے ایک آزاد ذرے کا تصور بے معنی ہے۔

 اس کا ہرگز یہ مطلب نہیں کہ قابل علیحدگی حل ہمارے کسی کام کے نہیں ہیں، کیونکہ یہ طبعی  مفہوم  سے آزاد، ریاضیاتی کردار ادا کرتے ہیں۔ تابع وقت شروڈنگر مساوات کا عمومی حل اب بھی قابل قابل علیحدگی حلوں کا خطی جوڑ ہو گا (صرف اتنا ہے کہ غیر مسلسل اشاریہ \عددی{n} پر مجموعہ کی بجائے  اب یہ استمراری متغیر \عددی{k} کے لحاظ سے تکمل  ہو گا)۔
\begin{align}
\Psi(x,t)=\frac{1}{\sqrt{2\pi}}\int_{-\infty}^{+\infty}\phi(k)e^{i(kx-\frac{\hslash k^{2}}{2m}t)}\dif{k}
\end{align}


(ہم \عددی{ \frac{1}{\sqrt{2 \pi }} } کو اپنی آسانی کیلئے تکمل کے باہر نکالتے ہیں؛ مساوات 
 \حوالہ{مساوات_شروڈنگر_عمومی_حل_مجموعہ} میں عددی سر \عددی{ c_{n} } کی جگہ یہاں \عددی{ ( 1/\sqrt{2\pi}) \phi(k) \dif k  } کردار ادا کرتا ہے۔) اب اس تفاعل موج کو (موزوں \عددی{ \phi(k) } کیلئے) معمول پر لایا جا سکتا ہے۔ تاہم اس میں \عددی{k} کی قیمتوں کی سعت پائی جائے گی، لہٰذا توانائیوں اور رفتاروں کی بھی سعت پائی جائیں گی۔ ہم اس کو \اصطلاح{موجی اکٹھ}\فرہنگ{موجی اکٹھ}\حاشیہب{wave packet}\فرہنگ{wave!packet} کہتے ہیں۔\حاشیہد{سائن نما امواج کی وسعت لامتناہی تک پہنچتی ہے اور یہ معمول پر لانے کے قابل نہیں ہوتی ہیں۔ تاہم ایسی امواج کا خطی میل تباہ کن مداخلت پیدا کرتا ہے، جس کی بنا مقام بندی اور معمول زنی ممکن ہوتی ہے۔}

عمومی کوانٹم مسئلہ میں ہمیں \عددی{ \Psi (x,0) } فراہم کر کے \عددی{ \Psi(x,t) } تلاش کرنے کو کہا جاتا ہے۔ آزاد ذرے کیلئے اس کا حل مساوات \حوالہء{ 2.100} کی صورت اختیار کرتا ہے۔ اب سوال یہ پیدا ہوتا ہے کہ ابتدائی تفاعل موج 
\begin{align}
\Psi (x,0) = \frac{1}{\sqrt{2\pi}} \int_{- \infty}^{+ \infty} \phi (k) e^{ikx} \dif k
\end{align}
پر پورا اترتا ہوا \عددی{ \psi(k) } کیسے تعین کیا جائے؟ یہ فوریئر تجزیہ کا کلاسیکی مسئلہ ہے جس کا جواب \اصطلاح{مسئلہ پلانشرال}\فرہنگ{مسئلہ!پلانشرال}\حاشیہب{Plancherel's theorem}\فرہنگ{theorem!Plancherel}: 
\begin{align}\label{مساوات_شروڈنگر_مسئلہ_پلانشرل}
f(x) = \frac{1}{\sqrt{2\pi}} \int_{- \infty}^{+ \infty} F(k) e^{ikx} \dif k \Leftrightarrow F(k) = \frac{1}{\sqrt{2\pi}} \int_{- \infty}^{+ \infty} f(x) e^{-ikx} \dif x
\end{align}
 پیش کرتا ہے (سوال \حوالہء{2.20} دیکھیں)۔ \عددی{F(k)} کو \عددی{f(x)} کا \اصطلاح{فوریئر بدل}\فرہنگ{فوریئر!بدل}\حاشیہب{Fourier transform}\فرہنگ{Fourier!transform} کہا جاتا ہے جبکہ \عددی{f(x)} کو \عددی{F(k)} کا \اصطلاح{الٹ فوریئر بدل}\فرہنگ{فوریئر!الٹ بدل}\حاشیہب{inverse Fourier transform}\فرہنگ{Fourier!inverse transform} کہتے ہیں (ان دونوں میں صرف قوت نما کی علامت کا فرق پایا جاتا ہے)۔ ہاں، اجازتی تفاعل پر کچھ پابندی ضرور عائد ہے: تکمل کا \ترچھا{موجود}\حاشیہد{تفاعل \عددی{f(x)} پر عائد لازم اور کافی پابندی یہ ہے کہ \عددی{\int_{-\infty}^{\infty}\abs{f(x)}^2\dif x} متناہی ہو۔ (ایسی صورت میں \عددی{\int_{-\infty}^{\infty}\abs{F(k)}^2\dif k} بھی متناہی ہو گا، اور حقیقتاً ان دونوں تکملات کی قیمتیں ایک دوسری جتنی ہوں گی۔ \text{Arfken} کے حصہ \text{15.5} میں حاشیہ \text{24} دیکھیں۔)}\شناخت{حاشیہ_شروڈنگر_مربع_تکملیت} ہونا لازم ہے۔ ہمارے مقاصد کے لئے، تفاعل \عددی{\Psi(x,0)} پر بذات خود معمول شدہ ہونے کی طبعی شرط مسلط کرنا اس کی ضمانت دے گا۔ یوں آزاد ذرے کے عمومی کوانٹم مسئلہ کا حل مساوات \حوالہء{ 2.100} ہو گا جہاں \عددی{\phi(k)} درج ذیل ہو گا۔ 
\begin{align}\label{مساوات_شروڈنگر_الٹ_بدل_سائے}
\phi(k) = \frac{1}{\sqrt{2\pi}} \int_{- \infty}^{+\infty} \Psi(x,0)e^{-ikx} \dif x
\end{align}
%===============
\ابتدا{مثال}
ایک آزاد ذرہ جو ابتدائی طور پر خطہ \عددی{  -a \leq x \leq a    } میں رہنے کا پابند  ہو کو وقت \عددی{t=0} پر چھوڑ دیا جاتا ہے:
\begin{align*}
\Psi (x,0) =
\begin{cases}
A, & -a < x < a,\\
 0, &\text{\RL{دیگر صورت}}
\end{cases}
\end{align*}
جہاں \عددی{   A  } اور \عددی{  a   } مثبت حقیقی مستقل ہیں۔ \عددی{   \Psi(x,t)  } تلاش کریں۔ 

حل: \quad
ہم پہلے \عددی{  \Psi(x,0)  } کو معمول پر لاتے ہیں۔ 
\begin{align*}
1 = \int_{-\infty}^{\infty} \left| \Psi (x,0) \right| ^{2} \dif x = \left| A \right|^{2} \int_{-a}^{a} \dif x = 2a \left| A \right|^{2} \Rightarrow A = \frac{1}{\sqrt{2a}}
\end{align*}
اس کے بعد مساوات \حوالہ{مساوات_شروڈنگر_الٹ_بدل_سائے} استعمال کرتے ہوئے \عددی{  \psi(k)   }  تلاش کرتے ہیں۔
\begin{align*}
\phi(k) =& \frac{1}{\sqrt{2\pi}} \frac{1}{\sqrt{2a}} \int_{-a}^{a} e^{-ikx} \dif x = \frac{1}{2\sqrt{\pi a}} \left. \frac{e^{-ikx}}{-ik} \right|_{-a}^{a} \\
=& \frac{1}{k\sqrt{\pi a}} \left( \frac{e^{ikx} - e^{-ikx}}{2i} \right) = \frac{1}{\sqrt{\pi a}} \frac{\sin(ka)}{k}
\end{align*}
آخر میں ہم اس کو دوبارہ مساوات \حوالہء{ 2.100} میں پر کرتے ہیں۔ 
\begin{align}
\Psi (x,t) = \frac{1}{\pi \sqrt{2a}} \int_{-\infty}^{\infty} \frac{\sin (ka)}{k} e^{i(kx-\frac{\hslash k^{2}}{2m}t)} \dif k
\end{align}
بد قسمتی سے اس تکمل کو بنیادی تفاعل کی صورت میں حل کرنا ممکن نہیں ہے، تاہم اس کی قیمت کو اعدادی تراکیب سے حاصل کیا جا سکتا ہے (شکل \حوالہء{2.8})۔ (ایسی بہت کم صورتیں حقیقتاً پائی جاتی ہیں جن کے لئے \عددی{\Psi(x,t)} کا تکمل (مساوات \حوالہء{ 2.100}) صریحاً حل کرنا ممکن ہو۔ سوال \حوالہ{سوال_شروڈنگر_گاوسی_موجی_اکٹھ} میں ایسی ایک بالخصوص خوبصورت مثال پیش کی گئی ہے۔)

آئیں ایک تحدیدی صورت پر غور کریں۔ اگر \عددی{a} کی قیمت بہت کم ہو تب ابتدائی تفاعل موج  خوبصورت مقامی نوکیلی صورت اختیار کرتی ہے (شکل \حوالہء{2.9})۔ ایسی صورت میں ہم چھوٹے زاویوں کے لئے تخمیناً  \عددی{  \sin ka \approx ka    } لکھ کر درج ذیل حاصل کرتے ہیں
\begin{align*}
\phi (k)  \approx \sqrt{\frac{a}{\pi}}
\end{align*}
جو  \عددی{k} کی مختلف قیمتوں کا آپس میں کٹ جانے کی بنا افقی ہے (شکل \حوالہء{2.9})۔ یہ مثال ہے اصول عدم یقینیت  کی: اگر ذرے کے مقام میں پھیلاو  کم ہو،  تب اس کی معیار حرکت (لہٰذا \عددی{k}، مساوات \حوالہء{ 2.96} دیکھیں) کا پھیلاو  لازماً زیادہ ہو گا۔ اس کی دوسری انتہا  (بڑی  \عددی{a}) کی صورت میں مقام کا پھیلاو  زیادہ  ہو گا (شکل \حوالہء{2.10}) لہٰذا درج ذیل ہو گا۔ 
\begin{align*}
\phi (k)  = \sqrt{\frac{a}{\pi}} \frac{\sin ka }{ka }
\end{align*}
اب \عددی{ \sin z/z  } کی زیادہ سے زیادہ قیمت   \عددی{ z=0  }  پر پائی جاتی ہے جو گھٹ کر \عددی{   z=\pm\pi   }  ( جو یہاں \عددی{  k=\pm \pi/a } کو ظاہر کرتا ہے) پر  صفر ہوتی ہے۔ یوں بڑی \عددی{  a } کیلئے \عددی{   k=0 } پر \عددی{\phi (k)} نوکیلی صورت اختیار کرے گا (شکل \حوالہء{2.10})۔   اس بار ذرے کی معیار حرکت اچھی طرح معین ہے جبکہ اس کا مقام صحیح طور پر معلوم نہیں ہے۔
\انتہا{مثال}
%=================
آئیں اب اس تضاد پر دوبارہ بات کریں جس کا ذکر ہم پہلے کر چکے: جہاں مساوات \حوالہء{ 2.94} میں دیا گیا علیحدگی   حل \عددی{\Psi_{k} (x, t)}، ٹھیک اس ذرہ کی رفتار سے حرکت نہیں کرتی ہے جس کو یہ بظاہر ظاہر کرتی ہے۔ حقیقتاً  یہ مسئلہ وہیں پر ختم ہو گیا تھا جب ہم جان چکے کہ \عددی{\Psi_{k}} طبعی طور پر قابل حصول حل نہیں ہے۔ بحر حال آزاد ذرے کی تفاعل موج (مساوات \حوالہء{ 2.100}) میں سموئی سمتی رفتار کی معلومات پر غور کرنا دلچسپی کا باعث ہے۔ بنیادی تصور کچھ یوں ہے: سائن نما تفاعلات کا خطی میل جس کے حیطہ کو \عددی{\phi} ترمیم کرتا ہو (شکل \حوالہء{2.11})   موجی اکٹھ ہو گا؛ یہ "غلاف" میں ڈھانکے ہوئے "لہروں" پر مشتمل ہو گا۔ انفرادی لہر کی رفتار، جس کو \اصطلاح{دوری سمتی رفتار}\فرہنگ{رفتار!دوری سمتی}\حاشیہب{phase velocity}\فرہنگ{velocity!phase} کہتے ہیں، ہرگز ذرے کی سمتی رفتار کو ظاہر نہیں کرتی ہے بلکہ غلاف کی رفتار، جس کو \اصطلاح{گروہی سمتی رفتار}\فرہنگ{رفتار!گروہی سمتی}\حاشیہب{group velocity}\فرہنگ{velocity!group} کہتے ہیں، ذرے کی رفتار ہو گی۔ غلاف کی سمتی رفتار لہروں کی فطرت پر منحصر ہو گی؛  یہ لہروں کی سمتی رفتار سے زیادہ، کم یا اس کے برابر ہو سکتی  ہے۔ ایک دھاگے پر امواج کی گروہی سمتی رفتار اور دوری سمتی رفتار  ایک دوسرے کے برابر ہوتی ہیں۔ پانی کی امواج کیلئے یہ دوری سمتی رفتار کی نصف ہو گی، جیسا آپ نے جھیل میں پتھر پھینک کر دیکھا ہو گا (اگر آپ پانی کی ایک مخصوص لہر پر نظر جمائے رکھیں تو آپ دیکھیں گے کہ، پیچھے سے آگے کی طرف بڑھتے ہوئے، آغاز میں اس لہر کا حیطہ  بڑھتا ہے جبکہ آخر میں آگے پہنچ کر اس کا حیطہ گھٹ کر صفر ہو جاتا ہے؛ اس دوران یہ تمام بطور ایک مجموعہ  نصف رفتار سے حرکت کرتا ہے۔) یہاں میں نے دکھانا ہو گا کہ کوانٹم میکانیات میں آزاد ذرے کے تفاعل موج کی گروہی سمتی رفتار اس کی دوری سمتی رفتار سے دگنی ہے، جو عین ذرے کی کلاسیکی  رفتار ہے۔

 ہمیں درج ذیل عمومی صورت کے موجی اکٹھ کی گروہی سمتی رفتار تلاش کرنی ہو گی۔ 
\begin{align*}
\Psi (x,t) = \frac{1}{\sqrt{2\pi}} \int_{- \infty}^{+ \infty} \phi (k) e^{i(kx - \omega t)} \dif k
\end{align*}
 (یہاں \عددی{\omega = (\hslash k^{2} /2m)} ہے، لیکن جو کچھ میں کہنے جا رہا ہوں وہ کسی بھی موجی اکٹھ کیلئے، اس کے \اصطلاح{انتشاری رشتہ}\فرہنگ{انتشاری!رشتہ}\حاشیہب{dispersion relation}\فرہنگ{dispersion!relation} (\عددی{\omega} کا متغیر \عددی{k} کے لحاظ سے کلیہ) سے قطع نظر،  درست ہو گا۔)  ہم فرض کرتے ہیں کہ کسی مخصوص قیمتی \عددی{k_{0}} پر  \عددی{\phi (k)} نوکیلی صورت اختیار کرتا ہے۔ (ہم زیادہ وسعت کا \عددی{k} بھی لے سکتے ہیں لیکن ایسے موجی اکٹھ کے مختلف اجزاء مختلف رفتار سے حرکت کرتے ہیں جس کی بنا یہ موجی اکٹھ بہت تیزی سے اپنی شکل و صورت تبدیل کرتا ہے اور کسی مخصوص سمتی رفتار پر حرکت کرتے ہوئے ایک مجموعہ  کا تصور بے معنی ہو جاتا ہے۔) چونکہ  \عددی{k_{0}} سے دور متکمل قابل نظر انداز ہے لہٰذا ہم تفاعل \عددی{\omega (k)} کو اس نقطہ کے گرد ٹیلر تسلسل سے پھیلا کر صرف ابتدائی اجزاء لیتے ہیں:
\begin{align*}
\omega (k) \cong \omega_{0} + \omega_{0}^{'} (k-k_{0})
\end{align*}
 جہاں نقطہ \عددی{k_{0}} پر \عددی{k} کے لحاظ سے \عددی{\omega} کا تفرق \عددی{\omega_{0}^{'}} ہے۔
 
(تکمل کے وسط کو \عددی{k_{0}} پر منتقل کرنے کے غرض سے)  ہم متغیر \عددی{k} کی جگہ متغیر \عددی{s=k-k_{0}} استعمال کرتے ہیں۔ یوں درج ذیل ہو گا۔ 
\begin{align*}
\Psi (x,t) \cong \frac{1}{\sqrt{2 \pi }} \int_{- \infty}^{+ \infty} \phi (k_{0} + s) e^{i[(k_{0} +s)x-(\omega_{0} + \omega_{0}^{'}s )t]} \dif s
\end{align*}
 وقت \عددی{     t=0  }  پر 
\begin{align*}
\Psi (x,0) = \frac{1}{\sqrt{2 \pi }} \int_{- \infty}^{+ \infty} \phi (k_{0} + s) e^{i(k_{0} +s)x} \dif s
\end{align*}
 جبکہ بعد کے وقت پر درج ذیل ہو گا۔ 
\begin{align*}
\Psi (x,t) \cong \frac{1}{\sqrt{2 \pi}} e^{i(-\omega_{0}t+k_{0}\omega_{0}^{'}t)} \int_{-\infty}^{+\infty} \phi (k_{0} + s) e^{i(k_{0} + s ) ( x - \omega_{0}^{'}t)} \dif s
\end{align*}
 ما سوائے \عددی{x}  کو \عددی{(x - \omega_{0}^{'}t)} منتقل کرنے کے یہ \عددی{\Psi(x,0)} میں پایا جانے والا تکمل ہے۔ یوں درج ذیل ہو گا۔ 
\begin{align}
\Psi(x,t) \cong e^{-i(\omega_{0} - k_{0} \omega_{0}^{'})t} \,\Psi(x-\omega_{0}^{'}t,0)
\end{align}
 ماسوائے  دوری جزو ضرب  کے (جو کسی بھی صورت میں \عددی{\left| \Psi \right|^{2}}  کی قیمت پر اثر انداز نہیں ہو گا) یہ موجی اکٹھ بظاہر سمتی رفتار \عددی{\omega_{0}^{'}} سے حرکت کرے گا: 
\begin{align}
v_{\text{گروہی}} = \frac{\dif \omega}{\dif k}
\end{align}
 (جس کی قیمت کا حساب \عددی{k = k_{0}} پر کیا جائے گا)۔  آپ دیکھ سکتے ہیں کہ یہ دوری رفتار سے مختلف ہے جسے درج ذیل مساوات پیش کرتی ہے۔ 
\begin{align}
v_{\text{دوری}} = \frac{\omega}{k}
\end{align}
 یہاں  \عددی{     \omega = (\hslash k^{2} /2m)  } یعنی \عددی{   \omega/k = (\hslash k/2m)   } ہے جبکہ \عددی{    \dif \omega / \dif k  = (\hslash k /m)   } ہے جو  دگنا ہے۔ یہ اس بات کی تصدیق کرتا ہے کہ موجی اکٹھ کی گروہی سمتی رفتار نا کہ ساکن حالات کی دوری سمتی رفتار کلاسیکی ذرے کی رفتار دے گی۔ 
\begin{align}
v_{\text{کلاسیکی}} = v_{\text{گروہی}} = 2v_{\text{دوری}}
\end{align}
%=====================
\ابتدا{سوال}
دکھائیں کہ متغیر \عددی{    x    } کے کسی بھی تفاعل کو لکھنے کے دو معادل طریقے \عددی{   [ Ae^{ikx}+Be^{-ikx}]    } اور  \عددی{    [C\cos kx + D\sin kx ]   }  ہیں۔ مستقلات \عددی{    C    } اور  \عددی{    D    } کو مستقلات  \عددی{    A    } اور \عددی{      B  } کی صورت میں لکھیں۔ اسی طرح مستقلات \عددی{   A   } اور  \عددی{    B    } کو مستقلات  \عددی{    C    } اور \عددی{      D  } کی صورت میں لکھیں۔ \ترچھا{تبصرہ:}  کوانٹم میکانیات میں جب  \عددی{     V=0  } ہو، قوت نمائی تفاعل حرکت کرتے امواج کو ظاہر کرتی ہے اور انہیں استعمال کرتے ہوئے آزاد ذرے پر تبصرہ کرنا زیادہ آسان ہوتا ہے،  جبکہ  \عددی{     \sin  } اور  \عددی{    \cos   } ساکن امواج کو ظاہر کرتی ہے جو لامتناہی چکور  کنواں میں پائی جاتی ہے۔ 
\انتہا{سوال}
%================
\ابتدا{سوال}\شناخت{سوال_شروڈنگر_احتمال_بہاو_رو}
مساوات \حوالہء{2.94} میں دی گئی آزاد ذرے کے تفاعل موج کا احتمال رو \عددی{  J} تلاش کریں (سوال \حوالہء{1.14} دیکھیں)۔ احتمال رو کے بہاو کا رخ کیا ہو گا؟
\انتہا{سوال}
%=============== 
\ابتدا{سوال}
اس سوال میں آپ کو مسئلہ پلانشرال کا ثبوت حاصل کرنے میں مدد دیا جائے گا۔ آپ متناہی وقفہ کے فوریئر تسلسل سے آغاز کر کے اس وقفہ کو وسعت دیتے ہوئے لامتناہی تک بڑھاتے گے۔ 

\begin{enumerate}[a. ]
\item 
مسئلہ ڈرشلے کہتا ہے کہ وقفہ \عددی{  [-a, +a]   } پر کسی بھی تفاعل  \عددی{   f(x)  } کو فوریئر تسلسل کے پھیلاو سے ظاہر کیا جا سکتا ہے:
\begin{align*}
f(x) = \sum_{n=0}^{\infty} [ a_{n}\sin(  n\pi x/a )  + b_{n}\cos(  n\pi x/a )]
\end{align*}
دکھائیں کہ اس کو درج ذیل معادل روپ میں بھی لکھا جا سکتا ہے۔
\begin{align*}
f(x) = \sum_{n=-\infty}^{\infty} c_{n}e^{i n \pi x /a }
\end{align*}
\عددی{  a_{n}  } اور  \عددی{   b_{n} } کی صورت میں \عددی{  c_{n}   } کیا ہو گا ؟
 \item
 فوریئر تسلسل کے عددی سروں کے حصول کی مساواتوں سے درج ذیل اخذ کریں۔
\begin{align*}
c_{n} = \frac{1}{2a} \int_{-a}^{+a} f(x)e^{-in\pi x/a} \dif x
\end{align*}
\item
\عددی{  n   }  اور  \عددی{  c_{ n }   } کی جگہ نئے متغیرات \عددی{    k=( \tfrac{n \pi}{a}) }  اور 
 \عددی{   F(k) = \sqrt{\tfrac{2}{\pi}}\,ac_{n } } استعمال کرتے ہوئے  دکھائیں کہ جزو-ا اور جزو-ب درج ذیل روپ اختیار کرتے ہیں 
\begin{align*}
f(x) &= \frac{1}{\sqrt{2 \pi }} \sum_{n=-\infty}^{\infty} F(k)e^{ikx} \Delta k; && F(k) = \frac{1}{\sqrt{2\pi}} \int_{-a}^{+a} f(x) e^{-ikx} \dif x,
\end{align*}
جہاں ایک \عددی{ n    } سے اگلی \عددی{ n    } تک \عددی{  k   } میں تبدیلی  \عددی{  \Delta k   } ہے۔ 
\item
حد  \عددی{  a \rightarrow \infty   } لیتے ہوئے مسئلہ پلانشرال حاصل کریں۔  \ترچھا{تبصرہ:} \عددی{   F(k) } کی صورت میں \عددی{   f(x) } اور \عددی{   f(x) }  کی صورت میں  \عددی{   F(k) }  کے کلیات کے آغاز دو بالکل مختلف جگہوں  ہوئیں۔  اس کے باوجود  حد \عددی{  a \rightarrow \infty   } کی صورت میں ان دونوں کی ساخت ایک دوسرے کے ساتھ مشابہت رکھتی ہیں۔ 
\end{enumerate}
\انتہا{سوال}
%=======================
\ابتدا{سوال}
ایک آزاد ذرے کا ابتدائی تفاعل موج درج ذیل ہے 
\begin{align*}
\Psi (x,0) = Ae^{ -a \left| x \right| } 
\end{align*}
جہاں \عددی{  A   } اور  \عددی{  a   } مثبت حقیقی مستقل ہیں۔
\begin{enumerate}[a.]
\item 
\عددی{  \Psi(x,0) } کو معمول پر لائیں۔ 
\item
\عددی{  \phi(k)} تلاش کریں۔ 
\item
\عددی{ \Psi(x,t) } کو تکمل کی صورت میں تیار کریں۔ 
\item
تحدیدی صورتوں پر (جہاں \عددی{a   } بہت بڑا ہو، اور  جہاں \عددی{a}  بہت چھوٹا ہو) پر تبصرہ کریں۔ 
\end{enumerate}
\انتہا{سوال}
%==========================
\ابتدا{سوال}\شناخت{سوال_شروڈنگر_گاوسی_موجی_اکٹھ} \quad \موٹا{گاوسی موجی اکٹھ}
 ایک آزاد ذرے کا ابتدائی تفاعل موج درج ذیل ہے
\begin{align*}
\Psi(x,0) = A e^{-ax^{2}}
\end{align*}
جہاں \عددی{  A } اور \عددی{  a } مستقلات ہیں (\عددی{a} حقیقی اور مثبت ہے)۔ 
\begin{enumerate}[a.]
\item
\عددی{  \Psi(x,0) } کو معمول پر لائیں۔ 
\item
\عددی{  \Psi(x,t) } تلاش کریں۔ \ترچھا{اشارہ:} "مربع مکمل کرتے ہوئے" درج ذیل روپ کے تکمل با آسانی حل ہوتے ہیں۔ 
\begin{align*}
\int_{-\infty}^{+\infty} e^{-(ax^{2} +bx)} \dif x
\end{align*}
مان لیں \عددی{  y \equiv \sqrt{a} [ x + (b/2a)] } ہے۔ یوں \عددی{  ( ax^{2} + bx ) = y^{2} - (b^{2}/4a) } ہو گا۔ جواب : 
\begin{align*}
\Psi(x,t) = \left( \frac{2a}{\pi} \right) ^{1/4} \frac{e^{-ax^{2}/[1+(2i\hslash at/m)]}}{\sqrt{1+(2i\hslash at/m)}}
\end{align*}
\item
\عددی{  \left| \Psi(x,t) \right|^{2} } تلاش کریں۔ اپنا جواب درج ذیل مقدار کی صورت میں لکھیں۔ 
\begin{align*}
\omega \equiv \sqrt{\frac{a}{1+(2\hslash at/m)^{2}}}
\end{align*}
وقت \عددی{ t = 0 } پر \عددی{ \left| \Psi \right|^{2} } کا خاکہ  (بطور \عددی{ x } کا تفاعل)  بنائیں۔ کسی بڑے \عددی{ t } پر دوبارہ خاکہ کھینچیں۔ وقت  گزرنے کے ساتھ ساتھ  \عددی{ \left| \Psi \right|^{2} } کو کیا ہو گا ؟
\item 
توقعاتی قیمتیں \عددی{\langle x\rangle}، \عددی{\langle p\rangle}، \عددی{\langle x^2\rangle} اور \عددی{\langle p^2\rangle}؛ اور احتمالات  \عددی{\sigma_x} اور \عددی{\sigma_p}  تلاش کریں۔
 \ترچھا{جزوی جواب:} \عددی{ \langle p^{2}\rangle = a\hslash^{2}}، تاہم  جواب کو  اس سادہ روپ میں لانے کیلئے آپ کو کافی الجبرا کرنا ہو گا۔ 
\item
کیا عدم یقینیت کا اصول یہاں کار آمد ہے؟ کس لمحہ \عددی{ t } پر یہ نظام عدم یقینیت کی حد کے قریب تر ہو گا ؟
\end{enumerate}
\انتہا{سوال}

%=======================
\حصہ{ڈیلٹا تفاعل مخفیہ}
\جزوحصہ{مقید حالات اور بکھراو حالات}
 ہم  غیر تابع وقت شروڈنگر مساوات کے دو مختلف حل دیکھ چکے ہیں:  لا متناہی چکور  کنواں اور ہارمونی مرتعش کے حل \ترچھا{معمول} پر لانے کے \ترچھا{قابل} تھے اور انہیں \ترچھا{غیر مسلسل اعشاریہ} \عددی{n} کے لحاظ سے نام دیا جاتا ہے؛  آزاد ذرے کے لیے یہ \ترچھا{معمول} پر لانے کے \ترچھا{قابل نہیں} ہیں اور انہیں \ترچھا{استمراری متغیر} \عددی{k} کے لحاظ  سے نام دیا جاتا ہے۔ اول الذکر بذات خود طبعی طور پر قابل حصول حل کو ظاہر کرتے ہیں جبکہ موخر الذکر  ایسا نہیں کرتے ہیں؛ تاہم دونوں صورتوں میں تابع وقت شروڈنگر  مساوات کے عمومی حل ساکن حالات کا خطی جوڑ ہو گا۔ پہلی قسم میں یہ جوڑ  (\عددی{n} پر لیا گیا) \ترچھا{مجموعہ} ہو گا، جبکہ دوسرے میں یہ (\عددی{k} پر) \ترچھا{تکمل} ہو گا۔ اس امتیاز کی طبعی اہمیت کیا ہے؟

 کلاسیکی میکانیات میں یک بعدی غیر تابع وقت مخفیہ دو مکمل طور پر مختلف حرکات پیدا کر سکتی ہے۔ اگر  \عددی{V(x)} ذرے کی کل توانائی \عددی{E} سے دونوں  جانب زیادہ بلند ہو(شکل  \حوالہء{2.12}) تب یہ ذرہ اس مخفی توانائی کے کنواں میں "پھنسا" رہے گا: یہ \اصطلاح{واپسی نقاط}\فرہنگ{واپسی نقاط}\حاشیہب{turning points}\فرہنگ{turning points} کے بیچ آگے پیچھے حرکت کرتا رہے گا اور کنواں سے باہر نہیں نکل سکے گا (ماسوائے اس صورت میں کہ آپ اسے اضافی توانائی فراہم کریں جس کی ابھی ہم بات نہیں کر رہے ہیں)۔ ہم اسے \اصطلاح{مقید حال}\فرہنگ{حال!مقید}\حاشیہب{bound state}\فرہنگ{state!bound} کہتے ہیں۔ اس کے برعکس اگر \عددی{E}  ایک (یا دونوں) جانب    \عددی{V(x)} سے  تجاوز کرے تب،  لامتناہی سے آتے ہوئے، مخفی توانائی کے زیر اثر ذرہ اپنی رفتار کم یا زیادہ  کرے گا اور اس کے بعد واپس لا متناہی کو لوٹے گا   (شکل  \حوالہء{2.12})۔ (یہ ذرہ مخفی توانائی میں پھنس نہیں سکتا ہے، ماسوائے اس صورت  کہ اس کی توانائی (مثلاً رگڑ کی بنا) گھٹے، لیکن ہم یہاں بھی ایسی صورت کی بات نہیں کر رہے ہیں۔) ہم اسے \اصطلاح{بکھراو حال}\فرہنگ{حال!بکھراو}\حاشیہب{scattering state}\فرہنگ{state!scattering} کہتے ہیں۔ بعض مخفی توانائیاں صرف مقید حال پیدا کرتی ہیں (مثلاً ہارمونی  مرتعش)؛  بعض صرف بکھراو حال پیدا کرتی  ہیں (مثلاً  پہاڑ مخفیہ  جو کہیں پر بھی نیچے نہ جھکتا ہو)؛ اور بعض، ذرہ کی توانائی پر منحصر، دونوں اقسام کے حال پیدا کرتی ہیں۔

 شروڈنگر مساوات کے حلوں  کے دو اقسام ٹھیک انہیں مقید اور بکھراو حال کو ظاہر کرتی ہیں۔ کوانٹم کے دائرہ کار میں یہ فرق اس سے بھی زیادہ واضح ہے جہاں  \اصطلاح{سرنگ زنی}\فرہنگ{سرنگ زنی}\حاشیہب{tunneling}\فرہنگ{tunneling}   (جس پر ہم کچھ دیر میں بات کریں گے) ایک ذرے کو کسی بھی متناہی مخفیہ رکاوٹ کے اندر سے گزرنے دیتی ہے، لہٰذا مخفیہ کی قیمت صرف لامتناہی پر اہم ہو گی   (شکل  \حوالہء{c 2.12})۔
\begin{align}
\begin{cases}
E<[V(-\infty)\, \text{اور}\, V(+\infty)]\Rightarrow \text{\RL{مقید حال}}\\
E>[V(-\infty)\, \text{یا}\, V(+\infty)]\Rightarrow \text{\RL{بکھراو حال}}
\end{cases}
\end{align} 
"روز مرہ زندگی" میں  لامتناہی پر عموماً مخفیہ صفر  کو پہنچتی ہیں۔ ایسی صورت میں مسلمہ معیار مزید سادہ صورت اختیار کرتی ہے:
\begin{align}
\begin{cases}
E<0\Rightarrow \text{\RL{مقید حال}}\\
E>0\Rightarrow \text{\RL{بکھراو حال}}
\end{cases}
\end{align}
چونکہ   \عددی{x\to\pm\infty} پر لامتناہی چکور  کنواں اور ہارمونی مرتعش کی مخفی توانائیاں لامتناہی کو پہنچتی ہیں لہٰذا یہ صرف مقید حالات پیدا کرتی ہیں جبکہ آزاد ذرے کی مخفی توانائی ہر مقام پر صفر ہوتی ہے لہٰذا  یہ صرف بکھراو حال\حاشیہد{آپ کو یہاں پریشانی کا سامنا ہو سکتا ہے کیونکہ عمومی مسئلہ جس کے لئے \عددی{E>V_{\text{کمتر}}} درکار ہے (سوال \حوالہء{2.3})، بکھراو حال، جو معمول پر لانے کے قابل نہیں ہیں، پر لاگو نہیں ہو گا۔ اگر آپ اس سے مطمئن نہیں ہیں تب \عددی{E\le 0} کے لئے مساوات شروڈنگر کو آزاد ذرہ کے لئے حل کر کے دیکھیں کہ اس کے خطی جوڑ بھی معمول پر لانے کے قابل نہیں ہیں۔ صرف مثبت مخفی توانائی  حل \ترچھا{مکمل} سلسلہ دیں گے۔} پیدا کرتی ہے۔  اس حصہ میں (اور اگلے حصہ میں) ہم ایسی مخفی توانائیوں پر غور کریں گے جو  دونوں اقسام کے حالات پیدا کرتی ہیں۔ 

\جزوحصہ{ڈیلٹا تفاعل کنواں} 
مبدا پر  لا متناہی کم چوڑائی اور لامتناہی بلند ایسا نوکیلا تفاعل  جس کا رقبہ اکائی ہو  (شکل \حوالہء{2.13}) \اصطلاح{ڈیلٹا تفاعل}\فرہنگ{تفاعل!ڈیلٹا}\حاشیہب{Dirac delta function}\فرہنگ{function!Dirac delta} کہلاتا ہے۔ 
\begin{align}\label{مساوات_شروڈنگر_ڈیلٹا_تفاعل_تعریف}
\delta(x)=&
\begin{cases}
0,& x\neq 0\\
\infty, &  x=0
\end{cases}
&&
\int_{-\infty}^{+\infty}\delta(x)\dif{x}=1
\end{align} 
نقطہ   \عددی{x=0} پر یہ تفاعل  متناہی نہیں ہے  لہٰذا تکنیکی طور پر اس کو تفاعل کہنا غلط ہو گا (ریاضی دان اسے \اصطلاح{\footnotesize{متعمم تفاعل}}\فرہنگ{متعمم!تفاعل}\حاشیہب{generalized function}\فرہنگ{generalized!function} یا \اصطلاح{\footnotesize{متعمم تقسیم}}\فرہنگ{متعمم!تقسیم}\حاشیہب{generalized distribution}\فرہنگ{generalized!distribution} کہتے ہیں)۔\حاشیہد{ڈیلٹا تفاعل کو ایسے مستطیل (یا مثلث) کی تحدیدی صورت تصور کیا جا سکتا ہے جس کی چوڑائی بتدریج کم اور قد بتدریج بڑھتا ہو۔} تاہم اس کا تصور نظریہ طبیعیات  میں اہم کردار ادا کرتا ہے۔  (مثال کے طور پر، برقی حرکیات کے میدان میں نقطی بار کی کثافت بار ایک ڈیلٹا تفاعل ہو گا۔) آپ دیکھ سکتے ہیں کہ  \عددی{\delta(x-a)} نقطہ \عددی{a}  پر اکائی رقبہ کا نوکیلی تفاعل ہو گا۔  چونکہ  \عددی{\delta(x-a)} اور ایک سادہ تفاعل \عددی{f(x)} کا حاصل ضرب نقطہ \عددی{a}    کے علاوہ ہر مقام پر صفر ہو گا لہٰذا \عددی{\delta(x-a)} کو  \عددی{f(x)} سے ضرب دینا، اسے \عددی{f(a)} سے ضرب دینے کے مترادف ہے:
\begin{align}
f(x)\delta(x-a)=f(a)\delta(x-a)
\end{align}
بالخصوص درج ذیل لکھا جا سکتا ہے جو  ڈیلٹا تفاعل کی اہم ترین خاصیت ہے۔ 
\begin{align}\label{مساوات_شروڈنگر_ڈیلٹا_تفاعل_اٹھاتا_ہے}
\int_{-\infty}^{+\infty}f(x)\delta(x-a)\dif{x}=f(a)\int_{-\infty}^{+\infty}\delta(x-a)\dif{x}=f(a)
\end{align}
 تکمل کی علامت کے اندر یہ نقطہ \عددی{a}   پر تفاعل \عددی{f(x)} کی قیمت "اٹھاتا" ہے۔ (لازمی نہیں کہ تکمل  \عددی{-\infty} تا   \عددی{+\infty} ہو،  صرف اتنا ضروری ہے کہ تکمل کے دائرہ کار میں نقطہ \عددی{a} شامل ہو لہٰذا \عددی{a-\epsilon} تا \عددی{a+\epsilon} تکمل لینا کافی ہو گا جہاں \عددی{\epsilon>0} ہے۔) 

آئیں درج ذیل روپ کے مخفیہ پر غور کریں جہاں  \عددی{\alpha} ایک مثبت مستقل ہے۔\حاشیہد{ڈیلٹا تفاعل کی اکائی ایک بٹا لمبائی ہے (مساوات \حوالہ{مساوات_شروڈنگر_ڈیلٹا_تفاعل_تعریف} دیکھیں) لہٰذا \عددی{\alpha} کا بعد توانائی ضرب لمبائی ہو گا۔}
\begin{align}\label{مساوات_شروڈنگر_کنواں_مخفیہ}
V(x)=-\alpha\delta(x)
\end{align}
یہ جان لینا ضروری ہے کہ (لامتناہی چکور  کنواں کی مخفیہ کی طرح) یہ ایک مصنوعی مخفیہ ہے، تاہم اس کے ساتھ کام کرنا نہایت آسان  ہے، اور جو کم سے کم  تحلیلی پریشانیاں پیدا کیے بغیر،  بنیادی نظریہ پر روشنی ڈالنے میں مددگار ثابت ہوتا ہے۔  ڈیلٹا تفاعل کنواں کے لیے شروڈنگر مساوات درج ذیل روپ اختیار کرتی ہے
\begin{align}
-\frac{\hslash^{2}}{2m}\frac{\dif^{\,2}\psi}{\dif{x^{2}}}-\alpha\delta(x)\psi=E\psi
\end{align} 
جو مقید حالات  \عددی{(E< 0)} اور بکھراو حالات  \عددی{(E> 0)} دونوں پیدا کرتی ہے۔

 ہم پہلے مقید حالات پر غور کرتے ہیں۔  خطہ \عددی{x< 0} میں  \عددی{V(x)=0} ہو گا لہٰذا 
\begin{align}\label{مساوات_شروڈنگر_ڈیلٹا_تفاعل_الف}
\frac{\dif^{\,2}\psi}{\dif{x^{2}}}=-\frac{2mE}{\hslash^{2}}\psi=k^{2}\psi
\end{align}
لکھا جا سکتا ہے جہاں \عددی{k} درج ذیل ہے (مقید حال کے لئے \عددی{E} منفی ہو گا لہٰذا \عددی{k}  حقیقی اور مثبت ہے۔)
\begin{align}\label{مساوات_شروڈنگر_تعریف_کے}
k\equiv\frac{\sqrt{-2mE}}{\hslash}
\end{align}
  مساوات \حوالہ{مساوات_شروڈنگر_ڈیلٹا_تفاعل_الف} کا عمومی حل 
\begin{align}
\psi(x)=Ae^{-kx}+Be^{kx}
\end{align}
ہو گا جہاں  \عددی{x\to-\infty} پر پہلا جزو لا متناہی کی طرف بڑھتا ہے لہٰذا ہمیں \عددی{A=0} منتخب کرنا ہو گا: 
\begin{align}
\psi(x)&=Be^{kx},&& (x<0)
\end{align}
خطہ  \عددی{x>0} میں بھی \عددی{V(x)} صفر ہے اور عمومی حل \عددی{F e^{-kx}+G e^{kx}}   ہو گا؛ اب \عددی{x\to +\infty} پر دوسرا جزو لامتناہی کی طرف بڑھتا ہے  لہٰذا \عددی{G=0} منتخب کرتے ہوئے درج ذیل لیا جائے گا۔
\begin{align}
\psi(x)&=Fe^{-kx},&& (x>0)
\end{align}
 ہمیں   نقطہ \عددی{x=0} پر سرحدی شرائط استعمال کرتے ہوئے ان دونوں تفاعل کو ایک دوسرے کے ساتھ جوڑنا ہو گا۔ میں     \عددی{\psi} کے معیاری سرحدی شرائط پہلے بیان کر چکا ہوں
\begin{align}
\begin{cases}
1. \quad\psi & \text{\RL{لازماً استمراری}}\\
2. \quad \frac{\dif{\psi}}{\dif{x}} & \text{\RL{استمراری، ماسوائے ان نقاط پر جہاں مخفیہ لامتناہی ہو}}
\end{cases}
\end{align}
 یہاں اول سرحدی شرط کے تحت \عددی{F=B}  ہو گا لہٰذا درج ذیل ہو گا۔ 
\begin{align}\label{مساوات_شروڈنگر-ڈیلٹا_تفاعل_حال}
\psi(x)=
\begin{cases}
Be^{kx},&(x\le0)\\
Be^{-kx},&(x\ge0)
\end{cases}
\end{align} 

تفاعل \عددی{\psi(x)}  کو شکل \حوالہء{2.14} میں ترسیم کیا گیا ہے۔ دوم سرحدی شرط  ہمیں ایسا کچھ نہیں بتاتی ہے؛  (لا متناہی چکور  کنواں کی طرح)  جوڑ پر مخفیہ لا متناہی ہے اور تفاعل کی ترسیل سے واضح ہے کہ  \عددی{x=0} پر اس میں بل پایا جاتا ہے۔ مزید اب تک کی کہانی میں ڈیلٹا تفاعل کا کوئی کردار نہیں پایا گیا۔ ظاہر ہے کہ \عددی{x=0}  پر \عددی{\psi} کے تفرق میں عدم استمرار یہی ڈیلٹا تفاعل تعین کرے گا۔ میں یہ عمل آپ کو کر کے دکھاتا ہوں جہاں آپ یہ بھی دیکھ پائیں  گے کہ کیوں \عددی{\tfrac{\dif \psi}{\dif x}} عموماً  استمراری ہوتا ہے۔ 

\begin{align}
-\frac{\hslash^2}{2m}\int_{-\epsilon}^{+\epsilon}\frac{\dif^{\,2}\psi}{\dif x^2}\dif x+\int_{-\epsilon}^{+\epsilon}V(x)\psi(x)\dif x=E\int_{-\epsilon}^{+\epsilon}\psi(x)\dif x
\end{align}
پہلا تکمل درحقیقت  دونوں آخری نقاط پر \عددی{\tfrac{\dif\psi}{\dif x}} کی قیمتیں ہوں گی؛ آخری تکمل اس پٹی کا رقبہ ہو گا، جس کا قد متناہی، اور \عددی{\epsilon\to 0} کی تحدیدی صورت میں، چوڑائی صفر کو پہنچتی ہو، لہٰذا یہ تکمل صفر ہو گا۔ یوں درج ذیل ہو گا۔
\begin{align}
\Delta\Big(\frac{\dif\psi}{\dif x}\Big)\equiv \left.\frac{\partial \psi}{\partial x}\right|_{+\epsilon}-\left.\frac{\partial \psi}{\partial x}\right|_{-\epsilon}=\frac{2m}{\hslash^2}\lim_{\epsilon\to0}\int_{-\epsilon}^{+\epsilon}V(x)\psi(x)\dif x
\end{align}
 عمومی طور پر دائیں ہاتھ پر حد صفر کے برابر ہو گا لہٰذا \عددی{\tfrac{\dif\psi}{\dif x}} عموماً استمراری ہو گا۔ لیکن جب سرحد پر \عددی{V(x)} لامتناہی ہو تب یہ دلیل قابل قبول نہیں ہو گی۔ بالخصوص \عددی{V(x)=-\alpha\delta(x)} کی صورت میں مساوات \حوالہ{مساوات_شروڈنگر_ڈیلٹا_تفاعل_اٹھاتا_ہے}  درج ذیل دے گی: 
\begin{align}\label{مساوات_شروڈنگر_مبدا_پر_تفرقات_کا_فرق}
\Delta\Big(\frac{\dif \psi}{\dif x}\Big)=-\frac{2m\alpha}{\hslash^2}\psi(0)
\end{align}
یہاں درج ذیل ہو گا (مساوات \حوالہ{مساوات_شروڈنگر-ڈیلٹا_تفاعل_حال}):
\begin{align*}
\begin{cases}
\frac{\dif \psi}{\dif x}=-Bke^{-kx},&(x>0)\quad \implies \left.\frac{\dif\psi}{\dif x}\right|_{+}=-Bk\\
\frac{\dif \psi}{\dif x}=+Bke^{+kx},&(x<0)\quad \implies \left.\frac{\dif\psi}{\dif x}\right|_{-}=+Bk
\end{cases}
\end{align*}
لہٰذا \عددی{\Delta(\dif\psi/\dif x)=-2Bk} ہو گا۔ساتھ ہی \عددی{\psi(0)=B} ہے۔ اس طرح مساوات \حوالہ{مساوات_شروڈنگر_مبدا_پر_تفرقات_کا_فرق}  درج ذیل کہتی ہے:
\begin{align}
k=\frac{m\alpha}{\hslash^2 }
\end{align}
اور اجازتی توانائیاں درج ذیل ہوں گی (مساوات \حوالہ{مساوات_شروڈنگر_تعریف_کے})۔
\begin{align}
E=-\frac{\hslash^2 k^2}{2m}=-\frac{m\alpha^2}{2\hslash^2}
\end{align}
 آخر میں \عددی{\psi} کو معمول پر لاتے ہوئے 
\begin{align*}
\int_{-\infty}^{+\infty}\abs{\psi(x)}^2\dif x=2\abs{B}^2\int_0^{\infty}e^{-2kx}\dif x=\frac{\abs{B}^2}{k}=1
\end{align*}
 (اپنی آسانی کے لیے مثبت حقیقی جذر کا انتخاب کر کے) درج ذیل حاصل ہو گا۔
\begin{align}
B=\sqrt{k}=\frac{\sqrt{m\alpha}}{\hslash}
\end{align}
آپ دیکھ سکتے ہیں کہ ڈیلٹا تفاعل،کی  "زور" \عددی{\alpha} کے قطع نظر،  ٹھیک ایک مقید حال دیتا ہے۔ 
\begin{align}\label{مساوات_شروڈنگر_مقید_حال_ڈیلٹا}
\psi(x)&=\frac{\sqrt{m\alpha}}{\hslash}e^{-m\alpha\abs{x}/\hslash^2};&&E=-\frac{m\alpha^2}{2\hslash^2}
\end{align}

ہم  \عددی{E>0} کی صورت میں \ترچھا{بکھراو حالات} کے بارے میں کیا کہہ سکتے ہیں؟ شروڈنگر مساوات \عددی{x<0} کے لئے  درج ذیل روپ اختیار کرتی ہے
\begin{align*}
\frac{\dif^{\,2}\psi}{\dif x^2}=-\frac{2mE}{\hslash^2}\psi=-k^2\psi
\end{align*}
 جہاں
\begin{align}\label{مساوات_شروڈنگر_مستقل_کے}
k\equiv\frac{\sqrt{2mE}}{\hslash}
\end{align}
 حقیقی اور مثبت ہے۔ اس کا عمومی حل  درج ذیل ہے
\begin{align}\label{مساوات_شروڈنگر_بایاں_حل_ڈیلٹا}
\psi(x)=Ae^{ikx}+Be^{-ikx}
\end{align}
جہاں کوئی بھی جزو بے قابو نہیں بڑھتا ہے لہٰذا انہیں رد نہیں کیا جا سکتا ہے۔ اسی طرح \عددی{x>0} کے لئے درج ذیل ہو گا۔
\begin{align}\label{مساوات_شروڈنگر_دایاں_حل_ڈیلٹا}
\psi(x)=Fe^{ikx}+Ge^{-ikx}
\end{align}
نقطہ \عددی{x=0} پر \عددی{\psi(x)} کے استمرار کی بنا درج ذیل ہو گا۔
\begin{align}\label{مساوات_شروڈنگر_شرط_اول}
F+G=A+B
\end{align}
تفرقات درج ذیل ہوں گے۔
\begin{align*}
\begin{cases}
\frac{\dif\psi}{\dif x}=ik(Fe^{ikx}-Ge^{-ikx}),& (x>0),\implies \left.\frac{\dif\psi}{\dif x}\right|_{+}=ik(F-G)\\
\frac{\dif\psi}{\dif x}=ik(Ae^{ikx}-Be^{-ikx}),& (x<0),\implies \left.\frac{\dif\psi}{\dif x}\right|_{-}=ik(A-B)
\end{cases}
\end{align*}
لہٰذا \عددی{\Delta(\dif\psi/\dif x)=ik(F-G-A+B)} ہو گا۔ساتھ ہی \عددی{\psi(0)=(A+B)} ہو گا لہٰذا دوسری سرحدی شرط (مساوات \حوالہ{مساوات_شروڈنگر_مبدا_پر_تفرقات_کا_فرق}) کہتی ہے
\begin{align}
ik(F-G-A+B)=-\frac{2m\alpha}{\hslash^2}(A+B)
\end{align}
یا مختصراً:
\begin{align}\label{مساوات_شروڈنگر_شرط_دوم}
F-G&=A(1+2i\beta)-B(1-2i\beta),&&\beta\equiv\frac{m\alpha}{\hslash^2 k}
\end{align}
دونوں سرحدی شرائط مسلط کرنے کے بعد ہمارے پاس دو مساوات (مساوات \حوالہ{مساوات_شروڈنگر_شرط_اول} اور \حوالہ{مساوات_شروڈنگر_شرط_دوم}) جبکہ چار نا معلوم مستقلات \عددی{A}، \عددی{B}، \عددی{C} اور \عددی{D} بلکہ \عددی{k} شامل کرتے ہوئے پانچ نا معلوم مستقل ہوں گے۔ یہ معمول پر لانے کے قابل حال نہیں ہے لہٰذا معمول پر لانا مدد گار ثابت نہیں ہو گا۔ بہتر ہو گا کہ ہم رک کر ان  مستقلات کی انفرادی طبعی اہمیت پر غور کریں۔  آپ کو یاد ہو گا کہ \عددی{e^{ikx}} (کے ساتھ  تابع وقت جزو ضربی \عددی{e^{-iEt/\hslash}} منسلک کرنے سے) دائیں رخ حرکت کرتا ہوا تفاعل موج پیدا ہوتا ہے۔ اسی طرح \عددی{e^{-ikx}} بائیں رخ حرکت کرتا ہوا موج دیتا ہے۔ یوں مساوات \حوالہ{مساوات_شروڈنگر_بایاں_حل_ڈیلٹا} میں مستقل \عددی{A} بائیں سے آمدی موج کا حیطہ ہے، \عددی{B} بائیں رخ واپس لوٹتے ہوئے موج کا حیطہ ہے، \عددی{F} (مساوات \حوالہ{مساوات_شروڈنگر_دایاں_حل_ڈیلٹا}) دائیں رخ نکل کر چلتے ہوئے موج کا حیطہ  جبکہ \عددی{H} دائیں سے آمدی موج کا حیطہ ہے (شکل \حوالہء{2.15} دیکھیں)۔ بکھراو کے عمومی تجربہ میں عموماً ایک رخ (مثلاً بائیں) سے ذرات پھینکے جاتے ہیں۔ ایسی صورت میں دائیں جانب سے آمدی موج کا حیطہ صفر ہو گا:
\begin{align}
G&=0,\quad\text{\RL{بائیں سے بکھراو}}
\end{align}
\اصطلاح{آمدی موج}\فرہنگ{موج!آمدی}\حاشیہب{incident wave}\فرہنگ{wave!incident} کا حیطہ \عددی{A}، \اصطلاح{منعکس موج}\فرہنگ{موج!منعکس}\حاشیہب{reflected wave}\فرہنگ{wave!reflected} کا حیطہ \عددی{B} جبکہ \اصطلاح{ترسیلی موج}\فرہنگ{موج!ترسیلی}\حاشیہب{transmitted wave}\فرہنگ{wave!transmitted} کا حیطہ \عددی{F} ہو گا۔   مساوات \حوالہ{مساوات_شروڈنگر_شرط_اول} اور  \حوالہ{مساوات_شروڈنگر_شرط_دوم}  کو \عددی{B} اور \عددی{F} کے لیے حل کر کے درج ذیل حاصل ہوں گے۔
\begin{align}
B=\frac{i\beta}{1-i\beta}A,\quad F=\frac{1}{1-i\beta}A
\end{align}
(اگر آپ دائیں سے بکھراو کا مطالعہ کرنا چاہیں تب \عددی{A=0} ہو گا؛ \عددی{G} آمدی حیطہ، \عددی{F} منعکس حیطہ، اور \عددی{B} ترسیلی حیطہ ہوں گے۔)

چونکہ کسی مخصوص مقام پر ذرے کی موجودگی کا احتمال \عددی{\abs{\psi}} ہوتا ہے لہٰذا آمدی  ذرہ کے انعکاس  کا   \ترچھا{تناسبی}\حاشیہد{یہ معمول پر لانے کے قابل تفاعل نہیں ہے لہٰذا کسی ایک مخصوص نقطہ پر ذرہ پایا جانے کا احتمال بے معنی ہو گا؛ بہر حال آمدی اور منعکس امواج کے احتمالات  کا تناسب معنی خیز ہے۔ اگلے پیراگراف میں اس پر مزید بات کی جائے گی۔} احتمال درج ذیل ہو گا

\begin{align}
R=\frac{\abs B^{2}}{\abs A^{2}}=\frac{\beta^{2}}{1+\beta^{2}}
\end{align}
جہاں  \عددی{ R}کو\اصطلاح{ شرح انعکاس}\فرہنگ{انعکاس!شرح}\حاشیہب{reflection coefficient}\فرہنگ{reflection!coefficient} کہتے ہیں۔ (اگر آپ کے پاس ذرات کی ایک شعاع ہو تو   \عددی{R} آپ کو بتائے گا کہ   ٹکرانے کے بعد ان میں سے کتنے ذرات  واپس لوٹ کر آئیں گے۔)  ترسیل کا  احتمال درج ذیل ہو گا جسے\اصطلاح{ شرح ترسیل}\فرہنگ{ترسیل!شرح}\حاشیہب{transmission coefficient}\فرہنگ{transmission!coefficient} کہتے ہیں۔
\begin{align}
T=\frac{\abs F^{2}}{\abs A^{2}}=\frac{1}{1+\beta^{2}}
\end{align}
 ظاہر ہے ان احتمال کا مجموعہ ایک \عددی{  (1)} ہو گا۔
  \begin{align}
  R+T=1
  \end{align}
 دھیان رہے کہ \عددی{ R  } اور  \عددی{  T } متغیر \عددی{ \beta  } کے لہٰذا  (مساوات \حوالہ{مساوات_شروڈنگر_مستقل_کے} اور \حوالہ{مساوات_شروڈنگر_شرط_دوم})    \عددی{ E  } کے تفاعل ہوں گے۔
  \begin{align}\label{مساوات_شروڈنگر_انعکاس_ترسیل_مستقل}
  R&=\frac{1}{1+\tfrac{2\hslash^{2}E}{m\alpha^{2}}},&& T =\frac{1}{1+\tfrac{m\alpha^{2}}{2\hslash^{2}E}}
  \end{align}  
  زیادہ  توانائی ترسیل کا احتمال بڑھاتی ہے جیسا کہ ظاہری طور پر ہونا چاہیے۔  
  
  یہاں تک باقی سب ٹھیک ہے لیکن ایک اصولی مسئلہ باقی  ہے جسے ہم نظرانداز نہیں کر سکتے ہیں.  چونکہ بکھراو   موج کے  تفاعل معمول پر لانے کے قابل نہیں ہیں لہٰذا یہ کسی صورت بھی حقیقی  ذرے کے حال کو ظاہر نہیں کر سکتے ہیں، لیکن ہم اس مسئلے کا حل جانتے ہیں۔ ہمیں ساکن حالات کے ایسے خطی جوڑ تیار کرنے ہونگے جو معمول پر لائے جانے کے قابل ہوں، جیسا ہم نے آزاد ذرہ  کے لیے کیا تھا۔   حقیقی طبی ذرات کو یوں تیار کردہ موجی اکٹھ  ظاہر کرے گا۔   یہ ظاہری طور پر سیدھا سادہ اصول ہے جو  عملی استعمال میں پیچیدہ ثابت ہوتا ہے لہٰذا یہاں سے آگے مسئلے کو کمپیوٹر کی مدد سے حل کرنا بہتر ہو گا۔\حاشیہد{کنواں اور  رکاوٹوں سے موجی اکٹھ کے  بکھراو کے اعدادی  مطالعہ  دلچسپ معلومات فراہم کرتے ہیں۔}  چونکہ  توانائی کی قیمتوں کا پورا سلسلہ استعمال کیے بغیر آزاد   ذرے کے تفاعل موج کو معمول پر نہیں لایا جا سکتا ہے لہٰذا   \عددی{ R  } اور  \عددی{  T }   کو (بالترتیب)   \عددی{ E  } کے قریب ذرات کی تخمینی شرح  انعکاس   اور شرح ترسیل سمجھنا چاہیے۔ 
  
   یہ ایک عجیب بات ہے کہ ہم لب لباب وقت کے تابع مسئلہ (جہاں ایک آمدی ذرہ مخفیہ سے  بکھر کر لامتناہی کی طرف رواں ہوتا ہے)  پر غور    \ترچھا{ ساکن   حالات} استعمال کرتے ہوئے کر پاتے  ہیں۔  آخر کار  (مساوات \حوالہ{مساوات_شروڈنگر_بایاں_حل_ڈیلٹا}  اور \حوالہ{مساوات_شروڈنگر_دایاں_حل_ڈیلٹا} میں) \عددی{  \psi }ایک مخلوط، غیر تابع وقت، سائن   نما تفاعل ہے جو (مستقل  حیطہ  کے ساتھ)  دونوں اطراف لا متناہی تک  پھیلا ہوا ہے۔ اس کے باوجود اس تفاعل پر موزوں سرحدی شرائط مسلط کر کے ہم  ایک ذرہ    (جسے مقامی موجی اکٹھ  سے ظاہر کیا گیا ہو)  کی مخفیہ سے انعکاس یا ترسیل کا احتمال تعین کر پاتے ہیں۔ اس ریاضیاتی کرامت کی وجہ میرے خیال میں یہ حقیقت ہے  کہ  ہم پوری  فضا میں پھیلے ہوئے تفاعل موج، جن کی تابعیت  وقت   نہ ہونے کے برابر ہو، کے خطی جوڑ لے کر ایک (حرکت پذیر)  نقطہ کے گرد ایسا تفاعل موج تیار کر سکتے ہیں جس پر  وقت کے لحاظ سے تفصیلًا غور کیا جا سکتا ہے(سوال \حوالہء{  2.43 })
  
متعلقہ مساوات جانتے ہوئے  آئیں ڈیلٹا تفاعل  رکاوٹ ( \حوالہء{   شکل2.16 }) کے مسئلہ پر غور کریں۔ ہمیں صرف \عددی{  \alpha}  کی علامت تبدیل کرنی   ہو گی۔ ظاہر ہے یہ تحدیدی حال کو ختم کرے گا  (\حوالہء{   سوال 2.2 })۔  دوسری جانب،   شرح  انعکاس اور شرح  ترسیل جو \عددی{  \alpha^2} پر منحصر  ہیں تبدیل نہیں ہوں گے۔ کتنی عجیب بات ہے کہ ایک ذرہ ایک رکاوٹ کے اندر سے یا ایک کنواں  کے اوپر سے ایک  جیسی  آسانی کے ساتھ گزرتا ہے۔   کلاسیکی طور پر جیسا کہ آپ جانتے ہیں،  ایک ذرہ کبھی بھی لا متناہی  قد کے رکاوٹ کو عبور نہیں کر سکتا،  چاہے اس کی توانائی کتنی ہی کیوں نہ ہو۔    حقیقتاً  کلاسیکی مسائل بکھراو  غیر دلچسپ ہوتے ہیں:  اگر \عددی{  E> V_{\text{بلندتر}}}  ہو تب \عددی{ R =0 } اور \عددی{  T=1 } ہو گا اور ذرہ ہر صورت رکاوٹ عبور کر پائے گا؛  اگر \عددی{ E< V_{\text{بلندتر}}}   ہو تب  \عددی{  T=0 } اور \عددی{ R =1 } ہو گا اور ذرہ  پہاڑی پر وہاں تک چڑھے گا جہاں تک اس میں دم ہو  اور اس  کے بعد اسی راستے واپس لوٹے گا۔  کوانٹائی بکھراو  زیادہ دلچسپ ہوتے ہیں:    اگر \عددی{ E< V_{\text{بلندتر}} }   ہو تب بھی ایک  ذرے کا مخفیہ  عبور کرنے کا احتمال غیر صفر ہو گا۔ اس مظہر کو  \اصطلاح{سرنگ زنی}\فرہنگ{سرنگ زنی}\حاشیہب{tunneling}\فرہنگ{tunneling}  کہتے ہیں جس پر جدید برقیات کا بیشتر حصہ منحصر ہے اور جو خوردبین میں حیرت انگیز  ترقی کے پشت پر ہے۔  اس کے برعکس  \عددی {  E> V_{\text{بلندتر}}} کی صورت میں  بھی ذرے کے انعکاس کا احتمال  غیر صفر   ہو گا؛  اگرچہ میں آپ کو کبھی بھی مشورہ نہیں دوں گا کہ چھت سے نیچے کودیں اور توقع رکھیں کہ کوانٹم میکانیات آپ کی جان بچا  پائے  گی ( سوال \حوالہء{   سوال 2.35 } دیکھیے گا)۔
 
\ابتدا{سوال}
درج ذیل تکملات کی قیمتیں تلاش کریں۔  
\begin{enumerate}[a.]
\item\(\int_{-3}^{+1}(x^{3}-3x^{2}+2x-1) \delta(x+2) \dif{x}\)
\item\( \int_{0}^{\infty}[\cos(3x)+2] \delta(x-\pi) \dif{x}\)
\item\(\int_{-1}^{+1}e^{(\abs x+3)} \delta (x-2) \dif{x}\)
\end{enumerate}
\انتہا{سوال}
\ابتدا{سوال}\شناخت{سوال_شروڈنگر_تفاعلات_برابر}
 ڈیلٹا تفاعلات  زیر علامت تکمل رہتے ہیں اور دو فقرے \عددی{ D_{1}(x) } اور \عددی{ D_{2}(x) }جو ڈیلٹا تفاعل پر مبنی ہیں صرف درج صورت میں ایک دوسرے کے برابر ہوں گے
\begin{align*}
\int_{-\infty}^{+\infty}f(x) D_1(x) \dif x =\int_{-\infty}^{+\infty}f(x) D_2(x) \dif x
\end{align*}
 جہاں  \عددی{f(x)} کوئی بھی سادہ تفاعل ہو سکتا ہے  ۔
\begin{enumerate}[a.]
\item 
درج ذیل دکھائیں
 \begin{align}\delta(cx)=\frac{1}{\abs{c}}\delta(x)\end{align}
جہاں \عددی{c } ایک حقیقی مستقل ہے۔ (منفی  \عددی{c}  کی صورت میں بھی تصدیق کریں۔) 
\item 
\اصطلاح{   سیڑھی تفاعل }\فرہنگ{سیڑھی تفاعل}\حاشیہب{step function}\فرہنگ{step function}\عددی{\theta(x)}  درج ذیل ہے ۔
\begin{align}
\theta(x)=
\begin{cases}
1&  x>0\\
0&  x<0
\end{cases}
\end{align}
 (اس نایاب صورت میں جہاں اس کی ضرورت پیش  آتی ہو ،  ہم \عددی{\theta(0)} کی تعریف \عددی{\frac{1}{2}} کرتے ہیں۔)  دکھائیں کہ   \(\dif{\theta}/\dif{x}=\delta(x)\) ہو گا۔
\end{enumerate}
\انتہا{سوال}
\ابتدا{سوال}
 عدم یقینیت کے اصول کو \حوالہ{مساوات_شروڈنگر_مقید_حال_ڈیلٹا}کے  تفاعل موج کے لئے پرکھیں۔ \ترچھا{ اشارہ} چونکہ \عددی{  \psi } کے تفرق کا\عددی{x=0} پر عدم استمرار پایا جاتا ہے لہٰذا\عددی{\langle p^2\rangle}  کا حساب پیچیدہ ہو گا۔ 
 سوال \حوالہ{سوال_شروڈنگر_تفاعلات_برابر}-ب  کا نتیجہ استعمال کریں۔ \ترچھا{جزوی جواب:} \عددی{\langle p^{2}\rangle =(m\alpha/\hslash)^{2}} 
\انتہا{سوال}
\ابتدا{سوال}\شناخت{سوال_شروڈنگر_ڈیلٹا_فوریئر_تبادل_کیا_ہے}
 تفاعل   \عددی{\delta(x)} کا فوریئر تبادل کیا  ہو گا؟ مسئلہ پلانشرل استعمال کر کے  درج ذیل دکھائیں۔ 
\begin{align}\label{مساوات_شروڈنگر_ڈیلٹا_پلانشرال}
\delta(x)=\frac{1}{2\pi}\int_{-\infty}^{+\infty} e^{ikx} \dif{k}
\end{align}
 \ترچھا{تبصرہ:} یہ کلیہ دیکھ کر  ایک   عزت مند ریاضی دان  پریشان  ضرور ہو گا۔   اگرچہ  \عددی{x=0} کے لئے  یہ تکمل  لامتناہی ہے اور \عددی{x\neq 0} کی صورت میں چونکہ  متکمل ہمیشہ کے لئے ارتعاش پذیر رہتا ہے لہٰذا یہ  (صفر  یا  کسی دوسرے عدد  کو)  مرکوز نہیں ہوتا ہے  ۔ اس کی پیوندکاری  کے طریقے پائے جاتے ہیں  (مثلاً،  ہم \عددی{-L} تا  \عددی{+L}  تکمل لے کر،     مساوات \حوالہ{مساوات_شروڈنگر_ڈیلٹا_پلانشرال} کو،    \عددی{L\to\infty} کرتے ہوئے  متناہی تکمل  کی اوسط قیمت تصور کر سکتے ہیں)۔ یہاں دشواری کا سبب یہ ہے کہ  مسئلہ   پلانشرل کے (مربع تکملیت) کی   بنیادی شرط   کو  ڈیلٹا تفاعل مطمئن  نہیں کرتا ہے(صفحہ \حوالہصفحہ{حاشیہ_شروڈنگر_مربع_تکملیت} پر مربع تکملیت کی شرط حاشیہ میں پیش کی گئی ہے)۔   اس کے باوجود  مساوات \حوالہ{مساوات_شروڈنگر_ڈیلٹا_پلانشرال} نہایت مددگار ثابت ہو سکتا ہے اگر اس کو احتیاط سے استعمال کیا جائے۔
\انتہا{سوال} 
\ابتدا{سوال}\شناخت{سوال_شروڈنگر_جڑواں_ڈیلٹا}
درج ذیل جڑواں ڈیلٹا تفاعل مخفیہ  پر غور کریں جہاں\عددی{  \alpha }  اور \عددی{  a}  مثبت مستقل  ہیں۔
\begin{align*}
V(x)=-\alpha[\delta(x+a)+\delta(x-a)]
\end{align*}
\begin{enumerate}[a.]
\item
 اس مخفیہ کا خاکہ کھینچیں۔
\item
  یہ کتنی مقید حالات پیدا کرتا ہے؟ \عددی{\alpha=\hslash^2/ma } اور \عددی{\alpha=\hslash^2/4ma }کیلئے اجازتی توانائیاں تلاش کریں اور تفاعلات موج کا خاکہ کھینچیں۔
\end{enumerate}
 \انتہا{سوال}
\ابتدا{سوال}
جڑواں ڈیلٹا تفاعل کے مخفیہ ( سوال\حوالہ{سوال_شروڈنگر_جڑواں_ڈیلٹا}) کے لئے  شرح ترسیل تلاش کریں۔ 
 \انتہا{سوال}

\حصہ{متناہی چکور  کنواں}
 ہم آخری مثال کے طور پر متناہی چکور  کنواں  کا مخفیہ
\begin{align}\label{مساوات_شروڈنگر_متناہی_چکور_کنواں_مخفیہ}
V(x)=
 \begin{cases} 
     -V_{0} & -a< x< a \\
      0 & \abs{x}> a
   \end{cases}
\end{align}
لیتے ہیں جہاں\عددی{ V_{0} } ایک( مثبت) مستقل ہے (شکل   \حوالہء{  2.17 })۔  ڈیلٹا تفاعل کنواں کی طرح یہ مخفیہ  مقید حالات ( جہاں\عددی{  E< 0 }  ہو گا )  کے ساتھ ساتھ بکھراو  حالات ( جہاں\عددی{  E> 0 } ہو گا) بھی پیدا  کرتا ہے۔ ہم پہلے مقید حالات پر غور کرتے ہیں۔
 
 خطہ\عددی{  x< -a } میں   جہاں مخفیہ  صفر ہے،  شروڈنگر  مساوات درج  ذیل روپ اختیار کرتی ہے
\begin{align*}
\frac{\dif^{\,2}\psi}{\dif{x^{2}}}=\kappa^{2}\psi \quad \text{یا}\quad -\frac{\hslash^{2}}{2m} \frac{\dif^{\,2} \psi}{\dif{x^{2}}}=E\psi
\end{align*} 
جہاں
 \begin{align}\label{مساوات_شروڈنگر_مستقل_کپا}
 \kappa\equiv \frac{\sqrt{-2mE}}{\hslash}
 \end{align}
 حقیقی اور مثبت ہے۔  اس کا عمومی حل\عددی{\Psi(x)=A e^{-kx}+ B e^{kx} }  ہے لیکن          \عددی{ x\to -\infty } کے صورت میں اس کا پہلا جزو بے قابو  بڑھتا ہے  لہٰذا ( ہمیشہ  طرح؛  مساوات\حوالہء{  2.119   } دیکھیں)   طبی طور پر قابل قبول حل درج ذیل ہو گا۔
\begin{align} 
\psi(x)&=Be^{kx}, && x<-a
\end{align}
 خطہ\عددی{  -a< x < a }  میں جہاں \عددی{ V(x)=-V_{0} } ہے مساوات  شروڈنگر درج ذیل روپ اختیار کرے گی
\begin{align*}
\frac{\dif ^{\,2}\psi}{\dif{x^2}}=-l^2\psi \quad \text{یا}\quad -\frac{\hslash^2}{2m}\frac{\dif^{\,2}\psi}{\dif x^2}=-V_0\psi
 \end{align*}
 جہاں \عددی{l} درج ذیل ہے۔
 \begin{align}\label{مساوات_شروڈنگر_مستقل_ایل}
 l\equiv \frac{\sqrt{2m(E+V_{0})}}{\hslash}  
 \end{align}
 اگرچہ مقید حالات کے لئے \عددی{ E } منفی ہے تاہم  \عددی{E>  V_{\text{کمتر}}} کی بنا  (  سوال\حوالہء{  2.2   }  دیکھیں)       اس کو  \عددی{ -V_{0} } سے بڑا ہونا  ہو گا ؛  لہٰذا \عددی{  l} بھی حقیقی اور مثبت ہو گا۔ اس کا عمومی حل\حاشیہد{آپ چاہیں تو  عمومی حل کو قوت نمائی روپ \عددی{} میں  لکھ سکتے ہیں۔اس سے بھی وہی اختتامی نتائج حاصل ہوں گے، تاہم تشاکلی  مخفیہ  کی بنا  ہم جانتے ہیں کہ حل جفت یا طاق ہوں گے، اور \عددی{\sin} اور \عددی{\cos} کا استعمال  اس حقیقت کو  بلا واسطہ بروئے کار لا سکتا ہے۔}
 \begin{align}
 \psi(x)&=C\sin(lx)+D\cos(lx), && -a<x<a
 \end{align}
 جہاں\عددی{  C } اور\عددی{ D } اختیاری مستقلات  ہیں۔  آخر میں، خطہ \عددی{x>a}  جہاں ایک بار پھر مخفیہ صفر ہے؛  عمومی حل\عددی{ \psi(x)=F e^{-\kappa x}+G  e^{\kappa x} } ہو گا لیکن یہاں\عددی{ x\to \infty} کی صورت میں دوسرا  جزو  بے قابو بڑھتا ہے لہٰذا قابل قبول حل درج ذیل ہو گا۔
\begin{align}
 \psi(x)&=F e^{-\kappa x}, && x>a
\end{align}

 اگلے قدم میں ہمیں سرحدی شرائط مسلط کرنے ہوں گے:  \عددی{ \psi }اور \عددی{ \frac{\dif{\psi}}{\dif{x}}  } نقاط  \عددی{ -a  } اور\عددی{  +a }  پر استمراری ہیں۔  یہ جانتے ہوئے کہ دیا گیا  مخفیہ جفت تفاعل ہے، ہم کچھ وقت بچا سکتے ہیں اور  فرض کر سکتے ہیں کہ حل مثبت یا  طاق  ہوں گے ( سوال \حوالہء{  2.1ج   })۔ اس کا فائدہ  یہ  ہے کہ ہمیں صرف ایک جانب (    مثلاً\عددی{ +a  } ) پر    سرحدی شرائط مسلط کرنی ہوں  گی؛   چونکہ\عددی{ \psi(-x)=\pm \psi(x)  } ہے لہٰذا  دوسری جانب کا حل  ہمیں خود  بخود   حاصل ہو گا۔  میں جفت حل حاصل کرتا ہوں جبکہ آپ کو سوال  \حوالہء{   2.29  }میں طاق   حل تلاش کرنے ہونگے۔ \عددی{ \cos}جفت ہے (جبکہ\عددی{ \sin} طاق ہے)  لہٰذا  میں درج ذیل  روپ کے حلوں  کی تلاش میں ہوں۔ 
\begin{align}\label{مساوات_شروڈنگر_تلاش_تفاعلات}
\psi(x)=
\begin{cases}
Fe^{-\kappa x} & x> a\\
D\cos(l x) & 0< x < a\\
\psi(-x) & x< 0
\end{cases}
\end{align}
نقطہ\عددی{ x=a  } پر  \عددی{\psi(x)}  کی استمرار  درج ذیل کہتی ہے  
\begin{align}\label{مساوات_شروڈنگر_استمرار_مستقل_الف}
 Fe^{-\kappa a}=D\cos(la) 
 \end{align}
 جبکہ \عددی{\frac{\dif{\psi}}{\dif{x}}   } کی استمرار درج ذیل  کہتی ہے
 \begin{align}\label{مساوات_شروڈنگر_استمرار_مستقل_ب}
 -\kappa Fe^{-\kappa a}=-lD\sin(la) 
 \end{align} 
 مساوات    \حوالہ{مساوات_شروڈنگر_استمرار_مستقل_ب} کو مساوات \حوالہ{مساوات_شروڈنگر_استمرار_مستقل_الف} سے تقسیم کرتے ہوئے درج ذیل  حاصل ہو گا۔

\begin{align}\label{مساوات_شروڈنگر_حل_کپا}
\kappa =l\tan(la) 
 \end{align}
 چونکہ \عددی{ \kappa}  اور \عددی{  l } دونوں \عددی{  E } کے تفاعل ہیں لہٰذا   اس  کلیہ سے  اجازتی توانائیاں  حاصل کی جا سکتی  ہیں۔اجازتی توانائی\عددی{ E  }  کے لئے حل کرنے سے پہلے ہم درج ذیل  بہتر علامتیں متعارف کرتے ہیں۔
\begin{align}
z\equiv l a \quad \text{اور}\quad z_0\equiv\frac{a}{\hslash}\sqrt{2mV_0}
\end{align} 
مساوات \حوالہ{مساوات_شروڈنگر_مستقل_کپا}  اور \حوالہ{مساوات_شروڈنگر_مستقل_ایل} کے تحت \عددی{ (\kappa^{2}+l^{2})=2mV_{0}/\hslash^{2}  } اور ہو گا لہٰذا  \عددی{ \kappa a=\sqrt{z_{0}^{2}-z^{2}}} ہو گا اور مساوات  \حوالہ{مساوات_شروڈنگر_حل_کپا}  درج ذیل  روپ اختیار کرے گی۔
\begin{align}
\tan z=\sqrt{(z_{0}/z)^{2}-1} 
\end{align}
 یہ \عددی{ z}  (لہٰذا  \عددی{ E})  کی ماورائی مساوات ہے جس کا متغیر \عددی{ z_{0}} ہے ( جو کنواں کی" جسامت" کی  ناپ ہے)۔  اس کو اعدادی طریقہ سے کمپیوٹر کے ذریعے حل کیا جا سکتا یا \عددی{ \tan z} اور \عددی{ \sqrt{(z_{0}/z)^{2}-1}} کو ایک ساتھ  ترسیم کر کے  ان کے نقاط تقاطع  لیتے ہوئے  حل کیا جا سکتا ہے ( شکل \حوالہء{ 2.18  })۔    دو تحدیدی صورتیں زیادہ دلچسپی کے حامل ہیں ۔ 
\begin{enumerate}[a.]
\item
 موٹا{ایک چوڑا اور  گہرا کنواں۔}\quad
 بہت بڑی  \عددی{ z_{0}}  کی صورت میں   طاق \عددی{n} کے لئے  نقاط تقاطع   \عددی{ z_{n}=n\pi/2}   سے  معمولی   نیچے ہوں گے؛   یوں درج ذیل  ہو گا۔
\begin{align}
E_{n}+V_{0}\cong \frac{n^{2}\pi^{2}\hslash^{2}}{2m(2a)^{2}}
 \end{align}
اب\عددی{E+V_{0}} کنواں کی تہہ کے  اوپر توانائی کو ظاہر کرتی ہے اور مساوات کا دایاں ہاتھ  ہمیں \عددی{2a} چوڑائی کے لامتناہی چکور  کنواں کی توانائیاں دیتا ہے  ( مساوات\حوالہ{مساوات_شروڈنگر_لامتناہی_چکور_کنواں_توانائیاں}  دیکھیں) ؛  بلکہ   \عددی{ n} یہاں طاق ہے لہٰذا     توانائیوں کی نصف تعداد   حاصل ہو گی۔ ( جیسا آپ سوال \حوالہء{ 2.29} میں دیکھیں گے کل توانائیوں کی باقی نصف تعداد  طاق  تفاعل موج سے حاصل ہو گی۔)    یوں\عددی{ V_{0}\to \infty } کرنے سے متناہی چکور  کنواں سے لامتناہی چکور  کنواں حاصل ہو گا؛  تاہم  کسی بھی متناہی \عددی{ V_{0}} کی صورت میں مقید حالات کی تعداد متناہی ہو گی۔ 
\item 
\موٹا{ کم گہرا، کم چوڑا  کنواں}\quad 
 جیسے جیسے\عددی{ z_{0}}  کی قیمت کم کی جاتی  ہے مقید حالات کی تعداد کم سے کم ہوتی جاتی ہے  حتٰی  کہ آخر کار (\عددی{ (z_{0}< \pi/2)  } کیلئے جہاں کم ترین  طاق  حال  بھی نہیں پایا جاتا )  صرف ایک  مقید حال رہ جائے گا۔    دلچسپ بات یہ ہے،  کنواں  جتنا   بھی  "کمزور"     کیوں نہ ہو ،   ایک عدد مقید حال ضرور   پایا جائے گا۔
\end{enumerate}


 اگر  آپ  \عددی{ \psi  } (مساوات \حوالہ{مساوات_شروڈنگر_تلاش_تفاعلات})   کو معمول پر لانے میں دلچسپی رکھتے ہیں( سوال \حوالہء{  2.30  }) تو  ایسا ضرور کریں   جبکہ  میں اب بکھراو  حالات \عددی{ E> 0  } کی طرف  بڑھنا چاہوں گا ۔  ہوں بائیں ہاتھ جہاں\عددی{ V(x)=0} ہے درج ذیل ہو گا 
\begin{align}
\psi(x)&=Ae^{i k x}+Be^{-i k x} && (x<-a) 
\end{align}
 جہاں  ہمیشہ  کی طرح درج ذیل ہو گا۔
 \begin{align}
 k\equiv \frac{\sqrt{2mE}}{\hslash} 
 \end{align} 
  کنواں کے اندر جہاں\عددی{ V(x)=-V_{0}}  ہے درج ذیل ہو گا
\begin{align}
\psi(x)&=C\sin(lx)+D\cos(lx)&& (-a<x<a)
 \end{align}
 جہاں پہلے کی طرح درج ذیل ہو گا۔
  \begin{align}
  l\equiv \frac{\sqrt{2m(E+V_{0})}}{\hslash}
   \end{align}
     دائیں جانب جہاں ہم فرض کرتے ہیں کہ کوئی آمدی  موج نہیں پائی جاتی  درج ذیل ہو گا ۔
 \begin{align}
 \psi(x)=Fe^{i k x} 
 \end{align}
  یہاں  آمدی حیطہ \عددی{ A  }،  انعکاسی  حیطہ \عددی{ B  }   اور ترسیلی  حیطہ  \عددی{  F }  ہے۔\حاشیہد{مقید حالات  کی صورت میں ہم نے طاق اور جفت تفاعلات تلاش کیے۔ ہم یہاں بھی ایسا کر سکتے ہیں، تاہم   مسئلہ بکھراو  میں امواج صرف ایک رخ سے آتے  ہیں لہٰذا  یہ مسئلہ   ذاتی طور پر  غیر تشاکلی ہے اور سیاق و سباق کے لحاظ سے  (حرکت پذیر امواج کے اظہار  کے لئے )    قوت نمائی  علامت کا استعمال  زیادہ موثر ہے۔  } 
  
   یہاں چار سرحدی شرائط پائے جاتے ہیں:  نقطہ  \عددی{-a}  پر  \عددی{\psi(x)}  کے  استمرار   کے تحت  درج ذیل ہو گا
\begin{align}\label{مساوات_شروڈنگر_اے}
Ae^{-ika}+Be^{ika} = -C\sin(la)+D\cos(la)
 \end{align}
نقطہ\عددی{ -a  }  پر\عددی{\tfrac{\dif \psi}{\dif x}} کا استمرار درج ذیل دے گا
\begin{align}\label{مساوات_شروڈنگر_اے_بی}
ik[Ae^{-ika}-Be^{ika}] =l[C\cos(la)+D\sin(la)] 
\end{align}

نقطہ \عددی{ +a  } پر \عددی{ \psi(x)  }کا استمرار درج ذیل دے گا 
\begin{align}\label{مساوات_شروڈنگر_سی}
C\sin (la)+D\cos(la)]=Fe^{ika} 
\end{align}
اور \عددی{ +a  } پر\عددی{\frac{\dif{\psi}}{\dif{x}}   }  کا استمرار درج ذیل  دے گا۔
\begin{align}\label{مساوات_شروڈنگر_ڈی}
l[C\cos(la)-D\sin(la)]=ikFe^{ika} 
\end{align}
 ہم ان میں سے دو  استعمال کرتے ہوئے   \عددی{ C  }  اور \عددی{ D  }  خارج کر کے باقی دو  حل کر کے   \عددی{ B  }اور \عددی{ F} تلاش کر سکتے ہیں (سوال \حوالہء{  2.32   }دیکھیے گا)۔
\begin{align}
B&=i\frac{\sin(2la)}{2kl}(l^{2}-k^{2})F \label{مساوات_شروڈنگر_بی}\\
F&=\frac{e^{-2ika}A}{\cos(2la)-i\frac{(k^{2}+l^{2})}{2kl}\sin(2la)}\label{مساوات_شروڈنگر_ایف}
\end{align}
  شرح ترسیل\عددی{(T=\abs{F}^2/\abs{A}^2)  }کو   اصل متغیرات کی صورت    میں لکھتے ہوئے  درج ذیل  حاصل ہو گا۔

\begin{align}\label{مساوات_شروڈنگر_ترسیلی_حل}
T^{-1}=1+\frac{V_{0}^{2}}{4E(E+V_{0})}\sin^{2}\big(\frac{2a}{\hslash}\sqrt{2m(E+V_{0})} \big) 
 \end{align}
 دھیان رہے کہ  جہاں بھی سائن کی قیمت صفر ہو، یعنی درج ذیل نقطوں پر جہاں \عددی{n} عدد صحیح ہے 
 \begin{align}
 \frac{2a}{\hslash}\sqrt{2m(E_{n}+V_{0})}=n\pi 
 \end{align}
  وہاں\عددی{T=1}   (اور کنواں "شفاف" )   ہو گا  ۔ یوں مکمل ترسیل کے لیے درکار توانائیاں  درج ذیل ہوں گی
\begin{align}
E_{n}+V_{0}=\frac{n^{2}\pi^{2}\hslash^{2}}{2m(2a)^{2}} 
\end{align}
 جو عین لامتناہی چکور  کنواں کی اجازتی توانائیاں ہیں۔  شکل\حوالہء{ 2.19  }میں توانائی  کے لحاظ سے \عددی{T}  ترسیم کیا گیا ہے۔


%==============================================================

\ابتدا{سوال} 
متناہی چکور  کنواں کے طاق مقید حال کے  تفاعل  موج کا تجزیہ کریں۔ اجازتی  توانائیوں کی ماورائی مساوات اخذ کر کے اسے ترسیمی طور پر حل کریں ۔ اس کے  دونوں تحدیدی صورتوں پر غور کریں۔  کیا ہر صورت ایک طاق مقید حال پایا جائے گا ؟ 
\انتہا{سوال}
\ابتدا{سوال}
 مساوات\حوالہ{مساوات_شروڈنگر_تلاش_تفاعلات}  میں دیا گیا \عددی{ \psi(x)}  معمول پر  لا کر مستقل\عددی{  D }  اور\عددی{  F }  تعین  کریں۔
\انتہا{سوال}
\ابتدا{سوال}
ڈائی رک ڈیلٹا تفاعل کو ایک ایسی مستطیل کی  تحدیدی صورت تصور کیا  جا سکتا ہے،  جس کا  رقبہ اکائی  \عددی{(1)  } رکھتے ہوئے اس کی چوڑائی صفر  تک  اور قد لامتناہی تک پہنچائی جائے۔ دکھائیں کہ  ڈیلٹا تفاعل کنواں (مساوات\حوالہ{مساوات_شروڈنگر_کنواں_مخفیہ})   لامتناہی گہرا  ہونے کے باوجود  \عددی{z_{0}\to 0   }کی بنا ایک" کمزور" مخفیہ ہے۔   ڈیلٹا تفاعل مخفیہ  کو  متناہی چکور  کنواں کی تحدیدی  صورت لیتے ہوئے اس کی مقید حال کی توانائی تعین کریں۔ تصدیق کریں کہ آپ کا جواب مساوات\حوالہ{مساوات_شروڈنگر_مقید_حال_ڈیلٹا}  کے مطابق ہے۔   دکھائیں کہ موزوں حد کی صورت میں مساوات\حوالہ{مساوات_شروڈنگر_ترسیلی_حل}  کی تخفیف مساوات\حوالہ{مساوات_شروڈنگر_انعکاس_ترسیل_مستقل}  دے گی ۔
\انتہا{سوال}
\ابتدا{سوال}
مساوات \حوالہ{مساوات_شروڈنگر_بی} اور \حوالہ{مساوات_شروڈنگر_ایف} اخذ کریں۔\ترچھا{  اشارہ:}  مساوات \حوالہ{مساوات_شروڈنگر_سی} اور\حوالہ{مساوات_شروڈنگر_ڈی} سے  \عددی{ C  } اور \عددی{  D }  کو  \عددی{ F  } کی صورت میں حاصل  کر کے
\begin{align*}
C&=[\sin(la)+i\frac{k}{l}\cos(la)]e^{ika}F; && D=[\cos(la)-i\frac{k}{l}\sin(la)]e^{ika}F
 \end{align*}
 انہیں واپس مساوات \حوالہ{مساوات_شروڈنگر_اے}  اور \حوالہ{مساوات_شروڈنگر_اے_بی} میں پر کریں۔  شرح ترسیل حاصل کر کے مساوات\حوالہ{مساوات_شروڈنگر_مقید_حال_ڈیلٹا} کی تصدیق کریں۔
\انتہا{سوال}
\ابتدا{سوال}
مستطیلی رکاوٹ ( جسے  خطہ\عددی{ -a< x< a  }  میں\عددی{V_(x)=+V_{0}> 0} لینے سے مساوات \حوالہ{مساوات_شروڈنگر_متناہی_چکور_کنواں_مخفیہ}  دیتی ہے)   کے لئے  شرح ترسیل  تعین کریں۔  تین صورتوں  \عددی{ E< V_{0}} ،
\عددی{ E=V_{0}  }اور \عددی{ E> v_{0}  } کو علیحدہ علیحدہ حل کریں۔(  آپ دیکھیں گے کہ رکاوٹ کے اندر تینوں صورتوں میں تفاعل موج ایک دوسرے سے مختلف ہوں  گے۔)  \ترچھا{ جزوی جواب:} \عددی{ E< V_{0}}  کے لئے درج ذیل ہو گا۔\حاشیہد{یہ سرنگ زنی کی ایک اچھی مثال ہے۔کلاسیکی طور پر ذرہ رکاوٹ سے  ٹکرانے کے بعد واپس لوٹے گا۔}
\begin{align*}
T^{-1}=1+\frac{V_{0}^2}{4E(V_{0}-E)}\sinh^{2}\Big(\frac{2a}{\hslash}\sqrt{2m(V_{0}-E)}   \Big) 
\end{align*}
\انتہا{سوال}
\ابتدا{سوال}\شناخت{سوال_شروڈنگر_سیڑھی_رکاوٹ}
 درج ذیل سیڑھی مخفیہ  پر غور کریں۔
\begin{align*}
V(x)=
\begin{cases}
0 & x\le 0\\
V_{0}&x> 0
\end{cases}
\end{align*}
%
\begin{enumerate}[a.]
\item
  شرح انعکاس    \عددی{ E< V_{0}}  صورت کیلئے  حاصل کر کے  جواب پر تبصرہ کریں۔ 
\item
شرح  انعکاس  \عددی{E> V_{0}}   صورت کے لئے حاصل  کریں۔ 
\item
ایسے  مخفیہ کے لئے   جو رکاوٹ کے دائیں جانب  واپس صفر  نہیں ہو  جاتا،  ترسیلی موج کی رفتار مختلف ہو گی   لہٰذا    شرح ترسیل\عددی{ \abs{F}^{2}/\abs{A}^{2}  }  نہیں ہو گی ( جہاں\عددی{A}   آمدی حیطہ   اور \عددی{  F }ترسیلی  حیطہ  ہے)۔ دکھائیں  کہ \عددی{E>V_0} کے لئے درج ذیل ہو گا۔
\begin{align}
 T=\sqrt{\frac{E-V_{0}}{E}  }\frac{\abs{F}^{2}}{\abs{A}^{2}} 
 \end{align}
 \ترچھا{اشارہ:}  آپ اسے مساوات\حوالہ{مساوات_شروڈنگر_کلاسیکی_کوانٹائی_رفتار} سے حاصل کر سکتے ہیں؛  یا زیادہ  خوبصورتی لیکن  کم معلومات کے ساتھ   احتمال رو  (سوال\حوالہ{سوال_شروڈنگر_احتمال_بہاو_رو}) سے حاصل کر سکتے ہیں۔   \عددی{ E< V_{0}  } کی صورت میں \عددی{ T}  کیا ہو گا؟
\item

صورت \عددی{  E> V_{0} } کے لیے سیڑھی  مخفیہ کے لئے شرح ترسیل تلاش کر کے \عددی{T+R=1} کی تصدیق کریں۔
\end{enumerate}
\انتہا{سوال}
\ابتدا{سوال}
ایک ذرہ جس کی کمیت \عددی{ m}  اور حرکی توانائی \عددی{E>0} ہو مخفیہ کی  ایک اچانک گہرائی   ( شکل\حوالہء{  2.34  })  کی طرف بڑھتا ہے۔ 

\begin{enumerate}[a.]
\item
  صورت\عددی{ E=V_{0}/3  }     میں اس کے  انعکاس کا احتمال کیا ہو گا؟ \ترچھا{ اشارہ:}  یہ بالکل سوال  \حوالہ{سوال_شروڈنگر_سیڑھی_رکاوٹ} کی طرح ہے،  بس یہاں سیڑھی  اوپر کی  بجائے نیچے کو  ہے ۔
\item 
 میں نے مخفیہ کی  شکل و صورت  یوں  پیش کی ہے گویا  ایک گاڑی  افقی  چٹان سے نیچے گرنے والی ہے تاہم  ایسی کھائی سے  گاڑی کا ٹکرا کر واپس لوٹنے   کا احتمال جزو -ا کے نتیجہ سے بہت کم ہو گا ۔یہ مخفیہ  کیوں ایک افقی  چٹان کی صحیح ترجمانی نہیں کرتا ہے؟  \ترچھا{ اشارہ:}  شکل\حوالہء{  2.20  } میں جیسے ہی گاڑی نقطہ \عددی{ x=0}  پر سے گزرتی ہے ، اس کی توانائی عدم استمرار کے ساتھ گر کر\عددی{-V_{0}}  ہو جاتی ہے؛  کیا   یہ   نیچے  گرتے  ہوئے  ایک گاڑی کے لیے  درست  ہو گا ؟
\item
ایک  نیوٹران مرکزہ  میں داخل ہوتے ہوئے  مخفیہ میں اچانک کمی محسوس کرتا ہے۔ باہر \عددی{V=0} جبکہ مرکزہ کے اندر \عددی{V=\SI{-12}{\mega\electronvolt}} ہو تا ہے۔  فرض کریں  بذریعہ  انشقاق خارج  ایک نیوٹران جس کی حرکی توانائی \عددی{\SI{4}{\mega\electronvolt}} ہو ایک ایسے مرکزہ کو ٹکراتا ہے۔ اس نیوٹران کا جذب  ہو کر  دوسرا  انشقاق پیدا کرنے کا احتمال کیا ہو گا؟ \ترچھا{اشارہ:}  آپ نے جزو-ا میں انعکاس کا  احتمال تلاش کیا؛ کلیہ \عددی{T=1-R} استعمال کر  کے سطح سے ترسیل کا احتمال حاصل کریں۔
\end{enumerate}
\انتہا{سوال}
\ابتدا{سوال}
عین مبدا پر \عددی{-a< x< +a}  کے بیچ لامتناہی چکور کنواں کے اندر\عددی{ V(x)=0  } اور اس کے باہر\عددی{ V(x)=\infty}  ہے ۔ غیر  تابع وقت  شروڈنگر مساوات پر موزوں سرحدی شرائط مسلط کر کے اسے حل کریں۔ تصدیق کریں کہ آپ کی توانائیاں عین  میری حاصل کردہ توانائیوں    (مساوات\حوالہ{مساوات_شروڈنگر_لامتناہی_چکور_کنواں_توانائیاں}) کے مطابق ہیں     اور تصدیق کریں کہ میری \عددی{ \psi} (مساوات    \حوالہ{مساوات_شروڈنگر_میری_سائے})  میں    \عددی{ x\to (x+a)/2  } پر کر کے،  موزوں  معمول  زنی   سے آپ کی  تمام \عددی{\psi} حاصل   ہوتی ہیں۔ اپنے اولین تین حل ترسیم کریں اور ان کا موازنہ شکل \حوالہء{    2.2 } سے کریں۔  دھیان رہے کہ یہاں کنواں کی چوڑائی \عددی{ 2a}  ہے ۔ 
\انتہا{سوال}
\ابتدا{سوال}
 لامتناہی چکور کنواں( مساوات \حوالہ{مساوات_شروڈنگر_لامتناہی_چکور})   میں ایک ذرے کا  ابتدائی تفاعل موج درج ذیل ہے۔
 \begin{align*}
  \Psi(x,0)&=A\sin^{3}(\pi x/a) && (0\le x\le a)
\end{align*}

 مستقل\عددی{ A}    اور \عددی{ \Psi(x,t)} تلاش کر کے   وقت کے لحاظ سے  \عددی{\langle x \rangle}   کا حساب لگائیں۔ توانائی کی توقعاتی قیمت کیا   ہو گی؟   \ترچھا{ اشارہ :} \عددی{\sin^n\theta} اور  \عددی{ \cos^n\theta} کو تخفیف کے بعد\عددی{\sin(m\theta)}  اور \عددی{\cos(m\theta)}کے خطی جوڑ  لکھا جا سکتا ہے جہاں \عددی{ m=0,1,2,\dotsc ,n  }  ہو گا۔
\انتہا{سوال}
\ابتدا{سوال}
 کمیت\عددی{  m} کا ایک ذرہ لامتناہی چکور کنواں ( مساوات \حوالہ{مساوات_شروڈنگر_لامتناہی_چکور})  میں زمینی حال میں ہے ۔ اچانک کنویں کا دایاں دیوار \عددی{ a } سے  \عددی{  2a}  منتقل ہوتا ہے جس سے کنواں کی چوڑائی دگنی ہو جاتی ہے۔ لمحاتی طور پر اس عمل سے  تفاعل موج اثر انداز نہیں ہوتا۔ اس ذرہ کی توانائی کی پیمائش اب کی جاتی ہے۔
\begin{enumerate}
\item
 کونسا  نتیجہ سب سے زیادہ امکان رکھتا ہے؟ اس نتیجے کے حصول کا احتمال کیا ہو گا؟ 
\item
 کونسا نتیجہ اس کے بعد زیادہ امکان رکھتا ہے اور اس کا احتمال کیا ہو گا؟
\item 
توانائی کی توقعاتی قیمت کیا ہو گی؟ \ترچھا{ اشارہ:}  اگر   آپ کو لامتناہی تسلسل کا سامنا ہو تب کوئی دوسری ترکیب استعمال کریں۔
\end{enumerate} 
 \انتہا{سوال}
\ابتدا{سوال}
\begin{enumerate}
\item   
 دکھائیں کہ لامتناہی چکور کنواں میں ایک ذرہ کا  تفاعل موج کوانٹائی\اصطلاح{ تجدیدی عرصہ}\فرہنگ{تجدیدی   عرصہ}\حاشیہب{revival time}\فرہنگ{revival time}  \عددی{T=4ma^{2}/\pi \hslash }  کے بعد دوبارہ اپنے اصل روپ میں واپس آتا ہے۔  یعنی  ( نہ صرف ساکن حال  ) بلکہ کسی بھی حال کے لئے  \عددی{ \Psi(x,T)=\Psi(x,0)} ہوتا ہے۔ 
\item   
 دیواروں سے ٹکرا کر دائیں سے بائیں اور بائیں سے دائیں حرکت کرتے ہوئے  ایک ذرہ جس کی توانائی \عددی{ E}  ہو کا کلاسیکی تجدیدی عرصہ کیا ہو گا ؟ 
\item   
 کس توانائی کیلئے یہ  تجدیدی  عرصے  ایک دوسرے کے برابر ہوں گے؟
\end{enumerate} 
\انتہا{سوال}
%86
\ابتدا{سوال}
  ایک ذرہ جس کی کمیت \عددی{ m } ہے  درج ذیل مخفی کو میں پایا جاتا ہے۔ 
\begin{align*}
V(x)=
\begin{cases}
\infty & (x< 0)\\
-32\hslash^{2}/ma^{2} & (0\le x \le a)\\
0 & (x> a)
\end{cases}
\end{align*}

\begin{enumerate}
\item
 اس کے مقید حلوں کی تعداد کیا ہو گی ؟
\item 
 مقید حال میں سب سے زیادہ توانائی کی صورت میں کنواں کے باہر (\عددی{ x> a})   ذرہ پائے جانے کا احتمال کیا ہو گا ؟
\ترچھا{جواب:}  \عددی{ 0.542  }،  اگرچہ یہ کنواں میں مقید ہے،  تاہم اس کا  کنواں سے باہر پائے جانے کا امکان زیادہ ہے۔ 
\end{enumerate}
\انتہا{سوال}
\ابتدا{سوال}
ایک ذرہ جس کی کمیت\عددی{m}  ہے ہارمونی مرتعش کی مخفیہ (مساوات   \حوالہ{مساوات_شروڈنگر_مخفیہ_ہارمونی})  میں درج ذیل حال سے آغاز  کرتا ہے  جہاں\عددی{A} کوئی مستقل ہے۔
\begin{align*}
\Psi(x,0)=A\Big(1-2\sqrt{\frac{m\omega}{\hslash}}\, x  \Big)^{2}e^{-\frac{m\omega}{2\hslash}x^{2}}
\end{align*}
%
\begin{enumerate}
\item
 توانائی کی توقعاتی قیمت کیا ہے ؟
\item
مستقبل کے لمحہ \عددی{T}  پر تفاعل موج درج ذیل ہو گا
\begin{align*}
\Psi(x,T)=B\Big(1+2\sqrt{\frac{m\omega}{\hslash}}\,x  \Big)^{2}e^{-\frac{m\omega}{2\hslash}x^{2}}   
\end{align*}
 جہاں \عددی{ B  } کوئی مستقل ہے۔ لمحہ  \عددی{  T } کی کم سے کم ممکنہ قیمت کیا ہو گی؟ 
\end{enumerate}
\انتہا{سوال}
\ابتدا{سوال}
درج ذیل نصف ہارمونی مرتعش کی اجازتی توانائیاں تلاش کریں۔
\begin{align*}
V(x)=
\begin{cases}
(1/2)m\omega^{2}x^{2}&x> 0\\
\infty & x< 0
\end{cases}
 \end{align*}
(مثلاً ایک ایسا  اسپرنگ  جس کو کھینچا تو جا سکتا ہے لیکن اسے دبایا  نہیں جا سکتا ہے۔)   \ترچھا{اشارہ:}  اس کو حل کرنے کے لئے آپ کو ایک بار اچھی طرح سوچنا  ہو  گا جبکہ حقیقی حساب بہت کم  درکار ہو گی۔ 
\انتہا{سوال}
\ابتدا{سوال}\شناخت{سوال_شروڈنگر_ساکن_گاوسی_آزاد_ذرہ_موجی_اکٹھ}
آپ نے سوال \حوالہ{سوال_شروڈنگر_گاوسی_موجی_اکٹھ} میں ساکن گاوسی آزاد ذرہ موجی اکٹھ  کا تجزیہ کیا۔ اب ابتدائی تفاعل موج 
\begin{align*}
\Psi(x,0)=Ae^{-ax^{2}}e^{ilx} 
\end{align*} 
 جہاں \عددی{ l}ایک حقیقی مستقل ہے سے آغاز  کرتے ہوئے  متحرک گاوسی  موجی اکٹھ  کے لیے یہی مسئلہ دوبارہ حل کریں۔
\انتہا{سوال}
\ابتدا{سوال}
مبدا  پر   لامتناہی چکور کنواں، جس کے وسط پر درج ذیل  ڈیلٹا تفاعل  رکاوٹ ہو،  کے لیے  غیر تابع وقت   شروڈنگر مساوات حل کریں۔
\begin{align*}
V(x)=
\begin{cases}
\alpha\delta (x)& -a< x< +a\\
\infty &\abs{x}\ge a
\end{cases}
 \end{align*} 
 جفت اور طاق تفاعل امواج کو علیحدہ علیحدہ حل کریں ۔انہیں معمول پر لانے کی ضرورت نہیں ہے ۔ اجازتی توانائیوں کو ( اگر ضرورت  پیش آئے)  ترسیمی طور پر تلاش کریں۔ ان کا موازنہ ڈیلٹا تفاعل کی غیر موجودگی میں مطابقتی  توانائیوں کے ساتھ کریں۔  طاق حلوں پر   ڈیلٹا تفاعل کا کوئی اثر نہ ہونے پر تبصرہ کریں ۔ تحدیدی صورتیں  \عددی{ a\to 0  } اور \عددی{ a\to \infty}پر  تبصرہ کریں۔ 
\انتہا{سوال}
\ابتدا{سوال}\شناخت{سوال_شروڈنگر_یکبعدی_منفرد_نا_ممکن}
 ایسے دو یا دو سے زیادہ  غیر تابع وقت  شروڈنگر  مساوات کے منفرد\حاشیہد{ایسے دو حل جن میں صرف جزو ضربی کا فرق پایا جاتا ہو (جن میں ایک مرتبہ معمول پر لانے کے بعد صرف دوری جزو \عددی{e^{i\phi}} کا فرق پایا جاتا ہو) درحقیقت ایک ہی حل کو ظاہر کرتے ہیں لہٰذا انہیں یہاں منفرد نہیں کہا جا سکتا ہے۔یہاں "منفرد" سے مراد "خطی طور پر غیر تابع" ہے۔}  حل جن کی توانائی \عددی{ E} ایک دوسرے جیسی ہو  کو \اصطلاح{ انحطاطی}\فرہنگ{انحطاطی}\حاشیہب{degenerate}\فرہنگ{degenerate} کہتے ہیں۔ مثال کے طور پر آزاد ذرہ  کے حال  دوہری انحطاطی ہیں۔  ان میں سے ایک حل دائیں رخ اور دوسرا بائیں رخ حرکت کو ظاہر کرتا ہے ۔  تاہم  ہم نے ایسے کوئی انحطاطی حل نہیں دیکھے جو معمول پر لانے کے قابل ہوں اور یہ محض ایک اتفاق نہیں ہے۔ 
 درج ذیل مسئلہ ثابت کریں : 
\ترچھا{یک بعدی  مقید  انحطاطی حال نہیں   پائے جاتے ہیں۔}\حاشیہد{جیسا ہم باب  \حوالہ{باب_تین_ابعادی_کوانٹم_میکانیات} میں دیکھیں گے،  بلند ابعاد میں ایسی  انحطاط عام پائی جاتی ہیں۔فرض کریں کہ مخفیہ علیحدہ علیحدہ حصوں پر مشتمل نہیں ہے جن کے بیچ خطہ میں \عددی{V=\infty} ہو۔ مثلاً دو  تنہا لامتناہی کنویں مقید انحطاطی حال دیں گے جہاں ذرہ ایک یا دوسرے کنواں میں پایا جائے گا۔} 
\ترچھا{اشارہ:}  فرض کریں
\عددی{  \psi_{1} }  اور \عددی{ \psi_{2}  } ایسے دو حل ہوں جن کی توانائی،  \عددی{ E}،   ایک دوسری  جیسی ہو ۔  حل \عددی{  \psi_{1} } کی شروڈنگر  مساوات کو \عددی{ \psi_{2}  } سے  ضرب دیں اور اس سے \عددی{  \psi_{2} }کی  شروڈنگر مساوات کو \عددی{ \psi_{1}  } سے ضرب دے کر منفی کر کے  دکھائیں کہ \عددی{\psi_{2}\dif{\psi_{1}}/\dif{x}- \psi_{1}\dif{\psi_{2}}/\dif{x}} ایک مستقل ہو گا۔
 اب   \عددی{\pm \infty} پر     معمول پر لائے جانے کے  قابل ہر حل   \عددی{\psi\to 0}  ہو گا۔  اس حقیقت کو استعمال کرتے ہوئے دکھائیں کہ یہ مستقل در حقیقت صفر ہو گا جس سے آپ   نتیجہ اخذ کر سکتے ہیں کہ \عددی{\psi_{2}} دراصل \عددی{\psi_{1}} کا مضرب ہے   لہٰذا یہ  حل دو  الگ الگ حل  نہیں ہو سکتے ہیں ۔
\انتہا{سوال}
\ابتدا{سوال}
فرض کریں کمیت\حوالہء{ m}   کا ایک موتی  ایک دائری چھلا  پر  بے رگڑ حرکت کرتا ہے۔ چھلے کا محیط \عددی{ L} ہے ۔ (یہ ایک آزاد ذرہ کی مانند  ہے تاہم یہاں\عددی{ \psi(x+L)=\psi(x)  } ہو گا۔)   اس کے ساکن حال تلاش کر کے  انہیں معمول پر لائیں اور ان کی  مطابقتی  اجازتی  توانائیاں دریافت  کریں ۔  آپ دیکھیں گے کہ ہر ایک توانائی\عددی{E_{n}} کے لئے دو آپس میں غیر تابع حل پائے جائیں گے جن میں سے ایک گھڑی وار اور دوسرا خلاف گھڑی حرکت کے لیے ہو گا،  جنہیں آپ\عددی{ \psi_{n}^{+}(x)  } اور \عددی{ \psi_{n}^{-}(x)  } کہہ سکتے ہیں۔سوال \حوالہ{سوال_شروڈنگر_یکبعدی_منفرد_نا_ممکن} کے مسئلہ  کو مد نظر رکھتے ہوئے آپ اس  انحطاط کے بارے میں کیا کہیں گے  (اور یہ مسئلہ یہاں کارآمد کیوں نہیں ہے)؟  
\انتہا{سوال}

%%%KKKKKKKKKK  the above is problem 2.46 on p99. ch2 completed to this place
