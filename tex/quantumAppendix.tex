\باب{ضمیمہ}

\حصہء{خطی الجبرا}\شناخت{ضمیمہ_خطی_الجبرا}
کالج  کی سطح پر پڑھائے جانے والے سادہ سمتیات کے حساب کو خطی الجبرا تصوراتی جامع    پہناتا اور  عمومیت دیتا ہے۔ عمومیت دو رخوں میں دی  جاتی ہے:  \عددی{(1)} ہم غیر سمتیات کو مخلوط اعداد ہونے کی اجازت دیتے ہیں،  اور \عددی{(2)} ہم اپنے آپ کو تین ابعاد میں  رہنے کا پابند نہیں رکھتے۔

\حصہ{سمتیات}\شناخت{ضمیمہ_سمتیات}
\اصطلاح{ سمتیات} \عددی{|\alpha\rangle}، \عددی{|\beta\rangle}، \عددی{|\gamma\rangle}، \نقطے   کے سلسلہ   اور 
    \اصطلاح{غیر سمتیات}   (\عددی{a}، \عددی{b}، \عددی{c}، \نقطے) \حاشیہد{ہمارے مقصد کے لئے غیر سمتیات سادہ مخلوط اعداد ہوں گے۔  ریاضی دان آپ کو    زیادہ پراسرار  میدانوں  پر سمتی فضاوں کے بارے میں بتا سکتے ہیں، تاہم ان کا کوانٹائی میکانیات میں کوئی کردار نہیں پایا جاتا۔ یاد رہے کہ \عددی{\alpha}، \عددی{\beta}، \عددی{\gamma}، \نقطے (عموماً) اعداد نہیں ہوں گے؛ یہ نام ہوں گے، مثلاً  \قول{جمشید}،
      یا  \قول{\تحریر{F43A-9GL}}، یا،  زیر غور سمتیہ کو جو بھی آپ پکارنا  چاہیں۔}  کے سلسلہ   پر  \اصطلاح{سمتی فضا}\حاشیہب{vector space}\فرہنگ{سمتی فضا}\فرہنگ{vector space} مشتمل ہو گا   جو    سمتی جمع اور غیر سمتی ضرب کے زیر عمل\اصطلاح{ بند}\فرہنگ{بند}\حاشیہب{closed}\فرہنگ{closed}  ہو گا۔\حاشیہد{یعنی یہ اعمال پوری طرح معین ہیں،  اور کبھی بھی آپ کو سمتی فضا سے باہر منتقل نہیں کریں گے۔}


\begin{itemize}
	\item \موٹا{سمتی جمع}
\end{itemize}
کسی بھی دو سمتیات کا مجموعہ بھی  سمتیہ ہوگا۔
\begin{align}
	|\alpha\rangle+|\beta\rangle=|\gamma\rangle
\end{align}
سمتی مجموعہ \اصطلاح{استبدالی}\فرہنگ{استبدالی}\حاشیہب{commutative}\فرہنگ{commutative}:  
\begin{align}
	|\alpha\rangle+|\beta\rangle = |\beta\rangle+|\alpha\rangle
\end{align}
اور  \اصطلاح{تلازمی}\فرہنگ{تلازمی}\حاشیہب{associative}\فرہنگ{associative}:
\begin{align}
	|\alpha\rangle+(|\beta\rangle+|\gamma\rangle)=(|\alpha\rangle+|\beta\rangle)+|\gamma\rangle
\end{align}
ہے۔ ایک \اصطلاح{ معدوم}\فرہنگ{معدوم}\حاشیہب{null}\فرہنگ{null} (یا \اصطلاح{صفر}\فرہنگ{صفر}\حاشیہب{zero}\فرہنگ{zero})  سمتیہ    \عددی{|0\rangle} پایا جاتا ہے\حاشیہد{جہاں غلط فہمی کا امکان نہ ہو، وہاں روایتی طور پر معدوم سمتیہ کو  سادہ صفر لکھا جاتا ہے:\عددی{|0\rangle \to 0}} جو ہر سمتیہ \عددی{|\alpha\rangle} کے لئے درجہ ذیل خاصیت رکھتا ہے
\begin{align}
	|\alpha\rangle+|0\rangle=|\alpha\rangle
\end{align}
اور ہر سمتیہ \عددی{|\alpha\rangle} کا شریک \اصطلاح{مخالف سمتیہ}\فرہنگ{مخالف سمتیہ}\حاشیہب{inverse vector}\فرہنگ{inverse vector} 
  \عددی{(|-\alpha\rangle)}\حاشیہد{یہ ایک  انوکھی  علامت ہے چونکہ \عددی{\alpha}  عدد نہیں۔ میں ایک سمتیہ جس کا نام  \قول{جمشید}  ہے کے مخالف سمتیہ کو \قول{جمشید-}   کا نام دے رہا ہوں۔ کچھ ہی دیر میں ہم بہتر   اصطلاح  دیکھ پائیں گے۔}  پایا جاتا ہے جو درجہ ذیل دیتا ہے۔
\begin{align}
	|\alpha\rangle+|-\alpha\rangle = |0\rangle
\end{align}

\begin{itemize}
	\item \موٹا{غیر سمتی ضرب}
\end{itemize}
کسی بھی غیر سمتیہ اور سمتیہ کا حاصل ضرب:
\begin{align}
	a|\alpha\rangle=|\gamma\rangle
\end{align}
 ایک سمتیہ ہوگا۔ غیر سمتی ضرب سمتی مجموعہ کے لحاظ سے \اصطلاح{ جزئیتی تقسیمی}\فرہنگ{جزئیتی تقسیمی}\حاشیہب{distributive}\فرہنگ{distributive}  
\begin{align}
	a(|\alpha\rangle+|\beta\rangle)=a|\alpha\rangle+a|\beta\rangle
\end{align}
اور  غیر سمتی جمع کے لحاظ سے بھی جزئیتی تقسیمی ہے۔
\begin{align}
	(a+b)|\alpha\rangle=a|\alpha\rangle+b|\alpha\rangle
\end{align}
یہ غیر سمتیات کے سادہ ضرب کے لحاظ سے \اصطلاح{تلازمی} بھی ہے۔
\begin{align}
	a(b|\alpha\rangle)=(ab)|\alpha\rangle
\end{align}
غیر سمتیات \عددی{0} اور \عددی{1} کے ساتھ  ضرب آپ کی  توقع کے مطابق نتائج دیں گے۔
\begin{align}
1|\alpha\rangle=|\alpha\rangle; \quad  0|\alpha\rangle=|0\rangle
\end{align}
ظاہر ہے \عددی{|-\alpha\rangle=(-1)|\alpha\rangle} ہوگا جس کو ہم \عددی{-|\alpha\rangle}   لکھتے ہیں۔

یہاں   جتنا  نظر آ رہا ہے،    حقیقتاً  اتنا ہے نہیں؛   پس میں نے سمتیات کی جوڑ توڑ کے عام فہم قواعد کو تصوراتی روپ  میں پیش کیا ہے۔ نتیجتاً دیگر نظام جو یہی باضابطہ خواص رکھتے ہوں پر ہم سادہ سمتیات کے رویّہ کے بارے میں  معلوم علم  اور وجدان بروئے کار  لا سکیں گے۔

سمتیات \عددی{|\alpha\rangle}، \عددی{|\beta\rangle}، \عددی{|\gamma\rangle}، \نقطے  
 کا   \اصطلاح{خطی  مجموعہ}\فرہنگ{خطی مجموعہ}\حاشیہب{linear combination}\فرہنگ{linear combination}   درجہ ذیل روپ کا  فقرہ ہوگا۔
\begin{align}
	a|\alpha\rangle+b|\beta\rangle+c|\gamma\rangle+\cdots
\end{align}
ایک سمتیہ \عددی{|\lambda\rangle} جس کو سلسلہ  \عددی{|\alpha\rangle}، \عددی{|\beta\rangle}، \عددی{|\gamma\rangle}، \نقطے کا      خطی مجموعہ لکھنا ممکن نہ ہو  \اصطلاح{خطی غیر تابع}\فرہنگ{خطی غیر تابع}\حاشیہب{linearly independent}\فرہنگ{linearly independent}  کہلاتا ہے۔ (مثلاً،  تین ابعاد میں اکائی سمتیہ \عددی{\hat{k}} سمتیات \عددی{\hat{i}} اور \عددی{\hat{j}} کا خطی غیر تابع ہے،  جبکہ \عددی{xy} مستوی میں ہر  سمتیہ \عددی{\hat{i}} اور \عددی{\hat{j}} کا خطی  \ترچھا{تابع} ہوگا۔)  اسی کی توسّط   سے، سمتیات کا  وہ  ذخیرہ جس میں ہر ایک سمتیہ  باقی تمام سمتیات  کا خطی غیر تابع ہو  "خطی غیر تابع"  کہلاتا ہے۔ جب ہر سمتیہ کو  سمتیات کے ایک ذخیرہ کے ارکان کا خطی مجموعہ لکھنا ممکن ہو، ہم کہتے ہیں کہ سمتیات کا یہ ذخیرہ  فضا کا  \اصطلاح{احاطہ}\فرہنگ{احاطہ}\حاشیہب{span}\فرہنگ{span}  کرتے\حاشیہد{فضا کا احاطہ کرنے والے سمتیات  کا سلسلہ   \اصطلاح{مکمل}\فرہنگ{مکمل}\فرہنگ{complete} (\تحریر{complete})  بھی  کہلاتا  ہے، اگرچہ میں اس  اصطلاح کو لامتناہی بُعد  کی صورت کے لئے رکھتا ہوں جہاں  ارتکاز پر سوالات اٹھائے  جا  سکتے ہیں۔}  ہیں۔ فضا کا احاطہ کرنے والے   \ترچھا{خطی غیر تابع} سمتیات کا سلسلہ  \اصطلاح{اساس}\فرہنگ{اساس}\حاشیہب{basis}\فرہنگ{basis} کہلاتا ہے۔ اساس میں سمتیات کی تعداد  فضا کا  \اصطلاح{بُعد}\فرہنگ{بُعد}\حاشیہب{dimension}\فرہنگ{dimension}  کہلاتا ہے۔ فی الحال ہم فرض کرتے ہیں کہ بُعد \عددی{(n)}  \ترچھا{متناہی} ہے۔

دیے گئے اساس 
\begin{align}
	|e_1\rangle, |e_2\rangle, \dots, |e_n\rangle
\end{align}
کے لحاظ سے کسی  بھی سمتیہ
\begin{align}
	|\alpha\rangle=a_1|e_1\rangle+a_2|e_2\rangle+\dots+a_n|e_n\rangle
\end{align}
کو اس اساس  کے \اصطلاح{ ارکان}  کی  (مرتب)  \عددی{n} اجزائی سلسلہ 
\begin{align}
	|\alpha\rangle\leftrightarrow(a_1, a_2, \dots, a_n)
\end{align}
سے یکتا طور پر ظاہر کیا جا سکتا ہے۔ عموماً  سمتیات کی بجائے  ان  اجزاء کے ساتھ کام کرنا زیادہ آسان  ہوتا ہے۔  سمتیات  جمع کرنے کے لئے  ان کے مطابقتی اجزاء آپس میں جمع کئے جاتے ہیں:
\begin{align}
	|\alpha\rangle+|\beta\rangle\leftrightarrow(a_1+b_1, a_2+b_2, \dots, a_n+b_n)
\end{align}
غیر سمتیہ سے ضرب  کے لئے ہر جزو کو اس غیر سمتیہ  سے  ضرب کریں:
\begin{align}
	c|\alpha\rangle\leftrightarrow(ca_1, ca_2, \dots, ca_n)
\end{align}
معدوم سمتیہ کو صفروں کی ایک کھڑی ظاہر کرتی ہے:
\begin{align}
	|0\rangle\leftrightarrow(0, 0, \dots, 0)
\end{align}
اور مخالف سمتیہ کے ارکان کی علماتیں  الٹ کی جاتی ہیں۔
\begin{align}
	|-\alpha\rangle\leftrightarrow(-a_1, -a_2, \dots, -a_n)
\end{align}
ارکان کے ساتھ کام کرنے کی واحد  قباحت  یہ ہے کہ  آپ کو کسی ایک مخصوص اساس کے ساتھ کام کرنا ہوگا،  اور یہی  حسابی عمل کسی دوسری  اساس میں بالکل مختلف نظر آئے گا۔

\ابتدا{سوال}
مخلوط اجزاء والے تین ابعادی سادہ سمتیات \عددی{(a_x\hat{i}+a_y\hat{j}+a_z\hat{k})} پر غور کریں۔
\begin{enumerate}[a]
\item
 کیا وہ ذیلی سلسلہ جس میں تمام سمتیات کے لئے \عددی{a_z=0} ہو سمتی فضا قائم کرتے ہیں؟ اگر کرتا ہو تب اس کا  بُعد کیا ہوگا؛  اگر  نہیں کرتا  تو کیوں نہیں  کرتا؟
\item
 اس  ذیلی سلسلہ کے بارے میں آپ کیا کہیں گے جن کا \عددی{z} جزو \عددی{1} کے برابر ہو؟ \ترچھا{ اشارہ}: کیا ایسے دو سمتیات کا مجموعہ اسی ذیلی سلسلہ میں پایا جائے گا؟  معدوم سمتیہ کے بارے میں سوچیں؟
\item
 ان سمتیات کے ذیلی سلسلہ کے بارے میں آپ کیا کہہ سکتے ہیں جن کے تمام ارکان برابر ہوں؟
 \end{enumerate}
\انتہا{سوال}
\ابتدا{سوال}
ان تمام کثیر رکنیوں، (جن کے عددی سر مخلوط ہوں اور)  جن کا  \عددی{x} میں درجہ  \عددی{N} سے کم ہو کے ذخیرہ پر غور کریں۔
\begin{enumerate}[a]
\item
کیا یہ سلسلہ سمتی فضا قائم کرتا ہے  (جہاں کثیر رکنیاں بطور  \قول{سمتیات}  ہوں)؟ اگر فضا قائم کرتا ہو تو  مناسب اساس تجویز کریں اور اس فضا کا بُعد بتائیں۔  اگر فضا قائم نہ کرتا ہو تو   تعریفی خصوصیات میں سے کونسی اس میں نہیں پائی جاتی (جاتیں)؟
\item
 اگر ہم چاہیں کہ تمام کثیر رکنیاں جفت تفاعلات ہوں تب کیا ہوگا؟
\item
 اگر ہم چاہیں کہ پہلا عددی سر ( جو \عددی{x^{N-1}} کو ضرب کرتا ہے)  \عددی{1} ہو تب کیا ہوگا؟
\item
 اگر ہم چاہیں کہ \عددی{x=1} پر کثیر رکنیوں کی قیمت \عددی{0} ہو تب کیا ہوگا؟
\item
 اگر ہم چاہیں کہ \عددی{x=0} پر کثیر رکنیوں کی قیمت \عددی{1} ہو تب کیا ہوگا؟ 	
 \end{enumerate}
\انتہا{سوال}
\ابتدا{سوال}
ثابت کریں  کہ کسی بھی ایک اساس کے لحاظ سے  سمتیہ کے ارکان \ترچھا{ یکتا}   ہوں گے۔
\انتہا{سوال}

\حصہ{اندرونی ضرب}\شناخت{ضمیمہ_اندرونی_ضرب}
تین ابعاد میں دو اقسام کے سمتی ضرب  پائے جاتے ہیں:  نقطی ضرب اور صلیبی ضرب۔ موخر الذکر کی قدرتی توسیع کسی طرح بھی \عددی{n} ابعاد سمتی فضاوں میں نہیں کی جا سکتی،   جبکہ  اول الذکر کی کی جا سکتی ہے؛  اور  اس سیاق و سباق میں اسے عموماً \اصطلاح{ اندرونی ضرب}\فرہنگ{اندرونی!ضرب}\حاشیہب{inner product}\فرہنگ{product!inner} پکارا  جاتا ہے۔ دو سمتیات (\عددی{|\alpha\rangle} اور \عددی{|\beta\rangle}) کا اندرونی ضرب ایک مخلوط عدد ہوگا جسے \عددی{\langle\alpha|\beta\rangle} لکھا جاتا ہے اور جس کے  خواص درج ذیل  ہیں۔
\begin{align}
	\langle\beta|\alpha\rangle &= \langle\alpha|\beta\rangle^*\\
	\langle\alpha|\alpha\rangle\geq0,\quad\text{\RL{اور}}\quad \langle\alpha|\alpha\rangle &= 0\leftrightarrow|\alpha\rangle = |0\rangle \label{مساوات_خطی_معیار_مثبت}\\
	\langle\alpha|(b|\beta\rangle+c|\gamma\rangle) &=  b\langle\alpha|\beta\rangle+c\langle\alpha|\gamma\rangle
\end{align}
مخلوط اعداد تک عمومیت کے علاوہ یہ مسلمات نقطی  ضرب کے جانے پہچانے رویّوں کو ریاضی کی زبان میں پیش کرتے ہیں۔ ایسی سمتی فضا جس میں اندرونی  ضرب بھی شامل ہو  \اصطلاح{اندرونی ضرب فضا}\فرہنگ{اندرونی!ضرب فضا}\حاشیہب{inner product space}\فرہنگ{inner!product space} کہلاتی ہے۔

چونکہ  سمتیہ کا اپنے ساتھ اندرونی ضرب غیر منفی عدد  ہے  (مساوات \حوالہ{مساوات_خطی_معیار_مثبت})  لہٰذا اس کا جذر حقیقی ہوگا؛ جو سمتیہ کا  \اصطلاح{معیار}\فرہنگ{معیار}\حاشیہب{norm}\فرہنگ{norm} کہلاتا ہے:
\begin{align}
	\|\alpha\|&\equiv\sqrt{\langle\alpha|\alpha\rangle}&  \text{معیار} 
\end{align}
اور جو  \قول{لمبائی} کے تصور کو وسعت دیتا ہے۔\اصطلاح{ اکائی سمتیہ}\فرہنگ{اکائی!سمتیہ}\حاشیہب{unit vector}\فرہنگ{unit!vector}   (جس کا معیار \عددی{1}  ہو گا)     \اصطلاح{معمول شدہ}\فرہنگ{معمول شدہ}\حاشیہب{normalized}\فرہنگ{normalized} کہلاتا ہے۔ دو سمتیات جن کا اندرونی ضرب صفر ہو   \اصطلاح{قائمہ}\فرہنگ{قائمہ}\حاشیہب{orthogonal}\فرہنگ{orthogonal}  کہلاتے ہیں  (جو \قول{سیدھا کھڑا}  ہونے کے تصور کو عمومیت دیتا ہے)۔ باہمی قائمہ معمول شدہ سمتیات:
\begin{align}
	\langle\alpha_i|\alpha_j\rangle=\delta_{ij}
\end{align}
کے ذخیرہ کو \اصطلاح{ معیاری عمودی سلسلہ}\فرہنگ{معیاری عمودی سلسلہ}\حاشیہب{orthonormal set}\فرہنگ{orthonormal set} کہتے ہیں۔ \ترچھا{معیاری عمودی اساس}  ہر صورت منتخب کیا جا سکتا ہے (سوال \حوالہ{سوال_ضمیمہ_گرام_شمد_ترکیب} دیکھیں) اور ایسا کرنا عموماً  بہتر بھی ثابت ہوتا ہے۔ ایسی صورت میں دو سمتیات کے  اندرونی ضرب  کو انکے اجزاء کے روپ  میں نہایت  خوبصورتی سے لکھا جا سکتا ہے:
\begin{align}
	\langle\alpha|\beta\rangle=a_1^*b_1+a_2^*b_2+\dots+a_n^*b_n
\end{align}
لہٰذا    معیار کا مربع
\begin{align}
	\langle\alpha|\alpha\rangle = \abs{a_1}^2+\abs{a_2}^2+\dots+\abs{a_n}^2
\end{align}
ہوگا  جبکہ اجزاء ازخود درجہ ذیل ہونگے۔
\begin{align}
	a_i=\langle e_i|\alpha\rangle
\end{align}
(یہ نتائج تین ابعادی معیاری عمودی اساس \عددی{\hat{i}}، \عددی{\hat{j}}، \عددی{\hat{k}} کے مشہور کلیات
 \عددی{\kvec{a}\cdot\kvec{b}=a_xb_x+a_yb_y+a_zb_z}، \عددی{\abs{\kvec{a}}^2=a_x^2+a_y^2+a_z^2} اور \عددی{a_x=\hat{i}\cdot\kvec{a}}، \عددی{a_y=\hat{j}\cdot \kvec{a}}، \عددی{a_z=\hat{k}\cdot\kvec{a}} کو عمومیت دیتے ہیں۔)  یہاں سے آگے ہم صرف معیاری عمودی اساس استعمال کریں گے،   ما سوائے جب صریحاً ایسا نہ کرنے کا کہا  گیا ہو۔

دو  سمتیات کے بیچ  \ترچھا{زاویہ}  ایسی    ہندسی مقدار  ہے جس کو ہم عمومیت دینا چاہیں گے۔ سادہ سمتی تجزیہ میں 
\عددی{\cos\theta=(\kvec{a}\cdot\kvec{b})/\abs{\kvec{a}}\abs{\kvec{b}}}  ہے۔ اب  اندرونی ضرب  عموماً مخلوط عدد ہو گا،    لہٰذا  (  اختیاری اندرونی ضرب فضا میں) مماثل کلیہ  (حقیقی)  زاویہ \عددی{\theta} نہیں دیگا۔ تاہم،  اس مقدار کی  \ترچھا{مطلق قیمت}  ایسا  عدد ہوگا جو \عددی{1} سے تجاوز نہیں کرتا۔
\begin{align}
	\abs{\langle\alpha|\beta\rangle}^2\leq\langle\alpha|\alpha\rangle\langle\beta|\beta\rangle
\end{align}
(اس اہم نتیجہ کو  \اصطلاح{شوارز عدم مساوات}\فرہنگ{شوارز عدم مساوات}\حاشیہب{Schwarz inequality}\فرہنگ{Schwarz inequality} کہتے ہیں؛  جس کا ثبوت سوال \حوالہ{ضمیمہ_اندرونی_ضرب_شوارز_عدم_مساوات_ثبوت} میں پیش کیا گیا ہے۔)  یوں،  آپ چاہیں تو،  \عددی{|\alpha\rangle} اور \عددی{|\beta\rangle} کے بیچ زاویہ کی  تعریف درج ذیل  لی جا سکتی ہے۔
\begin{align}
	\cos\theta = \sqrt{\frac{\langle\alpha|\beta\rangle\langle\beta|\alpha\rangle}{\langle\alpha|\alpha\rangle\langle\beta|\beta\rangle}}
\end{align}

%?????KKKK proof read till here
\ابتدا{سوال}\شناخت{سوال_ضمیمہ_گرام_شمد_ترکیب}
فرض کریں آپ اساس \عددی{(|e_1\rangle, |e_2\rangle, \dots, |e_n\rangle)} سے آغاز کرتے ہیں جو معیاری عمودی نہیں ہے۔ اس سے معیاری عمودی اساس \عددی{(|e'_1\rangle, |e'_2\rangle, \dots, |e'_n\rangle)} گرہم شمڈ حکمت عملی سے حاصل کی جا سکتی ہے جو ایک منظم ترکیب ہے۔ یہ کچھ یوں ہے

(الف) پہلی اساس سمتیہ کی معمولزنی معیار سے تقسیم کرتے ہوئے کریں
\begin{align*}
		|e_1'\rangle = \frac{|e_1\rangle}{\|e_1\|}
\end{align*}
(ب) پہلی سمتیہ پر دوسرے سمتیہ کے تزلیل کو دوسری سمتیہ سے منفی کریں
\begin{align*}
	|e_2\rangle-\langle e_1'|e_2\rangle|e_1'\rangle
\end{align*}
یہ سمتیہ \عددی{|e'_1\rangle} کا عمودی ہوگا۔
\انتہا{سوال}
\ابتدا{سوال}\شناخت{ضمیمہ_اندرونی_ضرب_شوارز_عدم_مساوات_ثبوت}
شوارز عدم مساوات ثابت کریں۔ اشارہ: \عددی{|\gamma\rangle=|\beta\rangle-(\langle\alpha|\beta\rangle/\langle\alpha|\alpha\rangle)|\alpha\rangle} اور \عددی{\langle\gamma|\gamma\rangle\geq0} استعمال کریں۔
\انتہا{سوال}
\ابتدا{سوال}
سمتیات \عددی{|\alpha\rangle=(1+i)\hat{i}+(1)\hat{j}+(i)\hat{k}} اور \عددی{|\beta\rangle=(4-i)\hat{i}+(0)\hat{j}+(2-2i)\hat{k}} کے بیچ مساوات \عددی{A.28} کی معنوں میں زاویہ تلاش کریں۔
\انتہا{سوال}
\ابتدا{سوال}
تکونی عدم مساوات \عددی{\|(|\alpha\rangle+|\beta\rangle)\|\leq\|\alpha\|+\|\beta\|} ثابت کریں۔
\انتہا{سوال}
\حصہ{قوالب}\شناخت{ضمیمہ_قالب}
فرض کریں آپ تین فضا میں ہر سمتیہ کو \عددی{17} سے ضرب دیں یا ہر سمتیہ کو \عددی{z} محور کے گرد \عددی{39\degree} گھمائیں یا \عددی{xy} مستوی میں ہر سمتیہ کا عکس لیں یہ تمام خطی تبالہ کی مثالیں ہیں خطی مبدل \عددی{\hat{T}} ایک سمتی فضا میں یر ایک سمتیہ کا کسی دوسرے سمتیہ \عددی{(|\alpha\rangle\rightarrow|\alpha'\rangle=\hat{T}|\alpha\rangle)} میں تبادلہ
\begin{align}
	\hat{T}(a|\alpha\rangle+b|\beta\rangle)=a(\hat{T}|\alpha\rangle)+b(\hat{T}|\beta\rangle)
\end{align}
کرتا ہے جہاں کسی بھی سمتیہ \عددی{|\alpha\rangle}، \عددی{|\beta\rangle} اور غیر سمتیات \عددی{a,b} کے لئے یہ عمل خطی ہوگا۔

یہ جانتے ہوئے کہ اساس سمتیات کے سلسلہ کو کوئی خطی مبدل کیا کرتا ہے آپ با آسانی معلوم کرسکتے ہیں کہ وہ کسی بھی سمتیہ کے ساتھ کیا کرے گا۔ مثال کے طور پر 
\begin{align*}
	\hat{T}|e_1\rangle &= T_{11}|e_1\rangle+T_{21}|e_2\rangle+\dots+T_{n1}|e_n\rangle\\
	\hat{T}|e_2\rangle &= T_{12}|e_1\rangle+T_{22}|e_2\rangle+\dots+T_{n2}|e_n\rangle\\
	&\dots\\
	\hat{T}|e_n\rangle &= T_{1n}|e_1\rangle+T_{2n}|e_2\rangle+\dots+T_{nn}|e_n\rangle
\end{align*}
یا مختصراً
\begin{align}
	\hat{T}|e_{j}\rangle=\sum_{i=1}^{n}T_{ij}|e_i\rangle,\quad(j=1, 2, \dots, n)
\end{align}
اگر \عددی{|\alpha\rangle} ایک اختیاری سمتیہ 
\begin{align}
	|\alpha\rangle=a_1|e_1\rangle+a_2|e_2\rangle+a_2|e_2\rangle+\dots+a_n|e_n\rangle=\sum_{j=1}^{n}a_j|e_j\rangle
\end{align}
ہو۔ تب درجہ ذیل ہوگا
\begin{align}
	\hat{T}|\alpha\rangle=\sum_{j=1}^{n}a_j\left(\hat{T}|e_j\rangle\right)=\sum_{j=1}^{n}\sum_{i=1}^{n}a_jT_{ij}|e_i\rangle=\sum_{i=1}^{n}\left(\sum_{j=1}^{n}T_{ij}a_j\right)|e_i\rangle
\end{align}
ظاہر ہے کہ \عددی{\hat{T}} ایک سمتیہ کو جس کے ارکان \عددی{a_1, a_2, \dots, a_n} ہوں کو ایک نئے سمتیہ میں لے جاتا ہے جن کے ارکان درجہ ذیل ہونگے
\begin{align}
	a'_i=\sum_{j=1}^{n}T_{ij}a_j
\end{align}
یوں جس طرح کسی اساس کے لحاظ سے \عددی{n} ارکان \عددی{a_i} سمتیہ یا \عددی{|\alpha\rangle} کو یکتہ ظاہر کرتے ہیں اسی طرح \عددی{T_{ij}} کے \عددی{n^2} ارکان خطی مبدل \عددی{\hat{T}} کو اسی اساس کے لحاظ سے یکتا طور پر بیان کرتے ہیں۔
\begin{align}
	\hat{T}\leftrightarrow(T_{11}, T_{12}, \dots, T_{nn})
\end{align}
اگر اساس معیاری عمودی ہو  مساوات \عددی{A.30} سے درجہ ذیل لکھا جا سکتا ہے
\begin{align}
	T_{ij}=\langle e_i|\hat{T}|e_j\rangle
\end{align}
ان مخلوط اعداد کو قالب کے روپ میں لکھنا بہتر ثابت ہوتا ہے 
\begin{align}
	T=
	\begin{pmatrix}
		T_{11} & T_{12} & \dots & T_{1n}\\
		T_{21} & T_{22} & \dots & T_{2n}\\
		\vdots & \vdots & & \vdots\\
		T_{n1} & T_{n2} & \dots & T_{nn}
	\end{pmatrix}
\end{align}
یوں خطی مبدل کا مطالعہ محز قوالب کے نظریہ کا مطالعہ ہوگا دو خطی مبدل کا مجموعہ \عددی{(\hat{S}+\hat{T})} کی تعریف ہماری توقع کے عین مطابق درجہ ذیل ہے
\begin{align}
	(\hat{S}+\hat{T})|\alpha\rangle=\hat{S}|\alpha\rangle+\hat{T}|\alpha\rangle
\end{align}
جو قوالب جمع کرنے کے مترادف ہے جہاں آپ انکے مطابقتی ارکان جمع کرتے ہیں
\begin{align}
	U = S+T \leftrightarrow U_{ij}=S_{ij}+T_{ij}
\end{align}
دو خطی مبدل کا حاصل ضرب \عددی{(\hat{S}\hat{T})} پہلے \عددی{\hat{T}} اور اسکے بعد \عددی{\hat{S}} عمل کرنے کے مترادف ہے
\begin{align}
	|\alpha'\rangle=\hat{T}|\alpha\rangle;\quad|\alpha''\rangle=\hat{S}|\alpha'\rangle=\hat{S}(\hat{T}|\alpha\rangle)=\hat{S}\hat{T}|\alpha\rangle
\end{align}
مجموعی مبدل \عددی{\hat{U}=\hat{S}\hat{T}} کو کونسا قالب \عددی{U} ظاہر کرے گا؟ اسے حاصل کرنا مشکل نہیں ہے
\begin{align*}
	a''_i=\sum_{j=1}^{n}S_{ij}a'_j=\sum_{j=1}^{n}S_{ij}\left(\sum_{k=1}^{n}T_{jk}a_k\right)=\sum_{k=1}^{n}\left(\sum_{j=1}^{n}S_{ij}T_{jk}\right)a_k=\sum_{k=1}^{n}U_{ik}a_k
\end{align*}
بظاہر درجہ ذیل ہوگا
\begin{align}
	U = ST\leftrightarrow U_{ik}=\sum_{j=1}^{n}S_{ij}T_{jk}
\end{align}
یہ قوالب ضرب کرنے کا رائج طریقہ ہے آپ \عددی{S} کے \عددی{i}ویں صف کو اور \عددی{T} کے \عددی{k}ویں قطار کو لیکر انکے مطابقتی اندراجات کا آپس میں ضرب لیکر تمام کو جمع کرتے ہیں اسی طریقہ کو استعمال کرتے ہوئے مستطیل قوالب کو آپس میں ضرب کیا جاسکتا ہے بس اتنا ضروری ہے کہ پہلے میں قطاروں کی تعداد دوسرے میں صفوں کی تعداد کے برابر ہو۔ بلخصوص \عددی{|\alpha\rangle} کے ارکان کا \عددی{n} اجزائی سلسلہ کو \عددی{n\times1} قطاری قالب یا قطاری سمتیہ
\begin{align}
	a\equiv
	\begin{pmatrix}
		a_1\\a_2\\\vdots\\a_n
	\end{pmatrix}
\end{align}
لکھ کر مبادلہ کے قائدہ کو قالبی حاصل ضرب
\begin{align}
	a'=Ta
\end{align}
کی صورت میں لکھا جاسکتا ہے

آئیں اب قالبی اصطلاحات سیکھیں:
ایک قالب کا تبدیل محل جس کو ہم اعراب کے ساتھ لکھتے ہیں \عددی{\tilde{T}} ٹلڈ انہی ارکان پر مشتمل ہوتا تاہم اس میں صف اور قطار آپس میں تبدیل ہوتے ہیں۔ بلخصوص قطاری قالب کا تبدیل محل صف قالب ہوگا
\begin{align}
	\tilde{a}=
	\begin{pmatrix}
		a_1 & a_2 & \dots & a_n
	\end{pmatrix}
\end{align}
چوکور قالب کے مرکزی وتر بالائی بائیں سے زیریں دائیں میں عکس اس کا تبدیل محل ہوگا 
\begin{align}
	\tilde{T}=
	\begin{pmatrix}
		T_{11} & T_{21} & \dots & T_{n1}\\
		T_{12} & T_{22} & \dots & T_{n2}\\
		\vdots & \vdots & & \vdots\\
		T_{1n} & T_{2n} & \dots & T_{nn}
	\end{pmatrix}
\end{align}
ایسا چوکور قالب جو اپنے تبدیل محل کے برابر ہو تشاکلی ہوگا اگر تبدیل محل کی علامت الٹ ہو تب یہ خلاف تشاکلی ہوگا
\begin{align}
	\tilde{T} = T \text{\RL{تشاکلی}};\quad \tilde{T} = -T \text{\RL{خلاف تشاکلی}}
\end{align}
ہر رکن کا مخلوط جوڑی دار لینے سے قالب کا مخلوط جوڑی دار جس کو ہم ہمیشہ کی طرح ستارہ سے ظاہر کرتے ہیں حاصل ہوگا
\begin{align}
	T^*=
	\begin{pmatrix}
		T_{11}^* & T_{12}^* & \dots & T_{1n}^*\\
		T_{21}^* & T_{22}^* & \dots & T_{2n}^*\\
		\vdots & \vdots & & \vdots\\
		T_{n1}^* & T_{n2}^* & \dots & T_{nn}^*
	\end{pmatrix}
	;\quad a^*=
	\begin{pmatrix}
		a_1^*\\a_2^*\\\vdots\\a_n^*
	\end{pmatrix}
\end{align}
تمام ارکان حقیقی ہونے کی صورت میں قالب حقیقی ہوگا اور تمام خیالی ہونے کی صورت میں خیالی ہوگا
\begin{align}
	T^*=T \text{\RL{حقیقی}};\quad T^*=-T\text{\RL{خیالی}}
\end{align}
ایک قالب کا تبدیل محل جوڑی دار اس کا ہرمیشی جوڑی دار یا شریق قالب جسے خنجر سے ظاہر کیا جاتا ہے ہوگا
\begin{align}
	T^\dagger\equiv\tilde{T}^*=
	\begin{pmatrix}
		T_{11}^* & T_{21}^* & \dots & T_{n1}^*\\
		T_{12}^* & T_{22}^* & \dots & T_{n2}^*\\
		\vdots & \vdots & & \vdots\\
		T_{1n}^* & T_{2n}^* & \dots & T_{nn}^*
	\end{pmatrix}
	;\quad a^\dagger\equiv\tilde{a}^*=
	\begin{pmatrix}
		a_1^* & a_2^* & \dots & a_n^*
	\end{pmatrix}
\end{align}
اگر ایک چکور قالب اپنے ہرمیشی جوڑی دار کے برابر ہو ہرمیشی یا خود شریق قالب ہوگا اگر ہرمیشی جوڑی دار منفی علامت متعارف کرتا ہو قالب منحرف ہرمیشی یا خلاف ہرمیشی ہوگا
\begin{align}
	T^\dagger=T\text{\RL{ہرمیشی}};\quad T^\dagger=-T\text{\RL{منحرف ہرمیشی}}
\end{align}
اس علامتیت میں دو سمتیات کے اندرونی ضرب کو کسی معیاری عمودی اساس کے لحاظ سے نہایت خوبصورتی کے ساتھ قالبی ضرب کی صورت میں لکھا جا سکتا ہے
\begin{align}
	\langle\alpha|\beta\rangle=a^\dagger b
\end{align}
دیہان رہے کہ اس رکوع میں متعارف تینوں اعمال کا دو مرتبہ اطلاق کرنے سے واپس اصل قالب حاصل ہوتا ہے۔ عام طور پر قالبی ضرب غیر مقلبی \عددی{ST\neq TS} ہوگا ضرب لکھنے کے دونوں طریقوں کے بیچ فرق کو مقلب کہتے ہیں
\begin{align}
	[S, T]\equiv ST-TS
\end{align}
حاصل ضرب کا تبدیل محل الٹ ترتیب میں تبدیل محلوں کا حاصل ضرب ہوگا 
\begin{align}
	(\widetilde{ST})=\tilde{T}\tilde{S}
\end{align}
اور یہی کچھ ہرمیشی جوڑی دار کے لئے بھی درست ہوگا
\begin{align}
	(ST)^\dagger=T^\dagger S^\dagger
\end{align}
اکائی قالب خطی تبادلہ کو ظاہر کرتا ہے جو ہر سمتیہ کو اپنے میں ہی لے جاتا ہے مرکزی وتر پر ایک اور باقی تمام ارکان صفر پر مشتمل ہوتا ہے  
\begin{align}
	I\equiv
	\begin{pmatrix}
		1 & 0 & \dots & 0\\
		0 & 1 & \dots & 0\\
		\vdots & \vdots & & \vdots\\
		0 & 0 & \dots & 1
	\end{pmatrix}
\end{align}
دوسرے لفظوں میں درجہ ذیل ہوگا
\begin{align}
	I_{ij}=\delta_{ij}
\end{align}
چوکور قالب کا معکوس جسے \عددی{T^{-1}} لکھا جاتا ہے کی تعریف ہمیشہ کی طرح درجہ ذیل ہوگی
\begin{align}
	T^{-1}T=TT^{-1}=I
\end{align}
ایک قالب کا معکوس صرف اور صرف اس صورت ہوگا جب سکا مقطع غیر صفر ہو در حقیقت 
\begin{align}
	T^{-1}=\frac{1}{\det T}\tilde{C}
\end{align}
ہوگا۔ جہاں ہم ضربیوں کا قالب \عددی{C} ہے رکن \عددی{T_{ij}} کا ہم ضربی \عددی{(-1)^{i+j}} ضرب اس ذیلی قالب کے مقطع کا حاصل ضرب ہوگا جو \عددی{T} کے \عددی{i}ویں صف اور \عددی{j}ویں قطار کو مٹانے سے حاصل ہوگا۔ ایسا قالب جس کا معکوس نہ پیا جاتا ہو نادر کہلاتا ہے۔ حاصل ضرب کا معکوس اگر موجود ہو انکئ معکوسوں کا الٹ نظم میں حاصل ضرب ہوگا
\begin{align}
	(ST)^{-1}=T^{-1}S^{-1}
\end{align}
ایسا قالب جس کا معکوس اس کے ہرمیشی جوڑی دار کے برابر ہو اکہرا کہلاتا ہے
\begin{align}
	U^\dagger = U^{-1}\text{\RL{اکہرا}}
\end{align}
یہ فرض کرتے ہوئے کہ اساس معیاری عمودی ہے اکہرا قالب کے قطار معیاری عمودی سلسلہ قائم کرتے ہیں اور اس کے صف بھی ایسا ہی کرتے ہیں مساوات \عددی{A.50} 
\begin{align}
	\langle\alpha'|\beta'\rangle=a'^\dagger b'=(Ua)^\dagger(Ub)=a^\dagger U^\dagger Ub=a^\dagger b=\langle\alpha|\beta\rangle
\end{align}
کی بدولت ایسے خطی تبادلہ جنہیں اکہرا قوالب ظاہر کرتے ہیں اندرونی حاصل ضرب برقرار رکھتے ہیں۔
\ابتدا{سوال}
درجہ ذیل قوالب لیتے ہوئے
\begin{align*}
	A=
	\begin{pmatrix}
		-1 & 1 & i\\
		2 & 0 & 3\\
		2i & -2i & 2
	\end{pmatrix}
	,\quad B=
	\begin{pmatrix}
		2 & 0 & -i\\
		0 & 1 & 0\\
		i & 3 & 2
	\end{pmatrix}
\end{align*}
درجہ ذیل کا حساب لگائیں: (الف) \عددی{A+B}، (ب) \عددی{AB}، (ج) \عددی{[A,B]}، (د) \عددی{\tilde{A}}، (ھ) \عددی{A^*}، (و) \عددی{A^\dagger}، (گ) \عددی{\det(B)}، (ح) \عددی{B^{-1}}۔ دیکھائیں کہ \عددی{BB^{-1}=I} ہے۔ کیا \عددی{A} کا معکوس مجود ہے؟
\انتہا{سوال}
\ابتدا{سوال}
قالبی قوالب
\begin{align*}
	a=
	\begin{pmatrix}
		i\\2i\\2
	\end{pmatrix}
	,\quad b=
	\begin{pmatrix}
		2\\(1-i)\\0
	\end{pmatrix}
\end{align*}
اور سوال \عددی{A.8} میں مستعمل چکور قالب استعمال کرتے ہوئے چرجہ ذیل حاصل کریں:

(الف) \عددی{Aa}

(ب) \عددی{a^\dagger b}

(ج) \عددی{\tilde{a}Bb}

(د) \عددی{ab^\dagger}
\انتہا{سوال}
\ابتدا{سوال}
درجہ ذیل میں صریحاً قوالب تیار کرتے ہوئے دیکھائیں کہ کسی بھی قالب \عددی{T} کو درجہ ذیل لکھا جا سکتا ہے

(الف) تشاکلی قالب \عددی{S} اور خلاف تشاکلی قالب \عددی{A} کا مجموعہ۔

(ب) حقیقی قالب \عددی{R} اور خیالی قالب \عددی{M} کا مجموعہ۔

(ج) ہرمیشی قالب \عددی{H} اور منحرف ہرمیشی قالب \عددی{K} کا مجموعہ۔
\انتہا{سوال}
\ابتدا{سوال}
مساوات \عددی{A.52}، \عددی{A.53} اور \عددی{A.58} ثابت کریں۔ دیکھائیں کہ دو اکہرا قوالب کا حاصل ضرب اکہرا ہوگا۔ کن شرائط کہ تحط دو ہرمیشی قوالب کا حآصل ضرب ہرمیشی ہوگا؟ کیا دو اکہرا قوالب کا مجموعہ اکہرا ہوگا؟ کیا دو ہرمیشی قوالب کا مجموعہ ہرمیشی ہوگا؟
\انتہا{سوال}
\ابتدا{سوال}
دیکھائیں کہ اکہرا قالب کے صف اور قطار عمودی معیاری سلسلہ قائم کرتے ہیں۔
\انتہا{سوال}
\ابتدا{سوال}
یہ یاد رکھتے ہوئے کہ \عددی{\det(\tilde{T})=\det(T)} دیکھائیں کہ ہرمیشی قالب کا مقطع حقیقی ہوگا اکہرا قالب کے مقطع کا معیار \عددی{1} ہوگا جس کی بنا اس کا نام اکہرا قالب ہے اور معیاری عمودی قالب کا مقطع یا \عددی{+1} یا \عددی{-1} ہوگا۔
\انتہا{سوال}
\حصہ{تبدیلی اساس}\شناخت{ضمیمہ_تبدیلی_اساس}
کسی بھی سمتیہ یا خطی تبادلہ کو ظاہر کرنے والے قالب کے ارکان اساس کا انتخاب پر منحصر ہوگا۔ اساس تبدیل کرنے سے یہ عداد کس طرح تبدیل ہوتے ہیں اس پر غور کرتے ہیں۔ پرانے اساسی سمتیات \عددی{|e_i\rangle} کسی بھی سمتیہ کی طرح ان نئی سمتیات \عددی{|f_i\rangle} کا خطی مجموعہ ہونگے
\begin{align*}
	|e_1\rangle &= S_{11}|f_1\rangle + S_{21}|f_2\rangle + \dots + S_{n1}|f_n\rangle\\
	|e_2\rangle &= S_{12}|f_1\rangle + S_{22}|f_2\rangle +
	\dots + S_{n2}|f_n\rangle\\
	&\dots\\
	|e_n\rangle &= S_{1n}|f_1\rangle + S_{2n}|f_2\rangle + \dots + S_{nn}|f_n\rangle
\end{align*}
جہاں \عددی{S_{ij}} مخلوط اعداد کا سلسلہ ہوگا یا مختصراً
\begin{align}
	|e_{j}\rangle = \sum_{i=1}^{n}S_{ij}|f_i\rangle,\quad(j=1, 2, \dots, n)
\end{align}
یہ خود ایک خطی تبادلہ ہے مساوات \عددی{A.30} سے موازنہ کریں اور ہم جانتے ہیں کہ ارکان کا تبادلہ کس طرح ہوگا
\begin{align}
	a_i^f = \sum_{j=1}^{n}S_{ij}a_j^e
\end{align}
جہاں زیر نوشت اساس کو ظاہر۔ قالبی روپ میں درجہ ذیل ہوگا
\begin{align}
	a^f = Sa^e
\end{align}
خطی تبادلہ \عددی{\hat{T}} کو ظاہر کرنے والا قالب اساس کی تبدیلی سے کس طرح تبدیل ہوگا؟ پرانے اساس میں 
\begin{align*}
	a^{'e} = T^ea^e
\end{align*}
اور مساوات \عددی{A.63} تھے دونوں اطراف کو \عددی{S^{-1}} سے ضرب دیتے ہوئے جس میں \عددی{a^e=S^{-1}a^f} کا منتقی نتیجہ شامل ہے لہٰذا
\begin{align*}
	a^{'f} = Sa^{'e} = S(T^ea^e) = ST^eS^{-1}a^f
\end{align*}
ظاہری طور پر
\begin{align}
	T^f = ST^eS^{-1}
\end{align}
ہوگا۔ عمومی طور پر دو قوالب \عددی{T_1} اور \عددی{T_2} اس صورت متشابہ ہونگے جب کسی غیر نادر قالب \عددی{S} کی صورت میں \عددی{T_2 = ST_1S^{-1}} ہو۔ یوں ہم دریافت کر چکے کہ مختلف اساس لے لحاظ سے ایک ہی خطی تبادلہ کو ظاہر کرنے والے قالب متشابہ ہونگے۔ اتفاقی طور پر اگر پہلی اساس معیاری عمودی ہو تب دوسری اساس صرف اس صورت میں معیاری عمودی ہوگا جب قالب \عددی{S} اکہرا ہو سوال \عددی{A.16} دیکھیں۔ چونکہ ہم صرف معیاری عمودی اساس میں کام کرتے ہیں لہٰذا ہماری دلچسپی بنیادی طور پر اکہرا متشابہت تبادلہ میں ہے۔

اگرچہ نئی اساسوں میں کوئی بھی خطی تبادلہ کے ارکان بہت مختلف نظر آتے ہیں قالب سے وابستہ دو اعداد یعنی مقطع اور آثار قالب تبدیل نہیں ہوتے چونکہ حاصل ضرب کا مقطع مقطع کا حاصل ضرب ہوگا لہٰذا درجہ ذیل ہوگا 
\begin{align}
	\det(T^f) = \det(ST^eS^{-1}) = \det(S)\det(T^e)\det(S^{-1}) = \det T^e
\end{align}
اور آثار قالب جو وتری ارکان کا مجموعہ ہے 
\begin{align}
	Tr(T)\equiv\sum_{i=1}^{m}T_{ii}
\end{align}
درجہ ذیل خاصیت سوال \عددی{A.17} دیکھیں
\begin{align}
	Tr(T_1T_2) = Tr(T_2T_1)
\end{align}
جہاں \عددی{T_1} اور \عددی{T_2} کوئی بھی دو قوالب ہیں لہٰذا درجہ ذیل ہوگا
\begin{align}
	Tr(T^f) = Tr(ST^eS^{-1}) = Tr(T_eS^{-1}S) = Tr(T^e)
\end{align}
\ابتدا{سوال}
تین ابعاد میں سمتیات کے لئے معیاری اساس \عددی{(\hat{i}, \hat{j}, \hat{k})} استعمال کرتے ہوئے۔

(الف) مبدہ کی طرف دیکھتے ہوئے خلاف گھڑی \عددی{z} محور کے گرد زاویہ \عددی{\theta} گھومنے کو ظاہر کرنے والا قالب تیار کریں۔

(ب) نقطہ \عددی{(1, 1, 1)} سے گزرتے ہوئے محور کے گرد محور سے مبدہ کی طرف دیکھتے ہوئے خلاف گھڑی \عددی{120\degree} گھومنے کو ظاہر کرنے والا قالب تیار کریں۔

(ج) مستوی \عددی{xy} میں عکس کو ظاہر کرنے والا قالب تیار کریں۔

(د) تصدیق کریں کہ یہ تمام قوالب معیاری عمودی ہیں اور ان کے مقطع کا حساب کریں۔
\انتہا{سوال}
\ابتدا{سوال}
عمومی اساس \عددی{(\hat{i}, \hat{j}, \hat{k})} استعمال کرتے ہوئے محور \عددی{x} کے گرد زاویہ \عددی{\theta} گھومنے کو ظاہر کرنے والا قالب \عددی{T_x} اور محور \عددی{y} کے گرد زاویہ \عددی{\theta} گھمونے کو ظاہر کرنے والے قالب \عددی{T_y} تیار کریں۔ فرض کریں اب ہم اساس تبدیل کر کے \عددی{\hat{i}'=\hat{j}}، \عددی{\hat{j}'=-\hat{i}}، \عددی{\hat{k}'=\hat{k}} لیتے ہیں اساس کی اس تبدیلی کو پیدہ کرنے والے قالب\عددی{S} تیار کریں اور تصدیق کریں کہ \عددی{ST_xS^{-1}} اور \عددی{ST_yS^{-1}} آپ کے توقعات کے عین مطابق ہے۔
\انتہا{سوال}
\ابتدا{سوال}
دیکھائیں کہ متشابہت قالبی ضرب برقرار رکھتا ہے یعنی اگر \عددی{A^eB^e=C^e} ہو تب \عددی{A^fB^f=C^f} ہوگا۔ عمومی طور پر متشابہت تشاکلی حقیقت یا ہرمیشی پن برقرار نہیں رکھتا لیکن دیکھائیں اگر \عددی{S} اکہرا ہو اور \عددی{H^e} ہرمیشی ہو تب \عددی{H^f} ہرمیشی ہوگا۔ دیکھائیں کہ \عددی{S} صرف اس صورت معیاری عمودی اساس کو دوسری معیاری عمودی اساس میں منتقل کرے گا اگر یہ اکہرا ہو۔
\انتہا{سوال}
\ابتدا{سوال}
دیکھائیں کہ \عددی{Tr(T_1T_2)=Tr(T_2T_1)} یوں \عددی{Tr(T_1T_2T_3)=Tr(T_2T_3T_1)} ہوگا کیا عام طور پر \عددی{Tr(T_1T_2T_3)=Tr(T_2T_1T_3)} ہوگا؟ اسکو ٹھیک یا غلط ثابت کریں۔ اشارہ: بہترین غلط ثبوت اسکی اُلٹ مثال پیش کرنا ہے جتنا سادہ ہو اتنا ہی بہتر ہے۔
\انتہا{سوال}
\حصہ{امتیازی سمتیات اور امتیازی اقدار}\شناخت{ضمیمہ_امتیازی_تفاعلات_و_اقدار}
تھرہ فضا میں کسی مخصوص محور کے گرد زاویہ \(\theta\)گھمانے کو ظاہر کرنے والے خطی تبادلہ پر غور کریں۔ زیادہ تر سمتیات پچیدہ انداز سے تبدیل ہوں گے یہ اس محور کے گرد مخروط پر حرکت کریں گے لیکن وہ سمتیات جو اسی محور پر پائے جاتے ہوں کا رویہ سادہ ہوگا وہ بلکل تبدیل نہیں ہوں گے \((\hat{T}\mid \alpha \rangle=\mid \alpha \rangle)\)۔ اگر \(\theta\) کی قیمت \(\num{180}\degree\) ہو تب استوائی مستوی میں پائے جانے والے سمتیات کی علامت تبدیل ہوگی \((\hat{T}\mid \alpha \rangle = -\mid \alpha \rangle)\)۔ مخلوط سمتی فضا میں ہر خطی تبادلہ کے اس طرح کے مخصوص سمتیات پائے جاتے ہیں جو اپنے آپ کے غیر سمتی مضرب میں تبدیل ہوتے۔
\begin{align}
	\hat{T}\mid\alpha\rangle = \lambda\mid\alpha\rangle
\end{align}
انہیں اس تبادلہ کے امتیازی سمتیات کہتے ہیں اور مخلوط عدد \(\lambda\) کو انکا امتیازی قدر کہتے ہیں۔ معدوم سمتیہ محمل معنوں میں مساوات \عددی{A.69} کو کسی بھی \(\hat{T}\) اور \(\lambda\) کے لئے مطمعن کرتا ہے اسے امتیازی سمتیات میں نہیں گنا جاتا تکنیکی طور پر ایک امتیازی سمتیہ سے مراد وہ غیر صفر سمتیہ ہے جو مساوات \عددی{A.69} کع مطمعن کرتا ہو۔ دیہان رہے کہ امتیازی سمتیہ کا کوئی بھی غیر صفر مضرب بھی امتیازی سمتیہ ہوگا جس کی امتیازی قدر وہی ہوگی۔

کسی مخصوص اساس کے لحاظ سے امتیازی سمتیہ مساوات قالبی روپ 
\begin{align}
	Ta = \lambda a
\end{align}
جہاں \عددی{a} غیر صفر ہے یا
\begin{align}
	(T-\lambda I)a = \num{0}
\end{align}
اختیار کرتا ہے۔ یہاں \num{0} ایسا صفر قالب ہے جس کے تمام ارکان صفر ہیں۔ اب اگر قالب \((T-\lambda I)\) کا معکوس ہوتا ہم مساوات \عددی{A.71} کو دونوں اطراف کو \((T-\lambda I)^{-1}\) سے ضرب دے کر \(a=0\) اخز کرتے۔ لیکن ہم \عددی{a} کو غیر صفر فرض کر چکے ہیں لہٰذا \((T-\lambda I)\) لاظماً نادر ہوگا جس سے مراد یہ ہے کہ اس کا مقطع صفر ہوگا
\begin{align}
	\det(T-\lambda I)=
	\begin{vmatrix}
		(T_{11}-\lambda) & T_{12} & \dots & T_{1n}\\
		T_{21} & (T_{22}-\lambda) & \dots & T_{2n}\\
		\vdots & \vdots & & \vdots\\
		T_{n1} & T_{n2} & \dots & (T_{nn}-\lambda)
	\end{vmatrix}
		= 0.
\end{align}
مقطع کھولنے سے \(\lambda\) کی الجبرائی مساوات
\begin{align}
	C_n\lambda^{n}+C_{n-1}\lambda^{n-1}+\dots+C_1\lambda+C_0 = 0
\end{align}
حاصل ہوےی ہے جہاں عددی سر \عددی{C_i} ارکان \عددی{T} کے تابع ہیں سوال \عددی{A.20} دیکھیں۔ اس کو قالب کی امتیازی مساوات کہتے ہیں اور اس کے حل امتیازی اقدار کا تعین کرتے ہیں۔ دیہان رہے کہ یہ \عددی{n} رطبی ماساوات ہے لہٰذا الجبرا کے بنیادی مسئلہ کے تحت اس کے \عددی{n} مخلوط ھزر ہوںگے البتہ ان میں سے چند مرکب جذر ہو سکتے ہیں لہٰذا ہم صرف اتنا کہہ سکتے ہیں کہ \(n \times n\) قالب کا کم سے کم ایک اور زیادہ سے زیادہ \عددی{n} منفرد امتیازی اقدار ہو سکتے ہیں۔ قالب کے تمام امتیازی اقدار کے زخیرہ کو اس کا طیف کہتے ہیں اگر دو یا دو سے زیادہ خطی غیر تابع امتیازی سمتیات کا ایک ہی امتیازی قدر ہو ہم کہتے ہیں کہ طیف انحطاطی ہے۔

امتیازی سمتیات تیار کرنے کا عام طور پر سادہ ترین طریقہ یہ ہے کہ مساوات \عددی{A.70} میں ہر ایک \(\lambda\) ڈال کر \عددی{a} کے ارکان کے لئے ہاتھ سے حل کریں۔ میں اس عمل کو ایک مثال کی صورت میں سمجھاتا ہوں۔
\ابتدا{مثال}
درج ذیل قالب کے امتیازی اقدار اور امتیازی سمتیات تلاش کریں:

\begin{align}
	M=
	\begin{pmatrix}
		2 & 0 & -2\\
		-2i & i & 2i\\
		1 & 0 & -1
	\end{pmatrix}
\end{align}
حل: اس کی امتیازی مساوات 
\begin{align}
	\begin{vmatrix}
		(2-\lambda) & 0 & -2\\
		-2i & (i-\lambda) & 2i\\
		1 & 0 & (-1-\lambda)
	\end{vmatrix}
		=-\lambda^3 + (1+i)\lambda^2-i\lambda = 0
\end{align}
ہے۔ جس کے جذر \عددی{1, 0} اور \عددی{i} ہے۔ پہلی امتیازی سمتیہ کے اقدار \عددی{(a_1, a_2, a_3)} لیتے ہوئے
\begin{align*}
	\begin{pmatrix}
		2 & 0 & -2\\
		-2i & i & 2i\\
		1 & 0 & -1
	\end{pmatrix}
	\begin{pmatrix}
		a_1\\
		a_2\\
		a_3
	\end{pmatrix}
		=0
	\begin{pmatrix}
		a_1\\
		a_2\\
		a_3
	\end{pmatrix}
		=
	\begin{pmatrix}
		0\\0\\0
	\end{pmatrix}
\end{align*}
ہوگا۔ جس سے درجہ ذیل تین مساوات ملتے ہیں
\begin{align*}
	2a_1 - 2a_3 &= 0\\
	-2ia_1 + ia_2 + 2ia_3 &= 0\\
	a_1 - a_3 &= 0
\end{align*}
ان میں سے پہلی ماساوات \عددی{a_1} کی صورت میں \عددی{a_3} کا تعین کرتی ہے \عددی{a_3 = a_1} دوسری \عددی{a_2} کا تعین کرتی ہے \عددی{a_2 = 0} اور تیسری فالتو مساوات ہے۔ ہم \عددی{a_1 = 1} چن سکتے ہیں چونکہ امتیازی سمتیہ کا کوئی بھی مضرب امتیازی سمتیہ ہی ہوگا
\begin{align}
	a^{(1)}=
	\begin{pmatrix}
		1\\0\\1
	\end{pmatrix}
		,\lambda_1 = 0 \text{\RL{کے لئے}}
\end{align}
دوسری امتیازی سمتیہ کے لئے ارکان کے لئے وہی علامتیں استعمال کرتے ہوئے 
\begin{align*}
	\begin{pmatrix}
		2 & 0 & -2\\
		-2i & i & 2i\\
		1 & 0 & -1
	\end{pmatrix}
	\begin{pmatrix}
		a_1\\a_2\\a_3
	\end{pmatrix}
		=1
	\begin{pmatrix}
		a_1\\a_2\\a_3
	\end{pmatrix}
		=
	\begin{pmatrix}
		a_1\\a_2\\a_3
	\end{pmatrix}
\end{align*}
ملتا ہے جس سے درجہ ذیل مساوات حاصل ہوںگے
\begin{align*}
	2a_1-2a_3 &= a_1\\
	-2ia_1 + ia_2 + 2ia_3 &= a_2\\
	a_1 - a_3 &= a_3
\end{align*}
جس کے حل \عددی{a_3 =(1/2)a_1, a_2=[(1-i)/2]a_1} ہیں اس مرتبہ میں \عددی{a_1 = 2} منتخب کرتا ہوں لہٰذا
\begin{align}
	a^{(2)} =
	\begin{pmatrix}
		2\\1-i\\1
	\end{pmatrix}
		, \lambda_2 = 1 \text{\RL{کے لئے}}
\end{align}
ہوگا۔ آخر میں تیسرا امتیازی سمتیہ
\begin{align*}
	\begin{pmatrix}
		2 & 0 & -2\\
		-2i & i & 2i\\
		1 & 0 & -1
	\end{pmatrix}
	\begin{pmatrix}
		a_1\\a_2\\a_3
	\end{pmatrix}
		=i
	\begin{pmatrix}
		a_1\\a_2\\a_3
	\end{pmatrix}
		=
	\begin{pmatrix}
		ia_1\\ia_2\\ia_3
	\end{pmatrix}
\end{align*}
درجہ ذیل مساوات دیگا
\begin{align*}
	2a_1-2a_3 &= ia_1\\
	-2ia_1 + ia_2 + 2ia_3 &= ia_2\\
	a_1 - a_3 &= ia_3
\end{align*}
جس کے حل \عددی{a_3 = a_1 = 0} ہیں جہاں \عددی{a_2} غیر متعین ہیں۔ ہم \عددی{a_2 = 1} منتخب کرتے ہیں یوں درجہ ذیل ہوگا
\begin{align}
	a^{(3)}=
	\begin{pmatrix}
		0\\1\\0
	\end{pmatrix}
	, \lambda_3 = i \text{\RL{کے لئے}}
\end{align}
\انتہا{مثال}
اگر امتیازی سمتیات فضا کا احاطہ کرتے ہوں جیسا گزشتہ مثال میں کرتے تھے ہم انہیں اساس کے طور پر استعمال کر سکتے ہیں
\begin{align*}
	\hat{T}\mid f_1\rangle &= \lambda_1\mid f_1\rangle,\\
	\hat{T}\mid f_2\rangle &= \lambda_2\mid f_2\rangle,\\
	&\dots\\
	\hat{T}\mid f_n\rangle &= \lambda_n\mid f_n\rangle\\
\end{align*}
اس اساس میں \عددی{\hat{T}} کو ظاہر کرنے والا قالب انتہائی سادہ روپ اختیار کرتا ہے جس میں امتیازی اقدار مرکزی وتر پر پائے جاتے ہیں جبکہ باقی تمام ارکان صفر ہوںگے
\begin{align}
	T=
	\begin{pmatrix}
		\lambda_1 & 0 & \dots & 0\\
		0 & \lambda_2 & \dots & 0\\
		\vdots & \vdots & & \vdots\\
		0 & 0 & \dots & \lambda_n
	\end{pmatrix}
\end{align}
اور معمول شدہ امتیازی سمتیات درجہ ذیل ہوںگے
\begin{align}
	\begin{pmatrix}
		1\\0\\0\\ \vdots\\0
	\end{pmatrix}
	,
	\begin{pmatrix}
		0\\1\\0\\\vdots\\0
	\end{pmatrix}
	, \dots,
	\begin{pmatrix}
		0\\0\\0\\\vdots\\1
	\end{pmatrix}
\end{align}
ایسا قالب جس کو اساس کی تبدیلی سے وتری روپ مساوات \عددی{A.79} کی صورت میں لایا جا سکے وتری کہلاتا ہے ظاہر ہے کہ ای کقالب صرف اس صورت میں وتری ہوگا جب اس کے امتیازی سمتیات فضا کا احاطہ کرتے ہوں۔ متشابہت قالب جو وتری بناتا ہے کو پرانی اساس میں معمول شدہ امتیازی سمتیات بطور \عددی{S^{-1}} کے قطار لیتے ہوئے تیار کیا جاسکتا ہے
\begin{align}
	(S^{-1})_{ij} = (a^{(j)})_i
\end{align}
\ابتدا{مثال}
مثال\عددی{A.1} میں
\begin{align*}
	S^{-1} =
	\begin{pmatrix}
		1 & 2 & 0\\
		0 & (1-i) & 1\\
		1 & 1 & 0
	\end{pmatrix}
\end{align*}
لہٰذا مساوات \عددی{A.57} استعمال کرتے ہوئے
\begin{align*}
	S=
	\begin{pmatrix}
		-1 & 0 & 2\\
		1 & 0 & -1\\
		(i-1) & 1 & (1-i)
	\end{pmatrix}
\end{align*}
اور آپ تصدیق کر سکتے ہیں کہ
\begin{align*}
	Sa^{(1)}=
	\begin{pmatrix}
		1\\0\\0
	\end{pmatrix}
	,
	Sa^{(2)}=
	\begin{pmatrix}
		0\\1\\0
	\end{pmatrix}
	,
	Sa^{(3)}=
	\begin{pmatrix}
		0\\0\\1
	\end{pmatrix}
\end{align*}
اور
\begin{align*}
	SMS^{-1}=
	\begin{pmatrix}
		0 & 0 & 0\\
		0 & 1 & 0\\
		0 & 0 & i
	\end{pmatrix}
\end{align*}
ہوگا۔
\انتہا{مثال}
قالب کو وتری روپ میں لانے کا صاف نظر آنے والا فائدہ ہے اس کے ساتھ کام کرنا نہایت آسان ہے۔ بد قسمتی سے ہر قالب کو وتری نہیں بنایا جا سکتا امتایزی سمتیات کو فضا کا احاطہ کرنا ہوگا۔ اگر امتیازی مساوات کے \عددی{n}  منفرد جذر ہوں تب قالب لظاماً وتری بنایا جا سکتا ہے لیکن مرقب جذر کی صورت میں بھی ممکن ہے کہ یہ وتری بنانے کے قابل ہو۔ وتری بنانے کے غیر قابل قالب کے لئے سوال \عددی{A.19} دیکیھیں۔ کیا بہتر ہوتا اگر تمام امتیازی سمتیات معلوم کرنے سے قبل ہم جان سکتے کہ آیہ ایک قالب وتری بنانے کے قابل ہے یا نہیں ایک کافی لیکن لاظمی نہیں شرط درجہ ذیل ہے ایک قالب جو اپنے ہرمیشی جوڑی دار کے ساتھ مقلوب ہو عمودی قالب کہلاتا ہے
\begin{align}
	[N^\dagger,N] = 0, \text{\RL{سادہ}}
\end{align}
ہر عمودی قالب وتری بنانے کے قابل ہے اس کے امتیازی سمتیات فضا کا احاطہ کرتے ہیں۔ بلخصوص ہر ہرمیشی قالب اور اکہرا قالب وتری بنانے کے قابل ہے۔

فرض کریں ہمارے پاس دو وتری بنانے کے قابل قالب ہیں قونٹائی معملات میں عموماً ایک سوال کھڑا ہوتا ہے کیا انہیں بیک وقت وتری بنایا جا سکتا ہے یعنی ایک ہی مشابہت قالب \عددی{S} کے ذریع؟ دوسرے لفظوں میں کیا ایسی اساس موجود ہے جس میں دونوں وتری بنائے جا سکتے ہیں؟ اس کا جواب ہے کہ صرف اور صرف اس صورت یہ ممکن ہوگا جب دونوں قالب مقلوبی ہوں سوال \عددی{A.22} دیکھیں۔
\ابتدا{سوال}
مستوی \عددی{xy} میں گھومنے کو ظاہر کرنے والا \عددی{2 \times 2} قالب
\begin{align}
	T=
	\begin{pmatrix}
		\cos\theta & -\sin\theta\\
		\sin\theta & \cos\theta
	\end{pmatrix}
\end{align}
دیکجھائیں کہ ماسوائے مخصوص زاویوں کے بتائیں وہ کونسے زاویہ ہیں؟ اس قالب کا کوئی حقیقی امتیازی قدر نہیں پایا جاتا۔ یہ اس ہندسی حقیقت کا اخز ہے کہ مستوی میں کوئی بھی سمتیہ گھمانے کے ذریع اپنے آپ میں نہیں پہنچایا جا سکتا اس کا موازنہ تین ابعاد میں گھمانے سے کریں۔ البتہ اس قالب کے مخلوط امتیازی اقدار اور امتیازی سمتیات ہوںگے۔ انہیں تلاش کریں۔ قالب \عددی{T} کو وتری بنانے والا قالب \عددی{S} تیار کریں۔ متشابہت تبادلہ \عددی{STS^{-1}} صریحاً کریں اور دیکھائیں کہ یہ \عددی{T}  کو وتری روپ میں گھٹاتا ہے۔
\انتہا{سوال}
\ابتدا{سوال}
درجہ ذیل قالب کے امتیازی اقدار اور امتیازی سمتیات تلاش کریں
\begin{align*}
	M=
	\begin{pmatrix}
		1 & 1\\
		0 & 1
	\end{pmatrix}
\end{align*}
کیا یہ قالب وتری بنانے کے قابل ہے؟
\انتہا{سوال}
\ابتدا{سوال}
دیکھائیں کہ امتیازی مساوات مساوات \عددی{A.73} کا پہلا، دوسرا اور آخری عددی سر درجہ ذیل ہیں
\begin{align}
	C_n = (-1)^n, C_{n-1} = (-1)^{n-1}Tr(T), \text{\RL{اور}} C_0 = \det(T)
\end{align}
ایک \عددی{3 \times 3} قالب جس کے ارکان \عددی{T_{ij}} ہوں کا \عددی{C_1} کیا ہوگا؟
\انتہا{سوال}
\ابتدا{سوال}
صاف ظاہر ہے کہ وتری قالب کا آسار قالب اس کے امتیازی اقدار کا مجموعہ اور اس کا مقطع ان کا حاصل ضرب ہوگا صرف مساوات \عددی{A.79} کو دیکھنے کی دیر ہے یوں مساوات \عددی{A.65} اور \عددی{A.68} کے تحت کسی بھی وتری بنانے کے قابل قالب کے لئے بھی ایسا ہی ہوگا۔ درجہ ذیل حقیقت کسی بھی قالب کے لئے ثابت کریں
\begin{align}
	\det(T) = \lambda_1\lambda_2\dots\lambda_n,\quad Tr(T)=\lambda_1+\lambda_2+\dots+\lambda_n
\end{align}
امتیازی مساوات کے \عددی{n} حل یہاں \عددی{\lambda} ہیں مرقب جذر کی صورت میں حلوں سے کم خطی غیر تابع امتیازی سمتیات ہوسکتے ہیں لیکن ہم \عددی{\lambda} کو اتنی مرتبہ ہی گنتے ہیں جتنی مرتبہ یہ پایا جاتا ہو۔ اشارہ: امتیازی مساوات کو درجہ ذیل روپ میں لکھیں
\begin{align*}
	(\lambda_1-\lambda)(\lambda_2-\lambda)\dots(\lambda_n-\lambda) = 0
\end{align*}
اور سوال \عددی{A.20} کا نتیجہ استعمال کریں۔
\انتہا{سوال}
\ابتدا{سوال}

(الف) دیکھائیں اگر دو قالب کسی ایک اساس میں مقلوبی ہوں تب وہ ہر اساس میں مقلوبی ہوںگے یعنی درجہ ذیل ہوگا
\begin{align}
	[T^e_1, T^e_2] = 0 \Rightarrow [T^f_1, T^f_2] = 0
\end{align}
اشارہ: مساوات \عددی{A.64} استعمال کریں۔

(ب) دیکھائیں کہ اگر دو قالب بیک وقت وتری بنانے کے قابل ہوں تو وہ مقلوبی ہوںگے۔
\انتہا{سوال}
\ابتدا{سوال}
درجہ ذیل قالب لیں
\begin{align*}
	M=
	\begin{pmatrix}
		1 & 1\\
		1 & i
	\end{pmatrix}
\end{align*}
(الف) کیا یہ عمودی قالب ہے؟

(ب) کیا یہ وتری بنانے کے قالب ہے؟
\انتہا{سوال}
\حصہ{ہرمیشی تبادلہ}\شناخت{ضمیمہ_ہرمشی_تبادلے}
میں نے مساوات \عددی{A.48} میں قالب کے تبدیل محل و جوڑی دار \عددی{T^\dagger = \overset{\sim}{T^*}} کو اس کی ہرمیشی جوڑی دار یا شریک قالب کی تعریف قرار دیا۔ یہاں میں خطی تبادلہ کے ہرمیشی جوڑی دار کا زیادہ بنیادی تعریف پیش کرتا ہوں یہ وہ تبادلہ \عددی{\hat{T}^\dagger} ہے جس کا اطلاق اندرونی ضرب کے پہلے رکن پر وہی نتیجہ دیتا ہے جو دوسرے سمتیہ پر خود \عددی{\hat{T}} کا اطلاق دیگا
\begin{align}
	\langle\hat{T}^{\dagger}\alpha\mid\beta\rangle = \langle\alpha\mid\hat{T}\beta\rangle
\end{align}
جہاں \عددی{|\alpha\rangle} اور \عددی{|\beta\rangle} کوئی بھی سمتیات ہو سکتے ہیں۔ ہاشیہ اس کا مقابل اگر دو وتری بنانے کے قالب مقلوبی ہوں تب وہ بیک وقت وتری بنانے کے قابل ہوںگے ثابت کرنا اتنا آسان نہیں ہے۔ ہاشیہ آپ پوچھ سکتے ہیں ایسا تبادلہ لاظماً موجود ہوگا یہ ایک اچھا سوال ہے اس کا جواب ہے جی ہاں۔ میں آپ کو خبردار کرتا چلوں کہ اگرچہ ہر کوئی اسے استعمال کرتا ہے یہ فرصودہ علامتیت ہے سمتیات \عددی{\alpha} اور \عددی{\beta} ہیں جبکہ \عددی{|\alpha\rangle} اور \عددی{ؔ|\beta\rangle} سمتیات نہیں بلکہ نام ہیں۔ بلخصوص ان کے کوئی ریاضیاتی خواص نہیں پائے جاتے اور \عددی{\hat{T} \beta} کا فقرہ بے معانی ہے خطی تبادلہ کسی سمتیہ پر نا کہ نام پر عمل کرتے ہیں۔ لیکن اس علامت کا مطلب صاف ظاہر ہے سمتیہ \عددی{\hat{T}\beta} کا نام \عددی{\hat{T}|\beta\rangle} ہے اور سمتیہ \عددی{\hat{T}^\dagger|\alpha\rangle} اور سمتیہ \عددی{|\beta\rangle} کا اندرونی ضرب \عددی{\langle\hat{T}^\dagger\alpha|\beta\rangle} ہے۔ بلخصوص
\begin{align}
	\langle\alpha\mid c\beta\rangle = c\langle\alpha\mid\beta\rangle
\end{align}
جیاں کسی بھی غیر سمتی \عددی{c} کے لئے چرجہ ذیل ہوگا
\begin{align}
	\langle c\alpha\mid\beta\rangle = c^{*}\langle\alpha\mid\beta\rangle
\end{align}
اگر آپ ہمیشہ کی طرح معیاری عمودی اساس میں کام کر رہے ہوں خطی تبادلہ کے ہرمیشی جوڑی دار کو مطابقتی قالب کا ہرمیشی جوڑی دار ضاہر کریگا چونکہ مساوات \عددی{A.50} اور \عددی{A.53} استعمال کرتے ہوئے درجہ ذیل ہوگا
\begin{align}
	\langle\alpha\mid\hat{T}\beta\rangle = a^\dagger Tb = (T^\dagger a)^\dagger b = \langle\hat{T}^\dagger\alpha\mid\beta\rangle
\end{align}
یوں یہ علامتیت صوات ہے اور ہم چاہیں تو تبادلہ کی زبان اور چاہیں تو قوالب کی زبان میں بات کرستے ہیں۔

کوانٹائی میکانیات میں ہرمیشی تبادلہ \عددی{(\hat{T}^\dagger = \hat{T})} بنیادی کردار ادا کرتے ہیں۔ ہرمیشی تبادلہ کے امتیازی سمتیات اور امتیازی اقدار تین نہایت اہم خواص رکھتے ہیں۔

(الف) ہرمیشی تبادلہ کے امتیازی اقدار حقیقی ہیں:

ثبوت: فرض کریں \عددی{\hat{T}} کی ایک امتیازی قدر \عددی{\lambda} ہے \عددی{\hat{T}|\alpha\rangle = \lambda|\alpha\rangle} جہاں \عددی{|\alpha\rangle\neq|0\rangle} ہے۔ تب درجہ ذیل ہوگا
\begin{align*}
	\langle\alpha\mid\hat{T}\alpha\rangle = \langle\alpha\mid\lambda\alpha\rangle = \lambda\langle\alpha\mid\alpha\rangle
\end{align*}
ساتھ ہی \عددی{\hat{T}} ہرمیشی ہے لہٰذا درجہ ذیل ہوگا
\begin{align*}
	\langle\alpha\mid\hat{T}\alpha\rangle = \langle\hat{T}\alpha\mid\alpha\rangle = \langle\lambda\alpha\mid\alpha\rangle = \lambda^*\langle\alpha\mid\alpha\rangle
\end{align*}
لیکن \عددی{\langle\alpha|\alpha\rangle\neq0} ہے مساوات \عددی{A.20} لہٰذا \عددی{\lambda = \lambda^*} اور یوں \عددی{\lambda} حقیقی ہوگا۔

(ب) ہرمیشی تبادلہ کے منفرد امتیازی اقدار کے امتیازی سمتیات قائمہ ہونگے۔

ثبوت: فرض کریں \عددی{\hat{T}|\alpha\rangle = \lambda|\alpha\rangle} اور \عددی{\hat{T}|\beta\rangle = \mu|\beta\rangle} ہے جہاں \عددی{\lambda\neq\mu} ہے۔ تب
\begin{align*}
	\langle\alpha\mid\hat{T}\beta\rangle = \langle\alpha\mid\mu\beta\rangle = \mu\langle\alpha\mid\beta\rangle
\end{align*}
اور اگر \عددی{\hat{T}} ہرمیشی ہو درجہ ذیل ہوگا
\begin{align*}
	\langle\alpha\mid\hat{T}\beta\rangle = \langle\hat{T}\alpha\mid\beta\rangle = \langle\lambda\alpha\mid\beta\rangle = \lambda^*\langle\alpha|\beta\rangle
\end{align*}
لیکن \عددی{\lambda = \lambda^*} ہے جزو (الف) سے اور ہم فرض کر چکے ہیں کہ \عددی{\lambda\neq\mu} ہے لہٰذا \عددی{\langle\alpha|\beta\rangle = 0} ہوگا۔

(ج) ہرمیشی تبادلہ کے امتیازی سمتیات فضا کا احاطہ کرتے ہیں جیسا ہم دیکھ چکے ہیں یہ اس فکرے کے مترادف ہے کہ کسی بھی ہرمیشی قالب کو وتری بنایا جاسکتا ہے مساوات \عددی{A.82} دیکھیں۔ یہ حقیقت جو خاسی تکنیکی ہے وہ ریاضیاتی سہارا ہے جس پر زیادہ تر کوانٹائی میکانیات کھڑی ہے۔ چونکہ اس ثبوت کو لامتناہی ابعادی سمتی فضائوں تک وصت نہیں دی جا سکتی لہٰذا یہ ایک باریک لڑی ہے جس پر کوانٹائی میکانیات منحصر ہے۔

\ابتدا{سوال}
ہرمیشی خطی تبادلہ کو تمام سمتیات \عددی{|\alpha\rangle} اور \عددی{|\beta\rangle} کے لئے لاظماً \عددی{\langle\alpha|\hat{T}\beta\rangle = \langle\hat{T}\alpha|\beta\rangle} مطمعن کرنا ہوگا۔ دیکھائیں کہ اتنی حیرانی کی بات ہے کہ کافی ہے کہ تمام سمتیات \عددی{|\gamma\rangle} کے لئے \عددی{\langle\gamma|\hat{T}\gamma\rangle = \langle\hat{T}\gamma|\gamma\rangle} ہو۔ اشارہ: پہلے \عددی{|\gamma\rangle = |\alpha + |\beta\rangle} اور اس کے بعد \عددی{|\gamma\rangle = |\alpha\rangle + i|\beta\rangle} لیں۔ 
\انتہا{سوال}
\ابتدا{سوال}
درجہ ذیل لیں
\begin{align*}
	T=
	\begin{pmatrix}
		1 & 1-i\\
		1+i & 0
	\end{pmatrix}
\end{align*}
(الف) تصدیق کریں کہ \عددی{T} ہرمیشی ہے۔

(ب) اس کی امتیازی اقدار تلاش کریں (آپ دیکھیں گے کہ یہ حقیقی ہیں)۔

(ج) امتیازی سمتیات تلاش کر کے انکی معمولزنی کریں (آپ دیکھیں گے کہ یہ معیاری عمودی ہیں)۔

(د) اکہرا وتری بنانے والا قالب \عددی{S} تیار کریں اور صریحاً تصدیق کریں کہ یہ \عددی{T} کو وتری بناتا ہے۔

(ھ) تصدیق کریں کہ \عددی{T} اور سکے وتری روپ کے لئے مقطع \عددی{T} اور آسار \عددی{T} ایک جیسے ہیں۔
\انتہا{سوال}
\ابتدا{سوال}
درجہ ذیل ہرمیشی قالب لیں
\begin{align*}
	T=
	\begin{pmatrix}
		2 & i & 1\\
		-i & 2 & i\\
		1 & -i & 2
	\end{pmatrix}
\end{align*}
(الف) اس قالب کا مقطع \عددی{Tr(T)} اور آسار \عددی{\det(T)} تلاش کریں۔

(ب) قالب \عددی{T} کی امتیازی اقدار تلاش کریں۔ تصدیق کریں کہ انکا مجموعہ اور حاصل ضرب مساوات \عددی{A.85} کے معنوں میں جزو(الف) کے عین مطابق ہے۔ قالب \عددی{T} کو وتری روپ میں لکھیں۔

(ج) قالب \عددی{T} کےامتیازی سمتیات تلاش کریں۔ انحطاطی حلقہ میں دو خطی غیر طابع امتیازی سمتیات تیار کریں ہرمیشی قالب کے لئے یہ قدم ہر صورت ممکن ہوگا لیکن کسی بھی اختیاری قالب کے لئے لاظمی نہیں کہ ایسا ممکن ہو سوال \عددی{A.19} کے ساتھ موازنہ کریں۔ انہیں قائمہ بنائیں اور تصدیق کریں کہ تیسرے کے لحاظ سے دونوں قائمہ ہیں۔ تینوں امتیازی سمتیات کی معمولزنی کریں۔

(د) قالب \عددی{T} کو وتری بنانے والا اکہرا قالب \عددی{S} تیار کریں اور صریحاً دیکھائیں کہ متشابہت تبادلہ \عددی{S} کو استعمال کرتے ہوئے \عددی{T} کو موضوع وتری روپ میں گھٹاتا ہے۔
\انتہا{سوال}
\ابتدا{سوال}
اکہرا تبادلہ وہ ہے جس کے لئے \عددی{\hat{U}^\dagger\hat{U} = 1} ہو۔

(الف) دیکھائیں کہ کسی بھی سمتیات \عددی{|\alpha\rangle}، \عددی{|\beta\rangle} کے لئے \عددی{\langle\hat{U}\alpha|\hat{U}\beta\rangle = \langle\alpha|\beta\rangle} کے معنوں میں  اکہرا تبادلہ اندرونی حاصل ضرب برقرار رکھتے ہیں۔

(ب) دیکھائیں کہ اکہرا تبادلہ کا امتیازی اقدار کا معیار \عددی{1} ہے۔

(ج) دیکھائیں کہ منفرد امتیازیی اقدار سے متعلق اکہرا قالب کی امتیازی سمتیات قائمہ ہوںگے۔
\انتہا{سوال}
\ابتدا{سوال}
قوالب کے تفاعلات ٹیلر تصلصل توسیعات دیتے ہیں مثلاً 
\begin{align}
	e^M\equiv I + M +\frac{1}{2}M^2 + \frac{1}{3!}M^3 + \dots
\end{align}
(الف) درجہ ذیل کے لئے \عددی{\exp(M)} تلاش کریں
\begin{align*}
	(i) M=
	\begin{pmatrix}
		0 & 1 & 3\\
		0 & 0 & 4\\
		0 & 0 & 0
	\end{pmatrix}
		; (ii) M=
	\begin{pmatrix}
		0 & \theta\\
		-\theta & 0
	\end{pmatrix}
\end{align*}
(ب) اگر \عددی{M} وتری بنانے کے قابل ہو تب درجہ ذیل دیکھائیں
\begin{align}
	\det\left(e^M\right) = e^{Tr(M)}
\end{align}
تبصرہ: اگر \عددی{M} وتری بنانے کے قابل نہ ہو تب بھی یہ درست ہوگا تاہم ایسی عمومی صورت کے لئے اسکو ثابت کرنا مشکل ہے۔

(ج) دیکھائیں اگر قوالب \عددی{M} اور \عددی{N} مقلوبی ہوں تب درجہ ذیل ہوگا
\begin{align}
	e^{M+N} = e^Me^N
\end{align}
ثابت کریں کہ غیر مقلوبی قالب کے لئے مساوات \عددی{A.93} درست نہیں سادہ ترین متضاد مثال دیکر ایسا کریں۔

(د) اگر \عددی{H} ہرمیشی ہوں تب دیکھائیں کہ\عددی{e^{iH}} اکہرا ہوگا۔
\انتہا{سوال}
