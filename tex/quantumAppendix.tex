\addcontentsline{toc}{chapter}{ضمیمہ}
\باب{خطی الجبرا}\شناخت{ضمیمہ_خطی_الجبرا}
کالج  کی سطح پر پڑھائے جانے والے سادہ سمتیات کے حساب کو خطی الجبرا تصوراتی جامع    پہناتا اور  عمومیت دیتا ہے۔ عمومیت دو رخوں میں دی  جاتی ہے:  \عددی{(1)} ہم غیر سمتیات کو مخلوط اعداد ہونے کی اجازت دیتے ہیں،  اور \عددی{(2)} ہم اپنے آپ کو تین ابعاد میں  رہنے کا پابند نہیں رکھتے۔

\حصہ{سمتیات}\شناخت{ضمیمہ_سمتیات}
\اصطلاح{ سمتیات} \عددی{|\alpha\rangle}، \عددی{|\beta\rangle}، \عددی{|\gamma\rangle}، \نقطے   کے سلسلہ   اور 
    \اصطلاح{غیر سمتیات}   (\عددی{a}، \عددی{b}، \عددی{c}، \نقطے) \حاشیہد{ہمارے مقصد کے لئے غیر سمتیات سادہ مخلوط اعداد ہوں گے۔  ریاضی دان آپ کو    زیادہ پراسرار  میدانوں  پر سمتی فضاوں کے بارے میں بتا سکتے ہیں، تاہم ان کا کوانٹائی میکانیات میں کوئی کردار نہیں پایا جاتا۔ یاد رہے کہ \عددی{\alpha}، \عددی{\beta}، \عددی{\gamma}، \نقطے (عموماً) اعداد نہیں ہوں گے؛ یہ نام ہوں گے، مثلاً  \قول{جمشید}،
      یا  \قول{در-73ءم}، یا،  زیر غور سمتیہ کو جو بھی آپ پکارنا  چاہیں۔}  کے سلسلہ   پر  \اصطلاح{سمتی فضا}\حاشیہب{vector space}\فرہنگ{سمتی فضا}\فرہنگ{vector space} مشتمل ہو گا   جو    سمتی جمع اور غیر سمتی ضرب کے زیر عمل\اصطلاح{ بند}\فرہنگ{بند}\حاشیہب{closed}\فرہنگ{closed}  ہو گا۔\حاشیہد{یعنی یہ اعمال پوری طرح معین ہیں،  اور کبھی بھی آپ کو سمتی فضا سے باہر منتقل نہیں کریں گے۔}


\begin{itemize}
	\item \موٹا{سمتی جمع}
\end{itemize}
کسی بھی دو سمتیات کا مجموعہ بھی  سمتیہ ہوگا۔
\begin{align}
	|\alpha\rangle+|\beta\rangle=|\gamma\rangle
\end{align}
سمتی مجموعہ \اصطلاح{استبدالی}\فرہنگ{استبدالی}\حاشیہب{commutative}\فرہنگ{commutative}:  
\begin{align}
	|\alpha\rangle+|\beta\rangle = |\beta\rangle+|\alpha\rangle
\end{align}
اور  \اصطلاح{تلازمی}\فرہنگ{تلازمی}\حاشیہب{associative}\فرہنگ{associative}:
\begin{align}
	|\alpha\rangle+(|\beta\rangle+|\gamma\rangle)=(|\alpha\rangle+|\beta\rangle)+|\gamma\rangle
\end{align}
ہے۔ ایک \اصطلاح{ معدوم}\فرہنگ{معدوم}\حاشیہب{null}\فرہنگ{null} (یا \اصطلاح{صفر}\فرہنگ{صفر}\حاشیہب{zero}\فرہنگ{zero})  سمتیہ    \عددی{|0\rangle} پایا جاتا ہے\حاشیہد{جہاں غلط فہمی کا امکان نہ ہو، وہاں روایتی طور پر معدوم سمتیہ کو  سادہ صفر لکھا جاتا ہے:\عددی{|0\rangle \to 0}} جو ہر سمتیہ \عددی{|\alpha\rangle} کے لئے درجہ ذیل خاصیت رکھتا ہے
\begin{align}
	|\alpha\rangle+|0\rangle=|\alpha\rangle
\end{align}
اور ہر سمتیہ \عددی{|\alpha\rangle} کا شریک \اصطلاح{مخالف سمتیہ}\فرہنگ{مخالف سمتیہ}\حاشیہب{inverse vector}\فرہنگ{inverse vector} 
  \عددی{(|-\alpha\rangle)}\حاشیہد{یہ ایک  انوکھی  علامت ہے چونکہ \عددی{\alpha}  عدد نہیں۔ میں ایک سمتیہ جس کا نام  \قول{جمشید}  ہے کے مخالف سمتیہ کو \قول{جمشید-}   کا نام دے رہا ہوں۔ کچھ ہی دیر میں ہم بہتر   اصطلاح  دیکھ پائیں گے۔}  پایا جاتا ہے جو درجہ ذیل دیتا ہے۔
\begin{align}
	|\alpha\rangle+|-\alpha\rangle = |0\rangle
\end{align}

\begin{itemize}
	\item \موٹا{غیر سمتی ضرب}
\end{itemize}
کسی بھی غیر سمتیہ اور سمتیہ کا حاصل ضرب:
\begin{align}
	a|\alpha\rangle=|\gamma\rangle
\end{align}
 ایک سمتیہ ہوگا۔ غیر سمتی ضرب سمتی مجموعہ کے لحاظ سے \اصطلاح{ جزئیتی تقسیمی}\فرہنگ{جزئیتی تقسیمی}\حاشیہب{distributive}\فرہنگ{distributive}  
\begin{align}
	a(|\alpha\rangle+|\beta\rangle)=a|\alpha\rangle+a|\beta\rangle
\end{align}
اور  غیر سمتی جمع کے لحاظ سے بھی جزئیتی تقسیمی ہے۔
\begin{align}
	(a+b)|\alpha\rangle=a|\alpha\rangle+b|\alpha\rangle
\end{align}
یہ غیر سمتیات کے سادہ ضرب کے لحاظ سے \اصطلاح{تلازمی} بھی ہے۔
\begin{align}
	a(b|\alpha\rangle)=(ab)|\alpha\rangle
\end{align}
غیر سمتیات \عددی{0} اور \عددی{1} کے ساتھ  ضرب آپ کی  توقع کے مطابق نتائج دیں گے۔
\begin{align}
1|\alpha\rangle=|\alpha\rangle; \quad  0|\alpha\rangle=|0\rangle
\end{align}
ظاہر ہے \عددی{|-\alpha\rangle=(-1)|\alpha\rangle} ہوگا جس کو ہم \عددی{-|\alpha\rangle}   لکھتے ہیں۔

یہاں   جتنا  نظر آ رہا ہے،    حقیقتاً  اتنا ہے نہیں؛   پس میں نے سمتیات کی جوڑ توڑ کے عام فہم قواعد کو تصوراتی روپ  میں پیش کیا ہے۔ نتیجتاً دیگر نظام جو یہی باضابطہ خواص رکھتے ہوں پر ہم سادہ سمتیات کے رویّہ کے بارے میں  معلوم علم  اور وجدان بروئے کار  لا سکیں گے۔

سمتیات \عددی{|\alpha\rangle}، \عددی{|\beta\rangle}، \عددی{|\gamma\rangle}، \نقطے  
 کا   \اصطلاح{خطی  مجموعہ}\فرہنگ{خطی مجموعہ}\حاشیہب{linear combination}\فرہنگ{linear combination}   درجہ ذیل روپ کا  فقرہ ہوگا۔
\begin{align}
	a|\alpha\rangle+b|\beta\rangle+c|\gamma\rangle+\cdots
\end{align}
ایک سمتیہ \عددی{|\lambda\rangle} جس کو سلسلہ  \عددی{|\alpha\rangle}، \عددی{|\beta\rangle}، \عددی{|\gamma\rangle}، \نقطے کا      خطی مجموعہ لکھنا ممکن نہ ہو  \اصطلاح{خطی غیر تابع}\فرہنگ{خطی غیر تابع}\حاشیہب{linearly independent}\فرہنگ{linearly independent}  کہلاتا ہے۔ (مثلاً،  تین ابعاد میں اکائی سمتیہ \عددی{\hat{k}} سمتیات \عددی{\hat{i}} اور \عددی{\hat{j}} کا خطی غیر تابع ہے،  جبکہ \عددی{xy} مستوی میں ہر  سمتیہ \عددی{\hat{i}} اور \عددی{\hat{j}} کا خطی  \ترچھا{تابع} ہوگا۔)  اسی کی توسّط   سے، سمتیات کا  وہ  ذخیرہ جس میں ہر ایک سمتیہ  باقی تمام سمتیات  کا خطی غیر تابع ہو  "خطی غیر تابع"  کہلاتا ہے۔ جب ہر سمتیہ کو  سمتیات کے ایک ذخیرہ کے ارکان کا خطی مجموعہ لکھنا ممکن ہو، ہم کہتے ہیں کہ سمتیات کا یہ ذخیرہ  فضا کا  \اصطلاح{احاطہ}\فرہنگ{احاطہ}\حاشیہب{span}\فرہنگ{span}  کرتے\حاشیہد{فضا کا احاطہ کرنے والے سمتیات  کا سلسلہ   \اصطلاح{مکمل}\فرہنگ{مکمل}\فرہنگ{complete} (\تحریر{complete})  بھی  کہلاتا  ہے، اگرچہ میں اس  اصطلاح کو لامتناہی بُعد  کی صورت کے لئے رکھتا ہوں جہاں  ارتکاز پر سوالات اٹھائے  جا  سکتے ہیں۔}  ہیں۔ فضا کا احاطہ کرنے والے   \ترچھا{خطی غیر تابع} سمتیات کا سلسلہ  \اصطلاح{اساس}\فرہنگ{اساس}\حاشیہب{basis}\فرہنگ{basis} کہلاتا ہے۔ اساس میں سمتیات کی تعداد  فضا کا  \اصطلاح{بُعد}\فرہنگ{بُعد}\حاشیہب{dimension}\فرہنگ{dimension}  کہلاتا ہے۔ فی الحال ہم فرض کرتے ہیں کہ بُعد \عددی{(n)}  \ترچھا{متناہی} ہے۔

دیے گئے اساس 
\begin{align}
	|e_1\rangle, |e_2\rangle, \dots, |e_n\rangle
\end{align}
کے لحاظ سے کسی  بھی سمتیہ
\begin{align}
	|\alpha\rangle=a_1|e_1\rangle+a_2|e_2\rangle+\dots+a_n|e_n\rangle
\end{align}
کو اس اساس  کے \اصطلاح{ ارکان}  کی  (مرتب)  \عددی{n} اجزائی سلسلہ 
\begin{align}
	|\alpha\rangle\leftrightarrow(a_1, a_2, \dots, a_n)
\end{align}
سے یکتا طور پر ظاہر کیا جا سکتا ہے۔ عموماً  سمتیات کی بجائے  ان  اجزاء کے ساتھ کام کرنا زیادہ آسان  ہوتا ہے۔  سمتیات  جمع کرنے کے لئے  ان کے مطابقتی اجزاء آپس میں جمع کئے جاتے ہیں:
\begin{align}
	|\alpha\rangle+|\beta\rangle\leftrightarrow(a_1+b_1, a_2+b_2, \dots, a_n+b_n)
\end{align}
غیر سمتیہ سے ضرب  کے لئے ہر جزو کو اس غیر سمتیہ  سے  ضرب کریں:
\begin{align}
	c|\alpha\rangle\leftrightarrow(ca_1, ca_2, \dots, ca_n)
\end{align}
معدوم سمتیہ کو صفروں کی ایک کھڑی ظاہر کرتی ہے:
\begin{align}
	|0\rangle\leftrightarrow(0, 0, \dots, 0)
\end{align}
اور مخالف سمتیہ کے ارکان کی علماتیں  الٹ کی جاتی ہیں۔
\begin{align}
	|-\alpha\rangle\leftrightarrow(-a_1, -a_2, \dots, -a_n)
\end{align}
ارکان کے ساتھ کام کرنے کی واحد  قباحت  یہ ہے کہ  آپ کو کسی ایک مخصوص اساس کے ساتھ کام کرنا ہوگا،  اور یہی  حسابی عمل کسی دوسری  اساس میں بالکل مختلف نظر آئے گا۔

\ابتدا{سوال}
مخلوط اجزاء والے تین ابعادی سادہ سمتیات \عددی{(a_x\hat{i}+a_y\hat{j}+a_z\hat{k})} پر غور کریں۔
\begin{enumerate}[a]
\item
 کیا وہ ذیلی سلسلہ جس میں تمام سمتیات کے لئے \عددی{a_z=0} ہو سمتی فضا قائم کرتے ہیں؟ اگر کرتا ہو تب اس کا  بُعد کیا ہوگا؛  اگر  نہیں کرتا  تو کیوں نہیں  کرتا؟
\item
 اس  ذیلی سلسلہ کے بارے میں آپ کیا کہیں گے جن کا \عددی{z} جزو \عددی{1} کے برابر ہو؟ \ترچھا{ اشارہ}: کیا ایسے دو سمتیات کا مجموعہ اسی ذیلی سلسلہ میں پایا جائے گا؟  معدوم سمتیہ کے بارے میں سوچیں؟
\item
 ان سمتیات کے ذیلی سلسلہ کے بارے میں آپ کیا کہہ سکتے ہیں جن کے تمام ارکان برابر ہوں؟
 \end{enumerate}
\انتہا{سوال}
\ابتدا{سوال}
ان تمام کثیر رکنیوں، (جن کے عددی سر مخلوط ہوں اور)  جن کا  \عددی{x} میں درجہ  \عددی{N} سے کم ہو کے ذخیرہ پر غور کریں۔
\begin{enumerate}[a]
\item
کیا یہ سلسلہ سمتی فضا قائم کرتا ہے  (جہاں کثیر رکنیاں بطور  \قول{سمتیات}  ہوں)؟ اگر فضا قائم کرتا ہو تو  مناسب اساس تجویز کریں اور اس فضا کا بُعد بتائیں۔  اگر فضا قائم نہ کرتا ہو تو   تعریفی خصوصیات میں سے کونسی اس میں نہیں پائی جاتی (جاتیں)؟
\item
 اگر ہم چاہیں کہ تمام کثیر رکنیاں جفت تفاعلات ہوں تب کیا ہوگا؟
\item
 اگر ہم چاہیں کہ پہلا عددی سر ( جو \عددی{x^{N-1}} کو ضرب کرتا ہے)  \عددی{1} ہو تب کیا ہوگا؟
\item
 اگر ہم چاہیں کہ \عددی{x=1} پر کثیر رکنیوں کی قیمت \عددی{0} ہو تب کیا ہوگا؟
\item
 اگر ہم چاہیں کہ \عددی{x=0} پر کثیر رکنیوں کی قیمت \عددی{1} ہو تب کیا ہوگا؟ 	
 \end{enumerate}
\انتہا{سوال}
\ابتدا{سوال}
ثابت کریں  کہ کسی بھی ایک اساس کے لحاظ سے  سمتیہ کے ارکان \ترچھا{ یکتا}   ہوں گے۔
\انتہا{سوال}

\حصہ{اندرونی ضرب}\شناخت{ضمیمہ_اندرونی_ضرب}
تین ابعاد میں دو اقسام کے سمتی ضرب  پائے جاتے ہیں:  نقطی ضرب اور صلیبی ضرب۔ موخر الذکر کی قدرتی توسیع کسی طرح بھی \عددی{n} ابعاد سمتی فضاوں میں نہیں کی جا سکتی،   جبکہ  اول الذکر کی کی جا سکتی ہے؛  اور  اس سیاق و سباق میں اسے عموماً \اصطلاح{ اندرونی ضرب}\فرہنگ{اندرونی!ضرب}\حاشیہب{inner product}\فرہنگ{product!inner} پکارا  جاتا ہے۔ دو سمتیات (\عددی{|\alpha\rangle} اور \عددی{|\beta\rangle}) کا اندرونی ضرب ایک مخلوط عدد ہوگا جسے \عددی{\langle\alpha|\beta\rangle} لکھا جاتا ہے اور جس کے  خواص درج ذیل  ہیں۔
\begin{align}
	\langle\beta|\alpha\rangle &= \langle\alpha|\beta\rangle^*\\
	\langle\alpha|\alpha\rangle\geq0,\quad\text{\RL{اور}}\quad \langle\alpha|\alpha\rangle &= 0\leftrightarrow|\alpha\rangle = |0\rangle \label{مساوات_خطی_معیار_مثبت}\\
	\langle\alpha|(b|\beta\rangle+c|\gamma\rangle) &=  b\langle\alpha|\beta\rangle+c\langle\alpha|\gamma\rangle
\end{align}
مخلوط اعداد تک عمومیت کے علاوہ یہ مسلمات نقطی  ضرب کے جانے پہچانے رویّوں کو ریاضی کی زبان میں پیش کرتے ہیں۔ ایسی سمتی فضا جس میں اندرونی  ضرب بھی شامل ہو  \اصطلاح{اندرونی ضرب فضا}\فرہنگ{اندرونی!ضرب فضا}\حاشیہب{inner product space}\فرہنگ{inner!product space} کہلاتی ہے۔

چونکہ  سمتیہ کا اپنے ساتھ اندرونی ضرب غیر منفی عدد  ہے  (مساوات \حوالہ{مساوات_خطی_معیار_مثبت})  لہٰذا اس کا جذر حقیقی ہوگا؛ جو سمتیہ کا  \اصطلاح{معیار}\فرہنگ{معیار}\حاشیہب{norm}\فرہنگ{norm} کہلاتا ہے:
\begin{align}
	\|\alpha\|&\equiv\sqrt{\langle\alpha|\alpha\rangle}&  \text{\small{معیار}} 
\end{align}
اور جو  \قول{لمبائی} کے تصور کو وسعت دیتا ہے۔\اصطلاح{ اکائی سمتیہ}\فرہنگ{اکائی!سمتیہ}\حاشیہب{unit vector}\فرہنگ{unit!vector}   (جس کا معیار \عددی{1}  ہو گا)     \اصطلاح{معمول شدہ}\فرہنگ{معمول شدہ}\حاشیہب{normalized}\فرہنگ{normalized} کہلاتا ہے۔ دو سمتیات جن کا اندرونی ضرب صفر ہو   \اصطلاح{قائمہ}\فرہنگ{قائمہ}\حاشیہب{orthogonal}\فرہنگ{orthogonal}  کہلاتے ہیں  (جو \قول{سیدھا کھڑا}  ہونے کے تصور کو عمومیت دیتا ہے)۔ باہمی قائمہ معمول شدہ سمتیات:
\begin{align}
	\langle\alpha_i|\alpha_j\rangle=\delta_{ij}
\end{align}
کے ذخیرہ کو \اصطلاح{ معیاری عمودی سلسلہ}\فرہنگ{معیاری عمودی سلسلہ}\حاشیہب{orthonormal set}\فرہنگ{orthonormal set} کہتے ہیں۔ \ترچھا{معیاری عمودی اساس}  ہر صورت منتخب کیا جا سکتا ہے (سوال \حوالہ{سوال_ضمیمہ_گرام_شمد_ترکیب} دیکھیں) اور ایسا کرنا عموماً  بہتر بھی ثابت ہوتا ہے۔ ایسی صورت میں دو سمتیات کے  اندرونی ضرب  کو انکے اجزاء کے روپ  میں نہایت  خوبصورتی سے لکھا جا سکتا ہے:
\begin{align}\label{مساوات_ضمیمہ_قالبی_ضرب}
	\langle\alpha|\beta\rangle=a_1^*b_1+a_2^*b_2+\dots+a_n^*b_n
\end{align}
لہٰذا    معیار کا مربع
\begin{align}
	\langle\alpha|\alpha\rangle = \abs{a_1}^2+\abs{a_2}^2+\dots+\abs{a_n}^2
\end{align}
ہوگا  جبکہ اجزاء ازخود درجہ ذیل ہونگے۔
\begin{align}
	a_i=\langle e_i|\alpha\rangle
\end{align}
(یہ نتائج تین ابعادی معیاری عمودی اساس \عددی{\hat{i}}، \عددی{\hat{j}}، \عددی{\hat{k}} کے مشہور کلیات
 \عددی{\kvec{a}\cdot\kvec{b}=a_xb_x+a_yb_y+a_zb_z}، \عددی{\abs{\kvec{a}}^2=a_x^2+a_y^2+a_z^2} اور \عددی{a_x=\hat{i}\cdot\kvec{a}}، \عددی{a_y=\hat{j}\cdot \kvec{a}}، \عددی{a_z=\hat{k}\cdot\kvec{a}} کو عمومیت دیتے ہیں۔)  یہاں سے آگے ہم صرف معیاری عمودی اساس استعمال کریں گے،   ما سوائے جب صریحاً ایسا نہ کرنے کا کہا  گیا ہو۔

دو  سمتیات کے بیچ  \ترچھا{زاویہ}  ایسی    ہندسی مقدار  ہے جس کو ہم عمومیت دینا چاہیں گے۔ سادہ سمتی تجزیہ میں 
\عددی{\cos\theta=(\kvec{a}\cdot\kvec{b})/\abs{\kvec{a}}\abs{\kvec{b}}}  ہے۔تاہم    اندرونی ضرب  عموماً مخلوط عدد ہو گا،    لہٰذا  (  اختیاری اندرونی ضرب فضا میں) مماثل کلیہ  (حقیقی)  زاویہ \عددی{\theta} نہیں دیگا۔ بہرحال،  اس مقدار کی  \ترچھا{مطلق قیمت}  ایسا  عدد ہوگا جو \عددی{1} سے تجاوز نہیں کرتا۔  
\begin{align}\label{مساوات_ضمیمہ_شوارز_عدم_مساوات}
	\abs{\langle\alpha|\beta\rangle}^2\leq\langle\alpha|\alpha\rangle\langle\beta|\beta\rangle
\end{align}
(اس اہم نتیجہ کو  \اصطلاح{شوارز عدم مساوات}\فرہنگ{شوارز عدم مساوات}\حاشیہب{Schwarz inequality}\فرہنگ{Schwarz inequality} کہتے ہیں؛  جس کا ثبوت سوال \حوالہ{ضمیمہ_اندرونی_ضرب_شوارز_عدم_مساوات_ثبوت} میں پیش کیا گیا ہے۔)  یوں،  آپ چاہیں تو،  \عددی{|\alpha\rangle} اور \عددی{|\beta\rangle} کے بیچ زاویہ کی  تعریف درج ذیل  لی جا سکتی ہے۔
\begin{align}\label{مساوات_ضمیمہ_زاویہ}
	\cos\theta = \sqrt{\frac{\langle\alpha|\beta\rangle\langle\beta|\alpha\rangle}{\langle\alpha|\alpha\rangle\langle\beta|\beta\rangle}}
\end{align}
\ابتدا{سوال}\شناخت{سوال_ضمیمہ_گرام_شمد_ترکیب}
فرض کریں آپ اساس (\عددی{|e_1\rangle}،\عددی{|e_2\rangle}،\عددی{\cdots}،\عددی{|e_n\rangle})  سے آغاز کرتے ہیں جو معیاری عمودی \ترچھا{ نہیں} ہے۔ اس اساس سے، \اصطلاح{گرام و شمد حکمت عملی}\فرہنگ{گرام و شمد حکمت عملی}\حاشیہب{Gram-Schmidt procedure}\فرہنگ{Gram-Schmidt procedure}   کے ذریعہ ( جو ایک منظم ترکیب ہے)      معیاری عمودی اساس (\عددی{|e'_1\rangle}،\عددی{|e'_2\rangle}،\عددی{\cdots}،\عددی{|e'_n\rangle})    حاصل کی جا سکتی ہے۔ یہ ترکیب  کچھ یوں ہے:
\begin{enumerate}[a]
\item
  اساس کے پہلے سمتیہ \عددی{|e_1\rangle}  کو  ( اس کے معیار سے تقسیم کر کے)   معمول شدہ  بنائیں۔
\begin{align*}
		|e_1'\rangle = \frac{|e_1\rangle}{\|e_1\|}
\end{align*}
\item
 پہلے سمتیہ پر دوسرے سمتیہ کا تظلیل  دریافت کر کے اس تظلیل کو دوسرے سمتیہ سے منفی کریں۔
\begin{align*}
	|e_2\rangle-\langle e_1'|e_2\rangle|e_1'\rangle
\end{align*}
( سمتیہ \عددی{|e_2\rangle} کا \عددی{|e_1'\rangle} کے رخ غیر سمتیہ تظلیل \عددی{\langle e_1'|e_2\rangle}  ہے جس کے دائیں جانب اکائی سمتیہ  \عددی{|e_1'\rangle} چسپاں کرنے سے سمتیہ تظلیل حاصل  کیا گیا۔) درج بالا  سمتیہ \عددی{|e'_1\rangle} کا قائمہ ہوگا؛ اس کو معمول شدہ  کر کے \عددی{|e_2'\rangle} حاصل کریں۔
\item
سمتیہ \عددی{|e_3\rangle}  کی \عددی{|e'_1\rangle} پر تظلیل  اور \عددی{|e'_2\rangle} پر تظلیل کو \عددی{|e_3\rangle} سے منفی کریں۔
\begin{align*}
|e_3\rangle-\langle e_1'|e_3\rangle|e_1'\rangle-\langle e_2'|e_3\rangle|e_2'\rangle
\end{align*}
یہ \عددی{|e_1'\rangle} اور \عددی{|e_2'\rangle} کو قائمہ ہو گا؛ اس کو معمول شدہ کر کے \عددی{|e_3'\rangle} حاصل کریں؛ وغیرہ، وغیرہ۔
\end{enumerate}

گرام و شمد حکمت عملی استعمال کرتے ہوئے درج ذیل   تین ابعاد فضائی  اساس کو  معیاری عمود شدہ  کریں۔
\begin{align*}
|e_1\rangle=(1+i)\hat{i}+(1)\hat(j)+(i)\hat{k}, |e_2\rangle=(i)\hat{i}+(3)\hat{j}+(1)\hat{k}, |e_3\rangle=(0)\hat{i}+(28)\hat{j}+(0)\hat{k}
\end{align*}
\انتہا{سوال}
\ابتدا{سوال}\شناخت{ضمیمہ_اندرونی_ضرب_شوارز_عدم_مساوات_ثبوت}
شوارز عدم مساوات (مساوات \حوالہ{مساوات_ضمیمہ_شوارز_عدم_مساوات})   ثابت کریں۔ \ترچھا{اشارہ}: \عددی{|\gamma\rangle=|\beta\rangle-(\langle\alpha|\beta\rangle/\langle\alpha|\alpha\rangle)|\alpha\rangle} لیں  اور \عددی{\langle\gamma|\gamma\rangle\geq0} استعمال کریں۔
\انتہا{سوال}
\ابتدا{سوال}
سمتیات \عددی{|\alpha\rangle=(1+i)\hat{i}+(1)\hat{j}+(i)\hat{k}} اور \عددی{|\beta\rangle=(4-i)\hat{i}+(0)\hat{j}+(2-2i)\hat{k}} کے بیچ (مساوات \حوالہ{مساوات_ضمیمہ_زاویہ} کی معنوں میں)  زاویہ تلاش کریں۔
\انتہا{سوال}
\ابتدا{سوال}
تکونی عدم مساوات \عددی{\|(|\alpha\rangle+|\beta\rangle)\|\leq\|\alpha\|+\|\beta\|} ثابت کریں۔
\انتہا{سوال}

\حصہ{قوالب}\شناخت{ضمیمہ_قالب}
فرض کریں آپ  (تین بُعدی  فضا میں)  ہر سمتیہ کو \عددی{17} سے ضرب دیں،  یا ہر سمتیہ کو \عددی{z} محور کے گرد \عددی{39\degree} گھمائیں،  یا \عددی{xy} مستوی میں ہر سمتیہ کا عکس لیں؛  یہ تمام  \اصطلاح{خطی تبادلہ}\فرہنگ{خطی تبادلہ}\حاشیہب{linear transformation}\فرہنگ{transformation!linear} کی مثالیں ہیں۔ خطی مبدل \عددی{\hat{T}}\حاشیہد{اس باب میں خطی تبادلہ کو ٹوپی کی علامت  (\عددی{\hat{}}) سے ظاہر کیا جائے گا؛ جیسا ہم دیکھیں گے، کوانٹائی  عامل بھی خطی مبدل ہیں اور ان  کو بھی ٹوپی کی نشان سے ظاہر کیا جائے گا۔}   سمتی فضا میں ہر ایک سمتیہ کا کسی دوسرے سمتیہ \عددی{(|\alpha\rangle\rightarrow|\alpha'\rangle=\hat{T}|\alpha\rangle)} میں تبادلہ کرتا ہے  جہاں کسی بھی سمتیات  \عددی{|\alpha\rangle}، \عددی{|\beta\rangle} اور کسی بھی  غیر سمتیات \عددی{a}، \عددی{b}   کے لئے اس عمل کا  خطی ہونا:
\begin{align}
	\hat{T}(a|\alpha\rangle+b|\beta\rangle)=a(\hat{T}|\alpha\rangle)+b(\hat{T}|\beta\rangle)
\end{align}
 لازمی شرط  ہے۔

یہ جانتے ہوئے کہ  \ترچھا{اساسی} سمتیات کے سلسلہ کے ساتھ خطی مبدل کیا کرتا ہے،  آپ با آسانی جان   سکتے ہیں کہ وہ  \ترچھا{کسی بھی} سمتیہ کے ساتھ کیا کرے گا۔ مثلاً،   اگر  \عددی{|e_1\rangle}، \عددی{|e_2\rangle}،\عددی{\cdots}،\عددی{|e_n\rangle}  اساس قائم کرتے ہوں  اور خطی مبدل \عددی{\hat{T}} اساسی سمتیہ \عددی{|e_1\rangle} پر عمل (\عددی{\hat{T}|e_1\rangle}) کر کے  ایک  نیا سمتیہ  پیدا کرتا ہے؛ ظاہر ہے، کسی بھی سمتیہ کی طرح،  اس نئے سمتیہ کو بھی اس اساس میں لکھا جا سکتا ہے لہٰذا 
\عددی{\hat{T}|e_1\rangle=T_{11}|e_1\rangle+T_{21}|e_2\rangle+\cdots+T_{n1}|e_n\rangle} لکھا جا سکتا ہے جہاں \عددی{T_{11}}، \عددی{T_{21}}، \عددی{\cdots}، \عددی{T_{n1}} عددی سر ہیں۔ اسی طرح باقی اساسی سمتیات کے لئے ایسا کیا جا سکتا ہے:
\begin{align*}
	\hat{T}|e_1\rangle &= T_{11}|e_1\rangle+T_{21}|e_2\rangle+\dots+T_{n1}|e_n\rangle\\
	\hat{T}|e_2\rangle &= T_{12}|e_1\rangle+T_{22}|e_2\rangle+\dots+T_{n2}|e_n\rangle\\
	&\vdots\\
	\hat{T}|e_n\rangle &= T_{1n}|e_1\rangle+T_{2n}|e_2\rangle+\dots+T_{nn}|e_n\rangle
\end{align*}
جس کو مختصراً درج ذیل لکھتے ہیں۔
\begin{align}\label{مساوات_ضمیمہ_خطی_تبادلہ}
	\hat{T}|e_{j}\rangle=\sum_{i=1}^{n}T_{ij}|e_i\rangle,\quad(j=1, 2, \dots, n)
\end{align}
اگر \عددی{|\alpha\rangle} ایک اختیاری سمتیہ ہو  (جس کو  ہم ان اساسی سمتیات میں لکھتے ہیں) :
\begin{align}\label{مساوات_ضمیمہ_اختیاری_سمتیہ}
	|\alpha\rangle=a_1|e_1\rangle+a_2|e_2\rangle+a_3|e_3\rangle+\cdots+a_n|e_n\rangle=\sum_{j=1}^{n}a_j|e_j\rangle
\end{align}
  تب
\begin{align*}
\hat{T}|\alpha\rangle= a_1\hat{T}|e_1\rangle+ a_2\hat{T}|e_2\rangle+ a_3\hat{T}|e_3\rangle+\cdots+ a_n\hat{T}|e_n\rangle
\end{align*}
ہو گا  جس میں \عددی{\hat{T}|e_1\rangle = T_{11}|e_1\rangle+T_{21}|e_2\rangle+\dots+T_{n1}|e_n\rangle} وغیرہ پُر کر کے
\begin{align*}
\hat{T}|\alpha\rangle=\phantom{+}& a_1(T_{11}|e_1\rangle+T_{21}|e_2\rangle+T_{31}|e_3\rangle+\dots+T_{n1}|e_n\rangle)\\
+&a_2(T_{12}|e_1\rangle+T_{22}|e_2\rangle+T_{32}|e_3\rangle+\dots+T_{n2}|e_n\rangle)\\
+&a_3(T_{13}|e_1\rangle+T_{23}|e_2\rangle+T_{33}|e_3\rangle+\dots+T_{n3}|e_n\rangle)\\
&\phantom{+}\vdots\\
+&a_n(T_{1n}|e_1\rangle+T_{2n}|e_2\rangle+T_{3n}|e_3\rangle+\dots+T_{nn}|e_n\rangle)
\end{align*}
ترتیب نو  کرتے ہوئے اکائی سمتیات کے عددی سر اکٹھے کر کے درج ذیل لکھا جا سکتا ہے۔
\begin{align*}
\hat{T}|\alpha\rangle=\phantom{+}&(a_1T_{11}+a_2T_{12}+a_3T_{13}+\cdots+a_nT_{1n})|e_1\rangle\\
+&(a_1T_{21}+a_2T_{22}+a_3T_{23}+\cdots+a_nT_{2n})|e_2\rangle\\
+&(a_1T_{31}+a_2T_{32}+a_3T_{33}+\cdots+a_nT_{3n})|e_3\rangle\\
&\phantom{+}\vdots\\
+&(a_1T_{n1}+a_2T_{n2}+a_3T_{n3}+\cdots+a_nT_{nn})|e_n\rangle
\end{align*}
اس مساوات میں  اساسی سمتیہ \عددی{|e_1\rangle} کے عددی سر   \عددی{(a_1T_{11}+a_2T_{12}+\cdots+a_nT_{1n})}  کو \عددی{\sum_{j=1}^n a_jT_{1j}} لکھا جا سکتا ہے، اور اسی طرح باقی اساسی سمتیات کے عددی سروں کے لئے بھی لکھا جا سکتا ہے۔یوں درج ذیل حاصل ہو گا۔
 \begin{align*}
\hat{T}|\alpha\rangle &=\sum_{j=1}^n a_j T_{1j}|e_1\rangle+\sum_{j=1}^n a_j T_{2j}|e_2\rangle+\cdots+\sum_{j=1}^n a_j T_{nj}|e_n\rangle\\
&=\sum_{i=1}^n  \sum_{j=1}^n a_j T_{ij}|e_i\rangle
\end{align*}
ہم مساوات \حوالہ{مساوات_ضمیمہ_اختیاری_سمتیہ} سے یہاں تک کے حساب کو مختصراً درج ذیل لکھ سکتے ہیں۔
\begin{align}\label{مساوات_ضمیمہ_خطی_تبادلہ_مختصر}
	\hat{T}|\alpha\rangle=\sum_{j=1}^{n}a_j\left(\hat{T}|e_j\rangle\right)=\sum_{j=1}^{n}\sum_{i=1}^{n}a_jT_{ij}|e_i\rangle=\sum_{i=1}^{n}\left(\sum_{j=1}^{n}T_{ij}a_j\right)|e_i\rangle
\end{align}
ظاہر ہے کہ \عددی{\hat{T}} ایک سمتیہ کو جس کے ارکان \عددی{a_1}، \عددی{a_2}،\عددی{\cdots}،\عددی{a_n}   ہوں کا تبادلہ  ایک نئے سمتیہ میں کرتا ہے جس کے ارکان درجہ ذیل ہونگے۔ 
\begin{align}\label{مساوات_ضمیمہ_تبادلہ_قائدہ}
	a'_i=\sum_{j=1}^{n}T_{ij}a_j
\end{align}
(مساوات \حوالہ{مساوات_ضمیمہ_خطی_تبادلہ} اور مساوات \حوالہ{مساوات_ضمیمہ_خطی_تبادلہ_مختصر}  میں اشاریہ آگے پیچھے کئے گئے ہیں۔ یہ لکھتے ہوئے غلطی نہیں کی گئی۔ دوسرے لفظوں میں  (  \عددی{i} اور \عددی{j}  آپس میں تبدیل کرنے \عددی{i\leftrightarrow j} سے مراد یہ ہے کہ) اگر اجزاء کا تبادلہ  \عددی{T_{ij}} سے  ہو، تب اساسی سمتیات کا تبادلہ \عددی{T_{ji}} سے  ہو گا۔  )

یوں جس طرح کسی اساس کے لحاظ سے \عددی{n} ارکان \عددی{a_i} سمتیہ یا \عددی{|\alpha\rangle} کو یکتا طور ظاہر کرتے ہیں اسی طرح \عددی{T_{ij}} کے \عددی{n^2} \اصطلاح{ ارکان}\فرہنگ{ارکان}\حاشیہب{elements}\فرہنگ{elements}  خطی مبدل \عددی{\hat{T}} کو اسی اساس کے لحاظ سے یکتا طور پر بیان کرتے ہیں۔
\begin{align}
	\hat{T}\leftrightarrow(T_{11}, T_{12}, \cdots, T_{nn})
\end{align}
اگر اساس معیاری عمودی ہو،   مساوات \حوالہ{مساوات_ضمیمہ_خطی_تبادلہ}  کے تحت درج ذیل ہو گا۔
\begin{align}
	T_{ij}=\langle e_i|\hat{T}|e_j\rangle
\end{align}
ان مخلوط اعداد کو \اصطلاح{قالب}\فرہنگ{قالب}\حاشیہب{matrix}\فرہنگ{matrix} کے روپ\حاشیہد{میں  \ترچھا{چوکور}  قوالب کو موٹی لکھائی میں  لاطینی   بڑے حروف ، مثلاً \عددی{\mat{T}}،  سے ظاہر  کروں گا۔}  میں لکھنا بہتر ثابت ہوتا ہے۔ 
\begin{align}
	\mat{T}=
	\begin{pmatrix}
		T_{11} & T_{12} & \dots & T_{1n}\\
		T_{21} & T_{22} & \dots & T_{2n}\\
		\vdots & \vdots & & \vdots\\
		T_{n1} & T_{n2} & \dots & T_{nn}
	\end{pmatrix}
\end{align}
یوں خطی مبدل کا مطالعہ  محض قوالب کے نظریہ کا مطالعہ ہوگا۔  دو خطی مبدل کے  \ترچھا{مجموعہ}  \عددی{(\hat{S}+\hat{T})} کی تعریف:
\begin{align}
	(\hat{S}+\hat{T})|\alpha\rangle=\hat{S}|\alpha\rangle+\hat{T}|\alpha\rangle
\end{align}
  ہماری توقع کے عین مطابق   قوالب جمع کرنے کے مترادف ہے  (جہاں آپ انکے مطابقتی ارکان جمع کرتے ہیں)۔
\begin{align}
	\mat{U} = \mat{S}+\mat{T} \Leftrightarrow U_{ij}=S_{ij}+T_{ij}
\end{align}
دو خطی تبادلہ  کا  \ترچھا{حاصل ضرب}  \عددی{(\hat{S}\hat{T})}،  پہلے \عددی{\hat{T}} اور اس کے بعد \عددی{\hat{S}} تبادلہ  کرنے کے مترادف ہے۔
\begin{align}
	|\alpha'\rangle=\hat{T}|\alpha\rangle;\quad|\alpha''\rangle=\hat{S}|\alpha'\rangle=\hat{S}(\hat{T}|\alpha\rangle)=\hat{S}\hat{T}|\alpha\rangle
\end{align}
مجموعی مبدل \عددی{\hat{U}=\hat{S}\hat{T}} کو کونسا قالب \عددی{\mat{U}} ظاہر کرتا ہے؟ اسے حاصل کرنا مشکل نہیں۔
\begin{align*}
	a''_i=\sum_{j=1}^{n}S_{ij}a'_j=\sum_{j=1}^{n}S_{ij}\left(\sum_{k=1}^{n}T_{jk}a_k\right)=\sum_{k=1}^{n}\left(\sum_{j=1}^{n}S_{ij}T_{jk}\right)a_k=\sum_{k=1}^{n}U_{ik}a_k
\end{align*}
ظاہر ہے کہ  درجہ ذیل ہوگا۔
\begin{align}
	\mat{U} = \mat{S}\mat{T}\Leftrightarrow U_{ik}=\sum_{j=1}^{n}S_{ij}T_{jk}
\end{align}
قوالب ضرب کرنے کا   یہ رائج طریقہ ہے؛   آپ \عددی{\mat{S}} کے \عددی{i}ویں صف  اور \عددی{\mat{T}} کے \عددی{k}ویں قطار کے مطابقتی اندراج  آپس میں ضرب کر کے   تمام کا مجموعہ لے کر حاصل ضرب  \عددی{\mat{ST}}  کا \عددی{ik} ویں  رکن تلاش کرتے  ہیں۔یہی طریقہ کار  بروئے کار لاتے  ہوئے \ترچھا{ مستطیل قوالب}   ضرب کیے جاتے ہیں،  بس اتنا ضروری ہے کہ پہلے قالب  میں قطاروں کی تعداد دوسرے قالب  میں صفوں کی تعداد کے برابر ہو۔ بالخصوص \عددی{|\alpha\rangle} کے ارکان کے  \عددی{n} اجزائی سلسلہ کو \عددی{n\times1}  \اصطلاح{قطار قالب}\فرہنگ{قطار قالب}\حاشیہب{column matrix}\فرہنگ{column matrix} (  یا \قول{قطار سمتیہ}):\حاشیہد{میں قطار قوالب اور صف قوالب کو موٹی لکھائی میں  لاطینی چھوٹے حروف، مثلاً \عددی{\mat{a}}، سے ظاہر کروں گا۔}
\begin{align}
	\mat{a} \equiv
	\begin{pmatrix}
		a_1\\a_2\\\vdots\\a_n
	\end{pmatrix}
\end{align}
لکھ کر قاعدہ  تبادلہ   (مساوات \حوالہ{مساوات_ضمیمہ_تبادلہ_قائدہ})  کو قالبی حاصل ضرب:
\begin{align}\label{مساوات_ضمیمہ_عمل_اظہار}
	\mat{a}'=\mat{T}\mat{a}
\end{align}
 لکھا جا سکتا ہے۔

آئیں اب قالبی اصطلاحات سیکھیں:
\begin{itemize}
\item
قالب کا  \اصطلاح{تبدیل  محل}\فرہنگ{تبدیل محل}\حاشیہب{transpose}\فرہنگ{transpose} (  جس کو ہم   لاطینی حرف پر  \قول{مد}   ڈال کر لکھتے ہیں:\عددی{\tilde{T}}) انہی ارکان پر مشتمل ہوگا،  تاہم اس میں صف اور قطار آپس میں جگہیں تبدیل کرتی ہیں۔ بالخصوص \ترچھا{قطار قالب}  کا تبدیل محل \اصطلاح{ صف قالب}\فرہنگ{صف قالب}\حاشیہب{row matrix}\فرہنگ{row matrix}  ہوگا۔
\begin{align}
	\tilde{\mat{a}}=
	\begin{pmatrix}
		a_1 & a_2 & \dots & a_n
	\end{pmatrix}
\end{align}
\ترچھا{چوکور قالب} کے ( بالائی بائیں سے زیریں دائیں)   \اصطلاح{مرکزی وتر}\فرہنگ{وتر!مرکزی}\حاشیہب{main diagonal}\فرہنگ{diagonal!main}  میں عکس اس کا تبدیل محل ہوگا ۔
\begin{align}
	\tilde{\mat{T}}=
	\begin{pmatrix}
		T_{11} & T_{21} & \dots & T_{n1}\\
		T_{12} & T_{22} & \dots & T_{n2}\\
		\vdots & \vdots & & \vdots\\
		T_{1n} & T_{2n} & \dots & T_{nn}
	\end{pmatrix}
\end{align}
ایسا  (چوکور)  قالب جو اپنے تبدیل محل کے برابر ہو  \اصطلاح{تشاکلی}\فرہنگ{تشاکلی}\حاشیہب{symmetric}\فرہنگ{symmetric}  کہلاتا ہے؛   اگر تبدیل محل کی علامت الٹ ہو تب یہ  \اصطلاح{خلاف تشاکلی}\فرہنگ{خلاف تشاکلی}\حاشیہب{antisymmetric}\فرہنگ{antisymmetric} ہوگا۔
\begin{align}
	\tilde{\mat{T}} &= \mat{T} \quad \text{\small\RL{تشاکلی}}& \tilde{\mat{T}} = -\mat{T} \quad \text{\small\RL{خلاف تشاکلی}}
\end{align}
\item
ہر رکن کا مخلوط جوڑی دار لینے سے قالب کا (مخلوط)  \اصطلاح{جوڑی دار}\فرہنگ{جوڑی دار}\حاشیہب{conjugate}\فرہنگ{conjugate}   (جس کو ہم ہمیشہ کی طرح ستارہ، \عددی{\mat{T}^*} سے ظاہر کرتے ہیں)  حاصل ہوگا۔
\begin{align}
	\mat{T}^*&=
	\begin{pmatrix}
		T_{11}^* & T_{12}^* & \dots & T_{1n}^*\\[0.25em]
		T_{21}^* & T_{22}^* & \dots & T_{2n}^*\\
		\vdots & \vdots & & \vdots\\
		T_{n1}^* & T_{n2}^* & \dots & T_{nn}^*
	\end{pmatrix}&
	\mat{a}^*=
	\begin{pmatrix}
		a_1^*\\[0.25em]
		a_2^*\\\vdots\\a_n^*
	\end{pmatrix}
\end{align}
تمام ارکان حقیقی ہونے کی صورت میں قالب \اصطلاح{ حقیقی}\فرہنگ{حقیقی}\حاشیہب{real}\فرہنگ{real} ہوگا،  جبکہ  تمام ارکان  خیالی ہونے کی صورت میں قالب \اصطلاح{ خیالی}\فرہنگ{خیالی}\حاشیہب{imaginary}\فرہنگ{imaginary} ہوگا۔
\begin{align}
	\mat{T}^*&=\mat{T}\quad \text{\small\RL{حقیقی}}& \mat{T}^*=-\mat{T}\quad\text{\small\RL{خیالی}}
\end{align}
\item
قالب کا \ترچھا{تبدیل محل و  جوڑی دار} اس کا   \اصطلاح{ہرمشی جوڑی دار}\فرہنگ{ہرمشی  جوڑی دار}\حاشیہب{hermitian conjugate}\فرہنگ{hermitian conjugate} ( یا  \اصطلاح{شریک}\فرہنگ{شریک}\حاشیہب{adjoint}\فرہنگ{adjoint}) ہو گا (  جسے خنجر کے نشان، \عددی{\mat{T}^\dagger}  سے ظاہر کیا جاتا ہے)۔
\begin{align}\label{مساوات_ضمیمہ_ہرمشی}
	\mat{T}^\dagger\equiv\tilde{\mat{T}}^*&=
	\begin{pmatrix}
		T_{11}^* & T_{21}^* & \dots & T_{n1}^*\\
		T_{12}^* & T_{22}^* & \dots & T_{n2}^*\\
		\vdots & \vdots & & \vdots\\
		T_{1n}^* & T_{2n}^* & \dots & T_{nn}^*
	\end{pmatrix};
	& \mat{a}^\dagger\equiv\tilde{\mat{a}}^*=
	\begin{pmatrix}
		a_1^* & a_2^* & \dots & a_n^*
	\end{pmatrix}
\end{align}
ایسا  چوکور قالب جو  اپنے ہرمشی جوڑی دار کے برابر ہو \اصطلاح{ ہرمشی }\فرہنگ{ہرمشی}\حاشیہب{hermitian}\فرہنگ{hermitian} (یا \اصطلاح{خود شریک}\فرہنگ{خود شریک}\حاشیہب{adjoint}\فرہنگ{adjoint}) قالب کہلاتا ہے؛ اگر ہرمشی جوڑی دار منفی علامت متعارف کرتا ہو قالب \اصطلاح{ منحرف ہرمشی}\فرہنگ{ہرمشی!منحرف}\حاشیہب{skew hermitian}\فرہنگ{hermitian!skew}   (یا \اصطلاح{خلاف ہرمشی}\فرہنگ{ہرمشی!خلاف}\حاشیہب{anti-hermitian}\فرہنگ{hermitian!anti}) ہوگا۔  
\begin{align}
	\mat{T}^\dagger&=\mat{T}\quad \text{\small\RL{ہرمشی}}&\mat{T}^\dagger=-\mat{T}\quad \text{\small\RL{منحرف ہرمشی}}
\end{align}
اس علامتیت میں دو سمتیات کے اندرونی ضرب کو  ( معیاری عمودی اساس کے لحاظ سے)  نہایت خوبصورتی کے ساتھ قالبی ضرب (مساوات \حوالہ{مساوات_ضمیمہ_قالبی_ضرب})    لکھا جا سکتا ہے۔
\begin{align}\label{مساوات_ضمیمہ_برقرار}
	\langle\alpha|\beta\rangle=\mat{a}^\dagger \mat{b}
\end{align}
\end{itemize}
یاد  رہے کہ درج بالا  رکوع میں متعارف تینوں اعمال (تبدیلی محل، جوڑی دار، ہرمشی جوڑی دار)  کا  دو مرتبہ اطلاق  سے واپس اصل قالب حاصل ہو گا۔

 عام طور پر قالبی ضرب غیر مقلبی \عددی{\mat{ST}\neq \mat{TS}} ہوگا؛  ضرب لکھنے کے دونوں طریقوں کے بیچ \ترچھا{ فرق} کو \اصطلاح{مقلب}\فرہنگ{مقلب}\حاشیہب{commutator}\فرہنگ{commutator} کہتے ہیں۔\حاشیہد{صرف چوکور قوالب کے لئے مقلب معنی خیز ہے۔ غیر چوکور قوالب میں دونوں ضرب کی جسامت بھی  ایک جیسی نہیں ہو گی۔}
\begin{align}
	\mat{[S}, \mat{T}]&\equiv \mat{ST}-\mat{TS}& \text{\small{مقلب}}
\end{align}
حاصل ضرب کا تبدیل محل \ترچھا{ الٹ ترتیب} میں تبدیل محلوں کا حاصل ضرب:
\begin{align}\label{مساوات_ضمیمہ_سوال_میں_حل_پہلا}
	(\widetilde{\mat{ST}})=\tilde{\mat{T}}\tilde{\mat{S}}
\end{align}
ہو گا (سوال \حوالہ{سوال_ضمیمہ_ثبوت} دیکھیں)، اور یہی کچھ ہرمشی جوڑی دار کے لئے بھی درست ہوگا۔
\begin{align}\label{مساوات_ضمیمہ_سوال_میں_حل_دوسرا}
	(\mat{ST})^\dagger=\mat{T}^\dagger \mat{S}^\dagger
\end{align}

\اصطلاح{اکائی قالب}\فرہنگ{قالب!اکائی}\حاشیہب{unit matrix}\فرہنگ{matrix!unit}     کے مرکزی وتر پر ارکان کی قیمت   ایک اور باقیوں  کی قیمت   صفر ہو گی ۔
\begin{align}
	\mat{I}\equiv
	\begin{pmatrix}
		1 & 0 & \dots & 0\\
		0 & 1 & \dots & 0\\
		\vdots & \vdots & & \vdots\\
		0 & 0 & \dots & 1
	\end{pmatrix}
\end{align}
(اکائی قالب  خطی تبادلہ کو ظاہر کرتا ہے جو ہر سمتیہ کا تبادلہ اسی سمتی میں کرتا ہے۔)   دوسرے لفظوں میں درجہ ذیل ہوگا۔
\begin{align}
	\mat{I}_{ij}=\delta_{ij}
\end{align}
چوکور قالب کے  \اصطلاح{معکوس}\فرہنگ{معکوس}\حاشیہب{inverse}\فرہنگ{inverse}  ،جسے \عددی{\mat{T}^{-1}} لکھا جاتا ہے،  کی تعریف  بدیہی ہے۔\حاشیہد{دھیان رہے کہ بایاں معکوس، دائیں معکوس کے برابر ہے، چونکہ اگر \عددی{\mat{AT}=\mat{I}} اور \عددی{\mat{TB}=\mat{I}} ہوں، تب (دوسرے کو بائیں سے \عددی{\mat{A}} سے ضرب کر کے  پہلا  استعمال کر نے سے) ہمیں \عددی{\mat{B}=\mat{A}} حاصل ہو گا۔}
\begin{align}
	\mat{T}^{-1}\mat{T}=\mat{T}\mat{T}^{-1}=\mat{I}
\end{align}
قالب کا معکوس صرف اور صرف اس صورت ہوگا جب اس  کا \اصطلاح{مقطع}\فرہنگ{مقطع}\حاشیہب{determinant}\فرہنگ{determinant} غیر صفر ہو؛   در حقیقت 
\begin{align}\label{مساوات_ضمیمہ_معکوس}
	\mat{T}^{-1}&=\frac{1}{\abs{\mat{T}}}\tilde{\mat{C}}& \text{\small\RL{قالب کا معکوس}}
\end{align}
ہوگا،  جہاں  \اصطلاح{ہم ضربیوں}\فرہنگ{ہم ضربی}\حاشیہب{cofactors}\فرہنگ{cofactor} کا قالب \عددی{\mat{C}} ہے اور \عددی{\abs{\mat{T}}}  قالب  کا مقطع   ہے    (قالب  \عددی{\mat{T}} سے \عددی{i}ویں صف اور \عددی{j}ویں قطار خارج  کر کے  حاصل ذیلی قالب کے مقطع کو   \عددی{(-1)^{i+j}} سے ضرب دینے سے  رکن \عددی{T_{ij}} کا ہم ضربی حاصل ہو گا۔)۔ (چوکور)  قالب کے مرکزی ارکان کے مجموعہ کو قالب کے    \اصطلاح{آثار}\فرہنگ{آثار}\حاشیہب{trace}\فرہنگ{trace}  کہتے ہیں۔
\begin{align*}
(\mat{T})\text{آثار} =\sum_{i}^n T_{ii}
\end{align*}
 ایسا قالب جس کا معکوس نہ پایا  جاتا ہو \اصطلاح{نادر}\فرہنگ{نادر}\حاشیہب{singular}\فرہنگ{singular}  کہلاتا ہے۔ حاصل ضرب کا معکوس ( اگر موجود ہو)    \ترچھا{الٹ ترتیب} میں انفرادی معکوس  کا حاصل ضرب ہوگا۔
\begin{align}\label{مساوات_ضمیمہ_سوال_میں_حل_تیسرا}
	(\mat{ST})^{-1}=\mat{T}^{-1}\mat{S}^{-1}
\end{align}
ایسا قالب جس کا معکوس اس کے ہرمشی جوڑی دار کے برابر ہو \اصطلاح{اکہرا}\فرہنگ{اکہرا}\حاشیہب{unitary}\فرہنگ{unitary}  کہلاتا ہے۔\حاشیہد{\ترچھا{حقیقی} سمتیہ فضا   (یعنی جس میں غیر سمتیات حقیقی ہوں)   میں ہرمشی جوڑی دار  اور  تبدیل محل ایک ہوں گے، اور  اکہرا قالب  قائمہ: \عددی{\tilde{\mat{O}}=\mat{O}^{-1}}   ہو گا۔مثلاً، سادہ تین بُعدی فضا میں  گھومنے کو  قائمہ قوالب سے ظاہر کیا جاتا ہے۔}

\begin{align}
	\mat{U}^\dagger& = \mat{U}^{-1}&\text{\small\RL{اکہرا}}
\end{align}
یہ فرض کرتے ہوئے کہ اساس معیاری عمودی ہے،  اکہرا قالب کے قطار معیاری عمودی سلسلہ قائم کرتے ہیں،  اور اس کے صف بھی ایسا  کرتے ہیں  (سوال \حوالہ{سوال_ضمیمہ_اکہرا_قالب_خواص} دیکھیں)۔ایسے  خطی تبادلہ جنہیں اکہرا قوالب  ظاہر کرتے ہوں،   مساوات \حوالہ{مساوات_ضمیمہ_برقرار} کی بدولت،  اندرونی ضرب  برقرار رکھتے ہیں۔
\begin{align}
	\langle\alpha'|\beta'\rangle=\mat{a}'^\dagger \mat{b}'=(\mat{Ua})^\dagger(\mat{Ub})=\mat{a}^\dagger \mat{U}^\dagger \mat{Ub}=\mat{a}^\dagger \mat{b}=\langle\alpha|\beta\rangle
\end{align}

\ابتدا{سوال}\شناخت{سوال_ضمیمہ_دو_قوالب}
درجہ ذیل قوالب لیتے ہوئے
\begin{align*}
	\mat{A}&=
	\begin{pmatrix}
		-1 & 1 & i\\
		2 & 0 & 3\\
		2i & -2i & 2
	\end{pmatrix}
	,&\mat{B}=
	\begin{pmatrix}
		2 & 0 & -i\\
		0 & 1 & 0\\
		i & 3 & 2
	\end{pmatrix}
\end{align*}
درجہ ذیل کا حساب لگائیں: (الف) \عددی{\mat{A}+\mat{B}}، (ب) \عددی{\mat{AB}}، (ج) \عددی{[\mat{A},\mat{B}]}، (د) \عددی{\tilde{\mat{A}}}، (ہ) \عددی{\mat{A}^*}، (و) \عددی{\mat{A}^\dagger}، (ز)آثار( \عددی{\mat{B}})،  (ح) مقطع(  \عددی{\mat{B}})، اور  (ط) \عددی{\mat{B}^{-1}}۔ دکھائیں کہ \عددی{\mat{B}\mat{B}^{-1}=\mat{I}} ہے۔ کیا \عددی{\mat{A}} کا معکوس  پایا جاتا  ہے؟
\انتہا{سوال}
\ابتدا{سوال}
قطار قوالب
\begin{align*}
	\mat{a}&=
	\begin{pmatrix}
		i\\2i\\2
	\end{pmatrix}
	,& \mat{b}=
	\begin{pmatrix}
		2\\(1-i)\\0
	\end{pmatrix}
\end{align*}
اور سوال \حوالہ{سوال_ضمیمہ_دو_قوالب}  میں مستعمل چوکور قوالب استعمال کرتے ہوئے درجہ ذیل تلاش  کریں۔ (الف) \عددی{\mat{Aa}}، (ب) \عددی{\mat{a}^{\dagger}\mat{b}}، (ج) \عددی{\tilde{\mat{a}}\mat{Bb}}،  (د) \عددی{\mat{ab}^\dagger}
\انتہا{سوال}
\ابتدا{سوال}
درجہ ذیل میں صریحاً قوالب تیار کرتے ہوئے دکھائیں کہ کسی بھی قالب \عددی{T} کو درجہ ذیل لکھا جا سکتا ہے۔
\begin{enumerate}
\item
 تشاکلی قالب \عددی{\mat{S}} اور خلاف تشاکلی قالب \عددی{\mat{A}} کا مجموعہ۔
\item
 حقیقی قالب \عددی{\mat{R}} اور خیالی قالب \عددی{\mat{M}} کا مجموعہ۔
\item
ہرمشی قالب \عددی{\mat{H}} اور منحرف ہرمشی قالب \عددی{\mat{K}} کا مجموعہ۔
\end{enumerate}
\انتہا{سوال}
\ابتدا{سوال}\شناخت{سوال_ضمیمہ_ثبوت}
مساوات \حوالہ{مساوات_ضمیمہ_سوال_میں_حل_پہلا}، مساوات  \حوالہ{مساوات_ضمیمہ_سوال_میں_حل_دوسرا} اور مساوات  \حوالہ{مساوات_ضمیمہ_سوال_میں_حل_تیسرا}  ثابت کریں۔ دکھائیں کہ دو اکہرا قوالب کا حاصل ضرب اکہرا ہوگا۔ کن شرائط کہ تحت دو ہرمشی قوالب کا حاصل ضرب بھی  ہرمشی ہوگا؟ کیا دو اکہرا قوالب کا مجموعہ اکہرا ہوگا؟ کیا دو ہرمشی قوالب کا مجموعہ ہرمشی ہوگا؟
\انتہا{سوال}
\ابتدا{سوال}\شناخت{سوال_ضمیمہ_اکہرا_قالب_خواص}
دکھائیں کہ اکہرا قالب کے صف اور قطار عمودی معیاری سلسلہ قائم کرتے ہیں۔
\انتہا{سوال}
\ابتدا{سوال}
یہ  جانتے  ہوئے کہ \عددی{\tilde{\mat{T}}\text{مقطع}=\mat{T}\text{مقطع}}ہے  دکھائیں کہ ہرمشی قالب کا مقطع حقیقی ہوگا،  اکہرا قالب کے مقطع کا معیار \عددی{1} ہوگا  (جس کی بنا اس کا نام اکہرا قالب ہے)  اور معیاری عمودی قالب کا مقطع  \عددی{+1} یا \عددی{-1} ہوگا۔
\انتہا{سوال}

\حصہ{تبدیلی اساس}\شناخت{ضمیمہ_تبدیلی_اساس}
 خطی تبادلہ کو ظاہر کرنے والے قالب کے ارکان یا  سمتیہ کے ارکان  یقیناً  اساس کے  انتخاب پر منحصر ہوں گے۔آئیں اس بات پر غور کرتے ہیں کہ    اساس کی  تبدیل  سے یہ   ا عداد کس طرح   تبدیل  ہوں گے۔
 
  پرانے اساسی سمتیات \عددی{|e_i\rangle}،  کسی بھی سمتیہ کی طرح،  ان نئے سمتیات \عددی{|f_i\rangle} کا خطی مجموعہ ہونگے:
\begin{align*}
	|e_1\rangle &= S_{11}|f_1\rangle + S_{21}|f_2\rangle + \dots + S_{n1}|f_n\rangle\\
	|e_2\rangle &= S_{12}|f_1\rangle + S_{22}|f_2\rangle +
	\dots + S_{n2}|f_n\rangle\\
	&\dots\\
	|e_n\rangle &= S_{1n}|f_1\rangle + S_{2n}|f_2\rangle + \dots + S_{nn}|f_n\rangle
\end{align*}
(جہاں \عددی{S_{ij}} مخلوط اعداد کا سلسلہ ہوگا)  یا مختصراً درج  ذیل۔
\begin{align}
	|e_{j}\rangle = \sum_{i=1}^{n}S_{ij}|f_i\rangle,\quad(j=1, 2, \dots, n)
\end{align}
یہ از خود ایک خطی تبادلہ ہے  (مساوات  \حوالہ{مساوات_ضمیمہ_خطی_تبادلہ}  سے موازنہ کریں)،\حاشیہد{یاد رہے کہ یہاں  موجودہ بحث میں ہم  ایک ہی سمتیہ کا دو مکمل مختلف اساس میں بات کر رہے ہیں، جبکہ  وہاں بالکل مختلف سمتیہ  کی بات اسی  ایک  اساس میں کی جا رہی تھی۔}   اور یوں  ہم جانتے ہیں کہ ارکان کا تبادلہ کس طرح ہوگا:
\begin{align}
	a_i^f = \sum_{j=1}^{n}S_{ij}a_j^e
\end{align}
(جہاں زیر بالا  اساس کو ظاہر کرتی ہے, ، یعنی \عددی{a^e} سے مراد اساسی سمتیات \عددی{|e_i\rangle} میں لکھے گئے ارکان ہیں)۔ قالبی روپ میں درجہ ذیل ہوگا۔
\begin{align}\label{مساوات_ضمیمہ_پرانا_اساس}
	\mat{a}^f = \mat{S}\mat{a}^e
\end{align}

خطی تبادلہ \عددی{\hat{T}} کو ظاہر کرنے والا قالب،  اساس کی تبدیلی سے کس طرح تبدیل ہوگا؟ پرانے اساس میں ہمارے پاس   (مساوات \حوالہ{مساوات_ضمیمہ_عمل_اظہار})
\begin{align*}
	\mat{a}^{e'} = \mat{T}^e\mat{a}^e
\end{align*}
اور مساوات \حوالہ{مساوات_ضمیمہ_پرانا_اساس}  تھے؛مساوات \حوالہ{مساوات_ضمیمہ_پرانا_اساس}   کے   دونوں اطراف کو \عددی{\mat{S}^{-1}} سے ضرب دے کر    \عددی{\mat{a}^e=\mat{S}^{-1}\mat{a}^f} ،  لہٰذا   
\begin{align*}
	\mat{a}^{f'} = \mat{S}\mat{a}^{e'} = \mat{S}(\mat{T}^e\mat{a}^e) = \mat{S}\mat{T}^e\mat{S}^{-1}\mat{a}^f
\end{align*}
حاصل\حاشیہد{یاد رہے کہ \عددی{\mat{S}^{-1}} لازماً موجود ہو گا؛ اگر \عددی{\mat{S}} نادر ہوتا، تب \عددی{|f_i\rangle} فضا کا احاطہ نہ کرتے، لہٰذا  اساس  قائم نہ کرتے۔}  ہو گا ( مساوات  \حوالہ{مساوات_ضمیمہ_پرانا_اساس} میں \عددی{\mat{a}^f} کی جگہ \عددی{\mat{a}^{f'}}، وغیرہ لکھا گیا ہے)۔ ظاہری طور پر
\begin{align}\label{مساوات_ضمیمہ_دوسرا_اساس}
	\mat{T}^f = \mat{S}\mat{T}^e\mat{S}^{-1}
\end{align}
ہوگا۔ عمومی طور پر دو قوالب ( \عددی{\mat{T}_1} اور \عددی{\mat{T}_2}) اس صورت  \اصطلاح{متشابہ}\فرہنگ{متشابہ}\حاشیہب{similar}\فرہنگ{similar}  ہونگے جب کسی (غیر نادر)  قالب \عددی{\mat{S}} کے لئے  \عددی{\mat{T}_2 = \mat{S}\mat{T}_1\mat{S}^{-1}} ہو۔ یوں ہم دریافت کر چکے کہ،  مختلف اساس لے لحاظ سے،  ایک ہی خطی تبادلہ کو ظاہر کرنے والے قوالب متشابہ ہونگے۔ اتفاقی طور پر،  اگر پہلی اساس معیاری عمودی ہو تب دوسری اساس صرف اس صورت  معیاری عمودی ہوگی  جب قالب \عددی{\mat{S}} اکہرا ہو  (سوال \حوالہ{سوال_ضمیمہ_اکہرا_عمودی} دیکھیں)۔ چونکہ ہم صرف معیاری عمودی اساس میں کام کرتے ہیں لہٰذا ہماری دلچسپی بنیادی طور پر  اکہرا  متشابہت  تبادلہ میں ہے۔

اگرچہ نئی اساس میں  خطی تبادلہ کے \ترچھا{ارکان}  بہت مختلف نظر آتے ہیں،  قالب سے وابستہ دو اعداد،  مقطع اور \اصطلاح{آثار}\فرہنگ{آثار}\حاشیہب{trace}\فرہنگ{trace}   قالب،  تبدیل نہیں ہوتے۔ چونکہ حاصل ضرب کا مقطع،  مقطعوں  کا حاصل ضرب ہوگا،  لہٰذا درجہ ذیل ہوگا۔
\begin{align}\label{مساوات_ضمیمہ_حاصل_ضرب}
	\abs{\mat{T}^f} = \abs{\mat{S}\mat{T}^e\mat{S}^{-1}} = \abs{\mat{S}}\abs{\mat{T}^e}\abs{\mat{S}^{-1}} = \abs{\mat{T}^e}
\end{align}
 آثار قالب  (\عددی{\Tr}) جو وتری ارکان کا مجموعہ ہے :
\begin{align}
	\Tr(\mat{T})\equiv\sum_{i=1}^{m}T_{ii}
\end{align}
درجہ ذیل خاصیت رکھتا ہے  (سوال \حوالہ{سوال_ضمیمہ_آثار}  دیکھیں)
\begin{align}
	\Tr(\mat{T}_1\mat{T}_2) = \Tr(\mat{T}_2\mat{T}_1)
\end{align}
 (جہاں \عددی{\mat{T}_1} اور \عددی{\mat{T}_2} کوئی بھی دو قوالب ہیں)،   لہٰذا درجہ ذیل ہوگا۔
\begin{align}\label{مساوات_ضمیمہ_آثار}
	\Tr(\mat{T}^f) = \Tr(\mat{S}\mat{T}^e\mat{S}^{-1}) = \Tr(\mat{T}_e\mat{S}^{-1}\mat{S}) = \Tr(\mat{T}^e)
\end{align}


\ابتدا{سوال}
تین ابعاد میں سمتیات کی لئے معیاری اساس \عددی{(\hat{i}, \hat{j}, \hat{k})} استعمال کرتے ہوئے۔
\begin{enumerate}[a.]
\item
(مبدا  کی طرف نیچے دیکھتے ہوئے)  خلاف گھڑی \عددی{z} محور کے گرد زاویہ \عددی{\theta} گھومنے کو ظاہر کرنے والا قالب تیار کریں۔
\item
نقطہ \عددی{(1, 1, 1)} سے گزرتے ہوئے محور کے گرد (محور سے  مبدا کی طرف نیچے دیکھتے ہوئے) خلاف گھڑی \عددی{120\degree} گھومنے کو ظاہر کرنے والا قالب تیار کریں۔
\item
 مستوی \عددی{xy} میں عکس کو ظاہر کرنے والا قالب تیار کریں۔
\item
 تصدیق کریں کہ یہ تمام قوالب معیاری عمودی ہیں اور ان کے مقطعات تلاش  کریں۔
\end{enumerate}
\انتہا{سوال}
\ابتدا{سوال}
عمومی اساس \عددی{(\hat{i}, \hat{j}, \hat{k})} میں  محور \عددی{x} کے گرد زاویہ \عددی{\theta} گھومنے کو ظاہر کرنے والا قالب \عددی{\mat{T}_x}،  اور محور \عددی{y} کے گرد زاویہ \عددی{\theta} گھومنے کو ظاہر کرنے والے قالب \عددی{\mat{T}_y} تیار کریں۔ فرض کریں اب ہم اساس تبدیل کر کے \عددی{\hat{i}'=\hat{j}}، \عددی{\hat{j}'=-\hat{i}}، \عددی{\hat{k}'=\hat{k}} لیتے ہیں۔  اساس کی اس تبدیلی کو پیدا کرنے والا قالب\عددی{\mat{S}} تیار کریں،  اور تصدیق کریں کہ آیا
 \عددی{\mat{S}\mat{T}_x\mat{S}^{-1}} اور \عددی{\mat{S}\mat{T}_y\mat{S}^{-1}} آپ کے توقعات کے مطابق ہیں یا نہیں۔
\انتہا{سوال}
\ابتدا{سوال}\شناخت{سوال_ضمیمہ_اکہرا_عمودی}
دکھائیں کہ متشابہت قالبی ضرب برقرار رکھتا ہے (یعنی  \عددی{\mat{A}^e\mat{B}^e=\mat{C}^e} ہونے کی صورت میں  \عددی{\mat{A}^f\mat{B}^f=\mat{C}^f} ہوگا)۔  متشابہت عمومی طور پر   تشاکلی ،حقیقت یا ہرمشی پن برقرار \ترچھا{نہیں} رکھتا؛  لیکن،  دکھائیں اگر \عددی{\mat{S}} \ترچھا{اکہرا}  ہو،  اور \عددی{\mat{H}^e} ہرمشی ہو،  تب \عددی{\mat{H}^f} ہرمشی ہوگا۔ دکھائیں کہ \عددی{\mat{S}} صرف اور صرف  اس صورت معیاری عمودی اساس کو دوسری معیاری عمودی اساس میں منتقل کرے گا اگر یہ اکہرا ہو۔
\انتہا{سوال}
\ابتدا{سوال}\شناخت{سوال_ضمیمہ_آثار}
ثابت کریں  کہ \عددی{\Tr(\mat{T}_1\mat{T}_2)=\Tr(\mat{T}_2\mat{T}_1)} ہو گا۔ یوں
 \عددی{\Tr(\mat{T}_1\mat{T}_2\mat{T}_3)=\Tr(\mat{T}_2\mat{T}_3\mat{T}_1)} ہوگا، لیکن  کیا عام طور پر
  \عددی{\Tr(\mat{T}_1\mat{T}_2\mat{T}_3)=\Tr(\mat{T}_2\mat{T}_1\mat{T}_3)} ہوگا؟ اس کو ٹھیک یا غلط ثابت کریں۔ \ترچھا{اشارہ}: غلط ثابت کرنے کا  بہترین ثبوت  اسکی اُلٹ مثال پیش کرنا ہے؛  جتنا مثال سادہ ہو  اتنا ہی بہتر ہے۔
\انتہا{سوال}

\حصہ{امتیازی سمتیات اور امتیازی اقدار}\شناخت{ضمیمہ_امتیازی_تفاعلات_و_اقدار}
تہرا فضا میں کسی مخصوص محور کے گرد زاویہ \(\theta\)گھمانے کو ظاہر کرنے والے خطی تبادلہ پر غور کریں۔ زیادہ تر سمتیات پچیدہ انداز سے تبدیل ہوں گے (یہ اس محور کے گرد مخروط پر حرکت کریں گے)،   لیکن وہ سمتیات جو \ترچھا{اسی} محور پر پائے جاتے ہوں کا رویہ نہایت سادہ ہوگا:  وہ بالکل تبدیل نہیں ہوں گے \((\hat{T}| \alpha \rangle=| \alpha \rangle)\)۔ اگر \(\theta\) کی قیمت \(\num{180}\degree\) ہو تب \قول{استوائی} مستوی میں پائے جانے والے سمتیات کی علامت تبدیل ہوگی \((\hat{T}| \alpha \rangle = -| \alpha \rangle)\)۔ مخلوط سمتی فضا\حاشیہد{حقیقی سمتی فضا میں (جہاں غیر سمتیہ کی قیمتیں حقیقی ہونے کی پابند ہوں گی)   ایسا لازمی نہیں۔ سوال \حوالہ{سوال_ضمیمہ_حقیقی_فضا_میں} دیکھیں۔}  میں \ترچھا{ہر} خطی تبادلہ کے،  اس طرح کے  \قول{مخصوص} سمتیات پائے جاتے ہیں جو اپنے آپ کے غیر سمتی مضرب میں تبدیل ہوتے:
\begin{align}\label{مساوات_ضمیمہ_مخصوص_سمتیات}
		\hat{T}|\alpha\rangle = \lambda|\alpha\rangle
\end{align}
انہیں اس تبادلہ کے  \اصطلاح{امتیازی سمتیات}\فرہنگ{امتیازی!سمتیات}\حاشیہب{eigenvectors}\فرہنگ{eigenvectors}  کہتے ہیں،  اور  (مخلوط)  عدد \(\lambda\) ان کا  \اصطلاح{امتیازی قدر}\فرہنگ{امتیازی!قدر}\حاشیہب{eigenvalue}\فرہنگ{eigenvalue} ہے۔ (اگرچہ، معدوم سمتیہ مہمل  معنوں میں مساوات \حوالہ{مساوات_ضمیمہ_مخصوص_سمتیات}  کو کسی بھی \(\hat{T}\) اور \(\lambda\) کے لئے مطمئن کرتا ہے،  اسے امتیازی سمتیات میں نہیں گنا جاتا۔ تکنیکی طور پر  امتیازی سمتیہ سے مراد وہ  \ترچھا{غیر صفر} سمتیہ ہے جو مساوات \حوالہ{مساوات_ضمیمہ_مخصوص_سمتیات} کو  مطمئن کرتا ہو۔)  دھیان رہے کہ امتیازی سمتیہ کا ہر   (غیر صفر)  مضرب بھی امتیازی سمتیہ ہوگا،  اور اس  کی امتیازی قدر وہی ہوگی۔

کسی مخصوص اساس کے لحاظ سے، امتیازی سمتیہ مساوات قالبی روپ: 
\begin{align}\label{مساوات_ضمیمہ_قالبی_امتیازی}
	\mat{T}\mat{a} = \lambda \mat{a}
\end{align}
(جہاں \عددی{\mat{a}} غیر صفر ہے) یا
\begin{align}\label{مساوات_ضمیمہ_قالبی_روپ}
	(\mat{T}-\lambda \mat{I})\mat{a} = \mat{0}
\end{align}
اختیار کرتی ہے۔ ( یہاں \عددی{ \mat{0}}  ایسا \اصطلاح{ صفر قالب}\فرہنگ{قالب!صفر}\حاشیہب{zero matrix}\فرہنگ{matrix!zero} ہے جس کے تمام ارکان صفر ہیں۔)  اب،  اگر
 قالب  \عددی{(\mat{T}-\lambda \mat{I})} کا \ترچھا{معکوس} پایا جاتا،  ہم مساوات \حوالہ{مساوات_ضمیمہ_قالبی_روپ}  کے  دونوں اطراف
  کو  \عددی{(\mat{T}-\lambda \mat{I})^{-1}} سے ضرب دے کر  \عددی{\mat{a}=\mat{0}} اخذ کرتے۔ لیکن ہم \عددی{\mat{a}} کو غیر صفر فرض کر چکے
   ہیں،  لہٰذا \عددی{(\mat{T}-\lambda \mat{I})}  حقیقتاً  نادر ہوگا،  جس سے مراد یہ ہے کہ اس کا مقطع صفر ہوگا۔
\begin{align}
	(\mat{T}-\lambda \mat{I})\text{مقطع}=
	\begin{vmatrix}
		(T_{11}-\lambda) & T_{12} & \dots & T_{1n}\\
		T_{21} & (T_{22}-\lambda) & \dots & T_{2n}\\
		\vdots & \vdots & & \vdots\\
		T_{n1} & T_{n2} & \dots & (T_{nn}-\lambda)
	\end{vmatrix}
		= 0.
\end{align}
مقطع  کھولنے  سے \عددی{\lambda} کی الجبرائی مساوات:
\begin{align}\label{مساوات_ضمیمہ_امتیازی_مساوات}
	C_n\lambda^{n}+C_{n-1}\lambda^{n-1}+\dots+C_1\lambda+C_0 = 0\quad \text{\small\RL{امتیازی مساوات}}
\end{align}
حاصل ہوتی ہے،  جہاں عددی سر \عددی{C_i} کی قیمتیں   \عددی{T}  کے ارکان کی  تابع ہیں (سوال  \حوالہ{سوال_ضمیمہ_ارکان_کی_تابع}  دیکھیں)۔ اس کو قالب کی \اصطلاح{امتیازی مساوات}\فرہنگ{امتیازی!مساوات}\حاشیہب{characteristic equation}\فرہنگ{characteristic!equation}  کہتے ہیں؛  اور اس کے حل امتیازی اقدار کا تعین کرتے ہیں۔ یاد رہے کہ یہ \عددی{n} رتبی مساوات ہے،  لہٰذا  (\اصطلاح{الجبرا کے بنیادی مسئلہ}\فرہنگ{الجبرا کا بنیادی مسئلہ}\حاشیہب{fundamental theorem of algebra}\فرہنگ{fundamental theorem of algebra} کے تحت) اس کے \عددی{n} (مخلوط) جذر  ہوں گے۔\حاشیہد{یہ وہ مقام ہے جہاں حقیقی سمتی فضا کا مسئلہ مزید  پیچیدہ ہوتا ہے، چونکہ ضروری نہیں  امتیازی مساوات کا کوئی  بھی (حقیقی) حل پایا جاتا ہو۔ سوال \حوالہ{سوال_ضمیمہ_حقیقی_فضا_میں} دیکھیں۔} تاہم، ان میں سے چند  \اصطلاح{متعدد  جذر}\فرہنگ{جذر!متعدد}\حاشیہب{multiple roots}\فرہنگ{multiple roots}  ہو سکتے ہیں، لہٰذا ہم صرف اتنا کہہ سکتے ہیں کہ \عددی{n\times n}  قالب کا کم سے کم ایک اور زیادہ سے زیادہ \عددی{n} منفرد امتیازی اقدار ہو سکتے ہیں۔ قالب کے تمام امتیازی اقدار کے ذخیرہ کو اس کا \اصطلاح{طیف}\فرہنگ{طیف}\حاشیہب{spectrum}\فرہنگ{spectrum}  کہتے ہیں؛ اگر دو یا دو سے زیادہ خطی غیر تابع امتیازی سمتیات کا ایک ہی امتیازی قدر ہو،  ہم کہتے ہیں  طیف \اصطلاح{انحطاطی}\فرہنگ{انحطاطی}\حاشیہب{degenerate}\فرہنگ{degenerate} ہے۔

عام طور پر، امتیازی سمتیات تیار کرنے کا  سادہ ترین طریقہ یہ ہو گا کہ مساوات \حوالہ{مساوات_ضمیمہ_قالبی_امتیازی}  میں ہر ایک \عددی{\lambda}  ڈال کر \عددی{\mat{a}} کے ارکان کے لئے قلم و کاغذ  سے حل کیا جائے۔ میں یہ عمل ایک مثال حل کر کے   سمجھاتا ہوں۔

\ابتدا{مثال}\شناخت{مثال_ضمیمہ_امتیازی_اقدار_حصول}
درج ذیل قالب کے امتیازی اقدار اور امتیازی سمتیات تلاش کریں۔

\begin{align}
	\mat{M}=
	\begin{pmatrix}
		2 & 0 & -2\\
		-2i & i & 2i\\
		1 & 0 & -1
	\end{pmatrix}
\end{align}
\موٹا{حل}: اس کی امتیازی مساوات 
\begin{align}
	\begin{vmatrix}
		(2-\lambda) & 0 & -2\\
		-2i & (i-\lambda) & 2i\\
		1 & 0 & (-1-\lambda)
	\end{vmatrix}
		=-\lambda^3 + (1+i)\lambda^2-i\lambda = 0
\end{align}
ہے،  جس کے جذر \عددی{0}، \عددی{1}  اور \عددی{i} ہیں۔ پہلے امتیازی سمتیہ کے     جزو  \عددی{(a_1, a_2, a_3)} لیتے ہوئے
\begin{align*}
	\begin{pmatrix}
		2 & 0 & -2\\
		-2i & i & 2i\\
		1 & 0 & -1
	\end{pmatrix}
	\begin{pmatrix}
		a_1\\
		a_2\\
		a_3
	\end{pmatrix}
		=0
	\begin{pmatrix}
		a_1\\
		a_2\\
		a_3
	\end{pmatrix}
		=
	\begin{pmatrix}
		0\\0\\0
	\end{pmatrix}
\end{align*}
ہوگا،  جو درجہ ذیل تین مساوات دیتا ہے۔
\begin{align*}
	2a_1 - 2a_3 &= 0\\
	-2ia_1 + ia_2 + 2ia_3 &= 0\\
	a_1 - a_3 &= 0
\end{align*}
ان میں سے پہلی مساوات  (\عددی{a_1} کی صورت میں)  \عددی{a_3} کا تعین کرتی ہے:  \عددی{a_3 = a_1}؛  دوسری  مساوات \عددی{a_2} کا تعین کرتی ہے:  \عددی{a_2 = 0}؛  اور تیسری مساوات  زائد از ضرورت  م ہے۔ ہم \عددی{a_1 = 1} چن سکتے ہیں  (چونکہ امتیازی سمتیہ کا کوئی بھی مضرب امتیازی سمتیہ ہی ہوگا)۔
\begin{align}\label{مساوات_ضمیمہ_پہلا_امتیازی_قدر}
	\mat{a}^{(1)}&=
	\begin{pmatrix}
		1\\0\\1
	\end{pmatrix},
		&\text{\small\RL{کے لئے}}\, \lambda_1 = 0
\end{align}
دوسرے امتیازی سمتیہ کے لئے ( جزو کی  وہی  علامتیں استعمال کرتے ہوئے )
\begin{align*}
	\begin{pmatrix}
		2 & 0 & -2\\
		-2i & i & 2i\\
		1 & 0 & -1
	\end{pmatrix}
	\begin{pmatrix}
		a_1\\a_2\\a_3
	\end{pmatrix}
		=1
	\begin{pmatrix}
		a_1\\a_2\\a_3
	\end{pmatrix}
		=
	\begin{pmatrix}
		a_1\\a_2\\a_3
	\end{pmatrix}
\end{align*}
ہو گا،  جس سے درجہ ذیل مساوات حاصل ہوں گی:
\begin{align*}
	2a_1-2a_3 &= a_1\\
	-2ia_1 + ia_2 + 2ia_3 &= a_2\\
	a_1 - a_3 &= a_3
\end{align*}
جن کے حل  \عددی{a_3 =(1/2)a_1}، \عددی{ a_2=[(1-i)/2]a_1} ہیں؛  اس مرتبہ میں \عددی{a_1 = 2} لیتا  ہوں،  لہٰذا
\begin{align}\label{مساوات_ضمیمہ_دوسرا_امتیازی_قدر}
	\mat{a}^{(2)} &=
	\begin{pmatrix}
		2\\1-i\\1
	\end{pmatrix}
		, &\text{\small\RL{کے لئے}}\,\lambda_2 = 1 
\end{align}
ہوگا۔ آخر میں،  تیسرا امتیازی سمتیہ  کے لئے
\begin{align*}
	\begin{pmatrix}
		2 & 0 & -2\\
		-2i & i & 2i\\
		1 & 0 & -1
	\end{pmatrix}
	\begin{pmatrix}
		a_1\\a_2\\a_3
	\end{pmatrix}
		=i
	\begin{pmatrix}
		a_1\\a_2\\a_3
	\end{pmatrix}
		=
	\begin{pmatrix}
		ia_1\\ia_2\\ia_3
	\end{pmatrix}
\end{align*}
درجہ ذیل مساوات دیگا
\begin{align*}
	2a_1-2a_3 &= ia_1\\
	-2ia_1 + ia_2 + 2ia_3 &= ia_2\\
	a_1 - a_3 &= ia_3
\end{align*}
جس کے حل \عددی{a_3 = a_1 = 0} ہیں،  جہاں \عددی{a_2} غیر متعین ہے۔ ہم \عددی{a_2 = 1}چنتے ہیں، یوں درجہ ذیل ہوگا۔
\begin{align}\label{مساوات_ضمیمہ_تیسرا_امتیازی_قدر}
	\mat{a}^{(3)}&=
	\begin{pmatrix}
		0\\1\\0
	\end{pmatrix}
	, &\text{\small\RL{کے لئے}}\,\lambda_3 = i
\end{align}
\انتہا{مثال}

اگر امتیازی سمتیات فضا کا احاطہ کرتے ہوں (جیسا گزشتہ مثال میں کرتے تھے)،   ہم انہیں اساس کے طور پر استعمال کر سکتے ہیں۔
\begin{align*}
	\hat{T}| f_1\rangle &= \lambda_1| f_1\rangle,\\
	\hat{T}| f_2\rangle &= \lambda_2| f_2\rangle,\\
	&\dots\\
	\hat{T}| f_n\rangle &= \lambda_n| f_n\rangle
\end{align*}
اس اساس میں \عددی{\hat{T}} کو ظاہر کرنے والا قالب انتہائی سادہ روپ اختیار کرتا ہے،  جس میں امتیازی اقدار مرکزی وتر پر پائے جاتے ہیں، جبکہ باقی تمام ارکان صفر ہوں گے:
\begin{align}\label{مساوات_ضمیمہ_وتری_قالب}
	\mat{T}=
	\begin{pmatrix}
		\lambda_1 & 0 & \dots & 0\\
		0 & \lambda_2 & \dots & 0\\
		\vdots & \vdots & & \vdots\\
		0 & 0 & \dots & \lambda_n
	\end{pmatrix}
\end{align}
اور  (معمول شدہ)  امتیازی سمتیات درجہ ذیل ہوں گے۔
\begin{align}
	\begin{pmatrix}
		1\\0\\0\\ \vdots\\0
	\end{pmatrix}
	,\quad 
	\begin{pmatrix}
		0\\1\\0\\\vdots\\0
	\end{pmatrix}
	, \dots,\quad 
	\begin{pmatrix}
		0\\0\\0\\\vdots\\1
	\end{pmatrix}
\end{align}

ایسا قالب جس کو اساس کی تبدیلی سے \اصطلاح{وتری روپ}\فرہنگ{وتری!روپ}\حاشیہب{diagonal form}\فرہنگ{diagonal!form}   (مساوات \حوالہ{مساوات_ضمیمہ_وتری_قالب})   میں لایا جا سکے  \اصطلاح{وتر پذیر}\فرہنگ{وتر!پذیر}\حاشیہب{diagonalizable}\فرہنگ{diagonalizable}   کہلاتا ہے (ظاہر ہے کہ  ایک قالب  صرف اور صرف  اس صورت   وتر پذیر   ہوگا جب اس کے امتیازی سمتیات فضا کا احاطہ کرتے ہوں)۔   (پرانی اساس میں)  معمول شدہ امتیازی سمتیات  کو \عددی{\mat{S}^{-1}} کے قطار لیتے ہوئے،   متشابہت قالب جو \اصطلاح{وتری سازی}\فرہنگ{وتری سازی}\حاشیہب{diagonalization}\فرہنگ{diagonalization}  کرتا  ہے،  تیار کیا جا سکتا ہے۔
\begin{align}
	(\mat{S}^{-1})_{ij} = (\mat{a}^{(j)})_i
\end{align}
\ابتدا{مثال}
ہم  مثال \حوالہ{مثال_ضمیمہ_امتیازی_اقدار_حصول}  میں حاصل \عددی{\mat{a}^1}  (مساوات \حوالہ{مساوات_ضمیمہ_پہلا_امتیازی_قدر})، \عددی{\mat{a}^2}  (مساوات \حوالہ{مساوات_ضمیمہ_دوسرا_امتیازی_قدر}) اور \عددی{\mat{a}^3}  (مساوات \حوالہ{مساوات_ضمیمہ_تیسرا_امتیازی_قدر}) کو \عددی{\mat{S}^{-1}} کے قطار لکھتے ہیں:
\begin{align*}
	\mat{S}^{-1} =
	\begin{pmatrix}
		1 & 2 & 0\\
		0 & (1-i) & 1\\
		1 & 1 & 0
	\end{pmatrix}
\end{align*}
لہٰذا     (مساوات \حوالہ{مساوات_ضمیمہ_معکوس}  استعمال کرتے ہوئے)
\begin{align*}
	\mat{S}=
	\begin{pmatrix}
		-1 & 0 & 2\\
		1 & 0 & -1\\
		(i-1) & 1 & (1-i)
	\end{pmatrix}
\end{align*}
اور آپ تصدیق کر سکتے ہیں کہ
\begin{align*}
	\mat{Sa}^{(1)}=
	\begin{pmatrix}
		1\\0\\0
	\end{pmatrix}
	,\quad
	\mat{Sa}^{(2)}=
	\begin{pmatrix}
		0\\1\\0
	\end{pmatrix}
	,\quad
	\mat{Sa}^{(3)}=
	\begin{pmatrix}
		0\\0\\1
	\end{pmatrix}
\end{align*}
اور
\begin{align*}
	\mat{SMS}^{-1}=
	\begin{pmatrix}
		0 & 0 & 0\\
		0 & 1 & 0\\
		0 & 0 & i
	\end{pmatrix}
\end{align*}
ہوں گے۔
\انتہا{مثال}


قالب کو وتری روپ میں لانے کا  فائدہ صاف ظاہر  ہے:  اس کے ساتھ کام کرنا زیادہ آسان ہے۔ بد قسمتی سے،  ہر قالب کو وتری نہیں بنایا جا سکتا؛   امتیازی سمتیات کو فضا کا احاطہ کرنا ہوگا۔ اگر امتیازی مساوات کے \عددی{n}  منفرد جذر ہوں،  تب قالب لازماً وتر پذیر ہو گا، لیکن   بعض اوقات متعدد  جذر کی صورت میں بھی   یہ وتر پذیر  ہو گا۔ (غیر  وتر پذیر  قالب کی مثال  کے لئے سوال \حوالہ{سوال_ضمیمہ_غیر_وتر_پذیر} دیکھیں۔)  کیا بہتر ہوتا (اگر تمام امتیازی سمتیات معلوم کرنے سے قبل) ہم جان سکتے کہ آیا  قالب  وتر پذیر  ہے یا نہیں۔ ایک  کارآمد کافی (تاہم غیر لازمی)  شرط درجہ ذیل ہے:  ایک قالب جو اپنے ہرمشی جوڑی دار کے ساتھ مقلوب ہو  \اصطلاح{عمودی}\فرہنگ{عمودی}\حاشیہب{normal}\فرہنگ{normal} قالب کہلاتا ہے۔
\begin{align}\label{مساوات_ضمیمہ_معیار}
	[\mat{N}^\dagger,\mat{N}] &= \mat{0}, &\text{\small\RL{عمودی}}
\end{align}
\ترچھا{ہر عمودی قالب وتر پذیر  ہو گا} ( اس کے امتیازی سمتیات فضا کا احاطہ کرتے ہیں)۔ بالخصوص، ہر ہرمشی قالب، اور اکہرا قالب، وتر پذیر  ہو گا۔

فرض کریں ہمارے پاس دو وتر پذیر قوالب ہوں؛  کوانٹائی معاملات میں عموماً ایک سوال کھڑا  ہوتا ہے: کیا انہیں  ( ایک ہی متشابہت قالب \عددی{S} کے ذریعہ) \اصطلاح{ بیک وقت وتری}\فرہنگ{بیک وقت وتری}\حاشیہب{simultaneously diagonalized}\فرہنگ{diagonalized!simultaneously}  بنایا جا سکتا ہے؟ دوسرے لفظوں میں، کیا ایسی اساس موجود ہے جس میں دونوں وتری ہوں؟ اس کا جواب ہے کہ صرف اور صرف اس صورت ایسا ممکن ہوگا جب دونوں قالب آپس میں  مقلوبی ہوں  (سوال \حوالہ{سوال_ضمیمہ_قوالب_مقلوبی}  دیکھیں)۔


\ابتدا{سوال}\شناخت{سوال_ضمیمہ_حقیقی_فضا_میں}
درج ذیل قالب مستوی \عددی{xy} میں گھومنے کو ظاہر کرنے والا \عددی{2 \times 2} قالب ہے۔
\begin{align}
	\mat{T}=
	\begin{pmatrix}
		\cos\theta & -\sin\theta\\
		\sin\theta & \cos\theta
	\end{pmatrix}
\end{align}
دکھائیں  کہ (ماسوائے مخصوص زاویوں کے؛  بتائیں وہ کون سے زاویہ ہیں؟)  اس قالب کے کوئی حقیقی امتیازی اقدار نہیں پائے  جاتے۔  (یہ اس ہندسی حقیقت کی عکاسی کرتا   ہے کہ مستوی میں کسی بھی   سمتیہ کو  ایسا  گھما کر  اپنے آپ میں نہیں پہنچایا جا سکتا؛ اس کا موازنہ تین ابعاد میں گھمانے سے کریں۔)  اس قالب کے، البتہ،  مخلوط امتیازی اقدار اور امتیازی سمتیات پائے جاتے ہیں ۔ انہیں تلاش کریں۔ قالب \عددی{\mat{T}} کا  وتری ساز  قالب \عددی{\mat{S}} تیار کریں۔ متشابہت تبادلہ \عددی{\mat{STS}^{-1}} صریحاً کریں،  اور دکھائیں کہ یہ \عددی{\mat{T}}  کو وتری روپ میں گھٹاتا ہے۔
\انتہا{سوال}
\ابتدا{سوال}\شناخت{سوال_ضمیمہ_غیر_وتر_پذیر}
درجہ ذیل قالب کے امتیازی اقدار اور امتیازی سمتیات تلاش کریں۔
\begin{align*}
\mat{M}=
	\begin{pmatrix}
		1 & 1\\
		0 & 1
	\end{pmatrix}
\end{align*}
کیا یہ قالب وتر پذیر ہے؟
\انتہا{سوال}
\ابتدا{سوال}\شناخت{سوال_ضمیمہ_ارکان_کی_تابع}
دکھائیں کہ امتیازی مساوات  (مساوات \حوالہ{مساوات_ضمیمہ_امتیازی_مساوات})  کا  پہلا، دوسرا اور آخری عددی سر درجہ ذیل ہے۔
\begin{align}
	C_n = (-1)^n,\quad  C_{n-1} = (-1)^{n-1}\Tr(\mat{T}),\quad  \text{\RL{اور}}\quad  C_0 = \abs{\mat{T}}
\end{align}
ایک \عددی{3 \times 3} قالب جس کے ارکان \عددی{T_{ij}} ہوں کا \عددی{C_1} کیا ہوگا؟
\انتہا{سوال}
\ابتدا{سوال}
صاف ظاہر ہے کہ \ترچھا{وتری} قالب کا آثار، اس  قالب  کے امتیازی اقدار کا مجموعہ،  اور اس کا مقطع ان کا حاصل ضرب ہوگا   (صرف مساوات \حوالہ{مساوات_ضمیمہ_وتری_قالب}  کو دیکھنے کی دیر ہے)۔   یوں  (مساوات \حوالہ{مساوات_ضمیمہ_حاصل_ضرب}  اور مساوات \حوالہ{مساوات_ضمیمہ_آثار}  کے تحت)  کسی بھی وتر پذیر قالب کے لئے بھی ایسا ہی ہوگا۔ہر قالب کے لئے درج ذیل ہو گا؛ اسے ثابت کریں۔
\begin{align}\label{مساوات_ضمیمہ_مقطع_آثار}
	\abs{\mat{T}}= \lambda_1\lambda_2\cdots\lambda_n,\quad \Tr(\mat{T})=\lambda_1+\lambda_2+\cdots+\lambda_n
\end{align}
(یہاں کے \عددی{\lambda}،  امتیازی مساوات کے \عددی{n} حل   ہیں؛  متعدد  جذر کی صورت میں،  خطی غیر تابع امتیازی سمتیات کی تعداد،   حلوں کی تعداد  سے کم  ہو سکتی  ہے،   لیکن ہم \عددی{\lambda} کو اتنی مرتبہ ہی گنتے  ہیں جتنی مرتبہ یہ پایا جاتا ہے۔) \ترچھا{اشارہ}: امتیازی مساوات کو درجہ ذیل روپ میں لکھیں 
\begin{align*}
	(\lambda_1-\lambda)(\lambda_2-\lambda)\dots(\lambda_n-\lambda) = 0
\end{align*}
اور سوال  \حوالہ{سوال_ضمیمہ_ارکان_کی_تابع}  کا نتیجہ زیر  استعمال  لائیں۔
\انتہا{سوال}
\ابتدا{سوال}\شناخت{سوال_ضمیمہ_قوالب_مقلوبی}
\begin{enumerate}[a]
\item
دکھائیں اگر دو قالب کسی ایک اساس میں مقلوبی ہوں تب وہ ہر اساس میں مقلوبی ہوں گے۔  یعنی درجہ ذیل ہوگا۔
\begin{align}
	[\mat{T}^e_1, \mat{T}^e_2] = \mat{0} \Rightarrow [\mat{T}^f_1, \mat{T}^f_2] = \mat{0}
\end{align}
\ترچھا{اشارہ}: مساوات \حوالہ{مساوات_ضمیمہ_دوسرا_اساس}  استعمال کریں۔
\item
 دکھائیں کہ اگر دو قالب بیک وقت وتر پذیر ہوں، وہ مقلوبی ہوں گے۔\حاشیہد{اس کا الٹ (یعنی اگر دو وتر پذیر قالب مقلوبی ہوں تب وہ بیک وقت وتر پذیر ہوں گے) ثابت کرنا اتنا آسان نہیں۔}
 \end{enumerate}
\انتہا{سوال}
\ابتدا{سوال}
درجہ ذیل قالب لیں۔
\begin{align*}
	\mat{M}=
	\begin{pmatrix}
		1 & 1\\
		1 & i
	\end{pmatrix}
\end{align*}
\begin{enumerate}[a]
\item
 کیا یہ عمودی  ہے؟
\item
 کیا یہ وتر پذیر  ہے؟
 \end{enumerate}
\انتہا{سوال}


\حصہ{ہرمشی تبادلہ}\شناخت{ضمیمہ_ہرمشی_تبادلے}
میں نے مساوات \حوالہ{مساوات_ضمیمہ_ہرمشی} میں قالب کے تبدیل محل و جوڑی دار \عددی{\mat{T}^\dagger = {\tilde{\mat{T}}^*}} کو اس کے   ہرمشی جوڑی دار  (یا شریک قالب) کی تعریف قرار دیا۔  میں اب  خطی تبادلہ کے ہرمشی جوڑی دار کی  زیادہ بنیادی تعریف پیش کرتا ہوں۔  یہ وہ تبادلہ \عددی{\hat{\mat{T}}^\dagger} ہے جس کا اطلاق (ہر  \عددی{|\alpha\rangle} اور \عددی{|\beta\rangle} سمتیات کے)   اندرونی ضرب کے پہلے رکن پر وہی نتیجہ دیتا ہے جو دوسرے سمتیہ پر  \عددی{\hat{\mat{T}}} کا اطلاق دیگا۔\حاشیہد{آپ پوچھ سکتے ہیں،  کیا  ایسا تبادلہ لازماً موجود ہوگا؟ یہ ایک اچھا سوال ہے۔ اس کا جواب ہے \قول{ جی ہاں}۔}
\begin{align}
	\langle\hat{\mat{T}}^{\dagger}\alpha|\beta\rangle = \langle\alpha|\hat{\mat{T}}\beta\rangle
\end{align}
  میں آپ کو خبردار کرتا چلوں کہ اگرچہ ہر کوئی اسے استعمال کرتا ہے یہ فرسودہ علامتیت ہے۔ سمتیات  \عددی{|\alpha\rangle} اور \عددی{ؔ|\beta\rangle} ہیں نا کہ  \عددی{\alpha} اور \عددی{\beta} جو درحقیقت  محض \ترچھا {نام} ہیں۔ بالخصوص،  ان کے کوئی ریاضیاتی خواص نہیں پائے جاتے،  اور \قول{\عددی{\hat{T} \beta}}  کا فقرہ بے معنی ہے۔ خطی تبادلہ  سمتیہ پر نا کہ نام پر عمل کرتے ہیں۔ تاہم،  اس علامت کا مطلب صاف ظاہر ہے: سمتیہ  \عددی{\hat{T}|\beta\rangle}  کا نام    \عددی{\hat{T}\beta} ہے اور سمتیہ \عددی{\hat{T}^\dagger|\alpha\rangle} اور سمتیہ \عددی{|\beta\rangle} کا اندرونی ضرب \عددی{\langle\hat{T}^\dagger\alpha|\beta\rangle} ہے۔ بالخصوص
\begin{align}
	\langle\alpha| c\beta\rangle = c\langle\alpha|\beta\rangle
\end{align}
ہو گا، جبکہ  جہاں کسی بھی غیر سمتیہ \عددی{c} کے لئے درجہ ذیل ہوگا۔
\begin{align}
	\langle c\alpha|\beta\rangle = c^{*}\langle\alpha|\beta\rangle
\end{align}

اگر آپ (ہمیشہ کی طرح)  معیاری عمودی اساس میں کام کر رہے ہوں، خطی تبادلہ کے ہرمشی جوڑی دار کو مطابقتی قالب کا ہرمشی جوڑی دار ظاہر کریگا؛  چونکہ (مساوات \حوالہ{مساوات_ضمیمہ_برقرار} اور مساوات \حوالہ{مساوات_ضمیمہ_سوال_میں_حل_دوسرا}  استعمال کرتے ہوئے)  درجہ ذیل ہے۔
\begin{align}
	\langle\alpha|\hat{T}\beta\rangle = \mat{a}^\dagger \mat{Tb} = (\mat{T}^\dagger \mat{a})^\dagger \mat{b} = \langle\hat{T}^\dagger\alpha|\beta\rangle
\end{align}
یوں یہ علامتیت ثباتی  ہے،  اور ہم چاہیں تو تبادلہ کی زبان اور چاہیں تو قوالب کی زبان میں بات کر سکتے ہیں۔

کوانٹائی میکانیات میں،  \اصطلاح{ہرمشی تبادلہ}\فرہنگ{ہرمشی!تبادلہ}\حاشیہب{hermitian transformation}\فرہنگ{hermitian!transformation}
   \عددی{(\hat{T}^\dagger = \hat{T})} بنیادی کردار ادا کرتے ہیں۔ ہرمشی تبادلہ کے امتیازی سمتیات اور امتیازی اقدار تین نہایت اہم خواص رکھتے ہیں۔
\begin{enumerate}[a]
\item
\موٹا{ہرمشی تبادلہ کے امتیازی اقدار حقیقی ہوں گے۔}

\موٹا{ثبوت}: فرض کریں \عددی{\hat{T}} کی ایک امتیازی قدر \عددی{\lambda} ہے:  \عددی{\hat{T}|\alpha\rangle = \lambda|\alpha\rangle} ، جہاں \عددی{|\alpha\rangle\neq|0\rangle} ہے۔ تب درجہ ذیل ہوگا۔
\begin{align*}
	\langle\alpha|\hat{T}\alpha\rangle = \langle\alpha|\lambda\alpha\rangle = \lambda\langle\alpha|\alpha\rangle
\end{align*}
ساتھ ہی \عددی{\hat{T}} ہرمشی ہے لہٰذا درجہ ذیل ہوگا۔
\begin{align*}
	\langle\alpha|\hat{T}\alpha\rangle = \langle\hat{T}\alpha|\alpha\rangle = \langle\lambda\alpha|\alpha\rangle = \lambda^*\langle\alpha|\alpha\rangle
\end{align*}
لیکن \عددی{\langle\alpha|\alpha\rangle\neq0} ہے  (مساوات \حوالہ{مساوات_خطی_معیار_مثبت})  لہٰذا \عددی{\lambda = \lambda^*} اور یوں \عددی{\lambda} حقیقی ہوگا۔
\item
\موٹا{ہرمشی تبادلہ کے منفرد امتیازی اقدار والے  امتیازی سمتیات قائمہ ہونگے۔}

\موٹا{ثبوت}: فرض کریں \عددی{\hat{T}|\alpha\rangle = \lambda|\alpha\rangle} اور \عددی{\hat{T}|\beta\rangle = \mu|\beta\rangle} ہیں،  جہاں \عددی{\lambda\neq\mu} ہے۔ تب
\begin{align*}
	\langle\alpha|\hat{T}\beta\rangle = \langle\alpha|\mu\beta\rangle = \mu\langle\alpha|\beta\rangle
\end{align*}
اور اگر \عددی{\hat{T}} ہرمشی ہو درجہ ذیل ہوگا۔
\begin{align*}
	\langle\alpha|\hat{T}\beta\rangle = \langle\hat{T}\alpha|\beta\rangle = \langle\lambda\alpha|\beta\rangle = \lambda^*\langle\alpha|\beta\rangle
\end{align*}
لیکن   (جزو -الف کے تحت)  \عددی{\lambda = \lambda^*} ہے،   اور ہم فرض کر چکے ہیں کہ \عددی{\lambda\neq\mu} ہے،  لہٰذا \عددی{\langle\alpha|\beta\rangle = 0} ہوگا۔
\item
\موٹا{ہرمشی تبادلہ کے امتیازی سمتیات فضا کا احاطہ کرتے ہیں۔}

 جیسا ہم دیکھ چکے ہیں،  یہ اس فقرہ  کے مترادف ہے کہ ہر  ہرمشی قالب کو وتری بنایا جا سکتا ہے (مساوات \حوالہ{مساوات_ضمیمہ_معیار}  دیکھیں)۔ یہ    حقیقت جو خاصی تکنیکی ہے ،   وہ ریاضیاتی سہارا ہے جس پر، ایک لحاظ سے،  زیادہ تر کوانٹائی میکانیات کھڑی ہے۔ چونکہ اس ثبوت کو لامتناہی ابعادی سمتی فضاوں تک وسعت نہیں دی جا سکتی،  لہٰذا یہ ایک نہایت نازک اور  باریک لڑی ہے جس پر کوانٹائی میکانیات منحصر ہے۔
\end{enumerate}
%????KKKK
\ابتدا{سوال}
ہرمشی خطی تبادلہ کو تمام سمتیات \عددی{|\alpha\rangle} اور \عددی{|\beta\rangle} کے لئے لازماً \عددی{\langle\alpha|\hat{T}\beta\rangle = \langle\hat{T}\alpha|\beta\rangle} مطمئن کرنا ہوگا۔ دکھائیں کہ اتنا  کافی ہو گا  (جو ایک حیرانی کی بات ہے)   کہ تمام سمتیات \عددی{|\gamma\rangle} کے لئے \عددی{\langle\gamma|\hat{T}\gamma\rangle = \langle\hat{T}\gamma|\gamma\rangle} ہو۔\ترچھا{اشارہ}: پہلے \عددی{|\gamma\rangle = |\alpha + |\beta\rangle} اور اس کے بعد \عددی{|\gamma\rangle = |\alpha\rangle + i|\beta\rangle} لیں۔ 
\انتہا{سوال}
\ابتدا{سوال}
درجہ ذیل لیں۔
\begin{align*}
	\mat{T}=
	\begin{pmatrix}
		1 & 1-i\\
		1+i & 0
	\end{pmatrix}
\end{align*}
\begin{enumerate}[a]
\item
 تصدیق کریں کہ \عددی{\mat{T}} ہرمشی ہے۔
\item
 اس کی امتیازی اقدار تلاش کریں (آپ دیکھیں گے کہ یہ حقیقی ہیں)۔
\item
 امتیازی سمتیات تلاش کر کے ان کی معمول زنی کریں  (آپ دیکھیں گے کہ یہ معیاری عمودی ہیں)۔
\item
 اکہرا وتر ساز  قالب \عددی{\mat{S}} تیار کریں،  اور صریحاً تصدیق کریں کہ یہ \عددی{\mat{T}} کو وتری بناتا ہے۔
\item
 تصدیق کریں کہ \عددی{\mat{T}} کے لئے مقطع \عددی{\mat{T}} اور  \عددی{\Tr\mat{T}}  جو ہیں، وہی  اس کے وتری روپ کے لئے بھی ہیں۔
 \end{enumerate}
\انتہا{سوال}
\ابتدا{سوال}
درجہ ذیل ہرمشی قالب لیں۔
\begin{align*}
	\mat{T}=
	\begin{pmatrix}
		2 & i & 1\\
		-i & 2 & i\\
		1 & -i & 2
	\end{pmatrix}
\end{align*}
\begin{enumerate}[a]
\item
 اس قالب کا مقطع،  \عددی{|\mat{T}|} اور  \عددی{\Tr(\mat{T})} تلاش کریں۔
\item
 قالب \عددی{\mat{T}} کی امتیازی اقدار تلاش کریں۔ تصدیق کریں کہ ان کا مجموعہ اور حاصل ضرب  (مساوات  \حوالہ{مساوات_ضمیمہ_مقطع_آثار}  کے معنوں میں) جزو(الف) کے عین مطابق ہے۔ قالب \عددی{\mat{T}} کا  وتری روپ  لکھیں۔
\item
 قالب \عددی{\mat{T}} کے امتیازی سمتیات تلاش کریں۔ انحطاطی  حلقہ  کے اندر،   دو خطی غیر تابع امتیازی سمتیات تیار کریں (ہرمشی قالب کے لئے یہ قدم ہر صورت ممکن ہوگا،  لیکن کسی بھی اختیاری قالب کے لئے لازمی نہیں کہ ایسا ممکن ہو؛  سوال \حوالہ{سوال_ضمیمہ_غیر_وتر_پذیر}  کے ساتھ موازنہ کریں)۔ انہیں قائمہ بنائیں،  اور تصدیق کریں کہ تیسرے کے لحاظ سے دونوں قائمہ ہیں۔ تینوں امتیازی سمتیات کی معمول زنی کریں۔
\item
ا   اکہرا قالب \عددی{\mat{S}} تیار کریں جو \عددی{\mat{T}}  کی وتری سازی کرتا ہے،  اور صریحاً دکھائیں کہ،   \عددی{\mat{S}} کو استعمال کرتے ہوئے،  متشابہت تبادلہ قالب  \عددی{\mat{T}} کو موزوں  وتری روپ میں گھٹاتا ہے۔
 \end{enumerate}
\انتہا{سوال}
\ابتدا{سوال}
\ترچھا{اکہرا تبادلہ}  وہ ہے جس کے لئے \عددی{\hat{U}^\dagger\hat{U} = 1} ہو۔
\begin{enumerate}[a]
\item
 دکھائیں کہ کسی بھی سمتیات \عددی{|\alpha\rangle}، \عددی{|\beta\rangle} کے لئے \عددی{\langle\hat{U}\alpha|\hat{U}\beta\rangle = \langle\alpha|\beta\rangle} کے معنوں میں  اکہرا تبادلے اندرونی حاصل ضرب برقرار رکھتے ہیں۔
\item
 دکھائیں کہ اکہرا تبادلہ کے امتیازی اقدار کا معیار \عددی{1} ہو گا۔
\item
 دکھائیں کہ منفرد امتیازی اقدار سے متعلق اکہرا قالب کے امتیازی سمتیات قائمہ ہوں گے۔
 \end{enumerate}
\انتہا{سوال}
\ابتدا{سوال}
قوالب کے \ترچھا{تفاعلات} ان کے  ٹیلر تسلسل  اتساع دیتے ہیں؛  مثلاً درج ذیل۔ 
\begin{align}
	e^{\mat{M}}\equiv \mat{I} + \mat{M} +\frac{1}{2}\mat{M}^2 + \frac{1}{3!}\mat{M}^3 + \cdots
\end{align}
\begin{enumerate}[a]
\item
 درجہ ذیل کے لئے \عددی{\exp(\mat{M})} تلاش کریں۔
\begin{align*}
\mat{M}&=
	\begin{pmatrix}
		0 & \theta\\
		-\theta & 0
	\end{pmatrix}\quad \text{\small{(ب)}}
	& \mat{M}=
	\begin{pmatrix}
		0 & 1 & 3\\
		0 & 0 & 4\\
		0 & 0 & 0
	\end{pmatrix}\quad\text{\small{(الف)}}
\end{align*}
\item
 اگر \عددی{\mat{M}}  وتر پذیر   ہو،  تب درجہ ذیل دکھائیں۔
\begin{align}
	\abs{e^{\mat{M}}} = e^{\Tr(\mat{M})}
\end{align}
\ترچھا{تبصرہ}: اگر \عددی{\mat{M}}غیر وتر پذیر  ہو تب بھی یہ درست ہوگا،  تاہم ایسی عمومی صورت کے لئے اس کو ثابت کرنا زیادہ مشکل ہے۔
\item
 دکھائیں اگر قوالب \عددی{\mat{M}} اور \عددی{\mat{N}} مقلوبی ہوں تب درجہ ذیل ہوگا۔
\begin{align}\label{مساوات_ضمیمہ_آخری}
	e^{\mat{M}+\mat{N}} = e^{\mat{M}}e^{\mat{N}}
\end{align}
ثابت کریں  ( سادہ ترین متضاد مثال دے کر) کہ غیر مقلوبی قالب کے لئے مساوات  \حوالہ{مساوات_ضمیمہ_آخری}، عام طور،  درست نہیں۔
\item
دکھائیں  اگر \عددی{\mat{H}} ہرمشی ہو تب  \عددی{e^{i\mat{H}}} اکہرا ہوگا۔
 \end{enumerate}
\انتہا{سوال}
