
Harmony مروحتاءش کی شوڈنگر مساوات   
$ a \pm $
 کی مدد سے  درج ذیل شکل اختیار کرتی ہے ۔
\begin{equation}
 H=h \omega (a \pm a \mp + 1/2)  		
\end{equation}
اس  طرح کی مساوات میں آ پ بالاٗی علامتیں ایک ساتھ پرھتے ہو یا زیریں علامتیں ایک ساتھ پرھتے سکتے ہیں۔ ہم ایک اہم موڑ پر ہیں ۔ میں دعوہ کرتا ہوں کہ اگر  
$ \psi $
 تواناءی کی شوڈنگر مساوات کو مطمءن کرتا ہے ۔
$ ( H \psi = E \psi ) $
تب 
$a_+ \psi $
تواناءی کے شوڈنگر مساوات پر پورا اترتا ہے ،
$ ( E + \hbar \omega ) : H ( a_+ \psi ) = ( E + \hbar \omega) ( a_+ \psi ) $
ثبوت:
\begin{align}
H ( a_+ \psi ) = \hbar \omega ( a_+ a_- + \frac{1}{2}) (a_+ \psi) = \hbar \omega (a_+ a_- a_+ + ( \frac{1}{2} ) a_+) \psi
\\
  = \hbar \omega a_+ ( a_- a_+ + \frac{1}{2} ) \psi  = a_+[ \hbar \omega ( a_+ a_- + 1 + \frac{1}{2} ) \psi \\
  = a_+ ( H + \hbar \omega ) \psi = a_+ ( E + \hbar \omega ) \psi = ( E + \hbar \omega ) ( a + \psi )
\end{align}  
(میں نے دوسری لکیر میں مساوات 2٫55 استعمال کرتے ہوے 
$ a_- a_+ $
 کی جگہ 
$ a_- a_+ + 1$ 
 استعمال کیا ہے ۔ دیحان  رہے اگرچہ
$ a_+$
 اور
$  a_- $
 کی ترتیب بہت اہم ہے ۔ 
$ a \pm $
  یا کویٗ بھی مثتقل مثلاً 
$ \hbar $ 
 ، 
$ \omega$
اور E  کی ترتیب غیر اہم ہے ایک عامل ہر مستقل کے ساتھ commute کرتا ہے ۔ ) \\
 اسی طرح تواناءی 
$ ( E - \hbar \omega) $
کا حل
$ a_- \psi $
ہو گا ۔
\begin{align}
H(a_+ \psi ) = \hbar \omega ( a_- a_+  - 1 - \frac{1}{2}) (a_- \psi)    = \hbar \omega a_- ( a_+ a_-  - \frac{1}{2} ) \psi
 \\	           = a_- [ \hbar \omega ( a_- a_+  - 1 - \frac{1}{2} ) \psi ] = a_- ( H - \hbar \omega ) = a_- ( E - \hbar \omega ) \psi 
	\\	=( E - \hbar \omega ) ( a_- \psi )
\end{align}
یوں کسی ایک حل کو جانتے ہوےٗ خودکار ترکیب دریافت کر لی ہے ، جس سے بالا ٗی اور زیریں توانا ٰی کے ن ٗے حل حاصل کیے جا سکتے ہیں ۔ چونکہ 
$ a \pm $
کے زریعے ہم توانا ٗی میں اوپر چڑھ سکتے یا نیچے اتر سکتے ہیں ۔ لہذاٰ انہیں عمل سیڑھی پکارا جاتا ہے ۔
$ a_+ $
چڑھاتا عمل اور 
$ a_- $ 
اتارتا عمل ہے ۔ حالات کی سعڑھی کو شکل 2٫5 میں دکھایا گیا ہے ۔
%\newpage
ذرا رکیے ! اگر میں اترتا عمل بار بار استعمال کروں ، تب اخر کار میں 0 سے کم توانا ٗی حال تک پہنچ جاوٗں گا ۔ ( جو سوال 2٫2 میں عمومی مسلحہ کے تحت  نا ممکن ہے ۔ ) ن ٗےٗ حالات خود خار طریقے سے حاصل کرنے کی ترکیب  کسی نہ کسی جگہ  ناکامی کا شکار ہو گی ۔  ایسا کیوں ہو گا ؟
ہم جانتے ہیں کہ 
$ a_- \psi $
Schroudinger  کی  مساوات کا نیا حل ہو گا  ، لیکن ہم اس کی زمانت نہیں دے سکتے  کہ یہ معمول پر لانے کے قابل بھی ہو گا ۔ یہ 0 ہو سکتا ہے  یا اس کا مربع تکمل علامت نہیں ہو سکتا ہے ۔ حقیقت میں ان میں سے پہلا والا ہو گا اور سیڑھی کے سب سے نیچے والے ڈنڈے پر جس کو ہم 
$ (\psi_0) $
کہتے ہیں ۔ 
\begin{equation}
 a_- \psi_0 = 0 
\end{equation}  
اس کو استعمال کرتے ہوےٗ ہم 
$ \psi_0 (x) $
تعین کر سکتے ہیں
$$\frac{1}{\sqrt{2 \hbar m \omega}} ( \hbar \frac{d}{dx} + m \omega x ) \psi_0 = 0 $$
\newpage 
یا
\begin{align}
\frac{ d  \psi_0}{ dx } = \frac{ - m \omega }{ \hbar } x \psi_0
\end{align}
یہ تفرکی مساوات بآسانی حل ہو جاتی ہے ۔ 
\begin{align}
\int{\frac{ d \psi_0 }{ \psi_0 } } = \frac{ - m \omega }{ \hbar } \int{ x dx } \rightarrow ln \psi_0 =  \frac{ - m \omega }{ 2 \hbar } x^2 + constant
\end{align} 
لہذا
\begin{align}
\psi_0 (x) = A e^{\frac{ - m \omega }{ 2 \hbar } x^2}
\end{align}
ہم اس کو یہی پر معمول لاتے ہیں، 
\begin{align}
1 = (A)^2 \int_{ - \infty}^{ \infty } e^\frac{ - m \omega x^2 }{ \hbar } dx = ( A )^2  \sqrt{ \frac{ \pi \hbar }{ m \omega }}
\end{align}
لہذا
$ A^2 = \sqrt{ \frac{ m \omega }{ \pi \hbar } } $
\\
\begin{equation}
\psi_0 (x) = \frac{ m \omega }{ \pi  \hbar}^\frac{1}{4} e^{\frac{ - m \omega }{ 2 \hbar} x^2}
\end{equation}
اس حال کی توانا ٗی دریافت کرنے کی خاطر، ہم اس کو مساوات 2٫57 کی شکل میں شوڈنگر مساوات میں پر کرتے ہیں 
$$ \hbar \omega (a_+ a_- + \frac{1}{2} ) = E_0 \psi_0  $$
اب یہ جانتے ہوےٗ کہ
$ a_- \psi = 0 $
ہمیں درج ذیل حاصل ہو گا،
\begin{equation}
 E_0 = \frac{1}{2} \hbar \omega 
\end{equation}
سیڑھی کی نچلے ترین ڈنڈے پر پاوٗں رکھتے ہوےٗ جو کوانٹم ورتیش کی زمینی صورتحال ہے ، ہم بڑھاتا حامل بار بار استعمال کرتے ہوےٗ حجانی حالتیں معلوم کرتے ہیں ۔ جہاں ہر قدم پر توانا ٗی میں 
$ \hbar \omega $
کا اضافہ ہو گا -
\begin{equation}
\psi_n (x) = A_n ( a_+)^n \psi_0(x) , with E_n = ( n + \frac{1}{2}) \hbar \omega
\end{equation}
جہاں
$ A_n $
معمول پر لانے کے لیے مستقل ہے ۔ یوں 
$ \psi_0 $
پر چرھاتا عمل بار بار استعمال کرتے ہوےٗ ہم اصولی طور پر حارمونی مرتیش کے تمام ساکن حالات دریافت کر سکتے ہیں 
\newpage 
سریحاُ ایسے کیے بغیر ہم تمام  اجازتی توانایاں تعین کر پاےٗ ہیں -
\\
مثال 4.2
harmony مرتا ٗش کی پہلی حجانی حال تلاش کریں 
\\
حل ہم مساوات 1 2٫6 استعمال کریں گے 
\begin{align}
\psi_1 (x) = A_1 a + \psi_0 = \frac{ A_1 }{ \sqrt{ 2 \hbar m \omega } } ( - h \frac{d}{dx} + m \omega x) ( \frac{ m \omega}{ \pi \hbar} )^\frac{1}{4}    e^{\frac{ - m \omega }{ 2 \hbar } x^2}
\end{align}
\begin{equation}
 = A_1 ( \frac{ m \omega }{ \pi \hbar})^\frac{1}{4} \sqrt{ \frac{ 2 m \omega }{ \hbar} } x e^{\frac{ - m \omega }{ 2 \hbar } x^2}
\end{equation}
ہم اس کو قلم و کاغذ کے ساتھ  معمول پر لاتے ہیں ۔ 
\begin{align}
\int{ ( \psi )^2 dx} = ( A_1 )^2 \sqrt{ \frac{ m \omega}{ \pi \hbar } } ( \frac{ 2 m \omega }{ \hbar} ) \int_{ - \infty }^{ \infty } x^2  e^{\frac{ - m \omega }{ \hbar } x^2}  dx  =  ( A_1)^2
\end{align} 
لہذا ، 
$ A_1 = 1 $
ہو گا- 
میں 50 مرتبہ چڑھاتا عمل استعمال کر کے 
$ \psi_50 $ 
 حاصل نہیں کرنا چاہوں گا لیکن  اصولی طور پر مساوات 2٫61 اپنا کام کرتی ہے ۔ سواےٗ معمول پر لانے کے  آپ Algebric طریقے سے حجان حالات کو معمول پر بھی لا سکتے ہیں لیکن اس کے لیے بہت مہتات چہل قدمی کرنی ہو گی ۔ لہذا غور کی جیے گا
$ a \pm \psi_n $
اور
$  \psi_{n+1} $
 ایک دوسرے کے خاص متناصب ہیں ۔ 
\begin{equation} 
a_+ \psi_n = c_n  \psi_{n+1}  ,  a_- \psi_n = d_n \psi_{n-1}
\end{equation} 
لیکن تناسبی مستقل 
$ c_n $
اور 
$ d_n $
کی قیمتیں کیا ہیں ۔  پہلے یہ جان لیں کہ کسی بھی تفال 
$ f (x) $
اور 
$ g(x) $
کے لیے 
\begin{equation}
\int_{ - \infty }^{ \infty } f^\star( a \pm g ) dx = \int_{ - \infty }^{ \infty } ( a \pm f)^\star g dx
\end{equation}
خطی الجبرا کی زبان میں 
$ a \mp $
اور
$ a \pm $ 
( ایک دوسرے کے حارمشی جوڑی دار ہیں ۔)
ثبوت :
\begin{align}
\int_{ - \infty}^{ \infty } f*( a+g ) dx = \frac{1}{ \sqrt{ 2 \hbar m \omega } } \int_{ - \infty }^{ \infty } f^\star( \mp \hbar \frac{ d}{dx} + m \omega x ) g dx
\end{align}
%\newpage 
تکمل بل حسس کے ذریعے 
$ \int f^\star( \frac{ dg }{ dx }) dx $
  سے 
$ -\int ( \frac{ df }{ dx }^\star g dx $
تک حاصل ہو گا۔  جہاں حاشیا 22 میں دیے گےٗ  وجہ کی بنا سرحدی جزو 0 ہو گا۔
\begin{align}
\int_{ - \infty }^{ \infty } f^\star( a \pm g) dx = \frac{1}{ \sqrt{ 2 \hbar m \omega } } \int_{ - \infty }^{ \infty } [ ( \pm \hbar \frac{d}{dx} + m \omega x ) f]^\star g dx = \int_{ - \infty }^{ \infty } (a_\mp f )^\star g dx
\end{align}  
بالخصوص 
$$ \int_{ - \infty }^{ \infty } ( a_\pm \psi_n )^\star( a_\pm \psi_n ) dx = \int_{ - \infty }^{ \infty } ( a_+ a_\pm \psi_n )^\star \psi_n  dx $$
یہاں مساوات 2٫57 اور 2٫61 استعمال کرتے ہوےٗ 
\begin{equation}
a_+ a_-  \psi_n = n \psi_n , a_- a_+ \psi_n = ( n+1 ) \psi_n 
\end{equation}
لہذا 
\begin{align}
\int_{ - \infty }^{ \infty } ( a_+ \psi_n )^\star( a_+ \psi_n ) dx = | c_n |^2 \int_{ - \infty }^{ \infty } | \psi_{n+1} |^2  dx  = ( n+1 ) \int_{ - \infty }^{ \infty } | \psi_n |^2 dx
\\
\int_{ - \infty }^{ \infty } ( a_- \psi_n )^\star( a_- \psi_n ) dx = | d_n |^2 \int_{ - \infty }^{ \infty } | \psi_{n-1} |^2  dx  =  n \int_{ - \infty }^{ \infty } | \psi_n |^2 dx
\end{align}
چونکہ 
$ \psi_n $
اور 
$ \psi_{ n \pm 1 } $
معمول شدہ ہیں ، لہذا 
$ | c_n |^2 = n+1 $
اور
$ | d_n |^2 = n $
ہوں گے ، یوں درج ذیل ہو گا ۔
\begin{equation}
a_+ \psi_n = \sqrt{n+1} \psi_{n+1} , a_- \psi_n = \sqrt{n} \psi_{n-1}
\end{equation}
لہذا 
\begin{align}
\psi_1 = a_+ \psi_0 , \psi_2 = a_+ \psi_1 = \frac{1}{ \sqrt{ 2 } } (a_+)^2 \psi_0
\\
\psi_3 = \frac{ 1 }{ \sqrt{ 3 } } a_+ \psi_2 = \frac{ 1 }{ \sqrt{3 . 2} } ( a_+ )^3 \psi_0 ,  \psi_4 = \frac{1}{ \sqrt{4} } a_+ \psi_3 =\frac{1}{ \sqrt{4 . 3 . 2} } ( a_+)^4 \psi_0
\end{align}
صاف ظاہر ہے کہ درج ذیل ہو گا ۔ 
\begin{equation}
\psi_n = \frac{1}{ \sqrt{n!} } ( a_+ )^n \psi_0
\end{equation}
جس کے تحت مساوات 2٫61 میں معمول  لانے کا مستقل 
$ A_n = \frac{1}{ \sqrt{ n! } } $
بالخصوص 
$ A_1 = 1 $
ہو گا ۔ جو مثال 2٫4 میں ہمارے نتیجے کی تصدیق کرتا ہے ۔
\newpage 
جیسا لا مطناہی  چکور ، کی صورت میں تھا ۔ حارمونی مرتٗش کے ساکن حالات ایک دوسرے کے ساتھ  صمود ہں 
\begin{equation}
\int_{ - \infty }^{ \infty } \psi_m^\star\psi_n dx = \delta_mn
\end{equation}
ہم ایک بار مساوات 2٫65 اور دو بار مساوات 2٫64 استعمال کر کے  پہلے 
$ a_+ $
اور بعد میں 
$ a_-  $
اپنی جگہ سے ہلاتے ہوےٗ ، اس کا ثبوت پیش کر سکتے ہیں ۔
\begin{align}
\int_{ - \infty }{ \infty } \psi_m^\star(a_+ a_- )\psi dx =& n \int_{ - \infty }^{ \infty } \psi_m^\star \psi_n dx
\\
\int_{ - \infty }^{ \infty } ( a_- \psi_m  )^\star( a_- \psi_n ) dx =& \int_{ - \infty }^{ \infty } ( a_+ a_- \psi_m)^\star \psi_n dx
\\
=& m \int_{ - \infty }^{ \infty } \psi_m^\star \psi_n dx
\end{align}
جب تک 
$ m = n $
نہ ہو 
$ \int \psi_m^\star \psi_n dx  $
 صفر کے برابر ہو گا، معیاری عمودی ہونے کا مطلب  ہے کہ ہم 
$ \psi ( x , 0 ) $
کو  ساکن حالات یعنی  مساوات 2٫16 کا خطی جوڑ لکھ کر اس کے عددی صرو کو مساوات 2٫34 سے حاصل کر سکتے ہیں ۔ جہاں توانا ٗی کی پیما ٗش سے 
$ E_n $
کی قیمت کے حصول کا اہتمال 
$ | c_n |^2 $
ہو گا ۔
 


