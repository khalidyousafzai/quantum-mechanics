%missing sec 1.4 and Q1.12 onwards
\باب{تفاعل موج}\شناخت{باب_تفاعل_موج}
\حصہ{شروڈنگر مساوات} 
فرض کریں  کمیت  \عددی{m} کا ذرہ، جو  \عددی{x} محور پر رہنے کا پابند ہو، پر قوت \عددی{F(x,t)} عمل کرتی ہے۔ کلاسیکی میکانیات میں اس ذرے کا مقام \عددی{x(t)}  کسی بھی وقت \عددی{t} پر تعین کرنا درکار ہوتا ہے۔ ذرے کا مقام جاننے کے بعد ہم اس کی اسراع، سمتی رفتار  \عددی{ v = \tfrac{dx}{dt} }، معیار حرکت  \عددی{p = mv} یا حرکی توانائی  \عددی{T = \tfrac{1}{2} mv^{2} } یا کوئی اور حرکی متغیر جس میں ہم دلچسپی رکھتے ہوں تعین کر سکتے ہیں۔  سوال پیدا ہوتا ہے کہ ہم \عددی{x(t)} کیسے تعین  کریں گے۔ ہم نیوٹن کا دوسرا قانون  \عددی{F=ma}  بروئے کار لاتے ہیں۔ (بقائی نظام جو خوش قسمتی سے خوردبینی سطح پر واحد نظام ہے، میں قوت کو خفی توانائی\حاشیہد{مقناطیسی قوتوں کے لئے ایسا نہیں ہو گا لیکن یہاں ہم ان کی بات نہیں کر رہے ہیں۔ دیگر، اس کتاب میں ہم رفتار کو غیر اضافی \عددی{v\ll c} تصور کریں گے۔} پر تفرق لکھا جا سکتا ہے \عددی{F = -\tfrac{\partial V}{\partial x}}،  لہٰذا نیوٹن کا قانون  \عددی{m\tfrac{\dif{\,^2} x}{\dif t^{\,2}} = -\tfrac{\partial V}{\partial x}} لکھا جائے گا۔)   اس مساوات کے ساتھ ابتدائی معلومات، جو عموماً لمحہ \عددی{t=0} پر سمتی رفتار یا مقام ہوں گے، استعمال کرتے ہوئے ہم  \عددی{x(t)} دریافت کر سکتے ہیں۔
 
کوانٹم میکانیات اس مسئلے کو بالکل مختلف انداز سے دیکھتی  ہے۔ اب ہم ذرے کی \اصطلاح{تفاعل موج}\فرہنگ{تفاعل موج}\حاشیہب{wave function}\فرہنگ{wave function} جس کی علامت \عددی{\Psi(x,t)} ہے  کو \اصطلاح{شروڈنگر مساوات}\فرہنگ{شروڈنگر مساوات}\حاشیہب{Schrodinger equation}\فرہنگ{Schrodinger equation} حل کر کے حاصل کرتے ہیں
\begin{align}
 i \hslash \frac{\partial \Psi}{\partial t} = - \frac{\hslash^{2}}{2m} \frac{\partial \Psi^{2}}{\partial x^{2}} + V \Psi
\end{align}
جہاں  \عددی{i}  منفی ایک \عددی{(-1)} کا جذر  اور  \عددی{\hslash} پلانک  مستقل، بلکہ اصل پلانک مستقل تقسیم  \عددی{2\pi}   ہو گا: 
\begin{align}
 \hslash = \frac{h}{2\pi} = \SI{1.054572e-34}{\joule\second} 
\end{align}
شروڈنگر  مساوات نیوٹن کے دوسرے قانون کا مماثل کردار ادا کرتی ہے۔ دی گئی ابتدائی معلومات، جو عموماً \عددی{\Psi(x,0)}  ہو گا، استعمال کرتے ہوئے  شروڈنگر مساوات،  مستقبل کے تمام اوقات کے لئے، \عددی{\Psi (x,t)} تعین کرتی ہے، جیسا  کلاسیکی میکانیات میں تمام مستقبل اوقات کے لئے  قاعدہ نیوٹن  \عددی{x(t)}  تعین کرتا ہے۔ 
 
%=======================

\حصہ {شماریاتی مفہوم}
تفاعل موج حقیقت میں کیا ہوتا ہے اور یہ جانتے ہوئے آپ حقیقت میں کیا کر سکتے ہیں ، ایک ذرے کی خاصیت ہے کہ وہ ایک نقطے پر پایا جاتا ہو لیکن ایک تفاعل موج جیسا کہ اس کے نام سے ظاہر ہے فضا میں پھیلا ہوا پایا جاتا ہے۔ کسی بھی لمحے  \عددی{t} پر یہ\عددی{x} کا تفاعل ہو گا۔  ایک تفاعل ایک ذرے کی حالت کو کس طرح بیان کر پائے گا ، اس کا جواب تفاعل موج کے \اصطلاح{شماریاتی مفہوم}\فرہنگ{شماریاتی مفہوم}\حاشیہب{statistical interpretation}\فرہنگ{statistical!interpretation} پیش کر کے جناب بارن نے دیا جس کے تحت لمحہ \عددی{t} پر نقطہ \عددی{x} پر ایک ذرہ  پائے جانے کا احتمال \عددی{|\Psi(x,t)|^2} دیگا، بلکہ اس کا زیادہ درست روپ\حاشیہد{تفاعل موج ازخود مخلوط ہے لیکن \عددی{|\Psi|^2=\Psi^*\Psi} ( جہاں \عددی{\Psi^*} تفاعل موج \عددی{\Psi} کا مخلوط جوڑی دار ہے) حقیقی اور غیر منفی ہے، جیسا کہ ہونا بھی چاہیے۔}  درج ذیل ہے۔
\begin{align}
\int_a^b\abs{\Psi(x,t)}^2\dif x=
\begin{cases}
\begin{minipage}{0.25\textwidth}
\RL{
لمحہ \عددی{t} پر \عددی{a} اور \عددی{b} کے بیچ ایک ذرہ کے  پائے جانے کا احتمال
}
\end{minipage}
\end{cases}
\end{align}
احتمال \عددی{|\Psi|^2} کی ترسیم کے نیچے رقبہ کے برابر ہو گا۔شکل \حوالہء{1.2} کی تفاعل موج کے لئے  ذرہ غالباً   نقطہ \عددی{A} پر پایا جائے  گا جہاں \عددی{\abs{\Psi}^2} کی قیمت زیادہ سے زیادہ ہے جبکہ  نقطہ \عددی{B} پر ذرہ غالباً نہیں پایا جائے گا۔

شماریاتی مفہوم کی بنا اس نظریہ سے  ذرہ کے بارے میں تمام قابل حصول معلومات ، یعنی اس کا تفاعل موج، جانتے ہوئے بھی ہم  کوئی سادہ تجربہ کر کے ذرے کا مقام یا کوئی دیگر متغیر ٹھیک ٹھیک معلوم کرنے سے قاصر رہتے ہیں۔ کوانٹم میکانیات ہمیں تمام ممکنہ نتائج کے صرف شماریاتی معلومات فراہم کر سکتی ہے۔ یوں  کوانٹم میکانیات میں \اصطلاح{عدم تعین}\فرہنگ{عدم تعین}\حاشیہب{indeterminacy}\فرہنگ{indeterminacy} کا عنصر پایا جائے گا۔ کوانٹم میکانیات میں عدم تعین کا عنصر،   طبیعیات اور فلسفہ کے ماہرین کے لیے مشکلات کا سبب بنتا رہا ہے جو انہیں اس سوچ   میں مبتلا کرتی  ہے  کہ آیا یہ کائنات کی ایک حقیقت ہے یا کوانٹم میکانی نظریہ میں کمی کا نتیجہ۔

 فرض کریں کہ ہم ایک تجربہ کر کے  معلوم کرتے ہیں کہ ایک ذرہ مقام \عددی{C} پر پایا\حاشیہد{ظاہر ہے کوئی بھی پیمائشی آلہ کامل نہیں ہو سکتا ہے؛ میں صرف اتنا کہنا چاہتا ہوں کہ پیمائشی خلل کے اندر رہتے ہوئے  یہ ذرہ نقطہ \عددی{C} کے  قریب پایا گیا۔} جاتا ہے۔ اب سوال پیدا ہوتا ہے کہ پیمائش سے فوراً قبل یہ ذرہ کہاں ہوتا ہو گا؟ اس کے تین ممکنہ  جوابات ہیں جن سے آپ کو کوانٹم عدم تعین کے بارے میں مختلف طبقہ سوچ کے بارے میں علم ہو گا ۔ 

\quad \عددی{(1}
\اصطلاح{حقیقت پسند}\فرہنگ{سوچ!حقیقت پسند}\حاشیہب{realist}\فرہنگ{position!realist} سوچ:   \ترچھا{ذرہ مقام \عددی{C} پر تھا۔} یہ ایک معقول جواب ہے جس کی آئن شٹائن بھی وکالت کرتے تھے۔ اگر یہ درست ہو تب کوانٹم میکانیات ایک نا مکمل نظریہ ہو گا کیونکہ ذرہ دراصل نقطہ \عددی{C} پر ہی تھا اور کوانٹم میکانیات ہمیں یہ معلومات فراہم کرنے سے قاصر رہی۔حقیقت پسند سوچ رکھنے  والوں کے مطابق عدم تعین پن قدرتی میں نہیں پایا جاتا بلکہ  یہ ہماری لاعلمی کا نتیجہ ہے۔ ان کے تحت کسی بھی لمحے پر ذرے کا مقام غیر معین نہیں تھا بلکہ یہ صرف تجربہ کرنے والے کو معلوم نہیں تھا۔ یوں \عددی{\Psi} مکمل کہانی بیان نہیں کرتا ہے اور ذرے کو مکمل طور پر بیان کرنے کے لئے (\اصطلاح{خفیہ متغیرات}\فرہنگ{خفیہ متغیرات}\حاشیہب{hidden variables}\فرہنگ{hidden variables} کی صورت میں) مزید معلومات درکار ہوں گی۔

\quad \عددی{(2}
\اصطلاح{تقلید پسند}\فرہنگ{سوچ!تقلید پسند}\حاشیہب{orthodox}\فرہنگ{position!orthodox} سوچ: \ترچھا{ذرہ حقیقت میں کہیں پر بھی نہیں تھا۔} پیمائشی عمل ذرے کو مجبور کرتی ہے کہ وہ ایک مقام پر "کھڑا ہو جائے" ( وہ مقام \عددی{C} کو کیوں منتخب کرتا ہے،  اس بارے میں ہمیں سوال کرنے کی اجازت نہیں ہے)۔  مشاہدہ وہ عمل ہے جو نہ صرف پیمائش میں خلل پیدا کرتا ہے،  یہ پیمائشی نتیجہ بھی پیدا کرتا ہے۔ پیمائشی عمل ذرے کو مجبور کرتا ہے کہ وہ کسی مخصوص مقام کو اختیار کرے۔ ہم ذرہ کو کسی ایک مقام کو منتخب کرنے پر مجبور کرتے ہیں۔  " یہ تصور جو \اصطلاح{کوپن ہیگن مفہوم}\فرہنگ{کوپن ہیگن مفہوم}\حاشیہب{Copenhagen interpretation}\فرہنگ{Copenhagen interpretation} پکارا جاتا ہے  جناب بوہر اور ان کے ساتھیوں سے منسوب   ہے۔ ماہر طبیعیات میں یہ تصور سب سے زیادہ مقبول ہے۔اگر یہ سوچ درست ہو تب پیمائشی عمل ایک انوکھی عمل ہے جو نصف صدی سے زائد عرصہ کی بحث و مباحثوں کے بعد  بھی پراسراری کا شکار  ہے۔

\quad \عددی{(3}
\اصطلاح{انکاری}\فرہنگ{سوچ!انکاری}\حاشیہب{agnostic}\فرہنگ{position!agnostic} سوچ: \ترچھا{جواب دینے سے گریز کریں۔} یہ سوچ اتنی بیوقوفانہ نہیں  جتنی نظر آتی ہے۔ چونکہ کسی ذرے کا مقام جاننے کے لیے آپ کو ایک تجربہ کرنا ہو گا اور تجربے کے نتائج آنے تک وہ لمحہ  ماضی  بن چکا ہو گا۔ چونکہ کوئی بھی تجربہ ماضی کا حال نہیں  بتا پاتا لہٰذا اس کے بارے میں بات کرنا بے معنی ہے۔ 

  \سن{1964}  تک تینوں طبقہ سوچ کے حامی پائے جاتے تھے البتہ اس سال  جناب جان بل نے  ثابت کیا کہ تجربہ سے قبل   ذرہ کا مقام ٹھیک  ہونے یا   نہ  ہونے کا  تجربہ پر قابل مشاہدہ اثر پایا جاتا ہے (ظاہر ہے کہ ہمیں یہ مقام معلوم نہیں ہو گا)۔ اس ثبوت نے انکاری سوچ کو غلط ثابت کیا۔ اب حقیقت پسند اور تقلید پسند سوچ کے بیچ   فیصلہ کرنا باقی ہے جو تجربہ کر کے کیا جا سکتا ہے۔ اس پر کتاب کے آخر میں بات کی جائے گی جب آپ کی علمی سوچ اتنی بڑھ چکی ہو گی کہ آپ کو  جناب جان بل کی دلیل سمجھ آ سکے گی۔ یہاں اتنا بتانا کافی ہو گا کہ تجربات  جان بل کی تقلید پسند سوچ کی درستگی کی تصدیق کرتے ہیں\حاشیہد{یہ فقرہ کچھ زیادہ سخت ہے۔چند نظریاتی اور تجرباتی مسائل باقی ہیں جن میں سے چند پر میں بعد میں تبصرہ کروں گا۔ایسے غیر مقامی خفیہ متغیرات کے نظریات  اور دیگر تشکیلات مثلاً \اصطلاح{متعدد دنیا} تشریح جو ان تینوں سوچ کے ساتھ مطابقت نہیں رکھتے ہیں۔ بہر حال، اب کے لئے بہتر ہے کہ ہم کوانٹم نظریہ کی بنیاد سیکھیں اور بعد میں اس طرح کی مسائل کے بارے میں فکر کریں۔}۔جیسا جھیل میں  موج ایک نقطہ پر نہیں پائی جاتی،  یوں  قبل از تجربہ  ایک ذرہ ٹھیک کسی  ایک مقام پر  نہیں پایا جاتا ہے ۔  پیمائشی عمل ذرے کو ایک مخصوص عدد اختیار کرنے پر مجبور کرتے ہوئے  ایک مخصوص نتیجہ پیدا کرتی ہے ۔ یہ نتیجہ تفاعل موج کی مسلط کردہ شماریاتی وزن کی پابندی کرتا ہے۔

 کیا ایک پیمائش کے فوراً بعد دوسری پیمائش وہی مقام \عددی{C} دے گی یا نیا مقام حاصل ہو گا؟ اس کے جواب پر سب متفق ہیں۔ ایک تجربے کے فوراً  بعد (اسی ذرہ پر) دوسرا تجربہ لازماً وہی مقام دوبارہ دے گا۔  حقیقت میں اگر دوسرا تجربہ مقام \عددی{C} کی تصدیق نہ کرے تب یہ ثابت کرنا نہایت مشکل ہو گا کے پہلے تجربہ میں مقام \عددی{C} ہی حاصل ہوا تھا۔ تقلید پسند اس کو کس طرح دیکھتا ہے کہ دوسری پیمائش ہر صورت \عددی{C} قیمت دے گی؟ ظاہری طور پر پہلی پیمائش تفاعل موج میں  ایسی بنیادی تبدیلی پیدا کرتی ہے کہ تفاعل موج   \عددی{C} پر نوکیلی صورت اختیار کرتی ہے  جیسا  شکل \حوالہء{1.3} میں دکھایا گیا ہے۔ ہم کہتے ہیں کہ پیمائش کا عمل تفاعل موج کو نقطہ \عددی{C} پر \اصطلاح{گر کر}\فرہنگ{گر کر}\حاشیہب{collapses}\فرہنگ{collapses}  نوکیلی صورت اختیار کرنے پر مجبور کرتی ہے  (جس کے بعد تفاعل موج شروڈنگر مساوات کے تحت جلد  پھیل جائے گی لہٰذا دوسری پیمائش جلد کرنی ضروری ہے)۔ اس طرح دو بہت مختلف طبعی اعمال پائے جاتے ہیں۔ پہلی میں  تفاعل موج وقت کے ساتھ شروڈنگر مساوات کے تحت ارتقا پاتا ہے، اور دوسری جس میں  پیمائش \عددی{\Psi} کو فوراً ایک جگہ غیر استمراری طور پر گرنے پر مجبور کرتی ہے۔ 

%=============================

\حصہ{احتمال} 
\جزوحصہ{ غیر مسلسل متغیرات}\شناخت{جزوحصہ_غیر_مسلسل_متغیرات}
چونکہ کوانٹم میکانیات کی شماریاتی تشریح کی جاتی ہے  لہٰذا  اس میں احتمال کلیدی کردار ادا کرتا ہے۔ اسی لیے میں اصل موضوع سے ہٹ کر  نظریہ احتمال پر تبصرہ کرتا ہوں۔ ہمیں چند نئی علامتیں  اور اصطلاحات سیکھنا ہو گا جنہیں میں ایک سادہ مثال کی مدد سے  واضح کرتا ہوں۔ 
فرض کریں ایک کمرہ   میں  \( 14 \)  حضرات موجود ہیں جن  کی  عمریں درج ذیل ہیں۔ 
\begin{itemize}
\item
 \( 14 \) سال عمر کا ایک شخص، 
\item
 \( 15 \) سال عمر کا ایک شخص، 
\item
 \( 16 \) سال عمر کے تین اشخاص، 
\item
 \( 22 \) سال عمر کے دو اشخاص، 
\item
 \( 24 \) سال عمر کے دو اشخاص، 
\item
 اور  \( 25 \) سال عمر کے پانچ  اشخاص۔
\end{itemize} 
اگر  \( j \)  عمر  کے لوگوں کی  تعداد  کو\( N(j) \) لکھا جائے تب درج ذیل ہو گا۔
\begin{align*}
N(14) &= 1 \\
N(15) &= 1 \\
N(16) &= 3 \\
N(22) &= 2 \\
N(24) &= 2 \\
N(25) &= 5 \\
\end{align*}
جبکہ \عددی{N(17)}، مثال کے طور پر، صفر ہو گا۔کمرہ میں لوگوں کی کل تعداد درج ذیل ہو گی۔ 
\begin{align}
N = \sum_{j=0}^{\infty} N(j)
\end{align}

(اس مثال میں ظاہر ہے کہ \( N=14 \) ہو گا۔)  شکل \حوالہء{1.4}میں اس مواد کی مستطیلی ترسیم دکھائی  گئی ہے۔ اس تقسیم کے بارے میں  درج ذیل چند ممکنہ سوالات ہیں۔ 

\ترچھا{سوال 1}\quad
اگر ہم اس گروہ سے بلا منصوبہ ایک شخص منتخب کریں تو اس بات کا کیا \اصطلاح{احتمال} ہو گا کہ اس شخص کی عمر 15 سال ہو؟
\ترچھا{جواب :}\quad
چودہ میں ایک امکان  ہو گا کیونکہ کل  \عددی{14} اشخاص ہیں اور ہر ایک شخص کی انتخاب کا امکان ایک جیسا ہے لہٰذا ایسا ہونے کا احتمال چودہ میں سے ایک ہو گا۔  اگر \عددی{j}  عمر کا شخص کے انتخاب کا احتمال\( P(j) \) ہو تب \عددی{P(14)=1/14}، \عددی{P(15)=1/14}، \عددی{P(16)=3/14}، وغیرہ ہو گا۔اس کا عمومی کلیہ درج ذیل ہو گا۔
\begin{align}
P(j) = \frac{N(j)}{N} 
\end{align}
دھیان رہے کی چودہ یا پندرہ سال عمر کا شخص کے انتخاب کا احتمال ان دونوں کی انفرادی احتمال کا مجموعہ یعنی \( \frac{1}{7} \) ہو گا۔بالخصوص تمام احتمال کا مجموعہ اکائی \عددی{(1)} کے برابر ہو گا چونکہ آپ کسی نہ کسی عمر کے شخص کو ضرور منتخب کر پائیں گے۔ 
\begin{align}\label{مساوات_تفاعل_موج_کل_احتمال_اکائی}
\sum_{j=0}^{\infty} P(j) = 1
\end{align}
\ترچھا{سوال 2}\quad
کونسا عمر بلندتر احتمال رکھتا ہے؟
\ترچھا{جواب:} \quad 
\( 25 \)،  چونکہ پانچ اشخاص اتنی عمر رکھتے ہیں جبکہ اس کے بعد ایک جیسی عمر کے لوگوں کی اگلی زیادہ  تعداد تین ہے۔ عموماً سب سے زیادہ احتمال کا \عددی{j} وہی \( j \) ہو گا جس کے لئے\( P(j) \) کی قیمت زیادہ سے زیادہ ہو۔ 

\ترچھا{سوال 3} \اصطلاح{وسطانیہ}\فرہنگ{وسطانیہ}\حاشیہب{median}\فرہنگ{median} عمر کیا ہے؟  \ترچھا{جواب:}\quad
چونکہ \عددی{7} لوگوں کی عمر \عددی{23} سے کم اور \عددی{7} لوگوں  کی  عمر \عددی{23} سے زیادہ ہے۔ لہٰذا  جواب \( 23 \) ہو گا۔ (عمومی طور پر وسطانیہ \( j \) کی وہ قیمت ہو گی جس سے زیادہ  اور جس سے کم قیمت کے نتائج کے احتمال  ایک دوسرے جیسے ہوں۔)

\ترچھا{سوال 4}\quad
ان  کی  \اصطلاح{اوسط}\فرہنگ{اوسط}\حاشیہب{mean}\فرہنگ{mean} عمر کتنی ہے ؟\ترچھا{جواب:} 
\[ \frac{(14)+(15)+3(16)+2(22)+2(24)+5(25)}{14} = \frac{294}{14}=21 \]
عمومی طور پر \( j \) کی اوسط قیمت جس کو ہم  \( \langle j \rangle \) لکھتے ہیں، درج ذیل ہو گی۔ 
\begin{align}\label{مساوات_تفاعل_موج_اوسط}
\langle j \rangle = \frac{\sum  j N(j)}{N} = \sum_{j=0}^{\infty} jP(j) 
\end{align}
دھیان رہے کہ عین ممکن ہے کہ گروہ میں کسی کی بھی عمر گروہ کی اوسط یا وسطانیہ  کے برابر نہ ہو۔ مثال کے طور پر،  اس مثال میں کسی کی عمر بھی  \( 21 \) یا \( 23 \) سال نہیں ہے۔ کوانٹم میکانیات میں ہم عموماً اوسط قیمت میں دلچسپی رکھتے ہیں جس کو \اصطلاح{توقعاتی قیمت}\فرہنگ{توقعاتی!قیمت}\حاشیہب{expectation value}\فرہنگ{expectation!value}  کا نام دیا گیا ہے۔ 

\ترچھا{سوال 5}\quad عمروں کے مربعوں کا اوسط کیا ہو گا ؟ \ترچھا{جواب:} \quad  آپ \( \frac{1}{14} \) احتمال سے   \( 14^{2} = 196 \)  حاصل کر سکتے ہیں، یا \( \frac{1}{14} \) احتمال سے \( 15^{2} = 225 \)، یا \( \frac{3}{14} \) احتمال سے \( 16^{2} = 256 \) حاصل کر سکتے ہیں، وغیرہ وغیرہ۔یوں ان کے مربعوں کا اوسط درج ذیل ہو گا۔ 
\begin{align}\label{مساوات_تفاعل_موج_اوسط_مربع}
\langle j^{2} \rangle  = \sum_{j=0}^{\infty} j^{2} P(j) 
\end{align}
عمومی طور پر \( j \) کے کسی بھی تفاعل کی اوسط قیمت درج ذیل ہو گی۔ 
\begin{align}
 \langle f(j) \rangle = \sum_{j=0}^{\infty} f(j) P(j) 
\end{align}
(مساوات \حوالہ{مساوات_تفاعل_موج_کل_احتمال_اکائی}، \حوالہ{مساوات_تفاعل_موج_اوسط} اور \حوالہ{مساوات_تفاعل_موج_اوسط_مربع} اس  کی خصوصی صورتیں ہیں۔) دھیان رہے کہ مربع کا اوسط \عددی{\langle j^2\rangle} عموماً اوسط کے مربع \عددی{\langle j \rangle^2} کے برابر نہیں ہو گا۔ مثال کے طور پر اگر ایک کمرہ میں صرف دو بچے ہوں جنکی عمریں\( 1 \) اور\( 3 \) ہو تب 
\( \langle x^{2} \rangle = 5 \)  جبکہ
\( \langle x\rangle ^{2} = 4 \)  ہو گا۔

شکل \حوالہء{1.5} کی شکل و صورتوں میں واضح فرق پایا جاتا ہے اگرچہ ان کی اوسط قیمت، وسطانیہ، بلندتر قیمت احتمال اور اجزاء کی تعداد ایک جیسے ہیں۔ ان میں پہلی شکل اوسط کے قریب نوکیلی صورت رکھتی ہے جبکہ دوسری افقی چوڑی  صورت رکھتی ہے۔ (مثال کے طور پر کسی بڑے شہر میں ایک جماعت میں طلبہ  کی تعداد پہلی شکل  مانند ہو گی جبکہ دھاتی علاقہ  میں ایک ہی کمرہ پر مبنی مکتب  میں بچوں کی تعداد دوسری شکل ظاہر کرے گی۔) ہمیں اوسط قیمت کے لحاظ سے، کسی بھی مقدار کے تقسیم کا پھیلاو، عددی صورت میں درکار ہو گا۔ اس کا ایک سیدھا طریقہ یہ ہو سکتا ہے کہ ہم ہر انفرادی جزو کی قیمت اور اوسط قیمت کا فرق
\begin{align}
\Delta j = j-\langle j \rangle 
\end{align}
لے کر  تمام \( \Delta j \) کی اوسط تلاش کریں۔ ایسا کرنے سے یہ مسئلہ پیش آتا ہے کہ ان کا جواب صفر ہو گا چونکہ اوسط کی تعریف کے تحت اوسط سے زیادہ اور اوسط سے کم قیمتیں ایک برابر ہوں گی۔ 
\begin{align*}
\langle \Delta j \rangle &= \sum \left( j -  \langle j \rangle  \right) P(j) = \sum jP(j) - \langle j \rangle \sum P(j) \\
 &= \langle j \rangle - \langle j \rangle = 0
\end{align*}
(چونکہ \( \langle j \rangle \)   مستقل ہے لہٰذا اس کو مجموعہ کی علامت سے باہر لے جایا جا سکتا ہے۔) اس مسئلہ سے چھٹکارا حاصل کرنے کی خاطر  آپ  \( \Delta j \)    کی مطلق قیمتوں کا اوسط لے سکتے ہیں لیکن \( \Delta j \)    کی مطلق قیمتوں  کے ساتھ کام کرنا مشکلات پیدا کرتا ہے۔ اس کی بجائے، منفی علامت سے نجات حاصل کرنے کی خاطر، ہم مربع لینے کے بعد اوسط حاصل کرتے ہیں۔
\begin{align} \label{مساوات_تفاعل_موج_تعریف_معیاری_احتمال_الف}
 \sigma^{2} \equiv \langle \left( \Delta j \right)^{2} \rangle 
\end{align}
اس قیمت کو تقسیم کی \اصطلاح{تغیریت}\فرہنگ{تغیریت}\حاشیہب{variance}\فرہنگ{variance} کہتے ہیں جبکہ تغیریت کا جذر \( \sigma \)  کو \اصطلاح{معیاری انحراف}\فرہنگ{معیاری انحراف}\حاشیہب{standard deviation}\فرہنگ{standard deviation} کہتے ہیں۔ روایتی طور پر  \عددی{\sigma} کو اوسط  \( \langle j \rangle \) کے گرد پھیلاو کی پیمائش  مانا جاتا ہے۔

 ہم تغیریت کا ایک چھوٹا مسئلہ پیش کرتے ہیں۔ 
\begin{align*}
\sigma^{2} &= \langle ( \Delta j )^{2} \rangle = \sum ( \Delta j )^{2} P(j) = \sum (j- \langle j \rangle )^{2} P(j) \\
&= \sum (j^{2} -2j \langle j \rangle + \langle j \rangle ^{2} ) P(j) \\ 
&= \sum j^{2} P(j) -2 \langle j \rangle \sum jP(j) + \langle j \rangle ^{2} \sum P(j) \\ 
&= \langle j^{2}\rangle -2\langle j \rangle \langle j \rangle + \langle j \rangle ^{2} = \langle j^{2} \rangle - \langle j \rangle ^{2}
\end{align*}
اس کا جذر لے کر ہم معیاری انحراف کو درج ذیل لکھ سکتے ہیں۔ 
\begin{align}\label{مساوات_تفاعل_موج_تعریف_معیاری_احتمال_ب}
 \sigma = \sqrt{\langle j^{2} \rangle - \langle j \rangle ^{2}} 
\end{align}


عملی استعمال میں \( \sigma \)  اس کلیے سے بہت جلد حاصل ہو گا۔ آپ \( \langle j^{2} \rangle \) اور \( \langle j \rangle ^{2} \) معلوم کر کہ ان کے فرق کا جذر لیں گے۔ جیسا آپکو یاد ہو گا میں نے ذکر کیا \( \langle j^{2} \rangle \) اور \( \langle j \rangle ^{2} \) عموماً ایک دوسرے کے برابر نہیں ہوں گے۔ جیسا آپ مساوات \حوالہ{مساوات_تفاعل_موج_تعریف_معیاری_احتمال_الف} سے دیکھ سکتے ہیں    \( \sigma ^{2} \) غیر منفی ہو گا لہٰذا  مساوات \حوالہ{مساوات_تفاعل_موج_تعریف_معیاری_احتمال_ب} کے تحت  درج ذیل ہو گا
\begin{align}
 \langle j^2 \rangle \geq \langle j \rangle ^{2}
\end{align}
اور یہ دونوں صرف اس صورت برابر ہو سکتے ہیں جب \( \sigma = 0 \) ہو،   جو تب ممکن ہو گا جب تقسیم میں کوئی پھیلاو نہ پایا جاتا ہو یعنی ہر جزو ایک ہی قیمت کا ہو۔ 


%=========================
\جزوحصہ{استمراری متغیرات}
اب تک ہم غیر مسلسل متغیرات کی بات کرتے آ رہے ہیں  جن کی قیمتیں الگ تھلگ ہوتی ہیں۔ (گزشتہ مثال میں ہم نے افراد کی عمروں کی بات کی جن کو سالوں میں ناپا جاتا ہے  لہٰذا  \عددی{j} عدد صحیح تھا۔)  تاہم اس  کو آسانی سے   استمراری تقسیم  تک وسعت دی جا سکتی ہے۔ اگر میں گلی میں بلا منصوبہ ایک شخص کا انتخاب کر کے اس کی عمر پوچھوں تو اس کا احتمال \ترچھا{صفر} ہو گا کہ اس کی عمر ٹھیک \عددی{16} سال \عددی{4} گھنٹے، \عددی{27} منٹ اور \عددی{3.37524} سیکنڈ ہو۔ یہاں  اس کی عمر کا \عددی{16} اور \عددی{17} سال کے بیچ ہونے کے احتمال کی بات کرنا معقول ہو گا۔بہت کم وقفے کی صورت میں احتمال \ترچھا{وقفے کی لمبائی کے راست متناسب} ہو گا۔ مثال کے طور پر \عددی{16} سال اور \عددی{16} سال جمع دو دنوں کے بیچ عمر  کا احتمال \عددی{16} سال اور \عددی{16} سال جمع ایک دن کے بیچ عمر کے احتمال کا دگنا ہو گا۔(ماسوائے ایسی صورت میں جب  \عددی{16} سال قبل عین اسی دن  کسی وجہ سے بہت زیادہ بچے  پیدا ہوئے ہوں۔ ایسی صورت میں اس قاعدہ کی اطلاق کی نقطہ نظر سے   ایک یا دو دن کا وقفہ بہت لمبا وقفہ ہے۔ اگر زیادہ بچوں کی پیدائش کا دورانیہ چھ گھٹے پر مشتمل ہو تب ہم ایک سیکنڈ یا، زیادہ محفوظ طرف رہنے کی خاطر، اس سے بھی کم دورانیے کا وقفہ لیں گے۔ تکنیکی طور پر ہم لامتناہی چھوٹے وقفہ کی بات کر رہے ہیں۔)  اس طرح درج ذیل لکھا جا سکتا ہے۔
\begin{align}\label{مساوات_تفاعل_موج_وقفہ_پر_احتمال}
 \rho(x)dx =\begin{cases}
\begin{minipage}{0.30\textwidth}
\RL{
بلا منصوبہ منتخب کئے گئے رکن  کا \عددی{x} اور \عددی{(x+\dif x)} کے بیچ پائے جانے کا احتمال
}
\end{minipage}
\end{cases} 
\end{align}
اس مساوات میں تناسبی مستقل $ \rho(x) $ \اصطلاح{کثافت احتمال}\فرہنگ{کثافت!احتمال}\حاشیہب{probability density}\فرہنگ{probability!density} کہلاتا ہے۔  متناہی وقفہ \عددی{a} تا \عددی{b} کے بیچ \عددی{x}  پایا جانے کا احتمال $ \rho(x) $ کا تکمل دے گا:
 \begin{align}
 P_{ab}=\int_a^b \rho(x)\dif x 
 \end{align}
اور غیر مسلسل تقسیم کے لئے اخذ کردہ قواعد درج ذیل روپ اختیار کریں گے:
\begin{align}
1&=\int_{-\infty}^{\infty} \rho(x)\dif x,  \label{مساوات_تفاعل_موج_کل_احتمال_اکائی_ہو_گا}\\
<x>& =\int_{-\infty}^{\infty} x\rho(x)\dif x, \label{مساوات_تفاعل_موج_اوسط_تکمل}\\
\langle f(x)\rangle&=\int_{-\infty}^{+\infty} f(x)\rho(x)\dif x,\\
 \sigma^2&\equiv \langle(\Delta x)^2\rangle = \langle x^2\rangle-\langle x\rangle^2 
\end{align}

\ابتدا{مثال}\شناخت{مثال_تفاعل_موج_چٹان-سے_گرتا_پتھر}
ایک چٹان جس کی اونچائی \عددی{h} ہو سے ایک پتھر کو نیچے گرنے دیا جاتا ہے۔ گرتے ہوئے  پتھر کی بلا واسطہ وقتی فاصلوں پر دس لاکھ تصاویر کھینچے جاتے ہیں۔ ہر تصویر پر طے شدہ فاصلہ ناپا جاتا ہے۔ ان تمام فاصلوں کی اوسط قیمت کیا ہو گی؟ یعنی طے شدہ فاصلوں کا وقتی اوسط کیا ہو گا؟ 

حل: \quad
پتھر ساکن حال سے بتدریج بڑھتی ہوئی رفتار سے نیچے گرتا ہے۔ یہ چٹان کے  بالائی سر کے قریب زیادہ وقت گزارتا ہے لہٰذا  ہم توقع کرتے ہیں کہ
 فاصلہ \عددی{\frac{h}{2}} سے کم ہو گا۔ ہوائی رگڑ کو نظر انداز کرتے ہوئے،  لمحہ \عددی{t} پر فاصلہ \عددی{x} درج ذیل ہو گا۔
\begin{align*}
 x(t) = \frac{1}{2} gt^2 
\end{align*}
اس کی سمتی رفتار  \عددی{\tfrac{dx}{dt}=gt} ہو گی اور پرواز کا  دورانیہ \عددی{T=\sqrt{2h/g}} ہو گا۔ وقفہ  \عددی{\dif t} میں تصویر کھینچنے کا احتمال \عددی{\tfrac{\dif t}{T}} ہو گا۔ یوں اس کا احتمال کہ ایک تصویر  مطابقتی سعت \عددی{\dif x} میں فاصلہ دے درج ذیل ہو گا:
 \begin{align}
 \frac{\dif t}{T} = \frac{\dif x}{gt}\sqrt{\frac{g}{2h}} = \frac{1}{2\sqrt{hx}}\dif x
\end{align}   
 ظاہر ہے کہ کثافت احتمال  (مساوات \حوالہ{مساوات_تفاعل_موج_وقفہ_پر_احتمال}) درج ذیل ہو گا۔
 \begin{align}
 \rho(x)&=\frac{1}{2\sqrt{hx}} && (0\leq x\leq h)
 \end{align}
 (اس وقفہ کے باہر کثافت  احتمال صفر ہو گا۔) 

ہم مساوات \حوالہ{مساوات_تفاعل_موج_کل_احتمال_اکائی_ہو_گا} استعمال کر کے اس نتیجہ کی تصدیق کر سکتے ہیں۔
 \begin{align}
 \int_0^h \frac{1}{2\sqrt{hx}}\dif x = \frac{1}{2\sqrt{h}}\left .(2x^{\frac{1}{2}})\right\vert_0^h =1
 \end{align}
 مساوات \حوالہ{مساوات_تفاعل_موج_اوسط_تکمل} سے اوسط فاصلہ تلاش  کرتے ہیں
 \begin{align}
 \langle x\rangle =\int_0^h x\frac{1}{2\sqrt{hx}}\dif x= \frac{1}{2\sqrt{h}} \left . \big(\frac{2}{3}x^{\frac{3}{2}}\big)\right\vert_0^h = \frac{h}{3}
 \end{align}
 جو  \عددی{ \frac{h}{2}}   سے کچھ کم ہے  جیسا کہ ہم توقع کرتے ہیں۔

 شکل \حوالہء{1.6} میں   \عددی{ \rho(x)} کی ترسیم دکھائی گئی ہے۔آپ دیکھ سکتے ہیں کہ کثافت احتمال ازخود لامتناہی ہو سکتا ہے جبکہ  احتمال (یعنی \عددی{\rho} کا تکمل) لازمناً  متناہی (بلکہ  \عددی{1} یا \عددی{1} سے کم ہو گا)۔
\انتہا{مثال}
%===================
\ابتدا{سوال}
 حصہ \حوالہ{جزوحصہ_غیر_مسلسل_متغیرات} میں اشخاص کی عمروں کی تقسیم کے لیے درج ذیل کریں۔ 
\begin{enumerate}[a.]
\item
اوسط کا مربع \عددی{\langle i\rangle^2} اور مربع کا اوسط \عددی{\langle j^2\rangle} تلاش کریں۔ 
\item
 ہر \عددی{j}  کے لیے \عددی{\Delta j} دریافت کریں اور مساوات \حوالہ{مساوات_تفاعل_موج_تعریف_معیاری_احتمال_الف} استعمال کرتے ہوئے  معیاری انحراف دریافت کریں۔
\item
 جزو ا اور ب کے نتائج استعمال کرتے ہوئے مساوات  \حوالہ{مساوات_تفاعل_موج_تعریف_معیاری_احتمال_ب} کی تصدیق کریں۔ 
\end{enumerate}
\انتہا{سوال}
%==================
\ابتدا{سوال}
\begin{enumerate}[a.]
\item
 مثال \حوالہ{مثال_تفاعل_موج_چٹان-سے_گرتا_پتھر} کی تقسیم کے لیے معیاری انحراف تلاش کریں۔
\item
بلا واسطہ منتخب تصویر میں  اوسط فاصلے سے، ایک معیاری انحراف کے برابر، دور فاصلہ \عددی{x} پائے جانے کا احتمال کیا ہو گا؟
\end{enumerate}
\انتہا{سوال}
%==================
\ابتدا{سوال}
درج ذیل گاوسی تقسیم پر غور کریں  جہاں    \عددی{A}، \عددی{a} اور \عددی{\lambda} مستقل ہیں۔  
\begin{align*}
\rho(x)=Ae^{-\lambda(x-a)^2}
\end{align*}
   (ضرورت کے پیش آپ تکمل کسی جدول سے دیکھ سکتے ہیں۔)
\begin{enumerate}[a.]
\item
مساوات\حوالہ{مساوات_تفاعل_موج_کل_احتمال_اکائی_ہو_گا} استعمال کرتے ہوئے  \عددی{A} کی قیمت تعین کریں۔
\item
اوسط \عددی{\langle x \rangle}، مربعی اوسط \عددی{\langle x^2\rangle} اور معیاری انحراف \عددی{\sigma} تلاش کریں۔
\item
\عددی{\rho(x)} کی ترسیم کا خاکہ بنائیں۔
\end{enumerate}
\انتہا{سوال}

%====================================
% section 1.4 is missing altogeather
\حصہ{معیار حرکت}
 حال \عددی{\Psi} میں پائے جانے والے ذرہ  کے مقام \عددی{x} کی توقعاتی قیمت درج ذیل ہو گی۔
\begin{align}
\langle x\rangle=\int_{-\infty}^{+\infty}x|\Psi (x,t)|^{2}\dif{x}
\end{align}
اس کا مطلب کیا ہے؟ اس کا  ہرگز یہ مطلب نہیں ہے کہ اگر آپ ایک ہی ذرے کا مقام جاننے کے لیے بار بار پیمائش کریں تو آپ کو نتائج کی اوسط قیمت  \عددی{\int x|\Psi|^{2}\dif{x}} حاصل ہو گی۔ اس کے برعکس: پہلی پیمائش (جس کا نتیجہ غیر متعیین ہے) تفاعل موج کو اس قیمت پر بیٹھنے پر مجبور کرے گا جو پیمائش سے حاصل ہوئی ہو، اس کے بعد (اگر جلد) دوسری پیمائش کی جائے تو وہی نتیجہ دوبارہ حاصل ہو گا۔ حقیقت میں 
\عددی{\langle x\rangle} ان ذرات کی  پیمائشوں کی اوسط ہو گی جو یکساں حال \عددی{\Psi} میں پائے جاتے ہوں۔ یوں یا تو آپ ہر پیمائش کے بعد کسی طرح اس ذرہ کو دوبارہ ابتدائی حال \عددی{\Psi} میں لائیں گے اور یا آپ متعدد ذرات کی \اصطلاح{سگرا}\فرہنگ{سگرا}\حاشیہب{ensemble}\فرہنگ{ensemble} کو ایک ہی حال \عددی{\Psi}  میں لا کر تمام کے مقام کی  پیمائش کریں گے۔ ان نتائج کا اوسط \عددی{\langle x\rangle} ہو گا۔ (میں اس کی تصوراتی شکل یوں پیش کرتا ہوں کہ ایک الماری میں  قطار پر شیشہ کی بوتلیں کھڑی ہیں اور ہر بوتل میں ایک ذرہ پایا جاتا ہے ۔ تمام ذرات ایک جیسے  (بوتل کے وسط کے لحاظ سے) حال \عددی{\Psi} میں پائے جاتے ہیں۔ ہر بوتل کے قریب ایک طالب علم  کھڑا ہے جس کے ہاتھ میں ایک فیتا  ہے۔ جب اشارہ  دیا جائے تو تمام طلبہ اپنے اپنے ذرہ  کا مقام ناپتے ہیں۔ ان نتائج کا مستطیلی ترسیم تقریباً \عددی{|\Psi|^2} دیگا جبکہ ان کی اوسط قیمت تقریباً \عددی{\langle x \rangle} ہو گی۔ (چونکہ ہم متناہی تعداد کے ذرات پر تجربہ  کر رہے ہیں لہٰذا  یہ توقع نہیں کیا جا سکتا ہے کہ جوابات بالکل حاصل ہوں گے لیکن بوتلوں  کی تعداد بڑھانے سے نتائج نظریاتی جوابات کے  زیادہ قریب حاصل ہوں گے۔)) مختصراً توقعاتی قیمت ذرات کے  سگرا پر کیے جانے والے تجربات کی اوسط قیمت ہو گی نہ کہ کسی ایک ذرہ پر بار بار تجربات کی نتائج کی اوسط قیمت۔

 چونکہ \عددی{\Psi} وقت اور مقام کا تابع ہے لہٰذا وقت گزرنے کا ساتھ ساتھ \عددی{\langle x\rangle} تبدیل ہو گا۔ ہمیں اس کی سمتی رفتار جاننے میں دلچسپی ہو سکتی ہے۔ مساوات \حوالہء{1.25} اور \حوالہء{1.28} سے درج ذیل لکھا جا سکتا ہے۔
\begin{align}\label{مساوات_تفاعل_موج_رفتار_توقعاتی_الف}
\frac{\dif \,\langle x \rangle}{\dif{t}}=\int x\frac{\partial}{\partial{t}}|\Psi|^{2}\dif{x}=\frac{i\hslash}{2m}\int x\frac{\partial}{\partial{x}}\big (\Psi ^*\frac{\partial{\Psi}}{\partial{x}}-\frac{\partial{\Psi^ *}}{\partial{x}}\Psi \big)\dif{x}
\end{align}

تکمل بالحصص  کی مدد سے اس فقرے کی سادہ  صورت حاصل کرتے ہیں۔
\begin{align}
\frac{\dif\, \langle x \rangle}{\dif{t}}=-\frac{i\hslash}{2m}\int\big (\Psi^*\frac{\partial{\Psi}}{\partial{x}}-\frac{\partial{\Psi^*}}{\partial{x}}\Psi \big)\dif{x}
\end{align}
(میں نے یہاں \عددی{\tfrac{\partial{x}}{\partial{x}}=1} استعمال کیا اور سرحدی جزو کو اس بنا رد کیا کہ \عددی{(\pm)} لامتناہی پر \عددی{\Psi} کی قیمت  \عددی{0} ہو گی۔ دوسرے جزو پر  دوبارہ تکمل بالحصص لاگو کرتے ہیں۔
\begin{align}\label{مساوات_تفاعل_موج_توقعاتی_قیمت_رفتار}
\frac{\dif \,\langle x \rangle}{\dif{t}}=-\frac{i\hslash}{m}\int\Psi^*\frac{\partial{\Psi}}{\partial{x}} \dif{x}
\end{align}

اس نتیجے سے ہم کیا مطلب حاصل کر سکتے ہیں؟ یہ  \عددی{x} کی توقعاتی قیمت کی سمتی رفتار ہے نا کہ  ذرہ کی سمتی رفتار۔  ابھی تک ہم جو کچھ دیکھ چکے ہیں اس سے ذرہ  کی سمتی رفتار دریافت نہیں کی جا سکتی ہے۔ کوانٹم میکانیات میں ذرہ  کی  سمتی رفتار کا مفہوم واضح  نہیں ہے ۔ اگر پیمائش سے قبل ایک ذرے کا مقام  غیر تعیین  ہو تب اس کی سمتی رفتار بھی غیر تعیین ہو گی۔ ہم ایک مخصوص قیمت کا نتیجہ حاصل کرنے کے احتمال کی صرف بات کر سکتے ہیں ۔ ہم \عددی{\Psi} جانتے ہوئے کثافت احتمال کی بناوٹ کرنا باب \حوالہ{باب_قواعد_و_ضوابط} میں دیکھیں گے۔ اب کے لیے صرف اتنا جاننا کافی ہے کہ  \ترچھا{سمتی رفتار کی توقعاتی قیمت ذرہ کے مقام کی توقعاتی قیمت کا تفرق ہو گا}۔
\begin{align}\label{مساوات_تفاعل_موج_تعریف_رفتار}
\langle v \rangle=\frac{\dif\langle x \rangle}{\dif{t}}
\end{align}
مساوات \حوالہ{مساوات_تفاعل_موج_توقعاتی_قیمت_رفتار} ہمیں \عددی{\Psi} سے بلا واسطہ \عددی{\langle v\rangle} دیتی ہے۔ 

روایتی طور پر ہم سمتی رفتار کی بجائے  \اصطلاح{معیار حرکت}\فرہنگ{معیار حرکت}\حاشیہب{momentum}\فرہنگ{momentum} \عددی{p=mv} کے ساتھ کام کرتے ہیں۔
\begin{align}\label{مساوات_تفاعل_موج_تعریف_معیار_حرکت}
\langle p \rangle=m\frac{d\langle x \rangle}{\dif{t}}=-i\hslash\int\big (\Psi^*\frac{\partial{\Psi}}{\partial{x}} \big)\dif{x}
\end{align}
میں \عددی{\langle x\rangle} اور \عددی{\langle p \rangle} کو زیادہ معنی خیز طرز میں پیش کرتا ہوں۔
\begin{align}
\langle x \rangle &=\int \Psi^*(x)\Psi \dif{x}\label{مساوات_تفاعل_موج_توقعاتی_مقام}\\
\langle p \rangle&=\int \Psi^*\big (\frac{\hslash}{i}\frac{\partial}{\partial{x}}\big )\Psi \dif{x}\label{مساوات_تفاعل_موج_توقعاتی_معیار_حرکت}
\end{align}
کوانٹم میکانیات میں  مقام کو \اصطلاح{عامل}\فرہنگ{عامل}\حاشیہب{operator}\فرہنگ{operator} \عددی{x} اور معیار حرکت کو عامل \عددی{\frac{\hslash}{i}\frac{\dif }{\dif{x}}}  ظاہر کرتے ہیں۔  کسی بھی توقعاتی قیمت کے حصول کی خاطر ہم موزوں عامل کو \عددی{\Psi^*} اور \عددی{\Psi} کے بیچ لکھ کر تکمل لیتے ہیں۔

یہ سب بہت اچھا ہے  لیکن دیگر مقداروں کا کیا ہو گا؟ حقیقت یہ ہے کہ تمام کلاسیکی متغیرات کو مقام اور معیار حرکت کی صورت میں لکھا جا سکتا ہے۔ مثال کے طور پر حرکی توانائی کو
\begin{align*}
T=\frac{1}{2}mv^{2}=\frac{p^{2}}{2m}
\end{align*}
اور زاویائی معیار حرکت کو 
\begin{align*}
\kvec{L}=\kvec{r}\times m\kvec{v}=\kvec{r}\times \kvec{p}
\end{align*}
لکھا جا سکتا ہے (جہاں یک بعدی حرکت کے لئے  زاویائی معیار حرکت نہیں پایا جاتا ہے)۔ کسی بھی مقدار مثلاً \عددی{Q(x,p)} کی توقعاتی قیمت  حاصل کرنے کے لیے ہم ہر \عددی{p}  کی جگہ \عددی{\tfrac{\hslash}{i}\frac{d}{\dif{x}}} پر کر کے حاصل عامل کو \عددی{  \Psi^*} اور \عددی{\Psi} کے بیچ لپیٹ کر درج ذیل تکمل حاصل کرتے ہیں۔
\begin{align}\label{مساوات_تفاعل_موج_توقعاتی_قیمت_حصول}
\langle Q(x,p) \rangle=\int \Psi^*Q\big (x,\frac{\hslash}{i}\frac{\partial}{\partial{x}}\big )\Psi \dif{x}
\end{align}
مثال کے طور پر حرکی توانائی کی توقعاتی قیمت درج ذیل ہو گی۔
\begin{align}
\langle T \rangle=-\frac{\hslash ^{2}}{2m}\int \Psi^*\frac{\partial^{\,2}\Psi}{\partial{x^{2}}}\dif{x}
\end{align}
حال \عددی{\Psi} میں ایک ذرہ کی کسی بھی حرکی مقدار کی توقعاتی قیمت  مساوات \حوالہ{مساوات_تفاعل_موج_توقعاتی_قیمت_حصول}  سے حاصل ہو گی۔
مساوات \حوالہ{مساوات_تفاعل_موج_توقعاتی_مقام} اور \حوالہ{مساوات_تفاعل_موج_توقعاتی_معیار_حرکت}  اس کی دو مخصوص صورتیں ہیں۔ میں نے کوشش کی ہے کہ جناب بوہر کی شماریاتی تشریح کو مد نظر رکھتے ہوئے   مساوات  \حوالہ{مساوات_تفاعل_موج_توقعاتی_قیمت_حصول} قابل قبول نظر آئے، اگرچہ، حقیقتاً یہ کلاسیکی میکانیات سے بہت مختلف انداز ہے کام کرنے کا۔ ہم باب \حوالہء{3} میں اس کو زیادہ مضبوط نظریاتی بنیادوں پر کھڑا کریں گے، جب تک آپ اس کے استعمال  کی مشق کریں۔ فی الحال آپ اس کو ایک مسلمہ تصور کر سکتے ہیں۔

\ابتدا{سوال}
آپ کیوں مساوات \حوالہ{مساوات_تفاعل_موج_رفتار_توقعاتی_الف} کے وسطی فقرہ پر تکمل بالحصص کرتے ہوئے، وقتی تفرق کو \عددی{x} کے اوپر سے گزار کر،یہ جانتے ہوئے کہ \عددی{\tfrac{\partial x}{\partial t}=0} ہے، فیصلہ نہیں کر سکتے ہیں کہ \عددی{\tfrac{\dif  \langle x\rangle}{\dif t}=0} ہو گا؟
\انتہا{سوال}
%=====================  
\ابتدا{سوال}
\عددی{\tfrac{\dif \langle p\rangle}{\dif t}} کا حساب کریں۔\ترچھا{جواب:}
\begin{align}\label{مساوات_تفاعل_موج_مخفی_توانائی_سے_معیار_حرکت}
\frac{\dif\langle p \rangle}{\dif t}=\big\langle-\frac{\partial V}{\partial x} \big\rangle
\end{align}
مساوات \حوالہ{مساوات_تفاعل_موج_تعریف_رفتار} (مساوات \حوالہ{مساوات_تفاعل_موج_تعریف_معیار_حرکت} کا پہلا حصہ) اور \حوالہ{مساوات_تفاعل_موج_مخفی_توانائی_سے_معیار_حرکت} \اصطلاح{مسئلہ اہرنفسٹ}\فرہنگ{مسئلہ!اہرنفسٹ}\حاشیہب{Ehrenfest's theorem}\فرہنگ{theorem!Ehrenfest} کی مخصوص صورتیں ہیں، جو کہتا ہے کہ توقعاتی قیمتیں کلاسیکی قواعد کو مطمئن کرتے ہیں۔
\انتہا{سوال}
%==============
\ابتدا{سوال}
فرض کریں آپ مخفی توانائی کے ساتھ ایک مستقل جمع کرتے ہیں (مستقل سے میرا مراد ایسا مستقل ہے  جو \عددی{x} اور \عددی{t} کا تابع نہ ہو)۔ کلاسیکی میکانیات میں یہ کسی بھی چیز پر اثر انداز نہیں ہو گا البتہ کوانٹم میکانیات میں اس کے اثر پر غور کرنا باقی ہے۔ دکھائیں کہ تفاعل موج کو اب \عددی{e^{-iV_t/\hslash}} ضرب کرتا ہے جو وقت کا تابع جزو ہے۔اس کا کسی حرکی متغیر کی توقعاتی قیمت پر کیا اثر ہو گا؟ 
\انتہا{سوال}
%=========================

\حصہ{اصول عدم یقینیت}
فرض کریں آپ ایک لمبی رسی کا ایک سر اوپر نیچے  ہلا کر موج پیدا کرتے ہیں (شکل \حوالہء{1.7})۔ اب اگر پوچھا جائے کہ یہ موج ٹھیک  کہاں پائی جاتی ہے تو آپ غالباً اس کا جواب دینے سے قاصر ہونگے۔ موج کسی ایک جگہ نہیں بلکہ کئی میٹر لمبائی پر پائی جاتی ہے۔ اس کی بجائے اگر  \اصطلاح{طول موج}\فرہنگ{طول موج}\حاشیہب{wavelength}\فرہنگ{wavelength} پوچھی جائے  تو آپ اس کا معقول جواب دے سکتے ہیں:  اس کا طول موج تقریباً ایک میٹر ہے۔ اس کے برعکس اگر آپ رسی کو ایک جھٹکا دیں تو  ایک نوکیلی موج پیدا ہو گی (شکل \حوالہء{1.8})۔ یہ موج دوری نہیں ہے لہٰذا اس کے طول موج کی بات کرنا بے معنی ہو گا۔ اب آپ طول موج بتانے سے قاصر ہوں گے جبکہ  موج کا مقام بتانا ممکن ہو گا۔ اول الذکر  میں موج کا مقام پوچھنا بے معنی سوال ہو گا جبکہ موخر الذکر میں طول موج جاننا بے معنی  ہو گا۔ ہم ان دو صورتوں کے بیچ کے حالات بھی پیدا کر سکتے ہیں  جن میں  مقام موج   اور  طول موج خاصی حد تک قابل تعین ہوں۔ تاہم ان صورتوں میں طول موج بہتر سے بہتر جانتے ہوئے مقام موج کم سے کم بتانا ممکن ہو گا یا پھر مقام بہتر سے بہتر جانتے ہوئے طول موج کم سے کم قابل تعین  ہو گا۔ فوریئر تجزیہ کا ایک مسئلہ ان حقائق کو مضبوط بنیادوں پر کھڑا کرتا ہے ۔ فی الحال میں صرف کیفی دلائل پیش کرنا چاہتا ہوں ۔

 یہ حقائق ہر موجی مظہر، بشمول کوانٹم میکانی موج تفاعل، کے لیے درست ہیں۔ اب ایک ذرے کے \عددی{\Psi} کے طول موج اور  معیار حرکت کا تعلق  \اصطلاح{کلیہ ڈی بروگ لی}\فرہنگ{کلیہ!ڈی بروگ لی}\حاشیہب{De Broglie formula}\فرہنگ{formula!De Broglie}
\begin{align}\label{مساوات_تفاعل_موج_ڈی_بروگلی_معیار_حرکت}
p=\frac{h}{\lambda}=\frac{2\pi\hslash}{\lambda}
\end{align}
پیش کرتا ہے ۔یوں طول موج میں پھیلاو معیار حرکت میں پھیلاو کے مترادف  ہے اور  اب ہمارا عمومی مشاہدہ  یہ  ہو گا کہ کسی ذرے کا مقام ٹھیک ٹھیک جانتے ہوئے ہم اس کی معیار حرکت  کم سے کم  جان سکتے ہیں۔ اس کو ریاضیاتی روپ میں لکھتے ہیں:
\begin{align}\label{مساوات_تفاعل_موج_اصول_عدم_یقینیت}
\sigma_{x}\sigma_{p}\ge\frac{\hslash}{2}
\end{align}
جہاں \عددی{\sigma_x} اور \عددی{\sigma_p} بالترتیب \عددی{x} اور \عددی{p} کے معیاری انحراف ہیں۔ یہ جناب ہیزنبرگ کا مشہور \اصطلاح{اصول عدم یقینیت}\فرہنگ{عدم یقینیت اصول}\فرہنگ{اصول!عدم یقینیت}\حاشیہب{uncertainty principle}\فرہنگ{uncertainty principle}  ہے ۔ (اس کا ثبوت  باب \حوالہ{باب_قواعد_و_ضوابط} میں پیش کیا جائے گا۔میں نے اس کو یہاں اس لئے متعارف کیا کہ آپ باب \حوالہ{باب_غیر_تابع_وقت_شروڈنگر_مساوات} کی مثالوں میں اس کا استعمال کرنا سیکھیں۔)

 اس بات کی تسلی کر لیں کہ آپ کو اصول عدم یقینیت کا مطلب سمجھ آ گیا ہے۔  مقام کی پیمائش کی ٹھیک ٹھیک نتائج کی طرح معیار حرکت کی پیمائش بھی ٹھیک ٹھیک نتائج دے گی۔ یہاں  "پھیلاو" سے مراد یہ ہے کہ یکساں تیار کردہ  نظاموں پر پیمائشیں  بالکل ایک جیسے نتائج  نہیں دیں گی۔  آپ چاہیں تو (\عددی{\Psi} کو نوکیلی بنا کر) ایسا حال تیار کر سکتے ہیں جس پر مقام کی پیمائشیں قریب قریب نتائج  دیں لیکن ایسی صورت میں معیار حرکت کی پیمائشوں کے نتائج ایک دوسرے سے بہت مختلف ہوں گی۔ اس طرح آپ چاہیں تو (\عددی{\Psi} کو ایک لمبی سائن نما موج بنا کر) ایسا حال تیار کر سکتے ہیں جس پر معیار حرکت کی پیمائشوں کے نتائج ایک دوسرے کے قریب قریب ہوں گے لیکن ایسی صورت میں ذرے کے مقام کی پیمائشوں کے نتائج ایک دوسرے سے بہت مختلف ہوں گے۔اور ہاں آپ ایسا حال بھی تیار کر سکتے ہیں جس میں نہ تو مقام  اور نا ہی معیار حرکت ٹھیک سے معلوم ہو۔ مساوات \حوالہ{مساوات_تفاعل_موج_اصول_عدم_یقینیت} درحقیقت ایک عدم مساوات ہے جس میں \عددی{\sigma_x} اور \عددی{\sigma_p} کی جسامت پر کوئی حد مقرر نہیں ہے۔ آپ \عددی{\Psi} کو ایک لمبی بلدار لکیر بنا کر، جس میں بہت سارے ابھار اور گڑھے پائے جاتے ہوں اور جس میں کوئی تواتر نہ پایا جاتا ہو، \عددی{\sigma_x} اور \عددی{\sigma_p} کی قیمتیں جتنی چاہیں بڑھا سکتے ہیں۔
%================

\ابتدا{سوال}
ایک ذرہ جس کی کمیت \عددی{m} ہے درج ذیل حال میں پایا جاتا ہے 
\begin{align}
\Psi (x,t)=Ae^{-a[(mx^{2}/\hslash)+it]}
\end{align}
جہاں \عددی{A} اور \عددی{a} مثبت حقیقی مستقل ہیں۔
\begin{enumerate}[a.]
\item
مستقل \عددی{A} تلاش کریں۔
\item
کس مخفی توانائی تفاعل \عددی{V(x)} کے لیے \عددی{\Psi} شروڈنگر مساوات کو مطمئن کرتا ہے؟
\item
\عددی{x}، \عددی{x^2}، \عددی{p} اور \عددی{p^2} کی توقعاتی قیمتیں تلاش کریں۔
\item
\عددی{\sigma_x} اور \عددی{\sigma_p} کی قیمتیں تلاش کریں۔ کیا ان کا حاصل ضرب اصول عدم یقینیت پر پورا اترتے ہیں؟
\end{enumerate}
\انتہا{سوال}
%====================
\ابتدا{سوال} 
مستقل \عددی{\pi} کے  ہندسی پھیلاو کے اولین \عددی{25} ہندسوں \عددی{(3,1,4,1,5,9,\cdots)}  پر غور کریں۔
\begin{enumerate}[a.]
\item
اس گروہ سے بلا منصوبہ ایک ہندسہ  منتخب کیا جاتا ہے۔ صفر تا نو ہر ہندسہ کے انتخاب کا احتمال کیا ہو گا؟
\item
 کسی ہندسے کے انتخاب کا احتمال سب سے زیادہ ہو گا؟ وسطانیہ ہندسہ کونسا ہو گا؟ اوسط قیمت کیا ہو گی؟
\item
اس تقسیم کا معیاری انحراف کیا ہو گا؟
\end{enumerate}
\انتہا{سوال}
%===========================
\ابتدا{سوال}
گاڑی کی رفتار پیما کی خراب سوئی آزادانہ  طور پر حرکت کرتی ہے۔ ہر جھٹکا کے بعد یہ اطراف سے ٹکڑا کر \عددی{0} اور \عددی{\pi} زاویوں کے بیچ آ کر رک جاتی ہے۔
\begin{enumerate}[a.] 
\item

کثافت احتمال \عددی{\rho(\theta)} کیا ہو گا؟ اشارہ: زاویہ \عددی{\theta} اور \عددی{(\theta+\dif \theta)} کے بیچ سوئی رکنے کا احتمال \عددی{\rho (\theta)\dif\theta} ہو گا۔ متغیر \عددی{\theta} کے لحاظ سے \عددی{\rho(\theta)} کو وقفہ\عددی{-\tfrac{\pi}{2}} تا \عددی{\tfrac{3\pi}{2}} ترسیم کریں (ظاہر ہے اس وقفے کا کچھ حصہ درکار نہیں ہے جہاں \عددی{\rho} صفر ہو گا)۔ دھیان رہے کہ کل احتمال \عددی{1} ہو گا۔
\item
اس تقسیم کے لیے \عددی{\langle \theta \rangle}، \عددی{\langle \theta^2 \rangle} اور \عددی{\sigma} تلاش  کریں۔
\item
 اسی طرح \عددی{\langle \sin\theta \rangle}، \عددی{\langle \cos\theta \rangle} اور \عددی{\langle \cos^2\theta \rangle} تلاش کریں۔
\end{enumerate}
\انتہا{سوال}
