\باب{حرارت نا گزر تخمین}\شناخت{باب_حرارت_نا_گزر_تخمین}

\حصہ{مسئلہ حرارت ناگزر}
\جزوحصہ{حرارت ناگزر عمل}
فرض کریں ایک کامل لٹکن انتصابی ستہ میں بغیر کسی رگڑ یا ہوائی مزاحمت کے آگے پیچھے ارتعاش کرتا ہے اگر آپ اس لٹکن کو جٹکے سے ہلائیں تو یہ افراتفری کے ساتھ دائروی صورت میں حرکت کرنے لگے گا لیکن اگر آپ بغیر جھٹکے کے لٹکن کو آہستہ آہستہ ایک مقام سے دوسری مقام منتقل کریں شکل 10.1 تب لٹکن اسی سطح یا اس کے متوازی سطح میں شائستگی اور روانی سے اسی حیطہ کے ساتھ جلھولتا رہے گا بیرونی حالات کی بہت آہستہ آہستہ تبدیلی ہی حرارت نہ گزر عمل کی پہچان ہے دھیان رہے کہ یہاں دو مختلف امتیازی وقتتوں کی بات کی جا رہی ہے نظام کی حرکت جو یہاں لٹکن کی ارتعاش کا دوری عرصہ ہوگا کو ظاہر کرنے والا اندرونی وقت \عددی{T_i} اور نظام میں نمایاں تبدیلی مثلا لرزتے ہوئے چبوترا پر نصب لٹکن کی صورت میں چبوترے کی لرزش کا دوری عرصہ کو ظاہر کرنے والا بیرونی وقت \عددی{T_e} حرارت ناگزر عمل میں \عددی{T_e \gg T_i} ہوگا حرارت نہ گزر عمل کے تجزیہ کا بنیادی حکمت عملی یہ ہوگا کہ پہلے بیرونی عوامل مقدار معلوم کو غیر متغیر رکھتے ہوئے مسئلہ حل کیا جاتا ہے اور حساب کے بالکل آخر میں انہیں بہت آہستہ آہستہ وقت کے ساتھ تبدیل ہونے کی اجازت دی جاتی ہے مثال کے طور پر مقررہ لمبائی \عددی{L} کی لٹکن کا کلاسیکی دوری عرصہ \عددی{
2 \pi \sqrt{L/g}
} ہوگا اب اگر لمبائی آہستہ آہستہ تبدیل ہو تب دوری عرصہ  بظاہر  \عددی{
2 \pi \sqrt{L(t)/g}
} ہوگا حصہ 7.3 میں ہائیڈروجن سالمہ پر تبصرہ کے دوران ایک زیادہ باریک بیں مثال پیش کی گئی ہم نے آغاز میں مرکزہ کو ساکن تصور کرتے ہوئے ان کے بیچ فاصلہ \عددی{ R} کی صورت میں الیکٹرون کی حرکت کے لئے حل کیا نظام کی زمینی حال توانائی کو \عددی{ R} کے تفاعل کی صورت میں دریافت کرنے کے بعد ہم نے توازنی فاصلہ معلوم کرکے ترسیم کی ان حنا سے مرکزہ کی لرزش کا تعدد حاصل کیا سوال 7.10 طبیعت سالمہ میں اس ترکیب کو جس میں ساکن مرکزہ سے آغاز کرتے ہوئے الیکٹرانی تفاعلات موج کا حساب کر کے ان سے نسبتا سست رفتار مرکزہ کی مقامات اور حرکت کے بارے میں معلومات حاصل کرنے کو بارن و اوپن ہائيمر تخمین کہتے ہیں حرارت نہ گزر تخمین کے بنیادی تصور کو ایک مسئلہ کے روپ میں پیش کیا جا سکتا ہے فرض کریں ہيملٹنی ابتدائی روپ \عددی{H^i} سے بہت آہستہ آہستہ تبدیل ہوکر کسی اختتامی روپ  \عددی{H^f} تک پہنچتا ہے مسئلہ حرارت نہ گزر کہتا ہے کہ اگر ذرا ابتدائی طور پر \عددی{H^i} کے \عددی{n} وی امتیازی حال میں پایا جاتا ہوں تب یہ زیر مساوات شروڈنگر \عددی{H^f} کی \عددی{ n} وی امتیازی حال میں منتقل ہوگا میں یہاں فرض کرتا ہوں کہ \عددی{H^i} سے \عددی{H^f} تک تحویل کے دوران طیف غیر مسلسل اور غیر انحطاطی ہے یو حالات کی ترتیب کوئی شبہ نہیں پایا جائے گا امتیازی تفاعلات پر نظر رکھنے کی کوئی ترکیب وضع کرنے سے ان شرائط کو نرم بنایا جا سکتا ہے لیکن میں یہاں ایسا نہیں کروں گا مثال کے طور پر ہم لامتناہی چکور کنواں میں ایک ذرا کو زمینی حال میں تیار کرتے ہیں شکل 10.2(الف) 
\begin{align}
\psi^i (x) = \sqrt{\frac{2}{a}} \sin \big ( \frac{\pi}{a} x \big )
\end{align}
 اب دائیں  دیوار کو بہت آہستہ آہستہ مقام \عددی{2a} پر منتقل کیا جاتا ہے مسئلہ حرارت نہ گزر کے تحت ماسوائے جزو ضربی ہیّت کے یہ ذرہ توسیع شدہ کنواں کے زمینی حال میں منتقل ہوگا شکل 10.2(ب) 
\begin{align}
\psi^f (x) = \sqrt{\frac{1}{a}} \sin \big ( \frac{\pi}{2a} x \big )
\end{align}
دھیان رہے کے نظریہ اضطراب کی طرح ہم ہيملٹنی میں ایک چھوٹی تبدیلی کی بات نہیں کر رہے ہیں یہاں تبدیلی بہت بڑی ہے فقط اتنا ضروری ہے کہ تبدیلی بہت آہستہ آہستہ رونما ہو یہاں توانائی کی بقا نہیں ہوگی جو بھی دیوار کو حرکت دے رہا ہے نظام سے توانائی حاصل کرے گا جیسا کہ گاڑی کی انجن کے سلنڈر میں آہستہ آہستہ پھیلتا ہوا گیس بوکا کو توانائی فراہم کرتا ہے اس کے برعکس کنواں کی اچانک وسط کی صورت میں حال \عددی{
\psi^i (x)
} ہی رہتا ہے شکل 10.2(ج) جو نئے ہیملٹنی کے امتیازی حالات کا ایک پیچیدہ خطی جوڑ ہوگا سوال 2.38 یہاں توانائی کی بقا ہوگی کم ازکم اس کی توقعاتی قیمت کی ضرور ہوگی جیسا اچانک رکاوٹ ہٹانے سے خلا میں گیس کی آزادانہ پھیلاو سے کوئی کام نہیں ہوتا 
\ابتدا{سوال}
ایک لامتناہی چکور کنواں جس کی دائیں دیوار ایک مستقل سمتی رفتار \عددی{v} سے حرکت کرتے ہوئے کنواں کو وسیع بناتا ہے کو بالکل ٹھیک ٹھیک حل کرنا ممکن ہے اس کے حلوں کا مکمل سلسلہ درج ذیل ہوگا 
\begin{align}
\Phi n (x, t) \cong \sqrt{\frac{2}{\omega}} \sin \big ( \frac{n \pi}{\omega} x \big ) e^{i(mvx^2 -2E_n^i at)/\hslash \omega}
\end{align}
جہاں \عددی{
w(t) \equiv a + vt 
} کنواں کی لمحاتی چوڑائی اور چوڑائی \عددی{ a} کے اصل کنواں کی \عددی{ n} ویں اجازتی توانائی \عددی{
E_n^i \equiv n^2 \pi^2 \hslash^2 /2ma^2
} ہے عمومی حل ان \عددی{\Phi} کا ایک خطی جوڑ:
\begin{align}
\Psi (x, t) = \sum_{n = 1}^{\infty} c_n \Phi_n (x, t)
\end{align}
ہوگا جہاں عددی سر \عددی{c_n} وقت \عددی{ t} کے تابع نہیں ہوں گے 
\begin{enumerate}[a.]
\item
دیکھیں آیا تابع وقت شروڈنگر مساوات بمع مناسب سرحدی شرائط کو مساوات 10.3  مطمئن کرتی ہے 
\item
فرض کریں اصل کنواں کی زمینی حال میں ایک ذرہ آغاز \عددی{(t=0)}  کرتا ہے  
\begin{align*}
\Psi (x, 0) = \sqrt{\frac{2}{a}} \sin \big ( \frac{\pi}{a} x \big )
\end{align*}
دکھائیں کے پھیلاؤ کے عددى سروں کو درج ذیل روپ میں لکھا جا سکتا ہے 
\begin{align}
c_n = \frac{2}{\pi} \sum_0^{\pi} e^{-iaz^2} \sin(nz) \sin(z) \dif z
\end{align}
جہاں \عددی{
\alpha \equiv mva/2 \pi^2 \hslash
}
کنواں کی پھیلنے کی رفتار کی ایک بے بودی پیمائش ہے بدقسمتی سے اس تکمل کی قیمت کو بنیادی تفاعلات کی صورت میں حاصل نہیں کیا جا سکتا ہے 
\item
فرض کریں ہم کنواں کو ابتدائی چوڑائی کے دگنا چوڑائی تک پھیلنے دیتے ہیں یوں بیرونی وقت \عددی{
w(T_e) = 2a
} ہوگا ابتدائی زمینی حال کے تابع وقت قوت نمائی جزو ضربی کا دورانیہ اندرونی وقت ہوگا وقت \عددی{T_e} اور \عددی{T_i} تعین کرکے دیکھائے کے حرکت نہ گزر صورت حال سے مراد \عددی{\alpha \ll 1} ہوگا جس کے تحت تکمل کے دائرہ کار پر \عددی{
e^{-i \alpha z^2} \cong 1
} ہوگا اس کو استعمال کرتے ہوئے پھیلاؤ کے عددی سر \عددی{c_n} تعین کریں حال \عددی{
\Psi (x, t)
} تیار کرکے تصدیق کریں کہ یہ مسئلہ حرارت نہ گزر کے مطابق ہے 
\item
دکھائیں گے \عددی{
\Psi (x, t)
} میں جزو ہيّت کو درج ذیل روپ میں لکھا جا سکتا ہے 
\begin{align}
\theta (t) = - \frac{1}{\hslash} \int_0^1 E_1 (t') \dif t'
\end{align}
جہاں لمحہ \عددی{ t} پر لمحاتی امتیازی قدر \عددی{
E_n (t) \equiv n^2 \pi^2 \hslash^2 /2m \omega^2 
} ہوگا اس نتیجہ پر تبصرہ کریں 
\end{enumerate}
\انتہا{سوال}

