\باب{غیر تابع وقت نظریہ اضطراب}\شناخت{باب_غیر_تابع_وقت_نظریہ_اضطراب}

\حصہ{ غیر انحطاطی نظریہ اضطراب}
\جزوحصہ{ عمومی ضابطہ  بندی}
فرض کریں ہم کسی مخفیہ (مثلاً   یک  بعدی لامتناہی چکور کنواں) کے لئے غیر تابع وقت شروڈنگر مساوات:
\begin{align}\label{مساوات_اضطراب_پہلی}
H^0\psi_n^0=E_n^0\psi_n^0
\end{align}
حل کر کے معیاری عمودی امتیازی تفاعلات \عددی{\psi_n^0} کا مکمل سلسلہ
\begin{align}
\langle \psi_n^0 | \psi_m^0 \rangle = \delta_{nm}
\end{align}
اور ان کی مطابقتی امتیازی اقدار \عددی{E_n^0} حاصل کرتے ہیں۔ اب ہم مخفیہ میں معمولی اضطراب پیدا کرتے ہیں (مثلاً کنواں کی تہہ میں ایک چھوٹا موڑا ڈال کر؛  شکل  \حوالہء{6۔1}) ہم  نئے  امتیازی تفاعلات اور امتیازی اقدار جاننا چاہیں گے: 
\begin{align}\label{مساوات_اضطراب_بنیادی}
H\psi_n = E_n\psi_n
\end{align}
تاہم انتہائی خوش قسمتی کے علاوہ  کوئی وجہ   نہیں  پائی  جاتی کے ہم اس  پیچیدہ مخفیہ کے لیے مساوات شروڈنگر کو بالکل ٹھیک ٹھیک حل کر  پائیں  گے۔ \اصطلاح{ نظریہ   اضطراب} کو    غیر  مضطرب صورت کے معلوم ٹھیک ٹھیک حلوں کو لے کر قدم بقدم چلتے ہوئے مضطرب مسئلے کے تخمینی حل دیتا ہے ہم نئے  ہیملٹنی کو دو اجزاء کا  مجموعہ لکھ کر آغاز کرتے ہیں 
\begin{align}
H = H^0 + \lambda H'
\end{align}
جہاں \عددی{H'} اضطراب ہے زیر بالا میں \عددی{0} ہمیشہ  غیر  مضطرب مقدار کو ظاہر کرتا ہے ہم یہاں \عددی{\lambda} کو ایک چھوٹا عدد تصور کرتے ہیں بعد میں اس کی قیمت کو بڑھا کر ایک \عددی{(1)} کر دی جائے گی اور \عددی{H} اصل ہیملٹنی ہو گا اس کے بعد ہم \عددی{\psi_n} اور \عددی{E_n} کو \عددی{\lambda} کی طاقتی تسلسل کے صورت میں لکھتے ہیں 
\begin{align}
\psi_n &= \psi_n^0 + \lambda\psi_n^1 + \lambda^2\psi_n^2+\dotsc \label{مساوات_اضطراب_سائے_این}\\
E_n &= E_n^0 + \lambda E_n^1 + \lambda^2 E_n^2+\dotsc \label{مساوات_اضطراب_ای_این}
\end{align} 
یہاں \عددی{n} ویں امتیازی قدر کی قیمت میں \اصطلاح{اول رتبی تصحیح} کو \عددی{E_n^1} ظاہر کرتا ہے جبکہ \عددی{n} ویں امتیازی تفاعل میں \اصطلاح{اول رتبی تصحیح} کو  \عددی{\psi_n^1} ظاہر کرتا ہے اسی طرح \عددی{E_n^2} اور \عددی{\psi_n^2} دوم رتبی تصحيح ہوں گے و غیر ه و  غیر ه مساوات \حوالہ{مساوات_اضطراب_سائے_این} اور مساوات \حوالہ{مساوات_اضطراب_ای_این} کو مساوات \حوالہ{مساوات_اضطراب_بنیادی} میں پر کر کے 
\begin{multline*}
(H^0 + \lambda H')[\psi_n^0 + \lambda \psi_n^1 + \lambda^2 \psi_n^2 + \dotsc]\\
= (E_n^0 + \lambda E_n^1 + \lambda^2 E_n^2 + \dotsc)[\psi_n^0 + \lambda \psi_n^1 + \lambda^2 \psi_n^2 + \dotsc]
\end{multline*}
یا \عددی{\lambda} کے ایک جیسے طاقتوں کو اکٹھا لکھ کر درج ذیل لکھا جا سکتا ہے 
\begin{multline*}
H^0 \psi_n^0 + \lambda (H^0 \psi_n^1 + H' \psi_n^0) + \lambda^2 (H^0 \psi_n^2 + H' \psi_n^1) + \dotsc \\
= E_n^0 \psi_n^0 + \lambda (E_n^0 \psi_n^1 + E_n^1 \psi_n^0) + \lambda^2 (E_n^0 \psi_n^2 + E_n^1 \psi_n^1 + E_n^2 \psi_n^0) + \dotsc
\end{multline*}
 کمتر رتبہ \عددی{\lambda^0} کی صورت میں اس سے \عددی{H^0 \psi_n^0 = E_n^0 \psi_n^0} حاصل ہوتا ہے جو  کوئی ی نئی مساوات نہیں ہے (مساوات \حوالہ{مساوات_اضطراب_پہلی}) رتبہ اول \عددی{(\lambda^1)} تک درج ذیل ہو گا 
\begin{align}\label{مساوات_اضطراب_رتبہ_اول}
H^0 \psi_n^1 + H' \psi_n^0 = E_n^0 \psi_n^1 + E_n^1 \psi_n^0
\end{align}
رتبہ دوم \عددی{(\lambda^2)} تک درج ذیل ہو گا 
\begin{align}\label{مساوات_اضطراب_رتبہ_دوم}
H^0 \psi_n^2 + H' \psi_n^1 = E_n^0 \psi_n^2 + E_n^1 \psi_n^1 + E_n^2 \psi_n^0
\end{align}
و غیر ه و غیر ه (رتبہ  پر نظر رکھنے کی غرض سے ہم نے \عددی{\lambda} استعمال کیا اب اس کی ضرورت نہیں رہی لہٰذا اس کی قیمت ایک، \عددی{1}، کر دیں)

\جزوحصہ{اول رتبی نظریہ}
مساوات \حوالہ{مساوات_اضطراب_رتبہ_اول} کا \عددی{\psi_n^0} کے ساتھ اندرونی ضرب لیتے ہیں یعنی \عددی{(\psi_n^0)^*} سے ضرب دے کر تکمل لیتے ہیں 
\begin{align*}
\langle \psi_n^0 | H^0 \psi_n^1 \rangle + \langle \psi_n^0 | H' \psi_n^0 \rangle = E_n^0 \langle \psi_n^0 | \psi_n^0 | \psi_n^1 \rangle + E_n^1 \langle \psi_n^0 | \psi_n^0 \rangle
\end{align*}
تاہم \عددی{H^0} ہرمشی ہے لہٰذا
\begin{align*}
\langle \psi_n^0 | H^0 \psi_n^1 \rangle = \langle H^0 \psi_n^0 | \psi_n^1 \rangle = E_n^0 \langle \psi_n^0 | \psi_n^1 \rangle
\end{align*}
 ہو گا جو دائیں ہاتھ کے پہلے  جزو کو حدف کرے گا مزید  \عددی{
\langle \psi_n^0 | \psi_n^0 \rangle = 1
} 
کی بنا درج ذیل ہو گا 
\begin{align}
E_n^1 = \langle \psi_n^0 | H' | \psi_n^0 \rangle
\end{align}
یہ رتبہ اول نظریہ اضطراب کا بنیادی نتیجہ ہے بلکہ عملاً یہ پوری کوانٹم میکانیات میں غالباً سب سے اہم مساوات ہے یہ کہتی ہے کے  غیر  مضطرب حال میں اضطراب کی توقعاتی قیمت توانائی کی  اول رتبی   تصحیح ہو گی 

\ابتدا{مثال}
لامتناہی چکور کنواں کی غیر مضطرب تفاعلات موج مساوات \حوالہء {2.28} درج ذیل ہیں  
\begin{align*}
\psi_n^0 (x) = \sqrt{\frac{2}{a}} \sin (\frac{n \pi}{a} x)
\end{align*}
 فرض کریں ہم کنواں کی تہہ کو مستقل مقدار \عددی{V_0} اوپر اٹھاتے ہوئے اس نظام کو مضطرب کرتے ہیں شکل \حوالہء{6.2} توانائیوں میں رتبہ اول درستگی تلاش کریں 

حل: یہاں \عددی{H' = V_0}
 ہو گا لہٰذا   \عددی{n} ویں حال کی توانائی میں رتبہ اول تصحیح درج ذیل ہو گی
\begin{align*}
E_n^1 = \langle \psi_n^0 | V_0 | \psi_n^0 \rangle = V_0 \langle \psi_n^0 | \psi_n^0 \rangle = V_0
\end{align*}
یوں درست شدہ توانائیوں کی سطحیں  \عددی{
E_n \cong  E_n^0 + V_0
} ہونگے جی ہاں تمام کی تمام \عددی{V_0} مقدار سے اوپر  اٹھتی ہیں یہاں حیرانگی کی بات یہ ہے کہ رتبہ اول نظریہ بالکل ٹھیک جواب دیتا ہے یوں ظاہر ہے کہ مستقل اضطراب کی صورت میں تمام بلند رتبی تصحیح صفر ہوں گی \حاشیہد{یہاں  کوئی ی چیز لامتناہی چکور کنواں کی خصوصیات پر منحصر نہیں ہے لہٰذا یہی کچھ کسی بھی مخفیہ کے لیے مستقل اضطراب کی صورت میں درست ہو گا} اس کے برعکس کنواں کی نصف چوڑائی تک اضطراب کی وسعت کی صورت میں شکل
\حوالہء{6.3} ہو گا۔

\begin{align*}
E_n^1 = \frac{2V_0}{a} \int_0^{a/2} \sin^2 (\frac{n \pi}{a} x) \dif = \frac{V_0}{2}
\end{align*}
اب توانائی کی ہر سطح  \عددی{
\frac{V_0}{2}
}
اوپر  اٹھتی ہے یہ غالباً بالکل ٹھیک نتیجہ نہیں ہے لیکن اول رتبہ تخمین کی نقطہ نظر سے معقول جواب ہے۔
\انتہا{مثال}

 مساوات \حوالہء{6.9} ہمیں توانائی کی اول رتبی درستگی دیتی ہے تفاعل موج کے لئے اول رتبی تصحیح حاصل کرنے کی غرض سے ہم مساوات \حوالہء{6.7} کو درج ذیل روپ میں لکھتے ہے 
\begin{align}
(H^0 - E_n^0) \psi_n^1 = - (H' - E_n^1) \psi_n^0
\end{align}

چونکہ اس کا دایاں ہاتھ ایک معلوم تفاعل ہے لہٰذا یہ \عددی{\psi_n^1} میں ایک غیر  متجانس  تفرقی مساوات ہے اب غیر مضطرب تفاعلات موج ایک مکمل سلسلہ دیتے ہیں  لہٰذا  کسی بھی تفاعل کی طرح \عددی{\psi_n^1} کو ان کا خطی جوڑ لکھا جا سکتا ہے 
\begin{align}
\psi_n^1 = \sum_{m \ne n} c_m^{(n)} \psi_m^0
\end{align}
اگر \عددی{/psi_n^1} مساوات 6.10 کو مطمئن کرتا ہوں تب کسی بھی مستقل \عددی{\alpha} کے لیے \عددی{(\psi_n^1 + \alpha \psi_n^0)} بھی اس مساوات کو مطمئن کرے گا  لہٰذا  ہم جزو \عددی{\psi_n^0} کو منفی کر سکتے ہیں ایسے ہی کرتے ہوئے مساوات 6.11 کے مجموعہ میں \عددی{m = n} شامل نہیں کیا گیا عددی سر \عددی{c_m^{(n)}} تعین کر کے ہم مسئلہ حل کر سکتے ہیں ہم مساوات 6.10 میں مساوات 6.11 پر کرتے ہوئے یہ جانتے ہوئے کہ غیر مضطرب شروڈنگر  مساوات مساوات 6.1 کو \عددی{\psi_m^0} مطمئن کرتے ہیں درج ذیل حاصل کرتے ہیں 
\begin{align*}
\sum_{m \ne n} {(E_m^0 - E_n^0) c_m^{(n)} \psi_m^0} = - {(H' - E_n^1) \psi_n^0}
\end{align*}
اس کا \عددی{\psi_l^0} کے ساتھ اندرونی ضرب لیتے ہیں 
\begin{align*}
\sum_{m \ne n} (E_m^0 - E_n^0) c_m^{(n)} \langle \psi_l^0 | \psi_m^0 \rangle = - \langle \psi_l^0 | H' | \psi_n^0 \rangle + E_n^1 \langle \psi_l^0 | \psi_n^0 \rangle 
\end{align*}
اگر \عددی{l = n} ہو تب بایاں ہاتھ صفر ہو گا اور ہمیں دوبارہ مساوات 6.9 ملے گی اگر \عددی{l /ne n} ہو تو درج ذیل ہو گا 
\begin{align*}
(E_l^0 - E_n^0) c_l^{(n)} = - \langle \psi_l^0 | H' | \psi_n^0 \rangle
\end{align*}
یا 
\begin{align}
c_m^{(n)} = \frac{\langle \psi_m^0 | H' | \psi_n^0 \rangle}{E_n^0 - E_m^0}
\end{align}
 لہٰذا ا درج ذیل حاصل ہو گا 
\begin{align}
\psi_n^1 = \sum_{m \ne n} \frac{\langle \psi_m^0 | H' | \psi_n^0 \rangle}{(E_n^0 - E_m^0)} \psi_m^0
\end{align}
جب تک غیر مضطرب توانائی طیف غیر انحطاطی ہو  نسب نما  کوئی ی مسئلہ کڑا نہیں کرے گا  (چونکہ کسی بھی عددی سر کے لئے  \عددی{m = n} نہیں ہوتا) ہاں اس صورت میں جب دو غیر مضطرب حالات کی توانائیاں ایک دوسرے جتنی ہو تب مساوات 6.12 میں نسب نما میں صفر پایا جائے گا جو ہمیں مصیبت میں ڈالے گا ایسی صورت میں انحطاطی نظریہ اضطراب کی ضرورت پیش آئے گی جس پر حصہ 6.2 میں غور کیا جائے گا یوں اول رتبی  نظریہ اضطراب مکمل ہوتا ہے توانائی کی اول رتبی تصحیح \عددی{E_n^1} مساوات 6.9 دیتی ہے جبکہ تفاعل موج کی اول رتبی درستگی \عددی{\psi_n^1} مساوات 6.13 دیتی ہے میں آپ کو یہاں یہ ضرور بتانا چاہوں گا کہ اگرچہ نظریہ اضطراب عموماً  توانائیوں کی بہت درست قیمتیں دیتا ہے یعنی \عددی{E_n^0 + E_n^1} اصل قیمت \عددی{E_n} کے بہت قریب ہے اس سے حاصل تفاعلات موج عموماً   افسوس  کن  ہوتے ہیں  

\ابتدا{سوال} 
فرض کرے ہم لامتناہی چکور کنواں کے وسط میں \عددی{\delta} تفاعلی موڑا ڈالتے ہیں 
\begin{align*}
H' = \alpha \delta (x - \frac{a}{2})
\end{align*}
جہاں \عددی{\alpha} ایک مستقل ہے 
(الف) اجازتی توانائیوں کی اول رتبی تصحیح تلاش کریں بتائیں کہ جفت \عددی{n} کی صورت میں توانائیاں مضطرب کیوں نہیں ہونگی   
(ب) 

\انتہا{سوال} 

