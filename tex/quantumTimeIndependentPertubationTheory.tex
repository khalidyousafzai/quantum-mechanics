\باب{غیر تابع وقت نظریہ اضطراب}\شناخت{باب_غیر_تابع_وقت_نظریہ_اضطراب}


\حصہ{ غیر انحطاطی نظریہ اضطراب}
\جزوحصہ{ عمومی ضابطہ بندى}
فرض کریں ہم کسی مخفیہ (مثلا يک بعدی لامتناہی چکور کنواں) کے لئے غیر تابع وقت شروڈنگر مساوات:
\begin{align}\label{مساوات_اضطراب_پہلی}
H^0\psi_n^0=E_n^0\psi_n^0
\end{align}
حل کر کے معیاری عمودی امتیازی تفاعلات \عددی{\psi_n^0} کا مکمل سلسلہ
\begin{align}
\langle \psi_n^0 | \psi_m^0 \rangle = \delta_{nm}
\end{align}
اور ان کی متابقتى امتیازی اقدار \عددی{E_n^0} حاصل کرتے ہیں۔ اب ہم مخفیہ میں معمولی اضطراب پیدا کرتے ہیں (مثلا کنواں کی تہہ میں ایک چھوٹا موڑا ڈال کر؛ شكل \حوالہء{6۔1}) ہم نیے امتیازی تفاعلات اور امتیازی اقدار جاننا چاہیں گے: 
\begin{align}\label{مساوات_اضطراب_بنیادی}
H\psi_n = E_n\psi_n
\end{align}
تاہم انتہائی خوش كسمتى کے علاوه کوئ و جه نہیں پائ جاتی کے ہم اس پیچیدا مخفياه کے لیے مساوات شروڈنگر کو بالکل ٹھیک ٹھیک حل کر پائے گے۔ \اصطلاح{نظريه اضطراب} کو غير مضطرب صورت کے معلوم ٹھیک ٹھیک حلوں کو لے کر قدم بقدم چلتے ہوئے مضطرب مسئلے کے تخمينى حل دیتا ہے ہم نئے ہيملٹنی کو دو اجزاء کا مجموعه لکھ کر آغاز کرتے ہیں 
\begin{align}
H = H^0 + \lambda H'
\end{align}
جہاں \عددی{H'} اضطراب ہے زیر بالا میں \عددی{0} ہمیشہ غير مضطرب مقدار کو ظاہر کرتا ہے ہم یہاں \عددی{\lambda} کو ایک چھوٹا عدد تصور کرتے ہیں بعد میں اس کی قيمت کو بڑھا کر ایک \عددی{(1)} کردی جائے گی اور \عددی{H} اصل ہیملٹنی ہوگا اس کے بعد ہم \عددی{\psi_n} اور \عددی{E_n} کو \عددی{\lambda} کی طاقتی تسلسل کے صورت میں لکھتے ہیں 
\begin{align}
\psi_n &= \psi_n^0 + \lambda\psi_n^1 + \lambda^2\psi_n^2+\dotsc \label{مساوات_اضطراب_سائے_این}\\
E_n &= E_n^0 + \lambda E_n^1 + \lambda^2 E_n^2+\dotsc \label{مساوات_اضطراب_ای_این}
\end{align} 
یہاں \عددی{n} ویں امتيازى قدر کی قیمت میں \اصطلاح{اول رتبی تصحیح} کو \عددی{E_n^1} ظاہر کرتا ہے جبکہ \عددی{n} ویں امتيازى تفاعل میں \اصطلاح{اول رتبی تصحیح} کو  \عددی{\psi_n^1} ظاہر کرتا ہے اسی طرح \عددی{E_n^2} اور \عددی{\psi_n^2} دوم رتبی تصحيح ہوں گے وغيره و غيره مساوات \حوالہ{مساوات_اضطراب_سائے_این} اور مساوات \حوالہ{مساوات_اضطراب_ای_این} کو مساوات \حوالہ{مساوات_اضطراب_بنیادی} میں پر کرکے 
\begin{multline*}
(H^0 + \lambda H')[\psi_n^0 + \lambda \psi_n^1 + \lambda^2 \psi_n^2 + \dotsc]\\
= (E_n^0 + \lambda E_n^1 + \lambda^2 E_n^2 + \dotsc)[\psi_n^0 + \lambda \psi_n^1 + \lambda^2 \psi_n^2 + \dotsc]
\end{multline*}
یا \عددی{\lambda} کے ایک جیسے طاقتوں کو اکٹھا لکھ کر درج ذیل لکھا جا سکھتا ہے 
\begin{multline*}
H^0 \psi_n^0 + \lambda (H^0 \psi_n^1 + H' \psi_n^0) + \lambda^2 (H^0 \psi_n^2 + H' \psi_n^1) + \dotsc \\
= E_n^0 \psi_n^0 + \lambda (E_n^0 \psi_n^1 + E_n^1 \psi_n^0) + \lambda^2 (E_n^0 \psi_n^2 + E_n^1 \psi_n^1 + E_n^2 \psi_n^0) + \dotsc
\end{multline*}
كمتر رتبہ \عددی{\lambda^0} کی صورت میں اس سے \عددی{H^0 \psi_n^0 = E_n^0 \psi_n^0} حاصل ہوتا ہے جو کوئی نئی مساوات نہیں ہے (مساوات \حوالہ{مساوات_اضطراب_پہلی}) رتبہ اول \عددی{(\lambda^1)} تک درج ذیل ہوگا 
\begin{align}\label{مساوات_اضطراب_رتبہ_اول}
H^0 \psi_n^1 + H' \psi_n^0 = E_n^0 \psi_n^1 + E_n^1 \psi_n^0
\end{align}
رتبہ دوم \عددی{(\lambda^2)} تک درج ذیل ہوگا 
\begin{align}\label{مساوات_اضطراب_رتبہ_دوم}
H^0 \psi_n^2 + H' \psi_n^1 = E_n^0 \psi_n^2 + E_n^1 \psi_n^1 + E_n^2 \psi_n^0
\end{align}
وغيره وغيره (رتبہ  پر نظر رکھنے کی غرض سے ہم نے \عددی{\lambda} استمال کیا اب اس کی ضرورت نہیں رہی لہٰذا اس کی قیمت ایک، \عددی{1}، کردیں)

\جزوحصہ{اول رتبی نظریہ}
مساوات \حوالہ{مساوات_اضطراب_رتبہ_اول} کا \عددی{\psi_n^0} کے ساتھ اندرونی ضرب لیتے ہیں یعنی \عددی{(\psi_n^0)^*} سے ضرب دے کر تکمل لیتے ہیں 
\begin{align*}
\langle \psi_n^0 | H^0 \psi_n^1 \rangle + \langle \psi_n^0 | H' \psi_n^0 \rangle = E_n^0 \langle \psi_n^0 | \psi_n^0 | \psi_n^1 \rangle + E_n^1 \langle \psi_n^0 | \psi_n^0 \rangle
\end{align*}
تاہم \عددی{H^0} ہرمشی ہے لہاذا 
\begin{align*}
\langle \psi_n^0 | H^0 \psi_n^1 \rangle = \langle H^0 \psi_n^0 | \psi_n^1 \rangle = E_n^0 \langle \psi_n^0 | \psi_n^1 \rangle
\end{align*}
ہوگا جو دائیں ہاتھ کے پہلے  جزو کو حدف کرے گا مزيد \عددی{
\langle \psi_n^0 | \psi_n^0 \rangle = 1
} 
کی بنا درج ذیل ہوگا 
\begin{align}
E_n^1 = \langle \psi_n^0 | H' | \psi_n^0 \rangle
\end{align}
یہ رتبہ اول نظریہ اضطراب کا بنیادی نتیجہ ہے بلکہ عملاً یہ پوری کوانٹم میکانیات میں غالباً سب سے اہم مساوات ہے یہ کہتی ہے کے غير مضطرب حال میں اضطراب کی توقعاتى قیمت توانائی کی  اول رتبی  تصحيح ہوگی 

\ابتدا{مثال}
لامتناہی چکور کنواں کی غیر مضطرب تفاعلات موج مساوات \حوالہء {2.28} درج ذیل ہیں  
\begin{align*}
\psi_n^0 (x) = \sqrt{\frac{2}{a}} \sin (\frac{n \pi}{a} x)
\end{align*}
\انتہا{مثال}
. فرض کریں ہم کنواں کی تہہ کو مستقل مقدار \عددی{V_0} اوپر اٹھاتے ہوئے اس نظام کو مضطرب کرتے ہیں شکل \حوالہء{6.2} توانائیوں میں رتبہ اول درستگی تلاش کریں 

حل: یہاں \عددی{H' = V_0}
ہوگا لہٰذا   \عددی{n} ویں حال کی توانائی میں رتبہ اول تصحیح درج ذیل ہوگی 
\begin{align*}
E_n^1 = \langle \psi_n^0 | V_0 | \psi_n^0 \rangle = V_0 \langle \psi_n^0 | \psi_n^0 \rangle = V_0
\end{align*}
یوں درست شدہ توانائیوں کی صطحیں  \عددی{
E_n \cong  E_n^0 + V_0
} ہونگے جی ہاں تمام کی تمام \عددی{V_0} مقدار سے اوپر اٹھتی ہیں یہاں حیرانگی کی بات یہ ہے کہ رتبہ اول نظریہ بالکل ٹھیک جواب دیتا ہے یوں ظاہر ہے کہ مستقل اضطراب کی صورت میں تمام بلند رتبی تصحیح صفر ہوںگی \حاشیہد{یہاں کوئی چیز لامتناہی چکور کنواں کی خصوصیات پر منحصر نہیں ہے لہٰذا یہی کچھ کسی بھی مخفیہ کے لیے مستقل اضطراب کی صورت میں درست ہوگا} اس کے برعکس کنواں کی نصف چوڑائی تک اضطراب کی وسعت کی صورت میں شکل
\حوالہء{6.3} ہو گا۔

