\باب{غیر تابع وقت نظریہ اضطراب}\شناخت{باب_غیر_تابع_وقت_نظریہ_اضطراب}

\حصہ{ غیر انحطاطی نظریہ اضطراب}
\جزوحصہ{ عمومی ضابطہ  بندی}
فرض کریں ہم کسی مخفیہ (مثلاً   یک  بعدی لامتناہی چکور کنواں) کے لئے غیر تابع وقت شروڈنگر مساوات:
\begin{align}\label{مساوات_اضطراب_پہلی}
H^0\psi_n^0=E_n^0\psi_n^0
\end{align}
حل کر کے معیاری عمودی امتیازی تفاعلات \عددی{\psi_n^0} کا مکمل سلسلہ
\begin{align}
\langle \psi_n^0 | \psi_m^0 \rangle = \delta_{nm}
\end{align}
اور ان کی مطابقتی امتیازی اقدار \عددی{E_n^0} حاصل کرتے ہیں۔ اب ہم مخفیہ میں معمولی اضطراب پیدا کرتے ہیں (مثلاً کنواں کی تہہ میں ایک چھوٹا موڑا ڈال کر؛  شکل  \حوالہء{6۔1}) ہم  نئے  امتیازی تفاعلات اور امتیازی اقدار جاننا چاہیں گے: 
\begin{align}\label{مساوات_اضطراب_بنیادی}
H\psi_n = E_n\psi_n
\end{align}
تاہم انتہائی خوش قسمتی کے علاوہ  کوئی وجہ   نہیں  پائی  جاتی کے ہم اس  پیچیدہ مخفیہ کے لیے مساوات شروڈنگر کو بالکل ٹھیک ٹھیک حل کر  پائیں  گے۔ \اصطلاح{ نظریہ   اضطراب} کو    غیر  مضطرب صورت کے معلوم ٹھیک ٹھیک حلوں کو لے کر قدم بقدم چلتے ہوئے مضطرب مسئلے کے تخمینی حل دیتا ہے ہم نئے  ہیملٹنی کو دو اجزاء کا  مجموعہ لکھ کر آغاز کرتے ہیں 
\begin{align}
H = H^0 + \lambda H'
\end{align}
جہاں \عددی{H'} اضطراب ہے زیر بالا میں \عددی{0} ہمیشہ  غیر  مضطرب مقدار کو ظاہر کرتا ہے ہم یہاں \عددی{\lambda} کو ایک چھوٹا عدد تصور کرتے ہیں بعد میں اس کی قیمت کو بڑھا کر ایک \عددی{(1)} کر دی جائے گی اور \عددی{H} اصل ہیملٹنی ہو گا اس کے بعد ہم \عددی{\psi_n} اور \عددی{E_n} کو \عددی{\lambda} کی طاقتی تسلسل کے صورت میں لکھتے ہیں 
\begin{align}
\psi_n &= \psi_n^0 + \lambda\psi_n^1 + \lambda^2\psi_n^2+\cdots \label{مساوات_اضطراب_سائے_این}\\
E_n &= E_n^0 + \lambda E_n^1 + \lambda^2 E_n^2+\cdots \label{مساوات_اضطراب_ای_این}
\end{align} 
یہاں \عددی{n} ویں امتیازی قدر کی قیمت میں \اصطلاح{اول رتبی تصحیح} کو \عددی{E_n^1} ظاہر کرتا ہے جبکہ \عددی{n} ویں امتیازی تفاعل میں \اصطلاح{اول رتبی تصحیح} کو  \عددی{\psi_n^1} ظاہر کرتا ہے اسی طرح \عددی{E_n^2} اور \عددی{\psi_n^2} دوم رتبی تصحيح ہوں گے و غیر ه و  غیر ه مساوات \حوالہ{مساوات_اضطراب_سائے_این} اور مساوات \حوالہ{مساوات_اضطراب_ای_این} کو مساوات \حوالہ{مساوات_اضطراب_بنیادی} میں پر کر کے 
\begin{multline*}
(H^0 + \lambda H')[\psi_n^0 + \lambda \psi_n^1 + \lambda^2 \psi_n^2 + \cdots]\\
= (E_n^0 + \lambda E_n^1 + \lambda^2 E_n^2 + \cdots)[\psi_n^0 + \lambda \psi_n^1 + \lambda^2 \psi_n^2 + \cdots]
\end{multline*}
یا \عددی{\lambda} کے ایک جیسے طاقتوں کو اکٹھا لکھ کر درج ذیل لکھا جا سکتا ہے 
\begin{multline*}
H^0 \psi_n^0 + \lambda (H^0 \psi_n^1 + H' \psi_n^0) + \lambda^2 (H^0 \psi_n^2 + H' \psi_n^1) + \cdots \\
= E_n^0 \psi_n^0 + \lambda (E_n^0 \psi_n^1 + E_n^1 \psi_n^0) + \lambda^2 (E_n^0 \psi_n^2 + E_n^1 \psi_n^1 + E_n^2 \psi_n^0) + \cdots
\end{multline*}
 کمتر رتبہ \عددی{\lambda^0} کی صورت میں اس سے \عددی{H^0 \psi_n^0 = E_n^0 \psi_n^0} حاصل ہوتا ہے جو  کوئی ی نئی مساوات نہیں ہے (مساوات \حوالہ{مساوات_اضطراب_پہلی}) رتبہ اول \عددی{(\lambda^1)} تک درج ذیل ہو گا 
\begin{align}\label{مساوات_اضطراب_رتبہ_اول}
H^0 \psi_n^1 + H' \psi_n^0 = E_n^0 \psi_n^1 + E_n^1 \psi_n^0
\end{align}
رتبہ دوم \عددی{(\lambda^2)} تک درج ذیل ہو گا 
\begin{align}\label{مساوات_اضطراب_رتبہ_دوم}
H^0 \psi_n^2 + H' \psi_n^1 = E_n^0 \psi_n^2 + E_n^1 \psi_n^1 + E_n^2 \psi_n^0
\end{align}
و غیر ه و غیر ه (رتبہ  پر نظر رکھنے کی غرض سے ہم نے \عددی{\lambda} استعمال کیا اب اس کی ضرورت نہیں رہی لہٰذا اس کی قیمت ایک، \عددی{1}، کر دیں)

\جزوحصہ{اول رتبی نظریہ}
مساوات \حوالہ{مساوات_اضطراب_رتبہ_اول} کا \عددی{\psi_n^0} کے ساتھ اندرونی ضرب لیتے ہیں یعنی \عددی{(\psi_n^0)^*} سے ضرب دے کر تکمل لیتے ہیں 
\begin{align*}
\langle \psi_n^0 | H^0 \psi_n^1 \rangle + \langle \psi_n^0 | H' \psi_n^0 \rangle = E_n^0 \langle \psi_n^0 | \psi_n^0 | \psi_n^1 \rangle + E_n^1 \langle \psi_n^0 | \psi_n^0 \rangle
\end{align*}
تاہم \عددی{H^0} ہرمشی ہے لہٰذا
\begin{align*}
\langle \psi_n^0 | H^0 \psi_n^1 \rangle = \langle H^0 \psi_n^0 | \psi_n^1 \rangle = E_n^0 \langle \psi_n^0 | \psi_n^1 \rangle
\end{align*}
 ہو گا جو دائیں ہاتھ کے پہلے  جزو کو حدف کرے گا مزید  \عددی{
\langle \psi_n^0 | \psi_n^0 \rangle = 1
} 
کی بنا درج ذیل ہو گا 
\begin{align}
E_n^1 = \langle \psi_n^0 | H' | \psi_n^0 \rangle
\end{align}
یہ رتبہ اول نظریہ اضطراب کا بنیادی نتیجہ ہے بلکہ عملاً یہ پوری کوانٹم میکانیات میں غالباً سب سے اہم مساوات ہے یہ کہتی ہے کے  غیر  مضطرب حال میں اضطراب کی توقعاتی قیمت توانائی کی  اول رتبی   تصحیح ہو گی 

\ابتدا{مثال}
لامتناہی چکور کنواں کی غیر مضطرب تفاعلات موج مساوات \حوالہء {2.28} درج ذیل ہیں  
\begin{align*}
\psi_n^0 (x) = \sqrt{\frac{2}{a}} \sin \big (\frac{n \pi}{a} x\big )
\end{align*}
 فرض کریں ہم کنواں کی تہہ کو مستقل مقدار \عددی{V_0} اوپر اٹھاتے ہوئے اس نظام کو مضطرب کرتے ہیں شکل \حوالہء{6.2} توانائیوں میں رتبہ اول  تصحیح تلاش کریں 

حل: یہاں \عددی{H' = V_0}
 ہو گا لہٰذا   \عددی{n} ویں حال کی توانائی میں رتبہ اول تصحیح درج ذیل ہو گی
\begin{align*}
E_n^1 = \langle \psi_n^0 | V_0 | \psi_n^0 \rangle = V_0 \langle \psi_n^0 | \psi_n^0 \rangle = V_0
\end{align*}
یوں  تصحیح  شدہ توانائیوں کی سطحیں  \عددی{
E_n \cong  E_n^0 + V_0
} ہونگے جی ہاں تمام کی تمام \عددی{V_0} مقدار سے اوپر  اٹھتی ہیں یہاں حیرانگی کی بات یہ ہے کہ رتبہ اول نظریہ بالکل ٹھیک جواب دیتا ہے یوں ظاہر ہے کہ مستقل اضطراب کی صورت میں تمام بلند رتبی تصحیح صفر ہوں گی \حاشیہد{یہاں  کوئی ی چیز لامتناہی چکور کنواں کی خصوصیات پر منحصر نہیں ہے لہٰذا یہی کچھ کسی بھی مخفیہ کے لیے مستقل اضطراب کی صورت میں درست ہو گا} اس کے برعکس کنواں کی نصف چوڑائی تک اضطراب کی وسعت کی صورت میں شکل
\حوالہء{6.3} ہو گا۔

\begin{align*}
E_n^1 = \frac{2V_0}{a} \int_0^{a/2} \sin^2 \big(\frac{n \pi}{a} x\big ) \dif  x= \frac{V_0}{2}
\end{align*}
اب توانائی کی ہر سطح  \عددی{
\frac{V_0}{2}
}
اوپر  اٹھتی ہے یہ غالباً بالکل ٹھیک نتیجہ نہیں ہے لیکن اول رتبہ تخمین کی نقطہ نظر سے معقول جواب ہے۔
\انتہا{مثال}

 مساوات \حوالہء{6.9} ہمیں توانائی کی اول رتبی  تصحیح دیتی ہے تفاعل موج کے لئے اول رتبی تصحیح حاصل کرنے کی غرض سے ہم مساوات \حوالہء{6.7} کو درج ذیل روپ میں لکھتے ہے 
\begin{align}
(H^0 - E_n^0) \psi_n^1 = - (H' - E_n^1) \psi_n^0
\end{align}

چونکہ اس کا دایاں ہاتھ ایک معلوم تفاعل ہے لہٰذا یہ \عددی{\psi_n^1} میں ایک غیر  متجانس  تفرقی مساوات ہے اب غیر مضطرب تفاعلات موج ایک مکمل سلسلہ دیتے ہیں  لہٰذا  کسی بھی تفاعل کی طرح \عددی{\psi_n^1} کو ان کا خطی جوڑ لکھا جا سکتا ہے 
\begin{align}
\psi_n^1 = \sum_{m \ne n} c_m^{(n)} \psi_m^0
\end{align}
اگر \عددی{/psi_n^1} مساوات 6.10 کو مطمئن کرتا ہوں تب کسی بھی مستقل \عددی{\alpha} کے لیے \عددی{(\psi_n^1 + \alpha \psi_n^0)} بھی اس مساوات کو مطمئن کرے گا  لہٰذا  ہم جزو \عددی{\psi_n^0} کو منفی کر سکتے ہیں ایسے ہی کرتے ہوئے مساوات 6.11 کے مجموعہ میں \عددی{m = n} شامل نہیں کیا گیا عددی سر \عددی{c_m^{(n)}} تعین کر کے ہم مسئلہ حل کر سکتے ہیں ہم مساوات 6.10 میں مساوات 6.11 پر کرتے ہوئے یہ جانتے ہوئے کہ غیر مضطرب شروڈنگر  مساوات مساوات 6.1 کو \عددی{\psi_m^0} مطمئن کرتے ہیں درج ذیل حاصل کرتے ہیں 
\begin{align*}
\sum_{m \ne n} {(E_m^0 - E_n^0) c_m^{(n)} \psi_m^0} = - {(H' - E_n^1) \psi_n^0}
\end{align*}
اس کا \عددی{\psi_l^0} کے ساتھ اندرونی ضرب لیتے ہیں 
\begin{align*}
\sum_{m \ne n} (E_m^0 - E_n^0) c_m^{(n)} \langle \psi_l^0 | \psi_m^0 \rangle = - \langle \psi_l^0 | H' | \psi_n^0 \rangle + E_n^1 \langle \psi_l^0 | \psi_n^0 \rangle 
\end{align*}
اگر \عددی{l = n} ہو تب بایاں ہاتھ صفر ہو گا اور ہمیں دوبارہ مساوات 6.9 ملے گی اگر \عددی{l \ne n} ہو تو درج ذیل ہو گا 
\begin{align*}
(E_l^0 - E_n^0) c_l^{(n)} = - \langle \psi_l^0 | H' | \psi_n^0 \rangle
\end{align*}
یا 
\begin{align}
c_m^{(n)} = \frac{\langle \psi_m^0 | H' | \psi_n^0 \rangle}{E_n^0 - E_m^0}
\end{align}
 لہٰذا ا درج ذیل حاصل ہو گا 
\begin{align}
\psi_n^1 = \sum_{m \ne n} \frac{\langle \psi_m^0 | H' | \psi_n^0 \rangle}{(E_n^0 - E_m^0)} \psi_m^0
\end{align}
جب تک غیر مضطرب توانائی طیف غیر انحطاطی ہو  نسب نما  کوئی ی مسئلہ کھڑا نہیں کرے گا  (چونکہ کسی بھی عددی سر کے لئے  \عددی{m = n} نہیں ہوتا) ہاں اس صورت میں جب دو غیر مضطرب حالات کی توانائیاں ایک دوسرے جتنی ہو تب مساوات 6.12 میں نسب نما میں صفر پایا جائے گا جو ہمیں مصیبت میں ڈالے گا ایسی صورت میں انحطاطی نظریہ اضطراب کی ضرورت پیش آئے گی جس پر حصہ 6.2 میں غور کیا جائے گا یوں اول رتبی  نظریہ اضطراب مکمل ہوتا ہے توانائی کی اول رتبی تصحیح \عددی{E_n^1} مساوات 6.9 دیتی ہے جبکہ تفاعل موج کی اول رتبی  تصحیح \عددی{\psi_n^1} مساوات 6.13 دیتی ہے میں آپ کو یہاں یہ ضرور بتانا چاہوں گا کہ اگرچہ نظریہ اضطراب عموماً  توانائیوں کی بہت درست قیمتیں دیتا ہے یعنی \عددی{E_n^0 + E_n^1} اصل قیمت \عددی{E_n} کے بہت قریب ہے اس سے حاصل تفاعلات موج عموماً   افسوس  کن  ہوتے ہیں  

\ابتدا{سوال} 
فرض کرے ہم لامتناہی چکور کنواں کے وسط میں \عددی{\delta} تفاعلی موڑا ڈالتے ہیں 
\begin{align*}
H' = \alpha \delta (x - \frac{a}{2})
\end{align*}
جہاں \عددی{\alpha} ایک مستقل ہے 
\begin{enumerate}[a.]
\item
 اجازتی توانائیوں کی اول رتبی تصحیح تلاش کریں بتائیں کہ جفت \عددی{n} کی صورت میں توانائیاں مضطرب کیوں نہیں ہونگی   
\item
 زمینی حال کی  تصحیح \عددی{\psi_1^1} کی مساوات مساوات 6.13 کی پھیلاو  میں ابتدائی تین غیر صفر اجزاء تلاش کریں  
 \end{enumerate}
\انتہا{سوال} 

\ابتدا{سوال}
ہارمونی مرتعش \عددی{[V(x) = \tfrac{1}{2} kx^2]} 
کی اجازتی توانائیاں درج ذیل ہیں 
\begin{align*}
E_n &= \big(n + \frac{1}{2}\big) \hslash \omega  && (n = 0, 1, 2, \cdots )
\end{align*}
جہاں \عددی{
\omega = \sqrt{k/m}
} کلاسیکی تعدد ہے اب فرض کرے مقیاس لچک میں معمولی تبدیلی رونما ہوتی ہے \عددی{k \to (1 + \epsilon ) k}
\begin{enumerate}[a.]
\item
(الف) نہیں توانائیوں کی بالکل ٹھیک ٹھیک قیمتیں  حاصل  کرے آپ نے کل یہ کو دوم رتبہ تک \عددی{\epsilon} کی طاقتیں تسلسل میں پھیلائیں 
\item
 اب مساوات 6.9 استعمال کرتے ہوئے توانائی میں اول رتبی اضطراب کا حساب لگائیں یہاں \عددی{H'} کیا  ہو گا اپنے نتیجے کا جزو(الف) کے ساتھ موازنہ کرے اشارہ: نئے تکمل کی قیمت کے حصول کی نا ضرورت اور نہ اجازت ہے 
 \end{enumerate}
\انتہا{سوال} 
\ابتدا{سوال}
ایک لامتناہی چکور کنواں مساوات 2.19 میں دو یکساں بوسن رکھے جاتے ہیں یہ مخفیہ 
\begin{align*}
V(x_1, x_2) = -aV_0\delta (x_1 - x_2)
\end{align*}
 جہاں \عددی{V_0} ایک مستقل ہے جس کا بعد توانائی ہے  اور \عددی{a} کنواں کی چوڑائی ہے کے ذریعے ایک دوسرے پر بہت معمولی اثر انداز ہوتے ہیں 
\begin{enumerate}[a.]
\item
 پہلی قدم میں ذرات کے باہمی   اثر  کو نظر انداز کرتے ہوئے زمینی حال اور پہلے ہیجان حال کے تفاعلات موج اور مطابقتی توانائیاں تلاش کریں 
\item
 اول رتبی نظریہ اضطراب استعمال کرتے ہوئے زمینی حال اور پہلے    ہیجان  حال کے توانائیوں پر ذرات کے باہمی  اثر کا تخمین اول رتبی نظریہ اضطراب سے دریافت کریں 
 \end{enumerate}
\انتہا{سوال} 

%====================
\جزوحصہ{دوم رتبی توانائیاں}

یہاں بھی اسی طرح بڑھتے ہوئے ہم \عددی{\psi_n^0} اور دو رتبی مساوات مساوات 6.8 کا اندرونی ضرب لیتے ہیں 
\begin{align*}
\langle \psi_n^0 | H^0 \psi_n^2 \rangle + \langle \psi_n^0 | H' \psi_n^1 \rangle = E_n^0 \langle \psi_n^0 | \psi_n^2 \rangle + E_n^1 \langle \psi_n^0 | \psi_n^1 \rangle + E_n^2 \langle \psi_n^0 | \psi_n^0 \rangle
\end{align*}
یہاں بھی ہم \عددی{H^0} کی ہرمشی پن کو بروئے کار لاتے ہیں 
\begin{align*}
\langle \psi_n^0 | H^0 \psi_n^2 \rangle = \langle H^0 \psi_n^0 | \psi_n^2 \rangle = E_n^0 \langle \psi_n^0 | \psi_n^2 \rangle
\end{align*}
 لہٰذا  بائیں ہاتھ کا پہلا جزو دائیں ہاتھ کے پہلے جزو کے ساتھ کٹ جائے گا   ساتھ   ہی \عددی{
\langle \psi_n^0 | \psi_n^0 \rangle = 1
} ہو گا   لہٰذا  ہمارے پاس \عددی{E_n^2} کا درج ذیل  کلیہ رہ جاتا ہے 
\begin{align}
E_n^2 = \langle \psi_n^0 | H' | \psi_n^1 \rangle - E_n^1 \langle \psi_n^0 | \psi_n^1 \rangle
\end{align}
تاہم  مجموعہ  میں \عددی{m = n} شامل نہیں  اور  باقی تمام عمودی ہیں  لہٰذا  
\begin{align*}
\langle \psi_n^0 | \psi_n^1 \rangle  = \sum_{m \ne n} c_m^{(n)} \langle \psi_n^0 | \psi_m^0 \rangle = 0
\end{align*}
 ہو گا جس کی بنا 
\begin{align*}
E_n^2 = \langle \psi_n^0 | H' | \psi_n^1 \rangle = \sum_{m \ne n} c_m^{(n)} \langle \psi_n^0 | H' | \psi_m^0 \rangle = \sum{m \ne n} \frac{\langle \psi_m^0 | H' | \psi_n^0 \rangle \langle \psi_n^0 | H' | \psi_m^0 \rangle }{E_n^0 - E_m^0}
\end{align*}
یا  آخرکار
\begin{align}
E_n^2 = \sum_{m \ne n} \frac{ \abs{ \langle \psi_m^0 | H' | \psi_n^0 }^2 }{E_n^0 - E_m^0}
\end{align}
 ہو گا جو دو رتبی نظریہ اضطراب کا بنیادی نتیجہ ہے۔

 اگرچہ ہم اسی طرح آگے بڑھتے ہوئے تفاعل موج  کی  دوم رتبی تصحیح  \عددی{\psi_n^2} توانائی کی سوم  رتبی تصحیح  وغیرہ وغیرہ حاصل کر سکتے ہیں لیکن عملاً اس ترکیب کو صرف مساوات 6.15 تک استعمال کرنا سودمند  ہو گا۔
\ابتدا{سوال}
\begin{enumerate}[a.]
\item
توانائیوں کی  دوم رتبی  تصحیح  \عددی{(E_n^2)} سوال 6.1 کی مخفیہ کے لیے تلاش کریں۔ تبصرہ: آپ تسلسل کا مجموعہ صریحاً حاصل کر کے طاق \عددی{n} کیلئے
 \عددی{-2m( \alpha / \pi \hslash n)^2} حاصل کر سکتے ہیں۔
\item
زمینی حال توانائی کے لئے دوم رتبی تصحیح  \عددی{E_n^2} سوال 6.2 کے مخفیہ کے لیے تلاش کریں۔ تصدیق کیجیے گا کہ آپ کا نتیجہ بالکل درست نتیجہ کے مطابق ہے۔ 
\end{enumerate}
\انتہا{سوال}
\ابتدا{سوال}
ایک ایسے  باردار  ذرہ پر غور کریں جو یک بعدی ہارمونی ارتعاشی مخفیہ  میں پایا جاتا ہو۔ فرض کریں  ہم ایک کمزور برقی میدان \عددی{(E)} چالو کرتے ہیں جس کی بنا مخفی توانائی میں \عددی{H' = qEx} مقدار کی تبدیلی پیدا ہوتی ہے۔
\begin{enumerate}[a.]
\item
دکھائیں کہ توانائیوں کی  دو سطحوں میں کوئی اول رتبی تبدیلی پیدا نہیں ہو گی۔ دو رتبی تصحیح  تلاش کریں۔ اشارہ: سوال 3.33 دیکھیں۔
\item
تبدیلی متغیرات \عددی{x' \equiv x - (qE/m \omega^2)} استعمال کرتے ہوئے موجودہ صورت میں شروڈنگر مساوات کو بلا واسطہ حل کیا جا سکتا ہے۔ ایسا کرتے ہوئے ٹھیک ٹھیک توانائیاں تلاش کر کے دکھائیں کہ یہ نظریہ اضطراب کی تخمین کے مطابق ہے۔
\end{enumerate}
\انتہا{سوال}

\حصہ{انحطاطی نظریہ اضطراب}
اگر غیر مضطرب حالات انحطاطی ہوں یعنی دو یا دو سے زیادہ منفرد حالات \عددی{\psi_a^0} اور \عددی{\psi_b^0} کی توانائیاں ایک دوسرے جیسی ہوں تب سادہ نظریہ اضطراب غیر کارآمد ہو گا  چونکہ \عددی{c_a^{(b)}} مساوات 6.12 اور \عددی{E_a^2} مساوات 6.15 بےقابو بڑھتے ہیں شاید  ماسوائے اس صورت جب شمار کنندہ صفر ہو \عددی{
\langle \psi_a^0 | H' | \psi_b^0 \rangle = 0
} اور جس کو ہم بعد میں استعمال کریں گے۔ یوں  انحطاط صورت میں ہمیں توانائیوں کی اول 

%=====================
رتبی تصحیح مساوات 6.9 پر بھی یقین نہیں کرنا چاہیے اور ہمیں مسئلے کا کوئی دوسرا حل ڈھونڈنا   ہو گا۔

\جزوحصہ{دو پڑتا انحطاط}
درج ذیل فرض کریں جہاں \عددی{\psi_a^0} اور \عددی{\psi_b^0} معمول شدہ ہیں۔
\begin{align}
H^0 \psi_a^0 = E^0 \psi_a^0, \quad H^0 \psi_b^0 = E^0 \psi_b^0, \quad \langle \psi_a^0 | \psi_b^0 \rangle = 0
\end{align}
دھیان رہے کہ ان حالات کا ہر خطی جوڑ 
\begin{align}
\psi^0 = \alpha \psi_a^0 + \beta \psi_b^0
\end{align}
بھی \عددی{H^0} کا امتیازی حال ہو گا جس کا  امتیازی قدر \عددی{E^0} بھی وہی ہو گا 
\begin{align}
H^0 \psi^0 = E^0 \psi^0
\end{align}
عام طور پر اضطراب \عددی{(H')} انحطاط کو "توڑے"  (یا "منسوخ" کرے) گا جیسے جیسے ہم \عددی{\lambda} کی قیمت صفر سے ایک کی طرف بڑھاتے ہیں مشترک غیر مضطرب توانائی \عددی{E^0} دو ٹکڑوں میں تقسیم ہو گا شکل 6.4 مخالف چلتے ہوئے اگر ہم اضطراب کو بند  یعنی صفر  کر دیں تب بالائی حال کا تخفیف  \عددی{\psi_a^0} اور \عددی{\psi_b^0} کے ایک خطی جوڑ میں ہو گا جبکہ زیریں حال کا تخفیف کسی دوسرے عمودی خطی جوڑ میں ہو گا تاہم ہم قبل از وقت نہیں جان سکتے ہیں کہ یہ موزوں خطی جوڑ کیا ہوں گے چونکہ ہم غیر مضطرب حالات نہیں جانتے ہیں لہٰذا  یہی وجہ ہے کہ ہم اول رتبی توانائیاں مساوات 6.9 کا حساب نہیں کر سکتے ہیں 

اسی لیے ہم ان موزوں غیر مضطرب حالات کو فی الحال عمومی روپ مساوات 6.17 میں لکھتے ہیں جہاں \عددی{\alpha} اور \عددی{\beta} قابل تغیر ہوں گے ہم مساوات شروڈنگر
\begin{align}
H \psi = E \psi
\end{align}
کو \عددی{H = H^0 + \lambda H'} اور 
\begin{align}
E = E^0 + \lambda E^1 + \lambda^2 E^2 + \cdots, \quad \psi = \psi^0 + \lambda \psi^1 + \lambda^2 \psi^2 + \cdots
\end{align}
کیلئے حل کرنا چاہتے ہیں انہیں مساوات 6.19 میں پر کر کے پہلے کی طرح \عددی{\lambda} کی ایک جیسی طاقتوں کو اکٹھا کر کے درج ذیل حاصل ہو گا 
\begin{align*}
H^0 \psi^0 + \lambda (H' \psi^0 + H^0 \psi^1) + \cdots = E^0 \psi^0 + \lambda (E^1 \psi^0 + E^0 \psi^1) + \cdots
\end{align*}
اب \عددی{H^0 \psi^0 = E^0 \psi^0} مساوات 6.18 کی بنا اولین اجزاء ایک دوسرے کے ساتھ کٹ جائیں گے جبکہ \عددی{\lambda^1} رتبہ کے لیے درج ذیل ہو گا 
\begin{align}
H^0 \psi^1 + H' \psi^0 = E^0 \psi^1 + E^1 \psi^0
\end{align}
اس کا \عددی{\psi_a^0} کے ساتھ اندرونی ضرب لیتے ہیں 
\begin{align*} 
\langle \psi_a^0 | H^0 \psi^1 \rangle + \langle \psi_a^0 | H' \psi^0 \rangle = E^0 \langle \psi_a^0 | \psi^1 \rangle + E^1 \langle \psi_a^0 | \psi^0 \rangle
\end{align*}
چونکہ \عددی{H^0} ہرمشی ہے لہٰذا  بائیں ہاتھ پہلا جزو دائیں ہاتھ کے پہلے جزو کے ساتھ کٹ جائے گا مساوات 6.17 کو استعمال کرتے ہوئے اور معیاری عمودیت کی شرط مساوات 6.17 کو بروئے کار لاتے ہوئے 
\begin{align*}
\alpha \langle \psi_a^0 | H' | \psi_a^0 \rangle + \beta \langle \psi_a^0 | H' | \psi_b^0 \rangle = \alpha E^1
\end{align*}
یا مختصراً
\begin{align}
\alpha W_{aa} + \beta W_{ab} = \alpha E^1
\end{align}
حاصل ہو گا جہاں درج ذیل ہو گا 
\begin{align}
W_{ij} \equiv \langle \psi_i^0 | H' | \psi_j^0 \rangle, \quad (i, j = a, b)
\end{align}
اسی طرح \عددی{\psi_b^0} کے ساتھ اندرونی ضرب درج ذیل دے گا 
\begin{align}
\alpha W_{ba} + \beta W_{bb} = \beta E^1
\end{align}
دھیان رہے کہ اصولاً ہمیں تمام \عددی{W} معلوم ہے چونکہ یہ غیر مضطرب تفاعلات موج \عددی{\psi_a^0} اور \عددی{\psi_b^0} کے لحاظ سے \عددی{H'} کے ارکان قالب ہیں مساوات 6.24 کو \عددی{W_{ab}} سے ضرب دے کر مساوات 6.22 استعمال کر کے \عددی{\beta W_{ab}} کو خارج کر کے درج ذیل حاصل ہو گا 
\begin{align}
\alpha [W_{ab} W_{ba} - (E^1 - W_{aa}) (E^1 - W_{bb})] = 0
\end{align}
غیر صفر \عددی{\alpha} کی صورت میں مساوات 6.25 ہمیں \عددی{E^1} کی مساوات دیگی 
\begin{align}
(E^1)^2 - E^1 (W_{aa} + W_{bb}) + (W_{aa} + W_{bb} - W_{ab} W_{ba}) = 0
\end{align}
دو درجی کلیہ استعمال کرتے ہوئے اور مساوات 6.23 سے یہ جانتے ہوئے \عددی{W_{ba} = W^*_{ab}} ہم درج ذیل اخذ کرتے ہیں 
\begin{align}
E_{\pm}^1 = \frac{1}{2} \big [W_{aa} + W_{bb} \pm \sqrt{(W_{aa} - W_{bb})^2 + 4 \abs{W_{ab}}^2} \big ]
\end{align}
یہ انحطاطی نظریہ اضطراب کا بنیادی نتیجہ ہے جہاں دو  جذر دو مضطرب توانائیوں سے مطابقت رکھتے ہیں لیکن صفر \عددی{\alpha} کی صورت میں کیا ہو گا ایسی صورت میں \عددی{\beta = 1} ہو گا لہٰذا مساوات 6.22 کے تحت \عددی{
W_{ab} = 0
} اور مساوات 6.24 کے تحت \عددی{E^1 = W_{bb}} ہو گا یہ درحقیقت مساوات 6.27 کے عمومی نتیجہ میں منفی علامت کے ذریعے شامل ہے مثبت علامت \عددی{\alpha = 1}، \عددی{\beta = 0} کی صورت میں ہو گا۔ اس کے علاوہ ہمارے جوابات 
\begin{align*}
E_{+}^1 = W_{aa} = \langle \psi_a^0 | H' | \psi_a^0 \rangle, \quad E_{-}^1 = W_{bb} = \langle \psi_b^0 | H' | \psi_b^0 \rangle
\end{align*}
ٹھیک وہی ہیں جو ہم غیر انحطاطی نظریہ اضطراب سے حاصل کرتے ہیں مساوات  6.9 یہ محض ہماری خوش قسمتی ہے حالات \عددی{\psi_a^0} اور \عددی{\psi_b^0} پہلے سے موزوں خطی جوڑ تھے کیا اچھی بات ہوتی اگر ہم آغاز سے موزوں حالات جان پاتے ایسی صورت میں ہم غیر انحطاطی نظریہ اضطراب استعمال کر پاتے حقیقت میں درج ذیل مسئلہ کے تحت ہم عموماً ایسا کر پاتے ہیں 

\ابتدا{مسئلہ}
فرض کریں \عددی{A} ایک ایسا ہرمشی عامل ہے جو \عددی{H^0} اور \عددی{H'} کے ساتھ قابل تبادل ہے اگر \عددی{H^0} کے انحطاطی امتیازی تفاعلات \عددی{\psi_a^0} اور \عددی{\psi_b^0} عامل \عددی{A} کے بھی امتیازی تفاعلات ہوں جن کے منفرد امتیازی اقدار ہوں
\begin{align*}
\mu\ne \nu\quad \text{\RL{اور}}\quad A \psi_a^0 = \mu \psi_a^0, \quad A \psi_b^0 = \nu \psi_b^0 
\end{align*} 
تب \عددی{W_{ab} = 0} ہو گا لہٰذا  \عددی{\psi_a^0} اور \عددی{\psi_b^0} نظریہ اضطراب میں قابل استعمال موزوں حالات ہوں گے
\انتہا{مسئلہ}
\ابتدا{ثبوت}
ہم فرض کر چکے ہیں کہ \عددی{[A, H'] = 0} ہو گا جس کے تحت درج ذیل ہو گا 
\begin{align*}
\langle \psi_a^0 | [A, H'] \psi_b^0 \rangle &= 0 \\
&= \langle \psi_a^0 | A H' \psi_b^0 \rangle - \langle \psi_a^0 | H' A \psi_b^0 \rangle \\
&= \langle A \psi_a^0 | H' \psi_b^0 \rangle - \langle \psi_a^0 | H' \nu \psi_b^0 \rangle \\
&= (\mu - \nu) \langle \psi_a^0 | H' \psi_b^0 \rangle = (\mu - \nu) W_{ab} 
\end{align*}
اب \عددی{
\mu \ne \nu
} ہے لہٰذا  \عددی{W_{ab} = 0}
ہو گا 

سبق  اگر آپ کا سامنا انحطاطی حالات سے ہو ایسا ہرمشی عامل \عددی{A} تلاش کرنے کی کوشش کریں جو \عددی{H^0} اور \عددی{ H'} کے ساتھ قابل تبادل ہو \عددی{ H^0} اور \عددی{ A} کے بیک وقت امتیازی تفاعلات کو اپنے غیر مضطرب حالات منتخب کر کے سادہ اول رتبی نظریہ اضطراب بروئے کار لائے ایسا عامل تلاش نہ کرنے کی صورت میں آپ کو مساوات 6.27 استعمال کرنا ہو گا جس کی ضرورت عملاً  کم ہی پڑتی ہے 
\انتہا{ثبوت} 

%=============================

\ابتدا{سوال}
فرض کریں  دو موزوں غیر مضطرب حالات
\begin{align*}
\psi_\pm^0 = \alpha_\pm \psi_a^0 + \beta_\pm \psi_b^0
\end{align*}
جہاں \عددی{\alpha_\pm} اور \عددی{\beta_\pm} کو معمول شدگی تک مساوات 6.22 یا مساوات 6.24 تعین کرتے ہیں  صریحاً  درج ذیل دکھائیں 
\begin{enumerate}[a.]
\item
\عددی{\psi_\pm^0} عمودی ہے \عددی{
(\langle \psi_+^0 | \psi_-^0 \rangle = 0)
}  
\item
\عددی{
\langle \psi_+^0 | H' | \psi_-^0 \rangle = 0
} 
\item
\عددی{
\langle_\pm^0 | H' | \psi_\pm^0 \rangle = E_\pm^1
} جہاں \عددی{E^1} کی قیمت مساوات 6.27 دیتی ہے 
\end{enumerate}
\انتہا{سوال} 
\ابتدا{سوال} 
فرض کرے ایک ذرہ جس کی کمیت \عددی{m} ہے اپنے آپ پر بند یک بعدی خطہ جس کی لمبائی \عددی{L} ہے پر آزادی سے حرکت کرتا ہے 
\begin{enumerate}[a.]
\item
دکھائیں کے ساکن حالات کو درج ذیل روپ میں لکھا جا سکتا ہے 
 \begin{align*}
\psi_n (x) &= \frac{1}{\sqrt{L}} e^{2 \pi i n x/ L}, &&(-L/2 < x < L/2)
\end{align*}
جہاں \عددی{
 n = 0, \pm 1, \pm 2, \dotsc
} اور اجازتی توانائیاں درج ذیل ہیں 
\begin{align*}
E_n = \frac{2}{m} \big ( \frac{n \pi \hslash}{L} \big )^2
\end{align*}
دھیان رہے کہ زمینی حال \عددی{ n = 0} کے علاوہ تمام حالات دہرا انحطاطی ہے 
\item
فرض کریں ہم اب اضطراب 
\begin{align*}
H' = -V_0 e^{-x^2 / a^2}
\end{align*}
متعارف کرتے ہیں جہاں \عددی{
a \ll L
} ہو یہ \عددی{ x = 0} پر مخفیہ میں معمولی جھکاوٹ پیدا کرتا گویا تار کو یہاں مروڑا گیا ہوں مساوات 6.27 استعمال کرتے ہوئے \عددی{ E_n} کی اول رتبی  تصحیح  تلاش کریں اشارہ: چونکہ \عددی{H'} خطہ \عددی{
-a < x < a
} کے باہر تقریباً  صفر ہے اور \عددی{a \ll L} ہے لہٰذا تکمل کی قیمت حاصل کرتے وقت تکمل کی حدوں کو \عددی{\pm L/2} کی بجائے  \عددی{\pm \infty} رکھیں 
\item
اس مسئلہ کے لئے \عددی{\psi_n} اور \عددی{\psi_{-n}} کی موزوں خطی جوڑ کیا ہوں گے دکھائے کہ ان حالات کے ساتھ آپ کو مساوات 6.9 استعمال کرتے ہوئے اول رتبی  تصحیح  حاصل ہو گی 
\item
ایسا ہرمشی عامل \عددی{ A} تلاش کریں جو مسئلہ کے شرائط پر پورا اترتا ہو دکھائیں کہ \عددی{H^0} اور \عددی{A} کے بیک وقت امتیازی حالات ٹھیک وہی ہے  جو آپ نے جزو-ج میں استعمال کیے 
\end{enumerate}
\انتہا{سوال} 

%=======================
\جزوحصہ{ بلند رتبی انحطاط}

گزشتہ حصہ میں انحطاط کو دو پڑتا تصور کیا گیا تاہم ہم دیکھ سکتے ہیں کہ اس ترکیب کو کس طرح عمومی بنایا جا سکتا ہے مساوات 6.22 اور 6.24 کو ہم دوبارہ  قالبی  روپ میں لکھتے ہیں 
\begin{align}
\begin{pmatrix} 
W_{aa} & W_{ab} \\
W_{ba} & W_{bb}
\end{pmatrix}
\begin{pmatrix}
\alpha \\
\beta
\end{pmatrix}
= E^1
\begin{pmatrix}
\alpha \\
\beta
\end{pmatrix}
\end{align}
ظاہر ہے کہ \عددی{E^1} \عددی{ W} قالب کے امتیازی اقدار ہیں مساوات 6.26 اس قالب کی امتیازی مساوات ہے  اور غیر مضطرب حالات کے موزوں خطی جوڑ \عددی{ W} کے امتیازی سمتیات ہوں گے

 ہم \عددی{ n} پڑتا انحطاط کی صورت میں \عددی{n \times n}  قالب
\begin{align}
W_{i j} = \langle \psi_i^0 | H' | \psi_j^0
\end{align}
کے امتیازی اقدار تلاش کرتے ہیں الجبرا کی زبان میں موزوں غیر مضطرب تفاعلات موج کی تلاش سے مراد انحطاطی  ذيلی فضا میں ایسا اساس تیار کرنا ہے جو قالب \عددی{ W} کو  وتری بناتا ہو یہاں بھی ایک ایسا عامل \عددی{ A} تلاش کر کے جو \عددی{ H'} کا قابل تبادل ہو \عددی{A} اور \عددی{H'} کے بیک وقت امتیازی تفاعلات استعمال کر کے ہم قالب \عددی{ W} حاصل کریں گے جو از خود وتری ہو گا لہٰذا  آپ  کو امتیازی مساوات حل کرنے کی ضرورت پیش نہیں آئی گی اگر آپ کو میری دو پڑتا انحطاط کو عمومیت دیتے ہوئے \عددی{ n} پڑتا انحطاط پر یقین نہ ہو تب سوال 6.10 حل کر کے  اپنی تسلی کر لیں 

\ابتدا{مثال} 
تین آبادی لامتناہی کعبی کنواں سوال 4.2 پر غور کریں 
\begin{align}
V(x, y, z) = 
\begin{cases}
0, & 0 <x < a, \, 0 < y < a, \, 0 < z < a \\
\infty & \text{\RL{دیگر صورت}}
\end{cases}
\end{align}
ساکن حالات درج ذیل ہیں 
\begin{align}
\psi_{n_x n_y n_z}^0 (x, y, z) = \big ( \frac{2}{a} \big )^{3/2} \sin(\frac{n_x \pi}{a} x) \sin(\frac{n_y \pi}{a} y) \sin(\frac{n_z \pi}{a} z)
\end{align}
جہاں \عددی{n_x}، \عددی{n_y} اور \عددی{n_z} مثبت عدد صحیح ہیں ان کی مطابقتی اجازتی توانائیاں درج ذیل ہیں 
\begin{align}
E_{n_x n_y n_z}^0 = \frac{\pi^2 \hslash^2}{2 m a^2} (n_x^2 + n_y^2 + n_z^2)
\end{align}
دھیان رہے کہ زمینی حال \عددی{\psi_{111}} غیر انحطاطی ہے جس کی توانائی درج ذیل ہے 
\begin{align}
E_1^0 \equiv 3 \frac{\pi^2 \hslash^2}{2ma^2} 
\end{align}
تاہم پہلا ہیجان حال تہرا انحطاطی ہیں
\begin{align}
\psi_a \equiv \psi_{112}, \quad \psi_b \equiv \psi_{121}, \quad \psi_c \equiv \psi_{211}
\end{align}
اور ان تینوں کی توانائی 
\begin{align}
E_1^0 \equiv 3 \frac{\pi^2 \hslash^2}{ma^2}
\end{align}
ایک دوسری جیسی ہے۔ آئیے  اب درج ذیل اضطراب متعارف کرتے ہیں 
\begin{align} 
H' = 
\begin{cases}
V_0, & 0 < x < a/2, \, 0 < y < a/2 \\
0, & \text{\RL{دیگر صورت}}
\end{cases}
\end{align}
جو ڈبہ کے ایک چوتھائی  حصہ میں مخفیہ کو \عددی{V_0} مقدار  بڑھاتا ہے شکل 6.5 زمینی حال توانائی کی ایک رتبی  تصحیح  مساوات 6.9 دیتی ہے 
\begin{align}
E_0^1 &= \langle \psi_{111} | H' | \psi_{111} \rangle\nonumber \\
&= \big (\frac{2}{a} \big )^3 V_0 \int_0^{a/2} \sin^2 \big ( \frac{\pi}{a} x \big ) \dif x \int_0^{a/2} \sin^2 \big ( \frac{\pi}{a} y \big ) \dif y \int_0^a \sin^2 \big ( \frac{\pi}{a} z \big ) \dif z\nonumber \\
&= \frac{1}{4} V_0
\end{align}
%=============
جو ہمارے توقعات کے عین مطابق ہے اول ہیجان  حال جاننے کے لیے ہمیں انحطاطی نظریہ اضطراب کی  پوری صلاحیت درکار ہو گی پہلے قدم میں ہم قالب \عددی{ W} تیار کرتے ہیں اس کے  وتری ارکان وہی ہونگے جو زمینی حال کے ہیں ماسوائے ان میں سے ایک سائن جس کا دلیل دگنا ہے آپ درج ذیل کی خود تصدیق کر سکتے  ہیں 
\begin{align*}
W_{aa} = W_{bb} = W_{cc} = \frac{1}{4} V_0
\end{align*}
غیر وتری ارکان زیادہ دلچسپ ہے 
\begin{multline*}
W_{ab} = \big ( \frac{2}{a} \big )^3 V_0 \int_0^{a/2} \sin^2 \big ( \frac{\pi}{a} x \big ) \dif x\\
\times \int_0^{a/2} \sin \big (\frac{\pi}{a} y \big ) \sin \big ( \frac{2 \pi}{a} y \big ) \dif y \int_0^a \sin \big ( \frac{2 \pi}{a} z \big ) \sin \big ( \frac{\pi}{a} z \big ) \dif z
\end{multline*}
تاہم \عددی{ z} تکمل صفر ہو گا جیسا \عددی{W_{ac}} کے لیے بھی ہو گا لہٰذا  درج ذیل ہو گا 
\begin{align*}
W_{ab} = W_{ac} = 0
\end{align*}
الغرض  درج ذیل ہو گا 
\begin{multline*}
W_{bc} = \big ( \frac{2}{a} \big )^3 V_0 \int_0^{a/2} \sin \big ( \frac{\pi}{a} x \big ) \sin \big (\frac{2 \pi}{a} x \big ) \dif x\\
\times \int_0^{a/2} \sin \big (\frac{\pi}{a} y \big ) \sin \big ( \frac{\pi}{a} y \big ) \dif y \int_0^a \sin^2 \big ( \frac{\pi}{a} z \big ) \dif z = \frac{16}{9 \pi^2} V_0
\end{multline*}
یوں درج ذیل ہو گا جہاں \عددی{\kappa \equiv (8/3 \pi)^2 \approx 0.7205} ہے
\begin{align}
\bold{W} =
\begin{pmatrix}
W_{aa} & W_{ab} & W_{ac} \\
W_{ba} & W_{bb} & W_{bc} \\
W_{ca} & W_{cb} & W_{cc}
\end{pmatrix}=
 \frac{V_0}{4}
\begin{pmatrix}
1 & 0 & 0 \\
0 & 1 & \kappa \\
0 & \kappa & 1
\end{pmatrix}
\end{align}
قالب \عددی{\bold{W}} بلکہ \عددی{4 \bold{W}/V_0} جس کے ساتھ کام کرنا زیادہ آسان ہے کی امتیازی مساوات ضمیمہ \حوالہ{ضمیمہ_امتیازی_تفاعلات_و_اقدار} کے تحت
\begin{align*}
\begin{vmatrix}
1 - w & 0 & 0 \\
0 & 1 - w& \kappa \\
0 & \kappa & 1 - w
\end{vmatrix}
\end{align*}
یعنی
\begin{align*}
(1 - w)^3 - \kappa^2 (1 - w) = 0
\end{align*}
ہو گی جس کے امتیازی اقدار درج ذیل ہونگے 
\begin{align*}
w_1 = 1; \quad w_2 = 1+ \kappa \approx 1.7205; \quad w_3 = 1 - \kappa \approx 0.2795
\end{align*}
یوں \عددی{\lambda} کے اول رتبہ تک درج ذیل ہو گا 
\begin{align}
E_1 (\lambda) = 
\begin{cases}
E_1^0 + \lambda V_0/4 \\
E_1^0 + \lambda (1+ \kappa) V_0 /4 \\
E_1^0 + \lambda (1 - \kappa) V_0 /4
\end{cases}
\end{align}
جہاں \عددی{E_1^0} مشترکہ غیر مضطرب توانائی مساوات 6.35 ہے اضطراب توانائی \عددی{E_1^0} تین منفرد توانائیوں کی سطحوں میں تقسیم کر کے انحطاط ختم  کرتا ہے شکل 6.6 دیکھیں دھیان رہے اگر ہم بھولا پن میں اس مسئلے کو غیر انحطاطی نظریہ اضطراب سے حل کرتے تب ہم اخذ کرتے کہ اول رتبی تصحیح مساوات 6.9 تینوں حالات کے لئے ایک جیسی \عددی{V_0 /4} ہوتی جو درحقیقت صرف درمیانے حال کے لیے درست ہے 

مزید موزوں غیر مضطرب حالات درج ذیل روپ کے خطی جوڑ ہونگے 
\begin{align}
\psi^0 = \alpha \psi_a + \beta \psi_b + \gamma \psi_c 
\end{align}
جہاں عددی سر (\عددی{\alpha}،  \عددی{\beta} اور \عددی{\gamma})  قالب \عددی{\bold{W}} کے امتیازی سمتیات  ہوں گے 
\begin{align*}
\begin{pmatrix}
1 & 0 & 0 \\
0 & 1 & \kappa \\
0 & \kappa & 1
\end{pmatrix}
\begin{pmatrix}
\alpha \\
\beta \
\gamma
\end{pmatrix}
= w 
\begin{pmatrix}
\alpha \\
\beta \\
\gamma
\end{pmatrix}
\end{align*}
ہمیں \عددی{w = 1} کے لیے  \عددی{\alpha = 1}، \عددی{\beta = \gamma = 0} جبکہ  \عددی{w = 1 \pm \kappa} کے لیے \عددی{\alpha = 0}، \عددی{
\beta = \pm \gamma = 1/ \sqrt{2}
} حاصل ہوتے ہیں۔ میں نے ان کی معمول شدہ قیمتیں پی کی ہیں۔ ہوں موزوں حالات درج ذیل ہونگے 
\begin{align}
\psi^0 =
\begin{cases}
\psi_a \\
(\psi_b + \psi_c)/ \sqrt{2} \\
(\psi_b - \psi_c)/ \sqrt{2}
\end{cases}
\end{align}
\انتہا{مثال}
\ابتدا{سوال} 
لامتناہی کعبی کنواں مساوات 6.30 میں نقطہ \عددی{
(a/4, a/2, 3a/4)
} پر ڈیلٹا تفاعلی موڑا:
\begin{align*}
H' = a^3 V_0 \delta (x - a/4) \delta (y - a/2) \delta (z - 3a/4)
\end{align*}
 رکھ کر کنواں کو مضطرب کیا جاتا ہے۔ زمینی حال اور تہرا انحطاطی اول ہیجان حالات کی توانائیوں میں اول رتبی تصحیح تلاش کریں
\انتہا{سوال}
\ابتدا{سوال}
ایک ایسے کوانٹائی نظام پر غور کریں جس میں صرف تین خطی غیر تابع حالات پائے جاتے ہوں فرض کریں قالبی روپ میں اس کا ہیملٹنی درج ذیل ہے
\begin{align*}
\bold{H} = V_0 
\begin{pmatrix}
(1 - \epsilon) & 0 & 0 \\
0 & 1 & \epsilon \\
0 & \epsilon & 2
\end{pmatrix}
= \underbrace{V_0 
\begin{pmatrix}
1 & 0 & 0 \\
0 & 1 & 0 \\
0 & 0 & 2
\end{pmatrix}}_{H^0} 
+ \underbrace{\epsilon V_0 
\begin{pmatrix}
-1 & 0 & 0 \\
0 & 0 & 1 \\
0 & 1 & 0
\end{pmatrix}}_{H'}
\end{align*}
جہاں \عددی{V_0} ایک مستقل ہے اور \عددی{\epsilon} کوئی چھوٹا عدد  \عددی{(\epsilon \ll 1)}  ہے۔
\begin{enumerate}[a.]
\item
غیر مضطرب ہیملٹنی  \عددی{(\epsilon = 0)} کے امتیازی سمتیات اور امتیازی اقدار لکھیں 
\item
قالب \عددی{\bold{H}} کہ بالکل ٹھیک امتیازی اقدار  کے لئے حل کریں ان میں سے ہر ایک کو \عددی{\epsilon} کی صورت میں دوم رتبہ تک طاقتی تسلسل کی روپ میں پھیلائیں 
\item
اول رتبی اور دوم رتبی غیر انحطاطی نظریہ اضطراب استعمال کرتے ہوئے اس حال کی امتیازی قدر کی تخمینی قیمت تلاش کریں جو \عددی{H^0} کے غیر انحطاطی امتیازی سمتیہ سے پیدا ہوتا ہے آپ نے جواب کا جزو-ا کے بالکل ٹھیک جواب کے ساتھ موازنہ کریں 
\item
ابتدائی طور پر  انحطاطی  دو امتیازی اقدار  کی اول رتبی تصحیح  کو انحطاطی نظر یا ئے اضطراب سے تلاش کریں  بالکل ٹھیک نتائج کے ساتھ موازنہ کریں 
\end{enumerate}
\انتہا{سوال}
\ابتدا{سوال}
میں دعویٰ چکا ہوں کہ \عددی{ n} پڑتا انحطاطی توانائی کے اول رتبی تصحیح  قالب \عددی{ W} کے امتیازی اقدار ہوں گے میں نے دعویٰ کیا کہ یہ \عددی{ n = 2} صورت کی قدرتی عمومیت ہے۔ اس کو ثابت کرنے کے لئے،  حصہ 6.2.1 کی قدموں پر چل کر درج ذیل سے آغاز کر کے
\begin{align*}
\psi^0 = \sum_{j = 1}^n \alpha_j \psi_j^0
\end{align*}
(مساوات 6.17 کو  عمومیت  دیتے ہوئے)  دکھائیں کہ مساوات  6.22  کے مماثل کا مفہوم   قالب \عددی{\bold{W}} کی امتیازی قدر مساوات  لیا جا سکتا ہے۔ 
\انتہا{سوال} 

%========================================
\حصہ{ہائیڈروجن کا  مہین  ساخت}
ہائیڈروجن جوہر کے مطالعہ کے دوران حصہ 4.2 ہم نے   ہیملٹنی  درج ذیل لی 
\begin{align}
H = - \frac{\hslash^2}{2m} \nabla^2 - \frac{e^2}{4 \pi \epsilon_0} \frac{1}{r}
\end{align}
جو الیکٹران کی حرکی توانائی جمع کولمب مخفی توانائی ہے۔ تاہم یہ مکمل کہانی نہیں ہے ہم \عددی{ m} کی بجائے تخفیف   شدہ  کمیت سوال 5.1 استعمال  کر کے  ہیملٹنی میں حرکت مرکزہ کا اثر شامل  کرنا سیکھ چکے ہیں زیادہ اہم مہین ساخت ہے جو در حقیقت دو منفرد وجوہات،  اضافیتی تصحیح اور چکرو مدار   ربط،  کی بنا پیدا ہوتا ہے۔ بوہر توانائیوں مساوات 4.70 کے لحاظ سے مہین ساخت \عددی{\alpha^2} گنا کم  نہایت چھوٹا اضطراب ہے جہاں 
\begin{align}
\alpha \equiv \frac{e^2}{4 \pi \epsilon_0 \hslash c} \cong \frac{1}{137.036}
\end{align} 
مہین ساخت مستقل کہلاتا ہے اس سے بھی \عددی{\alpha} گنا چھوٹا  لیمب انتقال ہے جو بھر کی میدان کی کوانٹازنی سے وابستہ ہے اور اس سے مزید کم نہایت مہین ساخت کہلاتا ہے جو الیکٹران اور پروٹان کے جفت قطب  معیار اثر کے بیچ مقناطیسی باہم عمل سے پیدا ہوتا ہے اس  تنظیمی  ڈھانچہ کو جدول 6.1 میں پیش کیا گیا ہے اس حصہ میں ہم غیر تابع  وقت نظریہ اضطراب کی مثال کے طور پر ہائیڈروجن کی مہین ساخت پر غور کریں گے 
\ابتدا{سوال}
\begin{enumerate}[a.]
\item
بوہر توانائیوں کو مہین ساخت مستقل اور الیکٹران کی ساکن توانائی \عددی{mc^2} کی صورت میں لکھیں 
\item
\عددی{\epsilon_0}، \عددی{e}، \عددی{\hslash} اور \عددی{c} کی تجرباتی قیمتیں  استعمال کیے بغیر مہین ساخت مستقل کی قیمت تلاش کریں تبصرہ پوری طبیعیات میں بلاشبہ مہین ساخت مستقل سب سے زیادہ خالص بے بعدی  بنیادی عدد ہے یہ برقناطیسیت  الیکٹران کا بار اضافیت روشنی کی رفتار اور کوانٹم میکانیات پلانک مستقل کے بنیادی مستقلات کے بیچ رشتہ بیان کرتا ہے اگر آپ جزو -ب حل کر پائیں  یقیناً   آپ کو نوبیل انعام سے نوازا جائے گا البتہ میرا مشورہ ہوگا کہ اس وقت اس پر بہت وقت ضائع نہ کریں بہت سارے انتہائی قابل لوگ ایسا  کر کے  ناکام ہو چکے ہیں  
\end{enumerate}
\انتہا{سوال} 
\جزوحصہ{اضافیتی  تصحیح}
ہیملٹنی کا پہلا جزو بظاہر حرکی توانائی کو ظاہر کرتا ہے 
\begin{align}
T = \frac{1}{2} mv^2 = \frac{p^2}{2m} 
\end{align}
جس میں باضابطہ متبادل \عددی{
\bold{p} \to (\hslash/i) \nabla^2
} پر کر کے درج ذیل عامل حاصل ہوگا 
\begin{align}
T = - \frac{\hslash^2}{2m} \nabla^2
\end{align} 
تاہم مساوات 6.44 حرکی توانائی کا کلاسیکی کلیہ ہے اضافیتی کلیہ درج ذیل ہے 
\begin{align}
T = \frac{mc^2}{\sqrt{1 - (v/c)^2}} - mc^2
\end{align}
جہاں پہلا جزو کل اضافیتی  توانائی ہے جس میں مخفی توانائی شامل نہیں ہے اور جس سے ہمیں فی الحال غرض بھی نہیں ہے جبکہ دوسرا جزو ساکن توانائی ہے ان دونوں کے بیچ فرق کو حرکت سے منسوب کیا جا سکتا ہے ہمیں سمتی رفتار کی بجائے  اضافیتی معیار حرکت
\begin{align}
p = \frac{mv}{\sqrt{1 - (v/c)^2}}
\end{align}
 کی صورت میں \عددی{ T} کو لکھنا ہوگا۔دھیان رہے کہ
\begin{align*}
p^2 c^2 + m^2 c^4 = \frac{m^2 v^2 c^2 + m^2 c^4 [1 - (v/c)^2]}{1 - (v/c)^2} = \frac{m^2 c^4}{1 - (v/c)^2} = (T + mc^2)^2
\end{align*}
ہو گا جس کی بنا درج ذیل ہوگا 
\begin{align}
T = \sqrt{p^2 c^2 + m^2 c^4} - mc^2
\end{align}
غیر اضافیتی حد \عددی{p \ll mc} کی صورت میں حرکی توانائی کی اضافیتی مساوات تخفیف کے بعد کلاسیکی نتائج مساوات 6.44 دیتی ہے ایک چھوٹا عدد \عددی{(p/mc)} کی طاقتی تسلسل میں پھیلا کر درج ذیل حاصل ہوگا 
\begin{align}
T = mc^2 \big [ \sqrt{1 + \big(\frac{p}{mc}\big)^2}  - 1 \big ] &= mc^2 \big [ 1 + \frac{1}{2} \big(\frac{p}{mc}\big)^2 - \frac{1}{8} \big(\frac{p}{mc}\big)^4 \cdots - 1 \big ] \nonumber \\
&= \frac{p^2}{2m} - \frac{p^4}{8m^3 c^2} + \cdots .
\end{align}
ہیملٹنی کی کم سے کم رتبی اضافیتی تصحیح درج ذیل ہے 
\begin{align}
H'_r = - \frac{p^4}{8m^3 c^2}
\end{align}
غیر مضطرب حال میں \عددی{H'} کی توقعاتی قیمت رتبہ اول نظریہ اضطراب میں \عددی{E_n} کی تصحیح  ہو گی مساوات 6.9 
\begin{align}
E_r^1 = \langle H'_r \rangle = - \frac{1}{8 m^3 c^2} \langle \psi | p^4 \psi \rangle = - \frac{1}{8m^3 c^2} \langle p^2 \psi | p^2 \psi \rangle
\end{align}
اب غیر مضطرب حالات کے لئے شروڈنگر مساوات کہتی ہے 
\begin{align}
p^2 \psi = 2m (E - V) \psi
\end{align}
 لہٰذا   درج ذیل ہوگا 
\begin{align}
E_r^1 = - \frac{1}{2mc^2} \langle (E - V)^2 \rangle = - \frac{1}{2mc^2} [E^2 - 2E \langle V \rangle + \langle V^2 \rangle]
\end{align}
اب تک یہ مکمل طور پر ایک عمومی نتیجہ ہے تاہم ہمیں ہائیڈروجن میں دلچسپی ہے جس کے لیے \عددی{
-(1/4 \pi \epsilon_0)e^2 /r
} ہوگا 
\begin{align} 
E_r^1 = - \frac{1}{2mc^2} \big [ E_n^2 + 2E_n \big(\frac{e^2}{4 \pi \epsilon_0}\big) \big\langle \frac{1}{r} \big\rangle + \big(\frac{e^2}{4 \pi \epsilon_0}\big)^2 \big\langle \frac{1}{r^2} \big\rangle \big ]
\end{align} 
جہاں \عددی{E_n} زیر غور حال کی بوہر توانائی توانائی ہے یہ کام مکمل کرنے کی خاطر ہمیں غیر مضطرب حال \عددی{\psi_{nlm}} مساوات 4.89 میں \عددی{1/r} اور \عددی{1/r^2} کی توقعاتی قیمتیں درکار ہوں گی پہلا آسان ہے سوال 6.12 دیکھیں 
\begin{align}
\big\langle \frac{1}{r} \big\rangle = \frac{1}{n^2 a}
\end{align}
جہاں \عددی{ a} رداس بوہر مساوات 4.72 ہے دوسرا اتنا آسان نہیں ہے سوال 6.33 دیکھیں تاہم اس کا جواب درج ذیل ہے 
\begin{align}
\big\langle \frac{1}{r^2} \big\rangle = \frac{1}{(l + 1/2)n^3 a^2}
\end{align}
یوں درج ذیل ہوگا 
\begin{align*}
E_r^1 = - \frac{1}{2mc^2} \big [ E_n^2 + 2 E_n \big(\frac{e^2}{4 \pi \epsilon_0}\big) \frac{1}{n^2 a} + \big(\frac{e^2}{4 \pi \epsilon_0}\big)^2 \frac{1}{(l + 1/2)n^3 a^2} \big ]
\end{align*}
یا مساوات 4.72 استعمال کرتے ہوئے \عددی{ a} کو خارج  کر کے  باقی کو \عددی{E_n} مساوات 4.70 کی صورت میں لکھ کے درج ذیل حاصل ہوگا 
\begin{align}
E_r^1 = - \frac{(E_n)^2}{2mc^2} \big [ \frac{4n}{l + 1/2} - 3 \big ]
\end{align}
ظاہر ہے کہ اضافیتی تصحیح کی مقدار \عددی{E_n} سے تقریباً  \عددی{
E_n/mc^2 = \num{2e-5}
} گنا کم ہے 

اگرچہ ہائیڈروجن جوہر بہت زیادہ انحطاطی ہے اس کے باوجود میں نے حساب کے دوران غیر انحطاطی نظریہ اضطراب استعمال کیا مساوات 6.51 یہاں اضطراب کروی تشاکلی ہے  لہٰذا    یہ \عددی{L^2} اور \عددی{L_z} کا مقلوب ہو گا مزید کسی \عددی{E_n} کے \عددی{n^2}  حالات کے لئے ان (تمام) عاملین کے امتیازی تفاعلات کے منفرد امتیازی اقدار ہوں گے۔  یوں خوش قسمتی سے تفاعلات \عددی{\psi_{nlm}} اس مسئلہ کے موزوں حالات  ہوں گے  یا جیسا ہم کہتے ہیں \عددی{n}، \عددی{l} اور \عددی{m} موزوں کوانٹم اعداد ہیں  لہٰذا   غیر انحطاطی نظریہ اضطراب کا استعمال درست تھا 

\ابتدا{سوال}  
مسئلہ وریل سوال 4.40 استعمال کرتے ہوئے مساوات 6.55 ثابت کریں 
\انتہا{سوال} 
\ابتدا{سوال} 
آپ نے سوال 4.43 میں حال \عددی{\psi_{321}} کے لیے \عددی{r^s} کی توقعاتی قیمت حاصل کی اپنے جواب کی تصدیق \عددی{s = 0} غیر اہم صفر \عددی{s = -1} مساوات 6.55 \عددی{s = -2} مساوات 6.56 اور \عددی{s = -3} مساوات 6.64 کے لیے کریں \عددی{s = -7} کی صورت میں کیا ہوگا اس پر تبصرہ کریں 
\انتہا{سوال} 
\ابتدا{سوال}
 یک بعدی ہارمونی مرتعش کی توانائی کی سطحوں کے لیے کم سے کم رتبی اضافیتی تصحیح تلاش کریں اشارہ: مثال 2.5 میں مستعمل ترکیب بروئے کار لائیں 
\انتہا{سوال}
\ابتدا{سوال}
دکھائیں کہ ہائیڈروجن حالات کے لیے \عددی{l = 0} لیتے ہوئے \عددی{p^2} ہرمشی ہے لیکن \عددی{p^4} ہرمشی نہیں ہے ان حالات کے لئے \عددی{\psi} متغیرات \عددی{\theta} اور \عددی{\phi} کا غیر تابع ہے  لہٰذا   درج ذیل ہوگا 
\begin{align*}
p^2 = - \frac{\hslash^2}{r^2} \frac{\dif }{\dif r} \big ( r^2 \frac{\dif}{\dif r} \big )
\end{align*}
مساوات 4.13 تکمل بالحصص استعمال کرتے ہوئے درج ذیل دکھائیں 
\begin{align*}
\langle f | p^2 g \rangle = - 4 \pi \hslash^2 
\big ( r^2 f \frac{\dif g}{\dif r} - r^2 g \frac{\dif f}{\dif r} \big ) \big \rvert_0^{\infty} + \langle p^2 f | g \rangle 
\end{align*}
تصدیق کیجئے گا کہ \عددی{\psi_{n00}} کے لیے، جو مبدا کے قریب درج ذیل ہوگا، سرحدی جزو صفر ہے۔
\begin{align*}
\psi_{n00} \sim \frac{1}{\sqrt{\pi} (na)^{3/2}} e^{(-r/na)}
\end{align*}
اب یہی کچھ \عددی{p^4} کے لئے کر کے دیکھیں اور لکھائی کہ سرحدی اجزاء صفر نہیں ہونگے۔  درحقیقت درج ذیل ہوگا 
\begin{align*}
\langle \psi_{n00} | p^4 \psi_{m00} \rangle = \frac{8\hslash^4}{a^4} \frac{(n - m)}{(nm)^{5/2}} + \langle p^4 \psi_{n00} | \psi_{m00} \rangle
\end{align*}
\انتہا{سوال}

