%done till eq 4.97 on p172 and then after MISSING PAGES
%from  sec 4.4.1 spin 1/2 page 185 till start of prob 4.26
%missing part is from eq 4.97 to prob 4.25 (p184) and
%missing part is from prob 4.26 till prob 4.49 (p206)

\باب{تین ابعادی کوانٹم میکانیات}\شناخت{باب_تین_ابعادی_کوانٹم_میکانیات}
\حصہ{کروی محدد میں مساوات شروڈنگر}
تین ابعاد تک توسیع  باآسانی کی جا سکتی ہے۔ مساوات شروڈنگر درج ذیل کہتی ہے
\begin{align}
i\hslash\frac{\dif{\Psi}}{\dif{t}}=H\Psi;
\end{align}
معیاری طریقہ کار کا اطلاق \عددی{x}  کے ساتھ ساتھ \عددی{y} اور \عددی{z} پر  کر کے:
\begin{align}\label{مساوات_تین_ابعاد_عاملین_الف}
p_{x}\to \frac{\hslash}{i}\frac{\partial}{\partial{x}},\quad p_{y}\to \frac{\hslash}{i}\frac{\partial}{\partial{y}},\quad p_{z}\to \frac{\hslash}{i}\frac{\partial}{\partial{z}} 
\end{align}
ہیملٹنی\حاشیہد{جہاں کلاسیکی مشہود اور عامل میں فرق کرنا دشوار ہو، وہاں میں عامل پر "ٹوپی" کا نشان بناتا ہوں۔ اس باب میں ایسا کوئی موقع نہیں پایا جاتا جہاں ان کی پہچان مشکل ہو لہٰذا یہاں سے عاملین پر "ٹوپی" کا نشان نہیں ڈالا جائے گا۔} عامل \عددی{H} کو کلاسیکی توانائی
\begin{align*}
\frac{1}{2}mv^{2}+V=\frac{1}{2m}(p_{x}^{2}+p_{y}^{2}+p_{z}^{2})+V
\end{align*}
سے حاصل کیا جاتا ہے۔ مساوات \حوالہ{مساوات_تین_ابعاد_عاملین_الف} کو مختصراً درج ذیل لکھا  جا سکتا ہے۔
\begin{align}
p\to \frac{\hslash}{i}\nabla
\end{align}
یوں درج ذیل ہو گا
\begin{align}
i\hslash\frac{\partial{\Psi}}{\partial{t}}=-\frac{\hslash^{2}}{2m}\nabla^{2}\Psi+V\Psi
\end{align}
جہاں
\begin{align}
\nabla^{2}\equiv \frac{\partial^{\,2}}{\partial{x^2}}+\frac{\partial^{\,2}}{\partial{y^2}}+\frac{\partial^{\,2}}{\partial{z^2}} 
\end{align}
کارتیسی محدد میں \اصطلاح{لاپلاسی}\فرہنگ{لاپلاسی}\حاشیہب{Laplacian}\فرہنگ{Laplacian} ہے۔

مخفی توانائی \عددی{V}   اور تفاعل موج  \عددی{\Psi}    اب \عددی{\kvec{r}=(x,y,z)} اور \عددی{t} کے تفاعلات ہیں۔ لا متناہی چھوٹے حجم  \عددی{\dif^{\,3}\kvec{r}=\dif{x}\dif{y}\dif{z}} میں ایک ذرہ  پایا جانے کا احتمال
  \عددی{\abs{\Psi(\kvec{r},t)}^{2}\dif^{\,3}\kvec{r}}   ہو گا اور معمول زنی شرط درج ذیل ہو گی
\begin{align}\label{مساوات_ابعادی_معمول_زنی_شرط}
\int\abs{\Psi}^{2}\dif^{\,3}\kvec{r}=1 
\end{align}
جہاں تکمل کو پوری فضا پر لینا ہو گا۔ اگر مخفی توانائی وقت کی تابع نہ ہو  تب ساکن حالات کا مکمل سلسلہ پایا جائے گا:
\begin{align}
\Psi_{n}(\kvec{r},t)=\psi_{n}(\kvec{r})e^{-iE_{n}t/\hslash} 
\end{align}
جہاں فضائی تفاعل موج  \عددی{\psi_{n}}   غیر تابع وقت شروڈنگر مساوات 
\begin{align}
-\frac{\hslash^{2}}{2m}\nabla^{2}\psi+V\psi=E\psi
\end{align}
کو مطمئن کرتا ہے۔ تابع وقت شروڈنگر مساوات کا عمومی حل درج ذیل ہو گا۔
\begin{align}\label{مساوات_ابعادی_عمومی_حل_الف}
\Psi(\kvec{r},t)=\sum c_{n}\psi_{n}(\kvec{r})e^{-iE_{n}t/\hslash} 
\end{align}
جہاں مستقلات \عددی{c_{n}}  ہمیشہ کی طرح ابتدائی تفاعل موج  \عددی{\Psi(\kvec{r},0)}   سے حاصل کیے جائیں گے۔ (اگر مخفیہ   \اصطلاح{استمراریہ}\فرہنگ{استمراریہ}\حاشیہب{continuum}\فرہنگ{continuum}  حالات دیتی ہو تب مساوات \حوالہ{مساوات_ابعادی_عمومی_حل_الف} میں مجموعہ کی بجائے تکمل ہو گا۔)

\ابتدا{سوال}
\begin{enumerate}[a.]
\item
عاملین \عددی{\kvec{r}} اور \عددی{\kvec{p}} کے تمام \اصطلاح{باضابطہ  مقلبیت  رشتے}\فرہنگ{مقلبیت!باضابطہ رشتے}\حاشیہب{canonical commutation relations}\فرہنگ{commutation!canonical relations}: \عددی{[x,y]}، \عددی{[x,p_{y}]}، \عددی{[x,p_{x}]}، \عددی{[p_{y},p_{z}]}، وغیرہ وغیرہ، حاصل کریں۔

\ترچھا{جواب:}
 \begin{align}
[r_{i},p_{j}]=-[p_{i},r_{j}]=i\hslash\delta_{ij},\quad [r_{i},r_{j}]=[p_{i},p_{j}]=0 
\end{align}
جہاں اشاریہ \عددی{x}،\عددی{y} اور \عددی{z} کو ظاہر کرتے ہیں جبکہ \عددی{r_{x}=x}، \عددی{r_{y}=y} اور  \عددی{r_{z}=z}
ہیں۔
\item
تین ابعاد کے لیے مسئلہ اہرنفسٹ کی تصدیق کریں:
\begin{align}
\frac{\dif }{\dif{t}}\langle \kvec{p}\rangle=\langle -\nabla V\rangle  \quad \text{اور}\quad\frac{\dif}{\dif{t}}\langle \kvec{r}\rangle =\frac{1}{m}\langle \kvec{p}\rangle
\end{align}
(ان میں سے ہر ایک در حقیقت تین مساوات کو ظاہر کرتی ہے۔ ایک مساوات ایک جزو کے لیے ہو گا۔)  \ترچھا{اشارہ:} 
پہلے  تصدیق کر لیں کہ مساوات \حوالہء{3.71} تین ابعاد کے لیے بھی کارآمد ہے۔
\item
ہیزنبرگ عدم یقینیت کے اصول کو تین ابعاد کے لیے بیان کریں۔ 

\ترچھا{جواب:}
 \begin{align}
\sigma_{x}\sigma_{p_x}\geq\frac{\hslash}{2},\quad \sigma_{y}\sigma_{p_y}\geq\frac{\hslash}{2},\quad \sigma_{z}\sigma_{p_z}\geq\frac{\hslash}{2}
\end{align}
تاہم  (مثلاً) \عددی{\sigma_{x}\sigma_{p_y}}  پر کوئی پابندی عائد نہیں ہوتی۔
\end{enumerate}
\انتہا{سوال}


\جزوحصہ{علیحدگی متغیرات} 
  عموماً مخفیہ صرف مبدا سے فاصلہ کا تفاعل ہو گا۔ ایسی صورت میں \اصطلاح{کروی محدد}\فرہنگ{محدد!کروی}\حاشیہب{spherical coordinates}\فرہنگ{coordinates!spherical}  \عددی{(r,\theta,\phi)}   کا استعمال بہتر ثابت ہو گا (شکل \حوالہء{4۔1})۔  
کروی محدد میں لاپلاسی درج ذیل روپ اختیار کرتا ہے۔
\begin{align}
\nabla^{2}=\frac{1}{r^{2}}\frac{\partial}{\partial{r}}\big (r^{2}\frac{\partial}{\partial{r}}\big )+\frac{1}{r^{2}\sin{\theta}}\frac{\partial}{\partial{\theta}}\big(\sin{\theta}\frac{\partial}{\partial{\theta}}\big )+\frac{1}{r^{2}\sin^{2}{\theta}}\big(\frac{\partial^{\,2}}{\partial{\phi^{2}}}\big ) 
\end{align}
یوں کروی محدد میں تابع وقت شروڈنگر مساوات درج ذیل ہو گی۔ 
\begin{multline}\label{مساوات_ابعادی_لاپلاسی_ب}
-\frac{\hslash^{2}}{2m}\big [\frac{1}{r^{2}}\frac{\partial}{\partial{r}}\big (r^{2}\frac{\partial\psi}{\partial{r}}\big )+\frac{1}{r^{2}\sin{\theta}}\frac{\partial}{\partial{\theta}}\big(\sin{\theta}\frac{\partial\psi}{\partial{\theta}}\big )+\frac{1}{r^{2}\sin^{2}{\theta}}\big(\frac{\partial^{2}\psi}{\partial{\phi^{2}}}\big )\big ]\\
+V\psi=E\psi 
\end{multline}
ہم ایسے حل کی تلاش میں ہیں جن کو حاصل ضرب کی صورت میں علیحدہ علیحدہ لکھنا ممکن ہو:
\begin{align}
\psi(r,\theta,\phi)=R(r)Y(\theta,\phi) 
\end{align}
اس کو مساوات \حوالہ{مساوات_ابعادی_لاپلاسی_ب} میں پر کر کے
\begin{multline*}
-\frac{\hslash^{2}}{2m}\big [\frac{Y}{r^{2}}\frac{\dif}{\dif{r}}\big (r^{2}\frac{\dif{R}}{\dif{r}}\big )+\frac{R}{r^{2}\sin{\theta}}\frac{\partial}{\partial{\theta}}\big(\sin{\theta}\frac{\partial{Y}}{\partial{\theta}}\big )+\frac{R}{r^{2}\sin^{2}{\theta}}\frac{\partial^{2}{Y}}{\partial{\phi^{2}}}\big ]+VRY=ERY 
\end{multline*}
دونوں اطراف کو \عددی{RY} سے تقسیم کر کہ \عددی{-2mr^{2}/\hslash^{2}} سے ضرب دیتے ہیں۔
\begin{align*}
&\big\{\frac{1}{R}\frac{d}{\dif{r}}\big(r^{2}\frac{\dif{R}}{\dif{r}}\big)-\frac{2mr^{2}}{\hslash^{2}}[V(r)-E]\big\} \\
+\frac{1}{Y}&\big\{\frac{1}{\sin{\theta}}\big(\sin{\theta}\frac{\partial{Y}}{\partial{\theta}}\big)+\frac{1}{\sin^{2}{\theta}}\frac{\partial^{2}{Y}}{\partial{\phi^{2}}}\big\}=0 
\end{align*}
پہلی خمدار قوسین میں جزو صرف \عددی{r} کا تابع ہے جبکہ باقی حصہ صرف  \عددی{\theta}   اور   \عددی{\phi}   کا تابع ہے؛  لہٰذا دونوں حصے انفرادی طور پر ایک مستقل کے برابر ہوں گے۔ اس علیحدگی مستقل کو ہم  \عددی{l(l+1)} روپ میں لکھتے ہیں جس کی وجہ کچھ دیر میں واضح ہو گی۔\حاشیہد{ایسا کرنے سے ہم عمومیت نہیں کھوتے ہیں، چونکہ یہاں \عددی{l} کوئی بھی مخلوط عدد ہو سکتا ہے۔ بعد میں ہم دیکھیں گے کہ \عددی{l} کو لازماً عدد صحیح ہونا ہو گا۔اسی نتیجہ کو ذہن میں رکھتے ہوئے میں نے علیحدگی مستقل کو اس عجیب روپ میں لکھا ہے۔}
\begin{align}
\frac{1}{R}\frac{d}{\dif{r}}\big(r^{2}\frac{\dif{R}}{\dif{r}}\big)-\frac{2mr^{2}}{\hslash^{2}}[V(r)-E]&=l(l+1)\label{مساوات_ابعاد_رداسی_الف} \\ 
\frac{1}{Y}\big\{\frac{1}{\sin{\theta}}\big(\sin{\theta}\frac{\partial{Y}}{\partial{\theta}}\big)+\frac{1}{\sin^{2}{\theta}}\frac{\partial^{2}{Y}}{\partial{\phi^{2}}}\big\}&=-l(l+1) \label{مساوات_ابعاد_غیر_رداسی_الف}
\end{align}
\ابتدا{سوال} 
کارتیسی محدد میں علیحدگی متغیرات  استعمال کرتے ہوئے لامتناہی مربعی کنواں (یا ڈبہ میں ایک ذرہ):
\begin{align*}
V(x,y,z)=\begin{cases}
0&\text{\RL{
اگر $x$، $y$ اور $z$ تینوں $0$ اور $a$ کے بیچ پائے جاتے ہوں
}}\\
\infty&\text{\RL{دیگر صورت}}
\end{cases} 
\end{align*}
حل کریں۔
\begin{enumerate}[a.]
\item
ساکن حالات  اور ان کی مطابقتی توانائیاں دریافت کریں۔
\item
بڑھتی توانائی کے لحاظ سے انفرادی توانائیوں کو \عددی{E_1}، \عددی{E_2}، \عددی{E_3}، وغیرہ، وغیرہ  سے ظاہر کر کے \عددی{E_1} تا \عددی{E_6} تلاش کریں۔ ان کی انحطاطیت (یعنی ایک ہی توانائی کے مختلف حلوں کی تعداد)  معلوم کریں۔ \ترچھا{تبصرہ:}  یک بعدی صورت میں انحطاطی مقید حالات نہیں پائے جاتے ہیں (سوال \حوالہء{2.45})، تاہم  تین ابعادی صورت میں یہ کثرت سے پائے جاتے ہیں۔
\item
توانائی  \عددی{E_{14}}   کی انحطاطیت کیا ہے اور یہ صورت کیوں دلچسپ ہے؟
\end{enumerate}
\انتہا{سوال}

\جزوحصہ{زاویائی مساوات}
مساوات \حوالہ{مساوات_ابعاد_غیر_رداسی_الف} متغیرات  \عددی{\theta}   اور    \عددی{\phi}  پر  \عددی{\psi} کی تابعیت  تعین کرتی ہے۔
اس کو  \عددی{Y\sin^{2}{\theta}}   سے ضرب دے کر درج ذیل حاصل ہو گا۔
\begin{align}\label{مساوات_ابعادی_زاویائی_مساوات}
\sin{\theta}\frac{\partial}{\partial{\theta}}\big(\sin{\theta}\frac{\partial{Y}}{\partial{\theta}}\big)+\frac{\partial^{2}{Y}}{\partial{\phi^{2}}}=-l(l+1)Y\sin^{2}{\theta} 
\end{align}
ہو سکتا ہے آپ اس مساوات کو پہچانتے ہوں۔ یہ کلاسیکی برقی حرکیات میں مساوات لاپلاس کے حل میں پائی جاتی ہے۔ ہمیشہ کی طرح ہم علیحدگی متغیرات:
\begin{align}
Y(\theta,\phi)=\Theta(\theta)\Phi(\phi) 
\end{align}
استعمال کر کے دیکھنا چاہیں گے۔ اس کو پر کر کے \عددی{\Theta\Phi} سے تقسیم کر کہ درج ذیل حاصل ہو گا۔ 
\begin{align*}
\big\{\frac{1}{\Theta}\big[\sin{\theta}\frac{\dif}{\dif\theta}\big(\sin\theta\frac{\dif{\Theta}}{\dif{\theta}}\big)\big]+l(l+1)\sin^{2}{\theta}\big\}+\frac{1}{\Phi}\frac{\dif^{\,2}{\Phi}}{\dif{\phi^{\,2}}}=0 
\end{align*}
پہلا جزو صرف  \عددی{\theta}   کا تفاعل ہے، جبکہ دوسرا صرف   \عددی{\phi}     کا تفاعل ہے، لہٰذا ہر ایک جزو ایک مستقل ہو گا۔ اس مرتبہ ہم علیحدگی مستقل\حاشیہد{یہاں بھی ہم عمومیت نہیں کھوتے ہیں، چونکہ \عددی{m} کوئی بھی مخلوط عدد ہو سکتا ہے؛ اگرچہ ہم جلد دیکھیں گے کہ \عددی{m} کو عدد صحیح ہونا ہو گا۔ \ترچھا{انتباہ:} اب حرف \عددی{m} دو مختلف چیزوں، کمیت اور علیحدگی مستقل، کو ظاہر کر رہا ہے۔امید ہے کہ آپ کو درست معنی جاننے میں مشکل درپیش نہیں ہو گی۔} کو  \عددی{m^{2}}   لکھتے ہیں۔
\begin{align}
\frac{1}{\Theta}\big[\sin{\theta}\frac{\dif}{\dif\theta}\big(\sin\theta\frac{\dif{\Theta}}{\dif{\theta}}\big)\big]+l(l+1)\sin^{2}{\theta}&=m^{2} \\
\frac{1}{\Phi}\frac{\dif^{\,2}{\Phi}}{\dif{\phi^{2}}}&=-m^{2} 
\end{align}
 متغیر \عددی{\phi}  کی مساوات زیادہ آسان ہے۔
 \begin{align}
\frac{\dif^{\,2}{\Phi}}{\dif{\phi^{2}}}=-m^{2}\Phi\implies \Phi(\phi)=e^{im\phi} 
\end{align}
[درحقیقت  دو حل پائے جاتے ہیں: \عددی{e^{im\phi}} اور \عددی{e^{-im\phi}}، تاہم \عددی{m} کو منفی ہونے کی اجازت دے کر ہم موخر الذکر کو بھی درج بالا حل میں شامل کرتے ہیں۔ اس کے علاوہ حل میں  جزو ضربی مستقل بھی پایا جا  سکتا ہے جسے ہم  \عددی{\Theta}   میں ضم کرتے ہیں۔ چونکہ برقی مخفی توانائی لازماً \ترچھا{حقیقی} ہو گی لہٰذا برقی حرکیات میں اسّمتی تفاعل  \عددی{(\Phi)}   کو سائن اور کوسائن کی صورت میں نہ کہ قوت نمائی صورت میں لکھا جاتا ہے۔ کوانٹم میکانیات میں ایسی کوئی پابندی نہیں پائی جاتی ہے اور قوت نمائی کے ساتھ کام کرنا زیادہ آسان ہوتا ہے۔] اب جب بھی  \عددی{\phi}   کی قیمت  میں \عددی{2\pi}  کا اضافہ آئے، ہے ہم فضا میں واپس اسی نقطہ پر پہنچتے ہیں (شکل \حوالہء{4۔1} دیکھیں) لہٰذا درج ذیل شرط\حاشیہد{یہ بظاہر معصوم شرط اتنی معصوم نہیں ہے۔ یاد رہے کہ \عددی{m} کی قیمت سے قطع نظر، احتمال کثافت \عددی{(\abs{\Phi}^2)} یک قیمتی ہے۔ ہم حصہ \حوالہء{4.3} میں ایک مختلف طریقہ سے، زیادہ پر زور دلیل پیش کر کے \عددی{m} پر مسلط شرط حاصل کریں گے۔} مسلط کی جا سکتی ہے۔
\begin{align}
\Phi(\phi+2\pi)=\Phi(\phi) 
\end{align}
دوسرے لفظوں میں  \عددی{e^{im(\phi+2\pi)}=e^{im\phi}} یا \عددی{e^{2\pi im}=1} ہو گا جس کے تحت \عددی{m} لازماً عدد صحیح ہو گا۔
\begin{align}
m=0,\pm 1,\pm 2,\cdots
\end{align}
 مساوات \عددی{\theta} 
\begin{align}\label{مساوات_ابعادی_تھیٹا_مساوات}
\sin{\theta}\frac{\dif}{\dif{\theta}}\big(\sin{\theta}\frac{\dif{\Theta}}{\dif{\theta}}\big)+[l(l+1)\sin^{2}{\theta}-m^{2}]\Theta=0 
\end{align}
اتنی سادہ نہیں ہے۔ اس کا حل درج ذیل ہے
\begin{align}
\Theta(\theta)=AP_{l}^{m}(\cos{\theta}) 
\end{align}
جہاں  \عددی{P_{l}^{m}}  \اصطلاح{شریک لیژانڈر تفاعل}\فرہنگ{لیژانڈر!شریک}\حاشیہب{associated Legendre function}\فرہنگ{Legendre!associated}  ہے جس کی تعریف درج ذیل ہے
\begin{align}\label{مساوات_ابعادی_شریک_لیژانڈر_تفاعلات_تعریف}
P_{l}^{m}(x)\equiv (1-x^{2})^{\abs{m}/2}\big(\frac{\dif}{\dif{x}}\big)^{\abs{m}}P_{l}(x) 
\end{align}
اور \عددی{l} ویں لیژانڈر کثیر رکنی  کو  \عددی{P_{l}(x)}   ظاہر کرتا ہے\حاشیہد{دھیان رہے کہ \عددی{P_l^{-m}=P_l^m} ہو گا۔} جس کی تعریف \اصطلاح{کلیہ روڈریگیس}\فرہنگ{روڈریگیس!کلیہ}\حاشیہب{Rodrigues formula}\فرہنگ{Rodrigues formula}
\begin{align}\label{مساوات_ابعادی_روڈریگیس}
P_{l}(x)\equiv\frac{1}{2^{l}l!}\big(\frac{\dif}{\dif{x}}\big)^{l}(x^{2}-1)^{l} 
\end{align}
دیتا ہے۔مثال کے طور پر درج ذیل ہونگے۔
\begin{align*}
P_{0}(x)&=1,\quad P_{1}(x)=\frac{1}{2}\frac{\dif}{\dif{x}}(x^{2}-1)=x ,\\
P_{2}(x)&=\frac{1}{4\cdot 2}\big(\frac{\dif}{\dif{x}}\big)^{2}(x^{2}-1)^{2}=\frac{1}{2}(3x^{2}-1) 
\end{align*}
%
\begin{table}
\caption{ابتدائی چند لیژانڈر کثیر رکنیاں۔}
\label{جدول_ابعاد_لیژانڈر_چند_ابتدائی}
\centering
\begin{tabular}{l}
$P_0=1$\\[0.25em]
$P_1=x$\\[0.25em]
$P_2=\frac{1}{2}(3x^2-1)$\\[0.25em]
$P_3=\frac{1}{2}(5x^3-3x)$\\[0.25em]
$P_4=\frac{1}{8}(35x^4-30x^2+3)$\\[0.25em]
$P_5=\frac{1}{8}(63x^5-70x^3+15x)$
\end{tabular}
\end{table}
جدول \حوالہ{جدول_ابعاد_لیژانڈر_چند_ابتدائی} میں ابتدائی چند لیژانڈر کثیر رکنیاں پیش کی گئی ہیں۔ جیسا کہ نام سی ظاہر ہے، \عددی{P_{l}(x)} متغیر \عددی{x} کی درجہ \عددی{l} کثیر رکنی ہے، اور \عددی{l} کی قیمت طے  کرتی ہے کہ آیا یہ جفت کا طاق ہو گی۔ تاہم  \عددی{P_{l}^{m}(x)}  عموماً کثیر رکنی نہیں ہو گا؛ اور طاق \عددی{m} کی صورت میں اس میں \عددی{\sqrt{1-x^2}} کا جزو ضربی پایا جائے گا:
\begin{align*}
P_{2}^{0}(x)&=\frac{1}{2}(3x^{2}-1), \quad P_{2}^{1}(x)=(1-x^{2})^{1/2}\frac{\dif}{\dif{x}}\big[\frac{1}{2}(3x^{2}-1)\big]=3x\sqrt{1-x^{2}}, \\
P_{2}^{2}(x)&=(1-x^{2})\big(\frac{\dif}{\dif{x}}\big)^{2}\big[\frac{1}{2}(3x^{2}-1)\big]=3(1-x^{2}),
\end{align*}
وغیرہ وغیرہ۔  (اب ہمیں \عددی{P_{l}^{m}(\cos\theta)} چاہیے اور چونکہ \عددی{\sqrt{1-\cos^{2}\theta}=\sin\theta} ہوتا ہے لہٰذا \عددی{P_{l}^{m}(\cos\theta)} ہر صورت \عددی{\cos\theta} کا کثیر رکنی ہو گا جسے طاق \عددی{m} کی صورت میں
 \عددی{\sin\theta} ضرب کرے گا۔ جدول \حوالہ{جدول_ابعادی_شریک_لیژانڈر_تفاعلات} میں \عددی{\cos\theta} کے چند شریک لیژانڈر تفاعلات پیش کیے گئے ہیں۔)
\begin{table}
\caption{چند شریک لیژانڈر تفاعلات \عددی{{P_l^m(\cos\theta)}}}
\label{جدول_ابعادی_شریک_لیژانڈر_تفاعلات}
\centering
\begin{tabular}{ll}
$P_0^0=1$ & $P_2^0=\frac{1}{2}(3\cos^2\theta-1)$\\[0.25em]
$P_1^1=\sin\theta$ & $P_3^3=15\sin\theta(1-\cos^2\theta)$\\[0.25em]
$P_1^0=\cos\theta$ & $P_3^2=15\sin^2\theta\cos\theta$\\[0.25em]
$P_2^2=3\sin^2\theta$ & $P_3^1=\frac{3}{2}\sin\theta(5\cos^2\theta-1)$\\[0.25em]
$P_2^1=3\sin\theta\cos\theta$ & $P_3^0=\frac{1}{2}(5\cos^3\theta-3\cos\theta)$
\end{tabular}
\end{table}

دھیان رہے کہ صرف غیر منفی عدد صحیح \عددی{l} کی صورت میں کلیہ روڈریگیس معنی خیز ہو گا؛ مزید \عددی{|m|\textgreater{l}} کی صورت میں مساوات \حوالہ{مساوات_ابعادی_شریک_لیژانڈر_تفاعلات_تعریف} کے تحت \عددی{P_{l}^{m}=0} ہو گا۔ یوں \عددی{l} کی کسی بھی مخصوص قیمت کے لئے \عددی{m} کی \عددی{(2l+1)} ممکنہ قیمتیں ہوں گی:
\begin{align}\label{مساوات_ابعادی_انحطاطی_قیمتیں}
l=0,1,2,\dotsc;\quad m=-l,-l+1,\dotsc-1,0,1,\dotsc l-1,l
\end{align}
ذرا رکیے! مساوات  \حوالہ{مساوات_ابعادی_تھیٹا_مساوات} دو رتبی تفرقی مساوات ہے:  \عددی{l} اور \عددی{m} کی کسی بھی قیمتوں کے لئے اس کے  دو خطی غیر تابع حل ہونگے۔ باقی حل کہاں ہیں؟ \ترچھا{جواب:} یقیناً  تفرقی مساوات کے ریاضی حلوں کی  صورت میں باقی حل ضرور موجود ہوں گے تاہم  \عددی{\theta=0} اور (یا) \عددی{\theta=\pi} پر ایسے حل بے قابو بڑھتے ہیں (سوال \حوالہ{سوال_ابعادی_طبعی_نا_قابل_قبول} دیکھیں ) جس کی بنا   یہ طبعی طور  پر ناقابل قبول ہوں گے۔ 

کروی محدد میں حجمی رکن درج ذیل ہو گا
\begin{align}
\dif^{\,3}\kvec{r}=r^{2}\sin\theta\dif r\dif \theta\dif \phi
\end{align}
لہٰذا معمول زنی شرط (مساوات \حوالہ{مساوات_ابعادی_معمول_زنی_شرط}) درج ذیل روپ اختیار کرتی ہے۔
\begin{align*}
\int\abs{\psi}^{2}r^{2}\sin\theta\dif r \dif\theta\dif\phi=\int\abs{R}^{2}r^{2}\dif r\int\abs{Y}^{2}\sin\theta\dif\theta\dif\phi=1 
\end{align*}
یہاں \عددی{R} اور \عددی{Y} کو علیحدہ علیحدہ  معمول پر لانا زیادہ آسان ثابت ہوتا ہے۔
\begin{align}\label{مساوات_ابعاد_علیحدہ_معمول_زنی_شرائط}
\int_{0}^{\infty}\abs{R}^{2}r^{2}\dif r=1\quad{\text{اور}}\quad \int_{0}^{2\pi}\int_{0}^{\pi}\abs{Y}^{2}\sin\theta\dif \theta\dif \phi=1 
\end{align}
معمول شدہ  زاویائی موجی تفاعلات\حاشیہد{معمول زنی مستقل کو سوال \حوالہء{4.54} میں حاصل کیا گیا ہے؛  نظریہ زاویائی معیار حرکت میں مستعمل علامتیت کے ساتھ  ہم آہنگی کی خاطر \عددی{\epsilon} (جس کی قیمت \عددی{1} یا \عددی{-1} ہو گی) کی علامت کا انتخاب کیا گیا ہے۔ دھیان رہے کہ \عددی{Y_l^{-m}=(-1)^m(Y_l^m)^*} ہو گا۔} کو \اصطلاح{کروی ہارمونیات}\فرہنگ{کروی!ہارمونیات}\حاشیہب{spherical harmonics}\فرہنگ{spherical!harmonics} کہتے ہیں :
\begin{align}\label{مساوات_ابعادی_کروی_ہارمونیات}
Y_{l}^{m}(\theta,\phi)=\epsilon\sqrt{\frac{(2l+1)}{4\pi}\frac{(l-\abs{m})!}{(l+\abs{m})!}}e^{i{m\phi}}P_{l}^{m}(\cos\theta 
\end{align}
جہاں \عددی{m\geq0} کے لئے \عددی{\epsilon=(-1)^{m}}  اور \عددی{m\leq0}  کے لئے \عددی{\epsilon=1} ہو گا۔ جیسا کہ ہم بعد میں ثابت کریں گے، کروی ہارمونیات عمودی ہیں لہٰذا درج ذیل ہو گا۔
\begin{align}\label{مساوات_ابعادی_کروی_ہارمونی_عمودیت}
\int_{0}^{2\pi}\int_{0}^{\pi}[Y_{l}^{m}(\theta,\phi)]^*[Y_{l'}^{m'}(\theta,\phi)]\sin\theta\dif\theta\dif\phi=\delta_{ll'}\delta_{mm'} 
\end{align}
 جدول \حوالہ{جدول_ابعادی_کروی_ہارمونیات} میں چند ابتدائی  کروی ہارمونیات پیش کیے گئے ہیں۔ تاریخی وجوہات کی بنا \عددی{l} کو \اصطلاح{اسّمتی کوانٹائی عدد}\فرہنگ{کوانٹائی عدد!اسّمتی}\حاشیہب{azimuthal quantum number}\فرہنگ{quantum number!azimuthal} جب کہ \عددی{m} کو \اصطلاح{مقناطیسی کوانٹائی عدد}\فرہنگ{کوانٹائی عدد!مقناطیسی}\حاشیہب{magnetic quantum number}\فرہنگ{quantum number!magnetic} کہتے ہیں۔
\begin{table}
\caption{ابتدائی چند کروی ہارمونیات، \عددی{Y_l^m(\theta,\phi)}}
\label{جدول_ابعادی_کروی_ہارمونیات}
\renewcommand{\arraystretch}{2} 
\centering
\begin{tabular}{ll}
$Y_0^0=(\frac{1}{4\pi})^{1/2}$ & $Y_2^{\pm 2}=(\frac{15}{32\pi})^{1/2}\sin^2\theta e^{\pm 2 i \phi}$\\
$Y_1^0=(\frac{3}{4\pi})^{1/2}\cos\theta$ & $Y_3^0=(\frac{7}{16\pi})^{1/2}(5\cos^3\theta-3\cos\theta)$\\
$Y_1^{\pm 1}=\mp(\frac{3}{8\pi})^{1/2}\sin\theta e^{\pm i\phi}$ & $Y_3^{\pm 1}=\mp(\frac{21}{64\pi})^{1/2}\sin\theta(5\cos^2\theta-1)e^{\pm i\phi}$\\
$Y_2^0=(\frac{5}{16\pi})^{1/2}(3\cos^2\theta-1)$ & $Y_3^{\pm 2}=(\frac{105}{32\pi})^{1/2}\sin^2\theta\cos\theta e^{\pm 2 i \phi}$\\
$Y_2^{\pm 1}=\mp(\frac{15}{8\pi})^{1/2}\sin\theta\cos\theta e^{\pm i\phi}$ & $Y_3^{\pm3}=\mp(\frac{35}{64\pi})^{1/2}\sin^3\theta e^{\pm 3 i \phi}$
\end{tabular}
\end{table}
\ابتدا{سوال}
مساوات \حوالہ{مساوات_ابعادی_شریک_لیژانڈر_تفاعلات_تعریف}، \حوالہ{مساوات_ابعادی_روڈریگیس} اور \حوالہ{مساوات_ابعادی_کروی_ہارمونیات} استعمال کر کے \عددی{Y_{0}^{0}} اور \عددی{Y_{2}^{1}} تیار کریں۔ تصدیق کریں کہ یہ معمول شدہ اور عمودی ہیں۔
\انتہا{سوال}
%
\ابتدا{سوال}\شناخت{سوال_ابعادی_طبعی_نا_قابل_قبول}
دکھائیں کہ \عددی{l=m=0} کے لئے
\begin{align*}
\Theta(\theta)=A\ln[\tan(\theta/2)] 
\end{align*}
 مساوات \عددی{\theta} (مساوات \حوالہ{مساوات_ابعادی_تھیٹا_مساوات}) کو مطمئن کرتی ہے۔یہ (وہ) ناقابل قبول دوسرا حل ہے؛ اس میں کیا خرابی ہے؟
\انتہا{سوال}
%
\ابتدا{سوال}
مساوات \حوالہ{مساوات_ابعادی_کروی_ہارمونیات} استعمال کر کے \عددی{Y_{l}^{l}(\theta,\phi)} اور 
\عددی{Y_{3}^{2}(\theta,\phi)} تشکیل دیں۔(آپ \عددی{P_{3}^{2}} کو جو جدول \حوالہ{جدول_ابعادی_شریک_لیژانڈر_تفاعلات} سے دیکھ سکتے ہیں، جبکہ \عددی{P_{l}^{l}} آپ کو مساوات \حوالہ{مساوات_ابعادی_شریک_لیژانڈر_تفاعلات_تعریف} اور \حوالہ{مساوات_ابعادی_روڈریگیس} کی مدد سے  تشکیل دینا ہو گا۔) تصدیق کیجیے کہ  \عددی{l} اور \عددی{m} کی موزوں قیمتوں کیلئے یہ زاویائی مساوات (مساوات \حوالہ{مساوات_ابعادی_زاویائی_مساوات}) کو  مطمئن کرتے ہیں۔
\انتہا{سوال}
%
\ابتدا{سوال}
کلیہ روڈریگیس سے ابتدا کر کے  لیژانڈر  کثیر رکنیوں کی معیاری عمودیت کی شرط:
\begin{align}
\int_{-1}^{1}P_{l}(x)P_{l'}(x)\dif x=\big(\frac{2}{2l+1}\big)\delta_{ll'} 
\end{align}
اخذ کریں۔ (\ترچھا{اشارہ:} تکمل بالحصص استعمال کریں۔)
\انتہا{سوال}

\جزوحصہ{رداسی مساوات}
دھیان رہے کہ تمام کروی تشاکلی  مخفیہ کے لئے تفاعل موج کا زاویائی حصہ، \عددی{Y(\theta,\phi)}،  ایک  دوسرے جیسا ہو گا؛ مخفیے 
\عددی{V(r)} کی شکل و صورت تفاعل موج کے صرف رداسی حصہ، \عددی{R(r)}، پر اثر انداز ہو گی جسے مساوات \حوالہ{مساوات_ابعاد_رداسی_الف} تعین کرتی ہے۔
\begin{align}
\frac{\dif}{\dif r}\big(r^{2}\frac{\dif R}{\dif r}\big)-\frac{2mr^{2}}{\hslash^{2}}[V(r)-E]R=l(l+1)R
\end{align}
نئے متغیرات استعمال کرتے ہوئے اس مساوات کی سادہ روپ حاصل کی جا سکتی ہے: درج ذیل لینے سے
\begin{align}\label{مساوات_ابعادی_نئے_متغیر_رداسی}
u(r)\equiv{rR(r)} 
\end{align}  
\عددی{R=u/r}، \عددی{\dif R/\dif r=[r(\dif u/\dif r)-u]/r^{2}}، \عددی{(\dif /\dif r)[r^{2}(\dif R/\dif r)]=r\dif^{\,2}u/\dif r^{2}}  لہٰذا درج ذیل ہو گا۔
\begin{align}\label{مساوات_ابعادی_رداسی}
-\frac{\hslash^{2}}{2m}\frac{\dif^{\,2}u}{\dif r^{2}}+\big[V+\frac{\hslash^{2}}{2m}\frac{l(l+1)}{r^{2}}\big]u=Eu
\end{align}
اس کو \اصطلاح{رداسی مساوات}\فرہنگ{رداسی مساوات}\حاشیہب{radial equation}\فرہنگ{radial equation} کہتے ہیں\حاشیہد{یہاں \عددی{m} کمیت کو ظاہر کرتی ہے؛ رداسی مساوات میں علیحدگی مستقل \عددی{m} نہیں پایا جاتا ہے۔} جو شکل و صورت کے لحاظ سے یک بعدی شروڈنگر مساوات (مساوات \حوالہ{مساوات_شروڈنگر_علیحدہ_دوم}) کی طرح ہے، تاہم یہاں \اصطلاح{موثر مخفیہ}\فرہنگ{مخفیہ!موثر}\حاشیہب{effective potential}\فرہنگ{potential!effective} درج ذیل ہے
\begin{align}
V_{\text{موثر}}=V+\frac{\hslash^{2}}{2m}\frac{l(l+1)}{r^{2}} 
\end{align}
 جس میں \عددی{(\hslash^{2}/2m)[l(l+1)/r^{2}]}  اضافی جزو پایا جاتا ہے جو \اصطلاح{مرکز گریز جزو}\فرہنگ{مرکز گریز جزو}\حاشیہب{centrifugal term}\فرہنگ{centrifugal term} کہلاتا ہے۔ یہ کلاسیکی میکانیات کے مرکز گریز (مجازی) قوت کی طرح، ذرہ کو (مبدا سے دور) باہر جانب دھکیلتا ہے۔ یہاں معمول زنی شرط (مساوات \حوالہ{مساوات_ابعاد_علیحدہ_معمول_زنی_شرائط}) درج ذیل روپ اختیار کرتی ہے۔ 
\begin{align}
\int_{0}^{\infty}\abs{u}^{2}\dif r=1 
\end{align}
کسی مخصوص مخفیہ \عددی{V(r)}   کے بغیر ہم آگے نہیں بڑھ سکتے ہیں۔
%==================

\ابتدا{مثال}
درج ذیل لامتناہی کروی کنواں پر غور کریں۔
\begin{align}
V(r)=\begin{cases}
0&r\le {a}\\
\infty&r>a
\end{cases} 
\end{align}
اس کے تفاعلات موج اور اجازتی توانائیاں تلاش کریں۔

\ترچھا{حل:}\quad
کنواں کے باہر تفاعل موج صفر ہے جب کے کنواں کے اندر رداسی مساوات درج ذیل ہے
\begin{align}\label{مساوات_ابعادی_مربعی_کنواں_الف}
\frac{\dif^{\,2}u}{\dif r^{2}}=\big[\frac{l(l+1)}{r^{2}}-k^{2}\big]u 
\end{align}
جہاں ہمیشہ کی طرح درج ذیل ہو گا۔
\begin{align}
k\equiv\frac{\sqrt{2mE}}{\hslash} 
\end{align}
ہم نے اس مساوات کو، سرحدی شرط \عددی{u(a)=0} مسلط کر کے، حل کرنا ہے۔ سب سے آسان صورت \عددی{l= 0} کی ہے۔
\begin{align*}
\frac{\dif^{\,2}u}{\dif r^{2}}=-k^{2}u\implies {u(r)=A\sin(kr)+B\cos(kr)} 
\end{align*}
یاد رہے،  اصل رداسی تفاعل موج \عددی{R(r)=u(r)/r} ہے اور \عددی{r\to{0}}  کی صورت میں \عددی{[\cos(kr)]/r}  بے قابو بڑھتا ہے۔یوں ہمیں \عددی{B=0} منتخب\حاشیہد{درحقیقت ہم صرف اتنا چاہتے ہیں کہ تفاعل موج معمول پر لانے کے قابل ہو؛ یہ ضروری نہیں کہ یہ متناہی ہو: مساوات \حوالہ{مساوات_ابعاد_علیحدہ_معمول_زنی_شرائط} میں \عددی{r^2} کی بنا مبدا پر \عددی{R(r)\sim 1/r} معمول پر لانے کے قابل ہے۔} کرنا ہو گا۔ اب سرحدی شرط پر پورا اترنے کے لئے ضروری ہے  کہ \عددی{\sin(ka)=0} ہو لہٰذا
 \عددی{ka=n\pi} ہو گا جہاں \عددی{n}  عدد صحیح ہے۔ ظاہر ہے کہ اجازتی توانائیاں درج ذیل ہوں گی۔ 
\begin{align}
E_{n0}&=\frac{n^{2}\pi^{2}\hslash^{2}}{2ma^{2}},&&(n=1,2,3,...). 
\end{align}
جو عین یک بعدی لامتناہی چکور کنواں کی توانائیاں ہیں (مساوات \حوالہ{مساوات_شروڈنگر_لامتناہی_چکور_کنواں_توانائیاں})۔ \عددی{u(r)}
کو معمول پر لانے سے \عددی{A=\sqrt{2/a}} حاصل ہو گا۔ زاویائی جزو (جو \عددی{Y_{0}^{0}(\theta,\phi)=1/\sqrt{4\pi}}
کی بنا غیر اہم ہے) کو ساتھ منسلک کرتے ہوئے درج ذیل حاصل ہو گا۔ 
\begin{align}
\psi_{n00}=\frac{1}{\sqrt{2\pi a}}\frac{\sin(n\pi r/a)}{r} 
\end{align}
[دھیان کیجیے کہ ساکن حالات کے نام تین \اصطلاح{کوانٹائی اعداد}\فرہنگ{کوانٹائی اعداد}\حاشیہب{quantum numbers}\فرہنگ{quantum numbers} \عددی{n}، \عددی{l} اور \عددی{m} استعمال کر کے رکھے جاتے ہیں: \عددی{\psi_{nml}(r,\theta,\phi)}؛ جبکہ توانائی، \عددی{E_{nl}}،   صرف \عددی{n} اور \عددی{l} پر منحصر ہو گی۔] 

(ایک اختیاری عدد صحیح \عددی{l} کے لئے) مساوات \حوالہ{مساوات_ابعادی_مربعی_کنواں_الف} کا عمومی حل
\begin{align}
u(r)=Arj_{l}(kr)+Brn_{l}(kr). 
\end{align}
 بہت جانا پہچانا نہیں ہے  جہاں \عددی{j_{l}(x)} رتبہ \عددی{l} کا  \اصطلاح{کروی بیسل تفاعل}\فرہنگ{بیسل!کروی تفاعل}\حاشیہب{spherical Bessel function}\فرہنگ{Bessel!spherical function} ہے اور \عددی{n_{l}(x)} رتبہ \عددی{l} کا  \اصطلاح{کروی نیومن تفاعل}\فرہنگ{نیومن!کروی تفاعل}\حاشیہب{spherical Neumann function}\فرہنگ{Neumann!spherical function}  ہے جن کی تعریفات درج ذیل ہیں۔  
\begin{align}\label{مساوات_ابعادی_تعریفات_کروی_بیسل_نیومن}
j_{l}(x)\equiv(-x)^{l}\big(\frac{1}{x}\frac{\dif}{\dif x}\big)^{l}\frac{\sin x}{x};\quad n_{l}(x)\equiv-(-x)^{l}
\big(\frac{1}{x}\frac{\dif}{\dif x}\big)^{l}\frac{\cos x}{x}
\end{align}
مثال کے طور پر درج ذیل ہوں گے، وغیرہ وغیرہ۔
\begin{align*}
j_{0}(x)&=\frac{\sin x}{x};\quad n_{0}(x)=-\frac{\cos x}{x};\\
j_1(x)&=(-x)\frac{1}{x}\frac{\dif}{\dif x}\big(\frac{\sin x}{x}\big)=\frac{\sin x}{x^{2}}-\frac{\cos x}{x};\\
j_{2}(x)&=(-x)^{2}\big(\frac{1}{x}\frac{\dif}{\dif x}\big)^{2}\frac{\sin x}{x}=x^{2}\big(\frac{1}{x}\frac{\dif}{\dif x}\big)\frac{x\cos x-\sin x}{x^{3}}\\
&=\frac{3\sin x-3x\cos x-x^2\sin x}{x^{3}}
\end{align*}
\begin{table}
\caption{
ابتدائی چند کروی بیسل اور نیومن تفاعلات، \عددی{j_n(x)} اور \عددی{n_l(x)}؛ چھوٹی \عددی{x} کے لئے متقاربی روپ۔
}
\label{جدول_ابعادی_کروی_بیسل_نیومن_تفاعلات}
\centering
\renewcommand{\arraystretch}{2} 
\begin{tabular}{ll}
\toprule
$j_0=\frac{\sin x}{x}$  &  $n_0=-\frac{\cos x}{x}$\\
$j_1=\frac{\sin x}{x^2}-\frac{\cos x}{x}$  &  $n_1=-\frac{\cos x}{x^2}-\frac{\sin x}{x}$\\
$j_2=\big(\frac{3}{x^3}-\frac{1}{x}\big)\sin x-\frac{3}{x^2}\cos x$  &  $n_2=-\big(\frac{3}{x^3}-\frac{1}{x}\big)\cos x-\frac{3}{x^2}\sin x$\\[0.5em]
\midrule
$j_l\to\frac{2^l l!}{(2l+1)!}x^l$  &  $n_l\to -\frac{(2l)!}{2^ll!}\frac{1}{x^{l+1}},\quad x\ll 1$\\
\bottomrule
\end{tabular}
\end{table}
جدول \حوالہ{جدول_ابعادی_کروی_بیسل_نیومن_تفاعلات} میں ابتدائی چند کروی بیسل اور نیومن تفاعلات پیش کیے گئے ہیں۔ متغیر \عددی{x} کی چھوٹی قیمت کے لئے جہاں
\begin{align*}
\sin x\approx x-\frac{x^{3}}{3!}+\frac{x^{5}}{5!}-\dotsb\quad \text{اور}\quad  \cos x \approx 1-\frac{x^{2}}{2!}+\frac{x^{4}}{4!}-\dotsb
\end{align*}
 ہوں گے، درج ذیل ہوں گے، وغیرہ وغیرہ۔
\begin{align*}
j_{0}(x)\approx1;\quad{n_{0}(x)\approx-\frac{1}{x}};\quad{j_{1}(x)\approx\frac{x}{3}};\quad{j_{2}(x)\approx\frac{x^{2}}{15}}; 
\end{align*}
 دھیان رہے کہ مبدا  پر بیسل تفاعلات متناہی  ہیں جبکہ مبدا پر نیومن تفاعلات بے قابو بڑھتے ہیں۔ یوں ہمیں لازماً  \عددی{B_{l}=0} منتخب کرنا ہو گا لہٰذا درج ذیل ہو گا۔
\begin{align}
R(r)=Aj_{l}(kr) 
\end{align}
اب سرحدی شرط \عددی{R(a)=0} کو مطمئن کرنا باقی ہے۔ ظاہر ہے کہ \عددی{k} کو درج ذیل کے تحت منتخب کرنا ہو گا
\begin{align}
j_{l}(ka)=0
\end{align}
یعنی \عددی{l} رتبی کروی بیسل تفاعل کا \عددی{(ka)} ایک صفر ہو گا۔ اب بیسل تفاعلات ارتعاشی ہیں (شکل \حوالہء{4.2} دیکھیں)؛ ہر ایک کے لامتناہی تعداد  صفر پائے جاتے ہیں۔ تاہم (ہماری بدقسمتی سے) یہ ایک جیسے فاصلوں پر نہیں پائے جاتے ہیں (جیسا کہ نقاط \عددی{n} یا نقاط
\عددی{n\pi}، وغیرہ پر)؛ انہیں اعدادی تراکیب سے حاصل کرنا ہو گا۔ بہرحال سرحدی شرط کے تحت درج ذیل ہو گا
\begin{align}
k=\frac{1}{a}\beta_{nl} 
\end{align}
جہاں \عددی{\beta_{nl}} رتبہ  \عددی{l}   کروی بیسل تفاعل کا \عددی{n} واں صفر   ہو گا۔ یوں اجازتی توانائیاں 
\begin{align}\label{مساوات_ابعاد_کروی_کنواں_اجازتی_توانائیاں}
E_{nl}=\frac{\hslash^{2}}{2ma^{2}}\beta_{nl}^{2}. 
\end{align}
اور تفاعلات موج درج ذیل ہوں گے 
\begin{align}
\psi_{nlm}(r,\theta,\phi)=A_{nl}j_{l}(\beta_{nl}r/a)Y_{l}^{m}(\theta,\phi). 
\end{align}
جہاں مستقل \عددی{A_{n1}} کا تعین معمول زنی سے کیا جاتا ہے۔چونکہ \عددی{l} کی ہر ایک قیمت کے لئے \عددی{m} کی \عددی{(2l+1)}
مختلف قیمتیں پائی جاتی ہیں لہٰذا توانائی کی ہر سطح \عددی{(2l+1)} گنّا انحطاطی ہو گی (مساوات \حوالہ{مساوات_ابعادی_انحطاطی_قیمتیں} دیکھیں)۔
\انتہا{مثال}
%===============
\ابتدا{سوال}
\begin{enumerate}[a.]
\item
کروی نیومن تفاعلات \عددی{n_{1}(x)} اور \عددی{n_{2}(x)} کو (مساوات \حوالہ{مساوات_ابعادی_تعریفات_کروی_بیسل_نیومن}) میں پیش کی گئی تعریفات سے تیار کریں۔
\item
سائن اور کوسائن کو پھیلا کر \عددی{x\ll1} کے لئے كارآمد  \عددی{n_{1}(x)} اور \عددی{n_{2}(x)} کے  تخمینی کلیات  اخذ کریں۔تصدیق کریں کہ یہ مبدا پر بےقابو بڑھتے ہیں۔
\end{enumerate}
\انتہا{سوال}
%
\ابتدا{سوال}
\begin{enumerate}[a.]
\item
 تصدیق کریں کہ \عددی{V(r)=0} اور \عددی{l=1} کے لئے \عددی{Arj_{l}(kr)} رداسی مساوات کو مطمئن کرتا ہے۔
\item
لامتناہی کروی کنواں کیلئے \عددی{l=1} کی صورت میں اجازتی توانائیاں ترسیم کی مدد سے تعین کریں۔ دکھائیں کہ \عددی{n} کی بڑی قیمت کے لئے 
\عددی{E_{n1}\approx(\hslash^{2}\pi^{2}/2ma^{2})(n+1/2)^{2}} ہو گا۔(\ترچھا{اشارہ:} پہلے  
\عددی{j_{1}(x)=0\implies {x}=\tan x} دکھائیں۔ اس کے بعد \عددی{x} اور \عددی{\tan x} کو ایک ساتھ ترسیم کرتے ہوئے ان کے نقاط تقاطع تلاش کریں۔)
\end{enumerate}
\انتہا{سوال}
%
\ابتدا{سوال}
 ایک ذرہ جس کی کمیت \عددی{m} ہے کو \ترچھا{متناہی} کروی کنواں:
\begin{align*}
V(r)=\begin{cases}-V_{0}&r\le a\\0&r>a\\\end{cases} 
\end{align*}
میں رکھا جاتا ہے۔ اس کا زمینی حال، \عددی{l=0}  کے لئے،  رداسی مساوات کے حل سے حاصل کریں۔ دکھائیں کے
\عددی{V_{0}a^{2}<\pi^{2}\hslash^{2}/8m} کی صورت میں کوئی مقید حال نہیں پایا جائے گا۔
\انتہا{سوال}
%======================
\حصہ{ہائیڈروجن جوہر}
ہائیڈروجن جوہر بار \عددی{e} کے  ایک بھاری پروٹان جس کے گرد بار \عددی{-e} کا ایک ہلکا الیکٹران طواف کرتا ہو پر مشتمل ہوتا ہے۔ پروٹان بنیادی طور پر ساکن رہتا ہے (جسے ہم مبدا پر تصور کر سکتے ہیں)۔  ان دونوں کے مخالف بار کے بیچ قوت کشش پائی جاتی ہے جو انہیں اکٹھے رکھتی ہے  (شکل \حوالہء{4.3} دیکھیں)۔  قانون کولمب کے تحت مخفی توانائی درج ذیل ہو گی  
 \begin{align}\label{مساوات_ابعادی_کولمب_مخفیہ}
V(r)=-\frac{e^{2}}{4\pi\epsilon_{0}}\frac{1}{r} 
\end{align}
لہٰذا رداسی مساوات \حوالہ{مساوات_ابعادی_رداسی} درج ذیل روپ اختیار کرے گی۔
\begin{align}\label{مساوات_ابعادی_رداسی_کولمب}
-\frac{\hslash^{2}}{2m}\frac{\dif^{\,2}{u}}{\dif r^{2}}+\big[-\frac{e^{2}}{4\pi\epsilon_{0}}\frac{1}{r}+\frac{\hslash^{2}}{2m}\frac{l(l+1)}{r^{2}}\big]u=Eu 
\end{align}
 ہم نے اس مساوات کو  \عددی{u(r)}   کے لئے حل کر کے اجازتی توانائیاں \عددی{E} تعین کرنی ہیں۔  ہائیڈروجن جوہر کا حل نہایت اہم ہے لہٰذا میں اس کو، ہارمونی مرتعش کے تحلیلی حل کی ترکیب سے، قدم با قدم حل کر کے پیش کرتا ہوں۔ (جس قدم پر آپ کو دشواری پیش آئے، حصہ \حوالہ{حصہ_شروڈنگر_تحلیلی_ترکیب} سے مدد لیں جہاں مکمل تفصیل پیش کی گئی ہے۔)  کولمب مخفیہ، مساوات \حوالہ{مساوات_ابعادی_کولمب_مخفیہ}،    (\عددی{E>0} کے لئے)  \ترچھا{استمراریہ} حالات، جو الیکٹران پروٹون بکھراو کو ظاہر کرتے ہیں، تسلیم کرنے کے ساتھ ساتھ غیر مسلسل \ترچھا{مقید} حالات، جو ہائیڈروجن جوہر کو ظاہر کرتے ہے، بھی تسلیم کرتا ہے۔ ہماری دلچسپی موخر الذکر میں ہے۔

\جزوحصہ{رداسی تفاعل موج}
 سب سے پہلے نئی علامتیں متعارف کرتے ہوئے مساوات کی بہتر (صاف) صورت حاصل کرتے ہیں۔ درج ذیل متعارف کر کے (جہاں مقید حالات کے لئے \عددی{e} منفی ہونے کی وجہ سے \عددی{\kappa }  حقیقی ہو گا)
 \begin{align}\label{مساوات_ابعادی_متعارف_الف}
\kappa \equiv \frac{\sqrt{-2mE}}{\hslash} 
\end{align}
 مساوات \حوالہ{مساوات_ابعادی_رداسی_کولمب} کو \عددی{E} سے تقسیم کرنے سے
\begin{align*}
\frac{1}{\kappa ^{2}}\frac{\dif^{\,2}{u}}{\dif{r^{2}}}=\big[1-\frac{me^{2}}{2\pi\epsilon_{0}\hslash^{2}\kappa }\frac{1}{(\kappa r)}+\frac{l(l+1)}{(\kappa r)^{2}}\big]u 
\end{align*}
حاصل ہو گا جس کو دیکھ کر ہمیں خیال آتا ہے کہ ہم درج ذیل علامتیں متعارف کریں 
\begin{align}\label{مساوات_ابعادی_متعارف_ب}
\rho\equiv \kappa r, \quad \rho_{0}\equiv\frac{me^{2}}{2\pi\epsilon_{0}\hslash^{2}\kappa } 
\end{align}
لہٰذا درج ذیل لکھا جائے گا۔
\begin{align}\label{مساوات_ابعادی_رداسی_کولمب_نئے_متغیرات}
\frac{\dif^{\,2}{u}}{\dif{\rho^{2}}}=\big[1-\frac{\rho_{0}}{\rho}+\frac{l(l+1)}{\rho^{2}}\big]u 
\end{align}

اس کے بعد ہم حالات کی متقاربی روپ پر غور کرتے ہیں۔ اب \عددی{\rho\to\infty} کرنے سے قوسین کے اندر مستقل جزو غالب ہو گا لہٰذا (تخمیناً) درج ذیل لکھا جا سکتا ہے۔
\begin{align*}
\frac{\dif^{\,2}{u}}{\dif{\rho^{2}}}=u 
\end{align*}
اس کا عمومی حال درج ذیل ہے
\begin{align}\label{مساوات_ابعاد_عمومی_حل_بے_قابو}
u(\rho)=Ae^{-\rho}+Be^{\rho} 
\end{align}
تا ہم  (\عددی{\rho\to\infty} کی صورت میں) \عددی{e^{\rho}} بے قابو بڑھتا ہے لہٰذا ہمیں \عددی{B=0} لینا ہو گا۔ یوں \عددی{\rho}
کی بڑی قیمتوں کے لیے درج  ذیل ہو گا۔
\begin{align}
u(\rho)\sim Ae^{-\rho} 
\end{align}
اس کے برعکس \عددی{\rho\to 0} کی صورت میں مرکز گریز جزو غالب ہو گا؛\حاشیہد{یہ دلیل \عددی{l=0} کی صورت میں کارآمد نہیں ہو گی (اگرچہ مساوات \حوالہ{مساوات_ابعادی_ہائیڈروجن_مبدا_قریب} میں پیش نتیجہ اس صورت کے لئے بھی درست ہے)۔ بہرحال، میرا مقصد نئی علامتیت (مساوات \حوالہ{مساوات_ابعادی_نئی_علامتیت}) کے استعمال کے لئے راستہ ہموار کرنا ہے۔} لہٰذا تخمیناً  درج ذیل لکھا جا سکتا ہے۔ 
\begin{align*}
\frac{\dif^{\,2}{u}}{\dif{\rho^{2}}}=\frac{l(l+1)}{\rho^{2}}u 
\end{align*}
 جس کا عمومی حل (تصدیق کیجیے) درج ذیل ہو گا
 \begin{align*}
u(\rho)=C\rho^{l+1}+D\rho^{-l} 
\end{align*}
 تاہم  (\عددی{\rho\to0}  کی صورت میں)  \عددی{\rho^{-l}}  بے قابو بڑھتا ہے لہٰذا \عددی{D=0}  ہو گا۔ یوں  \عددی{\rho}  کی چھوٹی قیمتوں کے لیے درج ذیل ہو گا۔
 \begin{align}\label{مساوات_ابعادی_ہائیڈروجن_مبدا_قریب}
u(\rho)\sim C\rho^{l+1} 
\end{align}

  اگلے قدم پر متقاربی رویہ  کو چھیلنے کی خاطر  نیا تفاعل \عددی{v(\rho)}:
  \begin{align}\label{مساوات_ابعادی_نئی_علامتیت}
u(\rho)=\rho^{l+1}e^{-\rho}v(\rho) 
\end{align}
  اس امید سے متعارف کرتے ہے کہ    \عددی{u(\rho)}     سے  \عددی{v(\rho)} زیادہ سادہ ہو گا۔ ابتدائی نتائج 
\begin{align*}
\frac{\dif u}{\dif \rho}=\rho^l e^{-\rho} \big[(l+1-\rho)v+\rho\frac{\dif v}{\dif \rho}\big]
\end{align*}
اور
\begin{align*}
\frac{\dif^{\,2}u}{\dif \rho^2}=\rho^le^{-\rho}\big\{\big[-2l-2+\rho+\frac{l(l+1)}{\rho}\big]v+2(l+1-\rho)\frac{\dif v}{\dif \rho}+\rho\frac{\dif^{\,2}v}{\dif \rho^2}\big\}
\end{align*}
خوش آئین نظر نہیں آتے ہیں۔   اس طرح \عددی{v(\rho)} کی صورت میں رداسی مساوات (مساوات \حوالہ{مساوات_ابعادی_رداسی_کولمب_نئے_متغیرات})   درج ذیل روپ اختیار کرتی ہے۔
  \begin{align}\label{مساوات_ابعادی_رداسی_نئی_روپ}
\rho\frac{\dif^{\,2}{v}}{\dif{\rho^{2}}}+2(l+1-\rho)\frac{\dif{v}}{\dif{\rho}}+[\rho_{0}-2(l+1)]v=0 
\end{align}

  آخر میں ہم فرض کرتے ہیں کہ حل،   \عددی{v(\rho)}،  کو \عددی{\rho} کا طاقتی تسلسل لکھا جا سکتا ہے۔
  \begin{align}
v(\rho)=\sum_{j=0}^{\infty}c_{j}\rho^{j} 
\end{align}
ہمیں عددی سر (\عددی{c_0}، \عددی{c_1}، \عددی{c_2}، وغیرہ) تلاش کرنے ہوں گے۔  جزو در جزو تفرق لیتے ہیں۔
\begin{align*}
\frac{\dif{v}}{\dif{\rho}}=\sum_{j=0}^{\infty}jc_{j}\rho^{j-1}=\sum_{j=0}^{\infty}(j+1)c_{j+1}\rho^{j} 
\end{align*}
[میں نے دوسرے مجموعہ میں "فرضی اشاریہ" \عددی{j} کو \عددی{j+1} کہا ہے۔ اگر آپکو یقین نہ  ہو تو اولین چند اجزاء صریحاً لکھ کر تصدیق کر لیں۔ آپ سوال اٹھا سکتے ہیں کے نیا مجموعہ \عددی{j=-1}  سے کیوں  شروع نہیں کیا گیا؛ تاہم جزو ضربی \عددی{(j+1)}  اس جزو کو ختم کرتا ہے لہٰذا ہم صفر سے بھی شروع کر سکتے ہیں۔] دوبارہ تفرق لیتے ہیں۔
\begin{align*}
\frac{\dif^{\,2}{v}}{\dif{\rho^{2}}}=\sum_{j=0}^{\infty}j(j+1)c_{j+1}\rho^{j-1} 
\end{align*}
انہیں مساوات \حوالہ{مساوات_ابعادی_رداسی_نئی_روپ} میں پر کرتے ہیں۔
\begin{multline*}
\sum_{j=0}^{\infty}j(j+1)c_{j+1}\rho^{j}+2(l+1)+\sum_{j=0}^{\infty}(j+1)c_{j+1}\rho^{j} \\
-2\sum_{j=0}^{\infty}jc_{j}\rho^{j}+[\rho_{0}-2(l+1)]\sum_{j=0}^{\infty}c_{j}\rho^{j}=0 
\end{multline*}
ایک جیسی طاقتوں کے عددی سروں کو مساوی رکھتے ہوئے
\begin{align*}
j(j+1)c_{j+1}+2(l+1)(j+1)c_{j+1}-2jc_{j}+[\rho_{0}-2(l+1)]c_{j}=0 
\end{align*}
یا
\begin{align}\label{مساوات_ابعادی_کولمب_کلیہ_توالی}
c_{j+1}=\left\{\frac{2(j+l+1)-\rho_{0}}{(j+1)(j+2l+2)}\right\}c_{j} 
\end{align}
ہو گا۔ یہ کلیہ توالی عددی سر تعین کرتے ہوئے تفاعل \عددی{v(\rho)} تعین کرتا ہے۔ ہم \عددی{c_{0}} سے شروع کر کے (جو مجموعی مستقل کا روپ اختیار کرتا ہے جسے آخر میں معمول زنی سے حاصل کیا جائے گا)، مساوات \حوالہ{مساوات_ابعادی_کولمب_کلیہ_توالی}  سے \عددی{c_{1}} تعین کرتے ہے؛ جس کو واپس اسی مساوات میں پر کر کے \عددی{c_{2}} تعین ہو گا، وغیرہ، وغیرہ۔\حاشیہد{آپ پوچھ سکتے ہیں: طاقتی تسلسل کی ترکیب \عددی{u(\rho)} پر ہی کیوں لاگو نہیں کی گئی؛ اس ترکیب کے اطلاق سے قبل متقاربی رویہ کو کیوں (جزو ضربی کی صورت میں) باہر نکالا گیا؟ درحقیقت اس کی وجہ نتائج کی خوبصورتی ہے۔جزو ضربی \عددی{\rho^{l+1}} باہر نہ نکالنے سے تسلسل کے  ابتدائی اجزاء صفر ہوں گے (پہلا غیر صفر عددی سر \عددی{c_{l+1}} ہو گا)؛ \عددی{\rho^{l+1}}  باہر نکالنے سے تسلسل کا پہلا جزو \عددی{\rho^0} حاصل ہو گا۔ اس کے برعکس جزو ضربی \عددی{e^{-\rho}}  باہر نکالنا زیادہ ضروری ہے؛ اسے باہر نہ نکالنے سے \عددی{c_{j+2}}، \عددی{c_{j+1}} اور \عددی{c_j} پر مشتمل  تین اجزائی کلیہ توالی حاصل ہوتا ہے (کر کے دیکھیں!) جس کے ساتھ کام کرنا زیادہ مشکل ثابت ہوتا ہے۔}

 آئے \عددی{j} کی بڑی قیمت (جو \عددی{\rho}  کی بڑی قیمت کے مطابقتی ہوں گے جہاں بلند طاقتیں غالب ہوں گی) کے لئے عددی سروں کی صورت دیکھے ۔ یہاں  کلیہ توالی درج ذیل کہتا ہے۔\حاشیہد{آپ پوچھ سکتے ہیں: شمار کنندہ میں \عددی{2(l+1)-\rho_0} اور نسب نما میں \عددی{2l+2} رد کرنے کی طرح \عددی{j+1} میں \عددی{1} کیوں رد نہیں کیا جاتا؟ اس تخمین میں ایسا کیا جا سکتا ہے، تاہم اسے رد نہ کرنے سے دلیل زیادہ واضح ہو گا۔ آپ \عددی{1} کو رد کر کے دیکھ سکتے ہیں کہ میں کیا کہنا چاہتا ہوں۔}
\begin{align*}
c_{j+1}\cong\frac{2j}{j(j+1)}c_{j}=\frac{2}{j+1}c_{j} 
\end{align*}
ایک لمحہ کے لیے فرض کرے کہ یہ بالکل ٹھیک ٹھیک رشتہ ہے۔ تب 
\begin{align}
c_{j}=\frac{2^{j}}{j!}c_{0} 
\end{align}
لہٰذا
\begin{align*}
v(\rho)=c_{0}\sum_{j=0}^{\infty}\frac{2^{j}}{j!}\rho^{j}=c_{0}e^{2\rho} 
\end{align*}
 اور یوں  درج ذیل ہو گا
\begin{align}
u(\rho)=c_{0}\rho^{l+1}e^{\rho} 
\end{align}
 جو   \عددی{\rho}  کی بڑی قیمتوں کے لیے  بے قابو بڑھتا ہے۔ مثبت قوت نما وہی غیر پسندیدہ متقاربی رویہ دیتا ہے جو مساوات \حوالہ{مساوات_ابعاد_عمومی_حل_بے_قابو} میں پایا گیا۔ (در حقیقت  متقاربی حل بھی رداسی مساوات کے جائز حل ہیں البتہ ہم ان میں دلچسپی نہیں رکھتے ہیں کیونکہ یہ معمول پر  لانے کے قابل نہیں ہیں۔) اس المیہ سے نجات کا صرف ایک ہی راستہ ہے؛ تسلسل کو کہیں نہ کہیں اختتام پذیر ہونا ہو گا۔ لازمی طور پر ایک ایسا زیادہ سے زیادہ عدد صحیح، \عددی{j_{\text{بلندتر}}}، پایا جائے گا جس پر درج ذیل ہو۔
 \begin{align}
c_{(j_{\text{بلندتر}}+1)}=0
\end{align}
 (یوں کلیہ توالی کے تحت باقی تمام (زیادہ بلند) عددی سر صفر ہوں گے۔)  مساوات \حوالہ{مساوات_ابعادی_کولمب_کلیہ_توالی} سے ظاہر ہے کہ درج ذیل ہو گا۔
 \begin{align*}
2(j_{\text{بلندتر}}+l+1)-\rho_{0}=0 
\end{align*}
\اصطلاح{صدر کوانٹم  عدد}\فرہنگ{کوانٹم!صدر عدد}\حاشیہب{principal quantum number}\فرہنگ{quantum!principle number}
 \begin{align}\label{مساوات_ابعادی_صدر_کوانٹائی_عدد}
n\equiv j_{\text{بلندتر}}+l+1 
\end{align}
 متعارف کرتے ہوئے درج ذیل ہو گا۔
 \begin{align}\label{مساوات_ابعادی_رو_این}
\rho_{0}=2n 
\end{align}
 اب \عددی{E} کو  \عددی{\rho_{0}}  تعین کرتا ہے (مساوات \حوالہ{مساوات_ابعادی_متعارف_الف} اور \حوالہ{مساوات_ابعادی_متعارف_ب})
 \begin{align}
E=-\frac{\hslash^{2}\kappa ^{2}}{2m}=-\frac{me^{4}}{8\pi^{2}\epsilon^{2}\hslash^{2}\rho^{2}} 
\end{align}
 لہٰذا اجازتی توانائیاں درج ذیل ہوں گی۔ 
 \begin{align}\label{مساوات_ابعادی_ہائیڈروجن_اجازتی_توانائیاں}
E_{n}&=-\big[\frac{m}{2\hslash^{2}}\big(\frac{e^{2}}{4\pi\epsilon}\big)^{2}\big]\frac{1}{n^{2}}=\frac{E_{1}}{n^{2}}, && n=1,2,3,\dotsc
\end{align}
 یہ مشہور  زمانہ \اصطلاح{کلیہ بوہر}\فرہنگ{بوہر!کلیہ}\حاشیہب{Bohr formula}\فرہنگ{Bohr formula} ہے جو غالباً پورے کوانٹم میکانیات میں  اہم ترین نتیجہ ہے۔ جناب بوہر نے \سن{1913} میں،  ناقابل استعمال کلاسیکی طبیعیات اور نیم کوانٹم میکانیات کے ذریعہ  یہ کلیہ کو اخذ کیا۔ مساوات شروڈنگر \سن{1924} میں منظر عام ہوئی۔)

 مساوات  \حوالہ{مساوات_ابعادی_متعارف_ب} اور \حوالہ{مساوات_ابعادی_رو_این} کو ملا کر درج ذیل حاصل ہو گا
\begin{align}
\kappa =\big(\frac{me^{2}}{4\pi\epsilon_{0}\hslash^{2}}\big)\frac{1}{n}=\frac{1}{an} 
\end{align}
جہاں
\begin{align}
a\equiv\frac{4\pi\epsilon_{0}\hslash^{2}}{me^{2}}=\SI{0.529e-10}{\meter}
\end{align}
\اصطلاح{رداس بوہر}\فرہنگ{بوہر!رداس}\حاشیہب{Bohr radius}\فرہنگ{Bohr!radius} کہلاتا\حاشیہد{رداس بوہر کو روایتی طور پر زیر نوشت کے ساتھ لکھا جاتا ہے: \عددی{a_0}، تاہم یہ  غیر ضروری  ہے لہٰذا میں اس کو صرف \عددی{a} لکھوں گا۔} ہے۔ یوں (مساوات \حوالہ{مساوات_ابعادی_متعارف_ب} دوبارہ استعمال کرتے ہوئے) درج ذیل ہو گا۔
\begin{align}
\rho=\frac{r}{an} 
\end{align}
ہائیڈروجن جوہر کے فضائی تفاعلات  موج کے نام تین کوانٹائی اعداد (\عددی{n}، \عددی{l} اور \عددی{m}) استعمال کر کے رکھے جاتے ہیں 
 \begin{align}
\psi_{nlm}(r,\theta,\phi)=R_{nl}(r)Y_{l}^{m}(\theta,\phi) 
\end{align}
 جہاں مساوات \حوالہ{مساوات_ابعادی_نئے_متغیر_رداسی} اور \حوالہ{مساوات_ابعادی_نئی_علامتیت} کو دیکھتے ہوئے
 \begin{align}
R_{nl}(r)=\frac{1}{r}\rho^{l+1}e^{-\rho}v(\rho) 
\end{align} 
 ہو گا  جبکہ \عددی{v(\rho)} متغیر \عددی{\rho} میں درجہ \عددی{j_{\text{بلندتر}}=n-l-1}  کا کثیر رکنی ہو گا، جس کے عددی سر  درجہ ذیل کلیہ توالی دے گا (اور پورے تفاعل کو معمول پر لانا باقی ہے)۔
 \begin{align}\label{مساوات_ابعادی_کلیہ_توالی_کولمب_مخفیہ}
c_{j+1}=\frac{2(j+l+1-n)}{(j+1)(j+2l+2)}c_{j} 
\end{align}
\اصطلاح{زمینی حال}\فرہنگ{حال!زمینی}\حاشیہب{ground state}\فرہنگ{state!ground} (یعنی کم سے کم توانائی کے حال)  کے لیے 
 \عددی{n=1} ہو گا؛ طبعی مستقلات کی  قیمتیں پر کرتے ہوئے درجہ ذیل حاصل ہو گا۔
 \begin{align}
E_{1}=-\big[\frac{m}{2\hslash^{2}}\big(\frac{e^{2}}{4\pi\epsilon}\big)^{2}\big]=\SI{-13.6}{\electronvolt}
\end{align}
 ظاہر ہوا کہ  ہائیڈروجن کی \اصطلاح{بندشی توانائی}\فرہنگ{بندشی توانائی}\حاشیہب{binding energy}\فرہنگ{binding energy} (زمینی حال میں الیکٹران کو درکار  توانائی کی وہ مقدار جو جوہر کو باردارہ بنائے)  \عددی{\SI{13.6}{\electronvolt}} ہے۔ مساوات \حوالہ{مساوات_ابعادی_صدر_کوانٹائی_عدد} کے تحت \عددی{l=0}  لہٰذا \عددی{m=0}  ہو گا (مساوات \حوالہ{مساوات_ابعادی_انحطاطی_قیمتیں} دیکھیے) یوں درجہ ذیل ہو گا۔
 \begin{align}
\psi_{100}(r,\theta,\phi)=R_{10}(r)Y_{0}^{0}(\theta,\phi) 
\end{align}
 کلیہ توالی پہلے جزو پر ہی اختتام پذیر ہوتا ہے  ( مساوات \حوالہ{مساوات_ابعادی_کلیہ_توالی_کولمب_مخفیہ} سے \عددی{j=0} کے لئے  
 \عددی{c_{1}=0}  حاصل ہوتا ہے)،  لہٰذا \عددی{v(\rho)}  ایک مستقل   \عددی{(c_{0})}   ہو گا اور  یوں درجہ ذیل ہو گا۔
   \begin{align}
R_{10}(r)=\frac{c_{0}}{a}e^{-r/a} 
\end{align}
   اس کو مساوات \حوالہ{مساوات_ابعاد_علیحدہ_معمول_زنی_شرائط} کے تحت معمول پر لانے
سے
\begin{align*}
\int_{0}^{\infty}\abs{R_{10}}^{2}r^{2}\dif{r}=\frac{\abs{c_0}^{2}}{a^{2}}\int_{0}^{\infty}e^{-2r/a}r^{2}\dif{r}=\abs{c_{0}}^{2}\frac{a}{4}=1 
\end{align*}
یعنی \عددی{c_{0}=2/\sqrt{a}} حاصل ہو گا۔ مزید \عددی{Y_0^0=\tfrac{1}{\sqrt{4\pi}}} ہے لہٰذا  ہائیڈروجن کا زمینی حال درج ذیل ہو گا۔
\begin{align}
\psi_{100}(r,\theta,\phi)=\frac{1}{\sqrt{\pi a^{3}}}e^{-r/a} 
\end{align}
اسی طرح \عددی{n=2} کے لئے توانائی
\begin{align}
E_{2}=\frac{\SI{-13.6}{\electronvolt}}{4}=\SI{-3.4}{\electronvolt}
\end{align}
ہو گی جو پہلی ہیجان حال، یا حالات کی بندشی توانائی ہے کیونکہ   \عددی{l=0} ہو سکتا ہے (جس میں \عددی{m=0} ہو گا) یا \عددی{l=1} ہو سکتا ہے (جس  کے لئے یا  \عددی{m} کی قیمت \عددی{-1}، \عددی{0} یا \عددی{+1} ہو گی)؛ یوں چار مختلف حالات کی یہی توانائی ہو گی۔
 کلیہ توالی (مساوات \حوالہ{مساوات_ابعادی_کلیہ_توالی_کولمب_مخفیہ})  \عددی{l=0} کے لئے \عددی{j=0} استعمال کرتے ہوئے \عددی{c_1=-c_0} اور \عددی{j=1} استعمال کرتے ہوئے \عددی{c_2=0} دے گا لہٰذا \عددی{v(\rho)=c_{0}(1-\rho)} اور درجہ ذیل ہو گا۔
 \begin{align}\label{مساوات_تین_ابعادی_رداسی_بیس}
R_{20}(r)=\frac{c_{0}}{2a}\big(1-\frac{r}{2a}\big)e^{-r/2a} 
\end{align}
[دھیان رہے کہ مختلف کوانٹم اعداد \عددی{l} اور \عددی{n} کے لئے پھیلاو عددی سر  \عددی{\{c_j\}}  مکمل طور پر مختلف ہونگے۔] کلیہ توالی 
  \عددی{l= 1}   کی صورت میں پہلے جزو پر تسلسل کو اختتام پذیر کرتا ہے؛  \عددی{v(\rho)} ایک مستقل ہو گا  لہٰذا درجہ ذیل حاصل ہو گا۔
   \begin{align}\label{مساوات_تین_ابعادی_رداسی_اکیس}
R_{21}(r)=\frac{c_{0}}{4a^{2}}re^{-r/2a} 
\end{align}
(ہر منفرد صورت میں \عددی{c_0} معمول زنی سے تعین ہو گا سوال \حوالہء{4.11} دیکھیں)۔

 کسی بھی اختیاری  \عددی{n} کے لئے (مساوات \حوالہ{مساوات_ابعادی_صدر_کوانٹائی_عدد} سے ہم آہنگ) \عددی{l} کی ممکنہ قیمتیں درجہ ذیل ہوں گی
\begin{align}
l=0,1,2,\cdots, n-1 
\end{align}
جبکہ ہر  \عددی{l} کے لئے  \عددی{m}  کی ممکنہ قیمتوں کی تعداد  \عددی{(2l+1)}  ہو گی (مساوات \حوالہ{مساوات_ابعادی_انحطاطی_قیمتیں})،  لہٰذا   \عددی{E_{n}}   سطح توانائی کی کل انحطاطیت درج ذیل ہو گی۔
\begin{align}
d(n)=\sum_{l=0}^{n-1}(2l+1)=n^{2} 
\end{align}
کثیر رکنی \عددی{v(\rho)} (جو مساوات \حوالہ{مساوات_ابعادی_کلیہ_توالی_کولمب_مخفیہ} کے کلیہ توالی سے حاصل ہو گی) ایک ایسا تفاعل ہے جس سے عملی ریاضی دان بخوبی واقف ہیں؛  ماسوائے معمول زنی کے، اسے درج ذیل لکھا جا سکتا ہے۔
 \begin{align}\label{مساوات_تین_ابعادی_لاگیغ_الف}
v(\rho)=L_{n-l-1}^{2l+1}(2\rho) 
\end{align}
 جہاں
  \begin{align}\label{مساوات_تین_ابعادی_لاگیغ_ب}
L_{q-p}^{p}(x)\equiv(-1)^{p}\big(\frac{\dif}{\dif{x}}\big)^{p}L_{q}(x) 
\end{align} 
 ایک \اصطلاح{شریک لاگیغ  کثیر رکنی}\فرہنگ{لاگیغ!شریک کثیر رکنی}\حاشیہب{associated Laguerre polynomial}\فرہنگ{Laguerre!associated polynomial} ہے جبکہ 
\begin{align}\label{مساوات_تین_ابعادی_لاگیغ_پ}
 L_{q}(x)\equiv e^{x}\big(\frac{\dif}{\dif{x}}\big)^{q}(e^{-x}x^{q}) 
\end{align}
\begin{table}
\caption{ابتدائی چند لاگیغ کثیر رکنیاں، \عددی{L_q(x)}}
\label{جدول_ابعاد_لاگیغ_ابتدائی_چند}
\centering
\renewcommand{\arraystretch}{1.25}
\begin{tabular}{l}
\toprule
$L_0=1$\\
$L_1=-x+1$\\
$L_2=x^2-4x+2$\\
$L_3=-x^3+9x^2-18x+6$\\
$L_4=x^4-16x^3+72x^2-96x+24$\\
$L_5=-x^5+25x^4-200x^3+600x^2-600x+120$\\
$L_6=x^6-36x^5+450x^4-2400x^3+5400x^2-4320x+720$\\
\bottomrule
\end{tabular}
\end{table}
\begin{table}
\caption{ابتدائی چند شریک لاگیغ کثیر رکنیاں، \عددی{L_{q-p}^p(x)}}
\label{جدول_ابعادی_شریک_لاگیغ_کثیر_رکنیاں}
\centering
\renewcommand{\arraystretch}{1.25}
\begin{tabular}{ll}
\toprule
$L_0^0=1$  & $L_0^2=2$\\
$L_1^0=-x+1$  &  $L_1^2=-6x+18$\\
$L_2^0=x^2-4x+2$  &  $L_2^2=12x^2-96x+144$\\
$L_0^1=1$  &  $L_0^3=6$\\
$L_1^1=-2x+4$  &  $L_1^3=-24x+96$\\
$L_2^1=3x^2-18x+18$  &  $L_2^3=60x^2-600x+1200$\\
\bottomrule
\end{tabular}
\end{table}
\begin{table}
\caption{ہائیڈروجن کے ابتدائی چند رداسی تفاعلات، \عددی{R_{nl}(r)}}
\label{جدول_ابعادی_ہائیڈروجن_رداسی_تفاعل}
\centering
\renewcommand{\arraystretch}{2}
\begin{tabular}{l}
\toprule
$R_{10}=2a^{-3/2}e^{-r/a}$\\
\midrule
$R_{20}=\frac{1}{\sqrt{2}}a^{-3/2}\big(1-\frac{1}{2}\frac{r}{a}\big)e^{-r/2a}$\\
$R_{21}=\frac{1}{\sqrt{24}}a^{-3/2}\frac{r}{a}e^{-r/2a}$\\
\midrule
$R_{30}=\frac{2}{\sqrt{27}}a^{-3/2}\big(1-\frac{2}{3}\frac{r}{a}+\frac{2}{27}\big(\frac{r}{a}\big)^2\big)e^{-r/3a}$\\
$R_{31}=\frac{8}{27\sqrt{6}}a^{-3/2}\big(1-\frac{1}{6}\frac{r}{a}\big)\big(\frac{r}{a}\big)e^{-r/3a}$\\
$R_{32}=\frac{4}{81\sqrt{30}}a^{-3/2}\big(\frac{r}{a}\big)^2e^{-r/3a}$\\
\midrule
$R_{40}=\frac{1}{4}a^{-3/2}\big(1-\frac{3}{4}\frac{r}{a}+\frac{1}{8}\big(\frac{r}{a}\big)^2-\frac{1}{192}\big(\frac{r}{a}\big)^3\big)e^{-r/4a}$\\
$R_{41}=\frac{\sqrt{5}}{16\sqrt{3}}a^{-3/2}\big(1-\frac{1}{4}\frac{r}{a}+\frac{1}{80}\big(\frac{r}{a}\big)^2\big)\big(\frac{r}{a}\big)e^{-r/4a}$\\
$R_{42}=\frac{1}{64\sqrt{5}}a^{-3/2}\big(1-\frac{1}{12}\frac{r}{a}\big)\big(\frac{r}{a}\big)^2e^{-r/4a}$\\
$R_{43}=\frac{1}{768\sqrt{35}}a^{-3/2}\big(\frac{r}{a}\big)^3e^{-r/4a}$\\
\bottomrule
\end{tabular}
\end{table}
 \عددی{q} ویں \اصطلاح{لاگیغ کثیر رکنی}\فرہنگ{لاگیغ!کثیر رکنی}\حاشیہب{Laguerre polynomial}\فرہنگ{Laguerre!polynomial} ہے۔\حاشیہد{دیگر علامتوں کی طرح ان کے لئے بھی کئی علامتیں استعمال کی جاتی ہیں۔ میں نے سب سے زیادہ مقبول علامتیں استعمال کی ہیں۔} (جدول \حوالہ{جدول_ابعاد_لاگیغ_ابتدائی_چند} میں چند ابتدائی لاگیغ  کثیر رکنیاں  پیش کی گئی ہیں؛ جدول \حوالہ{جدول_ابعادی_شریک_لاگیغ_کثیر_رکنیاں} میں چند ابتدائی شریک لاگیغ  کثیر رکنیاں پیش کئے گئی ہیں؛  جدول \حوالہ{جدول_ابعادی_ہائیڈروجن_رداسی_تفاعل} میں چند ابتدائی رداسی تفاعل امواج پیش کئے گئے ہیں جنہیں شکل \حوالہء{4.4} میں ترسیم کیا گیا ہے۔) ہائیڈروجن کے معمول شدہ تفاعلات موج درجہ ذیل ہیں۔
  \begin{align}
\psi_{nlm}=\sqrt{\big(\frac{2}{na}\big)^3\frac{(n-l-1)!}{2n[(n+l)!]^{3}}}\,e^{-r/na}\big(\frac{2r}{na}\big)^{l}[L_{n-l-1}^{2l+1}(2r/na)]Y_{l}^{m}(\theta,\phi)
\end{align}
یہ تفاعلات خوفناک نظر آتے ہیں لیکن شکوہ  نہ کیجیے گا؛ یہ اُن چند حقیقی نظاموں میں سے ایک ہے جن کا  بند روپ میں ٹھیک ٹھیک حل حاصل کرنا ممکن ہے۔  دھیان رہے، اگرچہ تفاعلات موج تینوں کوانٹائی اعداد کے تابع ہیں، توانائیوں (مساوات \حوالہ{مساوات_ابعادی_ہائیڈروجن_اجازتی_توانائیاں}) کو صرف \عددی{n} تعین کرتا ہے۔ یہ کولمب توانائی کی ایک مخصوص خاصیت ہے؛ آپ کو یاد ہو گا کہ کروی کنواں میں توانائیاں \عددی{l} پر منحصر تھیں (مساوات \حوالہ{مساوات_ابعاد_کروی_کنواں_اجازتی_توانائیاں})۔ تفاعلات موج باہمی عمودی 
\begin{align}
\int\psi_{nlm}^{*}\psi_{n'l'm'}r^{2}\sin\theta\dif r\dif\theta\dif\phi=\delta_{nn'}\delta_{ll'}\delta_{mm'} 
\end{align}
ہیں۔ یہ کروی ہارمونیات کی عمودیت (مساوات \حوالہ{مساوات_ابعادی_کروی_ہارمونی_عمودیت})  اور \عددی{(n\neq n')} کی صورت میں  \عددی{H} کی منفرد امتیازی  اقدار کے امتیازی تفاعل ہونے کی بنا ہے۔

ہائیڈروجن تفاعلات موج کی تصویر کشی آسان کام نہیں ہے۔  ماہر کیمیا ان کے ایسے کثافتی اشکال بناتے ہیں جن کی چمک  \عددی{\abs{\psi}^{2}}
کا راست متناسب ہوتی ہے (شکل \حوالہء{4.5})۔ زیادہ معلومات مستقل کثافت احتمال کی سطحوں (شکل \حوالہء{4.6}) کے اشکال دیتی ہیں (جنہیں پڑھنا  نسبتاً مشکل ہو گا)۔


\ابتدا{سوال} 
کلیه توالی( مساوات \حوالہ{مساوات_ابعادی_کلیہ_توالی_کولمب_مخفیہ})  استعمال کرتے ہو ئے  تفاعل موج   \عددی{R_{30}}، \عددی{R_{31}} اور \عددی{R_{32}} حاصل کریں ۔ انہیں معمول پر لانے کی ضرورت نہیں۔ 
\انتہا{سوال}
\ابتدا{سوال}
\begin{enumerate}[a.]
\item
مساوات  \حوالہ{مساوات_تین_ابعادی_رداسی_بیس}  میں دیے گئے \عددی{R_{20}} کو معمول پر لا کر \عددی{\psi_{200}} تیار کریں ۔
\item
مساوات  \حوالہ{مساوات_تین_ابعادی_رداسی_اکیس}  میں دیے گئے  \عددی{R_{21}}  کو معمول پر لا کر  \عددی{\psi_{211}}،  \عددی{\psi_{210}} اور  \عددی{\psi_{21-1}}  تیار کریں۔ 
\end{enumerate}
\انتہا{سوال}
\ابتدا{سوال} 
\begin{enumerate}[a.]
\item
مساوات  \حوالہ{مساوات_تین_ابعادی_لاگیغ_پ} استعمال کرتے ہوئے ابتدائی چار لاگیغ کثیر رکنیاں حاصل کریں ۔
\item
مساوات  \حوالہ{مساوات_تین_ابعادی_لاگیغ_الف}،  \حوالہ{مساوات_تین_ابعادی_لاگیغ_ب} اور \حوالہ{مساوات_تین_ابعادی_لاگیغ_پ} استعمال کرتے ہوئے  \عددی{n=5} ،    \عددی{l=2} کی صورت میں  \عددی{v(\rho)} تلاش کریں ۔
\item
کلیہ توالی( مساوات \حوالہ{مساوات_ابعادی_کلیہ_توالی_کولمب_مخفیہ})  استعمال کرتے ہوئے \عددی{n=5} ،   \عددی{l=2}  کی صورت میں  \عددی{v(\rho)} تلاش کریں۔ 
\end{enumerate}
\انتہا{سوال}
\ابتدا{سوال} 
\begin{enumerate}[a.]
\item
ہائیڈروجن جو ہر کے زمینی حال میں الیکٹران   کے لیے  \عددی{\langle r \rangle} اور \عددی{\langle r^{2} \rangle} تلاش کریں ۔  اپنے جواب کو رداس بوہر کی صورت میں لکھیں۔ 
\item
ہائیڈروجن جوہر کے زمینی حال میں الیکٹران  کے لیے  \عددی{\langle x \rangle}  اور \عددی{\langle x^{2} \rangle}  تلاش کریں۔  \ترچھا{اشاره :}  آپکو کوئی نیا تکمل حاصل کرنے کی ضرورت نہیں ۔ دھیان  رہے کہ 
  \عددی{r^{2}=x^{2}+y^{2}+z^{2}} ہو گا    ، اور ا  زمینی حال میں تشاکلی کو بروئے کار لائیں۔ 
\item
حال \عددی{n=2} ، \عددی{l=1}، \عددی{m=1} کے لیے \عددی{\langle x^{2} \rangle} تلاش کریں ۔   \ترچھا{انتباہ:}  یہ حال \عددی{x} ، \عددی{y} اور \عددی{z} کے لحاظ سے  تشاکلی نہیں ہے ۔ یہاں
\عددی{x=r\sin{\theta}\cos{\phi}}  استعمال کرنا  ہو گا۔ 
\end{enumerate}
\انتہا{سوال}
\ابتدا{سوال} 
ہائیڈروجن کے زمینی حال میں  \عددی{r} کی کون سی قیمت زیادہ محتمل ہو گی۔ ( اس کا جواب صفر نہیں ہے !)  \ترچھا{اشاره :}  آپکو پہلے  معلوم کرنا ہو گا کہ \عددی{r} اور \عددی{r+\dif{r}} کے بیچ  الیکٹران  پائے جانے کا احتمال کیا ہو گا۔
\انتہا{سوال}
\ابتدا{سوال} 
ہائیڈروجن  جوہر ساکن  حال \عددی{n=2}، \عددی{l=1}، \عددی{m=1}  اور  \عددی{n=2}، \عددی{l=1}، \عددی{m=-1}  کے درج ذیل خطی جوڑ سے ابتداء کرتا ہے ۔
\begin{align*}
\Psi(\kvec{r},0)=\frac{1}{\sqrt{2}}(\psi_{211}+\psi_{21-1})
\end{align*}
\begin{enumerate}[a.]
\item
حال \عددی{\Psi(\kvec{r},t)} تیار کریں ۔ اس کی سادہ ترین صورت حاصل کریں ۔
\item
مخفی توانائی  کی توقعاتی قیمت ی   \عددی{\langle V \rangle} تلاش کریں ۔ (کیا یہ  \عددی{t} کی  تابع ہو گی؟)  اصل کلیہ اور عدد دی جواب   کو الیکٹران وولٹ تو صورت میں پیش کریں ۔
\end{enumerate}
\انتہا{سوال}

\جزوحصہ{ہائیڈروجن  کا   طیف}
اصولی طور پر ایک ہائیڈروجن جوہر جو ساکن حال \عددی{\psi_{nlm}} میں پایا جاتا ہو  ہمیشہ کے لیے اسی حال میں رہے گا۔ تاہم اس کو (  دوسرے جوہر کے ساتھ ٹکرا کر  یا اس پر روشنی  ڈال کر)  چھیڑنے سے الیکٹران کسی دوسرے ساکن حال میں \اصطلاح{عبور}\فرہنگ{عبور}\حاشیہب{transition}\فرہنگ{transition} کر سکتا ہے ۔ یہ توانائی جذب کر کے زیادہ توانائی  حال منتقل ہو سکتا ہے یا   (عموماً   برقناطیسی   فوٹان کے  اخراج   سے)     توانائی خارج کر  کے کم توانائی  حال  منتقل ہو سکتا ہے ۔\حاشیہد{فطراً،   اس میں تابع وقت باہم  عمل پایا جائے گا جس کی تفصیل باب \حوالہ{باب_تابع_وقت_نظریہ_اضطراب} میں پیش کی جائے گی۔یہاں اصل عمل جاننا ضروری نہیں ہے۔} عملاً  ایسی چھیڑخانیاں  ہر وقت پائی جائیں گی لہٰذا عبور (جنہیں  "کوانٹم  چھلانگ" کہتے ہیں)   مستقل طور پر ہوتے  رہیں    گے، جن  کی  بنا  ہائیڈروجن  سے ہر وقت   روشنی (فوٹان)  خارج ہو گی جس کی تونائی   ابتدائی اور اختتامی حالات کی  توانائیوں  کے فرق
\begin{align}
E_{\gamma}=E_{i}-E_{f}=\SI{-13.6}{\electronvolt}\,\big(\frac{1}{n^{2}_{i}}-\frac{1}{n^{2}_{f}}\big)
\end{align}
 کے برابر ہو گا ۔
 
اب \اصطلاح{کلیہ پلانک}\فرہنگ{پلانک!کلیہ}\حاشیہب{Planck's formula}\فرہنگ{Planck's!formula}\حاشیہد{فوٹان درحقیقت برقناطیسی اخراج  کا ایک کوانٹم ہے۔یہ ایک اضافیتی  چیز ہے جس پر غیر اضافی کوانٹم میکانیات قابل استعمال نہیں ہے۔اگرچہ ہم چند مواقع پر فوٹان کی بات کرتے ہوئے کلیہ پلانک  سے اس کی توانائی حاصل کریں گے، یاد رہے کہ اس کا اس نظریہ سے کوئی تعلق نہیں جس پر ہم بات کر رہے ہیں۔}  کے تحت  فوٹان  کی توانائی اس کے تعدد کے راست تناسب ہو گی:
\begin{align}
E_{\gamma}=h\nu
\end{align}
جبکہ   \اصطلاح{  طول موج}\فرہنگ{طول موج}    \عددی{\lambda=c/\nu} ہے لہٰذا درج ذیل ہو گا۔
\begin{align}\label{مساوات_تین_ابعادی_رڈبرگ_کلیہ}
\frac{1}{\lambda}=R\big(\frac{1}{n^{2}_{f}}-\frac{1}{n^{2}_{i}}\big)
\end{align}
جہاں 
\begin{align}
R\equiv \frac{m}{4\pi{c}\hslash^{3}}\big(\frac{e^{2}}{4\pi\epsilon_{o}}\big)^{2}=\SI{1.097e7}{\per\meter}
\end{align}
\اصطلاح{رڈبرگ  مستقل}\فرہنگ{رڈبرگ}\حاشیہب{Rydberg constant}\فرہنگ{Rydberg!constant} کہلاتا ہے ۔ مساوات \حوالہ{مساوات_تین_ابعادی_رڈبرگ_کلیہ} ہائیڈروجن کے طیف کا  \اصطلاح{کلیہ رڈبرگ}\فرہنگ{رڈبرگ!کلیہ}\حاشیہب{Rydberg formula}\فرہنگ{Rydberg!formula}  ہے ۔ یہ کلیہ انیسویں  صدی میں تجرباتی طور پر اخذ کیا گیا ۔نظریہ بوہر  کی سب سے بڑی فتح  اس کلیے کا حصول ہے جو قدرت کے بنیادی مستقلات  کی صورت میں \عددی{R}  کی قیمت دیتا ہے۔ زمینی حال  \عددی{(n_{f}=1)}  میں عبور،   بالائے بصری خطہ میں پائے  جاتے  ہیں  جنہیں    طیف پیمائی کار  \اصطلاح{لیمان تسلسل}\فرہنگ{تسلسل!لیمان}\حاشیہب{Lyman series}\فرہنگ{series!Lyman} کہتے ہیں۔ پہلی  ہیجان  حال  \عددی{(n_{f}=2)}
میں عبور،     دکھائی دینے والے خطہ میں  روشنی پیدا  کرتے  ہیں  جسے  \اصطلاح{بالمر تسلسل}\فرہنگ{تسلسل!بالمر}\حاشیہب{Balmer series}\فرہنگ{series!Balmer} کہتے ہیں۔  اسی طرح   \عددی{n_{f}=3}
میں  عبور،  \اصطلاح{پاسشن تسلسل}\فرہنگ{تسلسل!پاسشن}\حاشیہب{Paschen series}\فرہنگ{series!Paschen} دیتے ہیں  جو  زیر  بصری شعاع ہے، وغیرہ وغیرہ  (شکل\حوالہء{ 4.7}  دیکھیں)۔  (  رہائشی حرارت پر زیادہ تر  ہائیڈروجن  جوہر زمینی حال میں ہونگے؛ اخراجی طیف حاصل کرنے کی خاطر آپکو پہلے مختلف  ہیجان  حالات میں الیکٹران  آباد کرنے ہوں گے؛  ایسا عموماً گیس میں برقی شعلہ   پیدا کر کے کیا جاتا ہے۔)
 %===========================
\ابتدا{سوال}
ہائیڈروجن جوہر \عددی{Z} پروٹان  کے مرکزہ کے گرد طواف کرتے ہوئے  ایک  الیکٹران  پر مشتمل  ہے۔(از  خود ہائیڈروجن  میں  \عددی{Z=1}    جبکہ    باردارہ\اصطلاح{  ہیلیم}\فرہنگ{ہیلیم}\حاشیہب{Helium}\فرہنگ{Helium}  میں  \عددی{Z=2} اور دہری باردارہ   \اصطلاح{لتھیم }\فرہنگ{لتھیم}\حاشیہب{Lithium}\فرہنگ{Lithium} میں  \عددی{Z=3}  ہو گا،  وغیرہ وغیرہ ۔)   ہائیڈروجن  جوہر کی بوہر  توانائیاں  \عددی{E_{n}(Z)}،  بندشی  توانائی\عددی{E_{1}(Z)}،   رداس بوہر  \عددی{a(Z)}،  اور رڈبرگ  مستقل  
\عددی{R(Z)} تعین کریں ۔ (اپنے جوابات کو  ہائیڈروجن  کی متعلقہ قیمتوں کے لحاظ سے پیش کریں۔)   برقناطیسی طیف کے کس خطہ میں  \عددی{Z=2} اور  \عددی{Z=3} کی صورت میں   لیمان   تسلسل پائے جائیں گے؟   \ترچھا{اشارہ :} کسی نیے  حساب کی ضرورت نہیں ہے؛   مخفیہ (  مساوات \حوالہ{مساوات_ابعادی_کولمب_مخفیہ})   میں  \عددی{e^2\to Ze^2}  ہو گا لہٰذا تمام  نتائج میں بھی یہی کچھ پر کرنا  ہو  گا  ۔
\انتہا{سوال}
%==============================
\ابتدا{سوال}
زمین اور سورج کو ہائیڈروجن  جوہر کا متبادل تجاذبی نظام تصور کریں۔
\begin{enumerate}[a.]
\item
 مساوات  \حوالہ{مساوات_ابعادی_کولمب_مخفیہ}  کی جگہ مخفی توانائی تفاعل کیا  ہو گا ؟  (زمین کی کمیت  \عددی{m} جبکہ سورج کی کمیت \عددی{M} لیں۔)
\item
اس نظام کا   "رداس بوہر"  \عددی{a_{g}}  کیا ہو گا؟ اس کی عددی قیمت تلاش کریں ۔
\item
تجاذبی کلیہ  بوہر  لکھ کر  رداس  \عددی{r_0} کے مدار میں سیارہ  کے کلاسیکی توانائی کو  \عددی{E_n} کے برابر رکھ کر  دکھائیں کہ \عددی{n=\sqrt{r_0/a_g}} ہو گا۔ اس سے زمین کے کوانٹائی  عدد  \عددی{n}کی اندازاً قیمت تلاش کریں ۔
\item
فرض کریں  زمین  اگلی نچلی سطح   \عددی{(n-1)} میں عبور کرتی  ہے۔  کتنی توانائی  کا اخراج ہو گا؟  جواب   جاول   میں دیں ۔خارج   فوٹان(یا   زیادہ ممکنہ طور پر \اصطلاح{گریویٹان})   کا طول موج کیا ہو گا؟ ( اپنے جواب کو نوری سالوں میں پیش کریں۔ کیا یہ حیرت انگیز  نتیجہ محض ایک اتفاق ہے ۔)
\end{enumerate}
\انتہا{سوال}

\حصہ{زاویائی معیار حرکت}
ہم دیکھ چکے ہیں کہ ہائیڈروجن جو ہر کے ساکن حالات کو تین کوانٹائی اعداد \عددی{n}،  \عددی{l} اور \عددی{m} کے لحاظ سے نام دیا جاتا ہے ۔ صدر کوانٹم عدد  \عددی{(n)} حال کی توانائی تعین کرتا ہے  (مساوات  \حوالہ{مساوات_ابعادی_ہائیڈروجن_اجازتی_توانائیاں})؛  ہم دیکھیں گے کہ \عددی{l} اور \عددی{m} مداری زاویائی معیار حرکت سے تعلق رکھتے ہیں ۔ کلاسیکی نظریہ میں وسطی قوتیں،  توانائی اور معیار حرکت بنیادی بقائی مقداریں ہیں ،  اور یہ حیرت کی  بات نہیں  کہ کوانٹم میکانیات میں  زاویائی معیار حرکت( اس سے بھی زیادہ ) اہمیت  رکھتا ہے ۔

 کلاسیکی طور پر  (مبدا کے لحاظ سے)  ایک ذرہ کی زاویائی معیار حرکت درج ذیل کلیہ دیتا ہے 
\begin{align}
\kvec{L}=\kvec{r}\times\kvec{p}
\end{align}
جس کے تحت درج ذیل ہو گا۔
\begin{align}
L_{x}=yp_{z}-zp_{y}, \quad\quad L_{y}=zp_{x}-xp_{z}, \quad \quad L_{z}=xp_{y}-yp_{x}
\end{align}
ان کے متعلقہ کوانٹم عاملین معیاری نسخہ \عددی{p_x\to-i\hslash\partial/\partial x}، \عددی{p_y\to-i\hslash\partial/\partial y}،  \عددی{p_z\to-i\hslash\partial/\partial z}  سے حاصل ہوں گے۔ باب   \حوالہ{باب_غیر_تابع_وقت_شروڈنگر_مساوات} میں ہم نے ہارمونی مرتعش کے اجازتی  توانائیوں کو خالص الجبرائی ترکیب سے حاصل کیا۔  اگلے  حصہ میں  الجبرائی ترکیب  استعمال کرتے ہوئے زاویائی معیار حرکت  عاملین کے امتیازی اقدار حاصل کیے جائیں گے۔ یہ ترکیب،  عاملین کے  مقلبیت  تعلقات پر مبنی ہے ۔ اس کے بعد ہم امتیازی تفاعلات حاصل کریں گے جو  زیادہ دشوار کام ہے۔

\جزوحصہ{امتیازی اقدار}
عاملین \عددی{L_{x}} اور \عددی{L_{y}} آپس میں غیر مقلوب  ہیں۔ درحقیقت درج ذیل ہو گا۔
\begin{gather}
\begin{aligned}
[L_{x},L_{y}]&=[yp_{z}-zp_{y},zp_{x}-xp_{z}]\\
&=[yp_{z},zp_{x}]-[yp_{z},xp_{z}]-[zp_{y},zp_{x}]+[zp_{y},xp_{z}]\\
\end{aligned}
\end{gather}

%%%%%%KKKKKK the above is eq 4.97 on p172

%in between the above and below is missing, that is from eq 4.97 (p172) to prob 4.25 (p184)

%below is sec 4.4.1 spin 1/2 page 185 till example 4.2 (inclusive) page 187/188
\جزوحصہء{ چکر \عددی{1/2}}
%sec 4.4.1 spin 1/2 page 185
سادہ مادہ ( پروٹان،  نیوٹران،  الیکٹران)  کے ساتھ ساتھ \اصطلاح{کوارک}\حاشیہب{quarks}  اور تمام \اصطلاح{ لپٹان}\فرہنگ{لپٹان}\حاشیہب{leptons}\فرہنگ{leptons}  کیلۓ  \عددی{s=\tfrac{1}{2}}  ہو گا جو سب سے اہم ترین صورت ہے۔ مزید\عددی{1/2} چکر سمجھنے کے بعد زیادہ چکر کے ضوابط  دریافت کرنا نسبتاً   آسان ہے۔  صرف "دو"   عدد امتیازی تفاعلات  پائے جاتے ہیں:  پہلا \عددی{|\tfrac{1}{2}\tfrac{1}{2}\rangle}  ہے جسے  \اصطلاح{ہم میدان چکر}\فرہنگ{چکر!ہم میدان}\حاشیہب{spin up}\فرہنگ{spin up} ( یا غیر رسمی طور پر  \عددی{\uparrow}) اور دوسرا \عددی{|\tfrac{1}{2}(-\tfrac{1}{2})\rangle} ہے جس کو  \اصطلاح{مخالف میدان چکر}\فرہنگ{چکر!مخالف میدان}\حاشیہب{spin down}\فرہنگ{spin down} (\عددی{\downarrow}) کہتے ہیں۔ انہیں کو  اساس سمتیات لیتے   ہوئے   \عددی{1/2}  چکر ذرے  کے عمومی حال کو دو اجزائی قالب قطار   (یا\اصطلاح{ چکرکار}\فرہنگ{چکرکار}\حاشیہب{spinor}\فرہنگ{spinor}) سے ظاہر کر سکتے ہیں:
\begin{align}\label{مساوات_تین_بعدی_عمومی_حال_چائے}
 \chi=\begin{pmatrix} a \\ b \end{pmatrix}= a\chi_{+} + b\chi_{-} 
 \end{align}
جہاں
\begin{align} 
 \chi_{+}=\begin{pmatrix}1\\0 \end{pmatrix}
 \end{align}
ہم میدان چکر  کو ظاہر کرتا ہے اور  
\begin{align} 
 \chi_{-}=\begin{pmatrix}0 \\1 \end{pmatrix}
 \end{align}
مخالف میدان چکر کو ظاہر کرتا ہے۔

ساتھ ہی عاملین چکر   \عددی{2\times 2}  قالب ہوں گے جنہیں حاصل کرنے کی خاطر ہم ان کا  اثر  \عددی{\chi_+} اور \عددی{\chi_-} پر  دیکھتے ہیں۔  مساوات  \حوالہء{4.135}  درج ذیل کہتی ہے۔
\begin{align} \label{مساوات_تین_بعدی_ہم_میدان_خلاف_میدان}
 \bold{S}^2\chi_{+}=\frac{3}{4}\hslash^2\chi_{+}\quad \text{اور} \quad \bold{S}^2\chi_{-}= \frac{3}{4}\hslash^2 \chi_{-}
 \end{align}
ہم \عددی{ \bold{S}^2 } کو ( اب تک)  نا معلوم ارکان کا قالب 
\begin{align} 
 \bold{S}^2=\begin{pmatrix}c &d\\e & f\end{pmatrix}
 \end{align}
لکھ کر  مساوات \حوالہ{مساوات_تین_بعدی_ہم_میدان_خلاف_میدان} کی بائیں مساوات  کو درج ذیل لکھ سکتے ہیں
\begin{align*} 
   \begin{pmatrix}c\\e \end{pmatrix}= \begin{pmatrix}\tfrac{3}{4}\hslash^2 \\ 0 \end{pmatrix}\quad \text{یا}\quad \begin{pmatrix}c & d \\ e & f \end{pmatrix}
\begin{pmatrix}1\\0 \end{pmatrix}= \frac{3}{4}\hslash^2 \begin{pmatrix}\hslash \\0 \end{pmatrix}
 \end{align*}
لہٰذا \عددی{ c=\tfrac{3}{4}\hslash^2 } اور \عددی{ e=0 } ہو گا۔  مساوات \حوالہ{مساوات_تین_بعدی_ہم_میدان_خلاف_میدان} کی دائیں مساوات کے تحت
\begin{align*} 
\begin{pmatrix} d \\ f \end{pmatrix}= \begin{pmatrix}0 \\ \tfrac{3}{4}\hslash^2 \end{pmatrix} \quad \text{یا}\quad 
 \begin{pmatrix} c & d \\ e & f \end{pmatrix} \begin{pmatrix} 0 \\ 1 \end{pmatrix} = \frac{3}{4}\hslash^2
 \begin{pmatrix} 0 \\ 1 \end{pmatrix}
 \end{align*} 
 لہٰذا \عددی{ d=0 } اور \عددی{ f=\frac{3}{4}\hslash^2 } ہو گا۔  یوں درج ذیل حاصل ہوتا ہے۔
 \begin{align} 
\bold{S}^2= \frac{3}{4}\hslash^2 \begin{pmatrix} 1&0 \\ 0&1 \end{pmatrix} 
 \end{align} 
%page 174 
اسی طرح 
\begin{align} 
 \bold{S}_{z}\chi_{+}=\frac{\hslash}{2}\chi_{+}, \quad \bold{S}_{z} \chi_{-}=-\frac{\hslash}{2}\chi_{-}, 
 \end{align}
سے درج ذیل حاصل ہو گا۔
\begin{align} 
 \bold{S}_{z}= \frac{\hslash}{2}\begin{pmatrix} 1&0 \\ 0&-1 \end{pmatrix} 
 \end{align}
ساتھ ہی مساوات \حوالہء{ 4.136}  ذیل کہتی ہے۔
\begin{align*} 
 \bold{S}_{+} \chi_{-}=\hslash \chi_+, \quad \bold{S}_{-}\chi_{+}=\hslash \chi_- , \bold{S}_{+} \chi_{+}= \bold{S}_{-} \chi_{-}=0, 
 \end{align*}
لہٰذا درج ذیل ہو گا۔
\begin{align} 
 \bold{S}_{+}=\hslash\begin{pmatrix} 0&1 \\ 0&0 \end{pmatrix} , \quad \bold{S}_{-}=\hslash\begin{pmatrix}0&0 \\ 1&0 \end{pmatrix} 
 \end{align}
 اب     چونکہ \عددی{S_{\pm}=S_x{\pm}iS_y}  ہے لہٰذا    \عددی{S_x=\tfrac{1}{2}(S_++S_-)} اور \عددی{S_y=\tfrac{1}{2i}(S_+-S_-)} ہوں گے اور یوں  درج ذیل ہو گا۔
\begin{align} 
 \bold{S}_{x}=\frac{\hslash}{2}\begin{pmatrix} 0&1 \\1&0 \end{pmatrix} , \quad \bold{S}_{y}=\frac{\hslash}{2}\begin{pmatrix} 0&-i \\ i& 0 \end{pmatrix}
 \end{align}
چونکہ \عددی{ \bold{S}_{x} } , \عددی{ \bold{S}_{y} } , \عددی{ \bold{S}_z } تینوں میں \عددی{  \hslash /2 } کا جزو  ضربی پایا جاتا ہے لہٰذا انہیں زیادہ صاف روپ \عددی{ \bold{S}=\frac{\hslash}{2} \sigma } لکھا جا سکتا ہے جہاں درج ذیل ہوں گے۔
\begin{align} 
 \sigma_{x}\equiv \begin{pmatrix} 0&1 \\ 1&0 \end{pmatrix} , \quad \sigma_{y}\equiv  \begin{pmatrix} 0&-i \\ i&0 \end{pmatrix} , \quad \sigma_{z}\equiv \begin{pmatrix} 1&0 \\ 0&-1 \end{pmatrix} 
 \end{align}
یہ  \اصطلاح{پالی قالب چکر}\فرہنگ{پالی قالب چکر}\حاشیہب{Pauli spin matrices}\فرہنگ{Pauli spin matrices}  ہیں۔ دھیان رکھیں کہ \عددی{ \bold{S}_{x} }, \عددی{ \bold{S}_{y} } , \عددی{ \bold{S}_{z} }  اور \عددی{ \bold{S}^2 } تمام ہرمشی ہیں (جیسا کہ  انہیں ہونا  بھی  چاہیے کیونکہ یہ  قابل مشاہدہ کو ظاہر کرتے ہیں)۔ اس کے برعکس \عددی{ \bold{S}_{+} } اور  \عددی{ \bold{S}_{-} } غیر ہرمشی  ہیں؛  یہ  ناقابل مشاہدہ  ہیں۔

 \عددی{ \bold{S}_{z} }  کے امتیازی چکرکار (یقیناً)  درج ذیل ہوں گے۔
\begin{align} 
 \chi_{+}= \begin{pmatrix} 1 \\ 0 \end{pmatrix},\quad (+\frac{\hslash}{2}\,\text{\RL{امتیازی قدر}}); \quad \chi_{-}=\begin{pmatrix} 0 \\ 1 \end{pmatrix} , \quad (-\frac{\hslash}{2}\,\text{\RL{امتیازی قدر}})
 \end{align}
عمومی حال  \عددی{\chi} (  مساوات \حوالہ{مساوات_تین_بعدی_عمومی_حال_چائے})میں ایک ذرہ  کی \عددی{ S_{z} } کی پیمائش،   \عددی{ \abs{a}^2 } احتمال کے ساتھ \عددی{ +\hslash/2 } یا  \عددی{ \abs{b}^2 } احتمال کے
 ساتھ \عددی{ -\hslash/2  } دے سکتی ہے۔ چونکہ صرف یہی ممکنات ہیں لہٰذا درج ذیل ہو گا
\begin{align} 
 |a|^2+|b|^2=1 
 \end{align}
(یعنی  چکرکار لازماً   معمول شدہ ہو گا)۔\حاشیہد{ لوگ عموماً کہتے ہیں  کہ ہم میدان ذرہ  ہونے کا احتمال \عددی{\abs{a}^2} ہے۔ایسا کہنا درست نہیں۔ درحقیقت وہ کہنا چاہتے ہیں کہ اگر \عددی{S_z} کی پیمائش کی جائے تب \عددی{\tfrac{\hslash}{2}} نتیجہ حاصل ہونے کا احتمال \عددی{\abs{a}^2} ہو گا۔ (صفحہ \حوالہصفحہ{حاشیہ_قواعد_احتمال_درست_مطلب} پر حاشیہ \حوالہ{حاشیہ_قواعد_احتمال_درست_مطلب}  دیکھیں۔)}

%page 175 
تاہم  اس کی بجائے آپ \عددی{ S_{x} } کی پیمائش  کر سکتے ہیں۔ اس کے کیا نتائج اور ان کے انفرادی احتمالات کیا ہونگے؟ عمومی شماریاتی مفہوم کے تحت ہمیں  \عددی{ \bold{S}_{x}}  کے  امتیازی اقدار اور امتیازی  چکرکار  جاننے ہوں گے۔ امتیازی مساوات درج ذیل ہے۔
\begin{align*} 
 \begin{vmatrix} -\lambda & \hslash/2 \\   \hslash/2& -\lambda \end{vmatrix}=0 \implies \lambda^2=\big(\frac{\hslash}{2}\big)^2\implies \lambda={\pm}\frac{\hslash}{2} 
 \end{align*}
 یہ ہرگز حیرت کی بات نہیں  کہ  \عددی{ S_{x}}   کی ممکنہ   قیمتیں  وہی ہیں جو   \عددی{ S_{z}}   کی ہیں۔ امتیازی چکرکار  کو ہمیشہ کی طرز پر  حاصل کرتے  ہیں:
\begin{align*} 
 \frac{\hslash}{2}\begin{pmatrix}0&1 \\ 1&0 \end{pmatrix} \begin{pmatrix} \alpha \\ \beta \end{pmatrix}= {\pm}\frac{\hslash}{2}\begin{pmatrix}\alpha \\ \beta \end{pmatrix} \implies \begin{pmatrix}\beta \\ \alpha \end{pmatrix} ={\pm} \begin{pmatrix}\alpha \\ \beta \end{pmatrix} 
 \end{align*} 
لہٰذا \عددی{ \beta={\pm}\alpha } ہو گا۔ آپ دیکھ سکتے ہیں کہ \عددی{ \bold{S}_{x}} کے (معمول شدہ) امتیازی  چکرکار  درج ذیل ہوں گے۔
\begin{align} 
\renewcommand{\arraystretch}{1.5}
 \chi_{+}^{(x)}=\begin{pmatrix} \tfrac{1}{\sqrt{2}} \\[0.5ex] \tfrac{1}{\sqrt{2}} \end{pmatrix} , (+\frac{\hslash}{2}\text{\RL{امتیازی قدر}});\quad  \chi_{-}^{(x)}= \begin{pmatrix}\tfrac{1}{\sqrt{2}} \\[0.5ex] \tfrac{-1}{\sqrt{2}} \end{pmatrix} , (-\frac{\hslash}{2}\text{\RL{امتیازی  قدر}})
 \end{align}
بطور ہرمشی قالب کے امتیازی سمتیات یہ فضا کا احاطہ کرتے ہیں؛   عمومی چکرکار  \عددی{ \chi } (مساوات  \حوالہ{مساوات_تین_بعدی_عمومی_حال_چائے} )  کو ان کا خطی جوڑ لکھا جا سکتا ہے۔
\begin{align} 
  \chi=\big(\frac{a+b}{\sqrt{2}}\big)\chi_{+}^{(x)} +\big( \frac{a-b}{\sqrt{2}}\big)\chi_{-}^{(x)}
 \end{align} 
اگر آپ \عددی{ S_{x} } کی پیمائش کریں تب  \عددی{ +\hslash/2 }کے حصول کا احتمال    \عددی{ \frac{1}{2}|a+b|^2 }  اور  \عددی{ - \hslash/2 }  کے حصول کا احتمال    \عددی{ \frac{1}{2}|a-b|^2 }ہو گا۔ ( تصدیق  کیجیے  کہ ان احتمالات کا مجموعہ \عددی{1} کے برابر ہے۔)

%%%%%%%%%%%%%%
\ابتدا{مثال}
فرض کریں \عددی{ \tfrac{1}{2}} چکر کا ایک ذرہ درج ذیل حال میں ہے۔
\begin{align} 
 \chi=\frac{1}{\sqrt{6}}\begin{pmatrix} 1+i \\ 2 \end{pmatrix} 
 \end{align}
بتائیں  کہ \عددی{ S_{z} } اور \عددی{ S_{x} } کی پیمائش کرتے  ہوئے \عددی{ +\hslash/2 } اور \عددی{ -\hslash/2 } حاصل کرنے کے احتمالات کیا ہونگے۔


\موٹا{حل:}\quad
یہاں \عددی{ a=(1+i)\sqrt{6} } اور \عددی{ b=\frac{2}{\sqrt{6}} } ہے  لہٰذا \عددی{ S_{z} } کیلۓ \عددی{+ \tfrac{\hslash}{2}}  کے حصول کا احتمال
\begin{align*}
 \abs{\frac{1+i}{\sqrt{6}}}^2=\frac{1}{3} 
\end{align*}
  جبکہ \عددی{ -\tfrac{\hslash}{2} } حاصل کرنے کا احتمال
  \begin{align*}
   \abs{\frac{2}{\sqrt{6}}}^2 =\frac{2}{3} 
  \end{align*}
   ہو گا۔اسی طرح  \عددی{ S_{x} }  کیلۓ \عددی{ +\frac{\hslash}{2} } کے حصول کا احتمال \عددی{(1/2)\abs{(3+i)/\sqrt{6}}^2=5/6}   جبکہ \عددی{- \frac{\hslash}{2} } کے حصول کا احتمال
%page 176 
 \عددی{(1/2)\abs{(-1+i)/\sqrt{6}}^2=1/6}  ہو گا۔اتفاقاً \عددی{ S_{x} } کی توقعاتی قیمت درج ذیل ہے
\begin{align*} 
 \frac{5}{6}\big(+\frac{\hslash}{2}\big)+\frac{1}{6}\big(-\frac{\hslash}{2}\big)=\frac{\hslash}{3} 
 \end{align*}
جس کو ہم   بلاواسطہ  درج ذیل طریقہ سے بھی حاصل کر سکتے ہیں۔
\begin{align*} 
\renewcommand{\arraystretch}{1.5}
 \langle S_{x}\rangle=\chi^{\dagger}\bold{S}_{x}\chi=\begin{pmatrix}\frac{1-i}{\sqrt{6}} & \frac{2}{\sqrt{6}} \end{pmatrix} \begin{pmatrix} 0&\tfrac{\hslash}{2} \\ \tfrac{\hslash}{2}&0 \end{pmatrix} \begin{pmatrix}\tfrac{1+i}{\sqrt{6}} \\ \tfrac{2}{\sqrt{6}}\end{pmatrix} =\frac{\hslash}{3}
 \end{align*} 
\انتہا{مثال}

%%%%==========================================


میں آپ کو 
$1/2$
چکر سے متلقہ ایک فرضی پیمائسی تجربا سے گزرتا ہوں۔ چونکہ یہ ان تصوراتی خیالات کی وضاحت کرتا یے جن پر باب ۱ میں تبصرا کیا گیا۔ فرض کریں ایک زرا حال
$\psi_+$
 میں پایا جاتا ہے۔ اب اگر کوئی سوال پوچھے کہ اس زرے کی زاویائی چکری میارِ حرکت کا 
 z
 جز کیا ہے۔ تب ہم پورے یقین کے ساتھ جواب دے سکتے ہیں کہ اس کا جواب   
 $+\hslash/2$
 یوگا۔ چونکہ 
 z
 کی پیمائس لازمن یہی قیمت دے گی۔ اس کے بجائے اگر پوچھنے والا سوال کرے کہ اس زرے کی چکریا زاویائی میارِ حرکت کا
 x
 جز کیا ہوگا۔ تب ہم یہ کہنے پر مجبور یونگے کہ 
 $S_x$
 کی پیمائس سے 
 $+\hslash/2$
  یا
  $-\hslash/2$
کے حصول کا احتمال آدھا آدھا ہے۔ گر  سوال پوچھنے والا کلاسیکی ماحرِ تبیات یا حصہ ۱۔۲ کے نقطہِ نزر سے حقیقت پسند ہو تو وہ اس جواب کو ناکافی سمجھے گا۔ کیا آپ یہ کہنا چاہتے ہیں کہ آپ کو اس زرے کا حقیقی حال معلوم نہیں ہے۔ نہیں میں نے یہ تو نہیں کہا!۔ مجھے زرے کا حال تھیک تھیک معلوم ہے اور یہ 
$\psi_+$
یے۔ یب ایسا کیوں ہے کہ آپ مجھے اس کے چکر کا 
x
جز نہیں بتا سکتے اس لیئے کہ اس کے چکر کا کوئی مخصوس 
x
جز نہیں پایا جاتا ہے۔ یقینن ایسا ہی ہوگا۔ اگر 
$S_x$
اور
$S_z$
کی قیمتیں تائین ہوں تب اصولِ ادم یقینیت متمئن نہیں ہوگا۔ یہ سنتے ہی سوال کرنے والا زرے کی چکر کا 
x
 جز از خود پیمائس کرتا ہے۔ اب فرض کریں کہ وہ
 $+\hslash/2$
 قیمت حاصل کرتا ہے۔ وہ خوشی سے چلا اٹھا ہے۔ اس زرے کی 
 $S_x$
 قیمت ٹھیک
 $+\hslash/2$
 یے۔ جی آپ درست فرماتے ہیں اب اس کی یہی قیمت ہے۔ جس سے یہ بلکل سابت نہیں ہوتا کہ تجربہ سے پہلے بھی اس کی یہی قیمت تھی۔ اب ظاہر ہے آپ بال کی کھال اتار رہے ہو اور آپ کی ادم یقینیت اصول کا کیا بنا۔ میں اب 
 $S_x$
 اور 
  $S_z$
  دونوں کو جانتا ہوں۔ جی نہیں آپ نہیں جانتے  
ہیں۔ آپ نے پیمائس کے دوران زرے کا حال تبدیل کر دیا ہے۔ اب وہ  
$\psi_+$
%x^3 5:48
اور اگرچہ آپ اس کے 
$S_x$
کی قیمت جانتے ہیں۔ آپ
$S_z$
کی قیمت اب نہیں جانتے ہیں۔ لیکن میں نے 
 $S_x$
 کی پیمائس کے دوران ہم نے پوری کوسس کی کہ میں زرے کا سکون برباد نہ کروں۔ اچھا اگر آپ میری بات پر یقین نہیں کرتے تو خود تصدیق کریں۔ آپ
$S_z$
کی پیمائس کریں اور دیکھیں کہ کیا نتیجہ حاصل ہوتا ہے۔ عین ممکن ہے کہ وہ 
 $\hslash/2$
حاصل کرے جو میرے لیئے سرمندگی کا عصر ہوگا۔ اگر ہم اس پورے عمل کو بار بار دورائیں تو یہ سب اوقات اسے   
$-\hslash/2$
حاصل ہوگا۔ یہ کام آدمی کے لیئے


%==========================
%above is sec 4.4.1 spin 1/2 page 185 till start of prob 4.26 

%in between is missing. that is from after prob 4.26 till prob 4.49 (p206)

%below is prob 4.50 (p206) till end of chapter. unedited

%problem 4.50
\ابتدا{سوال} 
فرض کریں کہ ہم جانتے ہیں کہ دو عدد \( 1/2 \) چکر ذرات یکتا تنظیم  \حوالہ {  4.178 } میں پائے جاتے ہیں ۔مان لیں کہ اکائی سمتیا \عددی{S_{a}^{(1)}  } کے رخ ذرہ 1 کے چکری زاویائی معیار حرکت کا جز \عددی{  \hat{a} } ہے اسی طرح مان لیں کہ اکائی سمتیا \عددی{  S_{b}^{(2)}\ }  کے رخ ذرہ 2 کے چکری زاویائی معیار حرکت کا جز \عددی{  \hat{b} } ہے۔ درج ذیل دکھائیں جہاں \عددی{   \hat{a}} اور \عددی{  \hat{b} } کے بیچ زاویہ \عددی{  \theta } ہے 
\begin{align}
    \langle S_{a}^{(1)}S_{b}^{(2)}\rangle=-\frac{\hslash^{2}}{4}\cos\theta
\end{align}
\انتہا{سوال}
\ابتدا{سوال}
\begin{enumerate}[a.]
\item     کلیبش گورڈن عددی سروں کو \( s_1=1/2\)\( s_2=any thing\) کچھ بھی لیتے ہوئے حاصل کریں۔ آپ درج ذیل میں \عددی{ A  } اور \عددی{  B } عددی سروں کی وہ قیمت تلاش کرنا چاہتے ہیں جن کے لیے \عددی{  |sm\rangle } کا امتیازی حال ویکٹر \عددی{   S^2} ہو گا 
\begin{align*}
    |sm \rangle=A|\frac{1}{2}\frac{1}{2}\rangle|S_2(m-\frac{1}{2})\rangle+B|\frac{1}{2}(-\frac{1}{2})\rangle|S_2(m+\frac{1}{2})\rangle
\end{align*} 
مساوات 4.179 تا مساوات 4.182 کی ترکیب استعمال کریں ۔ اگر آپ یہ جاننے سے قاصر ہوں کہ \عددی{  S_{x}^{(2)} } مثلاً ویکٹر \عددی{  |s_{2} m_{2}\rangle } پر کیا کرتا ہے تو مساوات 4.136 سے رجوع کریں اور مساوات 4.147  سے قبل جملہ دوبارہ پڑھیں۔ جواب:
\begin{align*}
    A=\sqrt{\frac{s_2\pm m+1/2}{2s_2+1}}; 
     B=\pm \sqrt{\frac{s_2\mp m+1/2}{2s_2+1}} 
\end{align*}
جہاں  \( s=s_2\pm1/2 \)  علامتیں تعین کرتی ہیں۔
\item اس عمومی نتیجے کی تصدیق جدول 4.8 میں تین یا چار درجہ دیکھ کر کریں۔
\end{enumerate}
\انتہا{سوال}
\ابتدا{سوال}
ہمیشہ کی طرح \عددی{  S_z } کی امتیازی حالات کو اساس لیتے ہوئے 3/2چکر کے ذرے کے لیے قالب \عددی{  S_x } تلاش کریں۔ امتیازی مساوات حل کرتے ہوئے \عددی{ S_x  } کی امتیازی اقدار معلوم کریں۔
\انتہا{سوال}
\ابتدا{سوال}
مساوات 4.145 اور 4.147 میں 1/2 چکر سوال 4.31 میں ایک چکر اور سوال 4.52 میں 3/2 چکر کے قالبوں کی بات کی گئی۔ ان نتائج کو عمومیت دیتے ہوئے اختیاری \عددی{  s } چکر کے لیے چکری قالب تلاش کریں۔
جواب:
\begin{align*}
    \kvec S_z&=\hslash \begin{pmatrix}
    s&0&0&\cdots&0\\0&s-1&0&\cdots&0\\0&0&s-2&\cdots&0\\\vdots&\vdots&\vdots&\cdots&\vdots\\0&0&0&\cdots&-s
    \end{pmatrix} \\ 
    \kvec S_x&=\frac{\hslash}{2}\begin{pmatrix}0&b_s&0&0&\cdots&0&0\\b_s&0&b_{s-1}&0&\cdots&0&0\\0&b_{s-1}&0&b_{s-2}&\cdots&0&0\\0&0&b_{s-2}&0&\cdots&0&0\\\vdots&\vdots&\vdots&\vdots&\cdots&\vdots&\vdots\\0&0&0&0&\cdots&0&b_{-s+1}\\0&0&0&0&\cdots&b_{-s+1}&0\end{pmatrix} \\
    \kvec S_y&=\frac{\hslash}{2}\begin{pmatrix}0&\iota b_s&0&0&\cdots&0&0\\\iota b_s&0&-\iota b_{s-1}&0&\cdots&0&0\\0&\iota b{s-1}&0&-\iota b{s-2}&\cdots&0&0\\0&0&\iota b_{s-2}&0&\cdots&0&0\\\vdots&\vdots&\vdots&\vdots&\cdots&\vdots&\vdots\\0&0&0&0&\cdots&0&-\iota b_{-s+1}\\0&0&0&0&\cdots&\iota b_{-s+1}&0
    \end{pmatrix}
\end{align*} 
جہاں \(b_j=\sqrt{(s+j)(s+1-j)}\) ہو گا۔
\انتہا{سوال}
\ابتدا{سوال}
کروی ہارمونیات کے لیے،؟؟؟؟ ضربی جز درج ذیل طریقے سے حاصل کریں۔ ہم حصہ 4.1.2 سے درج ذیل جانتے ہیں 
\begin{align*} 
    Y_{l}^{m}=B_{    l}^{m}e^{\iota m\phi}P_{l}    ^{m}(\cos\theta)
\end{align*} 
آپ کو جز \عددی{  B_{l}^{m} } تعین کرنا ہو گا (جس کی قیمت تلاش کیے بغیر میں نے ذکر مساوات 4.32 میں کیا)۔ مساوات 4.120 ، 4.121 اور 4.130 استعمال کرتے ہوئے \عددی{  B_{l}^{m+1} } کی صورت میں \عددی{ B_{l}^{m}  } کا کلیہ توالی دریافت کریں۔ اس کو \عددی{  m } کے ریاضی ماخول کی ترکیب سے حل کرتے ہوئے \عددی{  B_{l}^{m} } کو مجموعی مستقل \عددی{ C(l)  } تک حل کریں۔آخر میں سوال 4.22 کا نتیجہ استعمال کرتے ہوئے اس مستقل کا بھی کچھ کریں۔ شریک لیجانڈر تفاعل کے تفرک کا درج ذیل کلیہ مددگار ثابت ہو سکتا ہے:
\begin{align}
    (1-x^{2})\frac{ d P_{l}^{m}} {dx}=\sqrt{1-x^{2}}  P_{l}^{m+1}-mxP_{l}^{m}
\end{align}
\انتہا{سوال}
\ابتدا{سوال}
ہائیڈروجن جوہر میں ایک الیکٹران درج ذیل چکر اور فضائی حال کے ملاپ میں پایا جاتا ہے 
\begin{align*}
    R_{21}(\sqrt{1/3}Y_{1}^{0}\chi + \sqrt{2/3}Y_{1}^{1}\chi-)
\end{align*} 
\begin{enumerate}[a.]
\item مداری زاویائی معیار حرکت کے مربع \عددی{ (L^{2})  } کی پیمائش سے کیا قیمتیں حاصل ہو سکتی ہیں؟ ہر قیمت کا انفرادی احتمال کیا ہو گا؟ 
\item یہی کچھ معیاری \(z\) زاویائی معیار حرکت کے \عددی{ (L_z) } جز کے لیے معلوم کریں۔
\item یہی کچھ چکری زاویائی معیار حرکت کے مربع سکیئر\((S^2)\) کے لیے معلوم کریں۔
\item یہی کچھ چکری زاویائی معیار\(z\) کے \عددی{  (S_z) } جز کے لیے کریں۔ کل زاویائی معیار حرکت کو \(\kvec{J}=\kvec{L}+\kvec{S}\) لیں ۔
\item آپ \عددی{  J^2 } کی پیمائش کرتے ہیں آپ کیا قیمتیں حاصل کرتے ہیں ان کا انفرادی احتمال کیا ہو گا
\item یہی کچھ \عددی{  J_z } کے لیے معلوم کریں۔
\item  آپ ذرے کے مقام کی پیمائش کرتے ہیں، اس کی \عددی{   r,\theta,\phi }  پر پائے جانے کی کثافت احتمال کیا ہو گا؟ 
\item آپ چکر کے \عددی{ z  } جز اور منبع سے فاصلہ کی پیمائش کرتے ہیں (یاد رہے کہ یہ ہم آہنگ مشہودات ہیں) ایک ذرے کا رداس  \عددی{  r } پر اور ہم میدان ہونے کا کثافت احتمال کیا ہو گا؟ 
\end{enumerate}
\انتہا{سوال}
\ابتدا{سوال}
\begin{enumerate}[a.]
\item  دکھائیں کہ ایک تفاعل \عددی{  f(\phi) } جس کو؟؟؟؟؟ تسلسل میں پھیلایا جا سکتا ہے، کے لیے درج ذیل ہو گا 
\begin{align*}
    f(\phi+\varphi)\equiv e^{\frac{\iota L_z\varphi}{\hslash}}f(\phi)
\end{align*} 
(جہاں \عددی{  \varphi } اختیاری زاویہ ہے) ۔اسی کی بنا \عددی{  L_z/\hbar } کو \عددی{  z } کے گرد گھومنے کا پیداکار کہتے ہیں۔ اشارہ: مساوات 4.129 استعمال کریں اور سوال 3.39 سے مدد لیں۔ زیادہ عمومی \عددی{  \kvec{L}.\hat{n}/\hslash } ہو گا جو \عددی{ \hat{n}  } کے رخ گھومنے کا پیداکار ہے یعنی \عددی{e^{(i \kvec{L}.\hat{n}\varphi/\hslash)}}   کے گرد دائیں ہاتھ سے     زاویہ  \عددی{ \varphi  } گھومنے کا اثر پیدا کرتا ہے۔ چکر کی صورت میں گھومنے کا پیداکار  \عددی{ \kvec{S}\cdot\hat{n}/\hbar } ہو گا بالخصوص  \(1/2\)  چکر کے لیے
\begin{align}
    \chi'=e^{\iota(\sigma.\hat{n})\varphi/2}\chi
\end{align} ہمیں چکر کاروں کے گھومنے کے بارے میں بتاتی ہے۔
\item محور \عددی{  x-axis } کے لحاظ سے 180 ڈگری گھومنے کو ظاہر کرنے والا \عددی{  (2\times2) } قالب تیار کریں اور دکھائیں کہ یہ ہماری توقعات کے عین مطابق ہمہ میدان \عددی{ (\chi_+)  } کو خلاف میدان \عددی{  (\chi_-) } میں تبدیل کرتا ہے 
\item محور \عددی{  y-axis } کے لحاظ سے 90 ڈگری گھومنے والا قالب تیار کریں اور دیکھیں کہ \عددی{ (\chi_+)  } پر اس کا اثر کیا ہو گا؟ 
\item محور \عددی{  z-axis } کے لحاظ سے 360 زاویہ گھومنے کو ظاہر کرنے والا قالب تیار کریں۔ کیا جواب آپ کی توقعات کے مطابق ہے؟ ایسا نہ ہونے کی صورت میں اس کی مضمرات پر تبصرہ کریں۔
\item درج ذیل دکھائیں 
\begin{align} e^{\iota(\sigma.\hat{n})\varphi/2}=\cos{(\varphi/2)}+\iota(\hat{n}.\sigma)\sin{(\varphi/2)}
\end{align}
\end{enumerate}
\انتہا{سوال}
\ابتدا{سوال}
زاویائی معیار حرکت کے بنیادی تبادلی رشتے (مساوات 4.99) امتیازی اقدار کے عدد صحیح قیمتوں کے ساتھ ساتھ نصف عدد صحیح قیمتوں کی بھی اجازت دیتے ہیں۔ جبکہ مداری زاویائی معیار حرکت کی صرف عدد صحیح قیمتیں پائی جاتی ہیں۔ یوں ہم توقع کریں گے کہ \عددی{ \kvec{L}=\kvec{r}\times\kvec{p}  } کے روپ میں کوئی اضافی شرط ضرور نصف عددی قیمتوں کو خارج کرتا ہو گا۔ ہم \عددی{  a } کو کوئی ایسا مستقل لیتے ہیں جسکا بود لمبائی ہو مثلاً ہائیڈروجن پر بات کرتے ہوئے رداس بوہر درج ذیل حاملین متعارف کرتے ہیں 
\begin{align*}
    q_1=\frac{1}{\sqrt{2}}[x+(a^2/\hslash)p_y] ; p_1\equiv\frac{1}{\sqrt{2}}[p_x-(\hslash/a^2)y];
\end{align*}
\begin{align*}
    q_2\equiv\frac{1}{\sqrt{2}}[x-(a^2/\hslash)p_y];p_2\equiv\frac{1}{\sqrt{2}}[p_x+(\hslash/a^2)y].
\end{align*}
\begin{enumerate}[a.]
\item  تصدیق کریں کہ \عددی{  [q_1,q_2]=[p_1,p_2]=0;[q_1,p_1]=[q_2,p_2]=\iota\hslash } یوں مقام اور معیار حرکت کی باضابطہ تبادلی رشتوں کو  \عددی{  q's } اور  \عددی{  p's } مطمئین کرتے ہیں اور اشاریہ  \عددی{  1 } کے حاملین اشاریہ \عددی{  2 } کے حاملین کے ہم آہنگ ہیں 
\item  درج ذیل دکھائیں 
\begin{align*}
    L_z=\frac{\hslash}{2a^2}(q_1^2-q_2^2)+\frac{a^2}{2\hslash}(q_1^2-q_2^2)
\end{align*}
\item تصدیق کریں کہ ایک ایسا ہارمونی مرتعش جس کی کمیت  
\(m=\hslash/a^2\) 
ہو اور تعدد \عددی{  \omega=1 }  ہو کہ ہر ایک ہیملٹنی \عددی{  H } کے لیے \(L_z=H_1-H_2\) گا۔ 
\item ہم جانتے ہیں کہ ہارمونی مرتعش کے ہیملٹنی کی   امتیازی اقدار\((n+1/2)\hslash\omega\)ہیں جہاں\(n=0,1,2,3,\cdots\) ہو گا (حصہ \حوالہ {  } کے الجبرائی نظریہ میں ہیملٹنی کی روپ اور باضابطہ تبادلی رشتوں سے یہ اخذ کیا گیا) اس کو استعمال کرتے ہوئے یہ اخذ کریں کہ \عددی{  L_z } کے امتیازی اقدار لازماً عدد ہوں گے ۔
\end{enumerate}
\انتہا{سوال}
\ابتدا{سوال}
عمومی حال مساوات 4.139 می\(1/2\) چکر کے \عددی{  S_z } اور \عددی{  S_y } کی کم سے کم عدم یقینیت کا شرط معلوم کریں یعنی \(\sigma_{S_{x}}\sigma_{S_{y}}\geq(\hslash/2)|\langle S_z\rangle|\) میں مساوات کی صورت میں تلاش کریں۔ جواب: عمومیت کھوئے بغیر \عددی{  a } کو حقیقی منتخب کر سکتے ہیں تب عدم یقینیت کی کم سے کم قیمت اس صورت میں حاصل ہو گی \عددی{  b } خالف حقیقی یا خالف خیالی ہو۔
\انتہا{سوال}
\ابتدا{سوال}
کلاسیکی برقی حرکیات میں ایک ذرہ جس کا؟؟؟؟ \عددی{  q } ہو اور جو مقناطیسی میدان \عددی{  \kvec{E} } اور  \عددی{  \kvec{B} } میں سمتی رفتار  \عددی{ \kvec{v}  } کے ساتھ حرکت کرتا ہو، پر قوت عمل کرتا ہے جو لورینز قوت کی مساوات دیتی ہے \begin{align}
    \kvec{F}=q(\kvec{E}+\kvec{v}\times\kvec{B})
\end{align}
اس قوت کو کسی بھی غیر سمتی مخفی توانائی تفاعل کی ڈھلوان کی صورت میں لکھا جا سکتا ہے لہذا مساوات شروڈنگر اپنی اصلی روپ میں (مساوات 1.1) اس کو قبول نہیں کر سکتی ہے تاہم اس کی نفیس روپ 
\begin{align}
    \iota\hslash\frac{\partial\psi}{\partial t}=H\psi
\end{align}
کوئی مسئلہ نہیں کھڑا کرتی ہے۔ کلاسیکی ہیملٹنی درج ذیل ہو گا 
\begin{align}
    H=\frac{1}{2m}(\kvec{p}-q\kvec{A})^2+q\varphi
\end{align}
 جہاں \عددی{ \kvec{A}  }  سمتی مخفی قوہ \(\kvec{B}=\nabla \times \kvec{A}\) اور \عددی{  \varphi } غیر سمتی مخفی قوہ \((\kvec{E}= -\nabla\varphi-\partial\kvec{A}/\partial t)\) ہیں لہٰذا شروڈنگر مساوات میں باضابطہ متبادل \((\kvec{p}\rightarrow((\hslash/\iota)\nabla)\) درج ذیل لکھا جا سکتا ہے۔
\begin{align}
    \iota\hslash\frac{\partial\psi}{\partial t}=[\frac{1}{2m}(\frac{\hslash}{\iota}\nabla-q\kvec{A})^2+q\varphi]\psi
\end{align}
\begin{enumerate}[a.]
\item درج ذیل دکھائیں 
  \begin{align}
    \frac{d\langle r \rangle }{dt}=\frac{1}{m}\langle(\kvec{p}-q\kvec{A})\rangle
\end{align}
\item ہمیشہ کی طرح مساوات 1.32 دیکھیں۔ ہم \عددی{  d\langle \kvec{r}\rangle/dt }  کو \عددی{ \langle \kvec{v} \rangle  } لیتے ہیں ۔ درج ذیل دکھائیں 
\begin{align}
    m\frac{d\langle v \rangle}{dt}=q\langle\kvec{E}\rangle+\frac{q}{2m}\langle(\kvec{p}\times\kvec{B}-\kvec{B}\times\kvec{p})\rangle-\frac{q^2}{m} \langle (\kvec{A}\times\kvec{B})\rangle
\end{align}
\item    بالخصوص موجی اکٹھ کے حجم پر یکساں \عددی{  \kvec{E} } اور \عددی{ \kvec{B}  } میدانوں کی صورت میں درج ذیل دکھائیں 
  \begin{align}
    m\frac{d\langle \kvec{v}\rangle}{dt}=q(\kvec{E}+\langle\kvec{v}\rangle\times\kvec{B}),
\end{align}
  اس طرح \عددی{ \langle\kvec{v\rangle}  } کی توقعاتی قیمت عین لورینز قوت کی مساوات کے تحت حرکت کرے گی جیسا ہم مسئلہ؟؟؟؟؟ کے تحت کرتے ہیں۔
\end{enumerate}
\انتہا{سوال}
\ابتدا{سوال}
( پس منظر جاننے کے لیے سوال 4.59 پر نظر ڈالیں) درج ذیل فرض کریں جہاں \عددی{  B_0 } اور  \عددی{  K } مستقلات ہیں 
\begin{align*}
    \kvec{A}=\frac{\kvec{B_0}}{2}(x_{\hat{j}}-y_{\hat{i}})
\end{align*}
;
\begin{align*}
    \varphi=Kz^2
\end{align*} 
\begin{enumerate}[a.]
\item  میدان \عددی{  E } اور \عددی{  B } تلاش کریں 
\item ان میدانوں میں جن کی کمیت \عددی{  m } اور بار \عددی{ q  } ہوں کے ساکن حالات کی اجازتی توانائیاں تلاش کریں۔ جواب
\begin{align}
    E(n_1,n_2)=(n_1+\frac{1}{2})\hslash\omega_1+(n_2+\frac{1}{2})\hslash\omega,       (n_1,n_2=0,1,2,3,\cdots)
\end{align}
جہاں \(\omega_1=q\kvec{B_0}/m\) اور   \( \omega_2\equiv\sqrt{2q\kvec{K}\m}  \) ہو گا۔ تبصرہ: \( \kvec{K} = 0 \) کی صورت میں یہ سائیکلوٹران حرکت کا کوانٹم مماثل ہو گا۔ کلاسیکی سائیکلوٹران تعدد \عددی{  \omega_1 } ہو گا اور یہ \عددی{  z } رخ میں آزاد ذرہ ہے۔ اجازتی توانائیاں \((n_1+\frac{1}{2})\hslash\omega\) ہوں گی جنہیں لانڈاؤ سطحیں کہتے ہیں۔
\end{enumerate}
\انتہا{سوال}
\ابتدا{سوال}
( پس منظر جاننے کی خاطر سوال 4.59 پر نظر ڈالیں) کلاسیکی برقی حرکیات میں مخفی قوہ \عددی{ \kvec{A}  } اور \عددی{  \varphi } یکتا طور پر تعین نہیں کیے جا سکتے ہیں، طبی مقداریں میدان \عددی{  \kvec{E} } اور \عددی{ \kvec{B}  } ہیں 
\begin{enumerate}[a.]
\item  دکھائیں کہ مخفی قوہ 
\begin{align}
    \varphi'\equiv\varphi-\frac{\partial\Lambda}{\partial t},  \kvec{A}'\equiv\kvec{A}+\nabla\Lambda
\end{align}
(جہاں مقام اور وقت کا \عددی{  \Lambda } ایک اختیاری حقیقی تفاعل ہے) بھی وہی میدان \عددی{  \varphi } اور \عددی{  \kvec{A} } دیتے ہیں۔ مساوات 4.210 گیج تبادلہ کہلاتی ہے جبکہ ہم کہتے ہیں کہ یہ نظریہ گیج غیر متغیر ہے۔
\item کوانٹم میکانیات میں مخفی قوہ کا کردار زیادہ براہ راست پایا جاتا ہے اور ہم جاننا چاہیں گے کہ ایا یہ نظریہ گیج متغیر رہتا ہے یا نہیں؟ دکھائیں کہ 
\begin{align}
    \Psi'\equiv e^{\iota q \Lambda/\hslash}\Psi
\end{align}
شروڈنگر مساوات ( مساوات 4.20) کو گیج تبادلہ مخفی قوہ  \عددی{  \varphi' } اور \عددی{ \kvec{A}  } لیتے ہوئے مطمئن کرتا ہے۔ چونکہ \عددی{ \Psi  } اور \عددی{ \Psi'  } میں صرف زاویائی جز کا فرق پایا جاتا ہے لہٰذا یہ ایک ہی طبی حال کو ظاہر کرتے ہیں اور یوں یہ نظریہ گیج غیر متغیر ہو گا۔ مزید معلومات کے لیے حصہ 10.2.3 سے رجوع کیجئے گا۔
\end{enumerate}
\انتہا{سوال}



