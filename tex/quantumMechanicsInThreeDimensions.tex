\باب{تین ابعادی کوانٹم میکانیات}\شناخت{باب-تین_ابعادی_کوانٹم_میکانیات}
\حصہ{کروی محدد میں مساوات شروڈنگر}
تین ابعاد تک توسیع  باآسانی کی جا سکتی ہے۔ مساوات شروڈنگر درج ذیل کہتی ہے
\begin{align}
i\hslash\frac{\dif{\Psi}}{\dif{t}}=H\Psi;
\end{align}
معیاری طریقہ کار کا اطلاق \عددی{x}  کے ساتھ ساتھ \عددی{y} اور \عددی{z} پر  کر کے:
\begin{align}\label{مساوات_تین_ابعاد_عاملین_الف}
p_{x}\to \frac{\hslash}{i}\frac{\partial}{\partial{x}},\quad p_{y}\to \frac{\hslash}{i}\frac{\partial}{\partial{y}},\quad p_{z}\to \frac{\hslash}{i}\frac{\partial}{\partial{z}} 
\end{align}
ہیملٹنی\حاشیہد{جہاں کلاسیکی مشہود اور عامل میں فرق کرنا دشوار ہو، وہاں میں عامل پر "ٹوپی" کا نشان بناتا ہوں۔ اس باب میں ایسا کوئی موقع نہیں پایا جاتا جہاں ان کی پہچان مشکل ہو لہٰذا یہاں سے عاملین پر "ٹوپی" کا نشان نہیں ڈالا جائے گا۔} عامل \عددی{H} کو کلاسیکی توانائی
\begin{align*}
\frac{1}{2}mv^{2}+V=\frac{1}{2m}(p_{x}^{2}+p_{y}^{2}+p_{z}^{2})+V
\end{align*}
سے حاصل کیا جاتا ہے۔ مساوات \حوالہ{مساوات_تین_ابعاد_عاملین_الف} کو مختصراً درج ذیل لکھا  جا سکتا ہے۔
\begin{align}
p\to \frac{\hslash}{i}\nabla
\end{align}
یوں درج ذیل ہو گا
\begin{align}
i\hslash\frac{\partial{\Psi}}{\partial{t}}=-\frac{\hslash^{2}}{2m}\nabla^{2}\Psi+V\Psi
\end{align}
جہاں
\begin{align}
\nabla^{2}\equiv \frac{\partial^{\,2}}{\partial{x^2}}+\frac{\partial^{\,2}}{\partial{y^2}}+\frac{\partial^{\,2}}{\partial{z^2}} 
\end{align}
کارتیسی محدد میں \اصطلاح{لاپلاسی}\فرہنگ{لاپلاسی}\حاشیہب{Laplacian}\فرہنگ{Laplacian} ہے۔

مخفی توانائی \عددی{V}   اور تفاعل موج  \عددی{\Psi}    اب \عددی{\kvec{r}=(x,y,z)} اور \عددی{t} کے تفاعلات ہیں۔ لا متناہی چھوٹے حجم  \عددی{\dif^{\,3}\kvec{r}=\dif{x}\dif{y}\dif{z}} میں ایک ذرہ  پایا جانے کا احتمال
  \عددی{\abs{\Psi(\kvec{r},t)}^{2}\dif^{\,3}\kvec{r}}   ہو گا اور معمول زنی شرط درج ذیل ہو گی
\begin{align}
\int\abs{\Psi}^{2}\dif^{\,3}\kvec{r}=1 
\end{align}
جہاں تکمل کو پوری فضا پر لینا ہو گا۔ اگر مخفی توانائی وقت کی تابع نہ ہو  تب ساکن حالات کا مکمل سلسلہ پایا جائے گا:
\begin{align}
\Psi_{n}(\kvec{r},t)=\psi_{n}(\kvec{r})e^{-iE_{n}t/\hslash} 
\end{align}
جہاں فضائی تفاعل موج  \عددی{\psi_{n}}   غیر تابع وقت شروڈنگر مساوات 
\begin{align}
-\frac{\hslash^{2}}{2m}\nabla^{2}\psi+V\psi=E\psi
\end{align}
کو مطمئن کرتا ہے۔ تابع وقت شروڈنگر مساوات کا عمومی حل درج ذیل ہو گا۔
\begin{align}\label{مساوات_ابعادی_عمومی_حل_الف}
\Psi(\kvec{r},t)=\sum c_{n}\psi_{n}(\kvec{r})e^{-iE_{n}t/\hslash} 
\end{align}
جہاں مستقلات \عددی{c_{n}}  ہمیشہ کی طرح ابتدائی تفاعل موج  \عددی{\Psi(\kvec{r},0)}   سے حاصل کیے جائیں گے۔ (اگر مخفیہ   \اصطلاح{استمراریہ}\فرہنگ{استمراریہ}\حاشیہب{continuum}\فرہنگ{continuum}  حالات دیتی ہو تب مساوات \حوالہ{مساوات_ابعادی_عمومی_حل_الف} میں مجموعہ کی بجائے تکمل ہو گا۔)

\ابتدا{سوال}
\begin{enumerate}[a.]
\item
عاملین \عددی{\kvec{r}} اور \عددی{\kvec{p}} کے تمام \اصطلاح{باضابطہ تبادلی رشتے}\فرہنگ{تبادلی!باضابطہ رشتے}\حاشیہب{canonical commutation relations}\فرہنگ{commutation!canonical relations}: \عددی{[x,y]}، \عددی{[x,p_{y}]}، \عددی{[x,p_{x}]}، \عددی{[p_{y},p_{z}]}، وغیرہ وغیرہ، حاصل کریں۔

\ترچھا{جواب:}
 \begin{align}
[r_{i},p_{j}]=-[p_{i},r_{j}]=i\hslash\delta_{ij},\quad [r_{i},r_{j}]=[p_{i},p_{j}]=0 
\end{align}
جہاں اشاریہ \عددی{x}،\عددی{y} اور \عددی{z} کو ظاہر کرتے ہیں جبکہ \عددی{r_{x}=x}، \عددی{r_{y}=y} اور  \عددی{r_{z}=z}
ہیں۔
\item
تین ابعاد کے لیے مسئلہ اہرنفسٹ کی تصدیق کریں:
\begin{align}
\frac{\dif }{\dif{t}}\langle \kvec{p}\rangle=\langle -\nabla V\rangle  \quad \text{اور}\quad\frac{\dif}{\dif{t}}\langle \kvec{r}\rangle =\frac{1}{m}\langle \kvec{p}\rangle
\end{align}
(ان میں سے ہر ایک در حقیقت تین مساوات کو ظاہر کرتی ہے۔ ایک مساوات ایک جزو کے لیے ہو گا۔)  \ترچھا{اشارہ:} 
پہلے  تصدیق کر لیں کہ مساوات \حوالہء{3.71} تین ابعاد کے لیے بھی کارآمد ہے۔
\item
ہیزنبرگ عدم یقینیت کے اصول کو تین ابعاد کے لیے بیان کریں۔ 

\ترچھا{جواب:}
 \begin{align}
\sigma_{x}\sigma_{p_x}\geq\frac{\hslash}{2},\quad \sigma_{y}\sigma_{p_y}\geq\frac{\hslash}{2},\quad \sigma_{z}\sigma_{p_z}\geq\frac{\hslash}{2}
\end{align}
تاہم  (مثلاً) \عددی{\sigma_{x}\sigma_{p_y}}  پر کوئی پابندی عائد نہیں ہوتی۔
\end{enumerate}
\انتہا{سوال}


\جزوحصہ{علیحدگی متغیرات} 
  عموماً مخفیہ صرف مبدا سے فاصلہ کا تفاعل ہو گا۔ ایسی صورت میں \اصطلاح{کروی محدد}\فرہنگ{محدد!کروی}\حاشیہب{spherical coordinates}\فرہنگ{coordinates!spherical}  \عددی{(r,\theta,\phi)}   کا استعمال بہتر ثابت ہو گا (شکل \حوالہء{4۔1})۔  
کروی محدد میں لاپلاسی درج ذیل روپ اختیار کرتا ہے۔
\begin{align}
\nabla^{2}=\frac{1}{r^{2}}\frac{\partial}{\partial{r}}\big (r^{2}\frac{\partial}{\partial{r}}\big )+\frac{1}{r^{2}\sin{\theta}}\frac{\partial}{\partial{\theta}}\big(\sin{\theta}\frac{\partial}{\partial{\theta}}\big )+\frac{1}{r^{2}\sin^{2}{\theta}}\big(\frac{\partial^{\,2}}{\partial{\phi^{2}}}\big ) 
\end{align}
یوں کروی محدد میں تابع وقت شروڈنگر مساوات درج ذیل ہو گی۔ 
\begin{multline}\label{مساوات_ابعادی_لاپلاسی_ب}
-\frac{\hslash^{2}}{2m}\big [\frac{1}{r^{2}}\frac{\partial}{\partial{r}}\big (r^{2}\frac{\partial\psi}{\partial{r}}\big )+\frac{1}{r^{2}\sin{\theta}}\frac{\partial}{\partial{\theta}}\big(\sin{\theta}\frac{\partial\psi}{\partial{\theta}}\big )+\frac{1}{r^{2}\sin^{2}{\theta}}\big(\frac{\partial^{2}\psi}{\partial{\phi^{2}}}\big )\big ]\\
+V\psi=E\psi 
\end{multline}
ہم ایسے حل کی تلاش میں ہیں جن کو حاصل ضرب کی صورت میں علیحدہ علیحدہ لکھنا ممکن ہو:
\begin{align}
\psi(r,\theta,\phi)=R(r)Y(\theta,\phi) 
\end{align}
اس کو مساوات \حوالہ{مساوات_ابعادی_لاپلاسی_ب} میں پر کر کے
\begin{multline*}
-\frac{\hslash^{2}}{2m}\big [\frac{Y}{r^{2}}\frac{\dif}{\dif{r}}\big (r^{2}\frac{\dif{R}}{\dif{r}}\big )+\frac{R}{r^{2}\sin{\theta}}\frac{\partial}{\partial{\theta}}\big(\sin{\theta}\frac{\partial{Y}}{\partial{\theta}}\big )+\frac{R}{r^{2}\sin^{2}{\theta}}\frac{\partial^{2}{Y}}{\partial{\phi^{2}}}\big ]+VRY=ERY 
\end{multline*}
دونوں اطراف کو \عددی{RY} سے تقسیم کر کہ \عددی{-2mr^{2}/\hslash^{2}} سے ضرب دیتے ہیں۔
\begin{align*}
&\big\{\frac{1}{R}\frac{d}{\dif{r}}\big(r^{2}\frac{\dif{R}}{\dif{r}}\big)-\frac{2mr^{2}}{\hslash^{2}}[V(r)-E]\big\} \\
+\frac{1}{Y}&\big\{\frac{1}{\sin{\theta}}\big(\sin{\theta}\frac{\partial{Y}}{\partial{\theta}}\big)+\frac{1}{\sin^{2}{\theta}}\frac{\partial^{2}{Y}}{\partial{\phi^{2}}}\big\}=0 
\end{align*}
پہلی خمدار قوسین میں جزو صرف \عددی{r} کا تابع ہے جبکہ باقی حصہ صرف  \عددی{\theta}   اور   \عددی{\phi}   کا تابع ہے؛  لہٰذا دونوں حصے انفرادی طور پر ایک مستقل کے برابر ہوں گے۔ اس علیحدگی مستقل کو ہم  \عددی{l(l+1)} روپ میں لکھتے ہیں جس کی وجہ کچھ دیر میں واضح ہو گی۔\حاشیہد{ایسا کرنے سے ہم عمومیت نہیں کھوتے ہیں، چونکہ یہاں \عددی{l} کوئی بھی مخلوط عدد ہو سکتا ہے۔ بعد میں ہم دیکھیں گے کہ \عددی{l} کو لازماً عدد صحیح ہونا ہو گا۔اسی نتیجہ کو ذہن میں رکھتے ہوئے میں نے علیحدگی مستقل کو اس عجیب روپ میں لکھا ہے۔}
\begin{align}
\frac{1}{R}\frac{d}{\dif{r}}\big(r^{2}\frac{\dif{R}}{\dif{r}}\big)-\frac{2mr^{2}}{\hslash^{2}}[V(r)-E]&=l(l+1)\label{مساوات_ابعاد_رداسی_الف} \\ 
\frac{1}{Y}\big\{\frac{1}{\sin{\theta}}\big(\sin{\theta}\frac{\partial{Y}}{\partial{\theta}}\big)+\frac{1}{\sin^{2}{\theta}}\frac{\partial^{2}{Y}}{\partial{\phi^{2}}}\big\}&=-l(l+1) \label{مساوات_ابعاد_غیر_رداسی_الف}
\end{align}
\ابتدا{سوال} 
کارتیسی محدد میں علیحدگی متغیرات  استعمال کرتے ہوئے لامتناہی مربعی کنواں (یا ڈبہ میں ایک ذرہ):
\begin{align*}
V(x,y,z)=\begin{cases}
0&\text{\RL{
اگر $x$، $y$ اور $z$ تینوں $0$ اور $a$ کے بیچ پائے جاتے ہوں
}}\\
\infty&\text{\RL{دیگر صورت}}
\end{cases} 
\end{align*}
حل کریں۔
\begin{enumerate}[a.]
\item
ساکن حالات  اور ان کی مطابقتی توانائیاں دریافت کریں۔
\item
بڑھتی توانائی کے لحاظ سے انفرادی توانائیوں کو \عددی{E_1}، \عددی{E_2}، \عددی{E_3}، وغیرہ، وغیرہ  سے ظاہر کر کے \عددی{E_1} تا \عددی{E_6} تلاش کریں۔ ان کی انحطاطیت (یعنی ایک ہی توانائی کے مختلف حلوں کی تعداد)  معلوم کریں۔ \ترچھا{تبصرہ:}  یک بعدی صورت میں انحطاطی مقید حالات نہیں پائے جاتے ہیں (سوال \حوالہء{2.45})، تاہم  تین ابعادی صورت میں یہ کثرت سے پائے جاتے ہیں۔
\item
توانائی  \عددی{E_{14}}   کی انحطاطیت کیا ہے اور یہ صورت کیوں دلچسپ ہے؟
\end{enumerate}
\انتہا{سوال}

\جزوحصہ{زاویائی مساوات}
مساوات \حوالہ{مساوات_ابعاد_غیر_رداسی_الف} متغیرات  \عددی{\theta}   اور    \عددی{\phi}  پر  \عددی{\psi} کی تابعیت  تعین کرتی ہے۔
اس کو  \عددی{Y\sin^{2}{\theta}}   سے ضرب دے کر درج ذیل حاصل ہو گا۔
\begin{align}
\sin{\theta}\frac{\partial}{\partial{\theta}}\big(\sin{\theta}\frac{\partial{Y}}{\partial{\theta}}\big)+\frac{\partial^{2}{Y}}{\partial{\phi^{2}}}=-l(l+1)Y\sin^{2}{\theta} 
\end{align}
ہو سکتا ہے آپ اس مساوات کو پہچانتے ہوں۔ یہ کلاسیکی برقی حرکیات میں مساوات لاپلاس کے حل میں پائی جاتی ہے۔ ہمیشہ کی طرح ہم علیحدگی متغیرات:
\begin{align}
Y(\theta,\phi)=\Theta(\theta)\Phi(\phi) 
\end{align}
استعمال کر کے دیکھنا چاہیں گے۔ اس کو پر کر کے \عددی{\Theta\Phi} سے تقسیم کر کہ درج ذیل حاصل ہو گا۔ 
\begin{align*}
\big\{\frac{1}{\Theta}\big[\sin{\theta}\frac{\dif}{\dif\theta}\big(\sin\theta\frac{\dif{\Theta}}{\dif{\theta}}\big)\big]+l(l+1)\sin^{2}{\theta}\big\}+\frac{1}{\Phi}\frac{\dif^{\,2}{\Phi}}{\dif{\phi^{\,2}}}=0 
\end{align*}
پہلا جزو صرف  \عددی{\theta}   کا تفاعل ہے، جبکہ دوسرا صرف   \عددی{\phi}     کا تفاعل ہے، لہٰذا ہر ایک جزو ایک مستقل ہو گا۔ اس مرتبہ ہم علیحدگی مستقل\حاشیہد{یہاں بھی ہم عمومیت نہیں کھوتے ہیں، چونکہ \عددی{m} کوئی بھی مخلوط عدد ہو سکتا ہے؛ اگرچہ ہم جلد دیکھیں گے کہ \عددی{m} کو عدد صحیح ہونا ہو گا۔ \ترچھا{انتباہ:} اب حرف \عددی{m} دو مختلف چیزوں، کمیت اور علیحدگی مستقل، کو ظاہر کر رہا ہے۔امید ہے کہ آپ کو درست معنی جاننے میں مشکل درپیش نہیں ہو گی۔} کو  \عددی{m^{2}}   لکھتے ہیں۔
\begin{align}
\frac{1}{\Theta}\big[\sin{\theta}\frac{\dif}{\dif\theta}\big(\sin\theta\frac{\dif{\Theta}}{\dif{\theta}}\big)\big]+l(l+1)\sin^{2}{\theta}&=m^{2} \\
\frac{1}{\Phi}\frac{\dif^{\,2}{\Phi}}{\dif{\phi^{2}}}&=-m^{2} 
\end{align}
 متغیر \عددی{\phi}  کی مساوات زیادہ آسان ہے۔
 \begin{align}
\frac{\dif^{\,2}{\Phi}}{\dif{\phi^{2}}}=-m^{2}\Phi\implies \Phi(\phi)=e^{im\phi} 
\end{align}
[درحقیقت  دو حل پائے جاتے ہیں: \عددی{e^{im\phi}} اور \عددی{e^{-im\phi}}، تاہم \عددی{m} کو منفی ہونے کی اجازت دے کر ہم موخر الذکر کو بھی درج بالا حل میں شامل کرتے ہیں۔ اس کے علاوہ حل میں  جزو ضربی مستقل بھی پایا جا  سکتا ہے جسے ہم  \عددی{\Theta}   میں ضم کرتے ہیں۔ چونکہ برقی مخفی توانائی لازماً \ترچھا{حقیقی} ہو گی لہٰذا برقی حرکیات میں اسّمتی تفاعل  \عددی{(\Phi)}   کو سائن اور کوسائن کی صورت میں نہ کہ قوت نمائی صورت میں لکھا جاتا ہے۔ کوانٹم میکانیات میں ایسی کوئی پابندی نہیں پائی جاتی ہے اور قوت نمائی کے ساتھ کام کرنا زیادہ آسان ہوتا ہے۔] اب جب بھی  \عددی{\phi}   کی قیمت  میں \عددی{2\pi}  کا اضافہ آئے، ہے ہم فضا میں واپس اسی نقطہ پر پہنچتے ہیں (شکل \حوالہء{4۔1} دیکھیں) لہٰذا درج ذیل شرط\حاشیہد{یہ بظاہر معصوم شرط اتنی معصوم نہیں ہے۔ یاد رہے کہ \عددی{m} کی قیمت سے قطع نظر، احتمال کثافت \عددی{(\abs{\Phi}^2)} یک قیمتی ہے۔ ہم حصہ \حوالہء{4.3} میں ایک مختلف طریقہ سے، زیادہ پر زور دلیل پیش کر کے \عددی{m} پر مسلط شرط حاصل کریں گے۔} مسلط کی جا سکتی ہے۔
\begin{align}
\Phi(\phi+2\pi)=\Phi(\phi) 
\end{align}
دوسرے لفظوں میں  \عددی{e^{im(\phi+2\pi)}=e^{im\phi}} یا \عددی{e^{2\pi im}=1} ہو گا جس کے تحت \عددی{m} لازماً عدد صحیح ہو گا۔
\begin{align}
m=0,\pm 1,\pm 2,\cdots
\end{align}
 مساوات \عددی{\theta} 
\begin{align}
\sin{\theta}\frac{\dif}{\dif{\theta}}\big(\sin{\theta}\frac{\dif{\Theta}}{\dif{\theta}}\big)+[l(l+1)\sin^{2}{\theta}-m^{2}]\Theta=0 
\end{align}
اتنی سادہ نہیں ہے۔ اس کا حل درج ذیل ہے
\begin{align}
\Theta(\theta)=AP_{l}^{m}(\cos{\theta}) 
\end{align}
جہاں  \عددی{P_{l}^{m}}  \اصطلاح{شریک لیژانڈر تفاعل}\فرہنگ{لیژانڈر!شریک}\حاشیہب{associated Legendre function}\فرہنگ{Legendre!associated}  ہے جس کی تعریف درج ذیل ہے
\begin{align}\label{مساوات_ابعادی_شریک_لیژانڈر_تفاعلات_تعریف}
P_{l}^{m}(x)\equiv (1-x^{2})^{\abs{m}/2}\big(\frac{\dif}{\dif{x}}\big)^{\abs{m}}P_{l}(x) 
\end{align}
اور \عددی{l} ویں لیژانڈر کثیر رکنی  کو  \عددی{P_{l}(x)}   ظاہر کرتا ہے\حاشیہد{دھیان رہے کہ \عددی{P_l^{-m}=P_l^m} ہو گا۔} جس کی تعریف \اصطلاح{کلیہ روڈریگیس}\فرہنگ{روڈریگیس!کلیہ}\حاشیہب{Rodrigues formula}\فرہنگ{Rodrigues formula}
\begin{align}
P_{l}(x)\equiv\frac{1}{2^{l}l!}\big(\frac{\dif}{\dif{x}}\big)^{l}(x^{2}-1)^{l} 
\end{align}
دیتا ہے۔مثال کے طور پر درج ذیل ہونگے۔
\begin{align*}
P_{0}(x)&=1,\quad P_{1}(x)=\frac{1}{2}\frac{\dif}{\dif{x}}(x^{2}-1)=x ,\\
P_{2}(x)&=\frac{1}{4\cdot 2}\big(\frac{\dif}{\dif{x}}\big)^{2}(x^{2}-1)^{2}=\frac{1}{2}(3x^{2}-1) 
\end{align*}
%
\begin{table}
\caption{ابتدائی چند لیژانڈر کثیر رکنیاں۔}
\label{جدول_ابعاد_لیژانڈر_چند_ابتدائی}
\centering
\begin{tabular}{l}
$P_0=1$\\[0.25em]
$P_1=x$\\[0.25em]
$P_2=\frac{1}{2}(3x^2-1)$\\[0.25em]
$P_3=\frac{1}{2}(5x^3-3x)$\\[0.25em]
$P_4=\frac{1}{8}(35x^4-30x^2+3)$\\[0.25em]
$P_5=\frac{1}{8}(63x^5-70x^3+15x)$
\end{tabular}
\end{table}
جدول \حوالہ{جدول_ابعاد_لیژانڈر_چند_ابتدائی} میں ابتدائی چند لیژانڈر کثیر رکنیاں پیش کی گئی ہیں۔ جیسا کہ نام سی ظاہر ہے، \عددی{P_{l}(x)} متغیر \عددی{x} کی درجہ \عددی{l} کثیر رکنی ہے، اور \عددی{l} کی قیمت طے  کرتی ہے کہ آیا یہ جفت کا طاق ہو گی۔ تاہم  \عددی{P_{l}^{m}(x)}  عموماً کثیر رکنی نہیں ہو گا؛ اور طاق \عددی{m} کی صورت میں اس میں \عددی{\sqrt{1-x^2}} کا جزو ضربی پایا جائے گا:
\begin{align*}
P_{2}^{0}(x)&=\frac{1}{2}(3x^{2}-1), \quad P_{2}^{1}(x)=(1-x^{2})^{1/2}\frac{\dif}{\dif{x}}\big[\frac{1}{2}(3x^{2}-1)\big]=3x\sqrt{1-x^{2}}, \\
P_{2}^{2}(x)&=(1-x^{2})\big(\frac{\dif}{\dif{x}}\big)^{2}\big[\frac{1}{2}(3x^{2}-1)\big]=3(1-x^{2}),
\end{align*}
وغیرہ وغیرہ۔  (اب ہمیں \عددی{P_{l}^{m}(\cos\theta)} چاہیے اور چونکہ \عددی{\sqrt{1-\cos^{2}\theta}=\sin\theta} ہوتا ہے لہٰذا \عددی{P_{l}^{m}(\cos\theta)} ہر صورت \عددی{\cos\theta} کا کثیر رکنی ہو گا جسے طاق \عددی{m} کی صورت میں
 \عددی{\sin\theta} ضرب کرے گا۔ جدول \حوالہ{جدول_ابعادی_شریک_لیژانڈر_تفاعلات} میں \عددی{\cos\theta} کے چند شریک لیژانڈر تفاعلات پیش کیے گئے ہیں۔)
\begin{table}
\caption{چند شریک لیژانڈر تفاعلات \عددی{{P_l^m(\cos\theta)}}}
\label{جدول_ابعادی_شریک_لیژانڈر_تفاعلات}
\centering
\begin{tabular}{ll}
$P_0^0=1$ & $P_2^0=\frac{1}{2}(3\cos^2\theta-1)$\\[0.25em]
$P_1^1=\sin\theta$ & $P_3^3=15\sin\theta(1-\cos^2\theta)$\\[0.25em]
$P_1^0=\cos\theta$ & $P_3^2=15\sin^2\theta\cos\theta$\\[0.25em]
$P_2^2=3\sin^2\theta$ & $P_3^1=\frac{3}{2}\sin\theta(5\cos^2\theta-1)$\\[0.25em]
$P_2^1=3\sin\theta\cos\theta$ & $P_3^0=\frac{1}{2}(5\cos^3\theta-3\cos\theta)$
\end{tabular}
\end{table}

دھیان رہے کہ صرف غیر منفی عدد صحیح \عددی{l} کی صورت میں کلیہ روڈریگیس معنی خیز ہو گا؛ مزید \عددی{|m|\textgreater{l}} کی صورت میں مساوات \حوالہ{مساوات_ابعادی_شریک_لیژانڈر_تفاعلات_تعریف} کے تحت \عددی{P_{l}^{m}=0} ہو گا۔ یوں \عددی{l} کی کسی بھی مخصوص قیمت کے لئے \عددی{m} کی \عددی{(2l+1)} ممکنہ قیمتیں ہوں گی:
\begin{align}
l=0,1,2,\dotsc;\quad{m=-l,-l+1,\dotsc-1,0,1,\dotsc l-1,l.} 
\end{align}
%%%%%%%%%%%%% AM HERE
لیکن رکیے مساوات 
4.25
ایک رتبی تفرقی مساوات ہے جسکے l اور m کی کسی بھو قیمتوں کے لئے دو خطی غیرتابہ حل ہونگے۔ باکی حل کہا ہیں؟
جواب :
یہ مساوات کی رعایتی حلوں کی  صورت میں ضرور موجود ہیں لیکن یہ طبی بنیادوں پر ناقابلِ قبول ہیں چونکہ
\عددی{\theta=0}
اور/یا
\عددی{\theta=\pi}
پر یہ بےقابو بڑھتے ہیں (سوال 4.4 دیکھیں )۔
اب قروی محدد میں ھجلی رکن درجذیل ہو گا
\begin{align}
d^{3}r=r^{2}sin\theta{dr}d\theta{d\phi}. 
\end{align}
لہٰذا معمول پر لانے کی شرط (مساوات 
4.6)
درجزیل اختیار کرے گی
\begin{align}
\int|\psi|^{2}r^{2}sin\theta{drd\theta}d\phi-\int|R|^{2}r^{2}dr\int|Y|^{2}sin\theta{d\theta}d\phi=1. 
\end{align}
R اور Y کو علیحدہ علیحدہ طور پر معمول پر لانا زیادہ آسان صابط ہوتا ہے:
\begin{align}
\int_{0}^{\infty}|R|^{2}r^{2}dr=1\quad{and}\int_{0}^{2\pi}\int_{0}^{\pi}|Y|^{2}sin\theta{d\theta}d\phi=1 
\end{align}
معمول شدہ  دائریعی موجی دفاتوں کو قروی ہارمونیات کہتے ہیں :
\begin{align}
\boxed{Y_{l}^{m}(\theta,\phi)=\epsilon\sqrt{\frac{(2l+1)}{4\pi}\frac{(l-|m|)!}{(l+|m|)!}}e^{\iota{m\phi}}P_{l}^{m}(cos\theta.} 
\end{align}
جہاں
\عددی{\epsilon=(-1)^{m}}
کے لئے
\عددی{m\geq0}
اور
\عددی{\epsilon=1}
کے لئے
\عددی{m\leq0}
ہو گا۔ جیسا کے ہم بعد میں ثابت کریں گے یہ ازخود عمودی ہیں
\begin{align}
\int_{0}^{2\pi}\int_{0}^{\pi}[Y_{l}^{m}(\theta,\phi)]^{*}[Y_{l'}^{m'}(\theta,\phi)]sin\theta{d\theta}d\phi=\delta_{ll'},\delta_{mm'} 
\end{align}
جو جدول
4.3
میں چند ابتدائ قروی ہارمونیات پیش کیے گئے ہیں۔تاریخی وجوہات کی بینا L کو شمالی quanti عدد جبکے M کو مقناطیسی quanti عدد کہتے ہیں۔
سوال4.3:
مساوات 4.27,4.28اور 4.32 استعمال کرتے ہوئے 
\عددی{Y_{0}^{0}}
اور
\عددی{Y_{2}^{1}}
تیار کریں۔ اسکی تصدیق کریں کے یہ عمودی اور معمول شدہ ہیں۔
سوال 4.4:
دیکھائیں کہ
\begin{align}
\Theta(\theta)=Aln[tan(\theta/2)] 
\end{align}
تمام
\عددی{l=m=0}
کے لئے مساوات
\عددی{\theta}
(مساوات 4.25) کو مطمئین کرتا ہے۔یہ وہ دوسرا ناقابلِ قبول حل ہے۔اس میں کیا خرابی ہے ؟
سوال 4.5:
مساوات 4.32 کو استعمال کرتے ہوے
\عددی{Y_{1}^{1}(\theta,\phi)}
اور
\عددی{Y_{3}^{2}(\theta,\phi)}
تیار کریں۔(آپ
\عددی{P_{3}^{2}}
کو جو جدول 4.2 سے لے سکتے ہیں،جبکے
\عددی{P_{1}^{1}}
کو مساوات 4.27 ,4.28کی مدد سے آپکو تیار کرنا ہو گا)۔تصدیق کیجئے کے l اور m کی موضوں قیمتوں کیلئے یہ زاویائ مساوات (مساوات 4.18)، کو  مطمعین کرتے ہیں۔
سوال 4.6:
قلیہ rodrigues سے ابتدہ کرتے ہوئے legendre کثیرقنیوں کی معیاری امودیت کی شرط
\begin{align}
\int_{-1}^{1}P_{l}(x)P_{l'}(x)dx=\big(\frac{2}{2l+1}\big)\delta_{ll'} 
\end{align}
اخذ کریں۔ (اشارہ :تکمل حصۓ استعمال کریں )
حصہ 4.1.3 راداسی مساوات:
دیہان رہے کہ قروی طور پر تشاقلی تمام مخفی توانائیوں کے لئے تفعال موج کا زاوئی حصہ
\عددی{Y(\theta,\phi)}
ایک جیسا ہو گا ؛مخفی توانائی
\عددی{V(r)}
کی شقل وصورت تفعال موج کی صرف راداسی حصہ
\عددی{R(r)}
پر اثر انداز ہو گا جسے مساوات 4.16 تعین کرے گا:
\begin{align}
\frac{d}{dr}\big(r^{2}\frac{dR}{dr}\big)-\frac{2mr^{2}}{\hslash^{2}}[V(r)-E]R=l(l+1)R. 
\end{align}
نیے متغیورات استعمال کرتے ہوے اس مساوات کی سادہ روپ حاصل کی جاسکتی ہیں۔ درج زیل فرض کریں
\begin{align}
u(r)\equiv{rR(r).} 
\end{align}  
تاکہ
\عددی{R=u/r,dR/dr=[r(du/dr)-u]/r^{2},(d/dr)[r^{2}(dR/dr)]=rd^{2}u/dr^{2},}
ہو لہٰذا درجزیل ہو گا
\begin{align}
\boxed{-\frac{\hslash^{2}}{2m}\frac{d^{2}u}{dr^{2}}+\big[V+\frac{\hslash^{2}}{2m}\frac{l(l+1)}{r^{2}}\big]u=Eu} 
\end{align}
اسکو قداسی مساوات کہتے ہیں۔یہ شکل و صورت کے لہٰذ سے یقبودی shrondinger مساوات 2.5 کی طرح ہے ماسواۓ مؤثر مغفی توانائی
\begin{align}
V_{eff}=V+\frac{\hslash^{2}}{2m}\frac{l(l+1)}{r^{2}} 
\end{align}
کہ جس میں
\عددی{(\hslash^{2}/2m)[l(l+1)/r^{2}]}
کا اضافی جزو پایا جاتا ہے جسے مرکز گریز جزو کہتے ہیں۔ یہ كلاسیكی مکانیات کے مرکز گریز نقلی قوت کی طرح زرہ کو ممدہ سے دور دکھیلتا ہے اب معمول پر لانے کی شرط (مساوات 4.31 )درج ذیل روپ اختیار کرتی ہے 
\begin{align}
\int_{0}^{\infty}|u|^{2}dr=1 
\end{align}
کوئی مخصوص مخفی توانائی
\عددی{V(r)}
نہ جانتے ہوۓ ہم صرف یہاں تک جاسکتے ہیں۔
مثال 4.1:
درج ذیل لامتناحی قروی قواں پر غور کریں,
\begin{align}
V(r)=\begin{cases}
0&r\leq{a}\\
\infty&r>a
\end{cases} 
\end{align}
اسکی تفعال موج اور اجازتی توانائیاں تلاش کریں۔
حل :
قواں کے باہر تفعال موج صفر ہے جبکے قواں کے اندر راداسی مساوات درج ذیل کہتی ہے
\begin{align}
\frac{d^{2}u}{dr^{2}}=\big[\frac{l(l+1)}{r^{2}}-k^{2}\big]u 
\end{align}
جہاں ہمیشہ کی طرح درج ذیل ہو گا 
\begin{align}
k\equiv\frac{\sqrt{2mE}}{\hslash} 
\end{align}
ہم نے اس مساوات کو سرحدی شرط
\عددی{u(a)=0}
مسلط کرتے ہوۓ حل کرنا ہے۔ سب سے آسان صورت L= 0 کی ہے:
\begin{align}
\frac{d^{2}u}{dr^{2}}=-k^{2}u\to{u(r)=Asin(kr)+Bcos(kr)} 
\end{align}
یاد رہے کے اصل تفعال موج
\عددی{R(r)=u(r)/r,}
ہے لیکن
\عددی{[cos(kr)]/r}
کی صورت میں
\عددی{r\to{0}}
بےقابو بڑھتا ہے۔یوں ہمیں
\عددی{B=0}
منتخب کرنا ہو گا اب سرحدی شرط پر پورا اترنے کے لئے ضروری ہے کہ
\عددی{sin(ka)=0,}
لہٰذا
\عددی{ka=n\pi}
جہا n کوئی بھی عدد سہی ہو سکتا ہے ظاہر ہے کہ اِجارتی توانائیاں درج ذیل ہونگی 
\begin{align}
E_{\pi0}=\frac{n^{2}\pi^{2}\hslash^{2}}{2ma^{2}},\quad(n=1,2,3,...). 
\end{align}
جو ہیں یقبودی لامتناہی چکور قواں کی توانائیاں ہیں (مساوات 2.7)
\عددی{u(r)}
کو معمول پر لانے سے
\عددی{A=\sqrt{2/a};}
حاصل ہو گا۔ زاویائ جزو
\عددی{Y_{0}^{0}(\theta,\phi)=1/\sqrt{4\pi},}
ہونے کے بینا غیر اہم ہے کو ساتھ منسلک کرتے ہوئے درج ذیل حاصل ہو گا
\begin{align}
\psi_{\pi00}=\frac{1}{\sqrt{2\pi{a}}}\frac{sin(n\pi{r/a})}{r} 
\end{align}
]دیہان کیجیے گا کہ ساکن حالت کو تین quanti اعداد n,l اور m سے نام دیے گئے ہیں:
\عددی{\psi_{nml}(r,\theta,\phi).}
جبکہ توانائی, صرف n اور l پر منحصر ہوگی:
\عددی{E{nl}.]}
کسی بھی ایک اختیاری عدد سےہی L کے لئے مساوات 4.41 کا عمومی حل اتنا جانا پہچانا نہیں ہے 
\begin{align}
u(r)=Arj_{l}(kr)+Brn_{l}(kr). 
\end{align}
جہاں
\عددی{j_{l}(x)}
رتبی قروی Bessel دفعال ہے اور 
\عددی{l,n_{i}(x)}
رتبی قروی Neumann دفعال l ہے انکی تعریفات درج ذیل ہیں 
\begin{align}
j_{l}(x)\equiv(-x)^{l}\big(\frac{1}{x}\frac{d}{dx}\big)^{l}\frac{sinx}{x}:\quad{n_{i}(x)\equiv-(-x)^{l}}\big(\frac{1}{x}\frac{d}{dx}\big)^{l}\frac{cosx}{x}. 
\end{align}
مثال کے طور پر,
\begin{align*}
j_{0}(x)&=\frac{sinx}{x};\quad{n_{0}(x)=-\frac{cosx}{x};}\\
j_1(x)&=(-x)\frac{1}{x}\frac{d}{dx}\big(\frac{sinx}{x}\big)=\frac{sinx}{x^{2}}-\frac{cosx}{x};\\
j_{2}(x)&=(-x)^{2}\big(\frac{1}{x}\frac{d}{dx}\big)^{2}\frac{sinx}{x}=x^{2}\big(\frac{1}{x}\frac{d}{dx}\big)\frac{xcosx-sinx}{x^{3}}\\
\end{align*}
\begin{align}
=\frac{3sinx-3xcosx-x^{2}sinx}{x^{3}}: 
\end{align}
جو جدول 4.4 میں ابتدائ چند قروی Bessel اور Neumann دفعال پیش کیے گئے ہیں متغیر x کی چھوٹی قیمت کے لئے (یہاں
\عددی{sinx=x-x^{3}/3!+x^{5}/5!...}
اور
\عددی{cosx=1-x^{2}/2+x^{4}/4!...),}
\begin{align}
j_{0}(x)\approx1;\quad{n_{0}(x)\approx-\frac{1}{x}};\quad{j_{1}(x)\approx\frac{x}{3}};\quad{j_{2}(x)\approx\frac{x^{2}}{15}}; 
\end{align}
وغیرہ ہونگے۔ دیہان رہے ممدا پر Bessel دفعال متناہی رہتے ہیں جبکہ Neumann دفعالات بےقابو بڑھتے ہیں یوں لازم ہے کے ہم
\عددی{B_{l}=0}
منتخب کریں لہٰذا درج ذیل ہو گا
\begin{align}
R(r)=Aj_{l}(kr) 
\end{align}
اب سرحدی شرط,
\عددی{R(a)=0}
کو مطمئن کرنا باکی ہے ظاہر ہے کے k کو درج ذیل کے تحت منتخب کرنا ہو گا
\begin{align}
j_{l}(ka)=0; 
\end{align}
یعنی l رتبی قروی Bessel دفعال کا(ka) ایک صفر ہو گا اب ایک Bessel دفعالات ایہتحاشی ہیں (شکل 4.2 دیکھیں) ; ہر ایک کے لامتناہی تعداد کےصفر پاے جاتے ہیں, لیکن ہماری بدقسمتی ہے کہ یہ ایک جیسے فاصلوں پر نہیں پائے جاتے ہیں جیسا کہ نقطہ n یا نقطہ
\عددی{n\pi}
وغیرہ پر۔انہیں اعدادی طور پر حاصل کرنا ہو گا۔بحرحال سرحدی شرط کے تحت درج ذیل ہو گا
\begin{align}
k=\frac{1}{a}\beta_{nl} 
\end{align}
جہاں
\عددی{\beta_{nl}}
قروی Bessel دفعال کا
\عددی{nth}
صفر
\عددی{lth}
ہو گا۔یوں اجارتی توانائیاں درج ذیل ہونگی
\begin{align}
E_{nl}=\frac{\hslash^{2}}{2ma^{2}}\beta_{nl}^{2}. 
\end{align}
اور دفعال موج درج ذیل ہونگے 
\begin{align}
\psi_{nlm}(r,\theta,\phi)=A_{nl}j_{l}(\beta_{nl}r/a)Y_{l}^{m}(\theta,\phi). 
\end{align}
جہاں معمول زنی مستقل 
\عددی{A_{n1}}
 تعین کریں گے۔چونکہ l کی ہر ایک قیمت کے لئے m کی
\عددی{(2l+1)}
مختلف قیمتیں پای جاتی ہیں لہٰذا توانائی کی ہر ستہ
\عددی{2l+1}
گناہ انہتاتی ہو گا۔
سوال 4.7:
جزو الف :
\عددی{n_{1}(x)}
اور
\عددی{n_{2}(x)}
کو (مساوات 4.46) میں پیش کی گئی تعریف سے تیار کریں۔
جزو ب :
\عددی{sines}
اور
\عددی{cosines}
کو پھیلا کر
\عددی{n_{1}(x)}
اور
\عددی{n_{2}(x)}
کی تخمینی قلیات اخذ کریں جو
\عددی{x\ll1}
کے لئے كارامد ہونگیں۔تصدیق کریں کے یہ ممدا پر بےقابو بڑھتے ہیں۔
سوال 4.8:
جزو الف : تصدیق کریں کہ
\عددی{V(r)=0}
اور
\عددی{l=1}
کی صورت میں
\عددی{Arj_{1}(kr)}
راداسی مساوات کو مطمئن كرتا ہے۔
جزو ب : لامتناہی قروی قواں کیلئے L =1 کی صورت میں اجاراتی توانایاں ترسیم کی مدد سے تعین کریں۔ دیکھائیں کے n کی بڑی قیمت کے لئے 
\عددی{E_{n1}\approx(\hslash^{2}\pi^{2}/2ma^{2})(n+1/2)^{2}}
ہو گا۔(اشارہ : پہلے دیکھائیں کے
\عددی{j_{1}(x)=0\to{x}=tanx}
اسکے بعد
\عددی{x}
اور
\عددی{tanx}
کو ایک ہی جگہ ترسیم کرتے ہوئے انکے نقاط تلاش کریں۔ )
سوال 4.9: ایک ذرہ جسکی قیمت m ہے کو متناہی قروی قواں:
\begin{align}
V(r)=\begin{cases}-V_{0}&r<a\\0&r>a\\\end{cases} 
\end{align}
میں رکھا جاتا ہے اسکے زمینی حال کو l=0 لیتے ہوئے راداسی مساوات کے حل سے حاصل کریں۔ دیکھائیں کے
\عددی{V_{0}a^{2}\textless\pi^{2}\hslash^{2}/8m}
کی صورت میں مقیت حل نہیں پایا جاتا ہے۔
%145-150
حصہ ۴.۲
ہائیڈروجن جوہر ایک بھاری پروٹان جس کو ساکن تصور کیا جا سکتا ہے اور جس کا بار e ہے کے  گرد ایک حلقے الیکٹرون جس کا بار منفی e طواف کرتا ہے پر مشتمل ہوتا ہے ان دونوں کے مخالف بارو کے بیچ قوتِ کشش پائی جاتی ہے۔
 شکل ۴.۳
 کولوم کے قانون کے تحت مخفی توانائی درج ذیل ہوگی۔  
 \begin{align}
V(r)=\frac{-e^{2}}{4\pi\epsilon_{0}}\frac{1}{r} 
\end{align}
لہٰذا رداسی مساوات ۴.۳۷ درج ذیل روپ اختیار کرے گی۔
\begin{align}
\frac{-\hslash^{2}}{2m}\frac{\dif^{2}{u}}{\dif{r^{2}}}+\big[\frac{-e^{2}}{4\pi\epsilon_{0}}\frac{1}{r}+\frac{\hslash^{2}}{2m}\frac{l(l+1)}{r^{2}}\big]u=Eu 
\end{align}
 ہم نے اس مساوات کو 
 \عددی{u(r)}
  کے لئے حل کرکے اجازتی توانائیاں E تعین کرنی ہے۔  ہائیڈروجن جوہر کا حل اتنا اہم ہے کہ میں اس کو ہارمونی مرتاہش کے تہلیلی حل کے طریقے سے قدم با قدم حل کر کے پیش کرتا ہوں، جس قدم میں آپ کو دشواری پیش آئے، حصہ ۲.۳.۲ سے مدد لیں وہاں مکمل تفصیل پیش کی گئی ہے۔ کولوم مخفی توانائی مساوات۴.۵۲ 
 \عددی{E>0}
 استمراری حالات جو الیکٹران پروٹون ٹکراؤ کو ظاہر کرتے ہیں تسلیم کرنے کے ساتھ ساتھ غیر مسلسل مقید حالات جو ہائیڈروجن جوہر کو ظاہر کرتے ہے بھی تسلیم کرتا ہے۔ ہماری دلچپسی موخر الذِکر میں ہے۔
 حصہ ۴.۲.۱
 رداسی تفاعل موج، سب سے پہلے نئی علامتیں متعارف کرتے ہوئے مساوات کی صدا صورت حاصل کرتے ہیں درج ذیل لیتے ہوئے جہاں مقید حالات کے لئے e منفی ہونے کی وجہ سے k  حقیقی ہو گا۔
 \begin{align}
k\equiv \frac{\sqrt{-2mE}}{\hslash} 
\end{align}
 مساوات ۴.۵۳ کو E سے تقسیم کرتے ہوئے درج ذیل حاصل ہو گا
\begin{align}
\frac{1}{k^{2}}\frac{\dif^{2}{u}}{\dif{r^{2}}}=\big[1-\frac{me^{2}}{2\pi\epsilon_{0}\hslash^{2}k}\frac{1}{(kr)}+\frac{l(l+1)}{(kr)^{2}}\big]u 
\end{align}
اس کو دیکھ کر ہمیں خیال آتا ہے کہ ہم درج ذیل علامتیں متعارف کرائے 
\begin{align}
\rho\equiv kr \quad \rho_{0}\equiv\frac{me^{2}}{2\pi\epsilon_{0}\hslash^{2}k} 
\end{align}
لہٰذا درج ذیل لکھا جائے گا۔
\begin{align}
\frac{\dif^{2}{u}}{\dif{\rho^{2}}}=\big[1-\frac{\rho_{0}}{\rho}+\frac{l(l+1)}{\rho^{2}}\big]u 
\end{align}
اِس کے بعد ہم حلو کی متاکاربی روپ پر غور کرتے ہیں۔ اب
\عددی{\rho\to\infty}
کرنے سے قوسین کے اندر مستقل جزو غالب ہو گا لہٰذا تخمینی طور پر درج ذیل لکھا جا سکتا ہے۔
\begin{align}
\frac{\dif^{2}{u}}{\dif{\rho^{2}}}=u 
\end{align}
جس کا عمومی حال درج زیل ہے۔
\begin{align}
u(\rho)=Ae^{-\rho}+Be^{+\rho} 
\end{align}
تا ہم 
\عددی{\rho\to\infty}
کی صورت میں 
\عددی{e^{\rho}}
پے قابو برتا ہے لہذا ہمیں بھی
B=0
یوں 
\عددی{\rho}
کی بڑی قیمتوں کے لیے درج  ذیل ہو گا 
\begin{align}
u(\rho)\sim Ae^{-\rho} 
\end{align}
اس کے برعکس 
\عددی{\rho\to 0}
کی صورت میں مرکز گریز جزو غالب ہو گا۔ لہٰذا تخمينی طور پر درج ذیل لکھا جاسکتا ہے۔ 
\begin{align}
\frac{\dif^{2}{u}}{\dif{\rho^{2}}}=\frac{l(l+1)}{\rho^{2}}u 
\end{align}
 جس کا عمومی حل تصدیق کیجئے درج ذیل ہو گا۔
 \begin{align}
u(\rho)=C\rho^{l+1}+D\rho^{-l} 
\end{align}
 تاہم
 \عددی{(\rho\to\infty)}
 کی صورت میں
 \عددی{\rho^{-l}}
 پے قابو بھڑتا ہے لہذا
 D=0
 ہو گا۔
 \عددی{\rho}
 یوں 
 کی چھوٹی قیمتوں کے لیے درج ذیل ہو گا۔
 \begin{align}
u(\rho)\sim C\rho^{l+1} 
\end{align}
  دوسری قدم پر ہمیں متاکاربی رویئے کو ہٹانا ہو گا اس کی خاطر ہم نیا تفالوی رو اس اُمید سے متعارف کرتے ہے کے یہ
   \عددی{u(\rho)}
    رو سے زیادہ سادہ ہو گا۔
  \begin{align}
u(\rho)=\rho^{l+1}e^{-\rho}x(\rho) 
\end{align}
  پہلے اشارے اچھے نظر نہیں آتے

  اس طرح v رو کی صورت میں رداسی مساوات ۴.۵۶ 
  درج ذیل روپ اختیار کرتی ہے۔
  \begin{align}
\rho\frac{\dif^{2}{v}}{\dif{\rho^{2}}}+2(1+l-\rho)\frac{\dif{v}}{\dif{\rho}}+[\rho_{0}-2(l+1)]v=0 
\end{align}
  آخرمیں ہم فرض کرتے ہیں کہ حل
  \عددی{v}
    رو کو رو کی طاقتی تسلسل لکھا جا سکتا ہے۔
  \begin{align}
v(\rho)=\sum_{j=0}^{\infty}c_{j}\rho^{j} 
\end{align}
اب ہمیں عددی سر 
\عددی{(c_{0},c_{1},c_{2}...)}
تعین کرنے ہے ہم جزو در جزو تفرق لیتے ہیں۔
\begin{align}
\frac{\dif{v}}{\dif{\rho}}=\sum_{j=0}^{\infty}jc_{j}\rho^{j-1}=\sum_{j=0}^{\infty}(j+1)c_{j+1}\rho^{j} 
\end{align}
میں نے دوسرے مجموعہ میں فرضی اشاریہ j کو 
\عددی{j\to j+1}
کہا ہے اگر آپکو اس سے پریشانی ہو تو آپ اولین چند اجزاء صریحاً لکھ کر اس کی درستگی کی تصدیق کر سکتے ہیں۔ آپ سوال اٹھا سکتے ہیں کے اب مجموعہ
\عددی{j=-1}
شروع ہونا چاہئے لیکن
\عددی{(j+1)}
کا جزو اس جزو کو ختم کرتا ہے لہذا ہم ۰ سے بھی شروع کر سکتے ہیں۔ دوبارہ تفرق لیتے ہیں
\begin{align}
\frac{\dif^{2}{v}}{\dif{\rho^{2}}}=\sum_{j=0}^{\infty}j(j+1)c_{j+1}\rho^{j-1} 
\end{align}
مساوات ۴.۶۱ میں پر کرتے ہیں
\begin{align}
\sum_{j=0}^{\infty}j(j+1)c_{j+1}\rho^{j}+2(l+1)+\sum_{j=0}^{\infty}(j+1)c_{j+1}\rho^{j} 
\end{align}
\begin{align}
-2\sum_{j=0}^{\infty}jc_{j}\rho^{j}+[\rho_{0}-2(l+1)]\sum{j=0}^{\infty}c_{j}\rho^{j}=0 
\end{align}
ایک جیسی طاقتوں کے عددی سروں کو مساوی رکھتے ہوئے درج ذیل حاصل ہو گا۔
\begin{align}
j(j+1)c_{j+1}+2(l+1)(j+1)c_{j+1}-2jc_{j}+[\rho_{0}-2(l+1)]c_{j}=0 
\end{align}
یا
\begin{align}
c_{j+1}=\big\{\frac{2(j+l+1)-\rho_{0}}{(j+1)(j+2l+2)}\big\}c_{j} 
\end{align}
یہ کلیہ توالی عددی سر تعین کرتے ہوئے تفالوي رو تعین کرتے ہیں۔ ہم
\عددی{c_{0}}
سے شروع کر کے جو مجموعی مستقل کا روپ اختیار کرتا ہے جس کو آخر میںے معمول پے لیتے ہوئے حاصل کیا جائےگا۔
مساوات ۴.۶۳ کی مدد سے 
\عددی{c_{1}}
حاصل کرتے ہے جس کو واپس اسی مساوات میں پر کرتے ہوئے
\عددی{c_{2}}
حاصل ہو گا وغیرہ وغیرہ آئے J کی بری قیمت کے لئے عددی سروں کی صورت دیکھے۔ J کی بڑی قیمت 
\عددی{\rho}
 کی بڑی قیمت کو ظاہر کرتی ہے جہاں بلند طاقتیں غالب ہونگی۔ اس صورت میں کلیہ توالي درج ذیل رہتی ہے۔
\begin{align}
c_{j+1}\cong\frac{2j}{j(j+1)}c_{j}=\frac{2}{(j+1)}c_{j} 
\end{align}
ایک لمحہ کے لیے فرض کرے کہ یہ بلکل ٹھیک رشتہ ہے تب درج ذیل ہو گا۔
\begin{align}
c_{j}=\frac{2^{j}}{j!}c_{0} 
\end{align}
جس کی بنا پر درج ذیل لکھا جا سکتا ہے۔
\begin{align}
v(\rho)=c_{0}\sum_{j=0}^{\infty}\frac{2^{j}}{j!}\rho^{j}=c_{0}e^{2\rho} 
\end{align}
لہذا درج ذیل ہو گا۔
\begin{align}
u(\rho)=c_{0}\rho^{l+1}e^{\rho} 
\end{align}
 جو
  \عددی{\rho} 
 کی بڑی قیمتوں کے لیے پے قابو بڑھتا ہے۔ مثبت قوت نما وہی متاکربی رویا دیتا ہے جو ہمیں مساوات ۴.۵۷ میں نہیں چاہیے تھا۔ حقیقت میں متاکربی حل بھی رداسی مساوات کے جائز حل ہے البتہ ہم ان میں دلچسپی نہیں رکھتے ہے چونکہ یہ معمول پر نہیں لائے جاسکتے ہیں۔ اس علمیہ سے نجات کا صرف ایک ہی راستہ ہے ، تسلسل کو کہیں نہ کہیں ختم ہونا ہو گا لازمی طور پر ایسا زیادہ سے زیادہ عادت سے ہی J بلند تر پایا جائیگا جہاں درج ذیل ہو گا۔
 \begin{align}
c_{j_{\text{max}}+1}=0 
\end{align}
 یوں کلیہ توالی کے تحت باقی تمام عددی سر ۰ ہونگے ظاہر ہے مساوات ۴.۶۳ کے درج ذیل ہو گا
 \begin{align}
2(j_{\text{max}}+l+1)-\rho_{0}=0 
\end{align}
 صدر کوانٹم 
 \begin{align}
n\equiv j_{\text{max}}+l+1 
\end{align}
 متعارف کرتے ہوئے درج ذیل ہو گا۔
 \begin{align}
\rho_{0}=2n 
\end{align}
 اب E کو 
 \عددی{\rho_{0}}
 تعین کرتا ہے مساوات ۴.۵۴ اور ۴.۵۵
 \begin{align}
E=\frac{-\hslash^{2}k^{2}}{2m}=\frac{-me^{4}}{8\pi^{2}\epsilon^{2}\hslash^{2}\rho^{2}} 
\end{align}
 لہٰذا اجزاتی توانائیاں درج ذیل ہونگی 
 \begin{align}
\boxed{E_{n}=-\big[\frac{m}{2\hslash^{2}}\big(\frac{e^{2}}{4\pi\epsilon}\big)^{2}\big]\frac{1}{n^{2}}=\frac{E_{1}}{n^{2}}\quad n=1,2,3,\dotsc} 
\end{align}
 یہ مشہور زمانہ کلیہِ بہر ہے جو غالباً پورے کوانٹم میکانیات میں  سب سے اہم ترین نتیجہ ہے جناب بہر نے سن۱۹۱۳ میں یہ کلیہ کوانٹم میکانیات سے قبل تقریبن اندازے سے اخذ کیا۔ شروڈنگر مساوات سن ۱۹۲۴ میں منظر پر آی مساوات  ۴.۵۵ اور ۴.۶۸ کو ملا کر درج ذیل حاصل ہو گا۔
\begin{align}
k=\big(\frac{me^{2}}{4\pi\epsilon_{0}\hslash^{2}}\big)\frac{1}{n}=\frac{1}{an} 
\end{align}
جہاں
\begin{align}
\boxed{a\equiv\frac{4\pi\epsilon_{0}\hslash^{2}}{me^{2}}=0.529\times 10^{-10}\text{m}} 
\end{align}
رداس بہر کہلاتا ہے۔ یوں مساوات ۴.۵۵ کے تحت درج ذیل ہو گا۔
\begin{align}
\rho=\frac{r}{an} 
\end{align}
%150-153
ہائیڈروجن جوہر لے فضائی تفاعل  امواج کو 3 کوانٹم اعداد M اور N٫L سے نام دیا جاتا ہے 
 \begin{align}
\psi_{n,l,m}(r,\theta,\phi)=R_{nl}(r)Y_{l}^{m}(\theta,\phi) 
\end{align}
 جہاں مساوات 4.36 اور 4.60 کو دیکھتے ہوئے
 \begin{align}
R_{n,l}(r)=\frac{1}{r}\rho^{l+1}e^{-\rho}v(\rho) 
\end{align} 
 ہو گا. جبکہ وی رو متغير رو میں جے بلندتر
\عددی{j_{max}=n-l-1}
درجہ کا كثير رکنی ہو گا جس کے عددی سر جنہے معمول پر لانا باقی ہو گا  درجہ ذیل کلیہ توالی دے گا
 \begin{align}
c_{j+1}=\frac{2(j+l+1-n)}{(j+1)(j+2l+2)}c_{j} 
\end{align}
كم سے كم توانائی کا حال جسے زمینی حال کہتے ہیں٬ کے لیے
 \عددی{n=1} 
ہو گا. تبی مستقلوں کی یہ قیمتیں پر کرتے ہوئے درجہ ذیل حاصل ہو گا
 \begin{align}
\boxed{E_{1}=-\big[\frac{m}{2\hslash^{2}}\big(\frac{e^{2}}{4\pi\epsilon}\big)^{2}\big]=13.6\text{eV}} 
\end{align}
. اس سے ظاہر ہے کہ  ہائیڈروجن کی توانائی بھندن 13.6 eV ہے. یہ وہ توانائی ہے جو زمینی حال میں الیکٹرون کو محیا کرنے سے ایٹم بدارا بن جائے گا. مساوات 4.67 کے تحت l=0  لہذا m=0  ہو گا. مساوات 4.29 دیکھے یوں درجہ ذیل ہو گا
 \begin{align}
\psi_{100}(r,\theta,\phi)=R_{10}(r)Y_{0}^{0}(\theta,\phi) 
\end{align}
 کلیہ توالی پہلے جزو پرشی رکھ جاتا ہے. j=0 کے لئیے مساوات 4.76 سے
 \عددی{c_{1}=0} 
 حاصل ہو گا. یوں وی رو ایک مستقل 
  \عددی{c_{0}}
  ہو گا. لہذا درجہ ذیل ہو گا
   \begin{align}
R_{10}(r)=\frac{c_{0}}{a}e^{-r/a} 
\end{align}
   اس کو مساوات 4.31 کے تحت معمول پر لانے
سے
..\begin{align}
\int_{0}^{\infty}\abs{R_{10}}^{2}r^{2}\dif{r}=\frac{c_{0}^{2}}{a^{2}}\int_{0}^{\infty}e^{-2r/a}r^{2}\dif{r}=\abs{c_{0}}^{2}\frac{a}{4}=1 
\end{align}
حاصل ہو گا. ساتھ ہی
\عددی{c_{0}=2/\sqrt{a}}
لہذا بائیڈروجن کا زمینی حال درجہ ذیل ہو گا
\begin{align}
\psi_{100}(r,\theta,\phi)=\frac{1}{\sqrt{\pi a^{3}}}e^{-r/a} 
\end{align}
اسی طرح n=2  کے لئے توانائی درجہ ذیل ہوگی
\begin{align}
E_{2}=\frac{-13.6\text{eV}}{4}=-3.4\text{eV} 
\end{align}
چونکہ
 \عددی{m=2} 
کے لئے یا 
\عددی{l=0} 
ہو گا جو
\عددی{ m=0} 
دیتا ہے يا
 \عددی{l=1} 
ہو گا جو 1 اور M= -1,0 دیتا ہے لہذا چار مختلف حالات کی توانائی E2 ہوگی. کلیہ توالی مساوات 4.76  L=0 کے لیئے درجہ ذیل دے گا:
\begin{align}
c_{1}=-c_{0} (j=0) \quad c_{2}=0 (j=1) 
\end{align}
لہذا
\عددی{v(\rho)=c_{0}(1-\rho)}
اور درجہ ذیل ہونگے
 \begin{align}
R_{20}(r)=\frac{c_{0}}{2a}\big(1-\frac{r}{2a}\big)e^{-r/2a} 
\end{align}
دیہان رہے کہ مختلف کوانٹم اعداد l اور N کے لئے پھیلاؤ کے عددی سر
 \عددی{c_{j}}
 مکمل تور پر مختلف ہونگے. کلیہ توالی
  \عددی{l= 1}
  کی صورت میں پہلے جزو پر اختتام پذیر ہو گا. وی رو ایک مستقل ہو گا لہذا درجہ ذیل حاصل ہو گا.
   \begin{align}
R_{21}(r)=\frac{c_{0}}{4a^{2}}re^{-r/2a} 
\end{align}
ہر ایک صورت میں معمول پر لانے سے
 \عددی{c_{0}} 
تعین ہو گا.
سوال 4.11 دیکھے. کسی بھی n کے لئے مساوات 4.67 کی بلا تضاد L کی ممکنہ قیمتیں درجہ ذیل ہوں گی 
\begin{align}
0,1,2\cdots n-1 
\end{align}
جبکہ ہر
 \عددی{l }
کے لئیے 
\عددی{m}
 کی ممکنہ قیمتوں کی تعداد
 \عددی{ 2l+1} 
 ہوگی (مساوات 4.29)۔ لہذا
  \عددی{E_{n}}
  سطح کی توانائی کے لئے کل انحطاطيت درجہ ذیل ہوگی.
\begin{align}
d(n)=\sum_{l=0}^{n-1}(2l+1)=n^{2} 
\end{align}
كثير رکنی وی رو جو مساوات 4۔76 سے حاصل ہو گی ایک ایسا تفاعل ہے جس سے عملی ریاضی دان بخوبی واقف ہیں. ما سوائے معمول زنی کے اسے تجذيل لکھا جاسکتا ہے.
 \begin{align}
v(\rho)=L_{n-l-1}^{2l+1}(2\rho) 
\end{align}
 جہاں
  \begin{align}
L_{q-p}^{p}(x)\equiv(-1)^{p}\big(\frac{\dif}{\dif{x}}\big)^{p}L_{q}(x) 
\end{align} 
 ایک شریک لاگیغ  كثير رکنی ہے جبکہ 
\begin{align}
 L_{q}(x)\equiv e^{x}\big(\frac{\dif}{\dif{x}}\big)^{q}(e^{-x}x^{q}) 
\end{align}
 لاگیغ كثير رکنی ہے. جدول 4.5 میں چند ابتدائی لاگيغ كثور رکنیا پیش کی گئی ہیں.  جبکہ جدول 4.6 میں چند شریک لاگيغ كثير رکنیا پیش کئے گئی ہیں. جدول 4.7 میں چند ابتدائی رداسی تفاعل امواج پیش کئے گئے ہیں جنہیں شکل 4.4 میں ترسیم کیا گیا ہے. ہائیڈروجن کے معمول شده تفاعل امواج درجہ ذیل ہیں.
  \begin{align}
\boxed{\psi_{nlm}=\sqrt{\big(\frac{2}{na^{3}}\big)\frac{(n-l-1)!}{2n[(n+1)!]^{3}}}e^{-r/na}\big(\frac{2r}{na}\big)^{l}[L_{n-l-1}^{2l+1}(2r/na)]Y_{l}^{m}(\theta,\phi)} 
\end{align}
یہ تفاعل کچ خوفناک ہیں٬ ليكن شكوه نہ کیجیے گا. یہ اُن چند حقیقی نظاموں میں سے ایک ہے جن کا مکمل حل حاصل کرنا ممکن ہے.  دیہان رہے کہ اگرچہ تفاعل امواج تینوں کوانٹم اعداد پر منحصر ہے جبکہ توانائیاں مساوات 4.70 کو صرف n تعین کرتا ہے. یہ کوولومب( coulomb) توانائی کی ایک خاصیت ہے. آپ کو یاد ہو گا کہ کروی كنواں کی صورت میں توانائیاں L پر منحصر تھی (مساوات 4.50).
تفاعل موج باہمی امودی ہوں گے 
.\begin{align}
\int\psi_{nlm}^{*}\psi_{nlm}r^{2}\sin{\theta}\dif{\theta}\dif{\phi}=\delta_{nn'}\delta_{ll'}\delta_{mm'} 
\end{align}
 یہ کروی ہارمونیات کی امودیت مساوات 4.33 کی بنا اور
\عددی{n\neq n'}
کی صورت میں تفاعلات موج کا H کی الگ تلگ امتیازی اعقدار کے امتیازی تفاعل ہونے کی بنا ہے.
پائیڈروجن تفاعلات موج کی تصویر کشی آسان کام نہیں ہے. ماحر کیمیا کسافتی اشکال بناتے ہیں جہاں چمک
\عددی{\abs{\psi}^{2}}
کا راست متناسب ہوتا ہے (شکل 4.5). ان سے زیادہ معلومات مستقل كسافت کے احتمال کی سطحوں کے اشکال دیتی ہے جنہیں پڑھنا نسبتاً مشکل ہو گا. (شکل 4.6).\\

