%%%%%%%%%%%%   NOT EDITED AT ALL YET
\باب{قواعد و ضوابط}\شناخت{باب_قواعد_و_ضوابط}


%126-128
% example 3.32
\ابتدا{سوال} 
توانائی اور وقت کی عدم یقینیت کے اصول کی ایک دلچسپ روپ
\(\Delta t=\tau/\pi\)
ہے جہاں ابتدائی حال
\(\Psi (x,0)\)
کے عموری حال تک ارتقا کے لیے درکار وقت
\(\tau\)
ہے۔ دو معیاری عموری ساکن حالات کے دو برابر حصوں پر مشتمل اختیاری مخفی قو
\(\Psi(x,0)=1/\sqrt{2}[\psi_{1}(x)+\psi_{2}(x)]\)
کا تفال موج استعمال کرتے ہوئے اس کی چانچ پڑتال کریں۔
\انتہا{سوال}

\ابتدا{سوال} 
%3.33\\
ہارمونی مرتعش کی ساکن حالات کی معیاری عموری اساس مساوات 2.67
میں
\(\langle n|x|n'\rangle\)
اور
\(\langle n|p|n'\rangle\)
تلاش کریں۔ آپ سوال 2.12 میں
\(n=n'\)
تلاش کر چکے ہیں۔ وہی ترقیب موجودہ عمومی مسئلے میں استعمال کریں۔ متکانبی لامتناہی
\(\text{X}\)
اور
\(\text{P}\)
تلاش کریں۔ دکھائیں کہ اس اساس میں
\((1/2m)\text{\text{P}}^{2}+(m\omega^{2}/2)\text{\text{X}}^{2}=\text{\text{H}}\)
وتری ہے۔ کیا اس کے وتر ی ارکان آپکے توقع کے مطابق ہیں۔
جزوی جواب
\[\langle n|x|n'\rangle=\sqrt{\frac{\hslash}{2m\omega}}(\sqrt{n'}\delta_{n,n'-1}+\sqrt{n}\delta_{n',n-1})\]
\انتہا{سوال}

\ابتدا{سوال}
ایک ہارمونی مرتعش ایسی حال میں ہے کہ اس کی توانائی کو پیمائش
\((1/2)\hslash\omega\)
یا
\((3/2)\hslash\omega\)
ایک دوسرے جیسے احتمال کے ساتھ دے گی۔ اس حال میں
\(\langle p\rangle\)
کی زیادہ سے زیادہ ممکنہ قیمت کیا ہوگی۔ اگر لمحہ
\(t=0\)
پر اس کی
\(\langle p\rangle\)
کی زیادہ سے زیادہ ممکنہ قیمت ہو تو
\(\Psi(x,t)\)
کیا ہوگا۔
\انتہا{سوال}

\ابتدا{سوال} 
3۔35\\
ہارمونی مرتعش کے منطقی حالات۔ \\
ہارمونی مرتعش کے ساکن حالات 
\(|n\rangle =\psi_{n}(x)\)
مساوات 2.67 میں صرف
\(n=0\)
عین عدم یقینیت کی حد
\(\sigma_{x}\sigma_{p}=\hslash/2\)
پر بیٹھتا ہے جیسا آپ سوال 3.12 میں معلوم کر چکے ہیں عمومی طور پر
\(\sigma_{x}\sigma_{p}=(2n+1)\hslash/2\)
ہوگا۔ تاہم چند خطی جوڑ جنہیں منطقی حالات کہتے ہیں بھی عدم یقینیت کے حاصل ضرب تو کم سے کم کرتے ہیں جیسا ہم دیکھتے ہیں یہ عامل س تکیل کے امتیازی تفال ہوتے ہیں۔\\
\[a_{-}|\alpha\rangle =\alpha|\alpha\rangle\]
جہاں امتیاز ی 
\(\alpha\)
کوئی بھی مخلوط عود ہوسکتا ہے۔\\
جز الف\\
حال 
\(|\alpha\rangle\)
میں
\(\langle x \rangle\)
,
\(\langle x^{2} \rangle\)
,
\(\langle p \rangle\)
اور
\(\langle p^{2} \rangle\)
دریافت کریں۔ مثال 2.5 کی ترقیب استعمال کیں۔ اور یاد رکھیں کہ
\(a_{-}\)
منفی کا پرمش جوڑی دار
\(a_{+}\)
ہے ساتھ ہی یہ فرض نہ کریں کہ
\(\alpha\)
حقیقی ہے۔\\
جز ( ب )\\
\(\sigma_{x}\)
اور
\(\sigma_{p}\)
تلاش کریں۔ دکھائیں کہ
\(\sigma{x}\sigma{p}=\hslash/2\)
ہوگا۔\\
جز ( ج )\\
کسی بھی دوسرے تفال موج کی طرح منطقی حال کو توانائی امتیازی حالات کی صورت میں پھیلایا جا سکتا ہے۔\\
\[|\alpha\rangle=\sum_{n=0}^{\infty}C_{n}|n\rangle\]
رکھا ئیں کہ پھیلاو کے عددی سر درج ذیل ہونگے۔\\
\[c_{n}=\frac{\alpha^{n}}{\sqrt{n!}}c_{0}\]
جر ( د )\\
\(|\alpha\rangle\)
کو معمول پر لاتے ہوئے
\(c_{0}\)
تعین کریں۔
جواب
\(\text{exp}(-\abs{\alpha}^{2}/2)\)\\
جز ( ح)\\
اس کے ساتھ وقت کی تابعیت\\
\[|n\rangle\rightarrow e^{-iE_{n}t/\hslash}|n\rangle\]
منسلک کرکے دکھائیں کہ
\(|\alpha(t)\rangle\)
اب بھی
\(a{-}\)
کا امتیازی حال ہوگا لیکن وقت کے ساتھ امتیازی قیمت ارتقا پزیر ہوگا\\
\[\alpha(t)=e^{-i\omega t}\alpha\]
یوں منظقی حال ہمیشہ منطقی حال بی رہے گا اور عدم یقینیت کے حاصل ضرب کو تم سے کم برقرار رکھا ہے۔\\
جز ( و ) کیا زمینی حال
\(|n=0\rangle\)
از خود منطقی حال ہے اگر ایسا ہو تب امتیازی قدر کیا ہوگا 
\انتہا{سوال}
%128-130
\ابتدا{سوال} 
3.36\\
مقصود عدم یقینیت کا اصول:\\
عمومی عدم یقینيت کا اصول مساوات  3.62 درجہ ذيل کہتا ہے
\[\sigma_{A}^{2}\sigma_{B}^{2}\geq\frac{1}{4}\langle C^{2} \rangle\]
جہاں
\[\hat{C}\equiv i[\hat{A},\hat{B}]\]
ہے\\ 
جذ الف\\
دکھائے کہ اس کو زیادہ مستحكم کرتے ہوئے درجہ ذیل روٹ میں لکھا جاسکتا ہے
\[\sigma_{A}^{2}\sigma_{B}^{2}\geq\frac{1}{4}(\langle C\rangle^{2}+\langle D \rangle ^{2})\]
جہاں
\(\hat{D}\equiv \hat{AB}+\hat{BA}+2\langle A \rangle \langle B \rangle\)
ہوگا\\
اشاره: مساوات 3.60 میں 
\(\text{Re}(z)\)
 اجزاه لیں\\
ب)\\
مساوات 3.99 کو
 A=B
  کے لئیے جانچیں چونکہ اس صورت میں
 C=0
   ہے لہذا معیاری عدم یقینیت غیر اہم ہوگا بدقسمتی سے مقصود عدم یقینيت کا اصول بھی زیادہ مددگار ثابت نہیں ہوتا\\
\انتہا{سوال}

\ابتدا{سوال}
 3.37:\\
ایک نظام جو 3 صدی ہے کہ ہیملٹونین کو درجہ ذيل کا الف ظاہر کرتی ہے
\[\text{\text{H}}=\begin{pmatrix}
a&0&b\\
0&c&0\\
b&0&a\\
\end{pmatrix}\]
جہاں c اور a, b حقیقی اعداد ہیں۔\\
ا) اگر اس نظام کا ابتدائی حال درجہ ذيل ہو
\[|\delta(0) \rangle=\begin{pmatrix}
0\\
1\\
0\\
\end{pmatrix}\]

تب
\(\delta(t)\)
  کیا ہو گا؟\\
ب) اگر اس نظام کا ابتدائی حال درجہ ذيل ہو
\[| \delta(0)\rangle =\begin{pmatrix}
0\\
0\\
1\\
\end{pmatrix}\]
تب
 \(\delta{t}\)
  کیا ہوگا؟\\
\انتہا{سوال}

\ابتدا{سوال}
 3.38:\\
ایک نظام جو 3 صدی ہے کہ ہيملٹونین کو درجہ ذیل کا الف ظاہر کرتی ہے:\\
\[\text{\text{H}}=\hslash\omega\begin{pmatrix}
1&0&0\\
0&2&0\\
0&0&2\\
\end{pmatrix}\]
دیگر دو مشہود B اور A کو درجہ ذيل کا الف ظاہر کرتی ہے\\
\[\text{A}=\lambda\begin{pmatrix}
0&1&0\\
1&0&0\\
0&0&2\\
\end{pmatrix} , \text{B}=\mu\begin{pmatrix}
2&0&0\\
0&0&1\\
0&1&0\\
\end{pmatrix}
\]
جہاں
\(\omega, \lambda\)
اور
\(\mu\)
حقیقی مثبت اعداد ہیں۔\\
H، A
اور
B
 کی امتیازی اقدار اور معمول پر لائے  گئے امتیازی تفاعل تلاش کریں۔ \\
ب) یہ نظام کسی عمومی حال
\[| \delta(0)  \rangle=\begin{pmatrix}
c_{1}\\
c_{2}\\
c_{3}\\
\end{pmatrix}\]
سے ابتداء کرتا ہے جہاں
\(\abs{c_{1}}^{2}+\abs{c_{2}}^{2}+\abs{c_{3}}^{2}=1\)
ہے۔\\
لمحہ t=0 پر B اور ,A Hکی توقعاتی قیمت تلاش كريں۔\\
ج)
 \(\delta(s)\) 
کیا ہوگا ؟
لمحہ t پر اس نظام کی توانائی کی پیمائش کیا قيمتيں دے سکتی ہے؟ اور ہر ایک کا انفرادى احتمال کیا ہوگا؟ اسی سوال کے جوابات B اور A کے لیے بھی تلاش كریں۔
\انتہا{سوال}

\ابتدا{سوال}
 3.39:
ا) ایک تفاعل
 \(f(x)\)
 جس کو Taylor تسلسل کی صورت میں پھیلایا جاسکتا ہے کے لیے درجہ ذيل  دکھائیں:\\
\[f(x+x_{0})=e^{i\hat{p}x_{0}/\hslash}f(x)\]
جہاں
\(x_{0}\)
کوئی بھی مستقل فاصلہ ہو سکتا ہے۔ اسی وجہ کے بنا 
\(\hat{p}/\hslash\)
کو فضا میں  انتقال کا پیداکار کہتے ہیں۔ یہاں دیہان رہے کہ ایک حامل کی قوت نمائی کی قوتی تسلسلی پھیلاؤ کی تعریف درجہ ذیل ہے
\(e^{\hat{Q}}=1+\hat{Q}+(1/2)\hat{Q}^{2}+(1/3!)\hat{Q}^{3}+\dotsc\)\\
ب) اگر
\(\Psi(x,t)\)
وقت تابع schrodinger مساوات کو مطمئن کرتا ہو تب درجہ ذیل دکھائیں\\
\[\Psi(x,t+t_{0})=e^{-i\hat{\text{H}}t_{0}/\hslash}\Psi(x,t)\]
جہاں
\(t_{0}\)
کوئی بھی مستقل وقت ہوگا۔ اسی وجہ کی بنا
\(-\hat{\text{H}}/\hslash\)
وقت میں انتقال کا پیدا کار کہتے ہے۔\\
ج) دکھائیں کے لمحہ
\(t+t_{0}\)
پر ہر کی متغير Q کی توقعاتی قیمت کو درجہ ذيل لکھا جا سکتا ہے\\
\[\langle Q \rangle _{t+t_{0}}=\langle \Psi(x,t)|e^{i\hat{\text{H}}t_{0}/\hslash}\hat{Q}(x,p,t+t_{0}e^{-i\hat{\text{H}}t_{0}/\hslash}|\Psi(x,t)\rangle\]
اس کو استعمال کرتے ہوئے مساوات 3۔71 حاصل كریں۔\\
اشاره:
 \(t_{0}=\dif{t}\) 
لیتے ہوئے dt میں پہلے رتبے تک پھیلائیں۔
\انتہا{سوال}

\ابتدا{سوال}
 3۔40:\\
ا) ایک آزاد ذرہ کے لیے وقت تابع schrodinger مساوات کو معیار حرکت فضاه میں لکھ کر حل کریں۔
جواب:  
\(\text{exp}(-ip^{2}t/2m\hslash)\Phi(p,0\))\\
ب) متحرک گاسی موجی اکٹ سوال 2۔43 کے لئیے
\(\Phi(p,0)\)
تلاش كريں اور
\(\Phi(p,t)\)
تیار كریں۔ ساتھ ہی
\(\abs{\Phi(p,t)}^{2}\)
تیار كريں۔ اب دیکھیں گے کہ یہ وقت کا تابع نہیں ہوگا۔\\
ج)
\(\Phi\)
پر مبنی موضوع تكملات حل کرتے ہوے
\(\langle p \rangle\)
اور
\(\langle p^{2} \rangle\)
تلاش کرکے سوال 2۔43 کی حاصل كرده جوابات کے ساتھ موازنہ كريں۔\\
د) دکھائیں کے
\(\langle H \rangle =\langle p \rangle ^{2}/2m+\langle H \rangle _{0}\)
ہوگا جہاں زیر نوہشت میں 0 ساکن گاسی کو ظاہر کرتا ہے اور اپنے نتیجے پہ تبصرہ کریں۔
\انتہا{سوال}


