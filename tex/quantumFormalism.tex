\باب{قواعد و ضوابط}\شناخت{باب_قواعد_و_ضوابط}

%sec 3.3 is complete and has been edited.
\حصہ{ہرمشی عامل کے امتیازی تفاعل}
یوں ہم  ہرمشی عاملین کے امتیازی تفاعل کی طرف متوجہ ہوتے ہیں (جو    طبی طور پر  قابل مشاہدہ  کے تعیین حالات ہوں گے)۔  ان کے دو اقسام ہیں:   اگر  طیف \اصطلاح{غیر مسلسل}\فرہنگ{غیر مسلسل}\حاشیہب{discrete}\فرہنگ{discrete} ہو ( یعنی امتیازی اقدار الگ الگ ہوں)   تب امتیازی تفاعلات ہلبرٹ  فضا  میں پائے جائیں گے   اور یہ طبی طور پر قابل حصول حالات  ہوں گے۔  اگر طیف \اصطلاح{استمراری}\فرہنگ{استمراری}\حاشیہب{continuous}\فرہنگ{continuous} ہو  (  یعنی امتیازی اقدار ایک  پوری سعت  کو بھرتے ہوں)  تب امتیازی تفاعلات معمول پر لانے کے قابل نہیں ہوں گے   اور یہ کسی بھی ممکنہ تفاعل موج کو ظاہر نہیں کر سکتے ہیں (اگرچہ   ان کے  خطی جوڑ،  جن میں  لازماً  امتیازی اقدار کی ایک وسعت موجود ہو گی،  معمول پر لانے کے قابل ہو سکتے ہیں)۔  کچھ  عاملین کا صرف غیر مسلسل طیف ہو گا (  مثلاً   ہارمونی مرتعش کی ہیملٹنی)،   کچھ  کا صرف استمراری  طیف  ہو گا (مثلاً  آزاد ذرہ کی ہیملٹنی)،   اور کچھ  کا  ایک  حصہ غیر مسلسل اور دوسرا  حصہ استمراری ہو گا (مثلاً متناہی  چکور کنواں کی  ہیملٹنی)۔  ان میں   غیر مسلسل صورت نبھانا زیادہ آسان ہے چونکہ ان کے متعلقہ اندرونی ضرب لازماً موجود ہوں گے؛   در حقیقت یہ متناہی ابعادی نظریہ سے بہت مشابہت رکھتا ہے (  ہرمشی \ترچھا{قالب} کے امتیازی سمتیات)۔  میں پہلے غیر مسلسل صورت کو  اور اس کے بعد  استمراری  صورت کو دیکھوں گا۔  

\جزوحصہ{غیر مسلسل طیف}
ریاضیاتی طور پر ہرمشی عامل کے معمول پر لانے کے قابل امتیازی تفاعل کی دو اہم خصوصیات پائے جاتے  ہیں:
  
\ابتدا{مسئلہ}\شناخت{مسئلہ_قواعد_امتیازی_اقدار_حقیقی}
ان کے امتیازی اقدار حقیقی ہوں گے۔  
\انتہا{مسئلہ}
\ابتدا{ثبوت} 
 فرض کریں
\begin{align*}
\hat{Q}f = qf 
\end{align*}
ہو (یعنی   \عددی{ \hat{Q}} کا امتیازی تفاعل  \عددی{f}   اور  امتیازی قدر \عددی{  q} ہو)       اور \حاشیہد{یہ وہ  موقع  ہے جہاں  ہم فرض کرتے ہیں کہ امتیازی تفاعلات ہلبرٹ فضا میں پائے جاتے ہیں۔دیگر  صورت   اندرونی ضرب غیر موجود ہو سکتا ہے۔}
\begin{align*}
\langle f | \hat{Q}  f \rangle = \langle \hat{Q} f | f  \rangle
\end{align*}
ہو (\عددی{  \hat{Q}} ہرمشی ہے)۔  تب درج ذیل ہو گا۔
\begin{align*}
q\langle f \left\vert f \right.  \rangle = q^{*} \langle f \left\vert f \right. \rangle
\end{align*}
(چونکہ \عددی{ q} ایک عدد ہے  لہٰذا  اس کو  تکمل سے باہر نکالا   جا سکتا ہے،   اور چونکہ اندرونی ضرب میں پہلا تفاعل مخلوط جوڑی دار ہے ( مساوات \حوالہء{3.6} )   لہٰذا  دائیں طرف \عددی{ q}  بھی جوڑی دار ہو گا)۔ تاہم  \عددی{   \langle f | f \rangle }  صفر نہیں ہو سکتا ہے (  قوانین کے تحت \عددی{   f(x)=0 } امتیازی تفاعل نہیں ہو سکتا ہے) لہٰذا  \عددی{  q=q^{*}}  یعنی \عددی{q}  حقیقی ہو گا۔
\انتہا{ثبوت}

  یہ باعث اطمینان ہے:   تعیین حال میں ایک ذرہ کی  قابل مشاہدہ  کی پیمائش ایک حقیقی عدد دے گی۔ 
   
\ابتدا{مسئلہ}\شناخت{مسئلہ_قواعد_انفرادی_اقدار_عمودی_تفاعلات}
انفرادی امتیازی اقدار کے متعلقہ امتیازی تفاعلات \ترچھا{عمودی} ہوں گے۔  
\انتہا{مسئلہ}
\ابتدا{ثبوت} 
درج ذیل کے ساتھ ساتھ  فرض کریں    \عددی{  \hat{Q}} ہرمشی ہے۔
\begin{align*}
\hat{Q}f = qf  \quad \text{اور }\quad  \hat{Q}g=q'g
\end{align*}
  تب \عددی{   \langle f | \hat{Q}g \rangle = \langle \hat{Q}f | g \rangle} ہو گا  لہٰذا  درج ذیل ہو گا۔
\begin{align*}
q' \langle f | g \rangle = q^{*} \langle f | g \rangle
\end{align*}
(یہاں بھی چونکہ ہم نے فرض کیا ہے کہ امتیازی تفاعلات   ہلبرٹ فضا  میں پائے جاتے ہیں لہٰذا  ان کے  اندرونی ضرب موجود ہوں  گے۔)   اب (مسئلہ \حوالہ{مسئلہ_قواعد_امتیازی_اقدار_حقیقی} کے تحت)  \عددی{   q} حقیقی ہے،    لہٰذا   \عددی{   q{'} \neq q} کی صورت میں   \عددی{\langle f|g \rangle = 0 } ہو گا۔  
\انتہا{ثبوت}

یہی وجہ ہے کہ لامتناہی چکور کنواں یا مثال کے طور پر ہارمونی مرتعش کے امتیازی حالات عمودی ہیں؛   یہ منفرد امتیازی اقدار والے  ہیملٹنی کے امتیازی تفاعلات   ہیں۔  تاہم  یہ خاصیت صرف انہیں    یا  ہیملٹنی کے لئے  مخصوص  نہیں    بلکہ کسی بھی قابل مشاہدہ  کے تعیین  حالات کی بھی ہو گی۔

  بدقسمتی سے مسئلہ  \حوالہ{مسئلہ_قواعد_انفرادی_اقدار_عمودی_تفاعلات}  ہمیں انحطاطی حالات \عددی{(  q'=q)} کے بارے میں کوئی معلومات فراہم نہیں کرتا۔  تاہم،  اگر دو (یا دو سے زیادہ)  امتیازی حالات ایک ہی (ایک دوسرے جیسا)  امتیازی قدر رکھتے ہوں،  تب ان کا  ہر خطی جوڑ  بھی اسی امتیازی قدر والا امتیازی حال ہو گا  (سوال \حوالہء{ 3.7a}  )  اور ہم \اصطلاح{گرام شمد ترکیب  عمودیت}\فرہنگ{گرام شمد!ترکیب عمودیت}\حاشیہب{Gram-Schmidt orthogonalization process}\فرہنگ{Gram-Schmidt!orthogonalization process} (سوال \حوالہء{ A4})  استعمال کرتے ہوئے ہر ایک انحطاطی ذیلی فضا  میں عمودی امتیازی تفاعلات  تشکیل دے  سکتے ہیں۔  اصولی طور پر  ایسا کرنا ہر صورت  ممکن ہو گا ،  تاہم (شکر اللہ کا)  ہمیں عموماً    ایسا کرنے کی ضرورت پیش نہیں آئے گی۔  یوں انحطاط کی صورت  میں بھی ہم عمودی امتیازی تفاعلات  منتخب کر سکتے ہیں،   اور کوانٹم میکانیات کے ضوابط  طے کرتے ہوئے ہم فرض کریں گے  کہ ہم ایسا کر چکے ہیں۔  یوں ہم فوریئر کی ترکیب استعمال کر سکتے ہیں   جو   اساس تفاعلات   کی معیاری عمودیت پر مبنی ہے۔
  
     متناہی بعدی  سمتی  فضا  میں ہرمشی قالب کے امتیازی سمتیات   تیسری بنیادی خاصیت بھی رکھتے ہیں۔  یہ  فضا  کو احاطہ کرتے ہیں (  یعنی ہر سمتیہ کو ان کا خطی جوڑ لکھا جا سکتا ہے)۔  بد قسمتی سے اس کے  ثبوت کو  لا متناہی بعدی   فضاوں   تک وسعت  نہیں دی جا سکتی  ہے۔  تاہم یہ خاصیت کوانٹم میکانیات کی اندرونی ہم آہنگی کیلئے لازم ہے   لہٰذا(   ڈیراک کی طرح)  ہم اسے ایک \ترچھا{مسلمہ}   (بلکہ  قابل مشاہدہ کو ظاہر کرنے والے ہرمشی عاملین پر اس  کو مسلط شرط ) لیتے ہیں۔
     
\موٹا{  مسلمہ:} قابل مشاہدہ  کے امتیازی تفاعلات \ترچھا{مکمل}  ہوں گے:  ( ہلبرٹ  فضا  میں) ہر تفاعل کو ان کا خطی جوڑ لکھا جا سکتا ہے۔ \حاشیہد{چند مخصوص صورتوں میں مکملیت  کو ثابت کیا جا سکتا ہے (مثلاً ہم جانتے ہیں کہ  مسئلہ ڈرشلے کے تحت، لامتناہی چکور کنواں کے ساکن حالات مکمل ہیں)۔ چند صورتوں میں قابل ثبوت پہلو کو مسلمہ کہنا درست نظر نہیں آتا  لیکن مجھے اس سے بہتر اصطلاح  نہیں ملی۔}
%=======================

% Problem 3.7
\ابتدا{سوال}
\begin{enumerate}[a.]
\item
  فرض کریں کہ عامل \عددی{ \hat{Q}} کے دو امتیازی تفاعلات  \عددی{ f(x) } اور  \عددی{ g(x) } ہیں اور ان دونوں کا امتیازی قدر \عددی{ q} ہے۔  دکھائیں کہ \عددی{ f } اور \عددی{ g } کا ہر خطی جوڑ ازخود \عددی{ \hat{Q} } کا امتیازی تفاعل ہو گا   اور اس  کا  امتیازی قدر \عددی{q} ہو گا۔
\item
 تصدیق کریں کہ \عددی{f(x) = e^x} اور \عددی{ g(x) = e^{-x} }  عامل \عددی{\dif^{\,2}/\dif x^2} کے امتیازی تفاعل ہیں اور ان کا امتیازی اقدار ایک دوسرے جیسا  ہے۔تفاعل   \عددی{  f} اور \عددی{ g} کے ایسے دو خطی جوڑ تشکیل  دیں جو وقفہ \عددی{ (-1,1) } پر \ترچھا{عمودی}  امتیازی تفاعلات ہوں۔
\end{enumerate}
\انتہا{سوال}
\ابتدا{سوال}
\begin{enumerate}[a.]
\item
 تصدیق کریں کہ مثال \حوالہء{ 3.1} میں ہرمشی عامل کے امتیازی اقدار حقیقی ہیں۔  دکھائیں کہ (منفرد امتیازی اقدار کے)  امتیازی تفاعلات عمودی ہیں۔ 
\item
 یہی کچھ سوال \حوالہء{ 3.6} کے عامل کے لیے کریں ۔
\end{enumerate}
\انتہا{سوال}


\جزوحصہ{استمراری طیف}
ہرمشی عامل کا طیف   \ترچھا{ استمراری}  ہونے کی صورت میں عین ممکن ہے کہ ان کے  اندرونی ضرب  غیر موجود   ہوں،  لہٰذا مسئلہ  \حوالہ{مسئلہ_قواعد_امتیازی_اقدار_حقیقی}  اور مسئلہ \حوالہ{مسئلہ_قواعد_انفرادی_اقدار_عمودی_تفاعلات}   کے ثبوت کارآمد نہیں ہوں گے  اور  امتیازی تفاعلات  معمول پر لانے کے قابل نہیں ہوں گے۔  اس کے باوجود ایک لحاظ سے  تین لازم خصوصیات (حقیقی ہونا، عمودیت اور مکملیت   ) اب بھی کارآمد ہوں گے۔    اس پراسرار    صورت کو ایک مخصوص مثال کی مدد سے سمجھنا بہتر ہو گا۔



%Example 3.2
\ابتدا{مثال}\شناخت{مثال_قواعد_معیار_حرکت_عامل_تفاعلات}
معیار حرکت عامل کے امتیازی تفاعلات  اور امتیازی اقدار تلاش کریں۔

\موٹا{ حل:}\quad
فرض کریں کہ \عددی{p } امتیازی قدر اور \عددی{f_{p}(x) } امتیازی تفاعل ہے۔
\begin{align}
\frac{\hslash}{i} \frac{\dif}{\dif x} f_{p}(x) = pf_{p}(x)
\end{align}
اس کا عمومی حل درج ذیل ہو گا۔
\begin{align*}
f_{p}(x) = Ae^{ipx/\hslash}
\end{align*}
چونکہ \عددی{ p} کی کسی بھی( مخلوط)  قیمت کے لیے یہ قابل تکامل مربع نہیں ہے؛   معیار حرکت عامل کے ہلبرٹ فضا میں کوئی امتیازی تفاعلات نہیں  پائے جاتے ہیں۔  اس کے باوجود،  اگر ہم حقیقی امتیازی اقدار تک اپنے آپ کو محدود رکھیں،  ہمیں متبادل "معیاری عمودیت" حاصل ہوتی ہے۔   سوال  \حوالہ{سوال_شروڈنگر_تفاعلات_برابر}-الف  اور  \حوالہ{سوال_شروڈنگر_ڈیلٹا_فوریئر_تبادل_کیا_ہے} کو دیکھ کر درج ذیل ہو گا۔
\begin{align}
\int_{-\infty}^{\infty} f_{p'}^{*}(x)f_{p}(x) \dif x = |A|^{2}\int_{-\infty}^{\infty}e^{i(p-p')x/\hslash} \dif x = |A|^{2}2\pi\hslash\delta(p-p')
\end{align}
اگر ہم \عددی{ A=1/\sqrt{2\pi\hslash}} لیں تب 
\begin{align}\label{مساوات_قواعد_امتیازی_تفاعل_معیار_حرکت}
f_{p}(x) = \frac{1}{\sqrt{2\pi\hslash}}e^{ipx/\hslash}
\end{align}
  لہٰذا 
\begin{align}\label{مساوات_قواعد_ڈیراک_معیاری_عمودیت}
\langle f_{p'} | f_{p} \rangle = \delta(p-p')
\end{align}
ہو گا جو حقیقی معیاری عمودیت (مساوات \حوالہء{ 3.10} )   یاد دلاتی  ہے؛   یہاں اشاریہ استمراری متغیرات ہیں،  اور کرونیکر ڈیلٹا کی جگہ ڈیراک ڈیلٹا پایا جاتا ہے؛   تاہم  ان  کے علاوہ یہ ایک دوسرے جیسے نظر آتے ہیں۔  میں مساوات \حوالہ{مساوات_قواعد_ڈیراک_معیاری_عمودیت} کو \اصطلاح{ڈیراک معیاری عمودیت}\فرہنگ{ڈیراک!معیاری عمودیت}\حاشیہب{Dirac orthonormality}\فرہنگ{Dirac!orthonormality} کہوں گا۔  

سب سے اہم بات یہ ہے کہ یہ امتیازی تفاعلات مکمل ہیں اور ان کے مجموعہ ( مساوات \حوالہء{ 3.11}) کی جگہ اب تکمل استعمال ہوتا ہے:  کسی بھی (قابل تکامل مربع ) تفاعل \عددی{f(x) } کو درج ذیل روپ میں لکھا جا سکتا ہے۔
\begin{align}\label{مساوات_قواعد_تفاعل_بطور_فوریئر_تکمل}
f(x) = \int_{-\infty}^{\infty} c(p)f_{p}(x)\dif p = \frac{1}{\sqrt{2\pi\hslash}}\int_{-\infty}^{\infty}c(p)e^{ipx/\hslash}\dif p
\end{align}
پھیلاو  عددی سر (جو اب  تفاعل  \عددی{ c(p)} ہو گا)  کو فوریئر  ترکیب سے حاصل کیا جا سکتا ہے۔
\begin{align}
\langle f_{p'} | f \rangle = \int_{-\infty}^{\infty} c(p) \langle f_{p'} | f \rangle \dif p = \int_{\infty}^{\infty}c(p)\delta (p-p') \dif p = c(p')
\end{align}
چونکہ یہ پھیلاو (مساوات \حوالہ{مساوات_قواعد_تفاعل_بطور_فوریئر_تکمل})   درحقیقت ایک  فوریئر تبادل ہے لہٰذا   انہیں مسئلہ پلانشرال ( مساوات \حوالہ{مساوات_شروڈنگر_مسئلہ_پلانشرل})  سے بھی حاصل کیا جا سکتا ہے۔ 
\انتہا{مثال}

معیار حرکت کے امتیازی تفاعلات ( مساوات \حوالہ{مساوات_قواعد_امتیازی_تفاعل_معیار_حرکت})  سائن  نما ہیں جن کی طول موج درج ذیل ہے۔
\begin{align}
\lambda = \frac{2\pi\hslash}{p} 
\end{align}
یہ وہ ڈی بروگ لی کلیہ (مساوات \حوالہ{مساوات_تفاعل_موج_ڈی_بروگلی_معیار_حرکت}) ہے جس کا ثبوت موزوں وقت پر پیش کرنے کا  وعدہ میں نے کیا تھا۔  یہ کلیہ  ڈی بروگ لی کے تصور  سے   زیادہ پراسرار ہے،  چونکہ ہم اب  جانتے ہیں کہ  حقیقت میں ایسا کوئی ذرہ نہیں پایا جاتا جس کا معیار حرکت  تعیین ہو۔  ہاں ہم تنگ سعت  کی معیار حرکت کا ایسا موجی اکٹھ  تشکیل دے  سکتے ہیں جو معمول پر لانے کے قابل ہو اور جس پر  ڈی بروگ لی کا تعلق  لاگو ہو گا۔

ہم مثال  \حوالہ{مثال_قواعد_معیار_حرکت_عامل_تفاعلات} سے کیا مطلب لیں؟ اگرچہ \عددی{\hat{p} } کا کوئی بھی امتیازی تفاعل ہلبرٹ فضا میں نہیں رہتا،    ان کا ایک مخصوص کنبہ  (جن کے امتیازی اقدار حقیقی ہوں گے) قریبی "مضافات"  میں  رہتے ہیں اور یہ بظاہر معمول پر لانے کے قابل ہیں۔  یہ طبعی طور پر ممکنہ حالات کو ظاہر نہیں کرتے  لیکن اس کے باوجود کارآمد ثابت ہوتے ہیں (جیسا یک بعدی بکھراو  پر غور کے دوران ہم نے دیکھا)۔\حاشیہد{غیر حقیقی امتیازی اقدار والے امتیازی تفاعلات کے بارے میں کیا کہا جا سکتا ہے؟ یہ نا صرف  معمول پر لانے کے قابل نہیں بلکہ \عددی{\pm \infty} پر بے قابو بڑھتے ہیں۔ اس خطہ میں،  جس کو میں "مضافات" کہہ چکا ہوں،  اگرچہ تفاعلات کا  اپنا  (متناہی) اندرونی ضرب  نہیں پایا جاتا، تاہم یہ ہلبرٹ فضا میں تمام ارکان کے ساتھ اندرونی ضرب دیتے ہیں۔ ایسا  \عددی{\hat{p}}  کے ان امتیازی تفاعلات کے لئے درست نہیں ہو گا جن کے امتیازی اقدار غیر حقیقی ہوں۔ بالخصوص، میں دکھا جھکا ہوں کہ  ہلبرٹ فضا میں  تفاعلات کے لئے معیار حرکت عامل ہرمشی ہو گا، اگرچہ اس کا  دلیل پیش کرتے ہوئے    (مساوات \حوالہء{3.9} میں)   سرحدی جزو کو رد کیا گیا۔(جب تک \عددی{f} ہلبرٹ فضا میں پایا جاتا ہو) یہ رکن   تب بھی صفر ہو گا جب \عددی{\hat{p}} کا امتیازی تفاعل \عددی{g} ہو  جس کا امتیازی قدر حقیقی ہو، تاہم  امتیازی قدر کا  خیالی  حصہ ہونے کی صورت میں ایسا نہیں ہو گا۔ اس نقطہ نظر سے ہر مخلوط عدد،  عامل \عددی{\hat{p}} کا  امتیازی قدر ہو گا، تاہم صرف حقیقی اعداد ہرمشی عامل \عددی{\hat{p}} کے امتیازی اقدار ہوں گے؛ باقی اعداد اس خطہ سے باہر  پائے جائیں گے جس میں  \عددی{\hat{p}}  ہرمشی ہو۔}


\ابتدا{مثال}
عامل مقام کے امتیازی اقدار اور امتیازی تفاعلات تلاش کریں۔

\موٹا{حل:}\quad
فرض کریں کہ \عددی{y} امتیازی قدر اور \عددی{ g_{y}(x)} امتیازی تفاعل ہے۔
\begin{align}
xg_{y}(x) = yg_{y}(x) 
\end{align}
یہاں (کسی بھی ایک امتیازی تفاعل کے لیے)  \عددی{ y} ایک مقررہ عدد ،  جبکہ \عددی{x} استمراری متغیر ہے۔متغیر  \عددی{ x} کا  ایسا کون سا   تفاعل ہو گا جس کی  خاصیت یہ ہو  کہ اسے  \عددی{ x} سے ضرب دینا،  اس کو  \عددی{ y} سے ضرب دینے کے مترادف  ہو؟ ظاہر ہے کہ   ماسوائے نقطہ  \عددی{ x=y} کے    ایسی خاصیت والا تفاعل  صفر ہی ہو گا؛  درحقیقت یہ ڈیراک ڈیلٹا تفاعل ہو گا۔
\begin{align*}
g_{y}(x) = A\delta(x-y)
\end{align*}
اس  مرتبہ امتیازی   قدر کو    لازماً حقیقی   ہونا ہو  گا؛  امتیازی تفاعلات قابل تکامل مربع نہیں ہیں، تاہم  اب بھی یہ ڈیراک معیاری  عمودیت  پر پورا اترتے ہیں۔
\begin{align}
\int_{-\infty}^{\infty}g_{y'}^{*}g_{y}(x) \dif x = |A|^{2}\int_{-\infty}^{\infty}\delta(x-y')\delta(x-y) \dif x = |A|^{2} \delta (y-y')
\end{align}
اگر ہم \عددی{A=1} لیں تا کہ
\begin{align}
g_{y}(x) = \delta (x-y)
\end{align}
ہو تب درج ذیل ہو گا۔
\begin{align}
\langle g_{y'} | g_{y} \rangle = \delta (y-y')
\end{align}
یہ  امتیازی  تفاعلات بھی مکمل ہیں:
\begin{align}
f(x) = \int_{-\infty}^{\infty} c(y)g_{y}(x) \dif y = \int_{-\infty}^{\infty} c(y)\delta(x-y)\dif y,
\end{align}
جہاں درج ذیل ہو گا
\begin{align}
c(y) = f(y)
\end{align}
(جس کا حصول اس مثال میں  نہایت آسان   تھا، تاہم  آپ اس کو ترکیب  فوریئر  سے بھی حاصل کر سکتے ہیں)۔
\انتہا{مثال}

اگر ایک ہرمشی عامل کا طیف استمراری ہو  (لہٰذا  اس کے امتیازی اقدار کو استمراری متغیر   \عددی{ p} یا   یہاں پیش  مثالوں   میں  \عددی{y }، اور  بعد ازاں  عموماً  \عددی{z} سے نام  دیا جائے)،    امتیازی تفاعلات معمول پر لانے کے قابل نہیں ہوں گے، یہ ہلبرٹ فضا میں نہیں پائے جاتے  اور یہ  کسی بھی ممکنہ  طبعی حالات  کو ظاہر نہیں کرتے ہیں؛   ہاں حقیقی امتیازی اقدار والے  امتیازی تفاعلات   ڈیراک معیاری  عمودیت  پر پورا اترتے    اور مکمل ہوں گے (جہاں مجموعہ کی جگہ اب تکمل  ہو گا)۔  خوش قسمتی سے ہمیں صرف اتنا ہی چاہیے تھا۔
% Question 3.9
\ابتدا{سوال}
\begin{enumerate}[a.]
\item
  باب \حوالہ{باب_غیر_تابع_وقت_شروڈنگر_مساوات} سے   (ہارمونی مرتعش کے علاوہ)  ایک ایسے ہیملٹنی کی نشاندہی کریں  جس کا  طیف صرف \ترچھا{غیر مسلسل}  ہو۔ 
\item
باب \حوالہ{باب_غیر_تابع_وقت_شروڈنگر_مساوات}  سے ( آزاد ذرہ کے علاوہ ) ایک  ایسے ہیملٹنی  کی نشاندہی کریں  جس کا طیف صرف \ترچھا{استمراری} ہو۔ 
\item
 باب \حوالہ{باب_غیر_تابع_وقت_شروڈنگر_مساوات}    سے ( متناہی چکور کنواں کے علاوہ) ایک ایسے  ہیملٹنی  کی نشاندہی کریں جس کے طیف کا کچھ حصہ غیر مسلسل اور کچھ استمراری ہو۔ 
\end{enumerate}
\انتہا{سوال}
%Question 3.10
\ابتدا{سوال}
کیا لا متناہی چکور کنواں کا زمینی حال معیار حرکت کا  امتیازی تفاعل ہے؟ اگر ایسا ہے تب  اس کا معیار حرکت کیا ہو گا؟   اگر ایسا نہیں ہے تب  ایسا  کیوں نہیں ہے؟
\انتہا{سوال}

%the above is sec 3.3 it is complete and has been edited.


%%%%% the part here is missing


%%%%%KKKKKKK what follows is prob 3.32 on page 138 of the book
%126-128
% example 3.32
\ابتدا{سوال} 
توانائی  و  وقت کی عدم یقینیت کے اصول کا ایک دلچسپ روپ \عددی{\Delta t=\tau/\pi} ہے جہاں ابتدائی حال  \عددی{\Psi (x,0)} کے عمودی حال تک \عددی{\Psi(x,t)} کی  ارتقا کے لیے درکار وقت \عددی{\tau} ہے۔ دو (معیاری عمودی) ساکن حالات کے   برابر حصوں پر مشتمل ( اختیاری)  مخفیہ  کا تفاعل موج    \عددی{\Psi(x,0)=1/\sqrt{2}[\psi_{1}(x)+\psi_{2}(x)]} استعمال کرتے ہوئے اس کی چانچ    پڑتال کریں۔
\انتہا{سوال}

\ابتدا{سوال} 
%3.33\\
ہارمونی مرتعش کے  ساکن  حالات کی (معیاری عمودی)  اساس  (مساوات \حوالہ{مساوات_شروڈنگر_ہارمونی_ساکن_حالات}) میں قالبی ارکان   \عددی{\langle n|x|n'\rangle}  اور  \عددی{\langle n|p|n'\rangle} تلاش کریں۔ آپ سوال \حوالہ{سوال_شروڈنگر_تصدیق_کریں} میں  قالبی وتری رکن  \عددی{n=n'} دریافت کر چکے ہیں؛  وہی ترکیب موجودہ عمومی مسئلے میں استعمال کریں۔  متعلقہ(  لامتناہی) قالب   \عددی{\bold{X}}  اور \عددی{\bold{P}}  تشکیل دیں۔ دکھائیں کہ اس اساس میں
\عددی{\tfrac{1}{2m}\bold{P}^2+\tfrac{m\omega^2}{2}\bold{X}^2=\bold{H}}\ترچھا{  وتری} ہو گا۔ کیا اس کے وتر ی ارکان آپ کے توقع کے مطابق ہیں؟
\ترچھا{جزوی جواب:} 
\begin{align}
\langle n|x|n'\rangle=\sqrt{\frac{\hslash}{2m\omega}}(\sqrt{n'}\delta_{n,n'-1}+\sqrt{n}\delta_{n',n-1})
\end{align}
\انتہا{سوال}

\ابتدا{سوال}
ایک ہارمونی مرتعش ایسے حال میں ہے کہ اس کی توانائی کی  پیمائش،  ایک دوسرے جتنے  احتمال کے ساتھ،  \عددی{(1/2)\hslash\omega} یا \عددی{(3/2)\hslash\omega} دے گی۔ اس حال میں  \عددی{\langle p\rangle}  کی زیادہ سے زیادہ ممکنہ قیمت کیا  ہو گی؟  اگر لمحہ  \عددی{t=0} پر اس کی     قیمت (یہی زیادہ سے زیادہ  قیمت)  ہو تب  \عددی{\Psi(x,t)} کیا ہو گا؟
\انتہا{سوال}

%%%%%KKKKK AM HERE

\ابتدا{سوال} 
3۔35\\
\موٹا{ہارمونی مرتعش کے اتساقی  حالات۔ } \quad
ہارمونی مرتعش کے ساکن حالات 
\عددی{|n\rangle =\psi_{n}(x)}
مساوات 2.67 میں صرف
\عددی{n=0}
عین عدم یقینیت کی حد
\عددی{\sigma_{x}\sigma_{p}=\hslash/2}
پر بیٹھتا ہے جیسا آپ سوال 3.12 میں معلوم کر چکے ہیں عمومی طور پر
\عددی{\sigma_{x}\sigma_{p}=(2n+1)\hslash/2}
ہوگا۔ تاہم چند خطی جوڑ جنہیں منطقی حالات کہتے ہیں بھی عدم یقینیت کے حاصل ضرب تو کم سے کم کرتے ہیں جیسا ہم دیکھتے ہیں یہ عامل س تکیل کے امتیازی تفال ہوتے ہیں۔\\
\begin{align}a_{-}|\alpha\rangle =\alpha|\alpha\rangle\end{align}
جہاں امتیاز ی 
\عددی{\alpha}
کوئی بھی مخلوط عود ہوسکتا ہے۔\\
جز الف\\
حال 
\عددی{|\alpha\rangle}
میں
\عددی{\langle x \rangle}
,
\عددی{\langle x^{2} \rangle}
,
\عددی{\langle p \rangle}
اور
\عددی{\langle p^{2} \rangle}
دریافت کریں۔ مثال 2.5 کی ترقیب استعمال کیں۔ اور یاد رکھیں کہ
\عددی{a_{-}}
منفی کا پرمش جوڑی دار
\عددی{a_{+}}
ہے ساتھ ہی یہ فرض نہ کریں کہ
\عددی{\alpha}
حقیقی ہے۔\\
جز ( ب )\\
\عددی{\sigma_{x}}
اور
\عددی{\sigma_{p}}
تلاش کریں۔ دکھائیں کہ
\عددی{\sigma{x}\sigma{p}=\hslash/2}
ہوگا۔\\
جز ( ج )\\
کسی بھی دوسرے تفال موج کی طرح منطقی حال کو توانائی امتیازی حالات کی صورت میں پھیلایا جا سکتا ہے۔\\
\begin{align}|\alpha\rangle=\sum_{n=0}^{\infty}C_{n}|n\rangle\end{align}
رکھا ئیں کہ پھیلاو کے عددی سر درج ذیل ہونگے۔\\
\begin{align}c_{n}=\frac{\alpha^{n}}{\sqrt{n!}}c_{0}\end{align}
جر ( د )\\
\عددی{|\alpha\rangle}
کو معمول پر لاتے ہوئے
\عددی{c_{0}}
تعین کریں۔
جواب
\عددی{\text{exp}(-\abs{\alpha}^{2}/2)}\\
جز ( ح)\\
اس کے ساتھ وقت کی تابعیت\\
\begin{align}|n\rangle\rightarrow e^{-iE_{n}t/\hslash}|n\rangle\end{align}
منسلک کرکے دکھائیں کہ
\عددی{|\alpha(t)\rangle}
اب بھی
\عددی{a{-}}
کا امتیازی حال ہوگا لیکن وقت کے ساتھ امتیازی قیمت ارتقا پزیر ہوگا\\
\begin{align}\alpha(t)=e^{-i\omega t}\alpha\end{align}
یوں منظقی حال ہمیشہ منطقی حال بی رہے گا اور عدم یقینیت کے حاصل ضرب کو تم سے کم برقرار رکھا ہے۔\\
جز ( و ) کیا زمینی حال
\عددی{|n=0\rangle}
از خود منطقی حال ہے اگر ایسا ہو تب امتیازی قدر کیا ہوگا 
\انتہا{سوال}
%128-130
\ابتدا{سوال} 
3.36\\
مقصود عدم یقینیت کا اصول:\\
عمومی عدم یقینيت کا اصول مساوات  3.62 درجہ ذيل کہتا ہے
\begin{align}\sigma_{A}^{2}\sigma_{B}^{2}\geq\frac{1}{4}\langle C^{2} \rangle\end{align}
جہاں
\begin{align}\hat{C}\equiv i[\hat{A},\hat{B}]\end{align}
ہے\\ 
جذ الف\\
دکھائے کہ اس کو زیادہ مستحكم کرتے ہوئے درجہ ذیل روٹ میں لکھا جاسکتا ہے
\begin{align}\sigma_{A}^{2}\sigma_{B}^{2}\geq\frac{1}{4}(\langle C\rangle^{2}+\langle D \rangle ^{2})\end{align}
جہاں
\عددی{\hat{D}\equiv \hat{AB}+\hat{BA}+2\langle A \rangle \langle B \rangle}
ہوگا\\
اشاره: مساوات 3.60 میں 
\عددی{\text{Re}(z)}
 اجزاه لیں\\
ب)\\
مساوات 3.99 کو
 A=B
  کے لئیے جانچیں چونکہ اس صورت میں
 C=0
   ہے لہٰذا  معیاری عدم یقینیت غیر اہم ہوگا بدقسمتی سے مقصود عدم یقینيت کا اصول بھی زیادہ مددگار ثابت نہیں ہوتا\\
\انتہا{سوال}

\ابتدا{سوال}
 3.37:\\
ایک نظام جو 3 صدی ہے کہ ہیملٹنین کو درجہ ذيل کا الف ظاہر کرتی ہے
\begin{align}\text{\text{H}}=\begin{pmatrix}
a&0&b\\
0&c&0\\
b&0&a\\
\end{pmatrix}\end{align}
جہاں c اور a, b حقیقی اعداد ہیں۔\\
ا) اگر اس نظام کا ابتدائی حال درجہ ذيل ہو
\begin{align}|\delta(0) \rangle=\begin{pmatrix}
0\\
1\\
0\\
\end{pmatrix}\end{align}

تب
\عددی{\delta(t)}
  کیا ہو گا؟\\
ب) اگر اس نظام کا ابتدائی حال درجہ ذيل ہو
\begin{align}| \delta(0)\rangle =\begin{pmatrix}
0\\
0\\
1\\
\end{pmatrix}\end{align}
تب
 \عددی{\delta{t}}
  کیا ہوگا؟\\
\انتہا{سوال}

\ابتدا{سوال}
 3.38:\\
ایک نظام جو 3 صدی ہے کہ ہيملٹونین کو درجہ ذیل کا الف ظاہر کرتی ہے:\\
\begin{align}\text{\text{H}}=\hslash\omega\begin{pmatrix}
1&0&0\\
0&2&0\\
0&0&2\\
\end{pmatrix}\end{align}
دیگر دو قابل مشاہدہ B اور A کو درجہ ذيل کا الف ظاہر کرتی ہے\\
\begin{align}\text{A}=\lambda\begin{pmatrix}
0&1&0\\
1&0&0\\
0&0&2\\
\end{pmatrix} , \text{B}=\mu\begin{pmatrix}
2&0&0\\
0&0&1\\
0&1&0\\
\end{pmatrix}
\end{align}
جہاں
\عددی{\omega, \lambda}
اور
\عددی{\mu}
حقیقی مثبت اعداد ہیں۔\\
H، A
اور
B
 کی امتیازی اقدار اور معمول پر لائے  گئے امتیازی تفاعل تلاش کریں۔ \\
ب) یہ نظام کسی عمومی حال
\begin{align}| \delta(0)  \rangle=\begin{pmatrix}
c_{1}\\
c_{2}\\
c_{3}\\
\end{pmatrix}\end{align}
سے ابتداء کرتا ہے جہاں
\عددی{\abs{c_{1}}^{2}+\abs{c_{2}}^{2}+\abs{c_{3}}^{2}=1}
ہے۔\\
لمحہ t=0 پر B اور ,A Hکی توقعاتی قیمت تلاش كريں۔\\
ج)
 \عددی{\delta(s)} 
کیا ہوگا ؟
لمحہ t پر اس نظام کی توانائی کی پیمائش کیا قيمتيں دے سکتی ہے؟ اور ہر ایک کا انفرادى احتمال کیا ہوگا؟ اسی سوال کے جوابات B اور A کے لیے بھی تلاش كریں۔
\انتہا{سوال}

\ابتدا{سوال}
 3.39:
ا) ایک تفاعل
 \عددی{f(x)}
 جس کو Taylor تسلسل کی صورت میں پھیلایا جاسکتا ہے کے لیے درجہ ذيل  دکھائیں:\\
\begin{align}f(x+x_{0})=e^{i\hat{p}x_{0}/\hslash}f(x)\end{align}
جہاں
\عددی{x_{0}}
کوئی بھی مستقل فاصلہ ہو سکتا ہے۔ اسی وجہ کے بنا 
\عددی{\hat{p}/\hslash}
کو فضا میں  انتقال کا پیداکار کہتے ہیں۔ یہاں دیہان رہے کہ ایک عامل کی قوت نمائی کی قوتی تسلسلی پھیلاؤ کی تعریف درجہ ذیل ہے
\عددی{e^{\hat{Q}}=1+\hat{Q}+(1/2)\hat{Q}^{2}+(1/3!)\hat{Q}^{3}+\dotsc}\\
ب) اگر
\عددی{\Psi(x,t)}
وقت تابع schrodinger مساوات کو مطمئن کرتا ہو تب درجہ ذیل دکھائیں\\
\begin{align}\Psi(x,t+t_{0})=e^{-i\hat{\text{H}}t_{0}/\hslash}\Psi(x,t)\end{align}
جہاں
\عددی{t_{0}}
کوئی بھی مستقل وقت ہوگا۔ اسی وجہ کی بنا
\عددی{-\hat{\text{H}}/\hslash}
وقت میں انتقال کا پیدا کار کہتے ہے۔\\
ج) دکھائیں کے لمحہ
\عددی{t+t_{0}}
پر ہر کی متغير Q کی توقعاتی قیمت کو درجہ ذيل لکھا جا سکتا ہے\\
\begin{align}\langle Q \rangle _{t+t_{0}}=\langle \Psi(x,t)|e^{i\hat{\text{H}}t_{0}/\hslash}\hat{Q}(x,p,t+t_{0}e^{-i\hat{\text{H}}t_{0}/\hslash}|\Psi(x,t)\rangle\end{align}
اس کو استعمال کرتے ہوئے مساوات 3۔71 حاصل كریں۔\\
اشاره:
 \عددی{t_{0}=\dif{t}} 
لیتے ہوئے dt میں پہلے رتبے تک پھیلائیں۔
\انتہا{سوال}

\ابتدا{سوال}
 3۔40:\\
ا) ایک آزاد ذرہ کے لیے وقت تابع schrodinger مساوات کو معیار حرکت فضاه میں لکھ کر حل کریں۔
جواب:  
\عددی{\text{exp}(-ip^{2}t/2m\hslash)\Phi(p,0})\\
ب) متحرک گاسی موجی اکٹ سوال 2۔43 کے لئیے
\عددی{\Phi(p,0)}
تلاش كريں اور
\عددی{\Phi(p,t)}
تیار كریں۔ ساتھ ہی
\عددی{\abs{\Phi(p,t)}^{2}}
تیار كريں۔ اب دیکھیں گے کہ یہ وقت کا تابع نہیں ہوگا۔\\
ج)
\عددی{\Phi}
پر مبنی موضوع تكملات حل کرتے ہوے
\عددی{\langle p \rangle}
اور
\عددی{\langle p^{2} \rangle}
تلاش کرکے سوال 2۔43 کی حاصل كرده جوابات کے ساتھ موازنہ كريں۔\\
د) دکھائیں کے
\عددی{\langle H \rangle =\langle p \rangle ^{2}/2m+\langle H \rangle _{0}}
ہوگا جہاں زیر نوہشت میں 0 ساکن گاسی کو ظاہر کرتا ہے اور اپنے نتیجے پہ تبصرہ کریں۔
\انتہا{سوال}


