\باب{بکھراو}\شناخت{باب_بکھراو}

\حصہ{تعارف}
\جزوحصہ{کلاسیکی نظریہ بکھراو}
فرض کریں کسی مرکز بکھراو پر ایک ذرہ کا آمد ہوتا ہے مثلاً ایک پروٹان کو ایک بھاری مرکزہ پر داغا جاتا ہے یہ توانائی \عددی{E}  اور ٹکراو مقدار معلوم \عددی{b} کے ساتھ آکر کسی زاویائے بکھراو \عددی{\theta} پر اُبھرتا ہے \حوالہء{شکل \num{11.1}} دیکھیں۔ میں اپنی آسانی کے لیئے فرض کرتا ہوں کہ ہدف اسمتی تشاکلی ہے یوں خطِ حرکت ایک مستوی میں پایا جائے گا اور کہ نشانہ بھاری ہے لحاظہ تصداً کی بنا اس کی حرکت اُچھلنے کو نظرانداز کیا جاسکتا ہے۔ کلاسیکی نظریہ بکھراو کا بنیادی مسئلہ یہ ہوگا: ٹکراو مقدار معلوم کو جانتے ہوئے زاویائے بکھراو کا حساب کریں۔ یقیناً عام طور پر ٹکراو مقدار معلوم جتنا چھوٹا ہو زاویہ بکھراو اتنا بڑا ہوگا۔

\ابتدا{مثال}
\موٹا{سخت کرہ کا بکھراو}۔ فرض کریں ہدف رداس \عددی{R} کا ایک ٹھوس بھاری گیند ہے جبکہ آمدی ذرہ ہوائی بندوق کا ایک چھرہ ہے جو لچھکیلی ٹپکی کھا کر مڑتا ہے \حوالہء{شکل \num{11.2}}۔ زاویہ \عددی{\ایلفا} کی صورت میں ٹکراو مقدار معلوم \عددی{b=R\sin\ایلفا} اور زاویہ بکھراو \عددی{\تھیٹا=\پاے-2\ایلفا} ہوں گے۔ یوں درج ذیل ہوگا
\begin{align}
	b = R\sin\left(\frac{\pi}{2}-\frac{\theta}{2}\right) = R\cos\left(\frac{\theta}{2}\right)
\end{align}
ظاہری طور پر درج ذیل ہوگا
\begin{align}
	\theta =
	\begin{cases}
		2\cos^{-1}(b/R), & b\leq R \text{\RL{اگر}} \\
		0, & b\geq R \text{\RL{اگر}}
	\end{cases}
\end{align}
\انتہا{مثال}
عمومی طور پر لامتناہی چھوٹے رقبہ عمودی تراش \عددی{\dif\سگما} میں آمدی ذرات مطابقتی لامتناہی چھوٹے ٹھوس زاویہ \عددی{\dif\بڑااومیگا} میں بکھریں گے \حوالہء{شکل \num{11.3}}۔ بڑی \عددی{\dif\سگما} کی صورت میں \عددی{\dif\بڑااومیگا} بھی بڑا ہوگا تناسبی جز ضربی \عددی{D(\تھیٹا)\equiv\dif\سگما/\dif\بڑااومیگا} کو تفریقی بکھراو عمودی تراش کہتے ہیں 
\begin{align}
	\dif\sigma = D(\theta)\dif\Omega
\end{align}
ٹکراو مقدار معلوم اور اسمتی زاویہ \عددی{\فاے} کی صورت میں \عددی{\dif\سگما=b\dif b\dif\فاے} اور \عددی{\dif\بڑااومیگا=\sin\تھیٹا\dif\تھیٹا\dif\فاے} ہوں گے لحاظہ درج ذیل ہوگا
\begin{align}
	D(\theta) = \frac{b}{\sin\theta}\abs{\frac{\dif b}{\dif\theta}}
\end{align}
چونکہ عمومی طور پر \عددی{\تھیٹا} مقدار معلوم \عددی{b} کا گٹھتا ہوا تفاعل ہوگا لحاظہ یہ تفرق در حقیقت منفی ہوگا اسی لیئے مطلق قیمت لی گئی ہے۔

\ابتدا{مثال}
\موٹا{سخت کرہ کے بکھراو کی مثال جاری رکھتے ہیں}۔ سخت کرہ بکھراو \حوالہء{مثال \num{11.1}} کی صورت میں 
\begin{align}
	\frac{\dif b}{\dif\theta}=-\frac{1}{2}R\sin\left(\frac{\theta}{2}\right)
\end{align}
لحاظہ درج ذیل ہوگا 
\begin{align}
	D(\theta) = \frac{R\cos(\theta/2)}{\sin\theta}\left(\frac{R\sin(\theta/2)}{2}\right) = \frac{R^2}{4}
\end{align}
اس مثال میں تفریقی عمودی تراش \عددی{\تھیٹا} کا تابع نہیں ہے جو ایک غیر معمولی بات ہے۔
\انتہا{مثال}
کل عمودی تراش تمام ٹھوس زاویوں پر \عددی{D(\تھیٹا)} کا تکمل ہوگا
\begin{align}
	\sigma\equiv\int D(\theta)\dif\Omega	
\end{align}
اندازاً بات کرتے ہوئے یہ آمدی شعاع کا وہ رقبہ ہوگا جسے ہدف بکھیرتا ہے۔ مثال کے طور پر سخت کرہ بکھراو کی صورت میں درج ذیل ہوگا
\begin{align}
	\sigma = (R^2/4)\int \dif\Omega = \pi R^2
\end{align}
جو ہمارے توقعات کے عین مطابق ہے۔ یہ کرہ کا رقبہ عمودی تراش ہے۔ اس رقبہ میں آمدی چھرے ہدف کو نشانہ بنائیں گے جبکہ اس سے باہر چھرے ہدف کو خطا کریں گے۔ یہی تصورات نرم اہداف مثلاً مرکزہ کا کولمب میدان کے لیئے بھی کار آمد ہے جن میں صرف نشانے پر لگنا یا نہ لگنا نہیں ہوگا۔

آخر میں فرض کریں ہمارے پاس آمدی ذرات کی یکساں شدت تابندگی کی ایک شعاع ہو 
\begin{align}
	\mathcal{L}\equiv\text{number of incident particles per unit area per unit time}
\end{align}
فی اکائی وقت رقبہ \عددی{\dif\سگما} میں داخل ہونے والے ذرات اور یوں ٹھوس زاویہ \عددی{\dif\بڑااومیگا} میں بکھراو والے ذرات کی تعداد \عددی{\dif N = \mathcal{L}\dif\سگما=\mathcal{L}D(\تھیٹا)\dif\بڑااومیگا}  ہوگی لحاظہ درج ذیل ہوگا 

\begin{align}
	D(\theta) = \frac{1}{\mathcal{L}}\frac{\dif N}{\dif\Omega}
\end{align}
چونکہ یہ صرف ان مقداروں کی بات کرتا ہے جنہیں تجربہ گاہ میں با آسانی ناپا جا سکتا ہو لحاظہ اس کو عموماً تفریقی عمودی تراش کی تعریف لیا جاتا ہے۔ اگر ٹھوس زاویہ \عددی{\dif\بڑااومیگا} میں بکھرے ذرات کو محسوس کار دیکھتا ہو تب ہم اکائی وقت میں معلوم شدہ ذرات کی تعداد کو \عددی{\dif\بڑااومیگا} سے تقسیم کر کے آمدی شعاع کی تابندگی کے لحاظ سے معمول شدہ کرتے ہیں۔

\ابتدا{سوال}
\موٹا{ردرفورڈ بکھراو}۔ بار \عددی{q_1} اور حرکی توانائی \عددی{E} کا ایک آمدی ذرہ ایک بھاری ساکن ذرہ جس کا بار \عددی{q_2} ہو 	سے بکھرتا ہے۔

(الف) ٹکراو مقدار معلوم اور زاویہ بکھراو کے بیچ رشتہ اغز کریں۔

جواب: \عددی{b=(q_1q_2/8\pi\epsilon_0E)\cot(\theta/2)}

(ب) تفریقی بکھراو عمودی تراش تعین کریں۔

جواب:
\begin{align}
	D(\theta)=\left[\frac{q_1q_2}{16\pi\epsilon_0E\sin^2(\theta/2)}\right]^2
\end{align}
(ج) دیکھائیں کہ ردرفورڈ بکھراو کا کل عمودی تراش لامتناہی ہوگا۔ ہم کہتے ہیں \عددی{1/r} مخفیہ لامتناہی ساتھ رکھتا ہے آپ کولمب قوت سے بچ نہیں سکتے ہیں۔
\انتہا{سوال}
%==========================

\جزوحصہ{کوانٹم نظریہ بکھراو}
بکھراو کے کوانٹم نظریہ میں فرض کرتے ہیں کہ ایک آمدی مستوی موج \عددی{\psi(z) = Ae^{ikz}} جو محور \عددی{z} رخ حرکت کرتی ہو کا سامنا ایک بکھراو مخفیہ سے ہوتا ہے جس کے نتیجہ میں ایک کروی رخصتی موج پیدا ہوتی ہے \حوالہء{شکل \num{11.4}} یعنی ہم مساوات شروڈنگر کے وہ حل تلاش کرنا چاہتے ہیں جن کی عمومی روپ درج ذیل ہو
\begin{align}
	\psi(r, \theta)\approx A\left\{e^{ikz}+f(\theta)\frac{e^{ikr}}{r}\right\}, && \text{\RL{کے لیئے}} r \text{\RL{بڑے}}
\end{align}
کروی موج میں جز ضربی \عددی{1/r} پایا جاتا ہے چونکہ احتمال کی بقا کے خاطر \عددی{\abs{\psi}^2} کا یہ حصہ \عددی{1/r^2} کے لحاظ سے تبدیل ہوگا۔ عدد موج \عددی{k} کا آمدی ذرات کی توانائی کے ساتھ ہمیشہ کی طرح درج ذیل رشتہ ہوگا 
\begin{align}
	k\equiv\frac{\sqrt{2mE}}{\hslash}
\end{align}
یہاں بھی میں فرض کرتا ہوں کہ ہدف اسمتی تشاکلی ہے زیادہ عمومی صورت میں رخصتی کروی موج کا حیطہ \عددی{f} متغیرات \عددی{\فاے} اور \عددی{\تھیٹا} کا تابع ہوگا۔

ہمیں حیطہ بکھراو \عددی{f(\تھیٹا)} تعین کران ہوگا۔ یہ ہمیں کسی مخصوص رخ \عددی{\تھیٹا} میں بکھراو کا احتمال دیتا ہے اور یوں اس کا تعلق تفریقی عمودی تراش سے ہوگا۔ یقیناً سمتی رفتار \عددی{v} پر چلتے ہوئے ایک آمدی ذرہ کا وقت \عددی{\dif t} میں لامتناہی چھوٹی رقبہ \عددی{\dif\سگما} میں سے گزرنے کا احتمال \حوالہء{شکل \num{11.5}} دیکھیں درج ذیل ہوگا
\begin{align*}
	\dif P = \abs{\psi_{\text{\RL{آمدی}}}}^2\dif V = \abs{A}^2(v\dif t)\dif\sigma
\end{align*}
لیکن مطابقتی ٹھوس زاویہ \عددی{\dif\Omega} میں اس ذرہ کے بکھاو کا احتمال 
\begin{align*}
	\dif P = \abs{\psi_{\text{\RL{بکھرا}}}}^2\dif V = \frac{\abs{A}^2\abs{f}^2}{r^2}(v\dif t)r^2\dif\Omega
\end{align*}
بھی یہی ہوگا لحاظہ \عددی{\dif\sigma=\abs{f}^2\dif\Omega} اور درج ذیل ہوں گے
\begin{align}
		D(\theta) = \frac{\dif\sigma}{\dif\Omega} = \abs{f(\theta)}^2
\end{align}
ظاہر ہے کہ تفرقی عمودی تراش جس میں تجربہ کرنے والا دلچسمی رکھتا ہے حیطہ بکھراو جو مساوات ژروڈنگر کے حل سے حاصل ہوگا کی مطلق مربع کے برابر ہوگا آنے والے حصوں میں ہم حیطہ بکھراو کی حساب کے دو تراکیب جزوی موج تجزیہ اور بارن تخمین پر غور کریں گے۔

\ابتدا{سوال}
ایک بُعدی اور دو ابعادی بکھراو کے لیئے \حوالہء{مساوات \num{11.12}} کے مماثل تیار کریں۔
\انتہا{سوال} 



%=======================
% section 11.2 to end of chapter. unedited

\حصہ{جزوی موج تجزیہ}
\جزوحصہ{اصول و ضوابط}
ہم نے باب 4 میں دیکھا کہ کروی تشاکلی مخفیہ \عددی{V(r)} کے لیئے مساوات شروڈنگر قابلِ علیحدگی حلوں
\begin{align}
	\psi(r, \theta, \phi) = R(r)Y^m_l(\theta, \phi)
\end{align}
کا حامل ہوگا جہاں \(Y_l^m\) کروی ہارمونی مساوات \num{4.32} ہے اور \(u(r) = rR(r)\) رداسی مساوات مساوات \num{4.37} 
\begin{align}
	-\frac{\hbar^2}{2m}\frac{d^2u}{dr^2}+\left[V(r)+\frac{\hbar^2}{2m}\frac{l(l+1)}{r^2}\right]u = Eu
\end{align}
کو متمعن کرتا ہے بہت بڑی \عددی{r} کی صورت میں مخفیہ صفر کو پہنچتا ہے اور مرکز گریز حصہ قابلِ نظرابداز ہوگا۔ لحآظہ درج ذیل لکھا جاسکتا ہے۔
\begin{align*}
	\frac{d^2u}{dr^2} \approx-k^2u
\end{align*}
اس کا عمومی حل درج ذیل ہے
\begin{align*}
	u(r) = Ce^{ikr}+De^{-ikr}
\end{align*}
پہلا جز رخصتی کروی موج کو اور دوسرا جز آمدی موج کو ظاہر کرتا ہے پھر ہے کہ موج  بکھرائو کے لیئے ہم \(D=0\) چاہتے ہیں۔ یوں بہت بڑی \عددی{r} کی صورت میں درج ذیل ہوگا
\begin{align*}
	R(r)\sim\frac{e^{ikr}}{r}
\end{align*}
جہ ہم گزشتہ حصہ میں طبعی وجوہات سے اغز کر چکے ہیں مساوات  \num{11.12}۔

یہ بہت بڑی \عددی{r} کے لیئے تھا یا یہ کہنا زیادہ درست ہوگا کہ \(kr>>1\) کے لیئے تھا جسی بصریات میں خطہ اشاعی کہیں گے۔ یک بُعدی نظریہ بکھرائو کی طرح ہم یہاں فرض کرتے ہیں کہ مخفیہ مکامی ہے جس سے  ہمارا مراد یہ ہوگا کہ کسی متناہی بکھرائو خطہ کے باہر یہ تقریباً صفر ہوگا شکل \num{11.6}۔ درمیانی خطہ میں جہاں \عددی{V} کو رد کیا جا سکتا ہے لیکن مرکز گریز جز کو نظرانداز نہیں کیا جا سکتا رداسی مساوات درج ذیل روپ اختیار کرتی ہے۔  
\begin{align}
	\frac{d^2u}{dr^2}-\frac{l(l+1)}{r^2}u = -k^2u
\end{align}
جس کا عمومی حل مساوات \num{4.45} کروی بیسل تفاعلات کا خطی جوڑ ہوگا
\begin{align}
	u(r) = Arj_l(kr)+Brn_l(kr)
\end{align}
لیکن نہ ہی \عددی{j_l} جو سائن تفاعل کی طرح ہے اور نہ ہی \عددی{n_l} جو متعمم کوسائن کی طرح ہے کسی رخصتی یا آمدی موج کو ظاہر نہیں کرتے ہیں۔ ہمیں یہاں \(e^{ikr}\) اور \(e^{-ikr}\) طرز کے خطی جوڑ درکار ہوں گے جنہیں کروی ہینکل تفاعلات کہتے ہیں
\begin{align}
	h^{(1)}_l(x)\equiv j_l(x)+in_l(x);\quad h^{(2)}_l(x)\equiv j_l(x)-in_l(x)
\end{align}
جدول \num{11.1} میں چند ابتدائی کروی ہینکل تفاعلات پیش کیئے گئے ہیں۔
\begin{table}[h!]
\centering
\caption{کروی ہینکل تفاعلات $h_l^{(1)}(x)$ اور $h_l^{(2)}(x)$}
\label{table:1}
\begin{tabular}{|c c c|}
\hline
$h_0^{(1)} = -i\frac{e^{ix}}{x}$ & & $h_0^{(2)} = i\frac{e^{-ix}}{x}$ \\
$h_1^{(1)} = \left(-\frac{i}{x^2}-\frac{1}{x}\right)e^{ix}$ & & $h_1^{(2)} = \left(\frac{i}{x^2}-\frac{1}{x}\right)e^{-ix}$ \\
$h_2^{(1)} = \left(-\frac{3i}{x^3}-\frac{3}{x^2}+\frac{i}{x}\right)e^{ix}$ & & $h_2^{(2)} = \left(\frac{3i}{x^3}-\frac{3}{x^2}+\frac{i}{x}\right)e^{-ix}$\\
 & $\begin{matrix}
 	h_l^{(1)}\rightarrow\frac{1}{x}(-i)^{l+1}e^{ix} \\
 	h_2^{(2)}\rightarrow\frac{1}{x}(i)^{l+1}e^{-ix}
 \end{matrix}
	\Bigg\}x>>1\text{\RL{کے لیئے}}$ & \\
\hline
\end{tabular}
\end{table}
بڑی \عددی{r} کی صورت میں \(h_l^{(1)}(kr)\) جسے ہینکل تفاعک کا پہلا قسم کہتے ہیں \(e^{ikr}/r\) کے لحاظ سے تبدیل ہوتا ہے جبکہ \(h_l^{(2)}(kr)\) ہینکل تفاعل کی دوسری قسم \(e^{-ikr}/r\) کے لحاظ سے تبدیل ہوگا۔ یوں رخصتی امواج کے لیئے ہمیں کروی ہینکل تفاعلات کی پہلی قسم درکار ہوگی:
\begin{align}
	R(r)\sim h^{(1)}_l(kr)
\end{align}
اس طرح خطہ بکھرائو کے باہر جہاں \(V(r) = 0\) ہوگا بلکل ٹھیک تفاعل موج درج ذیل ہوگا
\begin{align}
	\psi(r, \theta, \phi) = A\left\{e^{ikz}+\sum_{l, m}C_{l, m}h^{(1)}_l(kr)Y^m_l(\theta, \phi)\right\}
\end{align}
اس کا پہلا جز آمدی مستوی موج ہے جبکہ مجموعہ جس کے عددی سر \عددی{C_{l, m}} ہے موج بکھرائو کو ظاہر کرتا ہے۔ چونکہ ہم فرض کر چکے ہیں کہ مخفیہ کروی تشاکلی ہے لحاظہ تفاعل موج \(\phi\) کا تابع نہیں ہوسکتا ہے۔ یوں صرف وہ اجزاء باقی رہیں گے جن میں \(m=0\) ہو یاد رہے \(Y_l^m\sim e^{im\phi}\) اب مساوات \num{4.27} اور \num{4.32} سے درج ذیل ہوگا
\begin{align}
	Y^0_l(\theta, \phi) = \sqrt{\frac{2l+1}{4\pi}}P_l(\cos\theta)
\end{align}
جہاں \عددی{l} ویں لیژانڈر کثیر رکنی کو \عددی{P_l} کو ظاہر کرتا ہے۔ روایتی طور پر \(C_{l, 0}\equiv i^{l+1}k\sqrt{4\pi(2l+1)}a_l\) لکھ کر عددی سروں کی تعریف یوں کی جاتی ہے:
\begin{align}
	\psi(r, \theta) = A\left\{e^{ikz}+k\sum_{l=0}^{\infty}i^{l+1}(2l+1)a_lh_l^{(1)}(kr)P_l(\cos\theta)\right\}
\end{align}
آپ کچھ ہی دیر میں دیکھیں گے کہ یہ مخصوص علامتیت کیوں بہتر ہے \عددی{a_l} کو \عددی{l} واں حیطہ جزوی موج کہتے ہیں۔

اب بہت بڑی \عددی{r} کی صورت میں ہینکل تفاعل \((-i)^{l+1}e^{ikr}/kr\) جدول \num{11.1} کے لحاظ سے تبدیل ہوگا لحاظہ درج ذیل ہوگا 
\begin{align}
	\psi(r, \theta)\approx A\left\{e^{ikz}+f(\theta)\frac{e^{(ikr)}}{r}\right\}
\end{align}
جہاں \(f(\theta)\) درج ذیل ہے
\begin{align}
	f(\theta) = \sum_{l=0}^{\infty}(2l+1)a_lP_l(\cos\theta)
\end{align}
یہ مساوات \num{11.12} میں میں پیش کی گئی عمومی ساخت کے اصول موضوعہ کی تصدیق کرتا ہے اور ہمیں دیکھاتا ہے کہ جزوی موج حیطوں \عددی{a_l} کی صورت میں حیطہ بکھرائو \(f(\theta)\) کس طرح حاصل ہوگا تفریقی عمودی تراش درج ذیل ہوگا
\begin{align}
	D(\theta) = \abs{f(\theta)}^2 = \sum_{l}\sum_{l^\prime}(2l+1)(2l^\prime+1)a^*_la_{l^\prime}P_l(\cos\theta)P_{l^\prime}(\cos\theta)
\end{align}
اور کل عمودی تراش درج ذیل ہوگا
\begin{align}
	\sigma=4\pi\sum_{l=0}^{\infty}(2l+1)\abs{a_l}^2
\end{align}
زاویائی تکمل کو حل کرنے کے لیئے میں نے لیژانڈر کثیررکنیوں کی عمودیت مساوات \num{4.34} استعمال کی۔
\جزوحصہ{لایاعمل}
زیرِ غور مخفیہ کے لیئے جزوی موج حیطوں \عددی{a_l} کا تعین کرنا باقی ہے۔ اندرونی خطہ جہاں \عددی{V(r)} غیر صفر ہے میں مساوات شروڈنگر کو حل کر کے اسے بیرونی حل مساوات \num{11.23} کے ساتھ مناسب سرحدی شرائط استعمال کرتے ہوئے ملانے سے ایسا کیا جاسکتا ہے۔ مثلا صرف اتنا ہے کہ میں نے بکھرائو موج کے لیئے کروی محدد جبکہ آمدی موج کے لیئے کارتیسی محدد استعمال کیئے ہیں۔ ہمیں تفاعل موج کو ایک جیسی علامتوں میں لکھنا ہوگا۔

یقیناً \(V=0\) کے لیئے مساوات شروڈنگر کو \(e^{ikz}\) متمعن کرتا ہے۔ ساتھ ہی میں دلائل پیشکر چکا ہوں کہ \(V=0\) کے لیئے مساوات شروڈنگر کا عمومی حل درج ذیل روپ کا ہوگا
\begin{align*}
	\sum_{l, m}\left[A_{l, m}j_l(kr)+B_{l, m}n_l(kr)\right]Y_l^m(\theta, \phi)
\end{align*}
یوں بلخصوص \(e^{ikz}\) کو اس طرح بیان کرنا ممکن ہونا چاہیئے اب مبدہ پر \(e^{ikz}\) متناہی ہے لحاظہ نیومن تفاعلات کی اجازت نہیں ہوگی \(r=0\) پر \(n_l(kr)\) بے قابو بڑھتے ہیں اور چونکہ \(z=r\cos\theta\) میں کوئی \(\phi\) نہیں پایا جاتا ہے لحاظہ صرف \(m=0\) اجزاء ہوں گے۔ مستوی موج کی کروی امواج کی صورت میں سریحاً پھیلائو کلیہ ریلے دیتی ہے۔
\begin{align}
	e^{ikz} = \sum_{l=0}^{\infty}i^l(2l+1)j_l(kr)P_l(\cos\theta)
\end{align}
اس کو استعمال کرتے ہوئے بیرونی خطہ میں تفاعل موج کو صرف \عددی{r} اور \(\theta\) کی صورت میں پیش کیا جا سکتا ہے
\begin{align}
	\psi(r, \theta) = A\sum_{l=0}^{\infty}i^l(2l+1)\left[j_l(kr)+ika_lh_l^{(1)}(kr)\right]P_l(\cos\theta)
\end{align}

%============================

\ابتدا{مثال}
کوانٹم سخت کرہ بکھرائو۔ درج ذیل فرض کریں
\begin{align}
	V(r)=
	\begin{cases}
		\infty, & r\leq a \text{\RL{کے لیئے}} \\
		0, & r>a \text{\RL{کے لیئے}}
	\end{cases}
\end{align}
سرحدی شرط تب درج ذیل ہوگا
\begin{align}
	\psi(a, \theta) = 0
\end{align}
یوں تمام \عددی{\theta} کے لیئے
\begin{align}
	\sum_{l=0}^{\infty}i^l(2l+1)\left[j_l(ka)+ika_lh_l^{(1)(ka)}\right]P_l(\cos\theta) = 0
\end{align}
ہوگا۔ جس سے درج ذیل حاصل ہوتا ہے \حوالہء{سوال\num{11.3}} 
\begin{align}
	a_l = i\frac{j_l(ka)}{kh_l^(1)(ka)}
\end{align}
بلخصوص کل عمودی تراش درج ذیل ہوگا
\begin{align}
	\sigma=\frac{4\pi}{k^2}\sum_{l=0}^{\infty}(2l+1)\abs{\frac{j_l(ka)}{h_l^{(1)}(ka)}}^2
\end{align}
یہ بلکل درست جواب ہے۔ لیکن اس کو دیکھ کر کچھ زیادہ نہیں کہا جاسکتا ہے آئیں کم توانائی بکھرائو \عددی{ka\ll1} کی تحدید صورت پر غور کریں \عددی{k=2\pi/\lambda} کی بنا یہ کہتا ہے کہ دوری عرصہ کرہ کے رداس سے بہت بڑا ہے۔ \حوالہء{جدول \num{4.4}} سے مدد لیتے ہوئے ہم دیکھتے ہیں کہ چھوٹی \عددی{z} کے لیئے \عددی{n_l(z)} کی مقدار \عددی{j_l(z)} سے بہت زیادہ ہوگی لحاظہ 
\begin{align}
	\frac{j_l(z)}{h_l^{(1)}(z)} &= \frac{j_l(z)}{j_l(z)+in_l(z)}\approx-i\frac{j_l(z)}{n_l(z)}\nonumber \\
	&\approx-i\frac{2^ll!z^l/(2l+1)!}{-(2l)!z^{-l-1}/2^ll!} = \frac{i}{2l+1}\left[\frac{2^ll!}{(2l)!}\right]^2z^{2l+1}
\end{align}
اور درج ذیل ہوگا 
\begin{align*}
	\sigma\approx\frac{4\pi}{k^2}\sum_{l=0}^{\infty}\frac{1}{2l+1}\left[\frac{2^ll!}{(2l)!}\right]^4(ka)^{4l+2}
\end{align*}
چونکہ ہم \عددی{ka\ll1} فرض کر رہے ہیں لحاظہ بلند طاقتیں قابلِ نظرانداز ہوں گی۔ کم توانائی تخمین میں \عددی{l=0} جز بکھرائو میں غالب ہوگا۔ یوں کلاسیکی صورت کے لیئے تفریقی عمودی تراش \عددی{\theta} کا تابع نہیں ہوگا۔ ظاہر ہے کہ کم تواانائی سخت کرہ بکھرائو کے لیئے درج ذیل ہوگا 
\begin{align}
	\sigma\approx4\pi a^2
\end{align}
حیرانی کی بات ہے کہ بکھراؤ عمودی تراش کی قیمت جو میٹرائی عمودی تراش کے چار گنا ہے۔ درحقیقت \عددی{\sigma} کی قیمت کرہ کی کل سطحی رقبہ کے برابر ہے۔ لمبی طولِ موج بکھراؤ کی ایک خاصیت بڑی معاصر جسامت ہے جو بصریات میں بھی ہوگا۔ ایک لحاظ سے یہ امواج کرہ کو چھوتے ہوئے اس کے اُپر سے گزرتے ہیں ناکہ کلاسیکی ذرات کی طرح جنہیں صرف سیدھا دیکھتے ہوئے عمودی تراش نظر آتا ہے۔
\انتہا{مثال}
\ابتدا{سوال}
\حوالہء{مساوات \num{11.32}} سے آغاز کرتے ہوئے \حوالہء{مساوات \num{11.33}} ثابت کریں۔ اشارہ: لیژانڈر کثیررکنی کی عمودیت بروئےکار لاتے ہوئے دیکھائیں کہ \عددی{l} کی مختلف قیمتوں والے عددی سر لاظماً صفر ہوں گے۔
\انتہا{سوال}
\ابتدا{سوال}
کروی ڈیلٹا تفاعل خول:
\begin{align*}
	V(r) = \alpha\delta(r-a)
\end{align*}
سے کم توانائی بکھراؤ کیصور پر غور کریں جہاں \عددی{\alpha} اور \عددی{a} مستقلات ہیں۔ حیطہ بکھراؤ \عددی{f(\theta)} تفریقی عمودی تراش \عددی{D(\theta)} اور کل عمودی تراش \عددی{\sigma} کا حساب کریں۔ ان میں \عددی{ka\ll1} فرض کریں لحاظہ صرف \عددی{l=0} جز خاطرخاہ حصہ ڈالیں گے۔ چیزوں کو آسان بنانے کی خاطر آغاز سے ہی \عددی{l\neq0} والے تمام اجزاء کو نظرانداز کریں۔ یہاں \عددی{a_0} تعین کرنا اصل مسئلہ ہے۔ اپنے جواب کو بے بُعدی مقدار \عددی{\beta\equiv2ma\alpha/\hslash^2} کی صورت میں پیش کریں۔

جواب: \عددی{\sigma=4\pi a^2\beta^2/(1+\beta)^2}	
\انتہا{سوال}
\حصہ{ینتقلات حیط}
پہلے نصف لکیر \عددی{x<0} پر مکامی مخفیہ \عددی{V(x)} سے یک بُعدی بکھراؤ کے مسئلے پر غور کرتے ہیں \حوالہء{شکل \num{11.7}} میں \عددی{x=0} پر اینٹون کی ایک دیوار کھڑی کرتا ہوں تاکہ بائیں سے آمدی موج 
\begin{align}
	\psi_i(x) = Ae^{ikx}&&(x<-a)
\end{align}
مکمل طور پر منعکس ہوگا
\begin{align}
	\psi_r(x) = Be^{-ikx}&&(x<-a)
\end{align}
باہم عمل خطہ \عددی{(-a<x<0)} میں جو کچھ بھی ہو احتمال کی بقا کی بنا منعکد موج کا حیطہ لاظماً آمدی موج کے حیطہ کے برابر ہوگا۔ تاہم ضروری نہیں کہ اس کا حیط وہی ہو اگر ماسوائے \عددی{x=0} پر دیوار کے کوئی مخفیہ نہیں پایا جاتا ہو تب چونکہ مبدہ پر آمدی جمع منعکس کل تفاعل موج صفر ہوگا 
\begin{align}
	\psi_0(x) = A\left(e^{ikx}-e^{-ikx}\right)&&(V(x)=0)
\end{align}
لحاظہ \عددی{B=-A} ہوگا۔ غیر صفر مخفیہ کی صورت میں \عددی{x<-a} کے لیئے تفاعل موج درج ذیل روپ اختیار کرتا ہے
\begin{align}
	\psi(x) = A\left(e^{ikx}-e^{i(2\delta-kx)}\right)&&(V(x)\neq0)
\end{align}
نظریہ بکھراؤ کی پوری کہانی کسی مخصوص مخفیہ کے لیئے \عددی{k} لحاظہ توانائی \عددی{E=\hslash^2k^2/2m} کی صورت میں ینتقل حیط کے حساب کا دوسرا نام ہے۔ ہم خطہ بکھراؤ \عددی{(-a<x<0)} میں مساوات زروڈنگر کو حل کر کے مناصب سرحدی شرائط مسلط کر کے ایسا کرتے ہیں \حوالہء{سوال \num{11.5}} دیکھیں۔ مخلوط حیطہ \عددی{B} کی بجائے ینتقل حیط کے ساتھ کرنے کا فائدہ یہ ہے کہ یہ طبیعات پر روشنی ڈالتا ہے۔ احتمال کی بقا کی بدولت مخفیہ منعکس موج کی صرف حیط تبدیل کرسکتا ہے اور ایک مخلوط مقدار جو دو حقیقی اعدات پر مشتمل ہوتا ہے کی بجائے ایک حقیقی مقدار کے ساتھ کام کرتے ہوئے ریاضی آسان ہوتی ہے۔

آئیں اب تین بُعدی صورت پر دوبارہ ڈالیں۔ آمدی مستوی موج \عددی{(Ae^{ikz})} کا \عددی{z} رخ میں کوئی زاویائی معیارِ حرکت نہیں پایا جاتا کلیہ ریلے میں \عددی{m\neq0} والا کوئی جز نہیں پایا جاتا۔ تاہم اس میں کل زیاویائی معیارِ حرکت \عددی{(l=0, 1, 2, \dots)} کی تمام قیمتیں شامل ہیں۔ چونکہ کروی تشاکلی مخفیہ زاویائی معیارِ حرکت کی بقا کرتا ہے لحاظہ ہر ایک جزوی موج جسے کسی ایک خصوصی \عددی{l} سے نام دیا جاتا ہے انفرادی طور پر بکھرے گی اور اس کے حیطہ میں کوئی تبدیلی رونما نہیں ہوگی تاہم اس کا حیطہ تبدیل ہوسکتا ہے۔ مخفیہ بلکل نہ ہونے کی صورت میں \عددی{\psi_0=Ae^{ikz}} ہوگا لحاظہ \عددی{l}ویں جزوی موج درج ذیل ہوگی \حوالہء{مساوات \num{11.28}}
\begin{align}
	\psi_0^{(l)} = Ai^l(2l+1)j_l(kr)P_l(\cos\theta)&&(V(r)=0)
\end{align}
لیکن \حوالہء{مساوات \num{11.19}} اور \حوالہء{جدول \num{11.1}} کے تحت درج ذیل ہوگا
\begin{align}
	j_l(x) = \frac{1}{2}\left[h^{(1)}(x)+h_l^{(2)}(x)\right]\approx\frac{1}{2x}\left[(-i)^{l+1}e^{ix}+i^{l+1}e^{-ix}\right]&&(x\gg1)
\end{align}
لحاظہ بڑی \عددی{r} کی صورت میں درج ذیل ہوگا
\begin{align}
	\psi_0^{(l)}\approx A\frac{(2l+1)}{2ikr}\left[e^{ikr}-(-1)^le^{-ikr}\right]P_l(\cos\theta)&&(V(r)=0)
\end{align}
چکور کوسین میں دوسرا جز آمدی کروی موج کو ظاہر کرتا ہے مخفیہ بکھراؤ متعارف کرمے نے یہ تبدیل نہیں ہوگا۔ پہلا جز رخصتی موج ہے جو ینتقل حیط \عددی{\delta_l} لیتا ہے
\begin{align}
	\psi^{(1)}\approx A\frac{(2l+1)}{2ikr}\left[e^{i(kr+2\delta_1)}-(-1)^le^{-ikr}\right]P_l(\cos\theta)&&(V(r)\neq0)
\end{align}
آپ \عددی{e^{ikz}} میں \عددی{h_l^{(2)}} جز کی بنا اس کو کروی مرتکز موج تصور کر سکتے ہیں جس میں \عددی{2\delta_l} ینتقل حیط پایا جاتا ہے اور جو \عددی{e^{ikz}} میں \عددی{h_l^{(1)}} حصہ کے ساتھ بکھرے موج کی بدولت رخصتی کرویہ موج کے طور پر اُبھرتا ہے۔

\حوالہء{حصہ 11.2.1} میں پورے نظریہ کو جزوی تفاعل حیطوں \عددی{a_l} کی صورت میں پیش کیا گیا یہاں اس کو ینتقل حیط \عددی{\delta_l} کی صورت میں پیش کیا گیا۔ ان دونوں کے بیچ ضرور کوئی تعلق پایا جاتا ہوگا۔ یقیناً \حوالہء{مساوات \num{11.23}} کی بڑی \عددی{r} کی صورت میں متقاربی روپ 
\begin{align}
	\psi^{(1)}\approx A\left\{\frac{(2l+1)}{2ikr}\left[e^{ikr}-(-1)^le^{-ikr}\right]+\frac{(2l+1)}{r}a_le^{ikr}\right\}P_l(\cos\theta)
\end{align}
کا \عددی{\delta_l} کی صورت میں عمومی کی صورت \حوالہء{مساوات \num{1.44}} کے ساتھ موازنہ کرنے ساے درج ذیل حاصل ہوگا
\begin{align}
	a_l=\frac{1}{2ik}\left(e^{2i\delta_l}-1\right)=\frac{1}{k}e^{i\delta_l}\sin(\delta_l)
\end{align}
اس طرح بلخصوص \حوالہء{مساوات \num{11.25}} 
\begin{align}
	f(\theta) = \frac{1}{k}\sum_{l=0}^{\infty}(2l+1)e^{i\delta_l}\sin(\delta_l)P_l(\cos\theta)
\end{align}
اور درج ذیل ہوگا \حوالہء{مساوات \num{11.27}} 
\begin{align}
	\sigma=\frac{4\pi}{k^2}\sum_{l=0}^{\infty}(2l+1)\sin^2(\delta_l)
\end{align}
اب بھی جزوی موج حیطوں کی بجائے ینتقلات حیط کے ساتھ کام کرنا بہتر ثابت ہوتا ہے چونکہ ان سے طبعی معلومات باآسانی حاصل ہوتی ہے اور ریاضی کی نقطہ نظر سے ان کے ساتھ کام کرنا آسان ہوتا ہے ینتقلی حیط زاویائی معیارِ حرکت کی بقا کو استعمال کرتے ہوئے مخلوط مقدار \عددی{a_l} جو دو حقیقی اعدات پر مشتمل ہوتا ہے کی بجائے ایک حقیقی عدد \عددی{\delta_l} استعمال کرتا ہے۔
%=================

\ابتدا{سوال}
ایک ذرہ جس کی کمیت \عددی{m} اور توانائی \عددی{E} ہو درج ذیل مخفیہ پر بائیں سے آمدی ہے
\begin{align*}
	V(x)=
	\begin{cases}
		0, & (x<-a). \\
		-V_0, & (-a\leq z\leq0). \\
		\infty, & (x>0).
	\end{cases}
\end{align*}
(الف) آمدی موج \عددی{Ae^{ikx}} جہاں \عددی{k=\sqrt{2mE}/\hslash} کی صورت میں منعکس موج تلاش کریں۔

جواب:
\begin{align*}
	Ae^{-2ika}\left[\frac{k-ik'\cot(k'a)}{k+ik'\cot(k'a)}\right]e^{-ikx}, && \text{\RL{جہاں}} k'=\sqrt{2m(E+V_0)}/\hslash 
\end{align*}
(ب) تصدیق کریں کہ منعکس موج کا حیطہ وہی ہے جو آمدی موج کا ہے۔

(ج) بہت گہرا کنواں \عددی{E\ll V_0} کے لیئے ینتقلات حیط \عددی{\delta} \حوالہء{مساوات \num{11.40}} تلاش کریں۔

جواب: \عددی{\delta=-ka}
\انتہا{سوال}
\ابتدا{سوال}
سخت کرہ بکھراؤ کے لیئے جزوی موج حیطی انتقال \عددی{\delta_l} کیا ہوں گے \حوالہء{مثال \num{11.3}}؟
\انتہا{سوال}
\ابتدا{سوال}
ایک ڈیلٹا تفاعل خول \حوالہء{سوال \num{11.4}} سے \عددی{S} موج \عددی{l=0} جزوی موج انتقال حیط \عددی{\delta_0(k)} تلاش کریں۔ ایسا کرتے ہوئے فرض کریں کہ \عددی{r\to\infty} پر رداسی تفاعل موج \عددی{u(r)} صفر کو پہنچے گا۔

جواب:
\begin{align*}
	-\cot^{-1}\left[\cot(ka)+\frac{ka}{\beta\sin^2(ka)}\right], &&\text{\RL{جہاں}} \beta\equiv\frac{2m\alpha a}{\hslash^2}
\end{align*}
\انتہا{سوال}
\حصہ{بارن تخمین}
\جزوحصہ{مساوات شروڈنگر کی تکملی روپ}
غیر تابع وقت شروڈنگر مساوات
\begin{align}
	-\frac{\hslash^2}{2m}\nabla\psi+V\psi=E\psi
\end{align}
کو مختصراً
\begin{align}
	(\nabla^2+k^2)\psi=Q
\end{align}
لکھا جا سکتا ہے جہاں درج ذیل ہوں گے
\begin{align}
	k\equiv\frac{\sqrt{2mE}}{\hslash} & \text{\RL{اور}} Q\equiv\frac{2m}{\hslash^2}V\psi
\end{align}
اس کا روپ سرسری طور پر مساوات ہلم ہولٹز کی طرح ہے۔ البتہ غیر متجانس جز \عددی{Q} از خود \عددی{\psi} کا تابع ہے۔

فرض کریں ہم ایک تفاعل \عددی{G(r)} دریافت کر پائیں جو ڈیلٹا تفاعلی منبع کے لیئے مساوات ہلم ہولٹز کو متمعن کرتا ہو
\begin{align}
	(\nabla^2+k^2)G(r)=\delta^3(r)
\end{align}
ایسی صورت میں ہم \عددی{\psi} کو بطور ایک تکمل لکھ سکتے ہیں
\begin{align}
	\psi(r)=\int G(r-r_0)Q(r_0)\dif^3r_0
\end{align}
ہم با آسانی دیکھا سکتے ہیں کہ یہ \حوالہء{مساوات \num{11.50}} روپ کی شروڈنگر مساوات کو متمعن کرتا ہے
\begin{align*}
	(\nabla^2+k^2)\psi(r) &= \int\left[(\nabla^2+k^2)G(r-r_0)\right]Q(r_0)\dif^3r_0 \\
	&= \int\delta^3(r-r_0)Q(r_0)\dif^3r_0 = Q(r)
\end{align*}
تفاعل \عددی{G(r)} کو مساوات ہلم ہولٹز کا تفاعل گرین کہتے ہیں۔ عمومی طور پر ایک خطی تفرقی مساوات کا تفاعل گرین ایک ڈیلٹا تفاعلی منبع کو ردِ عمل ظاہر کرتا ہے۔

ہمارا پہلا کام \عددی{G(r)} کے لیئے \حوالہء{مساوات \num{11.52}} کا حل تلاش کرنا ہے۔ ایسا کرنے کا آسان ترین طریقہ یہ ہے کہ ہم فوریر بدل لیں جو تفرقی مساوات کو ایک الجبرائی مساوات میں تبدیل کرتا ہے۔ درج ذیل لیں
\begin{align}
	G(r)=\frac{1}{(2\pi)^{3/2}}\int e^{is\cdot r}g(s)\dif^3s
\end{align}
تب 
\begin{align*}
	(\nabla^2+k^2)G(r) = \frac{1}{(2\pi)^{3/2}}\int\left[(\nabla^2+k^2)e^{is\cdot r}\right]g(s)\dif^3s
\end{align*}
ہوگا تاہم
\begin{align}
	\nabla^2e^{is\cdot r} = -s^2 e^{is\cdot r}
\end{align}
اور \حوالہء{مساوات \num{2.144}} دیکھیں
\begin{align}
	\delta^3(r)=\frac{1}{(2\pi)^3}\int e^{is\cdot r}\dif^3s
\end{align}
لحاظہ \حوالہء{مساوات \num{11.52}} درج ذیل کہے گی
\begin{align*}
	\frac{1}{(2\pi)^{3/2}}\int(-s^2+k^2)e^{is\cdot r}g(s)\dif^3s = \frac{1}{(2\pi)^3}\int e^{is\cdot r}\dif^3s
\end{align*}
یوں درج ذیل ہوگا 
\begin{align}
	g(s) = \frac{1}{(2\pi)^{3/2}(k^2-s^2)}
\end{align}
اس کو واپس \حوالہء{مساوات \num{11.54}} میںپُر کع کے درج ذیل ملتا ہے
\begin{align}
	G(r) = \frac{1}{(2\pi)^3}\int e^{is\cdot r}\frac{1}{(k^2-s^2)}\dif^3s
\end{align}
اب \عددی{s} تکمل کے نقطع نظر سے \عددی{r} غیر متغیر ہے ہم کروی محدد \عددی{(s, \theta, \phi)} کو یوں چنتے ہیں کہ \عددی{r} کتبی محور پر پایا جاتا ہو \حوالہء{شکل \num{11.8}}۔ یوں \عددی{s\cdot r = sr\cos\theta} ہوگا متغیر \عددی{\phi} کا تکمل \عددی{2\pi} ہوگا جبکہ \عددی{\theta} تکمل درج ذیل ہوگا
 
\begin{align}
	\int_{0}^{\pi}e^{isr\cos\theta}\sin\theta\dif\theta = -\frac{e^{isr\cos\theta}}{isr}\bigg|^\pi_{0} = \frac{2\sin(sr)}{sr}
\end{align}
یوں درج ذیل ہوگا
\begin{align}
	G(r) = \frac{1}{(2\pi^2)}\frac{2}{r}\int_{0}^{\infty}\frac{s \sin(sr)}{k^2-s^2}\dif s = \frac{1}{4\pi^2r}\int_{-\infty}^{\infty}\frac{s \sin(sr)}{k^2-s^2}\dif s
\end{align}
باقی تکمل اتنا آسان نہیں ہے۔ قوت نمائی علامتیت استعمال کرکے نصب نما کو اجزائے ضربی کی روپ میں لکھنا مددگا ثابت ہوتا ہے
\begin{align}
	G(r) &= \frac{i}{8\pi^2r}\left\{\int_{-\infty}^{\infty}\frac{se^{isr}}{(s-k)(s+k)}\dif s-\int_{-\infty}^{\infty}\frac{se^{-isr}}{(s-k)(s+k)}\dif s\right\}\nonumber \\
	&= \frac{i}{8\pi^2r}(I_1-I_2)
\end{align}
اگر \عددی{z_0} خطِ ارتفاہ کے اندر پایا جاتا ہو تب کوشی کلیہ تکمل 
\begin{align}
	\oint\frac{f(z)}{(z-z_0)}\dif z = 2\pi if(z_0)
\end{align}
استعملا کرتے ہوئے ان تکملات کی قیمت تلاش کی جا سکتی ہے دیگر صورت تکمل صفر ہوگا۔ یہاں حقیقی محور جو \عددی{\pm k} پر قطبی نادر نکات کے بلکل اوپر سے گزرتا ہے کے کے ساتھ ساتھ تکمل لیا جا رہا ہے۔ ہمیں قطبین کے اطراف سے گزرنا ہوگا میں \عددی{-k} پر بلائی جانب سے \عددی{+k} پر زیریں جانب سے گزروں گا \حوالہء{شکل \num{11.9}}۔ آپ کوئی نیا راستہ منتخب کر سکتے ہیں مثلاً آپ ہر قطب کے گرد سات مرتبہ چکر کاٹ کر راہ منتخب کر سکتے ہیں جس سے آپ کو ایک مختلف تفاعل گرین حاصل ہوگا لیکن میں کچھ ہی دیر میں دیکھاؤں گا کہ یہ تمام قابلِ قبول ہوں گے۔

\حوالہء{مساوات \num{11.61}} میں ہر ایک تکمل کے لیئے ہمیں خط استوا کو اس طرح بند کرنا ہوگا  کہ لامتناہی پر نصف دائرہ تکمل کی قیمت میں کوئی حصہ نہ ڈالے۔ تکمل \عددی{I_1} کی صورت میں اگر \عددی{s} کا خیالی جز بہت بڑا اور مثبت ہو تب جز ضربی \عددی{e^{isr}} صفر کو پہنچے گا اس تکمل کے لیئے ہم بالا نصف دائرہ لیتے ہیں \حوالہء{شکل \num{11.10} (الف)}۔ اب خط ارتفا صرف \عددی{s=+k} پر پائے جانے والا نادر نقطع کو گھیرتا ہے لحاظہ درج ذیل ہوگا
\begin{align}
	I_1 = \oint\left[\frac{se^{isr}}{s+k}\right]\frac{1}{s-k}\dif s = 2\pi i\left[\frac{se^{isr}}{s+k}\right]\bigg|_{s=k} = i\pi e^{ikr}
\end{align}
تکمل \عددی{I_2} کی صورت میں جب \عددی{s} کا خیالی جز بہت بڑی منفی مقدار ہو تب جز ضربی \عددی{e^{-isr}} صفر کو پہنچتا ہے لحاظہ ہم زیریں نصف دائراہ لیتے ہیں \حوالہء{شکل \num{11.10} (ب)}۔ اس مرتبہ خطِ ارتفا \عددی{s=-k} پر پائے جانے والے نادر نقطہ جو کو گھیرتا ہے اور یہ گھڑی وار ہے لحاظہ اس کے ساتھ اضافی منفی علامت ہوگا
\begin{align}
	I_2 = -\oint\left[\frac{se^{-isr}}{s-k}\right]\frac{1}{s+k}\dif s = -2\pi i\left[\frac{se^{-isr}}{s-k}\right]\bigg|_{s=-k} = -i\pi e^{ikr}
\end{align}
ماخوذ:
\begin{align}
	G(r) = \frac{i}{8\pi^2r}\left[\left(i\pi e^{ikr}\right)-\left(-i\pi e^{ikr}\right)\right] = -\frac{e^{ikr}}{4\pi r}
\end{align}
یہ \حوالہء{مساوات \num{11.52}} کا حل اور مساوات ہلم ہولٹز کا تفاعل گرین ہے اگر آپ کہیں ریاضیاتی تجزیہ میں گم ہوگئے ہوں تب بلاواسطہ تفرق کی مدد سے نتیجہ کی تصدیق کی جیئے گا \حوالہء{سوال \num{11.8}} دیکھیں۔ بلکہ یہ مساوات ہلم ہولٹز کا ایک تفاعل گرین ہے چونکہ ہم \عددی{G(r)} کے ساتھ ایسا کوئی بھی تفاعل \عددی{G_0(r)} جمع کر سکتے ہیں جو متجانز ہلم ہولٹز مساوات کو متمعن کرتا ہو
\begin{align}
	(\nabla^2+k^2)G_0(r) = 0
\end{align}
صاف ظاہر ہے کہ \حوالہء{مساوات \num{11.52}} کو \عددی{(G+G_0)} بھی متمعن کرتا ہے۔ اس ابہام کی وجہ قطبین کے قریب سے گزرتے ہوئے راہ کی بنا ہے راہ کی ایک مختلف انتخاب ایک مختلف تفاعل \عددی{G_0(r)} کے مترادف ہے۔

\حوالہء{مساوات \num{11.53}} کو دوبارہ دیکھتے ہوئے مساوات شروڈنگر کا عمومی حل درج ذیل روپ کا ہوگا
\begin{align}
	\psi(r) = \psi_0(r)-\frac{m}{2\pi\hslash^2}\int\frac{e^{ik\abs{r-r_0}}}{\abs{r-r_0}}V(r_0)\psi(r_0)\dif^3r_0
\end{align}
جہاں \عددی{\psi_0} آزاد ذرہ مساوات شروڈنگر کو متمعن کرتا ہے
\begin{align}
	(\nabla^2+k^2)\psi_0 = 0
\end{align}
\حوالہء{مساوات \num{11.67}} شروڈنگر مساوات کی تکملی روپ ہے جو زیادہ معروف تفرقی روپ کی مکمل طور پر معدل ہے۔ پہلی نظر میں ایسا معلوم ہوتا ہے کہ یہ کسی بھی مخفیہ کے لیئے مساوات شروڈنگر کا سری حل ہے جو ماننے والی بات نہیں ہے۔ دھوکہ مت کھائیں۔ دائیں ہاتھ تکمل کی علامت کے اندر \عددی{\psi} پایا جات ہے جسے جاننے بغیر آپ تکمل حاصل کر کے حل نہیں جان سکتے ہیں  تاہم تکملی روپ انتہائی طاقتور ثابت ہوتا ہے اور جیسا ہم اگلے حصہ میں دیکھیں گے یہ بلخصوص بکھراؤ مسائل کے لیئے نہایت موضوع ہے۔

\ابتدا{سوال}
\حوالہء{مساوات \num{11.65}} کو \حوالہء{مساوات \num{11.52}} میں پُر کر کے دیکھیں کہ یہ اسے متمعن کرتا ہے۔ اشارہ: \عددی{\nabla^2(1/r) = -4\pi\delta^3(r)}۔
\انتہا{سوال}
\ابتدا{سوال}
دیکھائیں کہ \عددی{V} اور \عددی{E} کی مناسب قیمتوں کے لیئے مساوات شروڈنگر کی تکملی روپ کو ہائڈروجن کا زمینی حال \حوالہء{مساوات \num{4.80}} متمعن کرتا ہے۔ دیہان رہے کہ \عددی{E} منفی ہے لحاظہ \عددی{k=i\kappa} ہوگا جہاں \عددی{\kappa\equiv\sqrt{-2mE}/\hslash} ہوگا۔
\انتہا{سوال}

%==========================

\جزوحصہ{بارن تخمین اوّل}
فرض کریں \عددی{r_0 = 0} پر \عددی{V(r_0)} مکامی مخفیہ ہے یعنی کسی متناہی خطہ کے باہر مخفیہ کی قیمت صفر ہے جو عموماً مسئلہ بکھراؤ میں ہعگا اور ہم مرکز بکھراؤ سے دور نکات پر \عددی{\psi(r)} جاننا چاہتے ہیں۔ ایسی صورت میں \حوالہء{مساوات \num{11.67}} کی تکمل میں حصہ ڈالنے والے تمام نکات کے لیئے \عددی{\abs{r}\gg\abs{r_0}} ہوگا لحاظہ
\begin{align}
	\abs{r-r_0}^2 = r^2+r_0^2-2r\cdot r_0 \cong r^2\left(1-2\frac{r\cdot r_0}{r^2}\right)
\end{align}
اور یوں درج ذیل ہوگا
\begin{align}
	\abs{r-r_0}^2\cong r-\hat{r}\cdot r_0
\end{align}
ہم 
\begin{align}
	k\equiv k\hat{r}
\end{align}
لیتے ہیں۔ یوں
\begin{align}
	e^{ik\abs{r-r_0}}\cong e^{ikr}e^{-ik\cdot r_0}
\end{align}
ہوگا۔ لحاظہ درج ذیل ہوگا 
\begin{align}
	\frac{e^{ik\abs{r-r_0}}}{\abs{r-r_0}}\cong\frac{e^{ikr}}{r}e^{-ik\cdot r_0}
\end{align}
نصب نما میں ہم زیادہ بڑی تخمین \عددی{\abs{r-r_0}\cong r} دے سکتے ہیں قوت نما میں ہمیں دوسرا جز بھی رکھنا ہوگا۔ اگر آپ یقین نہیں کر سکتے ہیں تو نصب نما میں دوسرے جز کو پہلا کر دیکھیں ہم یہاں ایک چھوٹی مقدار \عددی{(r_0/r)} کی قوتوں میں پھیلا کر کم سے کم رتبی جز کے علاوہ باقی تمام کو رد کرتے ہیں۔

بکھراؤ کی صورت میں ہم درج ذیل چاہتے ہیں۔ جو آمدی مستوی موج کو ظاہر کرتا ہے
\begin{align}
	\psi_0(r) = Ae^{ikz}
\end{align}
یوں بڑی \عددی{r} کے لیئے درج ذیل ہوگا 
\begin{align}
	\psi(r)\cong Ae^{ikz}-\frac{m}{2\pi\hslash^2}\frac{e^{ikr}}{r}\int e^{ik\cdot r_0}V(r_0)\psi(r_0)\dif^3r_0
\end{align}
یہ معیاری روپ \حوالہء{مساوات \num{11.12}} ہے جس سے ہم حیطہ بکھراؤ پڑھ سکتے ہیں
\begin{align}
	f(\theta, \phi) = -\frac{m}{2\pi\hslash^2A}\int e^{-ik\cdot r_0}V(r_0)\psi(r_0)\dif^3r_0 
\end{align}
یہاں تک یہ بلکل ایک درست جواب ہے ہم اب بارن تخمین باروہِ کار لاتے ہیں۔ فرض کریں آمدیہ مستوی موج کو مخفیہ قابلِ ذکر تبدیل نہیں کرتا ہو ایسی صورت میں درج ذیل استعمال کرنا معقول ہوگا
\begin{align}
	\psi(r_0)\approx\psi_0(r_0) = Ae^{ikz_0} = Ae^{ik'\cdot r_0}
\end{align}
جہاں تکمل کے اندر \عددی{k'} درج ذیل ہے
\begin{align}
	k'\equiv k\hat{z}
\end{align}
مخفیہ \عددی{V} صفر ہونے کی صورت میں یہ بلکل ٹھیک تفاعل موج ہوتا یہ بنیادی طور پر کمزور مخفیہ تخمین ہے۔ بارن تخمین میں یوں درج ذیل ہوگا 
\begin{align}
	f(\theta, \phi)\cong-\frac{m}{2\pi\hslash^2}\int e^{i(k'-k)\cdot r_0}V(r_0)\dif^3r_0
\end{align}
ہوسکتا ہے کہ آپ \عددی{k'} اور \عددی{k} کی تعریفات بھول چکے ہوں دونوں کی مقدار \عددی{k} ہے تاہم اوّل الذکر کا رخ آمدی شعاع کے رخ ہے جبکہ معاخرالذکر کا رخ کاشف کے رخ ہے \حوالہء{شکل \num{11.11}} دیکھیں۔ اس عمل میں \عددی{\hslash(k-k')} منتقلی معیارِ حرکت کو ظاہر کرے گا بلخصوص خطہ بکھراؤ پر کم توانائی لمبی طولِ موج بکھراؤ کے لیئے قوتِ نمائی جز ضربی بنیادی طر پر مستقل ہوگا اور یوں تخمین بارن درج ذیل سادہ روپ اختیار کرے گا   
\begin{align}
	f(\theta, \phi)\cong-\frac{m}{2\pi\hslash}\int V(r)\dif^3r, && \text{\RL{کم توانائی}}
\end{align}
میں نے یہاں \عددی{r} کے زیرِ نوشت میں کچھ نہیں لکھا اُید کی جاتی اس سے کوئی پریشانی پیدا نہیں ہوگی۔

\ابتدا{مثال}
کم توانائی نرم کرہ بکھراؤ درج ذیل مخفیہ لیں 
\begin{align}
	V(r)=
	\begin{cases}
		V_0, & r\leq a \text{\RL{اگر}} \\
		0, & r>a \text{\RL{اگر}}
	\end{cases}
\end{align}
کم توانائی کی صورت میں \عددی{\theta} اور \عددی{\phi} کا غیر تابع حیطہ مکھراؤ درج ذیل ہوگا۔
\begin{align}
	f(\theta, \phi)\cong-\frac{m}{2\pi\hslash^2}V_0\left(\frac{4}{3}\pi a^3\right)
\end{align}
تفریقی عمودی تراش 
\begin{align}
	\frac{\dif\sigma}{\dif\Omega}=\abs{f}^2\cong\left(\frac{2mV_0a^3}{3\hslash^2}\right)^2
\end{align}
اور کل عمودی تراش درج ذیل ہوگا۔ 
\begin{align}
	\sigma\cong4\pi\left(\frac{2mV_0a^3}{3\hslash^2}\right)^2
\end{align}
\انتہا{مثال}
ایک کروی تشاکلی مخفیہ \عددی{V(r)=V(r)} کے لیئے جو ضروری نہیں کہ کم توانائی پر ہو تخمین بارن دوبارہ سادہ روپ اختیار کرتا ہے۔ درج ذیل متعارف کرتے ہوئے 
\begin{align}
	\kappa\equiv k'-k
\end{align}
\عددی{r_0} تکمل کے قطبی محور کو \عددی{\kappa} پر رکھتے ہوئے درج ذیل ہوگا 
\begin{align}
	(k'-k)\cdot r_0 = \kappa r_0\cos\theta_0
\end{align}
یوں درج ذیل حاصل ہوگا
\begin{align}
	f(\theta)\cong-\frac{m}{2\pi\hslash^2}\int e^{i\kappa r_0\cos\theta_0}V(r_0)r^2_0\sin\theta_0\dif r_0\dif\theta_0\dif\phi_0
\end{align}
متغیر \عددی{\phi_0} کے لحاظ سے تکمل \عددی{2\pi} دیگا اور \عددی{\theta_0} تکمل کو ہم پہلے دیکھ چکے ہیں \حوالہء{مساوات \num{11.59}} دیکھیں۔ یوں \عددی{r} کے زیرِنوشت کو نہ لکھتے ہوئے درج ذیل رہ جائے گا
\begin{align}
	f(\theta)\cong-\frac{2m}{\hslash^2\kappa}\int_{0}^{\infty}rV(r)\sin(\kappa r)\dif r، && \text{\RL{کروی تشاکل}}
\end{align}
\عددی{f} کی زیویائی تابیعت \عددی{\kappa} میں سموئی گئی ہے \حوالہء{شکل \num{11.11}} کو دیکھ کر درج ذیل لکھا جا سکتا ہے
\begin{align}
	\kappa = 2k\sin(\theta/2)
\end{align}
\ابتدا{مثال}
یوکاوا بکھراؤ۔ یوکاوا مخفیہ جو جوہری مرکزہ کے بیچ بندشی قوت کا ایک سادہ نمونہ پیش کرتا ہے کا روپ درج ذیل ہے جہاں \عددی{\beta} اور \عددی{\mu} مستقلات ہیں
\begin{align}
	V(r) = \beta\frac{e^{-\mu r}}{r}
\end{align}
تخمین بارن درج ذیل دیگا 
\begin{align}
	f(\theta)\cong-\frac{2m\beta}{\hslash^2\kappa}\int_{0}^{\infty}e^{-\mu r}\sin(\kappa r)\dif r=-\frac{2m\beta}{\hslash(\mu^2+\kappa^2)}
\end{align}
آپ کو \حوالہء{سوال \num{11.11}} میں یہ تکمل حل کرنے کو کہا گیا ہے۔
\انتہا{مثال}
\ابتدا{مثال}
ردرفورڈ بکھراؤ۔ مخفیہ یوکاوا میں \عددی{\beta=q_1q_2/4\pi\epsilon_0} اور \عددی{\mu=0} پُر کرنے سے مخفیہ کولمب حاصل ہوگا جو دو نقطی باروں کے بیچ برقی باہم عمل کو بایان کرتا ہے۔ ظاہر ہے کہ حیطہ بکھراؤ درج ذیل ہوگا 
\begin{align}
	f(\theta)\cong-\frac{2mq_1q_2}{4\pi\epsilon_0\hslash^2\kappa^2}
\end{align}
یا \حوالہء{مساوات \num{11.89} اور \num{11.51}} استعمال کرتے ہوئے درج ذیل ہوگا 
\begin{align}
	f(\theta)\cong-\frac{q_1q_2}{16\pi\epsilon_0E\sin^2(\theta/2)}
\end{align}
اس کا مربع ہمیں تفریقی عمودی تراش دیگا 
\begin{align}
	\frac{\dif\sigma}{\dif\Omega}=\left[\frac{q_1q_2}{16\pi\epsilon_0E\sin^2(\theta/2)}\right]^2
\end{align}
جو ٹھیک کلیہ ردرفورڈ \حوالہء{مساوات \num{11.11}} ہے۔ آپ دیکھ سکتے ہیں کہ کولمب مخفیہ کے لیئے کالسیکی میکانیات تخمین بارن اور کوانٹم نظریہ میدان تمام ایک دوسرے جیسا نتیجہ دیتے ہیں۔ ہم کہہ سکتے ہیں کہ کلیہ ردرفورڈ ایک مضبوط کلیہ ہے۔
\انتہا{مثال}
\ابتدا{سوال}
اختیاری توانائی کے لیئے نرم کرہ بکھراؤ کا حیطہ بکھراؤ بارن تخمین سے حاصل کریں دیکھائیں کہ کم توانائی حد میں اس سے \حوالہء{مساوات \num{11.82}} حاصل ہوگا۔
\انتہا{سوال}
\ابتدا{سوال}
\حوالہء{مساوات \num{11.91}} میں تکمل کی قیمت تلا کر کے دائیں ہاتھ ریاضی فکرہ کی تصدیق کریں۔
\انتہا{سوال}
\ابتدا{سوال}
بارن تخمین میں یوکاوا مخفیہ سے بکھراؤ کا کل عمودی تراش تلاش کریں۔ اپنے جواب کو \عددی{E} کا تفاعل لکھیں۔
\انتہا{سوال}
\ابتدا{سوال}
درج ذیل اقدام \حوالہء{سوال \num{11.4}} کے مخفیہ کے لیئے کریں۔

(الف) کم توانائی تخمین بارن میں \عددی{f(\theta, D(\theta))} اور \عددی{\sigma} کا ھساب لگائیں۔

(ب) تخمین بارن میں اختیاری توانائیوں کے لیئے \عددی{f(\theta)} کا حساب لگائیں۔

(ج) دیکھائیں کہ آپ کے نتائج مناسب خطوں میں \حوالہء{سوال \num11.4} کے جواب کے مطابق ہیں۔
\انتہا{سوال}

%=================================

\جزوحصہ{تسلسل بارن}
تخمین بارن روح کے لحاظ سے کلاسیکی نظریہ بکھراؤ میں تخمین ضرب کی طرح ہے۔ ایک ذرہ کو منتقل عرضی ضرب کا حساب کرنے کے لیئے ہم تخمین ضرب میں فرض کرتے ہیں کہ ذرہ ایک سیدھی لیکر پر ہی چلے جاتا ہے \حوالہء{شکل \num{11.12}} ایسی صورت میں درج ذیل ہوگا
\begin{align}
	I=\int F_\perp\dif t
\end{align}
اگر ذرہ زیادہ نہیں مڑے تب یہ ذرہ کو منتقل معیارِ حرکت کی ایک اچھی تخمین ہوگی اور یوں زاویہ بکھراؤ درج ذیل ہوگا جہاں \عددی{p} آمدی معیارِ حرکت ہے 
\begin{align}
	\theta\cong\tan^{-1}(I/p)
\end{align}
اسے ہم رتبہ اوّل تخمین ضرب کہہ سکتے ہیں نہ مڑنےکی صورت کو صفر رتبی کہا ھائے گا اسی طرح صفر رتبی تخمین بارن میں آمدی مستوی موج بغیر کسی تبدیلی کے گزرے گی اور ہم نے جو کچھ گزشتہ حصہ میں دیکھا وہ در حقیقت اس کی رتبہ اوّل تصحیح ہے۔ ہم توقع کر سکتے ہیں کہ اسی تصور کو بار بار استعمال کرتے ہوئے ہم زیادہ بلند رتبی تصحیح کا ایک تسلسل پیدا کر کے بلکل ٹھیک جواب پر مرکوز ہو سکتے ہیں۔

مساوات شروڈنگر کی تکملی روپ درج ذیل ہے
\begin{align}
	\psi(r)=\psi_0(r)+\int g(r-r_0)V(r_0)\psi(r_0)\dif^3r_0
\end{align}
جہاں \عددی{\psi_0} آمدی موج ہے
\begin{align}
	g(r)\equiv-\frac{m}{2\pi\hslash^2}\frac{e^{ikr}}{r}
\end{align}
تفاعل گرین ہے۔ جس میں میں نے اپنی آسانی کے لیئے جز ضربی \عددی{2m/\hslash^2} شامل کیا ہے اور \عددی{V} مخفیہ بکھراؤ ہے۔ اس کو درج ذیل دیکھا جا سکتا ہے
\begin{align}
	\psi = \psi_0+\int gV\psi
\end{align}
فرض کریں ہم \عددی{\psi} کی اس ریاضی جملہ کو لیکر اسے تکمل کی علامت کے اندر لکھیں 
\begin{align}
	\psi=\psi_0+\int gV\psi_0+\iint gVgV\psi
\end{align}
اس عمل کہ بار بار دوہرانے سے ہمیں \عددی{\psi} کا ایک تسلسل حاصل ہوگا
\begin{align}
	\psi=\psi_0+\int gV\psi_0+\iint gVgV\psi_0+\iiint gVgVgV\psi_0+\dots
\end{align}
ہر متکمل میں آمدی تفاعل موج \عددی{\psi_0} کے علاوہ \عددی{gV} کے مزید زیادہ طاقتیں پائی جاتی ہیں۔ بارن کی تخمین اوّل اس تسلسل کو دوسرے جز کے بعد ختم کرتا ہے تاہم آپ دیکھ سکتے ہیں کہ بلند رتبی تصحیح کس طرح پیدا کی جائیں گی۔

بارن تسلسل کا خاکہ \حوالہء{شکل \num{11.13}} میں پیش کیا گیا ہے۔ صفر رتبی \عددی{\psi} پر مخفیہ کا کوئی اثر نہیں ہوگا رتبی اوّل میں اسے ایک چوٹ پڑتی ہے جس کے بعد یہ کسی نئے رخ چلے جائے گا۔ دوم رتبی میں اسے ایک چوٹ پڑتی ہے جس کے بعد یہ ایک نئے مقام پر پہنچتا ہے جہاں اسے دوبارہ ایک چوٹ پڑتی ہے جس کے بعد یہ ایک نئے راہ پر چل نکلتا ہے وغیرہ وغیرہ۔ اسی کے بنا بعض اوقات تفاعل گرین کو اشاعت کار کہا جاتا ہے جو ایک باہم عمل اور سورے کے بیچ خلل کی اشاعت کس طرح ہوتی ہے۔ تسلسل بارن اضافیتی کوانٹم میکانیات کی فینمن تشریح کا سبب بنا جس میں اشکال فینمن میں جز ضربی راس \عددی{V} اور اشاعت کار \عددی{g} کو ایک دوسرے کے ساتھ جوڑ کر سب کچھ بیان کیا جاتا ہے۔

\ابتدا{سوال}
تخمین ضرب میں ردرفورڈ بکھراؤ کے لیئے \عددی{\theta} کو ٹکراؤ مقدار معلوم کا تفاعل تلاش کریں۔ دیکھائیں کہ مناسب حدوں کے اندر آپ کا نتیجہ بلکل ٹھیک ریاضی فکرہ \حوالہء{سوال \num{11.1} (الف)} کے مطابق ہے۔
\انتہا{سوال}
\ابتدا{سوال}
بارن کی دوسری تخمین میں کم توانائی نرم کرہ بکھراو کے لیئے حیطہ بکھراو تلاش کریں۔

جواب: \عددی{-(2mV_0a^3/3\hslash^2)[1-(4mV_0a^2/5\hslash^2)]}
\انتہا{سوال}
\ابتدا{سوال}
یک بُعدی مساوات شروڈنگر کے لیئے تفاعل گریں تلاش کر کے \حوالہء{مساوات \num{11.67}} کا مماثل تکملی روپ تیار کریں۔

جواب:
\begin{align}
	\psi(x)=\psi_0(x)-\frac{im}{\hslash^2k}\int_{-\infty}^{\infty}e^{ik\abs{x-x_0}}V(x_0)\psi(x_0)\dif x_0
\end{align}
\انتہا{سوال}
\ابتدا{سوال}
مبدہ پر بغیر اینٹون کی دیوار کی صورت میں وقفہ \عددی{-\infty<x<\infty} پر یک بُعدی بکھراو کے لیئے \حوالہء{سوال \num{11.16}} کا نتیجہ استعمال کرتے ہوئے تخمین بارن تیار کریں۔ یعنی \عددی{\psi(x_0)\cong\psi_0(x_0)} تصور کرتے ہوئے \عددی{\psi_0(x)=Ae^{ikx}} منتخب کر کت تکمل کی قیمت تلاش کریں۔ دیکھائیں کہ انعکاسی عددی سر درج ذیل روپ اختیار کرتا ہے
\begin{align}
	R\cong\left(\frac{m}{\hslash^2k}\right)^2\abs{\int_{-\infty}^{\infty}e^{2ikx}V(x)\dif x}^2
\end{align}
\انتہا{سوال}
\ابتدا{سوال}
ایک ڈیلٹا تفاعل \حوالہء{مساوات \num{2.114}} اور ایک متناہی چکور کنواں \حوالہء{مساوات \num{2.145}} سے بکھراو کے لیئے تفصیلی عددی سر \عددی{(T = 1 - R)} کو یک بُعدی تخمین بارن \حوالہء{سوال \num{11.17}} کی مدد سے حاصل کریں۔ اپنے جوابات کا بلکل ٹھیک جوابات \حوالہء{مساوات \num{2.141} اور \num{2.169}} کے ساتھ موازنہی کریں۔
\انتہا{سوال}
\ابتدا{سوال}
آگے رخ ھیطہ بکھراو کے خیالی جز اور کل عمودی تراش کے بیچ رشتہ دینے والا مسئلہ بصریات ثابت کریں 
\begin{align}
	\sigma = \frac{4\pi}{k}Im(f(0))
\end{align}
اشارہ: \حوالہء{مساوات \num{11.47} اور \num{11.48}} استعمال کریں۔
\انتہا{سوال}
\ابتدا{سوال}
Missing Question
\begin{align}
	V(r) = Ae^{-\mu r^2}
\end{align}
\انتہا{سوال}

